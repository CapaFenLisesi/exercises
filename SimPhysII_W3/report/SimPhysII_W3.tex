%%%%%%%% Klassen-Optionen
\documentclass[12pt,a4paper]{scrartcl}

%%%%%%%% PAKETE: unverzichtbare Pakete mit Einstellungen
\usepackage[left=2.5cm, right=2cm, top=3cm, bottom=3cm, a4paper]{geometry} %Seitenrände
\usepackage[utf8x]{inputenc} % utf8-Kodierung und direkte Eingabe von Sonderzeichen
\usepackage{fixltx2e} % Verbessert einige Kernkompetenzen von LaTeX2e

%%%%%%%% PAKETE: AMS-Pakete
\usepackage{amsmath} % Mathe-Erweiterung
\usepackage{amsfonts} % Schrift-Erweiterung
\usepackage{amssymb} % Sonderzeichen-Erweiterung

%%%%%%%% PAKETE: Sonstiges
\usepackage[colorlinks, citecolor=black, filecolor=black, linkcolor=black, urlcolor=black]{hyperref} % Links
\usepackage{wrapfig} % ausgeklügekte Floatumgebung
\usepackage{float} % normale Floatumgebung
\restylefloat{figure} % ermöglicht die Verwendung von "H" (ist noch stärker als "h!")
\usepackage[small,it,singlelinecheck=false]{caption} % Bildunterschriften formatieren
\usepackage{multirow} % ermöglich Verbinden von Tabellenzeilen
\usepackage{multicol} % ermöglicht Spalten
\usepackage{fancyhdr} % ermöglicht Kopf- und Fußzeilen
\usepackage{graphicx} % Einbinden von Bildern möglich
\usepackage{units} % Einheiten
\usepackage{subcaption}

%%%%%%%% DEFINITIONEN: Titelseite
\author{April Cooper, Patrick Kreissl und Sebastian Weber}
\title{Worksheet 3: Diffusion processes and atomistic water model properties}
\publishers{University of Stuttgart}
\date{\today}

%%%%%%%% ANPASSUNGEN: Kopf-und Fußzeile
\fancypagestyle{plain}{} % redefine the plain pagestyle to match the fancy layout
\pagestyle{fancy} % aktiviere eigenen Seitenstil
\fancyhf{} % alle Kopf- und Fußzeilen bereinigen
\fancyhead[L]{Worksheet 3: Diffusion processes and atomistic water model properties}
\fancyhead[R]{\today}
\renewcommand{\headrulewidth}{0.6pt} % obere Trennlinie
\fancyfoot[L]{April Cooper, Patrick Kreissl und Sebastian Weber}
\fancyfoot[R]{Page \thepage}
\renewcommand{\footrulewidth}{0.6pt} % untere Trennlinie

%%%%%%%% ANPASSUNGEN: Absätze
\setlength{\parindent}{0em} % keine Absatzeinzüge
\setlength{\parskip}{0.5em} % Absatz-Abstand

%%%%%%%% ANPASSUNGEN: Abbildungsverzeichnis
\usepackage{tocloft} % Zum Anpassen der Verzeichnisse
%\renewcommand{\cftfigpresnum}{Abb. }
%\renewcommand{\cfttabpresnum}{Tab. }
\renewcommand{\cftfigaftersnum}{:}
\renewcommand{\cfttabaftersnum}{:}
\setlength{\cftfignumwidth}{2cm}
\setlength{\cfttabnumwidth}{2cm}
\setlength{\cftfigindent}{0cm}
\setlength{\cfttabindent}{0cm}

%%%%%%%% SONSTIGES
\usepackage{pdfpages}
\usepackage{pgf}
%\usepackage{subfigure}
\usepackage{graphicx}
\usepackage{caption}
\usepackage{subcaption}


% NÜTZLICH: http://truben.no/latex/table/

% Anfang des eigentlichen Dokuments
\begin{document}

\maketitle
\tableofcontents
\newpage

\section{Short Questions - Short Answers}


\subsubsection*{What are the main differences between various atomistic water models?}
\begin{itemize}
\item Geometry - some are planar, some tetrahedral, also the location and size of partial charges can differ 
\item Polarizability - some models take it into account some don't
\item Rigidness - some have fixed atom positions, others model atoms connected by "springs"
\end{itemize}

\subsubsection*{What is the difference between the SPC and the SPC/E water model? }
The SPC/E model takes the averaged polarization effects  into account, SPC doesn't.

\subsubsection*{What are the typical terms in an atomistic classical force field?}
Typical terms for the potential are: $E_{bond}$,$E_{torsion}$,$E_{angular}$,$E_{van-der-Waals}$, $E_{LJ}$ and 
$E_{coulomb}$

\subsubsection*{How is the Pauli exclusion principle incorporated into a classical force field?}
It is incorporated into the energy expression of the Lennard-Jones interactions $E_{LJ}$. If two (non-bonded) atoms get too close to each other their electron clouds overlap which results due to Pauli repulsion in a very strong repulsive force between these atoms. In the Lennard-Jones  potential the $r^{-12}$- term describes this strong (Pauli -) repulsion.

\section{Theoretical Task: Langevin equation - Calculation of particle
positions and velocities}
In this theoretical task, the Langevin equation describing the Brownian motion has to be solved:

\begin{equation}
	\text{d}v = - \gamma v \text ~ \text{d}t + \frac{\Gamma}{m}~\text{d}W
\end{equation}

The first term on the right hand side describes the dissipative force, the second the stochastic force.

\newpage
\subsection{Velocities of the particle}
Since the average force in the Langevin equation is already included in the first force term, the stochastic second one has to be zero on average: 
\begin{equation}
	\langle \text{d}W(t) \rangle = 0
\end{equation}
Therefore the second term can be neglected if one is only interested in computing the average force (force term one):
\begin{equation}
	\text{d} v = - \gamma v ~ \text{d} t
\end{equation}
This differential equation can be easily solved by separation of variables, which leads to to following solution (with $v_0 = v(t=0)$):
\begin{equation} 
	v(t) = v_0 \cdot \text{e}^{(-\gamma t)}
\end{equation}
The stochastic fluctuations of the second term also fulfill the following relation:
\begin{equation}
	\langle\text{d}W(t_1)\text{d}W(t_2)\rangle =  \delta_{t_1,t_2} ~ \text{d}t_1
\end{equation}
An explicit formal solution can be obtained as
\begin{equation}
	v(t) = v_0 \cdot \text{e}^{-\gamma t} + \frac{\Gamma}{m} \int_0^t \text{e}^ {-\gamma(t-s)}~\text{d}W(s)
	\label{eqn:v}
\end{equation}
Now one can calculate:
\begin{align}
	\langle v(t_1)v(t_2) \rangle &= \langle v_0 \cdot  v_0\rangle \cdot \text{e}^{-\gamma (t_1+t_2)}
 + \left( \frac{\Gamma}{m} \right)^2 \int_0^{t_1} \int_0^{t_2} \text{e}^ {-\gamma\cdot(t_1+t_2-(s_1+s_2))}~\langle\text{d}W(s_2)\text{d}W(s_1)\rangle\\
	&= \langle v_0^2\rangle \cdot \text{e}^{-\gamma\cdot (t_1+t_2)}	
	 + \left( \frac{\Gamma}{m} \right)^2  \int_0^{\min{t_1, t_2}} \text{e}^ {-\gamma\cdot(t_1+t_2-2s)}~\text{d}s\\
	&= \langle v_0^2\rangle \cdot \text{e}^{-\gamma\cdot (t_1+t_2)}+ \left( \frac{\Gamma}{m} \right)^2 \frac{1}{2 \gamma} \text{e}^ {-\gamma\cdot(t_1+t_2)}\left( \text{e}^ {2\gamma \min{t_1, t_2}} - 1 \right)
\end{align}
For $t_1=t_2=t$ this results in
\begin{align}
	\langle v(t)^2\rangle &= \langle v_0(t)^2\rangle \cdot \text{e}^{-2\gamma t}+ \left( \frac{\Gamma}{m} \right)^2 \frac{1}{2 \gamma} \text{e}^ {-2\gamma t}\left( \text{e}^ {2\gamma t} - 1 \right)\\
	&=\langle v_0(t)^2\rangle \cdot \text{e}^{-2\gamma t} - \frac{\Gamma}{2 m\gamma} \text{e}^ {-2\gamma t} + \left( \frac{\Gamma}{m} \right)^2 \frac{1}{2 \gamma}\\
	&=\left( \langle v_0(t)^2\rangle - \frac{\Gamma}{2 m\gamma} \right) \text{e}^ {-2\gamma t} + \left( \frac{\Gamma}{m} \right)^2 \frac{1}{2 \gamma} \label{eq:velsqrt}
\end{align}
Since in equilibrium we must have the equipartition theorem for one dimension ($\langle v(t)^2\rangle_{\text{eq}} = k_\text{B} \cdot T / m$, in three dimensions the right hand side would have to be multiplied by three) $\Gamma$ can be calculated (for $t \rightarrow \infty$ the first part of equation \ref{eq:velsqrt} vanishes):
\begin{equation}
	\Gamma = \sqrt{2 \gamma k_\text{B} T m}
\end{equation}

\subsection{Position of the particle}

By integrating equation \ref{eqn:v} we get an expression for the position.

\begin{align}
x(t) &= x_0 +  \int_0^t \left[ v_0 \cdot \text{e}^{-\gamma u} + \frac{\Gamma}{m} \int_0^u \text{e}^ {-\gamma(u-s)}~\text{d}W(s) \right] du \\
&= x_0 + \frac{v_0}{\gamma} \left(1-\text{e}^{-\gamma t} \right) + \frac{\Gamma}{m} \int_0^t \text{d}W(s) \int_s^t \text{e}^{-\gamma (u-s)} du \\
&= x_0 + \frac{v_0}{\gamma} \left(1-\text{e}^{-\gamma t} \right) +\frac{\Gamma}{m \gamma} \int_0^t \left(1-\text{e}^{-\gamma (t-s)} \right) ~\text{d}W(s) \label{eqn:x}
\end{align}
For the mean position, the last term gets zero because of it dependents on $\text{d}W(s)$. So the result is:
\begin{equation}
\langle x(t) \rangle = x_0 + \frac{v_0}{\gamma} \left(1-\text{e}^{-\gamma t} \right)
\end{equation}
This gives us in the limit of ...
\begin{itemize}
\item short time scales ($\gamma t << 1$) : $\langle x(t) \rangle \approx x_0 + v_0 \cdot t$
\item long time scales ($\tfrac{1}{\gamma t} << 1$) : $\langle x(t) \rangle \approx x_0 + \frac{v_0}{\gamma}$
\end{itemize} \vspace{2em}

In addition, equation \ref{eqn:x} results in the following mean-square displacement:
\begin{align}
\langle \Delta x(t)^2 \rangle &= \langle \left( x(t) - \langle x(t) \rangle \right)^2 \rangle = \langle x(t)^2 \rangle - \left( \langle x(t)\rangle \right)^2 \\
&= \left(\frac{\Gamma}{m \gamma} \right)^2 \int_0^t \int_0^t \left(1-\text{e}^{-\gamma (t-s_1)} \right) \left(1-\text{e}^{-\gamma (t-s_2)} \right) \langle ~\text{d}W(s_1) ~\text{d}W(s_2) \rangle \\
&= \left( \frac{\Gamma}{m \gamma} \right)^2 \int_0^t \left(1-\text{e}^{-\gamma (t-s)} \right)^2 ~\text{d}s \\
&= \frac{2 k_B T}{m \gamma} \int_0^t 1+\text{e}^{-2\gamma (t-s)}-2\text{e}^{-\gamma (t-s)} ~\text{d}s \\
&= \frac{2 k_B T}{m \gamma} \left( t + \frac{4\text{e}^{-\gamma t}-\text{e}^{-2\gamma t}}{2 \gamma} - \frac{3}{2g}\right)
%&= \frac{2 k_B T}{m \gamma} \left( t + \frac{1}{2 \gamma} (1-\text{e}^{-2 \gamma t}) - \frac{2}{\gamma} (1-\text{e}^{-\gamma t} ) \right)
%&\approx \frac{2 k_B T}{m \gamma} \left( t + \frac{1}{2 \gamma} (2 \gamma t + \frac{(2 \gamma t)^2}{2}) - \frac{2}{\gamma} (-\gamma t+ \frac{(\gamma t)^2}{2}) \right) \\
%&=\frac{2 k_B T}{m \gamma} \left( t + \frac{1}{2 \gamma} (2 \gamma t - \frac{(2 \gamma t)^2}{2}) - \frac{2}{\gamma} (\gamma t- \frac{(\gamma t)^2}{2}) \right)
\end{align}

%This results in the limit of ...
%\begin{itemize}
%\item short time scales: $\langle \Delta x(t)^2 \rangle  \varpropto \frac{k_B T}{m} t^2$
%\item long time scales: $\langle \Delta x(t)^2 \rangle \varpropto \frac{2 k_B T}{m \gamma} t$
%\end{itemize}

For long time scales the exponential functions approx zero. It results in:
\begin{equation}
\langle \Delta x(t)^2 \rangle \varpropto \frac{2 k_B T}{m \gamma} t
\end{equation}

With the equation  $\langle \Delta x(t)^2 \rangle = 2 D t$ (for three dimensions $6 D t$) you get the following diffusion constant:
\begin{equation}
D = \frac{k_B T}{m \gamma}
\end{equation}


\section{Computational Task: Atomistic water simulations with GROMACS}

\subsection{Radial distribution function}
\begin{figure}[H]
	\resizebox{\linewidth}{!}{%% Creator: Matplotlib, PGF backend
%%
%% To include the figure in your LaTeX document, write
%%   \input{<filename>.pgf}
%%
%% Make sure the required packages are loaded in your preamble
%%   \usepackage{pgf}
%%
%% Figures using additional raster images can only be included by \input if
%% they are in the same directory as the main LaTeX file. For loading figures
%% from other directories you can use the `import` package
%%   \usepackage{import}
%% and then include the figures with
%%   \import{<path to file>}{<filename>.pgf}
%%
%% Matplotlib used the following preamble
%%   \usepackage{fontspec}
%%   \setmainfont{DejaVu Serif}
%%   \setsansfont{DejaVu Sans}
%%   \setmonofont{DejaVu Sans Mono}
%%
\begingroup%
\makeatletter%
\begin{pgfpicture}%
\pgfpathrectangle{\pgfpointorigin}{\pgfqpoint{8.000000in}{3.500000in}}%
\pgfusepath{use as bounding box}%
\begin{pgfscope}%
\pgfsetrectcap%
\pgfsetroundjoin%
\definecolor{currentfill}{rgb}{1.000000,1.000000,1.000000}%
\pgfsetfillcolor{currentfill}%
\pgfsetlinewidth{0.000000pt}%
\definecolor{currentstroke}{rgb}{1.000000,1.000000,1.000000}%
\pgfsetstrokecolor{currentstroke}%
\pgfsetdash{}{0pt}%
\pgfpathmoveto{\pgfqpoint{0.000000in}{0.000000in}}%
\pgfpathlineto{\pgfqpoint{8.000000in}{0.000000in}}%
\pgfpathlineto{\pgfqpoint{8.000000in}{3.500000in}}%
\pgfpathlineto{\pgfqpoint{0.000000in}{3.500000in}}%
\pgfpathclose%
\pgfusepath{fill}%
\end{pgfscope}%
\begin{pgfscope}%
\pgfsetrectcap%
\pgfsetroundjoin%
\definecolor{currentfill}{rgb}{1.000000,1.000000,1.000000}%
\pgfsetfillcolor{currentfill}%
\pgfsetlinewidth{0.000000pt}%
\definecolor{currentstroke}{rgb}{0.000000,0.000000,0.000000}%
\pgfsetstrokecolor{currentstroke}%
\pgfsetdash{}{0pt}%
\pgfpathmoveto{\pgfqpoint{1.000000in}{0.350000in}}%
\pgfpathlineto{\pgfqpoint{7.200000in}{0.350000in}}%
\pgfpathlineto{\pgfqpoint{7.200000in}{3.150000in}}%
\pgfpathlineto{\pgfqpoint{1.000000in}{3.150000in}}%
\pgfpathclose%
\pgfusepath{fill}%
\end{pgfscope}%
\begin{pgfscope}%
\pgfpathrectangle{\pgfqpoint{1.000000in}{0.350000in}}{\pgfqpoint{6.200000in}{2.800000in}} %
\pgfusepath{clip}%
\pgfsetrectcap%
\pgfsetroundjoin%
\pgfsetlinewidth{1.003750pt}%
\definecolor{currentstroke}{rgb}{0.000000,0.000000,1.000000}%
\pgfsetstrokecolor{currentstroke}%
\pgfsetdash{}{0pt}%
\pgfpathmoveto{\pgfqpoint{1.000000in}{0.350000in}}%
\pgfpathlineto{\pgfqpoint{2.500400in}{0.350461in}}%
\pgfpathlineto{\pgfqpoint{2.512800in}{0.351452in}}%
\pgfpathlineto{\pgfqpoint{2.525200in}{0.354999in}}%
\pgfpathlineto{\pgfqpoint{2.537600in}{0.366397in}}%
\pgfpathlineto{\pgfqpoint{2.550000in}{0.389446in}}%
\pgfpathlineto{\pgfqpoint{2.562400in}{0.435840in}}%
\pgfpathlineto{\pgfqpoint{2.574800in}{0.517536in}}%
\pgfpathlineto{\pgfqpoint{2.587200in}{0.641090in}}%
\pgfpathlineto{\pgfqpoint{2.599600in}{0.817422in}}%
\pgfpathlineto{\pgfqpoint{2.612000in}{1.040588in}}%
\pgfpathlineto{\pgfqpoint{2.661600in}{2.095000in}}%
\pgfpathlineto{\pgfqpoint{2.674000in}{2.298888in}}%
\pgfpathlineto{\pgfqpoint{2.686400in}{2.462632in}}%
\pgfpathlineto{\pgfqpoint{2.698800in}{2.562552in}}%
\pgfpathlineto{\pgfqpoint{2.711200in}{2.616680in}}%
\pgfpathlineto{\pgfqpoint{2.723600in}{2.625680in}}%
\pgfpathlineto{\pgfqpoint{2.736000in}{2.585368in}}%
\pgfpathlineto{\pgfqpoint{2.748400in}{2.521704in}}%
\pgfpathlineto{\pgfqpoint{2.760800in}{2.426592in}}%
\pgfpathlineto{\pgfqpoint{2.785600in}{2.205016in}}%
\pgfpathlineto{\pgfqpoint{2.810400in}{1.959088in}}%
\pgfpathlineto{\pgfqpoint{2.822800in}{1.867264in}}%
\pgfpathlineto{\pgfqpoint{2.835200in}{1.741512in}}%
\pgfpathlineto{\pgfqpoint{2.847600in}{1.667328in}}%
\pgfpathlineto{\pgfqpoint{2.860000in}{1.577952in}}%
\pgfpathlineto{\pgfqpoint{2.884800in}{1.435064in}}%
\pgfpathlineto{\pgfqpoint{2.897200in}{1.374616in}}%
\pgfpathlineto{\pgfqpoint{2.909600in}{1.324680in}}%
\pgfpathlineto{\pgfqpoint{2.922000in}{1.284032in}}%
\pgfpathlineto{\pgfqpoint{2.946800in}{1.211288in}}%
\pgfpathlineto{\pgfqpoint{2.971600in}{1.159664in}}%
\pgfpathlineto{\pgfqpoint{3.008800in}{1.102794in}}%
\pgfpathlineto{\pgfqpoint{3.021200in}{1.096791in}}%
\pgfpathlineto{\pgfqpoint{3.033600in}{1.101372in}}%
\pgfpathlineto{\pgfqpoint{3.046000in}{1.079921in}}%
\pgfpathlineto{\pgfqpoint{3.058400in}{1.070282in}}%
\pgfpathlineto{\pgfqpoint{3.070800in}{1.072646in}}%
\pgfpathlineto{\pgfqpoint{3.083200in}{1.068967in}}%
\pgfpathlineto{\pgfqpoint{3.095600in}{1.067910in}}%
\pgfpathlineto{\pgfqpoint{3.108000in}{1.061810in}}%
\pgfpathlineto{\pgfqpoint{3.132800in}{1.057938in}}%
\pgfpathlineto{\pgfqpoint{3.157600in}{1.060258in}}%
\pgfpathlineto{\pgfqpoint{3.170000in}{1.059410in}}%
\pgfpathlineto{\pgfqpoint{3.182400in}{1.062355in}}%
\pgfpathlineto{\pgfqpoint{3.194800in}{1.058818in}}%
\pgfpathlineto{\pgfqpoint{3.207200in}{1.064017in}}%
\pgfpathlineto{\pgfqpoint{3.219600in}{1.063510in}}%
\pgfpathlineto{\pgfqpoint{3.232000in}{1.065397in}}%
\pgfpathlineto{\pgfqpoint{3.256800in}{1.072977in}}%
\pgfpathlineto{\pgfqpoint{3.269200in}{1.074415in}}%
\pgfpathlineto{\pgfqpoint{3.281600in}{1.078570in}}%
\pgfpathlineto{\pgfqpoint{3.294000in}{1.081422in}}%
\pgfpathlineto{\pgfqpoint{3.306400in}{1.075846in}}%
\pgfpathlineto{\pgfqpoint{3.318800in}{1.081504in}}%
\pgfpathlineto{\pgfqpoint{3.331200in}{1.082770in}}%
\pgfpathlineto{\pgfqpoint{3.343600in}{1.090947in}}%
\pgfpathlineto{\pgfqpoint{3.380800in}{1.099057in}}%
\pgfpathlineto{\pgfqpoint{3.405600in}{1.106490in}}%
\pgfpathlineto{\pgfqpoint{3.418000in}{1.107384in}}%
\pgfpathlineto{\pgfqpoint{3.430400in}{1.110536in}}%
\pgfpathlineto{\pgfqpoint{3.455200in}{1.123650in}}%
\pgfpathlineto{\pgfqpoint{3.467600in}{1.120113in}}%
\pgfpathlineto{\pgfqpoint{3.504800in}{1.138795in}}%
\pgfpathlineto{\pgfqpoint{3.517200in}{1.134814in}}%
\pgfpathlineto{\pgfqpoint{3.529600in}{1.143116in}}%
\pgfpathlineto{\pgfqpoint{3.542000in}{1.148637in}}%
\pgfpathlineto{\pgfqpoint{3.554400in}{1.155840in}}%
\pgfpathlineto{\pgfqpoint{3.566800in}{1.151752in}}%
\pgfpathlineto{\pgfqpoint{3.579200in}{1.163784in}}%
\pgfpathlineto{\pgfqpoint{3.591600in}{1.162496in}}%
\pgfpathlineto{\pgfqpoint{3.616400in}{1.168536in}}%
\pgfpathlineto{\pgfqpoint{3.628800in}{1.176264in}}%
\pgfpathlineto{\pgfqpoint{3.653600in}{1.175120in}}%
\pgfpathlineto{\pgfqpoint{3.666000in}{1.181872in}}%
\pgfpathlineto{\pgfqpoint{3.678400in}{1.182432in}}%
\pgfpathlineto{\pgfqpoint{3.690800in}{1.190448in}}%
\pgfpathlineto{\pgfqpoint{3.703200in}{1.187408in}}%
\pgfpathlineto{\pgfqpoint{3.740400in}{1.198480in}}%
\pgfpathlineto{\pgfqpoint{3.752800in}{1.200616in}}%
\pgfpathlineto{\pgfqpoint{3.765200in}{1.192864in}}%
\pgfpathlineto{\pgfqpoint{3.790000in}{1.199584in}}%
\pgfpathlineto{\pgfqpoint{3.802400in}{1.197072in}}%
\pgfpathlineto{\pgfqpoint{3.814800in}{1.200880in}}%
\pgfpathlineto{\pgfqpoint{3.839600in}{1.196776in}}%
\pgfpathlineto{\pgfqpoint{3.864400in}{1.195984in}}%
\pgfpathlineto{\pgfqpoint{3.876800in}{1.188208in}}%
\pgfpathlineto{\pgfqpoint{3.889200in}{1.197464in}}%
\pgfpathlineto{\pgfqpoint{3.901600in}{1.199304in}}%
\pgfpathlineto{\pgfqpoint{3.951200in}{1.188536in}}%
\pgfpathlineto{\pgfqpoint{3.963600in}{1.183136in}}%
\pgfpathlineto{\pgfqpoint{3.976000in}{1.185632in}}%
\pgfpathlineto{\pgfqpoint{3.988400in}{1.186008in}}%
\pgfpathlineto{\pgfqpoint{4.000800in}{1.184792in}}%
\pgfpathlineto{\pgfqpoint{4.013200in}{1.175744in}}%
\pgfpathlineto{\pgfqpoint{4.025600in}{1.176264in}}%
\pgfpathlineto{\pgfqpoint{4.038000in}{1.177904in}}%
\pgfpathlineto{\pgfqpoint{4.050400in}{1.171848in}}%
\pgfpathlineto{\pgfqpoint{4.062800in}{1.172392in}}%
\pgfpathlineto{\pgfqpoint{4.075200in}{1.167696in}}%
\pgfpathlineto{\pgfqpoint{4.087600in}{1.168456in}}%
\pgfpathlineto{\pgfqpoint{4.100000in}{1.163008in}}%
\pgfpathlineto{\pgfqpoint{4.112400in}{1.164136in}}%
\pgfpathlineto{\pgfqpoint{4.124800in}{1.155824in}}%
\pgfpathlineto{\pgfqpoint{4.137200in}{1.154272in}}%
\pgfpathlineto{\pgfqpoint{4.149600in}{1.155032in}}%
\pgfpathlineto{\pgfqpoint{4.162000in}{1.153808in}}%
\pgfpathlineto{\pgfqpoint{4.174400in}{1.150944in}}%
\pgfpathlineto{\pgfqpoint{4.186800in}{1.149984in}}%
\pgfpathlineto{\pgfqpoint{4.199200in}{1.147770in}}%
\pgfpathlineto{\pgfqpoint{4.211600in}{1.140789in}}%
\pgfpathlineto{\pgfqpoint{4.224000in}{1.138250in}}%
\pgfpathlineto{\pgfqpoint{4.236400in}{1.138426in}}%
\pgfpathlineto{\pgfqpoint{4.248800in}{1.135121in}}%
\pgfpathlineto{\pgfqpoint{4.261200in}{1.136905in}}%
\pgfpathlineto{\pgfqpoint{4.273600in}{1.129222in}}%
\pgfpathlineto{\pgfqpoint{4.286000in}{1.126374in}}%
\pgfpathlineto{\pgfqpoint{4.310800in}{1.126606in}}%
\pgfpathlineto{\pgfqpoint{4.348000in}{1.118412in}}%
\pgfpathlineto{\pgfqpoint{4.360400in}{1.117622in}}%
\pgfpathlineto{\pgfqpoint{4.385200in}{1.109515in}}%
\pgfpathlineto{\pgfqpoint{4.397600in}{1.108998in}}%
\pgfpathlineto{\pgfqpoint{4.410000in}{1.106697in}}%
\pgfpathlineto{\pgfqpoint{4.422400in}{1.102851in}}%
\pgfpathlineto{\pgfqpoint{4.434800in}{1.097734in}}%
\pgfpathlineto{\pgfqpoint{4.447200in}{1.102070in}}%
\pgfpathlineto{\pgfqpoint{4.459600in}{1.097085in}}%
\pgfpathlineto{\pgfqpoint{4.472000in}{1.096787in}}%
\pgfpathlineto{\pgfqpoint{4.496800in}{1.098658in}}%
\pgfpathlineto{\pgfqpoint{4.509200in}{1.092780in}}%
\pgfpathlineto{\pgfqpoint{4.521600in}{1.098177in}}%
\pgfpathlineto{\pgfqpoint{4.534000in}{1.092990in}}%
\pgfpathlineto{\pgfqpoint{4.546400in}{1.092944in}}%
\pgfpathlineto{\pgfqpoint{4.558800in}{1.097518in}}%
\pgfpathlineto{\pgfqpoint{4.571200in}{1.096010in}}%
\pgfpathlineto{\pgfqpoint{4.583600in}{1.098092in}}%
\pgfpathlineto{\pgfqpoint{4.608400in}{1.098692in}}%
\pgfpathlineto{\pgfqpoint{4.620800in}{1.101321in}}%
\pgfpathlineto{\pgfqpoint{4.633200in}{1.101790in}}%
\pgfpathlineto{\pgfqpoint{4.645600in}{1.104232in}}%
\pgfpathlineto{\pgfqpoint{4.658000in}{1.102274in}}%
\pgfpathlineto{\pgfqpoint{4.670400in}{1.104502in}}%
\pgfpathlineto{\pgfqpoint{4.682800in}{1.108556in}}%
\pgfpathlineto{\pgfqpoint{4.695200in}{1.107040in}}%
\pgfpathlineto{\pgfqpoint{4.720000in}{1.115915in}}%
\pgfpathlineto{\pgfqpoint{4.732400in}{1.113854in}}%
\pgfpathlineto{\pgfqpoint{4.744800in}{1.119067in}}%
\pgfpathlineto{\pgfqpoint{4.757200in}{1.118654in}}%
\pgfpathlineto{\pgfqpoint{4.769600in}{1.120197in}}%
\pgfpathlineto{\pgfqpoint{4.782000in}{1.126132in}}%
\pgfpathlineto{\pgfqpoint{4.794400in}{1.125161in}}%
\pgfpathlineto{\pgfqpoint{4.806800in}{1.131188in}}%
\pgfpathlineto{\pgfqpoint{4.819200in}{1.132588in}}%
\pgfpathlineto{\pgfqpoint{4.831600in}{1.129732in}}%
\pgfpathlineto{\pgfqpoint{4.844000in}{1.134229in}}%
\pgfpathlineto{\pgfqpoint{4.856400in}{1.134186in}}%
\pgfpathlineto{\pgfqpoint{4.868800in}{1.141806in}}%
\pgfpathlineto{\pgfqpoint{4.893600in}{1.146075in}}%
\pgfpathlineto{\pgfqpoint{4.906000in}{1.143191in}}%
\pgfpathlineto{\pgfqpoint{4.930800in}{1.157576in}}%
\pgfpathlineto{\pgfqpoint{4.943200in}{1.157896in}}%
\pgfpathlineto{\pgfqpoint{4.955600in}{1.153888in}}%
\pgfpathlineto{\pgfqpoint{4.968000in}{1.158136in}}%
\pgfpathlineto{\pgfqpoint{5.005200in}{1.162048in}}%
\pgfpathlineto{\pgfqpoint{5.017600in}{1.166112in}}%
\pgfpathlineto{\pgfqpoint{5.030000in}{1.164448in}}%
\pgfpathlineto{\pgfqpoint{5.042400in}{1.167920in}}%
\pgfpathlineto{\pgfqpoint{5.054800in}{1.165696in}}%
\pgfpathlineto{\pgfqpoint{5.067200in}{1.165184in}}%
\pgfpathlineto{\pgfqpoint{5.079600in}{1.168800in}}%
\pgfpathlineto{\pgfqpoint{5.116800in}{1.169920in}}%
\pgfpathlineto{\pgfqpoint{5.129200in}{1.174072in}}%
\pgfpathlineto{\pgfqpoint{5.141600in}{1.170832in}}%
\pgfpathlineto{\pgfqpoint{5.154000in}{1.175344in}}%
\pgfpathlineto{\pgfqpoint{5.166400in}{1.175896in}}%
\pgfpathlineto{\pgfqpoint{5.191200in}{1.180096in}}%
\pgfpathlineto{\pgfqpoint{5.216000in}{1.177176in}}%
\pgfpathlineto{\pgfqpoint{5.228400in}{1.181144in}}%
\pgfpathlineto{\pgfqpoint{5.240800in}{1.179296in}}%
\pgfpathlineto{\pgfqpoint{5.253200in}{1.181608in}}%
\pgfpathlineto{\pgfqpoint{5.265600in}{1.176320in}}%
\pgfpathlineto{\pgfqpoint{5.278000in}{1.181488in}}%
\pgfpathlineto{\pgfqpoint{5.302800in}{1.179960in}}%
\pgfpathlineto{\pgfqpoint{5.315200in}{1.178256in}}%
\pgfpathlineto{\pgfqpoint{5.327600in}{1.177992in}}%
\pgfpathlineto{\pgfqpoint{5.340000in}{1.180888in}}%
\pgfpathlineto{\pgfqpoint{5.352400in}{1.175720in}}%
\pgfpathlineto{\pgfqpoint{5.364800in}{1.177224in}}%
\pgfpathlineto{\pgfqpoint{5.377200in}{1.175736in}}%
\pgfpathlineto{\pgfqpoint{5.402000in}{1.174256in}}%
\pgfpathlineto{\pgfqpoint{5.414400in}{1.173040in}}%
\pgfpathlineto{\pgfqpoint{5.426800in}{1.170456in}}%
\pgfpathlineto{\pgfqpoint{5.439200in}{1.172696in}}%
\pgfpathlineto{\pgfqpoint{5.451600in}{1.169296in}}%
\pgfpathlineto{\pgfqpoint{5.476400in}{1.171232in}}%
\pgfpathlineto{\pgfqpoint{5.538400in}{1.164456in}}%
\pgfpathlineto{\pgfqpoint{5.550800in}{1.165704in}}%
\pgfpathlineto{\pgfqpoint{5.563200in}{1.159800in}}%
\pgfpathlineto{\pgfqpoint{5.575600in}{1.163512in}}%
\pgfpathlineto{\pgfqpoint{5.588000in}{1.160568in}}%
\pgfpathlineto{\pgfqpoint{5.600400in}{1.160496in}}%
\pgfpathlineto{\pgfqpoint{5.612800in}{1.156576in}}%
\pgfpathlineto{\pgfqpoint{5.625200in}{1.158232in}}%
\pgfpathlineto{\pgfqpoint{5.674800in}{1.153104in}}%
\pgfpathlineto{\pgfqpoint{5.687200in}{1.152728in}}%
\pgfpathlineto{\pgfqpoint{5.699600in}{1.150552in}}%
\pgfpathlineto{\pgfqpoint{5.712000in}{1.151872in}}%
\pgfpathlineto{\pgfqpoint{5.724400in}{1.148903in}}%
\pgfpathlineto{\pgfqpoint{5.736800in}{1.150776in}}%
\pgfpathlineto{\pgfqpoint{5.786400in}{1.144028in}}%
\pgfpathlineto{\pgfqpoint{5.798800in}{1.146785in}}%
\pgfpathlineto{\pgfqpoint{5.811200in}{1.146771in}}%
\pgfpathlineto{\pgfqpoint{5.836000in}{1.143955in}}%
\pgfpathlineto{\pgfqpoint{5.848400in}{1.143442in}}%
\pgfpathlineto{\pgfqpoint{5.860800in}{1.140650in}}%
\pgfpathlineto{\pgfqpoint{5.885600in}{1.141678in}}%
\pgfpathlineto{\pgfqpoint{5.898000in}{1.139192in}}%
\pgfpathlineto{\pgfqpoint{5.910400in}{1.138070in}}%
\pgfpathlineto{\pgfqpoint{5.947600in}{1.139078in}}%
\pgfpathlineto{\pgfqpoint{5.960000in}{1.136734in}}%
\pgfpathlineto{\pgfqpoint{5.972400in}{1.141408in}}%
\pgfpathlineto{\pgfqpoint{5.984800in}{1.139166in}}%
\pgfpathlineto{\pgfqpoint{5.997200in}{1.141811in}}%
\pgfpathlineto{\pgfqpoint{6.009600in}{1.138178in}}%
\pgfpathlineto{\pgfqpoint{6.022000in}{1.139374in}}%
\pgfpathlineto{\pgfqpoint{6.034400in}{1.136849in}}%
\pgfpathlineto{\pgfqpoint{6.059200in}{1.138397in}}%
\pgfpathlineto{\pgfqpoint{6.071600in}{1.136427in}}%
\pgfpathlineto{\pgfqpoint{6.096400in}{1.141136in}}%
\pgfpathlineto{\pgfqpoint{6.108800in}{1.138822in}}%
\pgfpathlineto{\pgfqpoint{6.121200in}{1.142399in}}%
\pgfpathlineto{\pgfqpoint{6.146000in}{1.139172in}}%
\pgfpathlineto{\pgfqpoint{6.158400in}{1.136557in}}%
\pgfpathlineto{\pgfqpoint{6.170800in}{1.140678in}}%
\pgfpathlineto{\pgfqpoint{6.183200in}{1.141116in}}%
\pgfpathlineto{\pgfqpoint{6.195600in}{1.143628in}}%
\pgfpathlineto{\pgfqpoint{6.220400in}{1.141944in}}%
\pgfpathlineto{\pgfqpoint{6.232800in}{1.140911in}}%
\pgfpathlineto{\pgfqpoint{6.245200in}{1.141946in}}%
\pgfpathlineto{\pgfqpoint{6.257600in}{1.144794in}}%
\pgfpathlineto{\pgfqpoint{6.270000in}{1.142543in}}%
\pgfpathlineto{\pgfqpoint{6.307200in}{1.144843in}}%
\pgfpathlineto{\pgfqpoint{6.319600in}{1.143163in}}%
\pgfpathlineto{\pgfqpoint{6.356800in}{1.144370in}}%
\pgfpathlineto{\pgfqpoint{6.369200in}{1.146700in}}%
\pgfpathlineto{\pgfqpoint{6.381600in}{1.144902in}}%
\pgfpathlineto{\pgfqpoint{6.394000in}{1.148912in}}%
\pgfpathlineto{\pgfqpoint{6.406400in}{1.148762in}}%
\pgfpathlineto{\pgfqpoint{6.418800in}{1.145104in}}%
\pgfpathlineto{\pgfqpoint{6.431200in}{1.145996in}}%
\pgfpathlineto{\pgfqpoint{6.443600in}{1.150544in}}%
\pgfpathlineto{\pgfqpoint{6.456000in}{1.148291in}}%
\pgfpathlineto{\pgfqpoint{6.480800in}{1.148051in}}%
\pgfpathlineto{\pgfqpoint{6.493200in}{1.148224in}}%
\pgfpathlineto{\pgfqpoint{6.505600in}{1.150304in}}%
\pgfpathlineto{\pgfqpoint{6.518000in}{1.148755in}}%
\pgfpathlineto{\pgfqpoint{6.530400in}{1.149467in}}%
\pgfpathlineto{\pgfqpoint{6.542800in}{1.148683in}}%
\pgfpathlineto{\pgfqpoint{6.580000in}{1.151544in}}%
\pgfpathlineto{\pgfqpoint{6.592400in}{1.148333in}}%
\pgfpathlineto{\pgfqpoint{6.604800in}{1.147322in}}%
\pgfpathlineto{\pgfqpoint{6.617200in}{1.151280in}}%
\pgfpathlineto{\pgfqpoint{6.642000in}{1.152352in}}%
\pgfpathlineto{\pgfqpoint{6.654400in}{1.150720in}}%
\pgfpathlineto{\pgfqpoint{6.691600in}{1.153952in}}%
\pgfpathlineto{\pgfqpoint{6.704000in}{1.151312in}}%
\pgfpathlineto{\pgfqpoint{6.704000in}{1.151312in}}%
\pgfusepath{stroke}%
\end{pgfscope}%
\begin{pgfscope}%
\pgfpathrectangle{\pgfqpoint{1.000000in}{0.350000in}}{\pgfqpoint{6.200000in}{2.800000in}} %
\pgfusepath{clip}%
\pgfsetrectcap%
\pgfsetroundjoin%
\pgfsetlinewidth{1.003750pt}%
\definecolor{currentstroke}{rgb}{0.000000,0.500000,0.000000}%
\pgfsetstrokecolor{currentstroke}%
\pgfsetdash{}{0pt}%
\pgfpathmoveto{\pgfqpoint{1.000000in}{0.350000in}}%
\pgfpathlineto{\pgfqpoint{2.500400in}{0.350872in}}%
\pgfpathlineto{\pgfqpoint{2.512800in}{0.352720in}}%
\pgfpathlineto{\pgfqpoint{2.525200in}{0.360123in}}%
\pgfpathlineto{\pgfqpoint{2.537600in}{0.376705in}}%
\pgfpathlineto{\pgfqpoint{2.550000in}{0.412981in}}%
\pgfpathlineto{\pgfqpoint{2.562400in}{0.493846in}}%
\pgfpathlineto{\pgfqpoint{2.574800in}{0.610342in}}%
\pgfpathlineto{\pgfqpoint{2.587200in}{0.782591in}}%
\pgfpathlineto{\pgfqpoint{2.599600in}{1.017948in}}%
\pgfpathlineto{\pgfqpoint{2.612000in}{1.297032in}}%
\pgfpathlineto{\pgfqpoint{2.649200in}{2.220072in}}%
\pgfpathlineto{\pgfqpoint{2.661600in}{2.453712in}}%
\pgfpathlineto{\pgfqpoint{2.674000in}{2.629112in}}%
\pgfpathlineto{\pgfqpoint{2.686400in}{2.783968in}}%
\pgfpathlineto{\pgfqpoint{2.698800in}{2.807664in}}%
\pgfpathlineto{\pgfqpoint{2.711200in}{2.810424in}}%
\pgfpathlineto{\pgfqpoint{2.723600in}{2.739184in}}%
\pgfpathlineto{\pgfqpoint{2.748400in}{2.518784in}}%
\pgfpathlineto{\pgfqpoint{2.798000in}{1.968040in}}%
\pgfpathlineto{\pgfqpoint{2.810400in}{1.817800in}}%
\pgfpathlineto{\pgfqpoint{2.835200in}{1.587016in}}%
\pgfpathlineto{\pgfqpoint{2.860000in}{1.409872in}}%
\pgfpathlineto{\pgfqpoint{2.872400in}{1.337568in}}%
\pgfpathlineto{\pgfqpoint{2.884800in}{1.272368in}}%
\pgfpathlineto{\pgfqpoint{2.909600in}{1.178328in}}%
\pgfpathlineto{\pgfqpoint{2.922000in}{1.138915in}}%
\pgfpathlineto{\pgfqpoint{2.934400in}{1.106167in}}%
\pgfpathlineto{\pgfqpoint{2.946800in}{1.079285in}}%
\pgfpathlineto{\pgfqpoint{2.959200in}{1.057297in}}%
\pgfpathlineto{\pgfqpoint{2.971600in}{1.046064in}}%
\pgfpathlineto{\pgfqpoint{2.984000in}{1.025803in}}%
\pgfpathlineto{\pgfqpoint{2.996400in}{1.020748in}}%
\pgfpathlineto{\pgfqpoint{3.008800in}{1.008582in}}%
\pgfpathlineto{\pgfqpoint{3.021200in}{1.005598in}}%
\pgfpathlineto{\pgfqpoint{3.033600in}{1.004106in}}%
\pgfpathlineto{\pgfqpoint{3.046000in}{1.000711in}}%
\pgfpathlineto{\pgfqpoint{3.058400in}{0.992346in}}%
\pgfpathlineto{\pgfqpoint{3.070800in}{0.992761in}}%
\pgfpathlineto{\pgfqpoint{3.083200in}{0.995970in}}%
\pgfpathlineto{\pgfqpoint{3.095600in}{1.004021in}}%
\pgfpathlineto{\pgfqpoint{3.108000in}{1.003232in}}%
\pgfpathlineto{\pgfqpoint{3.120400in}{1.007820in}}%
\pgfpathlineto{\pgfqpoint{3.132800in}{1.006679in}}%
\pgfpathlineto{\pgfqpoint{3.145200in}{1.011146in}}%
\pgfpathlineto{\pgfqpoint{3.157600in}{1.018055in}}%
\pgfpathlineto{\pgfqpoint{3.170000in}{1.017253in}}%
\pgfpathlineto{\pgfqpoint{3.194800in}{1.027322in}}%
\pgfpathlineto{\pgfqpoint{3.207200in}{1.031023in}}%
\pgfpathlineto{\pgfqpoint{3.219600in}{1.038656in}}%
\pgfpathlineto{\pgfqpoint{3.232000in}{1.033522in}}%
\pgfpathlineto{\pgfqpoint{3.244400in}{1.033419in}}%
\pgfpathlineto{\pgfqpoint{3.256800in}{1.055454in}}%
\pgfpathlineto{\pgfqpoint{3.269200in}{1.052287in}}%
\pgfpathlineto{\pgfqpoint{3.281600in}{1.059390in}}%
\pgfpathlineto{\pgfqpoint{3.294000in}{1.061436in}}%
\pgfpathlineto{\pgfqpoint{3.306400in}{1.069914in}}%
\pgfpathlineto{\pgfqpoint{3.318800in}{1.073108in}}%
\pgfpathlineto{\pgfqpoint{3.331200in}{1.079080in}}%
\pgfpathlineto{\pgfqpoint{3.343600in}{1.089822in}}%
\pgfpathlineto{\pgfqpoint{3.356000in}{1.090438in}}%
\pgfpathlineto{\pgfqpoint{3.368400in}{1.098053in}}%
\pgfpathlineto{\pgfqpoint{3.380800in}{1.101935in}}%
\pgfpathlineto{\pgfqpoint{3.393200in}{1.108361in}}%
\pgfpathlineto{\pgfqpoint{3.405600in}{1.117440in}}%
\pgfpathlineto{\pgfqpoint{3.418000in}{1.114439in}}%
\pgfpathlineto{\pgfqpoint{3.430400in}{1.131058in}}%
\pgfpathlineto{\pgfqpoint{3.442800in}{1.128233in}}%
\pgfpathlineto{\pgfqpoint{3.467600in}{1.144340in}}%
\pgfpathlineto{\pgfqpoint{3.480000in}{1.145658in}}%
\pgfpathlineto{\pgfqpoint{3.504800in}{1.162352in}}%
\pgfpathlineto{\pgfqpoint{3.529600in}{1.172000in}}%
\pgfpathlineto{\pgfqpoint{3.542000in}{1.172040in}}%
\pgfpathlineto{\pgfqpoint{3.554400in}{1.180240in}}%
\pgfpathlineto{\pgfqpoint{3.566800in}{1.180856in}}%
\pgfpathlineto{\pgfqpoint{3.579200in}{1.192952in}}%
\pgfpathlineto{\pgfqpoint{3.591600in}{1.197096in}}%
\pgfpathlineto{\pgfqpoint{3.604000in}{1.206088in}}%
\pgfpathlineto{\pgfqpoint{3.616400in}{1.208064in}}%
\pgfpathlineto{\pgfqpoint{3.628800in}{1.205208in}}%
\pgfpathlineto{\pgfqpoint{3.641200in}{1.216168in}}%
\pgfpathlineto{\pgfqpoint{3.653600in}{1.215024in}}%
\pgfpathlineto{\pgfqpoint{3.666000in}{1.219000in}}%
\pgfpathlineto{\pgfqpoint{3.678400in}{1.224192in}}%
\pgfpathlineto{\pgfqpoint{3.728000in}{1.232568in}}%
\pgfpathlineto{\pgfqpoint{3.740400in}{1.235056in}}%
\pgfpathlineto{\pgfqpoint{3.765200in}{1.231728in}}%
\pgfpathlineto{\pgfqpoint{3.777600in}{1.237024in}}%
\pgfpathlineto{\pgfqpoint{3.790000in}{1.240360in}}%
\pgfpathlineto{\pgfqpoint{3.802400in}{1.240048in}}%
\pgfpathlineto{\pgfqpoint{3.814800in}{1.241200in}}%
\pgfpathlineto{\pgfqpoint{3.827200in}{1.234896in}}%
\pgfpathlineto{\pgfqpoint{3.839600in}{1.238216in}}%
\pgfpathlineto{\pgfqpoint{3.852000in}{1.231632in}}%
\pgfpathlineto{\pgfqpoint{3.864400in}{1.229616in}}%
\pgfpathlineto{\pgfqpoint{3.876800in}{1.233808in}}%
\pgfpathlineto{\pgfqpoint{3.889200in}{1.229480in}}%
\pgfpathlineto{\pgfqpoint{3.901600in}{1.226408in}}%
\pgfpathlineto{\pgfqpoint{3.914000in}{1.220840in}}%
\pgfpathlineto{\pgfqpoint{3.926400in}{1.226512in}}%
\pgfpathlineto{\pgfqpoint{3.938800in}{1.216592in}}%
\pgfpathlineto{\pgfqpoint{3.951200in}{1.217616in}}%
\pgfpathlineto{\pgfqpoint{3.963600in}{1.214640in}}%
\pgfpathlineto{\pgfqpoint{3.976000in}{1.217296in}}%
\pgfpathlineto{\pgfqpoint{3.988400in}{1.207112in}}%
\pgfpathlineto{\pgfqpoint{4.000800in}{1.205088in}}%
\pgfpathlineto{\pgfqpoint{4.013200in}{1.197336in}}%
\pgfpathlineto{\pgfqpoint{4.025600in}{1.200256in}}%
\pgfpathlineto{\pgfqpoint{4.038000in}{1.193656in}}%
\pgfpathlineto{\pgfqpoint{4.050400in}{1.190200in}}%
\pgfpathlineto{\pgfqpoint{4.062800in}{1.183832in}}%
\pgfpathlineto{\pgfqpoint{4.087600in}{1.175872in}}%
\pgfpathlineto{\pgfqpoint{4.100000in}{1.169864in}}%
\pgfpathlineto{\pgfqpoint{4.112400in}{1.167496in}}%
\pgfpathlineto{\pgfqpoint{4.137200in}{1.155352in}}%
\pgfpathlineto{\pgfqpoint{4.149600in}{1.154520in}}%
\pgfpathlineto{\pgfqpoint{4.162000in}{1.144930in}}%
\pgfpathlineto{\pgfqpoint{4.174400in}{1.143098in}}%
\pgfpathlineto{\pgfqpoint{4.186800in}{1.144101in}}%
\pgfpathlineto{\pgfqpoint{4.199200in}{1.137533in}}%
\pgfpathlineto{\pgfqpoint{4.211600in}{1.133686in}}%
\pgfpathlineto{\pgfqpoint{4.236400in}{1.124146in}}%
\pgfpathlineto{\pgfqpoint{4.248800in}{1.125185in}}%
\pgfpathlineto{\pgfqpoint{4.273600in}{1.115172in}}%
\pgfpathlineto{\pgfqpoint{4.286000in}{1.113672in}}%
\pgfpathlineto{\pgfqpoint{4.298400in}{1.108087in}}%
\pgfpathlineto{\pgfqpoint{4.310800in}{1.104463in}}%
\pgfpathlineto{\pgfqpoint{4.323200in}{1.098729in}}%
\pgfpathlineto{\pgfqpoint{4.335600in}{1.099970in}}%
\pgfpathlineto{\pgfqpoint{4.348000in}{1.090779in}}%
\pgfpathlineto{\pgfqpoint{4.372800in}{1.087910in}}%
\pgfpathlineto{\pgfqpoint{4.385200in}{1.082652in}}%
\pgfpathlineto{\pgfqpoint{4.447200in}{1.073190in}}%
\pgfpathlineto{\pgfqpoint{4.459600in}{1.073275in}}%
\pgfpathlineto{\pgfqpoint{4.472000in}{1.069835in}}%
\pgfpathlineto{\pgfqpoint{4.496800in}{1.071303in}}%
\pgfpathlineto{\pgfqpoint{4.509200in}{1.064853in}}%
\pgfpathlineto{\pgfqpoint{4.521600in}{1.070933in}}%
\pgfpathlineto{\pgfqpoint{4.534000in}{1.068076in}}%
\pgfpathlineto{\pgfqpoint{4.546400in}{1.071161in}}%
\pgfpathlineto{\pgfqpoint{4.558800in}{1.071260in}}%
\pgfpathlineto{\pgfqpoint{4.583600in}{1.077916in}}%
\pgfpathlineto{\pgfqpoint{4.596000in}{1.082881in}}%
\pgfpathlineto{\pgfqpoint{4.608400in}{1.082369in}}%
\pgfpathlineto{\pgfqpoint{4.633200in}{1.089040in}}%
\pgfpathlineto{\pgfqpoint{4.645600in}{1.087387in}}%
\pgfpathlineto{\pgfqpoint{4.658000in}{1.093019in}}%
\pgfpathlineto{\pgfqpoint{4.670400in}{1.097329in}}%
\pgfpathlineto{\pgfqpoint{4.682800in}{1.097630in}}%
\pgfpathlineto{\pgfqpoint{4.695200in}{1.099874in}}%
\pgfpathlineto{\pgfqpoint{4.707600in}{1.104435in}}%
\pgfpathlineto{\pgfqpoint{4.720000in}{1.105465in}}%
\pgfpathlineto{\pgfqpoint{4.744800in}{1.115285in}}%
\pgfpathlineto{\pgfqpoint{4.757200in}{1.114692in}}%
\pgfpathlineto{\pgfqpoint{4.769600in}{1.116460in}}%
\pgfpathlineto{\pgfqpoint{4.782000in}{1.121412in}}%
\pgfpathlineto{\pgfqpoint{4.806800in}{1.127609in}}%
\pgfpathlineto{\pgfqpoint{4.819200in}{1.130264in}}%
\pgfpathlineto{\pgfqpoint{4.831600in}{1.135486in}}%
\pgfpathlineto{\pgfqpoint{4.844000in}{1.133481in}}%
\pgfpathlineto{\pgfqpoint{4.856400in}{1.139556in}}%
\pgfpathlineto{\pgfqpoint{4.881200in}{1.147874in}}%
\pgfpathlineto{\pgfqpoint{4.893600in}{1.150768in}}%
\pgfpathlineto{\pgfqpoint{4.906000in}{1.150696in}}%
\pgfpathlineto{\pgfqpoint{4.918400in}{1.152944in}}%
\pgfpathlineto{\pgfqpoint{4.930800in}{1.152944in}}%
\pgfpathlineto{\pgfqpoint{4.943200in}{1.160512in}}%
\pgfpathlineto{\pgfqpoint{4.955600in}{1.161256in}}%
\pgfpathlineto{\pgfqpoint{4.968000in}{1.164712in}}%
\pgfpathlineto{\pgfqpoint{4.980400in}{1.165176in}}%
\pgfpathlineto{\pgfqpoint{4.992800in}{1.167152in}}%
\pgfpathlineto{\pgfqpoint{5.005200in}{1.163168in}}%
\pgfpathlineto{\pgfqpoint{5.017600in}{1.171544in}}%
\pgfpathlineto{\pgfqpoint{5.030000in}{1.172896in}}%
\pgfpathlineto{\pgfqpoint{5.042400in}{1.177424in}}%
\pgfpathlineto{\pgfqpoint{5.067200in}{1.176776in}}%
\pgfpathlineto{\pgfqpoint{5.079600in}{1.176944in}}%
\pgfpathlineto{\pgfqpoint{5.092000in}{1.180400in}}%
\pgfpathlineto{\pgfqpoint{5.104400in}{1.180520in}}%
\pgfpathlineto{\pgfqpoint{5.116800in}{1.186216in}}%
\pgfpathlineto{\pgfqpoint{5.129200in}{1.184232in}}%
\pgfpathlineto{\pgfqpoint{5.141600in}{1.184344in}}%
\pgfpathlineto{\pgfqpoint{5.154000in}{1.182696in}}%
\pgfpathlineto{\pgfqpoint{5.166400in}{1.183872in}}%
\pgfpathlineto{\pgfqpoint{5.191200in}{1.184512in}}%
\pgfpathlineto{\pgfqpoint{5.203600in}{1.185208in}}%
\pgfpathlineto{\pgfqpoint{5.216000in}{1.182616in}}%
\pgfpathlineto{\pgfqpoint{5.228400in}{1.186832in}}%
\pgfpathlineto{\pgfqpoint{5.240800in}{1.184944in}}%
\pgfpathlineto{\pgfqpoint{5.253200in}{1.188776in}}%
\pgfpathlineto{\pgfqpoint{5.265600in}{1.189376in}}%
\pgfpathlineto{\pgfqpoint{5.278000in}{1.185424in}}%
\pgfpathlineto{\pgfqpoint{5.290400in}{1.188336in}}%
\pgfpathlineto{\pgfqpoint{5.302800in}{1.183352in}}%
\pgfpathlineto{\pgfqpoint{5.315200in}{1.187904in}}%
\pgfpathlineto{\pgfqpoint{5.327600in}{1.181848in}}%
\pgfpathlineto{\pgfqpoint{5.340000in}{1.183472in}}%
\pgfpathlineto{\pgfqpoint{5.352400in}{1.183504in}}%
\pgfpathlineto{\pgfqpoint{5.364800in}{1.181856in}}%
\pgfpathlineto{\pgfqpoint{5.377200in}{1.182072in}}%
\pgfpathlineto{\pgfqpoint{5.389600in}{1.177448in}}%
\pgfpathlineto{\pgfqpoint{5.414400in}{1.175136in}}%
\pgfpathlineto{\pgfqpoint{5.426800in}{1.175424in}}%
\pgfpathlineto{\pgfqpoint{5.439200in}{1.172728in}}%
\pgfpathlineto{\pgfqpoint{5.451600in}{1.173872in}}%
\pgfpathlineto{\pgfqpoint{5.464000in}{1.169728in}}%
\pgfpathlineto{\pgfqpoint{5.476400in}{1.174608in}}%
\pgfpathlineto{\pgfqpoint{5.488800in}{1.169144in}}%
\pgfpathlineto{\pgfqpoint{5.501200in}{1.169360in}}%
\pgfpathlineto{\pgfqpoint{5.513600in}{1.165456in}}%
\pgfpathlineto{\pgfqpoint{5.526000in}{1.163760in}}%
\pgfpathlineto{\pgfqpoint{5.550800in}{1.166072in}}%
\pgfpathlineto{\pgfqpoint{5.563200in}{1.160776in}}%
\pgfpathlineto{\pgfqpoint{5.588000in}{1.159720in}}%
\pgfpathlineto{\pgfqpoint{5.600400in}{1.155648in}}%
\pgfpathlineto{\pgfqpoint{5.612800in}{1.156448in}}%
\pgfpathlineto{\pgfqpoint{5.625200in}{1.158400in}}%
\pgfpathlineto{\pgfqpoint{5.637600in}{1.154120in}}%
\pgfpathlineto{\pgfqpoint{5.650000in}{1.152504in}}%
\pgfpathlineto{\pgfqpoint{5.662400in}{1.152792in}}%
\pgfpathlineto{\pgfqpoint{5.674800in}{1.147836in}}%
\pgfpathlineto{\pgfqpoint{5.699600in}{1.149461in}}%
\pgfpathlineto{\pgfqpoint{5.724400in}{1.146736in}}%
\pgfpathlineto{\pgfqpoint{5.736800in}{1.142926in}}%
\pgfpathlineto{\pgfqpoint{5.749200in}{1.142130in}}%
\pgfpathlineto{\pgfqpoint{5.761600in}{1.143359in}}%
\pgfpathlineto{\pgfqpoint{5.774000in}{1.142356in}}%
\pgfpathlineto{\pgfqpoint{5.798800in}{1.143210in}}%
\pgfpathlineto{\pgfqpoint{5.811200in}{1.139153in}}%
\pgfpathlineto{\pgfqpoint{5.848400in}{1.138659in}}%
\pgfpathlineto{\pgfqpoint{5.860800in}{1.142925in}}%
\pgfpathlineto{\pgfqpoint{5.873200in}{1.138745in}}%
\pgfpathlineto{\pgfqpoint{5.885600in}{1.142633in}}%
\pgfpathlineto{\pgfqpoint{5.898000in}{1.139682in}}%
\pgfpathlineto{\pgfqpoint{5.910400in}{1.139613in}}%
\pgfpathlineto{\pgfqpoint{5.922800in}{1.137184in}}%
\pgfpathlineto{\pgfqpoint{5.960000in}{1.136831in}}%
\pgfpathlineto{\pgfqpoint{5.972400in}{1.139958in}}%
\pgfpathlineto{\pgfqpoint{5.984800in}{1.137944in}}%
\pgfpathlineto{\pgfqpoint{6.022000in}{1.139514in}}%
\pgfpathlineto{\pgfqpoint{6.034400in}{1.137824in}}%
\pgfpathlineto{\pgfqpoint{6.046800in}{1.138607in}}%
\pgfpathlineto{\pgfqpoint{6.059200in}{1.140914in}}%
\pgfpathlineto{\pgfqpoint{6.071600in}{1.138831in}}%
\pgfpathlineto{\pgfqpoint{6.084000in}{1.140198in}}%
\pgfpathlineto{\pgfqpoint{6.096400in}{1.138636in}}%
\pgfpathlineto{\pgfqpoint{6.121200in}{1.139066in}}%
\pgfpathlineto{\pgfqpoint{6.146000in}{1.142501in}}%
\pgfpathlineto{\pgfqpoint{6.158400in}{1.142444in}}%
\pgfpathlineto{\pgfqpoint{6.170800in}{1.143513in}}%
\pgfpathlineto{\pgfqpoint{6.195600in}{1.141166in}}%
\pgfpathlineto{\pgfqpoint{6.232800in}{1.145003in}}%
\pgfpathlineto{\pgfqpoint{6.245200in}{1.143338in}}%
\pgfpathlineto{\pgfqpoint{6.257600in}{1.144828in}}%
\pgfpathlineto{\pgfqpoint{6.270000in}{1.143996in}}%
\pgfpathlineto{\pgfqpoint{6.307200in}{1.145100in}}%
\pgfpathlineto{\pgfqpoint{6.319600in}{1.148116in}}%
\pgfpathlineto{\pgfqpoint{6.344400in}{1.147170in}}%
\pgfpathlineto{\pgfqpoint{6.356800in}{1.151280in}}%
\pgfpathlineto{\pgfqpoint{6.381600in}{1.147877in}}%
\pgfpathlineto{\pgfqpoint{6.394000in}{1.147500in}}%
\pgfpathlineto{\pgfqpoint{6.406400in}{1.148243in}}%
\pgfpathlineto{\pgfqpoint{6.418800in}{1.146442in}}%
\pgfpathlineto{\pgfqpoint{6.431200in}{1.150280in}}%
\pgfpathlineto{\pgfqpoint{6.443600in}{1.151840in}}%
\pgfpathlineto{\pgfqpoint{6.456000in}{1.148431in}}%
\pgfpathlineto{\pgfqpoint{6.468400in}{1.149596in}}%
\pgfpathlineto{\pgfqpoint{6.480800in}{1.148087in}}%
\pgfpathlineto{\pgfqpoint{6.493200in}{1.150560in}}%
\pgfpathlineto{\pgfqpoint{6.505600in}{1.148707in}}%
\pgfpathlineto{\pgfqpoint{6.518000in}{1.150120in}}%
\pgfpathlineto{\pgfqpoint{6.530400in}{1.150144in}}%
\pgfpathlineto{\pgfqpoint{6.542800in}{1.152328in}}%
\pgfpathlineto{\pgfqpoint{6.555200in}{1.150136in}}%
\pgfpathlineto{\pgfqpoint{6.567600in}{1.149126in}}%
\pgfpathlineto{\pgfqpoint{6.580000in}{1.153344in}}%
\pgfpathlineto{\pgfqpoint{6.592400in}{1.151968in}}%
\pgfpathlineto{\pgfqpoint{6.604800in}{1.149290in}}%
\pgfpathlineto{\pgfqpoint{6.617200in}{1.151376in}}%
\pgfpathlineto{\pgfqpoint{6.629600in}{1.149829in}}%
\pgfpathlineto{\pgfqpoint{6.642000in}{1.151632in}}%
\pgfpathlineto{\pgfqpoint{6.654400in}{1.149699in}}%
\pgfpathlineto{\pgfqpoint{6.679200in}{1.149438in}}%
\pgfpathlineto{\pgfqpoint{6.691600in}{1.151512in}}%
\pgfpathlineto{\pgfqpoint{6.704000in}{1.150944in}}%
\pgfpathlineto{\pgfqpoint{6.704000in}{1.150944in}}%
\pgfusepath{stroke}%
\end{pgfscope}%
\begin{pgfscope}%
\pgfpathrectangle{\pgfqpoint{1.000000in}{0.350000in}}{\pgfqpoint{6.200000in}{2.800000in}} %
\pgfusepath{clip}%
\pgfsetrectcap%
\pgfsetroundjoin%
\pgfsetlinewidth{1.003750pt}%
\definecolor{currentstroke}{rgb}{1.000000,0.000000,0.000000}%
\pgfsetstrokecolor{currentstroke}%
\pgfsetdash{}{0pt}%
\pgfpathmoveto{\pgfqpoint{1.000000in}{0.350000in}}%
\pgfpathlineto{\pgfqpoint{2.500400in}{0.350824in}}%
\pgfpathlineto{\pgfqpoint{2.512800in}{0.352553in}}%
\pgfpathlineto{\pgfqpoint{2.525200in}{0.358864in}}%
\pgfpathlineto{\pgfqpoint{2.537600in}{0.372385in}}%
\pgfpathlineto{\pgfqpoint{2.550000in}{0.401380in}}%
\pgfpathlineto{\pgfqpoint{2.562400in}{0.455106in}}%
\pgfpathlineto{\pgfqpoint{2.574800in}{0.549347in}}%
\pgfpathlineto{\pgfqpoint{2.587200in}{0.675052in}}%
\pgfpathlineto{\pgfqpoint{2.599600in}{0.844681in}}%
\pgfpathlineto{\pgfqpoint{2.612000in}{1.058378in}}%
\pgfpathlineto{\pgfqpoint{2.649200in}{1.804592in}}%
\pgfpathlineto{\pgfqpoint{2.661600in}{2.021624in}}%
\pgfpathlineto{\pgfqpoint{2.674000in}{2.216936in}}%
\pgfpathlineto{\pgfqpoint{2.686400in}{2.353944in}}%
\pgfpathlineto{\pgfqpoint{2.698800in}{2.461632in}}%
\pgfpathlineto{\pgfqpoint{2.711200in}{2.510624in}}%
\pgfpathlineto{\pgfqpoint{2.723600in}{2.513168in}}%
\pgfpathlineto{\pgfqpoint{2.736000in}{2.474352in}}%
\pgfpathlineto{\pgfqpoint{2.748400in}{2.420424in}}%
\pgfpathlineto{\pgfqpoint{2.760800in}{2.343728in}}%
\pgfpathlineto{\pgfqpoint{2.773200in}{2.257952in}}%
\pgfpathlineto{\pgfqpoint{2.785600in}{2.163032in}}%
\pgfpathlineto{\pgfqpoint{2.810400in}{1.959928in}}%
\pgfpathlineto{\pgfqpoint{2.847600in}{1.689976in}}%
\pgfpathlineto{\pgfqpoint{2.884800in}{1.491280in}}%
\pgfpathlineto{\pgfqpoint{2.909600in}{1.391208in}}%
\pgfpathlineto{\pgfqpoint{2.922000in}{1.356512in}}%
\pgfpathlineto{\pgfqpoint{2.934400in}{1.311336in}}%
\pgfpathlineto{\pgfqpoint{2.946800in}{1.286496in}}%
\pgfpathlineto{\pgfqpoint{2.959200in}{1.248712in}}%
\pgfpathlineto{\pgfqpoint{2.971600in}{1.224728in}}%
\pgfpathlineto{\pgfqpoint{2.984000in}{1.211072in}}%
\pgfpathlineto{\pgfqpoint{2.996400in}{1.192344in}}%
\pgfpathlineto{\pgfqpoint{3.008800in}{1.181200in}}%
\pgfpathlineto{\pgfqpoint{3.021200in}{1.159784in}}%
\pgfpathlineto{\pgfqpoint{3.033600in}{1.153944in}}%
\pgfpathlineto{\pgfqpoint{3.046000in}{1.134162in}}%
\pgfpathlineto{\pgfqpoint{3.058400in}{1.132347in}}%
\pgfpathlineto{\pgfqpoint{3.070800in}{1.125364in}}%
\pgfpathlineto{\pgfqpoint{3.083200in}{1.119740in}}%
\pgfpathlineto{\pgfqpoint{3.095600in}{1.108876in}}%
\pgfpathlineto{\pgfqpoint{3.108000in}{1.110894in}}%
\pgfpathlineto{\pgfqpoint{3.120400in}{1.104726in}}%
\pgfpathlineto{\pgfqpoint{3.132800in}{1.105177in}}%
\pgfpathlineto{\pgfqpoint{3.157600in}{1.096452in}}%
\pgfpathlineto{\pgfqpoint{3.170000in}{1.105401in}}%
\pgfpathlineto{\pgfqpoint{3.182400in}{1.093952in}}%
\pgfpathlineto{\pgfqpoint{3.194800in}{1.100285in}}%
\pgfpathlineto{\pgfqpoint{3.207200in}{1.094132in}}%
\pgfpathlineto{\pgfqpoint{3.219600in}{1.089345in}}%
\pgfpathlineto{\pgfqpoint{3.232000in}{1.101030in}}%
\pgfpathlineto{\pgfqpoint{3.244400in}{1.101979in}}%
\pgfpathlineto{\pgfqpoint{3.256800in}{1.097018in}}%
\pgfpathlineto{\pgfqpoint{3.269200in}{1.096713in}}%
\pgfpathlineto{\pgfqpoint{3.281600in}{1.104290in}}%
\pgfpathlineto{\pgfqpoint{3.294000in}{1.098703in}}%
\pgfpathlineto{\pgfqpoint{3.306400in}{1.105471in}}%
\pgfpathlineto{\pgfqpoint{3.318800in}{1.099111in}}%
\pgfpathlineto{\pgfqpoint{3.331200in}{1.107222in}}%
\pgfpathlineto{\pgfqpoint{3.343600in}{1.104202in}}%
\pgfpathlineto{\pgfqpoint{3.368400in}{1.114428in}}%
\pgfpathlineto{\pgfqpoint{3.380800in}{1.110890in}}%
\pgfpathlineto{\pgfqpoint{3.393200in}{1.113786in}}%
\pgfpathlineto{\pgfqpoint{3.405600in}{1.114426in}}%
\pgfpathlineto{\pgfqpoint{3.418000in}{1.117141in}}%
\pgfpathlineto{\pgfqpoint{3.430400in}{1.115438in}}%
\pgfpathlineto{\pgfqpoint{3.467600in}{1.122854in}}%
\pgfpathlineto{\pgfqpoint{3.480000in}{1.122386in}}%
\pgfpathlineto{\pgfqpoint{3.492400in}{1.123835in}}%
\pgfpathlineto{\pgfqpoint{3.504800in}{1.132320in}}%
\pgfpathlineto{\pgfqpoint{3.517200in}{1.125370in}}%
\pgfpathlineto{\pgfqpoint{3.529600in}{1.130923in}}%
\pgfpathlineto{\pgfqpoint{3.542000in}{1.137786in}}%
\pgfpathlineto{\pgfqpoint{3.554400in}{1.137870in}}%
\pgfpathlineto{\pgfqpoint{3.566800in}{1.139780in}}%
\pgfpathlineto{\pgfqpoint{3.579200in}{1.145362in}}%
\pgfpathlineto{\pgfqpoint{3.591600in}{1.142786in}}%
\pgfpathlineto{\pgfqpoint{3.604000in}{1.145098in}}%
\pgfpathlineto{\pgfqpoint{3.616400in}{1.149760in}}%
\pgfpathlineto{\pgfqpoint{3.641200in}{1.143587in}}%
\pgfpathlineto{\pgfqpoint{3.653600in}{1.149308in}}%
\pgfpathlineto{\pgfqpoint{3.678400in}{1.147154in}}%
\pgfpathlineto{\pgfqpoint{3.703200in}{1.147983in}}%
\pgfpathlineto{\pgfqpoint{3.715600in}{1.156472in}}%
\pgfpathlineto{\pgfqpoint{3.728000in}{1.154216in}}%
\pgfpathlineto{\pgfqpoint{3.740400in}{1.147632in}}%
\pgfpathlineto{\pgfqpoint{3.752800in}{1.152152in}}%
\pgfpathlineto{\pgfqpoint{3.765200in}{1.151360in}}%
\pgfpathlineto{\pgfqpoint{3.777600in}{1.146029in}}%
\pgfpathlineto{\pgfqpoint{3.790000in}{1.152200in}}%
\pgfpathlineto{\pgfqpoint{3.802400in}{1.155296in}}%
\pgfpathlineto{\pgfqpoint{3.814800in}{1.156032in}}%
\pgfpathlineto{\pgfqpoint{3.827200in}{1.153264in}}%
\pgfpathlineto{\pgfqpoint{3.839600in}{1.156704in}}%
\pgfpathlineto{\pgfqpoint{3.876800in}{1.148515in}}%
\pgfpathlineto{\pgfqpoint{3.889200in}{1.153968in}}%
\pgfpathlineto{\pgfqpoint{3.901600in}{1.154648in}}%
\pgfpathlineto{\pgfqpoint{3.914000in}{1.149618in}}%
\pgfpathlineto{\pgfqpoint{3.938800in}{1.150920in}}%
\pgfpathlineto{\pgfqpoint{3.951200in}{1.150488in}}%
\pgfpathlineto{\pgfqpoint{3.963600in}{1.152968in}}%
\pgfpathlineto{\pgfqpoint{3.976000in}{1.149849in}}%
\pgfpathlineto{\pgfqpoint{4.000800in}{1.146242in}}%
\pgfpathlineto{\pgfqpoint{4.013200in}{1.146527in}}%
\pgfpathlineto{\pgfqpoint{4.025600in}{1.145439in}}%
\pgfpathlineto{\pgfqpoint{4.038000in}{1.148988in}}%
\pgfpathlineto{\pgfqpoint{4.050400in}{1.145886in}}%
\pgfpathlineto{\pgfqpoint{4.062800in}{1.152304in}}%
\pgfpathlineto{\pgfqpoint{4.075200in}{1.145982in}}%
\pgfpathlineto{\pgfqpoint{4.087600in}{1.149856in}}%
\pgfpathlineto{\pgfqpoint{4.100000in}{1.145469in}}%
\pgfpathlineto{\pgfqpoint{4.112400in}{1.144458in}}%
\pgfpathlineto{\pgfqpoint{4.124800in}{1.144567in}}%
\pgfpathlineto{\pgfqpoint{4.137200in}{1.148729in}}%
\pgfpathlineto{\pgfqpoint{4.149600in}{1.147969in}}%
\pgfpathlineto{\pgfqpoint{4.162000in}{1.144282in}}%
\pgfpathlineto{\pgfqpoint{4.174400in}{1.143095in}}%
\pgfpathlineto{\pgfqpoint{4.186800in}{1.146627in}}%
\pgfpathlineto{\pgfqpoint{4.199200in}{1.145049in}}%
\pgfpathlineto{\pgfqpoint{4.211600in}{1.145311in}}%
\pgfpathlineto{\pgfqpoint{4.224000in}{1.147084in}}%
\pgfpathlineto{\pgfqpoint{4.236400in}{1.151600in}}%
\pgfpathlineto{\pgfqpoint{4.248800in}{1.144186in}}%
\pgfpathlineto{\pgfqpoint{4.273600in}{1.147254in}}%
\pgfpathlineto{\pgfqpoint{4.286000in}{1.143282in}}%
\pgfpathlineto{\pgfqpoint{4.298400in}{1.143111in}}%
\pgfpathlineto{\pgfqpoint{4.310800in}{1.141822in}}%
\pgfpathlineto{\pgfqpoint{4.323200in}{1.146500in}}%
\pgfpathlineto{\pgfqpoint{4.335600in}{1.141994in}}%
\pgfpathlineto{\pgfqpoint{4.348000in}{1.141874in}}%
\pgfpathlineto{\pgfqpoint{4.360400in}{1.137528in}}%
\pgfpathlineto{\pgfqpoint{4.372800in}{1.139638in}}%
\pgfpathlineto{\pgfqpoint{4.385200in}{1.138189in}}%
\pgfpathlineto{\pgfqpoint{4.397600in}{1.140370in}}%
\pgfpathlineto{\pgfqpoint{4.422400in}{1.132490in}}%
\pgfpathlineto{\pgfqpoint{4.434800in}{1.132814in}}%
\pgfpathlineto{\pgfqpoint{4.447200in}{1.137898in}}%
\pgfpathlineto{\pgfqpoint{4.459600in}{1.129594in}}%
\pgfpathlineto{\pgfqpoint{4.472000in}{1.128982in}}%
\pgfpathlineto{\pgfqpoint{4.484400in}{1.126456in}}%
\pgfpathlineto{\pgfqpoint{4.496800in}{1.125448in}}%
\pgfpathlineto{\pgfqpoint{4.509200in}{1.128711in}}%
\pgfpathlineto{\pgfqpoint{4.521600in}{1.121410in}}%
\pgfpathlineto{\pgfqpoint{4.534000in}{1.125613in}}%
\pgfpathlineto{\pgfqpoint{4.546400in}{1.122402in}}%
\pgfpathlineto{\pgfqpoint{4.596000in}{1.122741in}}%
\pgfpathlineto{\pgfqpoint{4.608400in}{1.120300in}}%
\pgfpathlineto{\pgfqpoint{4.620800in}{1.121823in}}%
\pgfpathlineto{\pgfqpoint{4.633200in}{1.124862in}}%
\pgfpathlineto{\pgfqpoint{4.645600in}{1.125443in}}%
\pgfpathlineto{\pgfqpoint{4.658000in}{1.123974in}}%
\pgfpathlineto{\pgfqpoint{4.670400in}{1.125057in}}%
\pgfpathlineto{\pgfqpoint{4.682800in}{1.128093in}}%
\pgfpathlineto{\pgfqpoint{4.695200in}{1.127010in}}%
\pgfpathlineto{\pgfqpoint{4.732400in}{1.128258in}}%
\pgfpathlineto{\pgfqpoint{4.744800in}{1.132323in}}%
\pgfpathlineto{\pgfqpoint{4.757200in}{1.133912in}}%
\pgfpathlineto{\pgfqpoint{4.769600in}{1.132870in}}%
\pgfpathlineto{\pgfqpoint{4.782000in}{1.134040in}}%
\pgfpathlineto{\pgfqpoint{4.794400in}{1.137065in}}%
\pgfpathlineto{\pgfqpoint{4.806800in}{1.136978in}}%
\pgfpathlineto{\pgfqpoint{4.819200in}{1.140349in}}%
\pgfpathlineto{\pgfqpoint{4.831600in}{1.136548in}}%
\pgfpathlineto{\pgfqpoint{4.844000in}{1.143072in}}%
\pgfpathlineto{\pgfqpoint{4.856400in}{1.142604in}}%
\pgfpathlineto{\pgfqpoint{4.868800in}{1.145673in}}%
\pgfpathlineto{\pgfqpoint{4.881200in}{1.144209in}}%
\pgfpathlineto{\pgfqpoint{4.906000in}{1.148772in}}%
\pgfpathlineto{\pgfqpoint{4.943200in}{1.151144in}}%
\pgfpathlineto{\pgfqpoint{4.955600in}{1.152352in}}%
\pgfpathlineto{\pgfqpoint{4.980400in}{1.156824in}}%
\pgfpathlineto{\pgfqpoint{4.992800in}{1.155464in}}%
\pgfpathlineto{\pgfqpoint{5.005200in}{1.160784in}}%
\pgfpathlineto{\pgfqpoint{5.017600in}{1.157696in}}%
\pgfpathlineto{\pgfqpoint{5.030000in}{1.162728in}}%
\pgfpathlineto{\pgfqpoint{5.042400in}{1.160624in}}%
\pgfpathlineto{\pgfqpoint{5.054800in}{1.164728in}}%
\pgfpathlineto{\pgfqpoint{5.067200in}{1.164168in}}%
\pgfpathlineto{\pgfqpoint{5.079600in}{1.159472in}}%
\pgfpathlineto{\pgfqpoint{5.092000in}{1.167624in}}%
\pgfpathlineto{\pgfqpoint{5.104400in}{1.163968in}}%
\pgfpathlineto{\pgfqpoint{5.116800in}{1.164008in}}%
\pgfpathlineto{\pgfqpoint{5.129200in}{1.168408in}}%
\pgfpathlineto{\pgfqpoint{5.141600in}{1.167824in}}%
\pgfpathlineto{\pgfqpoint{5.154000in}{1.169888in}}%
\pgfpathlineto{\pgfqpoint{5.166400in}{1.165072in}}%
\pgfpathlineto{\pgfqpoint{5.178800in}{1.165360in}}%
\pgfpathlineto{\pgfqpoint{5.191200in}{1.168520in}}%
\pgfpathlineto{\pgfqpoint{5.203600in}{1.167168in}}%
\pgfpathlineto{\pgfqpoint{5.216000in}{1.168320in}}%
\pgfpathlineto{\pgfqpoint{5.240800in}{1.168744in}}%
\pgfpathlineto{\pgfqpoint{5.253200in}{1.170040in}}%
\pgfpathlineto{\pgfqpoint{5.265600in}{1.166000in}}%
\pgfpathlineto{\pgfqpoint{5.278000in}{1.169816in}}%
\pgfpathlineto{\pgfqpoint{5.290400in}{1.166304in}}%
\pgfpathlineto{\pgfqpoint{5.302800in}{1.168024in}}%
\pgfpathlineto{\pgfqpoint{5.315200in}{1.168400in}}%
\pgfpathlineto{\pgfqpoint{5.327600in}{1.165200in}}%
\pgfpathlineto{\pgfqpoint{5.352400in}{1.169104in}}%
\pgfpathlineto{\pgfqpoint{5.377200in}{1.164920in}}%
\pgfpathlineto{\pgfqpoint{5.414400in}{1.165928in}}%
\pgfpathlineto{\pgfqpoint{5.439200in}{1.159864in}}%
\pgfpathlineto{\pgfqpoint{5.464000in}{1.165376in}}%
\pgfpathlineto{\pgfqpoint{5.476400in}{1.161152in}}%
\pgfpathlineto{\pgfqpoint{5.488800in}{1.164504in}}%
\pgfpathlineto{\pgfqpoint{5.501200in}{1.163456in}}%
\pgfpathlineto{\pgfqpoint{5.513600in}{1.164144in}}%
\pgfpathlineto{\pgfqpoint{5.526000in}{1.163616in}}%
\pgfpathlineto{\pgfqpoint{5.538400in}{1.160400in}}%
\pgfpathlineto{\pgfqpoint{5.550800in}{1.159448in}}%
\pgfpathlineto{\pgfqpoint{5.563200in}{1.155880in}}%
\pgfpathlineto{\pgfqpoint{5.575600in}{1.157496in}}%
\pgfpathlineto{\pgfqpoint{5.600400in}{1.154600in}}%
\pgfpathlineto{\pgfqpoint{5.612800in}{1.156264in}}%
\pgfpathlineto{\pgfqpoint{5.625200in}{1.154384in}}%
\pgfpathlineto{\pgfqpoint{5.637600in}{1.154480in}}%
\pgfpathlineto{\pgfqpoint{5.650000in}{1.148988in}}%
\pgfpathlineto{\pgfqpoint{5.662400in}{1.153000in}}%
\pgfpathlineto{\pgfqpoint{5.674800in}{1.150024in}}%
\pgfpathlineto{\pgfqpoint{5.699600in}{1.149576in}}%
\pgfpathlineto{\pgfqpoint{5.712000in}{1.149424in}}%
\pgfpathlineto{\pgfqpoint{5.724400in}{1.146820in}}%
\pgfpathlineto{\pgfqpoint{5.749200in}{1.152536in}}%
\pgfpathlineto{\pgfqpoint{5.761600in}{1.149538in}}%
\pgfpathlineto{\pgfqpoint{5.774000in}{1.149287in}}%
\pgfpathlineto{\pgfqpoint{5.786400in}{1.146477in}}%
\pgfpathlineto{\pgfqpoint{5.798800in}{1.147527in}}%
\pgfpathlineto{\pgfqpoint{5.811200in}{1.145097in}}%
\pgfpathlineto{\pgfqpoint{5.873200in}{1.142197in}}%
\pgfpathlineto{\pgfqpoint{5.885600in}{1.144293in}}%
\pgfpathlineto{\pgfqpoint{5.910400in}{1.141171in}}%
\pgfpathlineto{\pgfqpoint{5.922800in}{1.141317in}}%
\pgfpathlineto{\pgfqpoint{5.935200in}{1.143106in}}%
\pgfpathlineto{\pgfqpoint{5.947600in}{1.141667in}}%
\pgfpathlineto{\pgfqpoint{5.960000in}{1.142322in}}%
\pgfpathlineto{\pgfqpoint{5.984800in}{1.140865in}}%
\pgfpathlineto{\pgfqpoint{5.997200in}{1.142651in}}%
\pgfpathlineto{\pgfqpoint{6.022000in}{1.141012in}}%
\pgfpathlineto{\pgfqpoint{6.034400in}{1.142010in}}%
\pgfpathlineto{\pgfqpoint{6.046800in}{1.137950in}}%
\pgfpathlineto{\pgfqpoint{6.059200in}{1.142191in}}%
\pgfpathlineto{\pgfqpoint{6.071600in}{1.139122in}}%
\pgfpathlineto{\pgfqpoint{6.108800in}{1.143423in}}%
\pgfpathlineto{\pgfqpoint{6.121200in}{1.143791in}}%
\pgfpathlineto{\pgfqpoint{6.133600in}{1.140961in}}%
\pgfpathlineto{\pgfqpoint{6.146000in}{1.141373in}}%
\pgfpathlineto{\pgfqpoint{6.158400in}{1.140286in}}%
\pgfpathlineto{\pgfqpoint{6.170800in}{1.142171in}}%
\pgfpathlineto{\pgfqpoint{6.183200in}{1.142134in}}%
\pgfpathlineto{\pgfqpoint{6.195600in}{1.145439in}}%
\pgfpathlineto{\pgfqpoint{6.208000in}{1.145030in}}%
\pgfpathlineto{\pgfqpoint{6.220400in}{1.139637in}}%
\pgfpathlineto{\pgfqpoint{6.232800in}{1.143934in}}%
\pgfpathlineto{\pgfqpoint{6.257600in}{1.145330in}}%
\pgfpathlineto{\pgfqpoint{6.270000in}{1.142975in}}%
\pgfpathlineto{\pgfqpoint{6.294800in}{1.142829in}}%
\pgfpathlineto{\pgfqpoint{6.307200in}{1.145161in}}%
\pgfpathlineto{\pgfqpoint{6.319600in}{1.143691in}}%
\pgfpathlineto{\pgfqpoint{6.332000in}{1.145426in}}%
\pgfpathlineto{\pgfqpoint{6.344400in}{1.145537in}}%
\pgfpathlineto{\pgfqpoint{6.356800in}{1.147719in}}%
\pgfpathlineto{\pgfqpoint{6.369200in}{1.147940in}}%
\pgfpathlineto{\pgfqpoint{6.381600in}{1.143962in}}%
\pgfpathlineto{\pgfqpoint{6.394000in}{1.146746in}}%
\pgfpathlineto{\pgfqpoint{6.418800in}{1.145334in}}%
\pgfpathlineto{\pgfqpoint{6.431200in}{1.146406in}}%
\pgfpathlineto{\pgfqpoint{6.456000in}{1.147287in}}%
\pgfpathlineto{\pgfqpoint{6.468400in}{1.151016in}}%
\pgfpathlineto{\pgfqpoint{6.493200in}{1.147399in}}%
\pgfpathlineto{\pgfqpoint{6.505600in}{1.149341in}}%
\pgfpathlineto{\pgfqpoint{6.530400in}{1.149168in}}%
\pgfpathlineto{\pgfqpoint{6.555200in}{1.150512in}}%
\pgfpathlineto{\pgfqpoint{6.567600in}{1.147386in}}%
\pgfpathlineto{\pgfqpoint{6.580000in}{1.149585in}}%
\pgfpathlineto{\pgfqpoint{6.629600in}{1.153800in}}%
\pgfpathlineto{\pgfqpoint{6.642000in}{1.151376in}}%
\pgfpathlineto{\pgfqpoint{6.666800in}{1.153680in}}%
\pgfpathlineto{\pgfqpoint{6.691600in}{1.151400in}}%
\pgfpathlineto{\pgfqpoint{6.704000in}{1.152440in}}%
\pgfpathlineto{\pgfqpoint{6.704000in}{1.152440in}}%
\pgfusepath{stroke}%
\end{pgfscope}%
\begin{pgfscope}%
\pgfpathrectangle{\pgfqpoint{1.000000in}{0.350000in}}{\pgfqpoint{6.200000in}{2.800000in}} %
\pgfusepath{clip}%
\pgfsetbuttcap%
\pgfsetroundjoin%
\pgfsetlinewidth{0.501875pt}%
\definecolor{currentstroke}{rgb}{0.000000,0.000000,0.000000}%
\pgfsetstrokecolor{currentstroke}%
\pgfsetdash{{1.000000pt}{3.000000pt}}{0.000000pt}%
\pgfpathmoveto{\pgfqpoint{1.000000in}{0.350000in}}%
\pgfpathlineto{\pgfqpoint{1.000000in}{3.150000in}}%
\pgfusepath{stroke}%
\end{pgfscope}%
\begin{pgfscope}%
\pgfsetbuttcap%
\pgfsetroundjoin%
\definecolor{currentfill}{rgb}{0.000000,0.000000,0.000000}%
\pgfsetfillcolor{currentfill}%
\pgfsetlinewidth{0.501875pt}%
\definecolor{currentstroke}{rgb}{0.000000,0.000000,0.000000}%
\pgfsetstrokecolor{currentstroke}%
\pgfsetdash{}{0pt}%
\pgfsys@defobject{currentmarker}{\pgfqpoint{0.000000in}{0.000000in}}{\pgfqpoint{0.000000in}{0.055556in}}{%
\pgfpathmoveto{\pgfqpoint{0.000000in}{0.000000in}}%
\pgfpathlineto{\pgfqpoint{0.000000in}{0.055556in}}%
\pgfusepath{stroke,fill}%
}%
\begin{pgfscope}%
\pgfsys@transformshift{1.000000in}{0.350000in}%
\pgfsys@useobject{currentmarker}{}%
\end{pgfscope}%
\end{pgfscope}%
\begin{pgfscope}%
\pgfsetbuttcap%
\pgfsetroundjoin%
\definecolor{currentfill}{rgb}{0.000000,0.000000,0.000000}%
\pgfsetfillcolor{currentfill}%
\pgfsetlinewidth{0.501875pt}%
\definecolor{currentstroke}{rgb}{0.000000,0.000000,0.000000}%
\pgfsetstrokecolor{currentstroke}%
\pgfsetdash{}{0pt}%
\pgfsys@defobject{currentmarker}{\pgfqpoint{0.000000in}{-0.055556in}}{\pgfqpoint{0.000000in}{0.000000in}}{%
\pgfpathmoveto{\pgfqpoint{0.000000in}{0.000000in}}%
\pgfpathlineto{\pgfqpoint{0.000000in}{-0.055556in}}%
\pgfusepath{stroke,fill}%
}%
\begin{pgfscope}%
\pgfsys@transformshift{1.000000in}{3.150000in}%
\pgfsys@useobject{currentmarker}{}%
\end{pgfscope}%
\end{pgfscope}%
\begin{pgfscope}%
\pgftext[left,bottom,x=0.867472in,y=0.168387in,rotate=0.000000]{{\sffamily\fontsize{12.000000}{14.400000}\selectfont 0.0}}
%
\end{pgfscope}%
\begin{pgfscope}%
\pgfpathrectangle{\pgfqpoint{1.000000in}{0.350000in}}{\pgfqpoint{6.200000in}{2.800000in}} %
\pgfusepath{clip}%
\pgfsetbuttcap%
\pgfsetroundjoin%
\pgfsetlinewidth{0.501875pt}%
\definecolor{currentstroke}{rgb}{0.000000,0.000000,0.000000}%
\pgfsetstrokecolor{currentstroke}%
\pgfsetdash{{1.000000pt}{3.000000pt}}{0.000000pt}%
\pgfpathmoveto{\pgfqpoint{2.240000in}{0.350000in}}%
\pgfpathlineto{\pgfqpoint{2.240000in}{3.150000in}}%
\pgfusepath{stroke}%
\end{pgfscope}%
\begin{pgfscope}%
\pgfsetbuttcap%
\pgfsetroundjoin%
\definecolor{currentfill}{rgb}{0.000000,0.000000,0.000000}%
\pgfsetfillcolor{currentfill}%
\pgfsetlinewidth{0.501875pt}%
\definecolor{currentstroke}{rgb}{0.000000,0.000000,0.000000}%
\pgfsetstrokecolor{currentstroke}%
\pgfsetdash{}{0pt}%
\pgfsys@defobject{currentmarker}{\pgfqpoint{0.000000in}{0.000000in}}{\pgfqpoint{0.000000in}{0.055556in}}{%
\pgfpathmoveto{\pgfqpoint{0.000000in}{0.000000in}}%
\pgfpathlineto{\pgfqpoint{0.000000in}{0.055556in}}%
\pgfusepath{stroke,fill}%
}%
\begin{pgfscope}%
\pgfsys@transformshift{2.240000in}{0.350000in}%
\pgfsys@useobject{currentmarker}{}%
\end{pgfscope}%
\end{pgfscope}%
\begin{pgfscope}%
\pgfsetbuttcap%
\pgfsetroundjoin%
\definecolor{currentfill}{rgb}{0.000000,0.000000,0.000000}%
\pgfsetfillcolor{currentfill}%
\pgfsetlinewidth{0.501875pt}%
\definecolor{currentstroke}{rgb}{0.000000,0.000000,0.000000}%
\pgfsetstrokecolor{currentstroke}%
\pgfsetdash{}{0pt}%
\pgfsys@defobject{currentmarker}{\pgfqpoint{0.000000in}{-0.055556in}}{\pgfqpoint{0.000000in}{0.000000in}}{%
\pgfpathmoveto{\pgfqpoint{0.000000in}{0.000000in}}%
\pgfpathlineto{\pgfqpoint{0.000000in}{-0.055556in}}%
\pgfusepath{stroke,fill}%
}%
\begin{pgfscope}%
\pgfsys@transformshift{2.240000in}{3.150000in}%
\pgfsys@useobject{currentmarker}{}%
\end{pgfscope}%
\end{pgfscope}%
\begin{pgfscope}%
\pgftext[left,bottom,x=2.107472in,y=0.168387in,rotate=0.000000]{{\sffamily\fontsize{12.000000}{14.400000}\selectfont 0.2}}
%
\end{pgfscope}%
\begin{pgfscope}%
\pgfpathrectangle{\pgfqpoint{1.000000in}{0.350000in}}{\pgfqpoint{6.200000in}{2.800000in}} %
\pgfusepath{clip}%
\pgfsetbuttcap%
\pgfsetroundjoin%
\pgfsetlinewidth{0.501875pt}%
\definecolor{currentstroke}{rgb}{0.000000,0.000000,0.000000}%
\pgfsetstrokecolor{currentstroke}%
\pgfsetdash{{1.000000pt}{3.000000pt}}{0.000000pt}%
\pgfpathmoveto{\pgfqpoint{3.480000in}{0.350000in}}%
\pgfpathlineto{\pgfqpoint{3.480000in}{3.150000in}}%
\pgfusepath{stroke}%
\end{pgfscope}%
\begin{pgfscope}%
\pgfsetbuttcap%
\pgfsetroundjoin%
\definecolor{currentfill}{rgb}{0.000000,0.000000,0.000000}%
\pgfsetfillcolor{currentfill}%
\pgfsetlinewidth{0.501875pt}%
\definecolor{currentstroke}{rgb}{0.000000,0.000000,0.000000}%
\pgfsetstrokecolor{currentstroke}%
\pgfsetdash{}{0pt}%
\pgfsys@defobject{currentmarker}{\pgfqpoint{0.000000in}{0.000000in}}{\pgfqpoint{0.000000in}{0.055556in}}{%
\pgfpathmoveto{\pgfqpoint{0.000000in}{0.000000in}}%
\pgfpathlineto{\pgfqpoint{0.000000in}{0.055556in}}%
\pgfusepath{stroke,fill}%
}%
\begin{pgfscope}%
\pgfsys@transformshift{3.480000in}{0.350000in}%
\pgfsys@useobject{currentmarker}{}%
\end{pgfscope}%
\end{pgfscope}%
\begin{pgfscope}%
\pgfsetbuttcap%
\pgfsetroundjoin%
\definecolor{currentfill}{rgb}{0.000000,0.000000,0.000000}%
\pgfsetfillcolor{currentfill}%
\pgfsetlinewidth{0.501875pt}%
\definecolor{currentstroke}{rgb}{0.000000,0.000000,0.000000}%
\pgfsetstrokecolor{currentstroke}%
\pgfsetdash{}{0pt}%
\pgfsys@defobject{currentmarker}{\pgfqpoint{0.000000in}{-0.055556in}}{\pgfqpoint{0.000000in}{0.000000in}}{%
\pgfpathmoveto{\pgfqpoint{0.000000in}{0.000000in}}%
\pgfpathlineto{\pgfqpoint{0.000000in}{-0.055556in}}%
\pgfusepath{stroke,fill}%
}%
\begin{pgfscope}%
\pgfsys@transformshift{3.480000in}{3.150000in}%
\pgfsys@useobject{currentmarker}{}%
\end{pgfscope}%
\end{pgfscope}%
\begin{pgfscope}%
\pgftext[left,bottom,x=3.347472in,y=0.168387in,rotate=0.000000]{{\sffamily\fontsize{12.000000}{14.400000}\selectfont 0.4}}
%
\end{pgfscope}%
\begin{pgfscope}%
\pgfpathrectangle{\pgfqpoint{1.000000in}{0.350000in}}{\pgfqpoint{6.200000in}{2.800000in}} %
\pgfusepath{clip}%
\pgfsetbuttcap%
\pgfsetroundjoin%
\pgfsetlinewidth{0.501875pt}%
\definecolor{currentstroke}{rgb}{0.000000,0.000000,0.000000}%
\pgfsetstrokecolor{currentstroke}%
\pgfsetdash{{1.000000pt}{3.000000pt}}{0.000000pt}%
\pgfpathmoveto{\pgfqpoint{4.720000in}{0.350000in}}%
\pgfpathlineto{\pgfqpoint{4.720000in}{3.150000in}}%
\pgfusepath{stroke}%
\end{pgfscope}%
\begin{pgfscope}%
\pgfsetbuttcap%
\pgfsetroundjoin%
\definecolor{currentfill}{rgb}{0.000000,0.000000,0.000000}%
\pgfsetfillcolor{currentfill}%
\pgfsetlinewidth{0.501875pt}%
\definecolor{currentstroke}{rgb}{0.000000,0.000000,0.000000}%
\pgfsetstrokecolor{currentstroke}%
\pgfsetdash{}{0pt}%
\pgfsys@defobject{currentmarker}{\pgfqpoint{0.000000in}{0.000000in}}{\pgfqpoint{0.000000in}{0.055556in}}{%
\pgfpathmoveto{\pgfqpoint{0.000000in}{0.000000in}}%
\pgfpathlineto{\pgfqpoint{0.000000in}{0.055556in}}%
\pgfusepath{stroke,fill}%
}%
\begin{pgfscope}%
\pgfsys@transformshift{4.720000in}{0.350000in}%
\pgfsys@useobject{currentmarker}{}%
\end{pgfscope}%
\end{pgfscope}%
\begin{pgfscope}%
\pgfsetbuttcap%
\pgfsetroundjoin%
\definecolor{currentfill}{rgb}{0.000000,0.000000,0.000000}%
\pgfsetfillcolor{currentfill}%
\pgfsetlinewidth{0.501875pt}%
\definecolor{currentstroke}{rgb}{0.000000,0.000000,0.000000}%
\pgfsetstrokecolor{currentstroke}%
\pgfsetdash{}{0pt}%
\pgfsys@defobject{currentmarker}{\pgfqpoint{0.000000in}{-0.055556in}}{\pgfqpoint{0.000000in}{0.000000in}}{%
\pgfpathmoveto{\pgfqpoint{0.000000in}{0.000000in}}%
\pgfpathlineto{\pgfqpoint{0.000000in}{-0.055556in}}%
\pgfusepath{stroke,fill}%
}%
\begin{pgfscope}%
\pgfsys@transformshift{4.720000in}{3.150000in}%
\pgfsys@useobject{currentmarker}{}%
\end{pgfscope}%
\end{pgfscope}%
\begin{pgfscope}%
\pgftext[left,bottom,x=4.587472in,y=0.168387in,rotate=0.000000]{{\sffamily\fontsize{12.000000}{14.400000}\selectfont 0.6}}
%
\end{pgfscope}%
\begin{pgfscope}%
\pgfpathrectangle{\pgfqpoint{1.000000in}{0.350000in}}{\pgfqpoint{6.200000in}{2.800000in}} %
\pgfusepath{clip}%
\pgfsetbuttcap%
\pgfsetroundjoin%
\pgfsetlinewidth{0.501875pt}%
\definecolor{currentstroke}{rgb}{0.000000,0.000000,0.000000}%
\pgfsetstrokecolor{currentstroke}%
\pgfsetdash{{1.000000pt}{3.000000pt}}{0.000000pt}%
\pgfpathmoveto{\pgfqpoint{5.960000in}{0.350000in}}%
\pgfpathlineto{\pgfqpoint{5.960000in}{3.150000in}}%
\pgfusepath{stroke}%
\end{pgfscope}%
\begin{pgfscope}%
\pgfsetbuttcap%
\pgfsetroundjoin%
\definecolor{currentfill}{rgb}{0.000000,0.000000,0.000000}%
\pgfsetfillcolor{currentfill}%
\pgfsetlinewidth{0.501875pt}%
\definecolor{currentstroke}{rgb}{0.000000,0.000000,0.000000}%
\pgfsetstrokecolor{currentstroke}%
\pgfsetdash{}{0pt}%
\pgfsys@defobject{currentmarker}{\pgfqpoint{0.000000in}{0.000000in}}{\pgfqpoint{0.000000in}{0.055556in}}{%
\pgfpathmoveto{\pgfqpoint{0.000000in}{0.000000in}}%
\pgfpathlineto{\pgfqpoint{0.000000in}{0.055556in}}%
\pgfusepath{stroke,fill}%
}%
\begin{pgfscope}%
\pgfsys@transformshift{5.960000in}{0.350000in}%
\pgfsys@useobject{currentmarker}{}%
\end{pgfscope}%
\end{pgfscope}%
\begin{pgfscope}%
\pgfsetbuttcap%
\pgfsetroundjoin%
\definecolor{currentfill}{rgb}{0.000000,0.000000,0.000000}%
\pgfsetfillcolor{currentfill}%
\pgfsetlinewidth{0.501875pt}%
\definecolor{currentstroke}{rgb}{0.000000,0.000000,0.000000}%
\pgfsetstrokecolor{currentstroke}%
\pgfsetdash{}{0pt}%
\pgfsys@defobject{currentmarker}{\pgfqpoint{0.000000in}{-0.055556in}}{\pgfqpoint{0.000000in}{0.000000in}}{%
\pgfpathmoveto{\pgfqpoint{0.000000in}{0.000000in}}%
\pgfpathlineto{\pgfqpoint{0.000000in}{-0.055556in}}%
\pgfusepath{stroke,fill}%
}%
\begin{pgfscope}%
\pgfsys@transformshift{5.960000in}{3.150000in}%
\pgfsys@useobject{currentmarker}{}%
\end{pgfscope}%
\end{pgfscope}%
\begin{pgfscope}%
\pgftext[left,bottom,x=5.827472in,y=0.168387in,rotate=0.000000]{{\sffamily\fontsize{12.000000}{14.400000}\selectfont 0.8}}
%
\end{pgfscope}%
\begin{pgfscope}%
\pgfpathrectangle{\pgfqpoint{1.000000in}{0.350000in}}{\pgfqpoint{6.200000in}{2.800000in}} %
\pgfusepath{clip}%
\pgfsetbuttcap%
\pgfsetroundjoin%
\pgfsetlinewidth{0.501875pt}%
\definecolor{currentstroke}{rgb}{0.000000,0.000000,0.000000}%
\pgfsetstrokecolor{currentstroke}%
\pgfsetdash{{1.000000pt}{3.000000pt}}{0.000000pt}%
\pgfpathmoveto{\pgfqpoint{7.200000in}{0.350000in}}%
\pgfpathlineto{\pgfqpoint{7.200000in}{3.150000in}}%
\pgfusepath{stroke}%
\end{pgfscope}%
\begin{pgfscope}%
\pgfsetbuttcap%
\pgfsetroundjoin%
\definecolor{currentfill}{rgb}{0.000000,0.000000,0.000000}%
\pgfsetfillcolor{currentfill}%
\pgfsetlinewidth{0.501875pt}%
\definecolor{currentstroke}{rgb}{0.000000,0.000000,0.000000}%
\pgfsetstrokecolor{currentstroke}%
\pgfsetdash{}{0pt}%
\pgfsys@defobject{currentmarker}{\pgfqpoint{0.000000in}{0.000000in}}{\pgfqpoint{0.000000in}{0.055556in}}{%
\pgfpathmoveto{\pgfqpoint{0.000000in}{0.000000in}}%
\pgfpathlineto{\pgfqpoint{0.000000in}{0.055556in}}%
\pgfusepath{stroke,fill}%
}%
\begin{pgfscope}%
\pgfsys@transformshift{7.200000in}{0.350000in}%
\pgfsys@useobject{currentmarker}{}%
\end{pgfscope}%
\end{pgfscope}%
\begin{pgfscope}%
\pgfsetbuttcap%
\pgfsetroundjoin%
\definecolor{currentfill}{rgb}{0.000000,0.000000,0.000000}%
\pgfsetfillcolor{currentfill}%
\pgfsetlinewidth{0.501875pt}%
\definecolor{currentstroke}{rgb}{0.000000,0.000000,0.000000}%
\pgfsetstrokecolor{currentstroke}%
\pgfsetdash{}{0pt}%
\pgfsys@defobject{currentmarker}{\pgfqpoint{0.000000in}{-0.055556in}}{\pgfqpoint{0.000000in}{0.000000in}}{%
\pgfpathmoveto{\pgfqpoint{0.000000in}{0.000000in}}%
\pgfpathlineto{\pgfqpoint{0.000000in}{-0.055556in}}%
\pgfusepath{stroke,fill}%
}%
\begin{pgfscope}%
\pgfsys@transformshift{7.200000in}{3.150000in}%
\pgfsys@useobject{currentmarker}{}%
\end{pgfscope}%
\end{pgfscope}%
\begin{pgfscope}%
\pgftext[left,bottom,x=7.067472in,y=0.168387in,rotate=0.000000]{{\sffamily\fontsize{12.000000}{14.400000}\selectfont 1.0}}
%
\end{pgfscope}%
\begin{pgfscope}%
\pgftext[left,bottom,x=3.842391in,y=-0.030045in,rotate=0.000000]{{\sffamily\fontsize{12.000000}{14.400000}\selectfont radius}}
%
\end{pgfscope}%
\begin{pgfscope}%
\pgfpathrectangle{\pgfqpoint{1.000000in}{0.350000in}}{\pgfqpoint{6.200000in}{2.800000in}} %
\pgfusepath{clip}%
\pgfsetbuttcap%
\pgfsetroundjoin%
\pgfsetlinewidth{0.501875pt}%
\definecolor{currentstroke}{rgb}{0.000000,0.000000,0.000000}%
\pgfsetstrokecolor{currentstroke}%
\pgfsetdash{{1.000000pt}{3.000000pt}}{0.000000pt}%
\pgfpathmoveto{\pgfqpoint{1.000000in}{0.350000in}}%
\pgfpathlineto{\pgfqpoint{7.200000in}{0.350000in}}%
\pgfusepath{stroke}%
\end{pgfscope}%
\begin{pgfscope}%
\pgfsetbuttcap%
\pgfsetroundjoin%
\definecolor{currentfill}{rgb}{0.000000,0.000000,0.000000}%
\pgfsetfillcolor{currentfill}%
\pgfsetlinewidth{0.501875pt}%
\definecolor{currentstroke}{rgb}{0.000000,0.000000,0.000000}%
\pgfsetstrokecolor{currentstroke}%
\pgfsetdash{}{0pt}%
\pgfsys@defobject{currentmarker}{\pgfqpoint{0.000000in}{0.000000in}}{\pgfqpoint{0.055556in}{0.000000in}}{%
\pgfpathmoveto{\pgfqpoint{0.000000in}{0.000000in}}%
\pgfpathlineto{\pgfqpoint{0.055556in}{0.000000in}}%
\pgfusepath{stroke,fill}%
}%
\begin{pgfscope}%
\pgfsys@transformshift{1.000000in}{0.350000in}%
\pgfsys@useobject{currentmarker}{}%
\end{pgfscope}%
\end{pgfscope}%
\begin{pgfscope}%
\pgfsetbuttcap%
\pgfsetroundjoin%
\definecolor{currentfill}{rgb}{0.000000,0.000000,0.000000}%
\pgfsetfillcolor{currentfill}%
\pgfsetlinewidth{0.501875pt}%
\definecolor{currentstroke}{rgb}{0.000000,0.000000,0.000000}%
\pgfsetstrokecolor{currentstroke}%
\pgfsetdash{}{0pt}%
\pgfsys@defobject{currentmarker}{\pgfqpoint{-0.055556in}{0.000000in}}{\pgfqpoint{0.000000in}{0.000000in}}{%
\pgfpathmoveto{\pgfqpoint{0.000000in}{0.000000in}}%
\pgfpathlineto{\pgfqpoint{-0.055556in}{0.000000in}}%
\pgfusepath{stroke,fill}%
}%
\begin{pgfscope}%
\pgfsys@transformshift{7.200000in}{0.350000in}%
\pgfsys@useobject{currentmarker}{}%
\end{pgfscope}%
\end{pgfscope}%
\begin{pgfscope}%
\pgftext[left,bottom,x=0.679389in,y=0.286971in,rotate=0.000000]{{\sffamily\fontsize{12.000000}{14.400000}\selectfont 0.0}}
%
\end{pgfscope}%
\begin{pgfscope}%
\pgfpathrectangle{\pgfqpoint{1.000000in}{0.350000in}}{\pgfqpoint{6.200000in}{2.800000in}} %
\pgfusepath{clip}%
\pgfsetbuttcap%
\pgfsetroundjoin%
\pgfsetlinewidth{0.501875pt}%
\definecolor{currentstroke}{rgb}{0.000000,0.000000,0.000000}%
\pgfsetstrokecolor{currentstroke}%
\pgfsetdash{{1.000000pt}{3.000000pt}}{0.000000pt}%
\pgfpathmoveto{\pgfqpoint{1.000000in}{0.750000in}}%
\pgfpathlineto{\pgfqpoint{7.200000in}{0.750000in}}%
\pgfusepath{stroke}%
\end{pgfscope}%
\begin{pgfscope}%
\pgfsetbuttcap%
\pgfsetroundjoin%
\definecolor{currentfill}{rgb}{0.000000,0.000000,0.000000}%
\pgfsetfillcolor{currentfill}%
\pgfsetlinewidth{0.501875pt}%
\definecolor{currentstroke}{rgb}{0.000000,0.000000,0.000000}%
\pgfsetstrokecolor{currentstroke}%
\pgfsetdash{}{0pt}%
\pgfsys@defobject{currentmarker}{\pgfqpoint{0.000000in}{0.000000in}}{\pgfqpoint{0.055556in}{0.000000in}}{%
\pgfpathmoveto{\pgfqpoint{0.000000in}{0.000000in}}%
\pgfpathlineto{\pgfqpoint{0.055556in}{0.000000in}}%
\pgfusepath{stroke,fill}%
}%
\begin{pgfscope}%
\pgfsys@transformshift{1.000000in}{0.750000in}%
\pgfsys@useobject{currentmarker}{}%
\end{pgfscope}%
\end{pgfscope}%
\begin{pgfscope}%
\pgfsetbuttcap%
\pgfsetroundjoin%
\definecolor{currentfill}{rgb}{0.000000,0.000000,0.000000}%
\pgfsetfillcolor{currentfill}%
\pgfsetlinewidth{0.501875pt}%
\definecolor{currentstroke}{rgb}{0.000000,0.000000,0.000000}%
\pgfsetstrokecolor{currentstroke}%
\pgfsetdash{}{0pt}%
\pgfsys@defobject{currentmarker}{\pgfqpoint{-0.055556in}{0.000000in}}{\pgfqpoint{0.000000in}{0.000000in}}{%
\pgfpathmoveto{\pgfqpoint{0.000000in}{0.000000in}}%
\pgfpathlineto{\pgfqpoint{-0.055556in}{0.000000in}}%
\pgfusepath{stroke,fill}%
}%
\begin{pgfscope}%
\pgfsys@transformshift{7.200000in}{0.750000in}%
\pgfsys@useobject{currentmarker}{}%
\end{pgfscope}%
\end{pgfscope}%
\begin{pgfscope}%
\pgftext[left,bottom,x=0.679389in,y=0.686971in,rotate=0.000000]{{\sffamily\fontsize{12.000000}{14.400000}\selectfont 0.5}}
%
\end{pgfscope}%
\begin{pgfscope}%
\pgfpathrectangle{\pgfqpoint{1.000000in}{0.350000in}}{\pgfqpoint{6.200000in}{2.800000in}} %
\pgfusepath{clip}%
\pgfsetbuttcap%
\pgfsetroundjoin%
\pgfsetlinewidth{0.501875pt}%
\definecolor{currentstroke}{rgb}{0.000000,0.000000,0.000000}%
\pgfsetstrokecolor{currentstroke}%
\pgfsetdash{{1.000000pt}{3.000000pt}}{0.000000pt}%
\pgfpathmoveto{\pgfqpoint{1.000000in}{1.150000in}}%
\pgfpathlineto{\pgfqpoint{7.200000in}{1.150000in}}%
\pgfusepath{stroke}%
\end{pgfscope}%
\begin{pgfscope}%
\pgfsetbuttcap%
\pgfsetroundjoin%
\definecolor{currentfill}{rgb}{0.000000,0.000000,0.000000}%
\pgfsetfillcolor{currentfill}%
\pgfsetlinewidth{0.501875pt}%
\definecolor{currentstroke}{rgb}{0.000000,0.000000,0.000000}%
\pgfsetstrokecolor{currentstroke}%
\pgfsetdash{}{0pt}%
\pgfsys@defobject{currentmarker}{\pgfqpoint{0.000000in}{0.000000in}}{\pgfqpoint{0.055556in}{0.000000in}}{%
\pgfpathmoveto{\pgfqpoint{0.000000in}{0.000000in}}%
\pgfpathlineto{\pgfqpoint{0.055556in}{0.000000in}}%
\pgfusepath{stroke,fill}%
}%
\begin{pgfscope}%
\pgfsys@transformshift{1.000000in}{1.150000in}%
\pgfsys@useobject{currentmarker}{}%
\end{pgfscope}%
\end{pgfscope}%
\begin{pgfscope}%
\pgfsetbuttcap%
\pgfsetroundjoin%
\definecolor{currentfill}{rgb}{0.000000,0.000000,0.000000}%
\pgfsetfillcolor{currentfill}%
\pgfsetlinewidth{0.501875pt}%
\definecolor{currentstroke}{rgb}{0.000000,0.000000,0.000000}%
\pgfsetstrokecolor{currentstroke}%
\pgfsetdash{}{0pt}%
\pgfsys@defobject{currentmarker}{\pgfqpoint{-0.055556in}{0.000000in}}{\pgfqpoint{0.000000in}{0.000000in}}{%
\pgfpathmoveto{\pgfqpoint{0.000000in}{0.000000in}}%
\pgfpathlineto{\pgfqpoint{-0.055556in}{0.000000in}}%
\pgfusepath{stroke,fill}%
}%
\begin{pgfscope}%
\pgfsys@transformshift{7.200000in}{1.150000in}%
\pgfsys@useobject{currentmarker}{}%
\end{pgfscope}%
\end{pgfscope}%
\begin{pgfscope}%
\pgftext[left,bottom,x=0.679389in,y=1.086971in,rotate=0.000000]{{\sffamily\fontsize{12.000000}{14.400000}\selectfont 1.0}}
%
\end{pgfscope}%
\begin{pgfscope}%
\pgfpathrectangle{\pgfqpoint{1.000000in}{0.350000in}}{\pgfqpoint{6.200000in}{2.800000in}} %
\pgfusepath{clip}%
\pgfsetbuttcap%
\pgfsetroundjoin%
\pgfsetlinewidth{0.501875pt}%
\definecolor{currentstroke}{rgb}{0.000000,0.000000,0.000000}%
\pgfsetstrokecolor{currentstroke}%
\pgfsetdash{{1.000000pt}{3.000000pt}}{0.000000pt}%
\pgfpathmoveto{\pgfqpoint{1.000000in}{1.550000in}}%
\pgfpathlineto{\pgfqpoint{7.200000in}{1.550000in}}%
\pgfusepath{stroke}%
\end{pgfscope}%
\begin{pgfscope}%
\pgfsetbuttcap%
\pgfsetroundjoin%
\definecolor{currentfill}{rgb}{0.000000,0.000000,0.000000}%
\pgfsetfillcolor{currentfill}%
\pgfsetlinewidth{0.501875pt}%
\definecolor{currentstroke}{rgb}{0.000000,0.000000,0.000000}%
\pgfsetstrokecolor{currentstroke}%
\pgfsetdash{}{0pt}%
\pgfsys@defobject{currentmarker}{\pgfqpoint{0.000000in}{0.000000in}}{\pgfqpoint{0.055556in}{0.000000in}}{%
\pgfpathmoveto{\pgfqpoint{0.000000in}{0.000000in}}%
\pgfpathlineto{\pgfqpoint{0.055556in}{0.000000in}}%
\pgfusepath{stroke,fill}%
}%
\begin{pgfscope}%
\pgfsys@transformshift{1.000000in}{1.550000in}%
\pgfsys@useobject{currentmarker}{}%
\end{pgfscope}%
\end{pgfscope}%
\begin{pgfscope}%
\pgfsetbuttcap%
\pgfsetroundjoin%
\definecolor{currentfill}{rgb}{0.000000,0.000000,0.000000}%
\pgfsetfillcolor{currentfill}%
\pgfsetlinewidth{0.501875pt}%
\definecolor{currentstroke}{rgb}{0.000000,0.000000,0.000000}%
\pgfsetstrokecolor{currentstroke}%
\pgfsetdash{}{0pt}%
\pgfsys@defobject{currentmarker}{\pgfqpoint{-0.055556in}{0.000000in}}{\pgfqpoint{0.000000in}{0.000000in}}{%
\pgfpathmoveto{\pgfqpoint{0.000000in}{0.000000in}}%
\pgfpathlineto{\pgfqpoint{-0.055556in}{0.000000in}}%
\pgfusepath{stroke,fill}%
}%
\begin{pgfscope}%
\pgfsys@transformshift{7.200000in}{1.550000in}%
\pgfsys@useobject{currentmarker}{}%
\end{pgfscope}%
\end{pgfscope}%
\begin{pgfscope}%
\pgftext[left,bottom,x=0.679389in,y=1.488070in,rotate=0.000000]{{\sffamily\fontsize{12.000000}{14.400000}\selectfont 1.5}}
%
\end{pgfscope}%
\begin{pgfscope}%
\pgfpathrectangle{\pgfqpoint{1.000000in}{0.350000in}}{\pgfqpoint{6.200000in}{2.800000in}} %
\pgfusepath{clip}%
\pgfsetbuttcap%
\pgfsetroundjoin%
\pgfsetlinewidth{0.501875pt}%
\definecolor{currentstroke}{rgb}{0.000000,0.000000,0.000000}%
\pgfsetstrokecolor{currentstroke}%
\pgfsetdash{{1.000000pt}{3.000000pt}}{0.000000pt}%
\pgfpathmoveto{\pgfqpoint{1.000000in}{1.950000in}}%
\pgfpathlineto{\pgfqpoint{7.200000in}{1.950000in}}%
\pgfusepath{stroke}%
\end{pgfscope}%
\begin{pgfscope}%
\pgfsetbuttcap%
\pgfsetroundjoin%
\definecolor{currentfill}{rgb}{0.000000,0.000000,0.000000}%
\pgfsetfillcolor{currentfill}%
\pgfsetlinewidth{0.501875pt}%
\definecolor{currentstroke}{rgb}{0.000000,0.000000,0.000000}%
\pgfsetstrokecolor{currentstroke}%
\pgfsetdash{}{0pt}%
\pgfsys@defobject{currentmarker}{\pgfqpoint{0.000000in}{0.000000in}}{\pgfqpoint{0.055556in}{0.000000in}}{%
\pgfpathmoveto{\pgfqpoint{0.000000in}{0.000000in}}%
\pgfpathlineto{\pgfqpoint{0.055556in}{0.000000in}}%
\pgfusepath{stroke,fill}%
}%
\begin{pgfscope}%
\pgfsys@transformshift{1.000000in}{1.950000in}%
\pgfsys@useobject{currentmarker}{}%
\end{pgfscope}%
\end{pgfscope}%
\begin{pgfscope}%
\pgfsetbuttcap%
\pgfsetroundjoin%
\definecolor{currentfill}{rgb}{0.000000,0.000000,0.000000}%
\pgfsetfillcolor{currentfill}%
\pgfsetlinewidth{0.501875pt}%
\definecolor{currentstroke}{rgb}{0.000000,0.000000,0.000000}%
\pgfsetstrokecolor{currentstroke}%
\pgfsetdash{}{0pt}%
\pgfsys@defobject{currentmarker}{\pgfqpoint{-0.055556in}{0.000000in}}{\pgfqpoint{0.000000in}{0.000000in}}{%
\pgfpathmoveto{\pgfqpoint{0.000000in}{0.000000in}}%
\pgfpathlineto{\pgfqpoint{-0.055556in}{0.000000in}}%
\pgfusepath{stroke,fill}%
}%
\begin{pgfscope}%
\pgfsys@transformshift{7.200000in}{1.950000in}%
\pgfsys@useobject{currentmarker}{}%
\end{pgfscope}%
\end{pgfscope}%
\begin{pgfscope}%
\pgftext[left,bottom,x=0.679389in,y=1.886971in,rotate=0.000000]{{\sffamily\fontsize{12.000000}{14.400000}\selectfont 2.0}}
%
\end{pgfscope}%
\begin{pgfscope}%
\pgfpathrectangle{\pgfqpoint{1.000000in}{0.350000in}}{\pgfqpoint{6.200000in}{2.800000in}} %
\pgfusepath{clip}%
\pgfsetbuttcap%
\pgfsetroundjoin%
\pgfsetlinewidth{0.501875pt}%
\definecolor{currentstroke}{rgb}{0.000000,0.000000,0.000000}%
\pgfsetstrokecolor{currentstroke}%
\pgfsetdash{{1.000000pt}{3.000000pt}}{0.000000pt}%
\pgfpathmoveto{\pgfqpoint{1.000000in}{2.350000in}}%
\pgfpathlineto{\pgfqpoint{7.200000in}{2.350000in}}%
\pgfusepath{stroke}%
\end{pgfscope}%
\begin{pgfscope}%
\pgfsetbuttcap%
\pgfsetroundjoin%
\definecolor{currentfill}{rgb}{0.000000,0.000000,0.000000}%
\pgfsetfillcolor{currentfill}%
\pgfsetlinewidth{0.501875pt}%
\definecolor{currentstroke}{rgb}{0.000000,0.000000,0.000000}%
\pgfsetstrokecolor{currentstroke}%
\pgfsetdash{}{0pt}%
\pgfsys@defobject{currentmarker}{\pgfqpoint{0.000000in}{0.000000in}}{\pgfqpoint{0.055556in}{0.000000in}}{%
\pgfpathmoveto{\pgfqpoint{0.000000in}{0.000000in}}%
\pgfpathlineto{\pgfqpoint{0.055556in}{0.000000in}}%
\pgfusepath{stroke,fill}%
}%
\begin{pgfscope}%
\pgfsys@transformshift{1.000000in}{2.350000in}%
\pgfsys@useobject{currentmarker}{}%
\end{pgfscope}%
\end{pgfscope}%
\begin{pgfscope}%
\pgfsetbuttcap%
\pgfsetroundjoin%
\definecolor{currentfill}{rgb}{0.000000,0.000000,0.000000}%
\pgfsetfillcolor{currentfill}%
\pgfsetlinewidth{0.501875pt}%
\definecolor{currentstroke}{rgb}{0.000000,0.000000,0.000000}%
\pgfsetstrokecolor{currentstroke}%
\pgfsetdash{}{0pt}%
\pgfsys@defobject{currentmarker}{\pgfqpoint{-0.055556in}{0.000000in}}{\pgfqpoint{0.000000in}{0.000000in}}{%
\pgfpathmoveto{\pgfqpoint{0.000000in}{0.000000in}}%
\pgfpathlineto{\pgfqpoint{-0.055556in}{0.000000in}}%
\pgfusepath{stroke,fill}%
}%
\begin{pgfscope}%
\pgfsys@transformshift{7.200000in}{2.350000in}%
\pgfsys@useobject{currentmarker}{}%
\end{pgfscope}%
\end{pgfscope}%
\begin{pgfscope}%
\pgftext[left,bottom,x=0.679389in,y=2.286971in,rotate=0.000000]{{\sffamily\fontsize{12.000000}{14.400000}\selectfont 2.5}}
%
\end{pgfscope}%
\begin{pgfscope}%
\pgfpathrectangle{\pgfqpoint{1.000000in}{0.350000in}}{\pgfqpoint{6.200000in}{2.800000in}} %
\pgfusepath{clip}%
\pgfsetbuttcap%
\pgfsetroundjoin%
\pgfsetlinewidth{0.501875pt}%
\definecolor{currentstroke}{rgb}{0.000000,0.000000,0.000000}%
\pgfsetstrokecolor{currentstroke}%
\pgfsetdash{{1.000000pt}{3.000000pt}}{0.000000pt}%
\pgfpathmoveto{\pgfqpoint{1.000000in}{2.750000in}}%
\pgfpathlineto{\pgfqpoint{7.200000in}{2.750000in}}%
\pgfusepath{stroke}%
\end{pgfscope}%
\begin{pgfscope}%
\pgfsetbuttcap%
\pgfsetroundjoin%
\definecolor{currentfill}{rgb}{0.000000,0.000000,0.000000}%
\pgfsetfillcolor{currentfill}%
\pgfsetlinewidth{0.501875pt}%
\definecolor{currentstroke}{rgb}{0.000000,0.000000,0.000000}%
\pgfsetstrokecolor{currentstroke}%
\pgfsetdash{}{0pt}%
\pgfsys@defobject{currentmarker}{\pgfqpoint{0.000000in}{0.000000in}}{\pgfqpoint{0.055556in}{0.000000in}}{%
\pgfpathmoveto{\pgfqpoint{0.000000in}{0.000000in}}%
\pgfpathlineto{\pgfqpoint{0.055556in}{0.000000in}}%
\pgfusepath{stroke,fill}%
}%
\begin{pgfscope}%
\pgfsys@transformshift{1.000000in}{2.750000in}%
\pgfsys@useobject{currentmarker}{}%
\end{pgfscope}%
\end{pgfscope}%
\begin{pgfscope}%
\pgfsetbuttcap%
\pgfsetroundjoin%
\definecolor{currentfill}{rgb}{0.000000,0.000000,0.000000}%
\pgfsetfillcolor{currentfill}%
\pgfsetlinewidth{0.501875pt}%
\definecolor{currentstroke}{rgb}{0.000000,0.000000,0.000000}%
\pgfsetstrokecolor{currentstroke}%
\pgfsetdash{}{0pt}%
\pgfsys@defobject{currentmarker}{\pgfqpoint{-0.055556in}{0.000000in}}{\pgfqpoint{0.000000in}{0.000000in}}{%
\pgfpathmoveto{\pgfqpoint{0.000000in}{0.000000in}}%
\pgfpathlineto{\pgfqpoint{-0.055556in}{0.000000in}}%
\pgfusepath{stroke,fill}%
}%
\begin{pgfscope}%
\pgfsys@transformshift{7.200000in}{2.750000in}%
\pgfsys@useobject{currentmarker}{}%
\end{pgfscope}%
\end{pgfscope}%
\begin{pgfscope}%
\pgftext[left,bottom,x=0.679389in,y=2.686971in,rotate=0.000000]{{\sffamily\fontsize{12.000000}{14.400000}\selectfont 3.0}}
%
\end{pgfscope}%
\begin{pgfscope}%
\pgfpathrectangle{\pgfqpoint{1.000000in}{0.350000in}}{\pgfqpoint{6.200000in}{2.800000in}} %
\pgfusepath{clip}%
\pgfsetbuttcap%
\pgfsetroundjoin%
\pgfsetlinewidth{0.501875pt}%
\definecolor{currentstroke}{rgb}{0.000000,0.000000,0.000000}%
\pgfsetstrokecolor{currentstroke}%
\pgfsetdash{{1.000000pt}{3.000000pt}}{0.000000pt}%
\pgfpathmoveto{\pgfqpoint{1.000000in}{3.150000in}}%
\pgfpathlineto{\pgfqpoint{7.200000in}{3.150000in}}%
\pgfusepath{stroke}%
\end{pgfscope}%
\begin{pgfscope}%
\pgfsetbuttcap%
\pgfsetroundjoin%
\definecolor{currentfill}{rgb}{0.000000,0.000000,0.000000}%
\pgfsetfillcolor{currentfill}%
\pgfsetlinewidth{0.501875pt}%
\definecolor{currentstroke}{rgb}{0.000000,0.000000,0.000000}%
\pgfsetstrokecolor{currentstroke}%
\pgfsetdash{}{0pt}%
\pgfsys@defobject{currentmarker}{\pgfqpoint{0.000000in}{0.000000in}}{\pgfqpoint{0.055556in}{0.000000in}}{%
\pgfpathmoveto{\pgfqpoint{0.000000in}{0.000000in}}%
\pgfpathlineto{\pgfqpoint{0.055556in}{0.000000in}}%
\pgfusepath{stroke,fill}%
}%
\begin{pgfscope}%
\pgfsys@transformshift{1.000000in}{3.150000in}%
\pgfsys@useobject{currentmarker}{}%
\end{pgfscope}%
\end{pgfscope}%
\begin{pgfscope}%
\pgfsetbuttcap%
\pgfsetroundjoin%
\definecolor{currentfill}{rgb}{0.000000,0.000000,0.000000}%
\pgfsetfillcolor{currentfill}%
\pgfsetlinewidth{0.501875pt}%
\definecolor{currentstroke}{rgb}{0.000000,0.000000,0.000000}%
\pgfsetstrokecolor{currentstroke}%
\pgfsetdash{}{0pt}%
\pgfsys@defobject{currentmarker}{\pgfqpoint{-0.055556in}{0.000000in}}{\pgfqpoint{0.000000in}{0.000000in}}{%
\pgfpathmoveto{\pgfqpoint{0.000000in}{0.000000in}}%
\pgfpathlineto{\pgfqpoint{-0.055556in}{0.000000in}}%
\pgfusepath{stroke,fill}%
}%
\begin{pgfscope}%
\pgfsys@transformshift{7.200000in}{3.150000in}%
\pgfsys@useobject{currentmarker}{}%
\end{pgfscope}%
\end{pgfscope}%
\begin{pgfscope}%
\pgftext[left,bottom,x=0.679389in,y=3.086971in,rotate=0.000000]{{\sffamily\fontsize{12.000000}{14.400000}\selectfont 3.5}}
%
\end{pgfscope}%
\begin{pgfscope}%
\pgftext[left,bottom,x=0.609945in,y=0.646118in,rotate=90.000000]{{\sffamily\fontsize{12.000000}{14.400000}\selectfont radial distribution function}}
%
\end{pgfscope}%
\begin{pgfscope}%
\pgfsetrectcap%
\pgfsetroundjoin%
\pgfsetlinewidth{1.003750pt}%
\definecolor{currentstroke}{rgb}{0.000000,0.000000,0.000000}%
\pgfsetstrokecolor{currentstroke}%
\pgfsetdash{}{0pt}%
\pgfpathmoveto{\pgfqpoint{1.000000in}{3.150000in}}%
\pgfpathlineto{\pgfqpoint{7.200000in}{3.150000in}}%
\pgfusepath{stroke}%
\end{pgfscope}%
\begin{pgfscope}%
\pgfsetrectcap%
\pgfsetroundjoin%
\pgfsetlinewidth{1.003750pt}%
\definecolor{currentstroke}{rgb}{0.000000,0.000000,0.000000}%
\pgfsetstrokecolor{currentstroke}%
\pgfsetdash{}{0pt}%
\pgfpathmoveto{\pgfqpoint{7.200000in}{0.350000in}}%
\pgfpathlineto{\pgfqpoint{7.200000in}{3.150000in}}%
\pgfusepath{stroke}%
\end{pgfscope}%
\begin{pgfscope}%
\pgfsetrectcap%
\pgfsetroundjoin%
\pgfsetlinewidth{1.003750pt}%
\definecolor{currentstroke}{rgb}{0.000000,0.000000,0.000000}%
\pgfsetstrokecolor{currentstroke}%
\pgfsetdash{}{0pt}%
\pgfpathmoveto{\pgfqpoint{1.000000in}{0.350000in}}%
\pgfpathlineto{\pgfqpoint{7.200000in}{0.350000in}}%
\pgfusepath{stroke}%
\end{pgfscope}%
\begin{pgfscope}%
\pgfsetrectcap%
\pgfsetroundjoin%
\pgfsetlinewidth{1.003750pt}%
\definecolor{currentstroke}{rgb}{0.000000,0.000000,0.000000}%
\pgfsetstrokecolor{currentstroke}%
\pgfsetdash{}{0pt}%
\pgfpathmoveto{\pgfqpoint{1.000000in}{0.350000in}}%
\pgfpathlineto{\pgfqpoint{1.000000in}{3.150000in}}%
\pgfusepath{stroke}%
\end{pgfscope}%
\begin{pgfscope}%
\pgfsetrectcap%
\pgfsetroundjoin%
\definecolor{currentfill}{rgb}{1.000000,1.000000,1.000000}%
\pgfsetfillcolor{currentfill}%
\pgfsetlinewidth{1.003750pt}%
\definecolor{currentstroke}{rgb}{0.000000,0.000000,0.000000}%
\pgfsetstrokecolor{currentstroke}%
\pgfsetdash{}{0pt}%
\pgfpathmoveto{\pgfqpoint{1.069417in}{2.427606in}}%
\pgfpathlineto{\pgfqpoint{1.926808in}{2.427606in}}%
\pgfpathlineto{\pgfqpoint{1.926808in}{3.080583in}}%
\pgfpathlineto{\pgfqpoint{1.069417in}{3.080583in}}%
\pgfpathlineto{\pgfqpoint{1.069417in}{2.427606in}}%
\pgfpathclose%
\pgfusepath{stroke,fill}%
\end{pgfscope}%
\begin{pgfscope}%
\pgfsetrectcap%
\pgfsetroundjoin%
\pgfsetlinewidth{1.003750pt}%
\definecolor{currentstroke}{rgb}{0.000000,0.000000,1.000000}%
\pgfsetstrokecolor{currentstroke}%
\pgfsetdash{}{0pt}%
\pgfpathmoveto{\pgfqpoint{1.166600in}{2.968161in}}%
\pgfpathlineto{\pgfqpoint{1.360967in}{2.968161in}}%
\pgfusepath{stroke}%
\end{pgfscope}%
\begin{pgfscope}%
\pgftext[left,bottom,x=1.513683in,y=2.890691in,rotate=0.000000]{{\sffamily\fontsize{9.996000}{11.995200}\selectfont spc}}
%
\end{pgfscope}%
\begin{pgfscope}%
\pgfsetrectcap%
\pgfsetroundjoin%
\pgfsetlinewidth{1.003750pt}%
\definecolor{currentstroke}{rgb}{0.000000,0.500000,0.000000}%
\pgfsetstrokecolor{currentstroke}%
\pgfsetdash{}{0pt}%
\pgfpathmoveto{\pgfqpoint{1.166600in}{2.764385in}}%
\pgfpathlineto{\pgfqpoint{1.360967in}{2.764385in}}%
\pgfusepath{stroke}%
\end{pgfscope}%
\begin{pgfscope}%
\pgftext[left,bottom,x=1.513683in,y=2.686915in,rotate=0.000000]{{\sffamily\fontsize{9.996000}{11.995200}\selectfont spce}}
%
\end{pgfscope}%
\begin{pgfscope}%
\pgfsetrectcap%
\pgfsetroundjoin%
\pgfsetlinewidth{1.003750pt}%
\definecolor{currentstroke}{rgb}{1.000000,0.000000,0.000000}%
\pgfsetstrokecolor{currentstroke}%
\pgfsetdash{}{0pt}%
\pgfpathmoveto{\pgfqpoint{1.166600in}{2.560609in}}%
\pgfpathlineto{\pgfqpoint{1.360967in}{2.560609in}}%
\pgfusepath{stroke}%
\end{pgfscope}%
\begin{pgfscope}%
\pgftext[left,bottom,x=1.513683in,y=2.483139in,rotate=0.000000]{{\sffamily\fontsize{9.996000}{11.995200}\selectfont tip3p}}
%
\end{pgfscope}%
\end{pgfpicture}%
\makeatother%
\endgroup%
}
    \caption{Radial distribution function. The peaks are marked with stars.} \label{fig:rdf}
\end{figure}

All water models give similar distances for the peaks. The height and visibility of the peaks is very different, however. \textit{SPC/E} produce the highest and \textit{TIP3P} the lowest maxima. \textit{SPC} is in between. 

\subsection{Hydrogen bond analysis}
\begin{figure}[H]
	\resizebox{\linewidth}{!}{%% Creator: Matplotlib, PGF backend
%%
%% To include the figure in your LaTeX document, write
%%   \input{<filename>.pgf}
%%
%% Make sure the required packages are loaded in your preamble
%%   \usepackage{pgf}
%%
%% Figures using additional raster images can only be included by \input if
%% they are in the same directory as the main LaTeX file. For loading figures
%% from other directories you can use the `import` package
%%   \usepackage{import}
%% and then include the figures with
%%   \import{<path to file>}{<filename>.pgf}
%%
%% Matplotlib used the following preamble
%%   \usepackage{fontspec}
%%   \setmainfont{DejaVu Serif}
%%   \setsansfont{DejaVu Sans}
%%   \setmonofont{DejaVu Sans Mono}
%%
\begingroup%
\makeatletter%
\begin{pgfpicture}%
\pgfpathrectangle{\pgfpointorigin}{\pgfqpoint{8.000000in}{3.500000in}}%
\pgfusepath{use as bounding box}%
\begin{pgfscope}%
\pgfsetrectcap%
\pgfsetroundjoin%
\definecolor{currentfill}{rgb}{1.000000,1.000000,1.000000}%
\pgfsetfillcolor{currentfill}%
\pgfsetlinewidth{0.000000pt}%
\definecolor{currentstroke}{rgb}{1.000000,1.000000,1.000000}%
\pgfsetstrokecolor{currentstroke}%
\pgfsetdash{}{0pt}%
\pgfpathmoveto{\pgfqpoint{0.000000in}{0.000000in}}%
\pgfpathlineto{\pgfqpoint{8.000000in}{0.000000in}}%
\pgfpathlineto{\pgfqpoint{8.000000in}{3.500000in}}%
\pgfpathlineto{\pgfqpoint{0.000000in}{3.500000in}}%
\pgfpathclose%
\pgfusepath{fill}%
\end{pgfscope}%
\begin{pgfscope}%
\pgfsetrectcap%
\pgfsetroundjoin%
\definecolor{currentfill}{rgb}{1.000000,1.000000,1.000000}%
\pgfsetfillcolor{currentfill}%
\pgfsetlinewidth{0.000000pt}%
\definecolor{currentstroke}{rgb}{0.000000,0.000000,0.000000}%
\pgfsetstrokecolor{currentstroke}%
\pgfsetdash{}{0pt}%
\pgfpathmoveto{\pgfqpoint{1.000000in}{0.350000in}}%
\pgfpathlineto{\pgfqpoint{7.200000in}{0.350000in}}%
\pgfpathlineto{\pgfqpoint{7.200000in}{3.150000in}}%
\pgfpathlineto{\pgfqpoint{1.000000in}{3.150000in}}%
\pgfpathclose%
\pgfusepath{fill}%
\end{pgfscope}%
\begin{pgfscope}%
\pgfpathrectangle{\pgfqpoint{1.000000in}{0.350000in}}{\pgfqpoint{6.200000in}{2.800000in}} %
\pgfusepath{clip}%
\pgfsetrectcap%
\pgfsetroundjoin%
\pgfsetlinewidth{1.003750pt}%
\definecolor{currentstroke}{rgb}{0.000000,0.000000,1.000000}%
\pgfsetstrokecolor{currentstroke}%
\pgfsetdash{}{0pt}%
\pgfpathmoveto{\pgfqpoint{1.000000in}{2.415000in}}%
\pgfpathlineto{\pgfqpoint{1.001240in}{2.170000in}}%
\pgfpathlineto{\pgfqpoint{1.002480in}{2.135000in}}%
\pgfpathlineto{\pgfqpoint{1.003720in}{2.695000in}}%
\pgfpathlineto{\pgfqpoint{1.004960in}{1.960000in}}%
\pgfpathlineto{\pgfqpoint{1.006200in}{2.030000in}}%
\pgfpathlineto{\pgfqpoint{1.007440in}{1.540000in}}%
\pgfpathlineto{\pgfqpoint{1.008680in}{1.855000in}}%
\pgfpathlineto{\pgfqpoint{1.009920in}{1.750000in}}%
\pgfpathlineto{\pgfqpoint{1.011160in}{1.750000in}}%
\pgfpathlineto{\pgfqpoint{1.012400in}{1.960000in}}%
\pgfpathlineto{\pgfqpoint{1.014880in}{1.785000in}}%
\pgfpathlineto{\pgfqpoint{1.016120in}{1.680000in}}%
\pgfpathlineto{\pgfqpoint{1.017360in}{1.785000in}}%
\pgfpathlineto{\pgfqpoint{1.018600in}{1.750000in}}%
\pgfpathlineto{\pgfqpoint{1.021080in}{2.205000in}}%
\pgfpathlineto{\pgfqpoint{1.022320in}{2.065000in}}%
\pgfpathlineto{\pgfqpoint{1.023560in}{1.540000in}}%
\pgfpathlineto{\pgfqpoint{1.024800in}{1.715000in}}%
\pgfpathlineto{\pgfqpoint{1.026040in}{2.135000in}}%
\pgfpathlineto{\pgfqpoint{1.027280in}{1.995000in}}%
\pgfpathlineto{\pgfqpoint{1.028520in}{2.345000in}}%
\pgfpathlineto{\pgfqpoint{1.031000in}{1.925000in}}%
\pgfpathlineto{\pgfqpoint{1.033480in}{2.240000in}}%
\pgfpathlineto{\pgfqpoint{1.034720in}{2.205000in}}%
\pgfpathlineto{\pgfqpoint{1.035960in}{1.855000in}}%
\pgfpathlineto{\pgfqpoint{1.038440in}{2.030000in}}%
\pgfpathlineto{\pgfqpoint{1.040920in}{1.855000in}}%
\pgfpathlineto{\pgfqpoint{1.042160in}{1.295000in}}%
\pgfpathlineto{\pgfqpoint{1.044640in}{1.925000in}}%
\pgfpathlineto{\pgfqpoint{1.045880in}{1.960000in}}%
\pgfpathlineto{\pgfqpoint{1.047120in}{1.295000in}}%
\pgfpathlineto{\pgfqpoint{1.052080in}{2.240000in}}%
\pgfpathlineto{\pgfqpoint{1.053320in}{1.680000in}}%
\pgfpathlineto{\pgfqpoint{1.055800in}{2.030000in}}%
\pgfpathlineto{\pgfqpoint{1.057040in}{2.100000in}}%
\pgfpathlineto{\pgfqpoint{1.058280in}{1.925000in}}%
\pgfpathlineto{\pgfqpoint{1.059520in}{2.170000in}}%
\pgfpathlineto{\pgfqpoint{1.060760in}{1.785000in}}%
\pgfpathlineto{\pgfqpoint{1.062000in}{1.995000in}}%
\pgfpathlineto{\pgfqpoint{1.064480in}{1.995000in}}%
\pgfpathlineto{\pgfqpoint{1.065720in}{2.065000in}}%
\pgfpathlineto{\pgfqpoint{1.066960in}{2.030000in}}%
\pgfpathlineto{\pgfqpoint{1.068200in}{2.100000in}}%
\pgfpathlineto{\pgfqpoint{1.070680in}{1.540000in}}%
\pgfpathlineto{\pgfqpoint{1.071920in}{1.505000in}}%
\pgfpathlineto{\pgfqpoint{1.073160in}{1.715000in}}%
\pgfpathlineto{\pgfqpoint{1.074400in}{2.135000in}}%
\pgfpathlineto{\pgfqpoint{1.075640in}{1.715000in}}%
\pgfpathlineto{\pgfqpoint{1.076880in}{2.240000in}}%
\pgfpathlineto{\pgfqpoint{1.078120in}{1.855000in}}%
\pgfpathlineto{\pgfqpoint{1.079360in}{1.855000in}}%
\pgfpathlineto{\pgfqpoint{1.080600in}{1.575000in}}%
\pgfpathlineto{\pgfqpoint{1.083080in}{2.100000in}}%
\pgfpathlineto{\pgfqpoint{1.084320in}{1.470000in}}%
\pgfpathlineto{\pgfqpoint{1.085560in}{2.240000in}}%
\pgfpathlineto{\pgfqpoint{1.088040in}{1.960000in}}%
\pgfpathlineto{\pgfqpoint{1.089280in}{1.855000in}}%
\pgfpathlineto{\pgfqpoint{1.091760in}{2.205000in}}%
\pgfpathlineto{\pgfqpoint{1.094240in}{1.855000in}}%
\pgfpathlineto{\pgfqpoint{1.095480in}{2.030000in}}%
\pgfpathlineto{\pgfqpoint{1.096720in}{1.960000in}}%
\pgfpathlineto{\pgfqpoint{1.099200in}{1.785000in}}%
\pgfpathlineto{\pgfqpoint{1.100440in}{2.030000in}}%
\pgfpathlineto{\pgfqpoint{1.101680in}{1.680000in}}%
\pgfpathlineto{\pgfqpoint{1.102920in}{1.715000in}}%
\pgfpathlineto{\pgfqpoint{1.104160in}{1.680000in}}%
\pgfpathlineto{\pgfqpoint{1.106640in}{1.995000in}}%
\pgfpathlineto{\pgfqpoint{1.107880in}{1.960000in}}%
\pgfpathlineto{\pgfqpoint{1.109120in}{2.065000in}}%
\pgfpathlineto{\pgfqpoint{1.110360in}{1.645000in}}%
\pgfpathlineto{\pgfqpoint{1.111600in}{2.205000in}}%
\pgfpathlineto{\pgfqpoint{1.112840in}{1.925000in}}%
\pgfpathlineto{\pgfqpoint{1.114080in}{1.925000in}}%
\pgfpathlineto{\pgfqpoint{1.115320in}{1.680000in}}%
\pgfpathlineto{\pgfqpoint{1.116560in}{1.750000in}}%
\pgfpathlineto{\pgfqpoint{1.117800in}{2.275000in}}%
\pgfpathlineto{\pgfqpoint{1.119040in}{1.855000in}}%
\pgfpathlineto{\pgfqpoint{1.121520in}{1.855000in}}%
\pgfpathlineto{\pgfqpoint{1.122760in}{1.680000in}}%
\pgfpathlineto{\pgfqpoint{1.124000in}{1.855000in}}%
\pgfpathlineto{\pgfqpoint{1.125240in}{1.750000in}}%
\pgfpathlineto{\pgfqpoint{1.127720in}{1.435000in}}%
\pgfpathlineto{\pgfqpoint{1.128960in}{1.890000in}}%
\pgfpathlineto{\pgfqpoint{1.130200in}{1.820000in}}%
\pgfpathlineto{\pgfqpoint{1.131440in}{1.925000in}}%
\pgfpathlineto{\pgfqpoint{1.132680in}{1.890000in}}%
\pgfpathlineto{\pgfqpoint{1.133920in}{2.170000in}}%
\pgfpathlineto{\pgfqpoint{1.135160in}{1.785000in}}%
\pgfpathlineto{\pgfqpoint{1.136400in}{1.820000in}}%
\pgfpathlineto{\pgfqpoint{1.137640in}{2.030000in}}%
\pgfpathlineto{\pgfqpoint{1.138880in}{1.610000in}}%
\pgfpathlineto{\pgfqpoint{1.140120in}{1.820000in}}%
\pgfpathlineto{\pgfqpoint{1.141360in}{1.680000in}}%
\pgfpathlineto{\pgfqpoint{1.142600in}{1.820000in}}%
\pgfpathlineto{\pgfqpoint{1.143840in}{1.785000in}}%
\pgfpathlineto{\pgfqpoint{1.145080in}{1.785000in}}%
\pgfpathlineto{\pgfqpoint{1.147560in}{2.275000in}}%
\pgfpathlineto{\pgfqpoint{1.148800in}{2.170000in}}%
\pgfpathlineto{\pgfqpoint{1.151280in}{1.715000in}}%
\pgfpathlineto{\pgfqpoint{1.152520in}{1.960000in}}%
\pgfpathlineto{\pgfqpoint{1.153760in}{1.260000in}}%
\pgfpathlineto{\pgfqpoint{1.156240in}{1.960000in}}%
\pgfpathlineto{\pgfqpoint{1.157480in}{1.925000in}}%
\pgfpathlineto{\pgfqpoint{1.158720in}{1.750000in}}%
\pgfpathlineto{\pgfqpoint{1.159960in}{1.960000in}}%
\pgfpathlineto{\pgfqpoint{1.161200in}{1.785000in}}%
\pgfpathlineto{\pgfqpoint{1.162440in}{1.855000in}}%
\pgfpathlineto{\pgfqpoint{1.163680in}{2.030000in}}%
\pgfpathlineto{\pgfqpoint{1.164920in}{1.820000in}}%
\pgfpathlineto{\pgfqpoint{1.166160in}{1.890000in}}%
\pgfpathlineto{\pgfqpoint{1.167400in}{1.890000in}}%
\pgfpathlineto{\pgfqpoint{1.168640in}{1.855000in}}%
\pgfpathlineto{\pgfqpoint{1.169880in}{1.890000in}}%
\pgfpathlineto{\pgfqpoint{1.171120in}{2.065000in}}%
\pgfpathlineto{\pgfqpoint{1.172360in}{1.960000in}}%
\pgfpathlineto{\pgfqpoint{1.173600in}{2.310000in}}%
\pgfpathlineto{\pgfqpoint{1.176080in}{1.995000in}}%
\pgfpathlineto{\pgfqpoint{1.177320in}{1.925000in}}%
\pgfpathlineto{\pgfqpoint{1.178560in}{2.030000in}}%
\pgfpathlineto{\pgfqpoint{1.179800in}{1.540000in}}%
\pgfpathlineto{\pgfqpoint{1.181040in}{1.610000in}}%
\pgfpathlineto{\pgfqpoint{1.182280in}{2.135000in}}%
\pgfpathlineto{\pgfqpoint{1.184760in}{1.610000in}}%
\pgfpathlineto{\pgfqpoint{1.187240in}{1.680000in}}%
\pgfpathlineto{\pgfqpoint{1.188480in}{1.680000in}}%
\pgfpathlineto{\pgfqpoint{1.189720in}{2.240000in}}%
\pgfpathlineto{\pgfqpoint{1.192200in}{1.680000in}}%
\pgfpathlineto{\pgfqpoint{1.193440in}{2.030000in}}%
\pgfpathlineto{\pgfqpoint{1.194680in}{1.820000in}}%
\pgfpathlineto{\pgfqpoint{1.195920in}{2.065000in}}%
\pgfpathlineto{\pgfqpoint{1.197160in}{2.065000in}}%
\pgfpathlineto{\pgfqpoint{1.198400in}{1.785000in}}%
\pgfpathlineto{\pgfqpoint{1.199640in}{1.820000in}}%
\pgfpathlineto{\pgfqpoint{1.200880in}{2.030000in}}%
\pgfpathlineto{\pgfqpoint{1.202120in}{1.995000in}}%
\pgfpathlineto{\pgfqpoint{1.203360in}{2.135000in}}%
\pgfpathlineto{\pgfqpoint{1.204600in}{2.135000in}}%
\pgfpathlineto{\pgfqpoint{1.205840in}{1.995000in}}%
\pgfpathlineto{\pgfqpoint{1.207080in}{2.205000in}}%
\pgfpathlineto{\pgfqpoint{1.209560in}{1.785000in}}%
\pgfpathlineto{\pgfqpoint{1.210800in}{2.170000in}}%
\pgfpathlineto{\pgfqpoint{1.212040in}{2.205000in}}%
\pgfpathlineto{\pgfqpoint{1.214520in}{1.960000in}}%
\pgfpathlineto{\pgfqpoint{1.215760in}{2.065000in}}%
\pgfpathlineto{\pgfqpoint{1.217000in}{1.645000in}}%
\pgfpathlineto{\pgfqpoint{1.220720in}{2.380000in}}%
\pgfpathlineto{\pgfqpoint{1.221960in}{1.995000in}}%
\pgfpathlineto{\pgfqpoint{1.223200in}{2.065000in}}%
\pgfpathlineto{\pgfqpoint{1.224440in}{2.345000in}}%
\pgfpathlineto{\pgfqpoint{1.225680in}{1.855000in}}%
\pgfpathlineto{\pgfqpoint{1.226920in}{2.065000in}}%
\pgfpathlineto{\pgfqpoint{1.228160in}{1.750000in}}%
\pgfpathlineto{\pgfqpoint{1.231880in}{2.485000in}}%
\pgfpathlineto{\pgfqpoint{1.235600in}{1.890000in}}%
\pgfpathlineto{\pgfqpoint{1.236840in}{2.030000in}}%
\pgfpathlineto{\pgfqpoint{1.238080in}{2.345000in}}%
\pgfpathlineto{\pgfqpoint{1.240560in}{2.100000in}}%
\pgfpathlineto{\pgfqpoint{1.241800in}{2.135000in}}%
\pgfpathlineto{\pgfqpoint{1.243040in}{2.100000in}}%
\pgfpathlineto{\pgfqpoint{1.245520in}{1.820000in}}%
\pgfpathlineto{\pgfqpoint{1.246760in}{1.995000in}}%
\pgfpathlineto{\pgfqpoint{1.250480in}{1.365000in}}%
\pgfpathlineto{\pgfqpoint{1.252960in}{1.750000in}}%
\pgfpathlineto{\pgfqpoint{1.254200in}{1.820000in}}%
\pgfpathlineto{\pgfqpoint{1.255440in}{1.505000in}}%
\pgfpathlineto{\pgfqpoint{1.256680in}{1.505000in}}%
\pgfpathlineto{\pgfqpoint{1.257920in}{1.855000in}}%
\pgfpathlineto{\pgfqpoint{1.259160in}{1.750000in}}%
\pgfpathlineto{\pgfqpoint{1.260400in}{1.960000in}}%
\pgfpathlineto{\pgfqpoint{1.261640in}{1.680000in}}%
\pgfpathlineto{\pgfqpoint{1.265360in}{2.380000in}}%
\pgfpathlineto{\pgfqpoint{1.266600in}{2.135000in}}%
\pgfpathlineto{\pgfqpoint{1.267840in}{2.170000in}}%
\pgfpathlineto{\pgfqpoint{1.269080in}{1.925000in}}%
\pgfpathlineto{\pgfqpoint{1.271560in}{2.310000in}}%
\pgfpathlineto{\pgfqpoint{1.272800in}{1.645000in}}%
\pgfpathlineto{\pgfqpoint{1.274040in}{1.890000in}}%
\pgfpathlineto{\pgfqpoint{1.275280in}{1.785000in}}%
\pgfpathlineto{\pgfqpoint{1.276520in}{2.030000in}}%
\pgfpathlineto{\pgfqpoint{1.277760in}{1.890000in}}%
\pgfpathlineto{\pgfqpoint{1.280240in}{2.240000in}}%
\pgfpathlineto{\pgfqpoint{1.281480in}{1.995000in}}%
\pgfpathlineto{\pgfqpoint{1.282720in}{1.995000in}}%
\pgfpathlineto{\pgfqpoint{1.283960in}{2.135000in}}%
\pgfpathlineto{\pgfqpoint{1.285200in}{1.820000in}}%
\pgfpathlineto{\pgfqpoint{1.286440in}{2.030000in}}%
\pgfpathlineto{\pgfqpoint{1.287680in}{1.855000in}}%
\pgfpathlineto{\pgfqpoint{1.290160in}{2.135000in}}%
\pgfpathlineto{\pgfqpoint{1.291400in}{1.820000in}}%
\pgfpathlineto{\pgfqpoint{1.292640in}{1.820000in}}%
\pgfpathlineto{\pgfqpoint{1.293880in}{1.995000in}}%
\pgfpathlineto{\pgfqpoint{1.295120in}{1.995000in}}%
\pgfpathlineto{\pgfqpoint{1.296360in}{2.135000in}}%
\pgfpathlineto{\pgfqpoint{1.297600in}{1.925000in}}%
\pgfpathlineto{\pgfqpoint{1.298840in}{2.415000in}}%
\pgfpathlineto{\pgfqpoint{1.300080in}{2.065000in}}%
\pgfpathlineto{\pgfqpoint{1.302560in}{2.275000in}}%
\pgfpathlineto{\pgfqpoint{1.303800in}{1.855000in}}%
\pgfpathlineto{\pgfqpoint{1.306280in}{2.345000in}}%
\pgfpathlineto{\pgfqpoint{1.307520in}{2.100000in}}%
\pgfpathlineto{\pgfqpoint{1.308760in}{1.575000in}}%
\pgfpathlineto{\pgfqpoint{1.312480in}{2.135000in}}%
\pgfpathlineto{\pgfqpoint{1.313720in}{1.505000in}}%
\pgfpathlineto{\pgfqpoint{1.316200in}{2.345000in}}%
\pgfpathlineto{\pgfqpoint{1.317440in}{1.645000in}}%
\pgfpathlineto{\pgfqpoint{1.319920in}{2.345000in}}%
\pgfpathlineto{\pgfqpoint{1.321160in}{1.925000in}}%
\pgfpathlineto{\pgfqpoint{1.322400in}{2.170000in}}%
\pgfpathlineto{\pgfqpoint{1.324880in}{1.610000in}}%
\pgfpathlineto{\pgfqpoint{1.326120in}{1.715000in}}%
\pgfpathlineto{\pgfqpoint{1.327360in}{1.785000in}}%
\pgfpathlineto{\pgfqpoint{1.328600in}{1.610000in}}%
\pgfpathlineto{\pgfqpoint{1.329840in}{1.855000in}}%
\pgfpathlineto{\pgfqpoint{1.332320in}{1.680000in}}%
\pgfpathlineto{\pgfqpoint{1.333560in}{2.170000in}}%
\pgfpathlineto{\pgfqpoint{1.334800in}{1.680000in}}%
\pgfpathlineto{\pgfqpoint{1.336040in}{1.855000in}}%
\pgfpathlineto{\pgfqpoint{1.337280in}{1.715000in}}%
\pgfpathlineto{\pgfqpoint{1.338520in}{1.820000in}}%
\pgfpathlineto{\pgfqpoint{1.339760in}{2.275000in}}%
\pgfpathlineto{\pgfqpoint{1.342240in}{1.505000in}}%
\pgfpathlineto{\pgfqpoint{1.347200in}{2.275000in}}%
\pgfpathlineto{\pgfqpoint{1.349680in}{1.680000in}}%
\pgfpathlineto{\pgfqpoint{1.350920in}{1.925000in}}%
\pgfpathlineto{\pgfqpoint{1.352160in}{1.680000in}}%
\pgfpathlineto{\pgfqpoint{1.353400in}{2.310000in}}%
\pgfpathlineto{\pgfqpoint{1.354640in}{1.890000in}}%
\pgfpathlineto{\pgfqpoint{1.357120in}{2.170000in}}%
\pgfpathlineto{\pgfqpoint{1.358360in}{1.715000in}}%
\pgfpathlineto{\pgfqpoint{1.359600in}{1.785000in}}%
\pgfpathlineto{\pgfqpoint{1.360840in}{2.100000in}}%
\pgfpathlineto{\pgfqpoint{1.363320in}{1.575000in}}%
\pgfpathlineto{\pgfqpoint{1.364560in}{1.680000in}}%
\pgfpathlineto{\pgfqpoint{1.365800in}{1.890000in}}%
\pgfpathlineto{\pgfqpoint{1.367040in}{1.890000in}}%
\pgfpathlineto{\pgfqpoint{1.368280in}{2.520000in}}%
\pgfpathlineto{\pgfqpoint{1.369520in}{2.065000in}}%
\pgfpathlineto{\pgfqpoint{1.370760in}{2.415000in}}%
\pgfpathlineto{\pgfqpoint{1.372000in}{2.205000in}}%
\pgfpathlineto{\pgfqpoint{1.373240in}{2.240000in}}%
\pgfpathlineto{\pgfqpoint{1.374480in}{2.310000in}}%
\pgfpathlineto{\pgfqpoint{1.375720in}{2.240000in}}%
\pgfpathlineto{\pgfqpoint{1.376960in}{1.855000in}}%
\pgfpathlineto{\pgfqpoint{1.378200in}{2.065000in}}%
\pgfpathlineto{\pgfqpoint{1.379440in}{1.960000in}}%
\pgfpathlineto{\pgfqpoint{1.380680in}{2.240000in}}%
\pgfpathlineto{\pgfqpoint{1.383160in}{2.240000in}}%
\pgfpathlineto{\pgfqpoint{1.385640in}{1.855000in}}%
\pgfpathlineto{\pgfqpoint{1.386880in}{2.030000in}}%
\pgfpathlineto{\pgfqpoint{1.388120in}{1.750000in}}%
\pgfpathlineto{\pgfqpoint{1.389360in}{1.785000in}}%
\pgfpathlineto{\pgfqpoint{1.390600in}{1.960000in}}%
\pgfpathlineto{\pgfqpoint{1.391840in}{1.925000in}}%
\pgfpathlineto{\pgfqpoint{1.393080in}{2.205000in}}%
\pgfpathlineto{\pgfqpoint{1.394320in}{2.205000in}}%
\pgfpathlineto{\pgfqpoint{1.396800in}{1.855000in}}%
\pgfpathlineto{\pgfqpoint{1.398040in}{2.170000in}}%
\pgfpathlineto{\pgfqpoint{1.399280in}{1.925000in}}%
\pgfpathlineto{\pgfqpoint{1.400520in}{1.960000in}}%
\pgfpathlineto{\pgfqpoint{1.403000in}{2.240000in}}%
\pgfpathlineto{\pgfqpoint{1.405480in}{1.365000in}}%
\pgfpathlineto{\pgfqpoint{1.406720in}{1.470000in}}%
\pgfpathlineto{\pgfqpoint{1.407960in}{1.470000in}}%
\pgfpathlineto{\pgfqpoint{1.410440in}{2.205000in}}%
\pgfpathlineto{\pgfqpoint{1.411680in}{2.310000in}}%
\pgfpathlineto{\pgfqpoint{1.414160in}{1.750000in}}%
\pgfpathlineto{\pgfqpoint{1.416640in}{1.995000in}}%
\pgfpathlineto{\pgfqpoint{1.417880in}{1.610000in}}%
\pgfpathlineto{\pgfqpoint{1.420360in}{2.170000in}}%
\pgfpathlineto{\pgfqpoint{1.421600in}{2.065000in}}%
\pgfpathlineto{\pgfqpoint{1.422840in}{1.680000in}}%
\pgfpathlineto{\pgfqpoint{1.424080in}{2.310000in}}%
\pgfpathlineto{\pgfqpoint{1.425320in}{1.820000in}}%
\pgfpathlineto{\pgfqpoint{1.427800in}{2.380000in}}%
\pgfpathlineto{\pgfqpoint{1.429040in}{2.485000in}}%
\pgfpathlineto{\pgfqpoint{1.430280in}{1.925000in}}%
\pgfpathlineto{\pgfqpoint{1.431520in}{2.100000in}}%
\pgfpathlineto{\pgfqpoint{1.432760in}{1.820000in}}%
\pgfpathlineto{\pgfqpoint{1.435240in}{1.995000in}}%
\pgfpathlineto{\pgfqpoint{1.437720in}{1.785000in}}%
\pgfpathlineto{\pgfqpoint{1.438960in}{2.100000in}}%
\pgfpathlineto{\pgfqpoint{1.440200in}{1.785000in}}%
\pgfpathlineto{\pgfqpoint{1.441440in}{1.785000in}}%
\pgfpathlineto{\pgfqpoint{1.442680in}{1.890000in}}%
\pgfpathlineto{\pgfqpoint{1.443920in}{1.470000in}}%
\pgfpathlineto{\pgfqpoint{1.446400in}{2.030000in}}%
\pgfpathlineto{\pgfqpoint{1.447640in}{1.925000in}}%
\pgfpathlineto{\pgfqpoint{1.448880in}{1.995000in}}%
\pgfpathlineto{\pgfqpoint{1.450120in}{1.645000in}}%
\pgfpathlineto{\pgfqpoint{1.452600in}{1.925000in}}%
\pgfpathlineto{\pgfqpoint{1.453840in}{1.890000in}}%
\pgfpathlineto{\pgfqpoint{1.455080in}{1.925000in}}%
\pgfpathlineto{\pgfqpoint{1.456320in}{2.205000in}}%
\pgfpathlineto{\pgfqpoint{1.458800in}{1.785000in}}%
\pgfpathlineto{\pgfqpoint{1.460040in}{1.820000in}}%
\pgfpathlineto{\pgfqpoint{1.461280in}{1.610000in}}%
\pgfpathlineto{\pgfqpoint{1.463760in}{2.240000in}}%
\pgfpathlineto{\pgfqpoint{1.465000in}{1.855000in}}%
\pgfpathlineto{\pgfqpoint{1.467480in}{2.030000in}}%
\pgfpathlineto{\pgfqpoint{1.468720in}{1.645000in}}%
\pgfpathlineto{\pgfqpoint{1.469960in}{1.890000in}}%
\pgfpathlineto{\pgfqpoint{1.471200in}{1.645000in}}%
\pgfpathlineto{\pgfqpoint{1.472440in}{1.890000in}}%
\pgfpathlineto{\pgfqpoint{1.473680in}{1.890000in}}%
\pgfpathlineto{\pgfqpoint{1.474920in}{2.100000in}}%
\pgfpathlineto{\pgfqpoint{1.476160in}{1.995000in}}%
\pgfpathlineto{\pgfqpoint{1.477400in}{1.505000in}}%
\pgfpathlineto{\pgfqpoint{1.479880in}{2.450000in}}%
\pgfpathlineto{\pgfqpoint{1.481120in}{1.610000in}}%
\pgfpathlineto{\pgfqpoint{1.483600in}{1.855000in}}%
\pgfpathlineto{\pgfqpoint{1.484840in}{1.715000in}}%
\pgfpathlineto{\pgfqpoint{1.486080in}{2.310000in}}%
\pgfpathlineto{\pgfqpoint{1.489800in}{1.400000in}}%
\pgfpathlineto{\pgfqpoint{1.491040in}{2.135000in}}%
\pgfpathlineto{\pgfqpoint{1.492280in}{1.505000in}}%
\pgfpathlineto{\pgfqpoint{1.494760in}{1.925000in}}%
\pgfpathlineto{\pgfqpoint{1.496000in}{1.610000in}}%
\pgfpathlineto{\pgfqpoint{1.498480in}{1.995000in}}%
\pgfpathlineto{\pgfqpoint{1.500960in}{1.400000in}}%
\pgfpathlineto{\pgfqpoint{1.502200in}{2.135000in}}%
\pgfpathlineto{\pgfqpoint{1.503440in}{1.750000in}}%
\pgfpathlineto{\pgfqpoint{1.504680in}{1.960000in}}%
\pgfpathlineto{\pgfqpoint{1.505920in}{1.855000in}}%
\pgfpathlineto{\pgfqpoint{1.507160in}{2.345000in}}%
\pgfpathlineto{\pgfqpoint{1.509640in}{1.680000in}}%
\pgfpathlineto{\pgfqpoint{1.512120in}{2.030000in}}%
\pgfpathlineto{\pgfqpoint{1.514600in}{1.575000in}}%
\pgfpathlineto{\pgfqpoint{1.515840in}{1.680000in}}%
\pgfpathlineto{\pgfqpoint{1.517080in}{1.540000in}}%
\pgfpathlineto{\pgfqpoint{1.518320in}{1.575000in}}%
\pgfpathlineto{\pgfqpoint{1.519560in}{1.400000in}}%
\pgfpathlineto{\pgfqpoint{1.520800in}{1.540000in}}%
\pgfpathlineto{\pgfqpoint{1.522040in}{1.890000in}}%
\pgfpathlineto{\pgfqpoint{1.523280in}{1.820000in}}%
\pgfpathlineto{\pgfqpoint{1.524520in}{1.855000in}}%
\pgfpathlineto{\pgfqpoint{1.527000in}{1.680000in}}%
\pgfpathlineto{\pgfqpoint{1.528240in}{1.505000in}}%
\pgfpathlineto{\pgfqpoint{1.530720in}{2.030000in}}%
\pgfpathlineto{\pgfqpoint{1.533200in}{1.890000in}}%
\pgfpathlineto{\pgfqpoint{1.534440in}{2.135000in}}%
\pgfpathlineto{\pgfqpoint{1.536920in}{1.680000in}}%
\pgfpathlineto{\pgfqpoint{1.538160in}{1.855000in}}%
\pgfpathlineto{\pgfqpoint{1.540640in}{1.715000in}}%
\pgfpathlineto{\pgfqpoint{1.541880in}{2.100000in}}%
\pgfpathlineto{\pgfqpoint{1.544360in}{1.785000in}}%
\pgfpathlineto{\pgfqpoint{1.545600in}{1.855000in}}%
\pgfpathlineto{\pgfqpoint{1.546840in}{1.855000in}}%
\pgfpathlineto{\pgfqpoint{1.548080in}{1.925000in}}%
\pgfpathlineto{\pgfqpoint{1.549320in}{1.925000in}}%
\pgfpathlineto{\pgfqpoint{1.550560in}{2.030000in}}%
\pgfpathlineto{\pgfqpoint{1.551800in}{1.750000in}}%
\pgfpathlineto{\pgfqpoint{1.553040in}{2.170000in}}%
\pgfpathlineto{\pgfqpoint{1.555520in}{1.680000in}}%
\pgfpathlineto{\pgfqpoint{1.556760in}{1.610000in}}%
\pgfpathlineto{\pgfqpoint{1.558000in}{2.205000in}}%
\pgfpathlineto{\pgfqpoint{1.559240in}{2.135000in}}%
\pgfpathlineto{\pgfqpoint{1.560480in}{2.170000in}}%
\pgfpathlineto{\pgfqpoint{1.561720in}{2.170000in}}%
\pgfpathlineto{\pgfqpoint{1.564200in}{1.435000in}}%
\pgfpathlineto{\pgfqpoint{1.566680in}{2.030000in}}%
\pgfpathlineto{\pgfqpoint{1.567920in}{1.610000in}}%
\pgfpathlineto{\pgfqpoint{1.569160in}{1.680000in}}%
\pgfpathlineto{\pgfqpoint{1.571640in}{2.135000in}}%
\pgfpathlineto{\pgfqpoint{1.572880in}{1.365000in}}%
\pgfpathlineto{\pgfqpoint{1.574120in}{1.995000in}}%
\pgfpathlineto{\pgfqpoint{1.576600in}{1.715000in}}%
\pgfpathlineto{\pgfqpoint{1.577840in}{2.450000in}}%
\pgfpathlineto{\pgfqpoint{1.579080in}{1.470000in}}%
\pgfpathlineto{\pgfqpoint{1.580320in}{1.995000in}}%
\pgfpathlineto{\pgfqpoint{1.582800in}{1.785000in}}%
\pgfpathlineto{\pgfqpoint{1.584040in}{1.925000in}}%
\pgfpathlineto{\pgfqpoint{1.585280in}{1.925000in}}%
\pgfpathlineto{\pgfqpoint{1.586520in}{1.610000in}}%
\pgfpathlineto{\pgfqpoint{1.590240in}{1.995000in}}%
\pgfpathlineto{\pgfqpoint{1.591480in}{1.925000in}}%
\pgfpathlineto{\pgfqpoint{1.592720in}{1.960000in}}%
\pgfpathlineto{\pgfqpoint{1.593960in}{1.715000in}}%
\pgfpathlineto{\pgfqpoint{1.595200in}{1.960000in}}%
\pgfpathlineto{\pgfqpoint{1.596440in}{1.890000in}}%
\pgfpathlineto{\pgfqpoint{1.597680in}{1.400000in}}%
\pgfpathlineto{\pgfqpoint{1.600160in}{1.890000in}}%
\pgfpathlineto{\pgfqpoint{1.601400in}{1.890000in}}%
\pgfpathlineto{\pgfqpoint{1.602640in}{1.995000in}}%
\pgfpathlineto{\pgfqpoint{1.603880in}{1.505000in}}%
\pgfpathlineto{\pgfqpoint{1.606360in}{1.855000in}}%
\pgfpathlineto{\pgfqpoint{1.607600in}{1.855000in}}%
\pgfpathlineto{\pgfqpoint{1.610080in}{1.960000in}}%
\pgfpathlineto{\pgfqpoint{1.611320in}{1.470000in}}%
\pgfpathlineto{\pgfqpoint{1.612560in}{1.505000in}}%
\pgfpathlineto{\pgfqpoint{1.615040in}{1.785000in}}%
\pgfpathlineto{\pgfqpoint{1.616280in}{2.310000in}}%
\pgfpathlineto{\pgfqpoint{1.617520in}{1.855000in}}%
\pgfpathlineto{\pgfqpoint{1.618760in}{2.240000in}}%
\pgfpathlineto{\pgfqpoint{1.620000in}{1.960000in}}%
\pgfpathlineto{\pgfqpoint{1.621240in}{1.995000in}}%
\pgfpathlineto{\pgfqpoint{1.622480in}{1.680000in}}%
\pgfpathlineto{\pgfqpoint{1.623720in}{1.890000in}}%
\pgfpathlineto{\pgfqpoint{1.624960in}{1.820000in}}%
\pgfpathlineto{\pgfqpoint{1.626200in}{2.100000in}}%
\pgfpathlineto{\pgfqpoint{1.627440in}{1.785000in}}%
\pgfpathlineto{\pgfqpoint{1.629920in}{2.100000in}}%
\pgfpathlineto{\pgfqpoint{1.631160in}{1.960000in}}%
\pgfpathlineto{\pgfqpoint{1.632400in}{1.960000in}}%
\pgfpathlineto{\pgfqpoint{1.633640in}{2.135000in}}%
\pgfpathlineto{\pgfqpoint{1.634880in}{1.715000in}}%
\pgfpathlineto{\pgfqpoint{1.636120in}{1.785000in}}%
\pgfpathlineto{\pgfqpoint{1.637360in}{2.065000in}}%
\pgfpathlineto{\pgfqpoint{1.639840in}{1.575000in}}%
\pgfpathlineto{\pgfqpoint{1.641080in}{1.855000in}}%
\pgfpathlineto{\pgfqpoint{1.643560in}{1.750000in}}%
\pgfpathlineto{\pgfqpoint{1.644800in}{1.540000in}}%
\pgfpathlineto{\pgfqpoint{1.646040in}{2.275000in}}%
\pgfpathlineto{\pgfqpoint{1.647280in}{1.995000in}}%
\pgfpathlineto{\pgfqpoint{1.648520in}{2.380000in}}%
\pgfpathlineto{\pgfqpoint{1.651000in}{1.540000in}}%
\pgfpathlineto{\pgfqpoint{1.653480in}{2.205000in}}%
\pgfpathlineto{\pgfqpoint{1.655960in}{1.855000in}}%
\pgfpathlineto{\pgfqpoint{1.657200in}{1.575000in}}%
\pgfpathlineto{\pgfqpoint{1.658440in}{2.065000in}}%
\pgfpathlineto{\pgfqpoint{1.659680in}{1.820000in}}%
\pgfpathlineto{\pgfqpoint{1.660920in}{2.310000in}}%
\pgfpathlineto{\pgfqpoint{1.662160in}{1.960000in}}%
\pgfpathlineto{\pgfqpoint{1.663400in}{1.925000in}}%
\pgfpathlineto{\pgfqpoint{1.664640in}{2.170000in}}%
\pgfpathlineto{\pgfqpoint{1.665880in}{1.855000in}}%
\pgfpathlineto{\pgfqpoint{1.668360in}{1.925000in}}%
\pgfpathlineto{\pgfqpoint{1.669600in}{1.925000in}}%
\pgfpathlineto{\pgfqpoint{1.670840in}{1.645000in}}%
\pgfpathlineto{\pgfqpoint{1.673320in}{2.100000in}}%
\pgfpathlineto{\pgfqpoint{1.675800in}{1.575000in}}%
\pgfpathlineto{\pgfqpoint{1.677040in}{1.645000in}}%
\pgfpathlineto{\pgfqpoint{1.679520in}{2.240000in}}%
\pgfpathlineto{\pgfqpoint{1.680760in}{1.610000in}}%
\pgfpathlineto{\pgfqpoint{1.682000in}{1.995000in}}%
\pgfpathlineto{\pgfqpoint{1.683240in}{1.960000in}}%
\pgfpathlineto{\pgfqpoint{1.684480in}{2.205000in}}%
\pgfpathlineto{\pgfqpoint{1.686960in}{1.960000in}}%
\pgfpathlineto{\pgfqpoint{1.688200in}{1.925000in}}%
\pgfpathlineto{\pgfqpoint{1.689440in}{2.065000in}}%
\pgfpathlineto{\pgfqpoint{1.690680in}{2.345000in}}%
\pgfpathlineto{\pgfqpoint{1.691920in}{1.680000in}}%
\pgfpathlineto{\pgfqpoint{1.693160in}{2.030000in}}%
\pgfpathlineto{\pgfqpoint{1.695640in}{1.890000in}}%
\pgfpathlineto{\pgfqpoint{1.696880in}{2.205000in}}%
\pgfpathlineto{\pgfqpoint{1.698120in}{1.540000in}}%
\pgfpathlineto{\pgfqpoint{1.699360in}{2.345000in}}%
\pgfpathlineto{\pgfqpoint{1.701840in}{1.645000in}}%
\pgfpathlineto{\pgfqpoint{1.703080in}{1.960000in}}%
\pgfpathlineto{\pgfqpoint{1.704320in}{1.890000in}}%
\pgfpathlineto{\pgfqpoint{1.705560in}{1.575000in}}%
\pgfpathlineto{\pgfqpoint{1.706800in}{1.575000in}}%
\pgfpathlineto{\pgfqpoint{1.708040in}{2.030000in}}%
\pgfpathlineto{\pgfqpoint{1.709280in}{1.435000in}}%
\pgfpathlineto{\pgfqpoint{1.710520in}{1.960000in}}%
\pgfpathlineto{\pgfqpoint{1.711760in}{1.610000in}}%
\pgfpathlineto{\pgfqpoint{1.713000in}{1.645000in}}%
\pgfpathlineto{\pgfqpoint{1.714240in}{1.960000in}}%
\pgfpathlineto{\pgfqpoint{1.716720in}{1.435000in}}%
\pgfpathlineto{\pgfqpoint{1.717960in}{1.575000in}}%
\pgfpathlineto{\pgfqpoint{1.720440in}{2.030000in}}%
\pgfpathlineto{\pgfqpoint{1.721680in}{1.645000in}}%
\pgfpathlineto{\pgfqpoint{1.722920in}{1.680000in}}%
\pgfpathlineto{\pgfqpoint{1.724160in}{1.645000in}}%
\pgfpathlineto{\pgfqpoint{1.727880in}{2.170000in}}%
\pgfpathlineto{\pgfqpoint{1.730360in}{1.540000in}}%
\pgfpathlineto{\pgfqpoint{1.731600in}{1.960000in}}%
\pgfpathlineto{\pgfqpoint{1.732840in}{1.715000in}}%
\pgfpathlineto{\pgfqpoint{1.734080in}{2.205000in}}%
\pgfpathlineto{\pgfqpoint{1.736560in}{1.715000in}}%
\pgfpathlineto{\pgfqpoint{1.737800in}{2.310000in}}%
\pgfpathlineto{\pgfqpoint{1.739040in}{1.960000in}}%
\pgfpathlineto{\pgfqpoint{1.740280in}{2.030000in}}%
\pgfpathlineto{\pgfqpoint{1.742760in}{1.785000in}}%
\pgfpathlineto{\pgfqpoint{1.745240in}{2.100000in}}%
\pgfpathlineto{\pgfqpoint{1.746480in}{1.995000in}}%
\pgfpathlineto{\pgfqpoint{1.747720in}{2.065000in}}%
\pgfpathlineto{\pgfqpoint{1.748960in}{2.345000in}}%
\pgfpathlineto{\pgfqpoint{1.750200in}{2.065000in}}%
\pgfpathlineto{\pgfqpoint{1.751440in}{2.380000in}}%
\pgfpathlineto{\pgfqpoint{1.752680in}{2.170000in}}%
\pgfpathlineto{\pgfqpoint{1.753920in}{2.310000in}}%
\pgfpathlineto{\pgfqpoint{1.755160in}{2.240000in}}%
\pgfpathlineto{\pgfqpoint{1.757640in}{1.820000in}}%
\pgfpathlineto{\pgfqpoint{1.758880in}{1.960000in}}%
\pgfpathlineto{\pgfqpoint{1.761360in}{1.820000in}}%
\pgfpathlineto{\pgfqpoint{1.762600in}{1.890000in}}%
\pgfpathlineto{\pgfqpoint{1.763840in}{1.750000in}}%
\pgfpathlineto{\pgfqpoint{1.765080in}{2.065000in}}%
\pgfpathlineto{\pgfqpoint{1.767560in}{1.715000in}}%
\pgfpathlineto{\pgfqpoint{1.768800in}{1.820000in}}%
\pgfpathlineto{\pgfqpoint{1.770040in}{1.820000in}}%
\pgfpathlineto{\pgfqpoint{1.772520in}{2.310000in}}%
\pgfpathlineto{\pgfqpoint{1.773760in}{1.855000in}}%
\pgfpathlineto{\pgfqpoint{1.775000in}{1.960000in}}%
\pgfpathlineto{\pgfqpoint{1.776240in}{1.890000in}}%
\pgfpathlineto{\pgfqpoint{1.777480in}{1.680000in}}%
\pgfpathlineto{\pgfqpoint{1.778720in}{1.715000in}}%
\pgfpathlineto{\pgfqpoint{1.779960in}{1.855000in}}%
\pgfpathlineto{\pgfqpoint{1.781200in}{1.540000in}}%
\pgfpathlineto{\pgfqpoint{1.782440in}{2.030000in}}%
\pgfpathlineto{\pgfqpoint{1.783680in}{2.065000in}}%
\pgfpathlineto{\pgfqpoint{1.784920in}{1.890000in}}%
\pgfpathlineto{\pgfqpoint{1.786160in}{2.065000in}}%
\pgfpathlineto{\pgfqpoint{1.787400in}{1.925000in}}%
\pgfpathlineto{\pgfqpoint{1.788640in}{2.030000in}}%
\pgfpathlineto{\pgfqpoint{1.789880in}{2.240000in}}%
\pgfpathlineto{\pgfqpoint{1.791120in}{2.065000in}}%
\pgfpathlineto{\pgfqpoint{1.792360in}{2.275000in}}%
\pgfpathlineto{\pgfqpoint{1.793600in}{2.100000in}}%
\pgfpathlineto{\pgfqpoint{1.794840in}{1.750000in}}%
\pgfpathlineto{\pgfqpoint{1.796080in}{1.820000in}}%
\pgfpathlineto{\pgfqpoint{1.797320in}{2.135000in}}%
\pgfpathlineto{\pgfqpoint{1.798560in}{1.890000in}}%
\pgfpathlineto{\pgfqpoint{1.799800in}{2.240000in}}%
\pgfpathlineto{\pgfqpoint{1.802280in}{1.890000in}}%
\pgfpathlineto{\pgfqpoint{1.803520in}{2.100000in}}%
\pgfpathlineto{\pgfqpoint{1.804760in}{1.575000in}}%
\pgfpathlineto{\pgfqpoint{1.806000in}{1.925000in}}%
\pgfpathlineto{\pgfqpoint{1.807240in}{1.820000in}}%
\pgfpathlineto{\pgfqpoint{1.808480in}{1.890000in}}%
\pgfpathlineto{\pgfqpoint{1.809720in}{1.820000in}}%
\pgfpathlineto{\pgfqpoint{1.810960in}{1.365000in}}%
\pgfpathlineto{\pgfqpoint{1.813440in}{1.925000in}}%
\pgfpathlineto{\pgfqpoint{1.814680in}{1.785000in}}%
\pgfpathlineto{\pgfqpoint{1.815920in}{1.995000in}}%
\pgfpathlineto{\pgfqpoint{1.817160in}{1.820000in}}%
\pgfpathlineto{\pgfqpoint{1.818400in}{1.470000in}}%
\pgfpathlineto{\pgfqpoint{1.819640in}{1.995000in}}%
\pgfpathlineto{\pgfqpoint{1.820880in}{1.820000in}}%
\pgfpathlineto{\pgfqpoint{1.823360in}{2.030000in}}%
\pgfpathlineto{\pgfqpoint{1.824600in}{1.505000in}}%
\pgfpathlineto{\pgfqpoint{1.825840in}{2.100000in}}%
\pgfpathlineto{\pgfqpoint{1.827080in}{1.855000in}}%
\pgfpathlineto{\pgfqpoint{1.828320in}{1.890000in}}%
\pgfpathlineto{\pgfqpoint{1.829560in}{2.240000in}}%
\pgfpathlineto{\pgfqpoint{1.830800in}{1.995000in}}%
\pgfpathlineto{\pgfqpoint{1.832040in}{1.505000in}}%
\pgfpathlineto{\pgfqpoint{1.833280in}{2.275000in}}%
\pgfpathlineto{\pgfqpoint{1.834520in}{1.645000in}}%
\pgfpathlineto{\pgfqpoint{1.835760in}{2.100000in}}%
\pgfpathlineto{\pgfqpoint{1.837000in}{2.065000in}}%
\pgfpathlineto{\pgfqpoint{1.838240in}{2.205000in}}%
\pgfpathlineto{\pgfqpoint{1.839480in}{1.785000in}}%
\pgfpathlineto{\pgfqpoint{1.840720in}{2.345000in}}%
\pgfpathlineto{\pgfqpoint{1.843200in}{1.680000in}}%
\pgfpathlineto{\pgfqpoint{1.844440in}{1.750000in}}%
\pgfpathlineto{\pgfqpoint{1.846920in}{2.135000in}}%
\pgfpathlineto{\pgfqpoint{1.848160in}{1.680000in}}%
\pgfpathlineto{\pgfqpoint{1.849400in}{2.135000in}}%
\pgfpathlineto{\pgfqpoint{1.850640in}{1.680000in}}%
\pgfpathlineto{\pgfqpoint{1.851880in}{1.645000in}}%
\pgfpathlineto{\pgfqpoint{1.854360in}{1.505000in}}%
\pgfpathlineto{\pgfqpoint{1.855600in}{1.960000in}}%
\pgfpathlineto{\pgfqpoint{1.858080in}{1.645000in}}%
\pgfpathlineto{\pgfqpoint{1.859320in}{1.645000in}}%
\pgfpathlineto{\pgfqpoint{1.860560in}{1.960000in}}%
\pgfpathlineto{\pgfqpoint{1.861800in}{1.715000in}}%
\pgfpathlineto{\pgfqpoint{1.864280in}{1.785000in}}%
\pgfpathlineto{\pgfqpoint{1.865520in}{1.435000in}}%
\pgfpathlineto{\pgfqpoint{1.866760in}{1.855000in}}%
\pgfpathlineto{\pgfqpoint{1.868000in}{1.575000in}}%
\pgfpathlineto{\pgfqpoint{1.870480in}{2.275000in}}%
\pgfpathlineto{\pgfqpoint{1.871720in}{1.785000in}}%
\pgfpathlineto{\pgfqpoint{1.874200in}{2.030000in}}%
\pgfpathlineto{\pgfqpoint{1.875440in}{2.030000in}}%
\pgfpathlineto{\pgfqpoint{1.876680in}{1.715000in}}%
\pgfpathlineto{\pgfqpoint{1.877920in}{1.995000in}}%
\pgfpathlineto{\pgfqpoint{1.879160in}{1.785000in}}%
\pgfpathlineto{\pgfqpoint{1.881640in}{2.240000in}}%
\pgfpathlineto{\pgfqpoint{1.882880in}{1.540000in}}%
\pgfpathlineto{\pgfqpoint{1.885360in}{2.030000in}}%
\pgfpathlineto{\pgfqpoint{1.886600in}{1.925000in}}%
\pgfpathlineto{\pgfqpoint{1.889080in}{1.365000in}}%
\pgfpathlineto{\pgfqpoint{1.891560in}{2.135000in}}%
\pgfpathlineto{\pgfqpoint{1.892800in}{1.435000in}}%
\pgfpathlineto{\pgfqpoint{1.895280in}{1.750000in}}%
\pgfpathlineto{\pgfqpoint{1.896520in}{1.750000in}}%
\pgfpathlineto{\pgfqpoint{1.897760in}{1.995000in}}%
\pgfpathlineto{\pgfqpoint{1.899000in}{1.680000in}}%
\pgfpathlineto{\pgfqpoint{1.900240in}{1.715000in}}%
\pgfpathlineto{\pgfqpoint{1.901480in}{1.645000in}}%
\pgfpathlineto{\pgfqpoint{1.902720in}{1.785000in}}%
\pgfpathlineto{\pgfqpoint{1.905200in}{2.415000in}}%
\pgfpathlineto{\pgfqpoint{1.906440in}{1.715000in}}%
\pgfpathlineto{\pgfqpoint{1.907680in}{1.785000in}}%
\pgfpathlineto{\pgfqpoint{1.908920in}{1.785000in}}%
\pgfpathlineto{\pgfqpoint{1.910160in}{1.470000in}}%
\pgfpathlineto{\pgfqpoint{1.912640in}{1.855000in}}%
\pgfpathlineto{\pgfqpoint{1.913880in}{1.890000in}}%
\pgfpathlineto{\pgfqpoint{1.915120in}{1.820000in}}%
\pgfpathlineto{\pgfqpoint{1.916360in}{1.925000in}}%
\pgfpathlineto{\pgfqpoint{1.917600in}{1.925000in}}%
\pgfpathlineto{\pgfqpoint{1.918840in}{1.890000in}}%
\pgfpathlineto{\pgfqpoint{1.920080in}{1.645000in}}%
\pgfpathlineto{\pgfqpoint{1.921320in}{1.750000in}}%
\pgfpathlineto{\pgfqpoint{1.922560in}{1.610000in}}%
\pgfpathlineto{\pgfqpoint{1.923800in}{1.785000in}}%
\pgfpathlineto{\pgfqpoint{1.925040in}{1.680000in}}%
\pgfpathlineto{\pgfqpoint{1.926280in}{1.820000in}}%
\pgfpathlineto{\pgfqpoint{1.927520in}{1.435000in}}%
\pgfpathlineto{\pgfqpoint{1.931240in}{2.345000in}}%
\pgfpathlineto{\pgfqpoint{1.933720in}{1.750000in}}%
\pgfpathlineto{\pgfqpoint{1.934960in}{1.610000in}}%
\pgfpathlineto{\pgfqpoint{1.936200in}{1.820000in}}%
\pgfpathlineto{\pgfqpoint{1.937440in}{1.715000in}}%
\pgfpathlineto{\pgfqpoint{1.938680in}{1.715000in}}%
\pgfpathlineto{\pgfqpoint{1.939920in}{1.225000in}}%
\pgfpathlineto{\pgfqpoint{1.942400in}{1.855000in}}%
\pgfpathlineto{\pgfqpoint{1.943640in}{1.855000in}}%
\pgfpathlineto{\pgfqpoint{1.944880in}{1.960000in}}%
\pgfpathlineto{\pgfqpoint{1.946120in}{1.925000in}}%
\pgfpathlineto{\pgfqpoint{1.947360in}{1.995000in}}%
\pgfpathlineto{\pgfqpoint{1.948600in}{1.645000in}}%
\pgfpathlineto{\pgfqpoint{1.951080in}{2.380000in}}%
\pgfpathlineto{\pgfqpoint{1.954800in}{1.330000in}}%
\pgfpathlineto{\pgfqpoint{1.956040in}{1.750000in}}%
\pgfpathlineto{\pgfqpoint{1.957280in}{1.575000in}}%
\pgfpathlineto{\pgfqpoint{1.958520in}{1.925000in}}%
\pgfpathlineto{\pgfqpoint{1.959760in}{1.785000in}}%
\pgfpathlineto{\pgfqpoint{1.961000in}{1.960000in}}%
\pgfpathlineto{\pgfqpoint{1.962240in}{1.890000in}}%
\pgfpathlineto{\pgfqpoint{1.963480in}{1.750000in}}%
\pgfpathlineto{\pgfqpoint{1.964720in}{1.855000in}}%
\pgfpathlineto{\pgfqpoint{1.965960in}{1.645000in}}%
\pgfpathlineto{\pgfqpoint{1.967200in}{1.890000in}}%
\pgfpathlineto{\pgfqpoint{1.968440in}{1.890000in}}%
\pgfpathlineto{\pgfqpoint{1.969680in}{1.960000in}}%
\pgfpathlineto{\pgfqpoint{1.970920in}{1.925000in}}%
\pgfpathlineto{\pgfqpoint{1.972160in}{1.820000in}}%
\pgfpathlineto{\pgfqpoint{1.973400in}{2.205000in}}%
\pgfpathlineto{\pgfqpoint{1.974640in}{2.030000in}}%
\pgfpathlineto{\pgfqpoint{1.975880in}{1.505000in}}%
\pgfpathlineto{\pgfqpoint{1.980840in}{2.100000in}}%
\pgfpathlineto{\pgfqpoint{1.982080in}{1.680000in}}%
\pgfpathlineto{\pgfqpoint{1.983320in}{1.960000in}}%
\pgfpathlineto{\pgfqpoint{1.985800in}{1.470000in}}%
\pgfpathlineto{\pgfqpoint{1.988280in}{1.855000in}}%
\pgfpathlineto{\pgfqpoint{1.989520in}{1.855000in}}%
\pgfpathlineto{\pgfqpoint{1.990760in}{2.170000in}}%
\pgfpathlineto{\pgfqpoint{1.992000in}{1.820000in}}%
\pgfpathlineto{\pgfqpoint{1.993240in}{2.170000in}}%
\pgfpathlineto{\pgfqpoint{1.994480in}{1.995000in}}%
\pgfpathlineto{\pgfqpoint{1.995720in}{1.400000in}}%
\pgfpathlineto{\pgfqpoint{1.998200in}{1.820000in}}%
\pgfpathlineto{\pgfqpoint{2.000680in}{1.610000in}}%
\pgfpathlineto{\pgfqpoint{2.001920in}{1.680000in}}%
\pgfpathlineto{\pgfqpoint{2.004400in}{2.030000in}}%
\pgfpathlineto{\pgfqpoint{2.005640in}{1.820000in}}%
\pgfpathlineto{\pgfqpoint{2.008120in}{2.240000in}}%
\pgfpathlineto{\pgfqpoint{2.009360in}{1.925000in}}%
\pgfpathlineto{\pgfqpoint{2.010600in}{2.030000in}}%
\pgfpathlineto{\pgfqpoint{2.011840in}{1.995000in}}%
\pgfpathlineto{\pgfqpoint{2.014320in}{1.715000in}}%
\pgfpathlineto{\pgfqpoint{2.015560in}{1.960000in}}%
\pgfpathlineto{\pgfqpoint{2.016800in}{1.890000in}}%
\pgfpathlineto{\pgfqpoint{2.018040in}{2.240000in}}%
\pgfpathlineto{\pgfqpoint{2.019280in}{1.610000in}}%
\pgfpathlineto{\pgfqpoint{2.020520in}{2.065000in}}%
\pgfpathlineto{\pgfqpoint{2.021760in}{1.855000in}}%
\pgfpathlineto{\pgfqpoint{2.023000in}{1.925000in}}%
\pgfpathlineto{\pgfqpoint{2.024240in}{2.205000in}}%
\pgfpathlineto{\pgfqpoint{2.025480in}{2.135000in}}%
\pgfpathlineto{\pgfqpoint{2.026720in}{2.135000in}}%
\pgfpathlineto{\pgfqpoint{2.029200in}{1.925000in}}%
\pgfpathlineto{\pgfqpoint{2.030440in}{2.240000in}}%
\pgfpathlineto{\pgfqpoint{2.031680in}{2.205000in}}%
\pgfpathlineto{\pgfqpoint{2.034160in}{1.750000in}}%
\pgfpathlineto{\pgfqpoint{2.035400in}{1.925000in}}%
\pgfpathlineto{\pgfqpoint{2.036640in}{1.610000in}}%
\pgfpathlineto{\pgfqpoint{2.037880in}{2.100000in}}%
\pgfpathlineto{\pgfqpoint{2.040360in}{1.715000in}}%
\pgfpathlineto{\pgfqpoint{2.041600in}{1.855000in}}%
\pgfpathlineto{\pgfqpoint{2.042840in}{1.855000in}}%
\pgfpathlineto{\pgfqpoint{2.044080in}{2.030000in}}%
\pgfpathlineto{\pgfqpoint{2.045320in}{1.785000in}}%
\pgfpathlineto{\pgfqpoint{2.046560in}{2.030000in}}%
\pgfpathlineto{\pgfqpoint{2.047800in}{1.960000in}}%
\pgfpathlineto{\pgfqpoint{2.049040in}{1.750000in}}%
\pgfpathlineto{\pgfqpoint{2.050280in}{1.925000in}}%
\pgfpathlineto{\pgfqpoint{2.051520in}{1.925000in}}%
\pgfpathlineto{\pgfqpoint{2.055240in}{1.365000in}}%
\pgfpathlineto{\pgfqpoint{2.057720in}{2.030000in}}%
\pgfpathlineto{\pgfqpoint{2.058960in}{1.400000in}}%
\pgfpathlineto{\pgfqpoint{2.060200in}{1.820000in}}%
\pgfpathlineto{\pgfqpoint{2.062680in}{1.365000in}}%
\pgfpathlineto{\pgfqpoint{2.063920in}{1.785000in}}%
\pgfpathlineto{\pgfqpoint{2.065160in}{1.820000in}}%
\pgfpathlineto{\pgfqpoint{2.066400in}{2.030000in}}%
\pgfpathlineto{\pgfqpoint{2.068880in}{1.645000in}}%
\pgfpathlineto{\pgfqpoint{2.070120in}{1.995000in}}%
\pgfpathlineto{\pgfqpoint{2.071360in}{1.750000in}}%
\pgfpathlineto{\pgfqpoint{2.072600in}{1.890000in}}%
\pgfpathlineto{\pgfqpoint{2.075080in}{1.750000in}}%
\pgfpathlineto{\pgfqpoint{2.076320in}{1.785000in}}%
\pgfpathlineto{\pgfqpoint{2.078800in}{1.960000in}}%
\pgfpathlineto{\pgfqpoint{2.080040in}{1.785000in}}%
\pgfpathlineto{\pgfqpoint{2.081280in}{2.065000in}}%
\pgfpathlineto{\pgfqpoint{2.082520in}{1.960000in}}%
\pgfpathlineto{\pgfqpoint{2.083760in}{2.100000in}}%
\pgfpathlineto{\pgfqpoint{2.085000in}{1.820000in}}%
\pgfpathlineto{\pgfqpoint{2.086240in}{1.855000in}}%
\pgfpathlineto{\pgfqpoint{2.087480in}{1.995000in}}%
\pgfpathlineto{\pgfqpoint{2.088720in}{1.855000in}}%
\pgfpathlineto{\pgfqpoint{2.089960in}{1.540000in}}%
\pgfpathlineto{\pgfqpoint{2.092440in}{2.100000in}}%
\pgfpathlineto{\pgfqpoint{2.094920in}{1.750000in}}%
\pgfpathlineto{\pgfqpoint{2.096160in}{1.855000in}}%
\pgfpathlineto{\pgfqpoint{2.097400in}{2.065000in}}%
\pgfpathlineto{\pgfqpoint{2.098640in}{1.855000in}}%
\pgfpathlineto{\pgfqpoint{2.099880in}{2.030000in}}%
\pgfpathlineto{\pgfqpoint{2.101120in}{1.925000in}}%
\pgfpathlineto{\pgfqpoint{2.102360in}{2.170000in}}%
\pgfpathlineto{\pgfqpoint{2.103600in}{2.170000in}}%
\pgfpathlineto{\pgfqpoint{2.104840in}{1.925000in}}%
\pgfpathlineto{\pgfqpoint{2.106080in}{1.995000in}}%
\pgfpathlineto{\pgfqpoint{2.107320in}{1.680000in}}%
\pgfpathlineto{\pgfqpoint{2.108560in}{1.890000in}}%
\pgfpathlineto{\pgfqpoint{2.109800in}{1.890000in}}%
\pgfpathlineto{\pgfqpoint{2.112280in}{1.960000in}}%
\pgfpathlineto{\pgfqpoint{2.113520in}{1.610000in}}%
\pgfpathlineto{\pgfqpoint{2.116000in}{2.065000in}}%
\pgfpathlineto{\pgfqpoint{2.117240in}{1.890000in}}%
\pgfpathlineto{\pgfqpoint{2.118480in}{1.890000in}}%
\pgfpathlineto{\pgfqpoint{2.119720in}{1.855000in}}%
\pgfpathlineto{\pgfqpoint{2.120960in}{1.435000in}}%
\pgfpathlineto{\pgfqpoint{2.124680in}{2.240000in}}%
\pgfpathlineto{\pgfqpoint{2.127160in}{1.820000in}}%
\pgfpathlineto{\pgfqpoint{2.128400in}{1.820000in}}%
\pgfpathlineto{\pgfqpoint{2.130880in}{2.380000in}}%
\pgfpathlineto{\pgfqpoint{2.132120in}{1.610000in}}%
\pgfpathlineto{\pgfqpoint{2.134600in}{1.960000in}}%
\pgfpathlineto{\pgfqpoint{2.135840in}{1.750000in}}%
\pgfpathlineto{\pgfqpoint{2.138320in}{2.030000in}}%
\pgfpathlineto{\pgfqpoint{2.139560in}{1.715000in}}%
\pgfpathlineto{\pgfqpoint{2.140800in}{2.135000in}}%
\pgfpathlineto{\pgfqpoint{2.142040in}{1.855000in}}%
\pgfpathlineto{\pgfqpoint{2.143280in}{1.960000in}}%
\pgfpathlineto{\pgfqpoint{2.147000in}{1.855000in}}%
\pgfpathlineto{\pgfqpoint{2.148240in}{1.855000in}}%
\pgfpathlineto{\pgfqpoint{2.149480in}{1.820000in}}%
\pgfpathlineto{\pgfqpoint{2.150720in}{1.890000in}}%
\pgfpathlineto{\pgfqpoint{2.151960in}{2.030000in}}%
\pgfpathlineto{\pgfqpoint{2.153200in}{1.855000in}}%
\pgfpathlineto{\pgfqpoint{2.154440in}{2.065000in}}%
\pgfpathlineto{\pgfqpoint{2.155680in}{1.610000in}}%
\pgfpathlineto{\pgfqpoint{2.159400in}{2.100000in}}%
\pgfpathlineto{\pgfqpoint{2.160640in}{1.435000in}}%
\pgfpathlineto{\pgfqpoint{2.161880in}{2.170000in}}%
\pgfpathlineto{\pgfqpoint{2.164360in}{1.820000in}}%
\pgfpathlineto{\pgfqpoint{2.165600in}{1.890000in}}%
\pgfpathlineto{\pgfqpoint{2.166840in}{2.310000in}}%
\pgfpathlineto{\pgfqpoint{2.168080in}{1.960000in}}%
\pgfpathlineto{\pgfqpoint{2.170560in}{2.240000in}}%
\pgfpathlineto{\pgfqpoint{2.171800in}{1.925000in}}%
\pgfpathlineto{\pgfqpoint{2.173040in}{2.380000in}}%
\pgfpathlineto{\pgfqpoint{2.174280in}{2.065000in}}%
\pgfpathlineto{\pgfqpoint{2.175520in}{2.205000in}}%
\pgfpathlineto{\pgfqpoint{2.176760in}{1.925000in}}%
\pgfpathlineto{\pgfqpoint{2.178000in}{1.995000in}}%
\pgfpathlineto{\pgfqpoint{2.180480in}{2.275000in}}%
\pgfpathlineto{\pgfqpoint{2.181720in}{1.995000in}}%
\pgfpathlineto{\pgfqpoint{2.182960in}{2.170000in}}%
\pgfpathlineto{\pgfqpoint{2.185440in}{1.820000in}}%
\pgfpathlineto{\pgfqpoint{2.186680in}{1.925000in}}%
\pgfpathlineto{\pgfqpoint{2.187920in}{1.890000in}}%
\pgfpathlineto{\pgfqpoint{2.189160in}{2.030000in}}%
\pgfpathlineto{\pgfqpoint{2.190400in}{1.925000in}}%
\pgfpathlineto{\pgfqpoint{2.191640in}{1.680000in}}%
\pgfpathlineto{\pgfqpoint{2.192880in}{2.135000in}}%
\pgfpathlineto{\pgfqpoint{2.194120in}{1.785000in}}%
\pgfpathlineto{\pgfqpoint{2.195360in}{2.135000in}}%
\pgfpathlineto{\pgfqpoint{2.196600in}{2.100000in}}%
\pgfpathlineto{\pgfqpoint{2.197840in}{1.960000in}}%
\pgfpathlineto{\pgfqpoint{2.200320in}{2.240000in}}%
\pgfpathlineto{\pgfqpoint{2.201560in}{2.030000in}}%
\pgfpathlineto{\pgfqpoint{2.202800in}{2.065000in}}%
\pgfpathlineto{\pgfqpoint{2.204040in}{1.645000in}}%
\pgfpathlineto{\pgfqpoint{2.207760in}{2.310000in}}%
\pgfpathlineto{\pgfqpoint{2.210240in}{1.715000in}}%
\pgfpathlineto{\pgfqpoint{2.211480in}{2.310000in}}%
\pgfpathlineto{\pgfqpoint{2.212720in}{2.310000in}}%
\pgfpathlineto{\pgfqpoint{2.213960in}{2.065000in}}%
\pgfpathlineto{\pgfqpoint{2.215200in}{2.205000in}}%
\pgfpathlineto{\pgfqpoint{2.216440in}{1.820000in}}%
\pgfpathlineto{\pgfqpoint{2.217680in}{1.890000in}}%
\pgfpathlineto{\pgfqpoint{2.218920in}{1.680000in}}%
\pgfpathlineto{\pgfqpoint{2.220160in}{1.925000in}}%
\pgfpathlineto{\pgfqpoint{2.221400in}{1.785000in}}%
\pgfpathlineto{\pgfqpoint{2.223880in}{2.240000in}}%
\pgfpathlineto{\pgfqpoint{2.226360in}{1.995000in}}%
\pgfpathlineto{\pgfqpoint{2.227600in}{2.275000in}}%
\pgfpathlineto{\pgfqpoint{2.230080in}{1.610000in}}%
\pgfpathlineto{\pgfqpoint{2.232560in}{2.310000in}}%
\pgfpathlineto{\pgfqpoint{2.236280in}{1.750000in}}%
\pgfpathlineto{\pgfqpoint{2.237520in}{1.855000in}}%
\pgfpathlineto{\pgfqpoint{2.238760in}{1.785000in}}%
\pgfpathlineto{\pgfqpoint{2.241240in}{2.065000in}}%
\pgfpathlineto{\pgfqpoint{2.243720in}{1.820000in}}%
\pgfpathlineto{\pgfqpoint{2.244960in}{2.135000in}}%
\pgfpathlineto{\pgfqpoint{2.246200in}{1.610000in}}%
\pgfpathlineto{\pgfqpoint{2.248680in}{2.275000in}}%
\pgfpathlineto{\pgfqpoint{2.249920in}{2.135000in}}%
\pgfpathlineto{\pgfqpoint{2.251160in}{1.820000in}}%
\pgfpathlineto{\pgfqpoint{2.252400in}{1.855000in}}%
\pgfpathlineto{\pgfqpoint{2.253640in}{1.855000in}}%
\pgfpathlineto{\pgfqpoint{2.254880in}{1.470000in}}%
\pgfpathlineto{\pgfqpoint{2.257360in}{2.205000in}}%
\pgfpathlineto{\pgfqpoint{2.258600in}{1.575000in}}%
\pgfpathlineto{\pgfqpoint{2.259840in}{1.995000in}}%
\pgfpathlineto{\pgfqpoint{2.262320in}{1.505000in}}%
\pgfpathlineto{\pgfqpoint{2.264800in}{1.855000in}}%
\pgfpathlineto{\pgfqpoint{2.266040in}{1.820000in}}%
\pgfpathlineto{\pgfqpoint{2.267280in}{1.680000in}}%
\pgfpathlineto{\pgfqpoint{2.268520in}{2.100000in}}%
\pgfpathlineto{\pgfqpoint{2.269760in}{2.065000in}}%
\pgfpathlineto{\pgfqpoint{2.271000in}{1.715000in}}%
\pgfpathlineto{\pgfqpoint{2.274720in}{2.030000in}}%
\pgfpathlineto{\pgfqpoint{2.277200in}{1.855000in}}%
\pgfpathlineto{\pgfqpoint{2.278440in}{2.030000in}}%
\pgfpathlineto{\pgfqpoint{2.279680in}{2.030000in}}%
\pgfpathlineto{\pgfqpoint{2.280920in}{1.890000in}}%
\pgfpathlineto{\pgfqpoint{2.282160in}{1.610000in}}%
\pgfpathlineto{\pgfqpoint{2.283400in}{1.715000in}}%
\pgfpathlineto{\pgfqpoint{2.285880in}{1.960000in}}%
\pgfpathlineto{\pgfqpoint{2.287120in}{1.610000in}}%
\pgfpathlineto{\pgfqpoint{2.288360in}{1.785000in}}%
\pgfpathlineto{\pgfqpoint{2.289600in}{1.645000in}}%
\pgfpathlineto{\pgfqpoint{2.292080in}{2.065000in}}%
\pgfpathlineto{\pgfqpoint{2.293320in}{1.890000in}}%
\pgfpathlineto{\pgfqpoint{2.295800in}{2.310000in}}%
\pgfpathlineto{\pgfqpoint{2.297040in}{2.030000in}}%
\pgfpathlineto{\pgfqpoint{2.299520in}{2.345000in}}%
\pgfpathlineto{\pgfqpoint{2.300760in}{1.995000in}}%
\pgfpathlineto{\pgfqpoint{2.302000in}{2.275000in}}%
\pgfpathlineto{\pgfqpoint{2.303240in}{2.170000in}}%
\pgfpathlineto{\pgfqpoint{2.304480in}{2.170000in}}%
\pgfpathlineto{\pgfqpoint{2.305720in}{1.960000in}}%
\pgfpathlineto{\pgfqpoint{2.306960in}{2.415000in}}%
\pgfpathlineto{\pgfqpoint{2.309440in}{1.855000in}}%
\pgfpathlineto{\pgfqpoint{2.310680in}{1.750000in}}%
\pgfpathlineto{\pgfqpoint{2.311920in}{1.890000in}}%
\pgfpathlineto{\pgfqpoint{2.313160in}{1.890000in}}%
\pgfpathlineto{\pgfqpoint{2.314400in}{1.925000in}}%
\pgfpathlineto{\pgfqpoint{2.316880in}{1.680000in}}%
\pgfpathlineto{\pgfqpoint{2.319360in}{2.450000in}}%
\pgfpathlineto{\pgfqpoint{2.320600in}{1.820000in}}%
\pgfpathlineto{\pgfqpoint{2.321840in}{1.890000in}}%
\pgfpathlineto{\pgfqpoint{2.323080in}{2.170000in}}%
\pgfpathlineto{\pgfqpoint{2.324320in}{2.065000in}}%
\pgfpathlineto{\pgfqpoint{2.325560in}{2.310000in}}%
\pgfpathlineto{\pgfqpoint{2.326800in}{2.205000in}}%
\pgfpathlineto{\pgfqpoint{2.328040in}{2.205000in}}%
\pgfpathlineto{\pgfqpoint{2.329280in}{2.135000in}}%
\pgfpathlineto{\pgfqpoint{2.330520in}{2.170000in}}%
\pgfpathlineto{\pgfqpoint{2.331760in}{2.345000in}}%
\pgfpathlineto{\pgfqpoint{2.333000in}{1.855000in}}%
\pgfpathlineto{\pgfqpoint{2.334240in}{2.205000in}}%
\pgfpathlineto{\pgfqpoint{2.335480in}{1.960000in}}%
\pgfpathlineto{\pgfqpoint{2.336720in}{1.960000in}}%
\pgfpathlineto{\pgfqpoint{2.337960in}{2.065000in}}%
\pgfpathlineto{\pgfqpoint{2.339200in}{2.065000in}}%
\pgfpathlineto{\pgfqpoint{2.340440in}{2.205000in}}%
\pgfpathlineto{\pgfqpoint{2.344160in}{1.750000in}}%
\pgfpathlineto{\pgfqpoint{2.345400in}{2.240000in}}%
\pgfpathlineto{\pgfqpoint{2.346640in}{2.170000in}}%
\pgfpathlineto{\pgfqpoint{2.347880in}{1.855000in}}%
\pgfpathlineto{\pgfqpoint{2.349120in}{2.030000in}}%
\pgfpathlineto{\pgfqpoint{2.351600in}{1.680000in}}%
\pgfpathlineto{\pgfqpoint{2.352840in}{1.680000in}}%
\pgfpathlineto{\pgfqpoint{2.355320in}{2.030000in}}%
\pgfpathlineto{\pgfqpoint{2.356560in}{2.100000in}}%
\pgfpathlineto{\pgfqpoint{2.357800in}{2.100000in}}%
\pgfpathlineto{\pgfqpoint{2.359040in}{2.135000in}}%
\pgfpathlineto{\pgfqpoint{2.360280in}{1.855000in}}%
\pgfpathlineto{\pgfqpoint{2.361520in}{2.030000in}}%
\pgfpathlineto{\pgfqpoint{2.362760in}{1.995000in}}%
\pgfpathlineto{\pgfqpoint{2.364000in}{1.855000in}}%
\pgfpathlineto{\pgfqpoint{2.365240in}{2.450000in}}%
\pgfpathlineto{\pgfqpoint{2.366480in}{1.855000in}}%
\pgfpathlineto{\pgfqpoint{2.367720in}{2.170000in}}%
\pgfpathlineto{\pgfqpoint{2.370200in}{1.890000in}}%
\pgfpathlineto{\pgfqpoint{2.371440in}{1.715000in}}%
\pgfpathlineto{\pgfqpoint{2.372680in}{1.855000in}}%
\pgfpathlineto{\pgfqpoint{2.373920in}{1.855000in}}%
\pgfpathlineto{\pgfqpoint{2.375160in}{1.645000in}}%
\pgfpathlineto{\pgfqpoint{2.376400in}{1.925000in}}%
\pgfpathlineto{\pgfqpoint{2.377640in}{1.925000in}}%
\pgfpathlineto{\pgfqpoint{2.378880in}{1.715000in}}%
\pgfpathlineto{\pgfqpoint{2.380120in}{1.960000in}}%
\pgfpathlineto{\pgfqpoint{2.381360in}{1.645000in}}%
\pgfpathlineto{\pgfqpoint{2.383840in}{1.960000in}}%
\pgfpathlineto{\pgfqpoint{2.385080in}{2.240000in}}%
\pgfpathlineto{\pgfqpoint{2.386320in}{2.205000in}}%
\pgfpathlineto{\pgfqpoint{2.388800in}{1.820000in}}%
\pgfpathlineto{\pgfqpoint{2.390040in}{1.995000in}}%
\pgfpathlineto{\pgfqpoint{2.391280in}{1.925000in}}%
\pgfpathlineto{\pgfqpoint{2.392520in}{2.135000in}}%
\pgfpathlineto{\pgfqpoint{2.393760in}{1.750000in}}%
\pgfpathlineto{\pgfqpoint{2.395000in}{1.855000in}}%
\pgfpathlineto{\pgfqpoint{2.396240in}{1.855000in}}%
\pgfpathlineto{\pgfqpoint{2.397480in}{1.715000in}}%
\pgfpathlineto{\pgfqpoint{2.398720in}{1.890000in}}%
\pgfpathlineto{\pgfqpoint{2.399960in}{1.855000in}}%
\pgfpathlineto{\pgfqpoint{2.401200in}{2.100000in}}%
\pgfpathlineto{\pgfqpoint{2.402440in}{1.610000in}}%
\pgfpathlineto{\pgfqpoint{2.403680in}{1.960000in}}%
\pgfpathlineto{\pgfqpoint{2.404920in}{1.995000in}}%
\pgfpathlineto{\pgfqpoint{2.406160in}{2.135000in}}%
\pgfpathlineto{\pgfqpoint{2.408640in}{1.540000in}}%
\pgfpathlineto{\pgfqpoint{2.409880in}{1.715000in}}%
\pgfpathlineto{\pgfqpoint{2.411120in}{1.645000in}}%
\pgfpathlineto{\pgfqpoint{2.412360in}{1.785000in}}%
\pgfpathlineto{\pgfqpoint{2.413600in}{2.275000in}}%
\pgfpathlineto{\pgfqpoint{2.416080in}{1.960000in}}%
\pgfpathlineto{\pgfqpoint{2.417320in}{1.890000in}}%
\pgfpathlineto{\pgfqpoint{2.418560in}{1.995000in}}%
\pgfpathlineto{\pgfqpoint{2.419800in}{1.785000in}}%
\pgfpathlineto{\pgfqpoint{2.421040in}{1.960000in}}%
\pgfpathlineto{\pgfqpoint{2.422280in}{1.820000in}}%
\pgfpathlineto{\pgfqpoint{2.423520in}{1.995000in}}%
\pgfpathlineto{\pgfqpoint{2.424760in}{1.960000in}}%
\pgfpathlineto{\pgfqpoint{2.426000in}{1.750000in}}%
\pgfpathlineto{\pgfqpoint{2.427240in}{1.925000in}}%
\pgfpathlineto{\pgfqpoint{2.428480in}{1.890000in}}%
\pgfpathlineto{\pgfqpoint{2.430960in}{2.205000in}}%
\pgfpathlineto{\pgfqpoint{2.433440in}{1.610000in}}%
\pgfpathlineto{\pgfqpoint{2.435920in}{1.995000in}}%
\pgfpathlineto{\pgfqpoint{2.437160in}{2.030000in}}%
\pgfpathlineto{\pgfqpoint{2.438400in}{1.925000in}}%
\pgfpathlineto{\pgfqpoint{2.439640in}{1.995000in}}%
\pgfpathlineto{\pgfqpoint{2.440880in}{1.855000in}}%
\pgfpathlineto{\pgfqpoint{2.444600in}{1.960000in}}%
\pgfpathlineto{\pgfqpoint{2.445840in}{1.890000in}}%
\pgfpathlineto{\pgfqpoint{2.447080in}{1.680000in}}%
\pgfpathlineto{\pgfqpoint{2.448320in}{2.100000in}}%
\pgfpathlineto{\pgfqpoint{2.449560in}{1.785000in}}%
\pgfpathlineto{\pgfqpoint{2.450800in}{2.170000in}}%
\pgfpathlineto{\pgfqpoint{2.452040in}{1.785000in}}%
\pgfpathlineto{\pgfqpoint{2.453280in}{1.925000in}}%
\pgfpathlineto{\pgfqpoint{2.454520in}{1.610000in}}%
\pgfpathlineto{\pgfqpoint{2.457000in}{1.715000in}}%
\pgfpathlineto{\pgfqpoint{2.458240in}{1.295000in}}%
\pgfpathlineto{\pgfqpoint{2.459480in}{2.030000in}}%
\pgfpathlineto{\pgfqpoint{2.460720in}{1.575000in}}%
\pgfpathlineto{\pgfqpoint{2.461960in}{1.890000in}}%
\pgfpathlineto{\pgfqpoint{2.463200in}{1.645000in}}%
\pgfpathlineto{\pgfqpoint{2.464440in}{1.820000in}}%
\pgfpathlineto{\pgfqpoint{2.465680in}{1.715000in}}%
\pgfpathlineto{\pgfqpoint{2.466920in}{1.715000in}}%
\pgfpathlineto{\pgfqpoint{2.468160in}{1.680000in}}%
\pgfpathlineto{\pgfqpoint{2.469400in}{1.505000in}}%
\pgfpathlineto{\pgfqpoint{2.470640in}{1.540000in}}%
\pgfpathlineto{\pgfqpoint{2.473120in}{2.415000in}}%
\pgfpathlineto{\pgfqpoint{2.474360in}{1.610000in}}%
\pgfpathlineto{\pgfqpoint{2.475600in}{2.065000in}}%
\pgfpathlineto{\pgfqpoint{2.476840in}{2.030000in}}%
\pgfpathlineto{\pgfqpoint{2.478080in}{1.715000in}}%
\pgfpathlineto{\pgfqpoint{2.479320in}{1.750000in}}%
\pgfpathlineto{\pgfqpoint{2.480560in}{2.240000in}}%
\pgfpathlineto{\pgfqpoint{2.485520in}{1.750000in}}%
\pgfpathlineto{\pgfqpoint{2.486760in}{2.345000in}}%
\pgfpathlineto{\pgfqpoint{2.489240in}{1.645000in}}%
\pgfpathlineto{\pgfqpoint{2.492960in}{2.415000in}}%
\pgfpathlineto{\pgfqpoint{2.494200in}{1.785000in}}%
\pgfpathlineto{\pgfqpoint{2.495440in}{2.030000in}}%
\pgfpathlineto{\pgfqpoint{2.496680in}{1.890000in}}%
\pgfpathlineto{\pgfqpoint{2.497920in}{1.575000in}}%
\pgfpathlineto{\pgfqpoint{2.499160in}{1.855000in}}%
\pgfpathlineto{\pgfqpoint{2.500400in}{1.820000in}}%
\pgfpathlineto{\pgfqpoint{2.501640in}{1.855000in}}%
\pgfpathlineto{\pgfqpoint{2.502880in}{1.960000in}}%
\pgfpathlineto{\pgfqpoint{2.505360in}{1.190000in}}%
\pgfpathlineto{\pgfqpoint{2.506600in}{1.540000in}}%
\pgfpathlineto{\pgfqpoint{2.507840in}{1.505000in}}%
\pgfpathlineto{\pgfqpoint{2.510320in}{2.065000in}}%
\pgfpathlineto{\pgfqpoint{2.511560in}{1.855000in}}%
\pgfpathlineto{\pgfqpoint{2.512800in}{1.890000in}}%
\pgfpathlineto{\pgfqpoint{2.514040in}{2.100000in}}%
\pgfpathlineto{\pgfqpoint{2.516520in}{1.820000in}}%
\pgfpathlineto{\pgfqpoint{2.517760in}{2.135000in}}%
\pgfpathlineto{\pgfqpoint{2.520240in}{1.470000in}}%
\pgfpathlineto{\pgfqpoint{2.521480in}{1.470000in}}%
\pgfpathlineto{\pgfqpoint{2.522720in}{1.785000in}}%
\pgfpathlineto{\pgfqpoint{2.523960in}{1.715000in}}%
\pgfpathlineto{\pgfqpoint{2.525200in}{1.470000in}}%
\pgfpathlineto{\pgfqpoint{2.527680in}{2.100000in}}%
\pgfpathlineto{\pgfqpoint{2.528920in}{1.785000in}}%
\pgfpathlineto{\pgfqpoint{2.530160in}{2.240000in}}%
\pgfpathlineto{\pgfqpoint{2.531400in}{1.890000in}}%
\pgfpathlineto{\pgfqpoint{2.532640in}{1.890000in}}%
\pgfpathlineto{\pgfqpoint{2.533880in}{1.925000in}}%
\pgfpathlineto{\pgfqpoint{2.535120in}{1.715000in}}%
\pgfpathlineto{\pgfqpoint{2.536360in}{1.785000in}}%
\pgfpathlineto{\pgfqpoint{2.537600in}{2.240000in}}%
\pgfpathlineto{\pgfqpoint{2.538840in}{1.855000in}}%
\pgfpathlineto{\pgfqpoint{2.540080in}{2.100000in}}%
\pgfpathlineto{\pgfqpoint{2.543800in}{1.855000in}}%
\pgfpathlineto{\pgfqpoint{2.545040in}{1.820000in}}%
\pgfpathlineto{\pgfqpoint{2.546280in}{1.680000in}}%
\pgfpathlineto{\pgfqpoint{2.548760in}{2.065000in}}%
\pgfpathlineto{\pgfqpoint{2.551240in}{1.715000in}}%
\pgfpathlineto{\pgfqpoint{2.552480in}{1.750000in}}%
\pgfpathlineto{\pgfqpoint{2.553720in}{2.135000in}}%
\pgfpathlineto{\pgfqpoint{2.554960in}{1.890000in}}%
\pgfpathlineto{\pgfqpoint{2.556200in}{2.065000in}}%
\pgfpathlineto{\pgfqpoint{2.557440in}{1.785000in}}%
\pgfpathlineto{\pgfqpoint{2.558680in}{1.960000in}}%
\pgfpathlineto{\pgfqpoint{2.559920in}{1.610000in}}%
\pgfpathlineto{\pgfqpoint{2.562400in}{1.890000in}}%
\pgfpathlineto{\pgfqpoint{2.563640in}{2.030000in}}%
\pgfpathlineto{\pgfqpoint{2.564880in}{1.680000in}}%
\pgfpathlineto{\pgfqpoint{2.566120in}{1.750000in}}%
\pgfpathlineto{\pgfqpoint{2.567360in}{1.680000in}}%
\pgfpathlineto{\pgfqpoint{2.568600in}{1.680000in}}%
\pgfpathlineto{\pgfqpoint{2.569840in}{1.715000in}}%
\pgfpathlineto{\pgfqpoint{2.571080in}{1.890000in}}%
\pgfpathlineto{\pgfqpoint{2.572320in}{1.785000in}}%
\pgfpathlineto{\pgfqpoint{2.574800in}{2.170000in}}%
\pgfpathlineto{\pgfqpoint{2.576040in}{1.820000in}}%
\pgfpathlineto{\pgfqpoint{2.578520in}{2.100000in}}%
\pgfpathlineto{\pgfqpoint{2.579760in}{1.575000in}}%
\pgfpathlineto{\pgfqpoint{2.581000in}{1.855000in}}%
\pgfpathlineto{\pgfqpoint{2.582240in}{1.785000in}}%
\pgfpathlineto{\pgfqpoint{2.583480in}{1.610000in}}%
\pgfpathlineto{\pgfqpoint{2.585960in}{2.100000in}}%
\pgfpathlineto{\pgfqpoint{2.587200in}{1.540000in}}%
\pgfpathlineto{\pgfqpoint{2.590920in}{2.240000in}}%
\pgfpathlineto{\pgfqpoint{2.592160in}{2.310000in}}%
\pgfpathlineto{\pgfqpoint{2.593400in}{1.820000in}}%
\pgfpathlineto{\pgfqpoint{2.594640in}{2.100000in}}%
\pgfpathlineto{\pgfqpoint{2.597120in}{2.030000in}}%
\pgfpathlineto{\pgfqpoint{2.598360in}{1.960000in}}%
\pgfpathlineto{\pgfqpoint{2.599600in}{1.470000in}}%
\pgfpathlineto{\pgfqpoint{2.600840in}{1.925000in}}%
\pgfpathlineto{\pgfqpoint{2.602080in}{1.890000in}}%
\pgfpathlineto{\pgfqpoint{2.604560in}{2.240000in}}%
\pgfpathlineto{\pgfqpoint{2.605800in}{2.205000in}}%
\pgfpathlineto{\pgfqpoint{2.608280in}{1.995000in}}%
\pgfpathlineto{\pgfqpoint{2.609520in}{2.240000in}}%
\pgfpathlineto{\pgfqpoint{2.610760in}{2.100000in}}%
\pgfpathlineto{\pgfqpoint{2.613240in}{1.785000in}}%
\pgfpathlineto{\pgfqpoint{2.614480in}{2.065000in}}%
\pgfpathlineto{\pgfqpoint{2.615720in}{1.785000in}}%
\pgfpathlineto{\pgfqpoint{2.616960in}{1.785000in}}%
\pgfpathlineto{\pgfqpoint{2.618200in}{1.540000in}}%
\pgfpathlineto{\pgfqpoint{2.620680in}{1.960000in}}%
\pgfpathlineto{\pgfqpoint{2.621920in}{2.065000in}}%
\pgfpathlineto{\pgfqpoint{2.623160in}{2.030000in}}%
\pgfpathlineto{\pgfqpoint{2.624400in}{1.400000in}}%
\pgfpathlineto{\pgfqpoint{2.625640in}{1.645000in}}%
\pgfpathlineto{\pgfqpoint{2.626880in}{2.275000in}}%
\pgfpathlineto{\pgfqpoint{2.629360in}{1.610000in}}%
\pgfpathlineto{\pgfqpoint{2.630600in}{1.435000in}}%
\pgfpathlineto{\pgfqpoint{2.631840in}{2.205000in}}%
\pgfpathlineto{\pgfqpoint{2.633080in}{1.785000in}}%
\pgfpathlineto{\pgfqpoint{2.634320in}{2.135000in}}%
\pgfpathlineto{\pgfqpoint{2.635560in}{1.785000in}}%
\pgfpathlineto{\pgfqpoint{2.636800in}{2.030000in}}%
\pgfpathlineto{\pgfqpoint{2.638040in}{1.715000in}}%
\pgfpathlineto{\pgfqpoint{2.639280in}{1.925000in}}%
\pgfpathlineto{\pgfqpoint{2.641760in}{1.645000in}}%
\pgfpathlineto{\pgfqpoint{2.643000in}{1.995000in}}%
\pgfpathlineto{\pgfqpoint{2.644240in}{1.960000in}}%
\pgfpathlineto{\pgfqpoint{2.645480in}{2.135000in}}%
\pgfpathlineto{\pgfqpoint{2.646720in}{1.750000in}}%
\pgfpathlineto{\pgfqpoint{2.647960in}{1.715000in}}%
\pgfpathlineto{\pgfqpoint{2.649200in}{2.030000in}}%
\pgfpathlineto{\pgfqpoint{2.651680in}{1.540000in}}%
\pgfpathlineto{\pgfqpoint{2.652920in}{1.855000in}}%
\pgfpathlineto{\pgfqpoint{2.654160in}{1.435000in}}%
\pgfpathlineto{\pgfqpoint{2.655400in}{2.065000in}}%
\pgfpathlineto{\pgfqpoint{2.656640in}{2.065000in}}%
\pgfpathlineto{\pgfqpoint{2.657880in}{1.855000in}}%
\pgfpathlineto{\pgfqpoint{2.659120in}{1.960000in}}%
\pgfpathlineto{\pgfqpoint{2.661600in}{1.470000in}}%
\pgfpathlineto{\pgfqpoint{2.662840in}{1.610000in}}%
\pgfpathlineto{\pgfqpoint{2.664080in}{1.925000in}}%
\pgfpathlineto{\pgfqpoint{2.665320in}{1.785000in}}%
\pgfpathlineto{\pgfqpoint{2.666560in}{1.855000in}}%
\pgfpathlineto{\pgfqpoint{2.667800in}{2.100000in}}%
\pgfpathlineto{\pgfqpoint{2.669040in}{2.065000in}}%
\pgfpathlineto{\pgfqpoint{2.670280in}{1.960000in}}%
\pgfpathlineto{\pgfqpoint{2.671520in}{1.435000in}}%
\pgfpathlineto{\pgfqpoint{2.674000in}{2.100000in}}%
\pgfpathlineto{\pgfqpoint{2.675240in}{1.575000in}}%
\pgfpathlineto{\pgfqpoint{2.677720in}{2.170000in}}%
\pgfpathlineto{\pgfqpoint{2.678960in}{2.065000in}}%
\pgfpathlineto{\pgfqpoint{2.681440in}{1.610000in}}%
\pgfpathlineto{\pgfqpoint{2.683920in}{1.995000in}}%
\pgfpathlineto{\pgfqpoint{2.685160in}{1.925000in}}%
\pgfpathlineto{\pgfqpoint{2.686400in}{1.785000in}}%
\pgfpathlineto{\pgfqpoint{2.687640in}{1.785000in}}%
\pgfpathlineto{\pgfqpoint{2.688880in}{2.170000in}}%
\pgfpathlineto{\pgfqpoint{2.691360in}{2.065000in}}%
\pgfpathlineto{\pgfqpoint{2.692600in}{1.785000in}}%
\pgfpathlineto{\pgfqpoint{2.693840in}{2.170000in}}%
\pgfpathlineto{\pgfqpoint{2.696320in}{1.470000in}}%
\pgfpathlineto{\pgfqpoint{2.697560in}{1.890000in}}%
\pgfpathlineto{\pgfqpoint{2.698800in}{1.890000in}}%
\pgfpathlineto{\pgfqpoint{2.700040in}{1.680000in}}%
\pgfpathlineto{\pgfqpoint{2.701280in}{2.380000in}}%
\pgfpathlineto{\pgfqpoint{2.703760in}{1.855000in}}%
\pgfpathlineto{\pgfqpoint{2.705000in}{2.345000in}}%
\pgfpathlineto{\pgfqpoint{2.706240in}{1.610000in}}%
\pgfpathlineto{\pgfqpoint{2.707480in}{2.135000in}}%
\pgfpathlineto{\pgfqpoint{2.709960in}{1.785000in}}%
\pgfpathlineto{\pgfqpoint{2.711200in}{1.855000in}}%
\pgfpathlineto{\pgfqpoint{2.712440in}{1.995000in}}%
\pgfpathlineto{\pgfqpoint{2.713680in}{1.750000in}}%
\pgfpathlineto{\pgfqpoint{2.714920in}{1.925000in}}%
\pgfpathlineto{\pgfqpoint{2.717400in}{1.785000in}}%
\pgfpathlineto{\pgfqpoint{2.718640in}{2.310000in}}%
\pgfpathlineto{\pgfqpoint{2.722360in}{1.715000in}}%
\pgfpathlineto{\pgfqpoint{2.723600in}{1.680000in}}%
\pgfpathlineto{\pgfqpoint{2.726080in}{1.470000in}}%
\pgfpathlineto{\pgfqpoint{2.727320in}{2.065000in}}%
\pgfpathlineto{\pgfqpoint{2.728560in}{1.575000in}}%
\pgfpathlineto{\pgfqpoint{2.729800in}{1.680000in}}%
\pgfpathlineto{\pgfqpoint{2.732280in}{2.065000in}}%
\pgfpathlineto{\pgfqpoint{2.733520in}{1.715000in}}%
\pgfpathlineto{\pgfqpoint{2.734760in}{1.890000in}}%
\pgfpathlineto{\pgfqpoint{2.736000in}{1.820000in}}%
\pgfpathlineto{\pgfqpoint{2.737240in}{1.820000in}}%
\pgfpathlineto{\pgfqpoint{2.738480in}{1.645000in}}%
\pgfpathlineto{\pgfqpoint{2.740960in}{2.135000in}}%
\pgfpathlineto{\pgfqpoint{2.742200in}{1.820000in}}%
\pgfpathlineto{\pgfqpoint{2.743440in}{1.855000in}}%
\pgfpathlineto{\pgfqpoint{2.745920in}{1.575000in}}%
\pgfpathlineto{\pgfqpoint{2.747160in}{1.960000in}}%
\pgfpathlineto{\pgfqpoint{2.748400in}{1.960000in}}%
\pgfpathlineto{\pgfqpoint{2.752120in}{1.610000in}}%
\pgfpathlineto{\pgfqpoint{2.753360in}{1.645000in}}%
\pgfpathlineto{\pgfqpoint{2.754600in}{2.065000in}}%
\pgfpathlineto{\pgfqpoint{2.755840in}{2.065000in}}%
\pgfpathlineto{\pgfqpoint{2.758320in}{1.785000in}}%
\pgfpathlineto{\pgfqpoint{2.759560in}{2.205000in}}%
\pgfpathlineto{\pgfqpoint{2.760800in}{1.960000in}}%
\pgfpathlineto{\pgfqpoint{2.762040in}{2.310000in}}%
\pgfpathlineto{\pgfqpoint{2.764520in}{1.995000in}}%
\pgfpathlineto{\pgfqpoint{2.765760in}{1.680000in}}%
\pgfpathlineto{\pgfqpoint{2.769480in}{2.345000in}}%
\pgfpathlineto{\pgfqpoint{2.771960in}{1.925000in}}%
\pgfpathlineto{\pgfqpoint{2.773200in}{1.995000in}}%
\pgfpathlineto{\pgfqpoint{2.774440in}{2.170000in}}%
\pgfpathlineto{\pgfqpoint{2.775680in}{2.030000in}}%
\pgfpathlineto{\pgfqpoint{2.776920in}{2.030000in}}%
\pgfpathlineto{\pgfqpoint{2.779400in}{1.785000in}}%
\pgfpathlineto{\pgfqpoint{2.780640in}{1.995000in}}%
\pgfpathlineto{\pgfqpoint{2.781880in}{1.645000in}}%
\pgfpathlineto{\pgfqpoint{2.784360in}{1.890000in}}%
\pgfpathlineto{\pgfqpoint{2.785600in}{1.925000in}}%
\pgfpathlineto{\pgfqpoint{2.786840in}{2.310000in}}%
\pgfpathlineto{\pgfqpoint{2.788080in}{1.610000in}}%
\pgfpathlineto{\pgfqpoint{2.790560in}{2.485000in}}%
\pgfpathlineto{\pgfqpoint{2.791800in}{1.960000in}}%
\pgfpathlineto{\pgfqpoint{2.793040in}{2.135000in}}%
\pgfpathlineto{\pgfqpoint{2.794280in}{1.855000in}}%
\pgfpathlineto{\pgfqpoint{2.795520in}{2.170000in}}%
\pgfpathlineto{\pgfqpoint{2.798000in}{1.890000in}}%
\pgfpathlineto{\pgfqpoint{2.799240in}{2.275000in}}%
\pgfpathlineto{\pgfqpoint{2.801720in}{1.785000in}}%
\pgfpathlineto{\pgfqpoint{2.802960in}{2.135000in}}%
\pgfpathlineto{\pgfqpoint{2.804200in}{1.645000in}}%
\pgfpathlineto{\pgfqpoint{2.805440in}{1.750000in}}%
\pgfpathlineto{\pgfqpoint{2.806680in}{2.030000in}}%
\pgfpathlineto{\pgfqpoint{2.807920in}{1.785000in}}%
\pgfpathlineto{\pgfqpoint{2.809160in}{1.890000in}}%
\pgfpathlineto{\pgfqpoint{2.810400in}{1.785000in}}%
\pgfpathlineto{\pgfqpoint{2.811640in}{2.205000in}}%
\pgfpathlineto{\pgfqpoint{2.812880in}{2.100000in}}%
\pgfpathlineto{\pgfqpoint{2.814120in}{2.135000in}}%
\pgfpathlineto{\pgfqpoint{2.815360in}{2.275000in}}%
\pgfpathlineto{\pgfqpoint{2.816600in}{1.960000in}}%
\pgfpathlineto{\pgfqpoint{2.820320in}{2.170000in}}%
\pgfpathlineto{\pgfqpoint{2.821560in}{1.750000in}}%
\pgfpathlineto{\pgfqpoint{2.824040in}{2.170000in}}%
\pgfpathlineto{\pgfqpoint{2.826520in}{1.820000in}}%
\pgfpathlineto{\pgfqpoint{2.827760in}{2.240000in}}%
\pgfpathlineto{\pgfqpoint{2.829000in}{1.960000in}}%
\pgfpathlineto{\pgfqpoint{2.830240in}{2.345000in}}%
\pgfpathlineto{\pgfqpoint{2.831480in}{1.960000in}}%
\pgfpathlineto{\pgfqpoint{2.833960in}{2.030000in}}%
\pgfpathlineto{\pgfqpoint{2.835200in}{2.030000in}}%
\pgfpathlineto{\pgfqpoint{2.836440in}{2.275000in}}%
\pgfpathlineto{\pgfqpoint{2.837680in}{1.855000in}}%
\pgfpathlineto{\pgfqpoint{2.838920in}{2.065000in}}%
\pgfpathlineto{\pgfqpoint{2.840160in}{1.645000in}}%
\pgfpathlineto{\pgfqpoint{2.842640in}{1.960000in}}%
\pgfpathlineto{\pgfqpoint{2.843880in}{1.785000in}}%
\pgfpathlineto{\pgfqpoint{2.845120in}{1.925000in}}%
\pgfpathlineto{\pgfqpoint{2.846360in}{1.645000in}}%
\pgfpathlineto{\pgfqpoint{2.847600in}{1.960000in}}%
\pgfpathlineto{\pgfqpoint{2.848840in}{1.155000in}}%
\pgfpathlineto{\pgfqpoint{2.850080in}{2.310000in}}%
\pgfpathlineto{\pgfqpoint{2.851320in}{1.680000in}}%
\pgfpathlineto{\pgfqpoint{2.852560in}{2.170000in}}%
\pgfpathlineto{\pgfqpoint{2.853800in}{1.715000in}}%
\pgfpathlineto{\pgfqpoint{2.855040in}{1.960000in}}%
\pgfpathlineto{\pgfqpoint{2.856280in}{2.450000in}}%
\pgfpathlineto{\pgfqpoint{2.857520in}{2.485000in}}%
\pgfpathlineto{\pgfqpoint{2.858760in}{1.575000in}}%
\pgfpathlineto{\pgfqpoint{2.860000in}{1.680000in}}%
\pgfpathlineto{\pgfqpoint{2.861240in}{1.330000in}}%
\pgfpathlineto{\pgfqpoint{2.863720in}{1.890000in}}%
\pgfpathlineto{\pgfqpoint{2.866200in}{1.645000in}}%
\pgfpathlineto{\pgfqpoint{2.867440in}{1.680000in}}%
\pgfpathlineto{\pgfqpoint{2.868680in}{1.960000in}}%
\pgfpathlineto{\pgfqpoint{2.869920in}{1.540000in}}%
\pgfpathlineto{\pgfqpoint{2.871160in}{1.610000in}}%
\pgfpathlineto{\pgfqpoint{2.872400in}{1.890000in}}%
\pgfpathlineto{\pgfqpoint{2.873640in}{1.680000in}}%
\pgfpathlineto{\pgfqpoint{2.874880in}{1.820000in}}%
\pgfpathlineto{\pgfqpoint{2.876120in}{2.240000in}}%
\pgfpathlineto{\pgfqpoint{2.877360in}{1.995000in}}%
\pgfpathlineto{\pgfqpoint{2.878600in}{1.995000in}}%
\pgfpathlineto{\pgfqpoint{2.879840in}{1.960000in}}%
\pgfpathlineto{\pgfqpoint{2.881080in}{1.960000in}}%
\pgfpathlineto{\pgfqpoint{2.882320in}{2.030000in}}%
\pgfpathlineto{\pgfqpoint{2.883560in}{1.995000in}}%
\pgfpathlineto{\pgfqpoint{2.884800in}{1.820000in}}%
\pgfpathlineto{\pgfqpoint{2.886040in}{1.995000in}}%
\pgfpathlineto{\pgfqpoint{2.888520in}{1.750000in}}%
\pgfpathlineto{\pgfqpoint{2.891000in}{1.820000in}}%
\pgfpathlineto{\pgfqpoint{2.892240in}{1.680000in}}%
\pgfpathlineto{\pgfqpoint{2.893480in}{2.310000in}}%
\pgfpathlineto{\pgfqpoint{2.894720in}{1.295000in}}%
\pgfpathlineto{\pgfqpoint{2.897200in}{1.820000in}}%
\pgfpathlineto{\pgfqpoint{2.898440in}{1.715000in}}%
\pgfpathlineto{\pgfqpoint{2.899680in}{1.715000in}}%
\pgfpathlineto{\pgfqpoint{2.900920in}{1.645000in}}%
\pgfpathlineto{\pgfqpoint{2.902160in}{2.100000in}}%
\pgfpathlineto{\pgfqpoint{2.903400in}{1.820000in}}%
\pgfpathlineto{\pgfqpoint{2.904640in}{1.820000in}}%
\pgfpathlineto{\pgfqpoint{2.905880in}{1.890000in}}%
\pgfpathlineto{\pgfqpoint{2.907120in}{2.100000in}}%
\pgfpathlineto{\pgfqpoint{2.908360in}{1.575000in}}%
\pgfpathlineto{\pgfqpoint{2.909600in}{2.135000in}}%
\pgfpathlineto{\pgfqpoint{2.910840in}{2.065000in}}%
\pgfpathlineto{\pgfqpoint{2.912080in}{1.750000in}}%
\pgfpathlineto{\pgfqpoint{2.913320in}{2.170000in}}%
\pgfpathlineto{\pgfqpoint{2.914560in}{2.135000in}}%
\pgfpathlineto{\pgfqpoint{2.917040in}{1.785000in}}%
\pgfpathlineto{\pgfqpoint{2.918280in}{1.960000in}}%
\pgfpathlineto{\pgfqpoint{2.919520in}{1.680000in}}%
\pgfpathlineto{\pgfqpoint{2.920760in}{2.030000in}}%
\pgfpathlineto{\pgfqpoint{2.922000in}{1.575000in}}%
\pgfpathlineto{\pgfqpoint{2.923240in}{2.030000in}}%
\pgfpathlineto{\pgfqpoint{2.924480in}{1.855000in}}%
\pgfpathlineto{\pgfqpoint{2.925720in}{2.205000in}}%
\pgfpathlineto{\pgfqpoint{2.928200in}{1.750000in}}%
\pgfpathlineto{\pgfqpoint{2.931920in}{2.415000in}}%
\pgfpathlineto{\pgfqpoint{2.933160in}{1.925000in}}%
\pgfpathlineto{\pgfqpoint{2.934400in}{2.065000in}}%
\pgfpathlineto{\pgfqpoint{2.935640in}{2.030000in}}%
\pgfpathlineto{\pgfqpoint{2.936880in}{1.960000in}}%
\pgfpathlineto{\pgfqpoint{2.938120in}{1.715000in}}%
\pgfpathlineto{\pgfqpoint{2.940600in}{1.890000in}}%
\pgfpathlineto{\pgfqpoint{2.941840in}{1.890000in}}%
\pgfpathlineto{\pgfqpoint{2.943080in}{1.540000in}}%
\pgfpathlineto{\pgfqpoint{2.944320in}{1.960000in}}%
\pgfpathlineto{\pgfqpoint{2.945560in}{1.890000in}}%
\pgfpathlineto{\pgfqpoint{2.946800in}{1.995000in}}%
\pgfpathlineto{\pgfqpoint{2.948040in}{1.540000in}}%
\pgfpathlineto{\pgfqpoint{2.949280in}{1.890000in}}%
\pgfpathlineto{\pgfqpoint{2.950520in}{1.505000in}}%
\pgfpathlineto{\pgfqpoint{2.951760in}{1.820000in}}%
\pgfpathlineto{\pgfqpoint{2.953000in}{1.470000in}}%
\pgfpathlineto{\pgfqpoint{2.954240in}{1.505000in}}%
\pgfpathlineto{\pgfqpoint{2.956720in}{1.995000in}}%
\pgfpathlineto{\pgfqpoint{2.959200in}{1.610000in}}%
\pgfpathlineto{\pgfqpoint{2.960440in}{1.855000in}}%
\pgfpathlineto{\pgfqpoint{2.961680in}{1.820000in}}%
\pgfpathlineto{\pgfqpoint{2.962920in}{1.365000in}}%
\pgfpathlineto{\pgfqpoint{2.965400in}{2.065000in}}%
\pgfpathlineto{\pgfqpoint{2.966640in}{1.260000in}}%
\pgfpathlineto{\pgfqpoint{2.967880in}{2.030000in}}%
\pgfpathlineto{\pgfqpoint{2.970360in}{1.855000in}}%
\pgfpathlineto{\pgfqpoint{2.971600in}{2.030000in}}%
\pgfpathlineto{\pgfqpoint{2.972840in}{1.575000in}}%
\pgfpathlineto{\pgfqpoint{2.974080in}{1.645000in}}%
\pgfpathlineto{\pgfqpoint{2.975320in}{2.030000in}}%
\pgfpathlineto{\pgfqpoint{2.976560in}{2.065000in}}%
\pgfpathlineto{\pgfqpoint{2.977800in}{1.995000in}}%
\pgfpathlineto{\pgfqpoint{2.979040in}{1.750000in}}%
\pgfpathlineto{\pgfqpoint{2.980280in}{2.065000in}}%
\pgfpathlineto{\pgfqpoint{2.982760in}{1.610000in}}%
\pgfpathlineto{\pgfqpoint{2.984000in}{1.890000in}}%
\pgfpathlineto{\pgfqpoint{2.985240in}{1.820000in}}%
\pgfpathlineto{\pgfqpoint{2.986480in}{2.135000in}}%
\pgfpathlineto{\pgfqpoint{2.987720in}{2.135000in}}%
\pgfpathlineto{\pgfqpoint{2.988960in}{1.855000in}}%
\pgfpathlineto{\pgfqpoint{2.990200in}{2.310000in}}%
\pgfpathlineto{\pgfqpoint{2.992680in}{1.540000in}}%
\pgfpathlineto{\pgfqpoint{2.993920in}{2.450000in}}%
\pgfpathlineto{\pgfqpoint{2.996400in}{1.540000in}}%
\pgfpathlineto{\pgfqpoint{3.001360in}{1.995000in}}%
\pgfpathlineto{\pgfqpoint{3.002600in}{1.925000in}}%
\pgfpathlineto{\pgfqpoint{3.003840in}{2.030000in}}%
\pgfpathlineto{\pgfqpoint{3.005080in}{1.645000in}}%
\pgfpathlineto{\pgfqpoint{3.007560in}{2.065000in}}%
\pgfpathlineto{\pgfqpoint{3.010040in}{1.750000in}}%
\pgfpathlineto{\pgfqpoint{3.012520in}{2.310000in}}%
\pgfpathlineto{\pgfqpoint{3.013760in}{1.960000in}}%
\pgfpathlineto{\pgfqpoint{3.015000in}{1.925000in}}%
\pgfpathlineto{\pgfqpoint{3.016240in}{1.680000in}}%
\pgfpathlineto{\pgfqpoint{3.017480in}{2.065000in}}%
\pgfpathlineto{\pgfqpoint{3.018720in}{1.785000in}}%
\pgfpathlineto{\pgfqpoint{3.021200in}{2.100000in}}%
\pgfpathlineto{\pgfqpoint{3.023680in}{1.575000in}}%
\pgfpathlineto{\pgfqpoint{3.024920in}{2.135000in}}%
\pgfpathlineto{\pgfqpoint{3.026160in}{2.135000in}}%
\pgfpathlineto{\pgfqpoint{3.027400in}{1.820000in}}%
\pgfpathlineto{\pgfqpoint{3.028640in}{2.170000in}}%
\pgfpathlineto{\pgfqpoint{3.029880in}{2.135000in}}%
\pgfpathlineto{\pgfqpoint{3.031120in}{2.380000in}}%
\pgfpathlineto{\pgfqpoint{3.032360in}{1.575000in}}%
\pgfpathlineto{\pgfqpoint{3.033600in}{2.065000in}}%
\pgfpathlineto{\pgfqpoint{3.034840in}{2.065000in}}%
\pgfpathlineto{\pgfqpoint{3.036080in}{2.485000in}}%
\pgfpathlineto{\pgfqpoint{3.037320in}{1.540000in}}%
\pgfpathlineto{\pgfqpoint{3.039800in}{1.715000in}}%
\pgfpathlineto{\pgfqpoint{3.041040in}{1.680000in}}%
\pgfpathlineto{\pgfqpoint{3.042280in}{1.680000in}}%
\pgfpathlineto{\pgfqpoint{3.043520in}{2.100000in}}%
\pgfpathlineto{\pgfqpoint{3.044760in}{2.030000in}}%
\pgfpathlineto{\pgfqpoint{3.046000in}{2.205000in}}%
\pgfpathlineto{\pgfqpoint{3.047240in}{1.785000in}}%
\pgfpathlineto{\pgfqpoint{3.049720in}{2.205000in}}%
\pgfpathlineto{\pgfqpoint{3.050960in}{1.855000in}}%
\pgfpathlineto{\pgfqpoint{3.052200in}{2.065000in}}%
\pgfpathlineto{\pgfqpoint{3.053440in}{1.890000in}}%
\pgfpathlineto{\pgfqpoint{3.054680in}{2.100000in}}%
\pgfpathlineto{\pgfqpoint{3.057160in}{1.330000in}}%
\pgfpathlineto{\pgfqpoint{3.058400in}{1.995000in}}%
\pgfpathlineto{\pgfqpoint{3.059640in}{2.030000in}}%
\pgfpathlineto{\pgfqpoint{3.060880in}{2.380000in}}%
\pgfpathlineto{\pgfqpoint{3.064600in}{1.890000in}}%
\pgfpathlineto{\pgfqpoint{3.065840in}{1.995000in}}%
\pgfpathlineto{\pgfqpoint{3.067080in}{1.680000in}}%
\pgfpathlineto{\pgfqpoint{3.068320in}{1.960000in}}%
\pgfpathlineto{\pgfqpoint{3.069560in}{1.890000in}}%
\pgfpathlineto{\pgfqpoint{3.070800in}{2.030000in}}%
\pgfpathlineto{\pgfqpoint{3.072040in}{2.625000in}}%
\pgfpathlineto{\pgfqpoint{3.075760in}{1.400000in}}%
\pgfpathlineto{\pgfqpoint{3.077000in}{1.715000in}}%
\pgfpathlineto{\pgfqpoint{3.078240in}{1.470000in}}%
\pgfpathlineto{\pgfqpoint{3.079480in}{1.820000in}}%
\pgfpathlineto{\pgfqpoint{3.080720in}{1.610000in}}%
\pgfpathlineto{\pgfqpoint{3.081960in}{1.925000in}}%
\pgfpathlineto{\pgfqpoint{3.084440in}{1.750000in}}%
\pgfpathlineto{\pgfqpoint{3.085680in}{1.855000in}}%
\pgfpathlineto{\pgfqpoint{3.086920in}{1.715000in}}%
\pgfpathlineto{\pgfqpoint{3.088160in}{2.345000in}}%
\pgfpathlineto{\pgfqpoint{3.089400in}{1.855000in}}%
\pgfpathlineto{\pgfqpoint{3.090640in}{2.100000in}}%
\pgfpathlineto{\pgfqpoint{3.091880in}{2.065000in}}%
\pgfpathlineto{\pgfqpoint{3.094360in}{1.750000in}}%
\pgfpathlineto{\pgfqpoint{3.095600in}{1.750000in}}%
\pgfpathlineto{\pgfqpoint{3.096840in}{2.030000in}}%
\pgfpathlineto{\pgfqpoint{3.098080in}{1.995000in}}%
\pgfpathlineto{\pgfqpoint{3.099320in}{1.680000in}}%
\pgfpathlineto{\pgfqpoint{3.100560in}{1.960000in}}%
\pgfpathlineto{\pgfqpoint{3.101800in}{1.925000in}}%
\pgfpathlineto{\pgfqpoint{3.104280in}{1.680000in}}%
\pgfpathlineto{\pgfqpoint{3.106760in}{2.135000in}}%
\pgfpathlineto{\pgfqpoint{3.109240in}{1.540000in}}%
\pgfpathlineto{\pgfqpoint{3.110480in}{2.030000in}}%
\pgfpathlineto{\pgfqpoint{3.111720in}{1.715000in}}%
\pgfpathlineto{\pgfqpoint{3.114200in}{2.065000in}}%
\pgfpathlineto{\pgfqpoint{3.115440in}{1.855000in}}%
\pgfpathlineto{\pgfqpoint{3.116680in}{2.275000in}}%
\pgfpathlineto{\pgfqpoint{3.117920in}{2.205000in}}%
\pgfpathlineto{\pgfqpoint{3.119160in}{1.960000in}}%
\pgfpathlineto{\pgfqpoint{3.120400in}{2.065000in}}%
\pgfpathlineto{\pgfqpoint{3.122880in}{2.485000in}}%
\pgfpathlineto{\pgfqpoint{3.124120in}{2.415000in}}%
\pgfpathlineto{\pgfqpoint{3.125360in}{1.785000in}}%
\pgfpathlineto{\pgfqpoint{3.126600in}{2.065000in}}%
\pgfpathlineto{\pgfqpoint{3.127840in}{1.855000in}}%
\pgfpathlineto{\pgfqpoint{3.129080in}{1.995000in}}%
\pgfpathlineto{\pgfqpoint{3.130320in}{1.785000in}}%
\pgfpathlineto{\pgfqpoint{3.131560in}{1.925000in}}%
\pgfpathlineto{\pgfqpoint{3.134040in}{1.505000in}}%
\pgfpathlineto{\pgfqpoint{3.135280in}{1.750000in}}%
\pgfpathlineto{\pgfqpoint{3.136520in}{1.750000in}}%
\pgfpathlineto{\pgfqpoint{3.137760in}{1.925000in}}%
\pgfpathlineto{\pgfqpoint{3.139000in}{1.890000in}}%
\pgfpathlineto{\pgfqpoint{3.140240in}{2.380000in}}%
\pgfpathlineto{\pgfqpoint{3.141480in}{2.205000in}}%
\pgfpathlineto{\pgfqpoint{3.142720in}{2.275000in}}%
\pgfpathlineto{\pgfqpoint{3.143960in}{1.820000in}}%
\pgfpathlineto{\pgfqpoint{3.145200in}{2.170000in}}%
\pgfpathlineto{\pgfqpoint{3.146440in}{2.100000in}}%
\pgfpathlineto{\pgfqpoint{3.147680in}{2.240000in}}%
\pgfpathlineto{\pgfqpoint{3.148920in}{1.995000in}}%
\pgfpathlineto{\pgfqpoint{3.150160in}{2.135000in}}%
\pgfpathlineto{\pgfqpoint{3.151400in}{1.715000in}}%
\pgfpathlineto{\pgfqpoint{3.153880in}{2.415000in}}%
\pgfpathlineto{\pgfqpoint{3.155120in}{1.820000in}}%
\pgfpathlineto{\pgfqpoint{3.156360in}{2.100000in}}%
\pgfpathlineto{\pgfqpoint{3.157600in}{2.100000in}}%
\pgfpathlineto{\pgfqpoint{3.160080in}{1.715000in}}%
\pgfpathlineto{\pgfqpoint{3.161320in}{1.890000in}}%
\pgfpathlineto{\pgfqpoint{3.163800in}{1.295000in}}%
\pgfpathlineto{\pgfqpoint{3.165040in}{1.715000in}}%
\pgfpathlineto{\pgfqpoint{3.167520in}{1.540000in}}%
\pgfpathlineto{\pgfqpoint{3.168760in}{1.575000in}}%
\pgfpathlineto{\pgfqpoint{3.170000in}{1.925000in}}%
\pgfpathlineto{\pgfqpoint{3.171240in}{1.750000in}}%
\pgfpathlineto{\pgfqpoint{3.172480in}{1.960000in}}%
\pgfpathlineto{\pgfqpoint{3.173720in}{1.575000in}}%
\pgfpathlineto{\pgfqpoint{3.174960in}{1.925000in}}%
\pgfpathlineto{\pgfqpoint{3.176200in}{1.890000in}}%
\pgfpathlineto{\pgfqpoint{3.177440in}{1.785000in}}%
\pgfpathlineto{\pgfqpoint{3.179920in}{1.330000in}}%
\pgfpathlineto{\pgfqpoint{3.181160in}{1.960000in}}%
\pgfpathlineto{\pgfqpoint{3.182400in}{1.575000in}}%
\pgfpathlineto{\pgfqpoint{3.183640in}{1.960000in}}%
\pgfpathlineto{\pgfqpoint{3.184880in}{1.645000in}}%
\pgfpathlineto{\pgfqpoint{3.186120in}{1.960000in}}%
\pgfpathlineto{\pgfqpoint{3.187360in}{1.575000in}}%
\pgfpathlineto{\pgfqpoint{3.188600in}{1.575000in}}%
\pgfpathlineto{\pgfqpoint{3.189840in}{1.540000in}}%
\pgfpathlineto{\pgfqpoint{3.191080in}{1.470000in}}%
\pgfpathlineto{\pgfqpoint{3.192320in}{1.995000in}}%
\pgfpathlineto{\pgfqpoint{3.193560in}{1.925000in}}%
\pgfpathlineto{\pgfqpoint{3.194800in}{2.135000in}}%
\pgfpathlineto{\pgfqpoint{3.196040in}{1.890000in}}%
\pgfpathlineto{\pgfqpoint{3.198520in}{2.240000in}}%
\pgfpathlineto{\pgfqpoint{3.199760in}{2.205000in}}%
\pgfpathlineto{\pgfqpoint{3.201000in}{1.820000in}}%
\pgfpathlineto{\pgfqpoint{3.202240in}{2.100000in}}%
\pgfpathlineto{\pgfqpoint{3.203480in}{2.030000in}}%
\pgfpathlineto{\pgfqpoint{3.204720in}{1.750000in}}%
\pgfpathlineto{\pgfqpoint{3.207200in}{1.995000in}}%
\pgfpathlineto{\pgfqpoint{3.208440in}{1.715000in}}%
\pgfpathlineto{\pgfqpoint{3.209680in}{1.785000in}}%
\pgfpathlineto{\pgfqpoint{3.210920in}{1.925000in}}%
\pgfpathlineto{\pgfqpoint{3.212160in}{2.240000in}}%
\pgfpathlineto{\pgfqpoint{3.213400in}{1.715000in}}%
\pgfpathlineto{\pgfqpoint{3.214640in}{1.855000in}}%
\pgfpathlineto{\pgfqpoint{3.215880in}{2.205000in}}%
\pgfpathlineto{\pgfqpoint{3.217120in}{1.890000in}}%
\pgfpathlineto{\pgfqpoint{3.218360in}{2.170000in}}%
\pgfpathlineto{\pgfqpoint{3.220840in}{2.030000in}}%
\pgfpathlineto{\pgfqpoint{3.222080in}{1.610000in}}%
\pgfpathlineto{\pgfqpoint{3.223320in}{2.100000in}}%
\pgfpathlineto{\pgfqpoint{3.224560in}{1.890000in}}%
\pgfpathlineto{\pgfqpoint{3.225800in}{1.960000in}}%
\pgfpathlineto{\pgfqpoint{3.228280in}{1.400000in}}%
\pgfpathlineto{\pgfqpoint{3.230760in}{1.820000in}}%
\pgfpathlineto{\pgfqpoint{3.232000in}{1.855000in}}%
\pgfpathlineto{\pgfqpoint{3.233240in}{1.120000in}}%
\pgfpathlineto{\pgfqpoint{3.234480in}{1.960000in}}%
\pgfpathlineto{\pgfqpoint{3.235720in}{1.820000in}}%
\pgfpathlineto{\pgfqpoint{3.236960in}{2.135000in}}%
\pgfpathlineto{\pgfqpoint{3.239440in}{1.470000in}}%
\pgfpathlineto{\pgfqpoint{3.240680in}{1.785000in}}%
\pgfpathlineto{\pgfqpoint{3.241920in}{1.610000in}}%
\pgfpathlineto{\pgfqpoint{3.244400in}{1.995000in}}%
\pgfpathlineto{\pgfqpoint{3.245640in}{1.750000in}}%
\pgfpathlineto{\pgfqpoint{3.246880in}{2.030000in}}%
\pgfpathlineto{\pgfqpoint{3.249360in}{1.610000in}}%
\pgfpathlineto{\pgfqpoint{3.250600in}{1.435000in}}%
\pgfpathlineto{\pgfqpoint{3.251840in}{1.855000in}}%
\pgfpathlineto{\pgfqpoint{3.254320in}{1.575000in}}%
\pgfpathlineto{\pgfqpoint{3.256800in}{1.715000in}}%
\pgfpathlineto{\pgfqpoint{3.258040in}{1.855000in}}%
\pgfpathlineto{\pgfqpoint{3.260520in}{1.505000in}}%
\pgfpathlineto{\pgfqpoint{3.261760in}{2.065000in}}%
\pgfpathlineto{\pgfqpoint{3.263000in}{1.890000in}}%
\pgfpathlineto{\pgfqpoint{3.264240in}{2.135000in}}%
\pgfpathlineto{\pgfqpoint{3.265480in}{1.750000in}}%
\pgfpathlineto{\pgfqpoint{3.267960in}{1.890000in}}%
\pgfpathlineto{\pgfqpoint{3.269200in}{1.400000in}}%
\pgfpathlineto{\pgfqpoint{3.270440in}{1.400000in}}%
\pgfpathlineto{\pgfqpoint{3.271680in}{1.750000in}}%
\pgfpathlineto{\pgfqpoint{3.272920in}{1.750000in}}%
\pgfpathlineto{\pgfqpoint{3.274160in}{2.100000in}}%
\pgfpathlineto{\pgfqpoint{3.275400in}{1.995000in}}%
\pgfpathlineto{\pgfqpoint{3.276640in}{1.995000in}}%
\pgfpathlineto{\pgfqpoint{3.277880in}{1.820000in}}%
\pgfpathlineto{\pgfqpoint{3.280360in}{2.345000in}}%
\pgfpathlineto{\pgfqpoint{3.281600in}{1.610000in}}%
\pgfpathlineto{\pgfqpoint{3.282840in}{2.170000in}}%
\pgfpathlineto{\pgfqpoint{3.284080in}{1.855000in}}%
\pgfpathlineto{\pgfqpoint{3.285320in}{2.135000in}}%
\pgfpathlineto{\pgfqpoint{3.286560in}{1.960000in}}%
\pgfpathlineto{\pgfqpoint{3.287800in}{2.065000in}}%
\pgfpathlineto{\pgfqpoint{3.289040in}{1.890000in}}%
\pgfpathlineto{\pgfqpoint{3.290280in}{2.100000in}}%
\pgfpathlineto{\pgfqpoint{3.292760in}{2.030000in}}%
\pgfpathlineto{\pgfqpoint{3.295240in}{1.785000in}}%
\pgfpathlineto{\pgfqpoint{3.296480in}{2.170000in}}%
\pgfpathlineto{\pgfqpoint{3.297720in}{1.855000in}}%
\pgfpathlineto{\pgfqpoint{3.298960in}{2.205000in}}%
\pgfpathlineto{\pgfqpoint{3.301440in}{1.995000in}}%
\pgfpathlineto{\pgfqpoint{3.302680in}{2.415000in}}%
\pgfpathlineto{\pgfqpoint{3.303920in}{1.960000in}}%
\pgfpathlineto{\pgfqpoint{3.305160in}{1.995000in}}%
\pgfpathlineto{\pgfqpoint{3.306400in}{1.785000in}}%
\pgfpathlineto{\pgfqpoint{3.307640in}{2.030000in}}%
\pgfpathlineto{\pgfqpoint{3.308880in}{1.365000in}}%
\pgfpathlineto{\pgfqpoint{3.311360in}{1.820000in}}%
\pgfpathlineto{\pgfqpoint{3.313840in}{2.065000in}}%
\pgfpathlineto{\pgfqpoint{3.315080in}{1.575000in}}%
\pgfpathlineto{\pgfqpoint{3.316320in}{2.030000in}}%
\pgfpathlineto{\pgfqpoint{3.317560in}{1.820000in}}%
\pgfpathlineto{\pgfqpoint{3.318800in}{1.855000in}}%
\pgfpathlineto{\pgfqpoint{3.320040in}{1.925000in}}%
\pgfpathlineto{\pgfqpoint{3.322520in}{1.750000in}}%
\pgfpathlineto{\pgfqpoint{3.323760in}{1.190000in}}%
\pgfpathlineto{\pgfqpoint{3.325000in}{1.435000in}}%
\pgfpathlineto{\pgfqpoint{3.326240in}{1.925000in}}%
\pgfpathlineto{\pgfqpoint{3.327480in}{1.540000in}}%
\pgfpathlineto{\pgfqpoint{3.328720in}{1.750000in}}%
\pgfpathlineto{\pgfqpoint{3.329960in}{2.205000in}}%
\pgfpathlineto{\pgfqpoint{3.331200in}{1.890000in}}%
\pgfpathlineto{\pgfqpoint{3.333680in}{2.240000in}}%
\pgfpathlineto{\pgfqpoint{3.334920in}{1.890000in}}%
\pgfpathlineto{\pgfqpoint{3.337400in}{2.065000in}}%
\pgfpathlineto{\pgfqpoint{3.339880in}{1.365000in}}%
\pgfpathlineto{\pgfqpoint{3.341120in}{2.065000in}}%
\pgfpathlineto{\pgfqpoint{3.342360in}{1.855000in}}%
\pgfpathlineto{\pgfqpoint{3.343600in}{2.205000in}}%
\pgfpathlineto{\pgfqpoint{3.346080in}{1.995000in}}%
\pgfpathlineto{\pgfqpoint{3.347320in}{1.750000in}}%
\pgfpathlineto{\pgfqpoint{3.348560in}{1.750000in}}%
\pgfpathlineto{\pgfqpoint{3.352280in}{2.030000in}}%
\pgfpathlineto{\pgfqpoint{3.353520in}{1.470000in}}%
\pgfpathlineto{\pgfqpoint{3.354760in}{1.855000in}}%
\pgfpathlineto{\pgfqpoint{3.356000in}{1.785000in}}%
\pgfpathlineto{\pgfqpoint{3.357240in}{1.960000in}}%
\pgfpathlineto{\pgfqpoint{3.358480in}{1.960000in}}%
\pgfpathlineto{\pgfqpoint{3.359720in}{1.505000in}}%
\pgfpathlineto{\pgfqpoint{3.360960in}{1.645000in}}%
\pgfpathlineto{\pgfqpoint{3.362200in}{1.925000in}}%
\pgfpathlineto{\pgfqpoint{3.364680in}{1.680000in}}%
\pgfpathlineto{\pgfqpoint{3.365920in}{1.750000in}}%
\pgfpathlineto{\pgfqpoint{3.367160in}{1.890000in}}%
\pgfpathlineto{\pgfqpoint{3.369640in}{1.680000in}}%
\pgfpathlineto{\pgfqpoint{3.372120in}{2.030000in}}%
\pgfpathlineto{\pgfqpoint{3.373360in}{1.820000in}}%
\pgfpathlineto{\pgfqpoint{3.374600in}{1.925000in}}%
\pgfpathlineto{\pgfqpoint{3.375840in}{2.135000in}}%
\pgfpathlineto{\pgfqpoint{3.377080in}{1.610000in}}%
\pgfpathlineto{\pgfqpoint{3.378320in}{1.995000in}}%
\pgfpathlineto{\pgfqpoint{3.379560in}{1.890000in}}%
\pgfpathlineto{\pgfqpoint{3.380800in}{1.400000in}}%
\pgfpathlineto{\pgfqpoint{3.382040in}{1.960000in}}%
\pgfpathlineto{\pgfqpoint{3.383280in}{1.820000in}}%
\pgfpathlineto{\pgfqpoint{3.384520in}{1.820000in}}%
\pgfpathlineto{\pgfqpoint{3.385760in}{2.030000in}}%
\pgfpathlineto{\pgfqpoint{3.388240in}{1.470000in}}%
\pgfpathlineto{\pgfqpoint{3.389480in}{1.435000in}}%
\pgfpathlineto{\pgfqpoint{3.390720in}{1.820000in}}%
\pgfpathlineto{\pgfqpoint{3.391960in}{1.470000in}}%
\pgfpathlineto{\pgfqpoint{3.393200in}{1.470000in}}%
\pgfpathlineto{\pgfqpoint{3.395680in}{2.030000in}}%
\pgfpathlineto{\pgfqpoint{3.396920in}{1.645000in}}%
\pgfpathlineto{\pgfqpoint{3.398160in}{2.065000in}}%
\pgfpathlineto{\pgfqpoint{3.399400in}{2.030000in}}%
\pgfpathlineto{\pgfqpoint{3.400640in}{1.610000in}}%
\pgfpathlineto{\pgfqpoint{3.401880in}{1.925000in}}%
\pgfpathlineto{\pgfqpoint{3.405600in}{1.715000in}}%
\pgfpathlineto{\pgfqpoint{3.406840in}{1.925000in}}%
\pgfpathlineto{\pgfqpoint{3.409320in}{1.575000in}}%
\pgfpathlineto{\pgfqpoint{3.410560in}{1.785000in}}%
\pgfpathlineto{\pgfqpoint{3.411800in}{2.310000in}}%
\pgfpathlineto{\pgfqpoint{3.413040in}{2.100000in}}%
\pgfpathlineto{\pgfqpoint{3.414280in}{2.100000in}}%
\pgfpathlineto{\pgfqpoint{3.415520in}{1.820000in}}%
\pgfpathlineto{\pgfqpoint{3.416760in}{1.820000in}}%
\pgfpathlineto{\pgfqpoint{3.418000in}{1.645000in}}%
\pgfpathlineto{\pgfqpoint{3.419240in}{2.100000in}}%
\pgfpathlineto{\pgfqpoint{3.420480in}{1.855000in}}%
\pgfpathlineto{\pgfqpoint{3.421720in}{2.240000in}}%
\pgfpathlineto{\pgfqpoint{3.422960in}{1.715000in}}%
\pgfpathlineto{\pgfqpoint{3.424200in}{1.820000in}}%
\pgfpathlineto{\pgfqpoint{3.425440in}{1.820000in}}%
\pgfpathlineto{\pgfqpoint{3.426680in}{1.995000in}}%
\pgfpathlineto{\pgfqpoint{3.427920in}{1.855000in}}%
\pgfpathlineto{\pgfqpoint{3.429160in}{1.960000in}}%
\pgfpathlineto{\pgfqpoint{3.431640in}{1.260000in}}%
\pgfpathlineto{\pgfqpoint{3.432880in}{2.065000in}}%
\pgfpathlineto{\pgfqpoint{3.434120in}{1.505000in}}%
\pgfpathlineto{\pgfqpoint{3.435360in}{1.995000in}}%
\pgfpathlineto{\pgfqpoint{3.436600in}{1.785000in}}%
\pgfpathlineto{\pgfqpoint{3.437840in}{1.820000in}}%
\pgfpathlineto{\pgfqpoint{3.439080in}{1.820000in}}%
\pgfpathlineto{\pgfqpoint{3.440320in}{1.995000in}}%
\pgfpathlineto{\pgfqpoint{3.441560in}{1.750000in}}%
\pgfpathlineto{\pgfqpoint{3.442800in}{1.820000in}}%
\pgfpathlineto{\pgfqpoint{3.444040in}{1.645000in}}%
\pgfpathlineto{\pgfqpoint{3.445280in}{2.170000in}}%
\pgfpathlineto{\pgfqpoint{3.446520in}{1.610000in}}%
\pgfpathlineto{\pgfqpoint{3.447760in}{1.855000in}}%
\pgfpathlineto{\pgfqpoint{3.449000in}{1.820000in}}%
\pgfpathlineto{\pgfqpoint{3.450240in}{1.890000in}}%
\pgfpathlineto{\pgfqpoint{3.451480in}{2.065000in}}%
\pgfpathlineto{\pgfqpoint{3.452720in}{1.960000in}}%
\pgfpathlineto{\pgfqpoint{3.453960in}{1.715000in}}%
\pgfpathlineto{\pgfqpoint{3.455200in}{1.785000in}}%
\pgfpathlineto{\pgfqpoint{3.456440in}{2.170000in}}%
\pgfpathlineto{\pgfqpoint{3.457680in}{1.820000in}}%
\pgfpathlineto{\pgfqpoint{3.458920in}{2.170000in}}%
\pgfpathlineto{\pgfqpoint{3.460160in}{1.890000in}}%
\pgfpathlineto{\pgfqpoint{3.462640in}{1.995000in}}%
\pgfpathlineto{\pgfqpoint{3.463880in}{1.890000in}}%
\pgfpathlineto{\pgfqpoint{3.465120in}{2.170000in}}%
\pgfpathlineto{\pgfqpoint{3.467600in}{1.995000in}}%
\pgfpathlineto{\pgfqpoint{3.468840in}{1.960000in}}%
\pgfpathlineto{\pgfqpoint{3.470080in}{1.960000in}}%
\pgfpathlineto{\pgfqpoint{3.471320in}{1.890000in}}%
\pgfpathlineto{\pgfqpoint{3.472560in}{1.750000in}}%
\pgfpathlineto{\pgfqpoint{3.473800in}{2.030000in}}%
\pgfpathlineto{\pgfqpoint{3.475040in}{2.030000in}}%
\pgfpathlineto{\pgfqpoint{3.476280in}{1.960000in}}%
\pgfpathlineto{\pgfqpoint{3.477520in}{1.295000in}}%
\pgfpathlineto{\pgfqpoint{3.478760in}{2.170000in}}%
\pgfpathlineto{\pgfqpoint{3.480000in}{1.715000in}}%
\pgfpathlineto{\pgfqpoint{3.481240in}{1.960000in}}%
\pgfpathlineto{\pgfqpoint{3.483720in}{1.680000in}}%
\pgfpathlineto{\pgfqpoint{3.484960in}{1.610000in}}%
\pgfpathlineto{\pgfqpoint{3.486200in}{1.750000in}}%
\pgfpathlineto{\pgfqpoint{3.487440in}{2.205000in}}%
\pgfpathlineto{\pgfqpoint{3.488680in}{2.030000in}}%
\pgfpathlineto{\pgfqpoint{3.489920in}{1.645000in}}%
\pgfpathlineto{\pgfqpoint{3.492400in}{2.240000in}}%
\pgfpathlineto{\pgfqpoint{3.493640in}{1.925000in}}%
\pgfpathlineto{\pgfqpoint{3.494880in}{2.100000in}}%
\pgfpathlineto{\pgfqpoint{3.496120in}{1.960000in}}%
\pgfpathlineto{\pgfqpoint{3.497360in}{1.995000in}}%
\pgfpathlineto{\pgfqpoint{3.498600in}{2.205000in}}%
\pgfpathlineto{\pgfqpoint{3.499840in}{2.030000in}}%
\pgfpathlineto{\pgfqpoint{3.501080in}{2.030000in}}%
\pgfpathlineto{\pgfqpoint{3.502320in}{2.065000in}}%
\pgfpathlineto{\pgfqpoint{3.503560in}{2.345000in}}%
\pgfpathlineto{\pgfqpoint{3.504800in}{1.610000in}}%
\pgfpathlineto{\pgfqpoint{3.507280in}{2.030000in}}%
\pgfpathlineto{\pgfqpoint{3.508520in}{1.820000in}}%
\pgfpathlineto{\pgfqpoint{3.509760in}{1.400000in}}%
\pgfpathlineto{\pgfqpoint{3.512240in}{2.030000in}}%
\pgfpathlineto{\pgfqpoint{3.514720in}{1.890000in}}%
\pgfpathlineto{\pgfqpoint{3.515960in}{1.995000in}}%
\pgfpathlineto{\pgfqpoint{3.517200in}{1.855000in}}%
\pgfpathlineto{\pgfqpoint{3.518440in}{1.925000in}}%
\pgfpathlineto{\pgfqpoint{3.519680in}{2.205000in}}%
\pgfpathlineto{\pgfqpoint{3.522160in}{1.750000in}}%
\pgfpathlineto{\pgfqpoint{3.523400in}{1.785000in}}%
\pgfpathlineto{\pgfqpoint{3.524640in}{2.100000in}}%
\pgfpathlineto{\pgfqpoint{3.527120in}{1.785000in}}%
\pgfpathlineto{\pgfqpoint{3.528360in}{1.680000in}}%
\pgfpathlineto{\pgfqpoint{3.529600in}{1.680000in}}%
\pgfpathlineto{\pgfqpoint{3.530840in}{1.925000in}}%
\pgfpathlineto{\pgfqpoint{3.532080in}{1.750000in}}%
\pgfpathlineto{\pgfqpoint{3.534560in}{2.065000in}}%
\pgfpathlineto{\pgfqpoint{3.535800in}{2.100000in}}%
\pgfpathlineto{\pgfqpoint{3.538280in}{1.645000in}}%
\pgfpathlineto{\pgfqpoint{3.539520in}{2.100000in}}%
\pgfpathlineto{\pgfqpoint{3.540760in}{1.645000in}}%
\pgfpathlineto{\pgfqpoint{3.542000in}{1.855000in}}%
\pgfpathlineto{\pgfqpoint{3.543240in}{1.715000in}}%
\pgfpathlineto{\pgfqpoint{3.544480in}{1.435000in}}%
\pgfpathlineto{\pgfqpoint{3.545720in}{2.135000in}}%
\pgfpathlineto{\pgfqpoint{3.548200in}{1.855000in}}%
\pgfpathlineto{\pgfqpoint{3.549440in}{1.925000in}}%
\pgfpathlineto{\pgfqpoint{3.550680in}{1.820000in}}%
\pgfpathlineto{\pgfqpoint{3.551920in}{1.435000in}}%
\pgfpathlineto{\pgfqpoint{3.553160in}{1.995000in}}%
\pgfpathlineto{\pgfqpoint{3.554400in}{1.540000in}}%
\pgfpathlineto{\pgfqpoint{3.555640in}{1.505000in}}%
\pgfpathlineto{\pgfqpoint{3.558120in}{1.855000in}}%
\pgfpathlineto{\pgfqpoint{3.559360in}{1.680000in}}%
\pgfpathlineto{\pgfqpoint{3.560600in}{2.030000in}}%
\pgfpathlineto{\pgfqpoint{3.561840in}{2.030000in}}%
\pgfpathlineto{\pgfqpoint{3.563080in}{1.715000in}}%
\pgfpathlineto{\pgfqpoint{3.564320in}{1.995000in}}%
\pgfpathlineto{\pgfqpoint{3.565560in}{1.855000in}}%
\pgfpathlineto{\pgfqpoint{3.566800in}{2.170000in}}%
\pgfpathlineto{\pgfqpoint{3.568040in}{2.170000in}}%
\pgfpathlineto{\pgfqpoint{3.569280in}{1.750000in}}%
\pgfpathlineto{\pgfqpoint{3.571760in}{2.310000in}}%
\pgfpathlineto{\pgfqpoint{3.574240in}{1.680000in}}%
\pgfpathlineto{\pgfqpoint{3.575480in}{1.995000in}}%
\pgfpathlineto{\pgfqpoint{3.577960in}{1.750000in}}%
\pgfpathlineto{\pgfqpoint{3.580440in}{2.065000in}}%
\pgfpathlineto{\pgfqpoint{3.581680in}{2.065000in}}%
\pgfpathlineto{\pgfqpoint{3.582920in}{2.240000in}}%
\pgfpathlineto{\pgfqpoint{3.584160in}{1.960000in}}%
\pgfpathlineto{\pgfqpoint{3.585400in}{2.240000in}}%
\pgfpathlineto{\pgfqpoint{3.586640in}{1.715000in}}%
\pgfpathlineto{\pgfqpoint{3.587880in}{1.680000in}}%
\pgfpathlineto{\pgfqpoint{3.590360in}{1.925000in}}%
\pgfpathlineto{\pgfqpoint{3.592840in}{1.680000in}}%
\pgfpathlineto{\pgfqpoint{3.594080in}{1.995000in}}%
\pgfpathlineto{\pgfqpoint{3.595320in}{1.785000in}}%
\pgfpathlineto{\pgfqpoint{3.597800in}{2.030000in}}%
\pgfpathlineto{\pgfqpoint{3.600280in}{1.750000in}}%
\pgfpathlineto{\pgfqpoint{3.602760in}{1.995000in}}%
\pgfpathlineto{\pgfqpoint{3.604000in}{2.485000in}}%
\pgfpathlineto{\pgfqpoint{3.605240in}{2.100000in}}%
\pgfpathlineto{\pgfqpoint{3.607720in}{2.380000in}}%
\pgfpathlineto{\pgfqpoint{3.608960in}{1.890000in}}%
\pgfpathlineto{\pgfqpoint{3.610200in}{2.590000in}}%
\pgfpathlineto{\pgfqpoint{3.611440in}{2.135000in}}%
\pgfpathlineto{\pgfqpoint{3.612680in}{2.170000in}}%
\pgfpathlineto{\pgfqpoint{3.613920in}{2.275000in}}%
\pgfpathlineto{\pgfqpoint{3.617640in}{2.065000in}}%
\pgfpathlineto{\pgfqpoint{3.618880in}{2.205000in}}%
\pgfpathlineto{\pgfqpoint{3.621360in}{1.925000in}}%
\pgfpathlineto{\pgfqpoint{3.622600in}{1.960000in}}%
\pgfpathlineto{\pgfqpoint{3.625080in}{1.750000in}}%
\pgfpathlineto{\pgfqpoint{3.626320in}{1.820000in}}%
\pgfpathlineto{\pgfqpoint{3.627560in}{1.960000in}}%
\pgfpathlineto{\pgfqpoint{3.628800in}{1.925000in}}%
\pgfpathlineto{\pgfqpoint{3.630040in}{1.855000in}}%
\pgfpathlineto{\pgfqpoint{3.631280in}{1.505000in}}%
\pgfpathlineto{\pgfqpoint{3.632520in}{2.065000in}}%
\pgfpathlineto{\pgfqpoint{3.633760in}{1.750000in}}%
\pgfpathlineto{\pgfqpoint{3.636240in}{2.205000in}}%
\pgfpathlineto{\pgfqpoint{3.639960in}{1.610000in}}%
\pgfpathlineto{\pgfqpoint{3.641200in}{2.345000in}}%
\pgfpathlineto{\pgfqpoint{3.642440in}{1.890000in}}%
\pgfpathlineto{\pgfqpoint{3.644920in}{2.100000in}}%
\pgfpathlineto{\pgfqpoint{3.646160in}{2.345000in}}%
\pgfpathlineto{\pgfqpoint{3.647400in}{2.310000in}}%
\pgfpathlineto{\pgfqpoint{3.648640in}{1.855000in}}%
\pgfpathlineto{\pgfqpoint{3.651120in}{1.855000in}}%
\pgfpathlineto{\pgfqpoint{3.653600in}{2.345000in}}%
\pgfpathlineto{\pgfqpoint{3.654840in}{1.855000in}}%
\pgfpathlineto{\pgfqpoint{3.656080in}{2.275000in}}%
\pgfpathlineto{\pgfqpoint{3.657320in}{1.855000in}}%
\pgfpathlineto{\pgfqpoint{3.658560in}{1.890000in}}%
\pgfpathlineto{\pgfqpoint{3.659800in}{1.680000in}}%
\pgfpathlineto{\pgfqpoint{3.663520in}{2.100000in}}%
\pgfpathlineto{\pgfqpoint{3.664760in}{2.100000in}}%
\pgfpathlineto{\pgfqpoint{3.666000in}{1.680000in}}%
\pgfpathlineto{\pgfqpoint{3.667240in}{2.030000in}}%
\pgfpathlineto{\pgfqpoint{3.669720in}{1.470000in}}%
\pgfpathlineto{\pgfqpoint{3.673440in}{2.380000in}}%
\pgfpathlineto{\pgfqpoint{3.674680in}{2.275000in}}%
\pgfpathlineto{\pgfqpoint{3.678400in}{1.925000in}}%
\pgfpathlineto{\pgfqpoint{3.679640in}{2.310000in}}%
\pgfpathlineto{\pgfqpoint{3.680880in}{1.855000in}}%
\pgfpathlineto{\pgfqpoint{3.683360in}{1.995000in}}%
\pgfpathlineto{\pgfqpoint{3.684600in}{1.995000in}}%
\pgfpathlineto{\pgfqpoint{3.687080in}{1.715000in}}%
\pgfpathlineto{\pgfqpoint{3.689560in}{2.030000in}}%
\pgfpathlineto{\pgfqpoint{3.690800in}{2.030000in}}%
\pgfpathlineto{\pgfqpoint{3.692040in}{2.100000in}}%
\pgfpathlineto{\pgfqpoint{3.694520in}{1.715000in}}%
\pgfpathlineto{\pgfqpoint{3.695760in}{2.310000in}}%
\pgfpathlineto{\pgfqpoint{3.698240in}{1.855000in}}%
\pgfpathlineto{\pgfqpoint{3.699480in}{1.680000in}}%
\pgfpathlineto{\pgfqpoint{3.700720in}{1.995000in}}%
\pgfpathlineto{\pgfqpoint{3.701960in}{1.785000in}}%
\pgfpathlineto{\pgfqpoint{3.703200in}{2.100000in}}%
\pgfpathlineto{\pgfqpoint{3.704440in}{1.820000in}}%
\pgfpathlineto{\pgfqpoint{3.705680in}{2.135000in}}%
\pgfpathlineto{\pgfqpoint{3.706920in}{2.065000in}}%
\pgfpathlineto{\pgfqpoint{3.709400in}{1.540000in}}%
\pgfpathlineto{\pgfqpoint{3.711880in}{1.610000in}}%
\pgfpathlineto{\pgfqpoint{3.714360in}{2.345000in}}%
\pgfpathlineto{\pgfqpoint{3.715600in}{1.820000in}}%
\pgfpathlineto{\pgfqpoint{3.716840in}{2.240000in}}%
\pgfpathlineto{\pgfqpoint{3.718080in}{2.240000in}}%
\pgfpathlineto{\pgfqpoint{3.719320in}{1.435000in}}%
\pgfpathlineto{\pgfqpoint{3.721800in}{2.065000in}}%
\pgfpathlineto{\pgfqpoint{3.723040in}{1.995000in}}%
\pgfpathlineto{\pgfqpoint{3.724280in}{1.680000in}}%
\pgfpathlineto{\pgfqpoint{3.725520in}{1.925000in}}%
\pgfpathlineto{\pgfqpoint{3.726760in}{1.575000in}}%
\pgfpathlineto{\pgfqpoint{3.729240in}{2.030000in}}%
\pgfpathlineto{\pgfqpoint{3.731720in}{1.680000in}}%
\pgfpathlineto{\pgfqpoint{3.732960in}{1.680000in}}%
\pgfpathlineto{\pgfqpoint{3.734200in}{1.785000in}}%
\pgfpathlineto{\pgfqpoint{3.736680in}{1.680000in}}%
\pgfpathlineto{\pgfqpoint{3.740400in}{2.485000in}}%
\pgfpathlineto{\pgfqpoint{3.741640in}{1.890000in}}%
\pgfpathlineto{\pgfqpoint{3.742880in}{1.995000in}}%
\pgfpathlineto{\pgfqpoint{3.744120in}{2.240000in}}%
\pgfpathlineto{\pgfqpoint{3.746600in}{1.575000in}}%
\pgfpathlineto{\pgfqpoint{3.749080in}{1.925000in}}%
\pgfpathlineto{\pgfqpoint{3.750320in}{1.890000in}}%
\pgfpathlineto{\pgfqpoint{3.751560in}{1.820000in}}%
\pgfpathlineto{\pgfqpoint{3.752800in}{2.065000in}}%
\pgfpathlineto{\pgfqpoint{3.754040in}{1.785000in}}%
\pgfpathlineto{\pgfqpoint{3.755280in}{1.820000in}}%
\pgfpathlineto{\pgfqpoint{3.756520in}{1.995000in}}%
\pgfpathlineto{\pgfqpoint{3.757760in}{2.380000in}}%
\pgfpathlineto{\pgfqpoint{3.761480in}{1.855000in}}%
\pgfpathlineto{\pgfqpoint{3.762720in}{2.450000in}}%
\pgfpathlineto{\pgfqpoint{3.765200in}{1.820000in}}%
\pgfpathlineto{\pgfqpoint{3.766440in}{2.100000in}}%
\pgfpathlineto{\pgfqpoint{3.770160in}{1.540000in}}%
\pgfpathlineto{\pgfqpoint{3.771400in}{1.995000in}}%
\pgfpathlineto{\pgfqpoint{3.772640in}{1.750000in}}%
\pgfpathlineto{\pgfqpoint{3.773880in}{1.855000in}}%
\pgfpathlineto{\pgfqpoint{3.775120in}{1.680000in}}%
\pgfpathlineto{\pgfqpoint{3.776360in}{2.205000in}}%
\pgfpathlineto{\pgfqpoint{3.777600in}{1.890000in}}%
\pgfpathlineto{\pgfqpoint{3.780080in}{2.135000in}}%
\pgfpathlineto{\pgfqpoint{3.781320in}{1.820000in}}%
\pgfpathlineto{\pgfqpoint{3.783800in}{2.135000in}}%
\pgfpathlineto{\pgfqpoint{3.786280in}{1.645000in}}%
\pgfpathlineto{\pgfqpoint{3.791240in}{2.380000in}}%
\pgfpathlineto{\pgfqpoint{3.793720in}{1.750000in}}%
\pgfpathlineto{\pgfqpoint{3.794960in}{2.065000in}}%
\pgfpathlineto{\pgfqpoint{3.797440in}{1.680000in}}%
\pgfpathlineto{\pgfqpoint{3.798680in}{1.785000in}}%
\pgfpathlineto{\pgfqpoint{3.799920in}{1.610000in}}%
\pgfpathlineto{\pgfqpoint{3.802400in}{2.345000in}}%
\pgfpathlineto{\pgfqpoint{3.804880in}{1.540000in}}%
\pgfpathlineto{\pgfqpoint{3.806120in}{1.925000in}}%
\pgfpathlineto{\pgfqpoint{3.807360in}{1.890000in}}%
\pgfpathlineto{\pgfqpoint{3.808600in}{2.135000in}}%
\pgfpathlineto{\pgfqpoint{3.811080in}{1.680000in}}%
\pgfpathlineto{\pgfqpoint{3.812320in}{1.750000in}}%
\pgfpathlineto{\pgfqpoint{3.813560in}{1.680000in}}%
\pgfpathlineto{\pgfqpoint{3.814800in}{1.890000in}}%
\pgfpathlineto{\pgfqpoint{3.816040in}{1.435000in}}%
\pgfpathlineto{\pgfqpoint{3.818520in}{1.890000in}}%
\pgfpathlineto{\pgfqpoint{3.819760in}{1.645000in}}%
\pgfpathlineto{\pgfqpoint{3.822240in}{1.960000in}}%
\pgfpathlineto{\pgfqpoint{3.823480in}{1.960000in}}%
\pgfpathlineto{\pgfqpoint{3.824720in}{1.750000in}}%
\pgfpathlineto{\pgfqpoint{3.825960in}{1.785000in}}%
\pgfpathlineto{\pgfqpoint{3.827200in}{2.100000in}}%
\pgfpathlineto{\pgfqpoint{3.828440in}{1.820000in}}%
\pgfpathlineto{\pgfqpoint{3.829680in}{1.855000in}}%
\pgfpathlineto{\pgfqpoint{3.830920in}{1.540000in}}%
\pgfpathlineto{\pgfqpoint{3.833400in}{2.240000in}}%
\pgfpathlineto{\pgfqpoint{3.835880in}{1.995000in}}%
\pgfpathlineto{\pgfqpoint{3.837120in}{1.680000in}}%
\pgfpathlineto{\pgfqpoint{3.838360in}{1.750000in}}%
\pgfpathlineto{\pgfqpoint{3.839600in}{1.715000in}}%
\pgfpathlineto{\pgfqpoint{3.840840in}{1.400000in}}%
\pgfpathlineto{\pgfqpoint{3.842080in}{2.240000in}}%
\pgfpathlineto{\pgfqpoint{3.843320in}{1.540000in}}%
\pgfpathlineto{\pgfqpoint{3.844560in}{1.925000in}}%
\pgfpathlineto{\pgfqpoint{3.845800in}{1.925000in}}%
\pgfpathlineto{\pgfqpoint{3.847040in}{1.995000in}}%
\pgfpathlineto{\pgfqpoint{3.848280in}{1.925000in}}%
\pgfpathlineto{\pgfqpoint{3.849520in}{2.345000in}}%
\pgfpathlineto{\pgfqpoint{3.850760in}{1.890000in}}%
\pgfpathlineto{\pgfqpoint{3.852000in}{1.890000in}}%
\pgfpathlineto{\pgfqpoint{3.853240in}{1.820000in}}%
\pgfpathlineto{\pgfqpoint{3.854480in}{1.855000in}}%
\pgfpathlineto{\pgfqpoint{3.855720in}{1.435000in}}%
\pgfpathlineto{\pgfqpoint{3.856960in}{1.855000in}}%
\pgfpathlineto{\pgfqpoint{3.858200in}{1.855000in}}%
\pgfpathlineto{\pgfqpoint{3.859440in}{1.645000in}}%
\pgfpathlineto{\pgfqpoint{3.860680in}{2.100000in}}%
\pgfpathlineto{\pgfqpoint{3.861920in}{1.645000in}}%
\pgfpathlineto{\pgfqpoint{3.863160in}{1.680000in}}%
\pgfpathlineto{\pgfqpoint{3.864400in}{1.680000in}}%
\pgfpathlineto{\pgfqpoint{3.865640in}{2.275000in}}%
\pgfpathlineto{\pgfqpoint{3.866880in}{1.855000in}}%
\pgfpathlineto{\pgfqpoint{3.868120in}{2.030000in}}%
\pgfpathlineto{\pgfqpoint{3.870600in}{1.470000in}}%
\pgfpathlineto{\pgfqpoint{3.871840in}{2.065000in}}%
\pgfpathlineto{\pgfqpoint{3.873080in}{1.785000in}}%
\pgfpathlineto{\pgfqpoint{3.874320in}{1.820000in}}%
\pgfpathlineto{\pgfqpoint{3.875560in}{1.540000in}}%
\pgfpathlineto{\pgfqpoint{3.878040in}{1.960000in}}%
\pgfpathlineto{\pgfqpoint{3.879280in}{2.100000in}}%
\pgfpathlineto{\pgfqpoint{3.880520in}{1.610000in}}%
\pgfpathlineto{\pgfqpoint{3.881760in}{2.065000in}}%
\pgfpathlineto{\pgfqpoint{3.883000in}{1.995000in}}%
\pgfpathlineto{\pgfqpoint{3.884240in}{2.205000in}}%
\pgfpathlineto{\pgfqpoint{3.885480in}{1.995000in}}%
\pgfpathlineto{\pgfqpoint{3.886720in}{2.275000in}}%
\pgfpathlineto{\pgfqpoint{3.887960in}{2.065000in}}%
\pgfpathlineto{\pgfqpoint{3.889200in}{2.065000in}}%
\pgfpathlineto{\pgfqpoint{3.890440in}{2.205000in}}%
\pgfpathlineto{\pgfqpoint{3.891680in}{1.820000in}}%
\pgfpathlineto{\pgfqpoint{3.892920in}{1.820000in}}%
\pgfpathlineto{\pgfqpoint{3.894160in}{1.925000in}}%
\pgfpathlineto{\pgfqpoint{3.895400in}{1.680000in}}%
\pgfpathlineto{\pgfqpoint{3.897880in}{2.275000in}}%
\pgfpathlineto{\pgfqpoint{3.901600in}{1.820000in}}%
\pgfpathlineto{\pgfqpoint{3.904080in}{2.100000in}}%
\pgfpathlineto{\pgfqpoint{3.906560in}{1.890000in}}%
\pgfpathlineto{\pgfqpoint{3.907800in}{1.995000in}}%
\pgfpathlineto{\pgfqpoint{3.909040in}{1.995000in}}%
\pgfpathlineto{\pgfqpoint{3.910280in}{1.680000in}}%
\pgfpathlineto{\pgfqpoint{3.911520in}{2.170000in}}%
\pgfpathlineto{\pgfqpoint{3.914000in}{1.645000in}}%
\pgfpathlineto{\pgfqpoint{3.915240in}{1.715000in}}%
\pgfpathlineto{\pgfqpoint{3.917720in}{2.240000in}}%
\pgfpathlineto{\pgfqpoint{3.918960in}{1.505000in}}%
\pgfpathlineto{\pgfqpoint{3.921440in}{1.995000in}}%
\pgfpathlineto{\pgfqpoint{3.922680in}{1.645000in}}%
\pgfpathlineto{\pgfqpoint{3.923920in}{2.065000in}}%
\pgfpathlineto{\pgfqpoint{3.926400in}{1.750000in}}%
\pgfpathlineto{\pgfqpoint{3.927640in}{1.715000in}}%
\pgfpathlineto{\pgfqpoint{3.928880in}{1.995000in}}%
\pgfpathlineto{\pgfqpoint{3.930120in}{1.820000in}}%
\pgfpathlineto{\pgfqpoint{3.931360in}{2.170000in}}%
\pgfpathlineto{\pgfqpoint{3.933840in}{1.715000in}}%
\pgfpathlineto{\pgfqpoint{3.935080in}{1.785000in}}%
\pgfpathlineto{\pgfqpoint{3.936320in}{2.100000in}}%
\pgfpathlineto{\pgfqpoint{3.937560in}{2.100000in}}%
\pgfpathlineto{\pgfqpoint{3.938800in}{2.170000in}}%
\pgfpathlineto{\pgfqpoint{3.940040in}{1.505000in}}%
\pgfpathlineto{\pgfqpoint{3.941280in}{1.470000in}}%
\pgfpathlineto{\pgfqpoint{3.943760in}{1.820000in}}%
\pgfpathlineto{\pgfqpoint{3.945000in}{1.750000in}}%
\pgfpathlineto{\pgfqpoint{3.946240in}{1.400000in}}%
\pgfpathlineto{\pgfqpoint{3.947480in}{1.715000in}}%
\pgfpathlineto{\pgfqpoint{3.948720in}{1.680000in}}%
\pgfpathlineto{\pgfqpoint{3.949960in}{1.715000in}}%
\pgfpathlineto{\pgfqpoint{3.951200in}{2.065000in}}%
\pgfpathlineto{\pgfqpoint{3.952440in}{2.065000in}}%
\pgfpathlineto{\pgfqpoint{3.954920in}{1.610000in}}%
\pgfpathlineto{\pgfqpoint{3.957400in}{2.170000in}}%
\pgfpathlineto{\pgfqpoint{3.958640in}{1.680000in}}%
\pgfpathlineto{\pgfqpoint{3.961120in}{2.205000in}}%
\pgfpathlineto{\pgfqpoint{3.962360in}{2.275000in}}%
\pgfpathlineto{\pgfqpoint{3.963600in}{1.610000in}}%
\pgfpathlineto{\pgfqpoint{3.964840in}{2.275000in}}%
\pgfpathlineto{\pgfqpoint{3.966080in}{2.240000in}}%
\pgfpathlineto{\pgfqpoint{3.967320in}{2.065000in}}%
\pgfpathlineto{\pgfqpoint{3.968560in}{1.645000in}}%
\pgfpathlineto{\pgfqpoint{3.969800in}{1.715000in}}%
\pgfpathlineto{\pgfqpoint{3.971040in}{2.135000in}}%
\pgfpathlineto{\pgfqpoint{3.972280in}{1.890000in}}%
\pgfpathlineto{\pgfqpoint{3.973520in}{1.960000in}}%
\pgfpathlineto{\pgfqpoint{3.974760in}{1.820000in}}%
\pgfpathlineto{\pgfqpoint{3.977240in}{1.995000in}}%
\pgfpathlineto{\pgfqpoint{3.978480in}{1.995000in}}%
\pgfpathlineto{\pgfqpoint{3.979720in}{1.785000in}}%
\pgfpathlineto{\pgfqpoint{3.980960in}{2.240000in}}%
\pgfpathlineto{\pgfqpoint{3.982200in}{2.100000in}}%
\pgfpathlineto{\pgfqpoint{3.983440in}{2.100000in}}%
\pgfpathlineto{\pgfqpoint{3.985920in}{2.415000in}}%
\pgfpathlineto{\pgfqpoint{3.987160in}{1.435000in}}%
\pgfpathlineto{\pgfqpoint{3.989640in}{2.065000in}}%
\pgfpathlineto{\pgfqpoint{3.990880in}{1.960000in}}%
\pgfpathlineto{\pgfqpoint{3.992120in}{2.065000in}}%
\pgfpathlineto{\pgfqpoint{3.993360in}{1.540000in}}%
\pgfpathlineto{\pgfqpoint{3.994600in}{1.925000in}}%
\pgfpathlineto{\pgfqpoint{3.995840in}{1.435000in}}%
\pgfpathlineto{\pgfqpoint{3.998320in}{2.100000in}}%
\pgfpathlineto{\pgfqpoint{4.000800in}{1.575000in}}%
\pgfpathlineto{\pgfqpoint{4.002040in}{1.680000in}}%
\pgfpathlineto{\pgfqpoint{4.003280in}{1.295000in}}%
\pgfpathlineto{\pgfqpoint{4.004520in}{2.030000in}}%
\pgfpathlineto{\pgfqpoint{4.007000in}{1.680000in}}%
\pgfpathlineto{\pgfqpoint{4.008240in}{1.645000in}}%
\pgfpathlineto{\pgfqpoint{4.009480in}{1.785000in}}%
\pgfpathlineto{\pgfqpoint{4.010720in}{1.540000in}}%
\pgfpathlineto{\pgfqpoint{4.013200in}{1.925000in}}%
\pgfpathlineto{\pgfqpoint{4.014440in}{1.855000in}}%
\pgfpathlineto{\pgfqpoint{4.016920in}{2.240000in}}%
\pgfpathlineto{\pgfqpoint{4.019400in}{2.240000in}}%
\pgfpathlineto{\pgfqpoint{4.024360in}{1.575000in}}%
\pgfpathlineto{\pgfqpoint{4.025600in}{1.575000in}}%
\pgfpathlineto{\pgfqpoint{4.026840in}{1.925000in}}%
\pgfpathlineto{\pgfqpoint{4.028080in}{1.960000in}}%
\pgfpathlineto{\pgfqpoint{4.029320in}{1.785000in}}%
\pgfpathlineto{\pgfqpoint{4.030560in}{1.960000in}}%
\pgfpathlineto{\pgfqpoint{4.031800in}{1.750000in}}%
\pgfpathlineto{\pgfqpoint{4.033040in}{1.785000in}}%
\pgfpathlineto{\pgfqpoint{4.035520in}{2.100000in}}%
\pgfpathlineto{\pgfqpoint{4.036760in}{2.030000in}}%
\pgfpathlineto{\pgfqpoint{4.038000in}{1.645000in}}%
\pgfpathlineto{\pgfqpoint{4.040480in}{2.310000in}}%
\pgfpathlineto{\pgfqpoint{4.041720in}{1.750000in}}%
\pgfpathlineto{\pgfqpoint{4.042960in}{1.960000in}}%
\pgfpathlineto{\pgfqpoint{4.045440in}{1.505000in}}%
\pgfpathlineto{\pgfqpoint{4.047920in}{2.065000in}}%
\pgfpathlineto{\pgfqpoint{4.049160in}{1.960000in}}%
\pgfpathlineto{\pgfqpoint{4.050400in}{1.715000in}}%
\pgfpathlineto{\pgfqpoint{4.052880in}{2.170000in}}%
\pgfpathlineto{\pgfqpoint{4.054120in}{1.540000in}}%
\pgfpathlineto{\pgfqpoint{4.055360in}{1.855000in}}%
\pgfpathlineto{\pgfqpoint{4.056600in}{1.610000in}}%
\pgfpathlineto{\pgfqpoint{4.057840in}{1.855000in}}%
\pgfpathlineto{\pgfqpoint{4.059080in}{1.505000in}}%
\pgfpathlineto{\pgfqpoint{4.060320in}{2.100000in}}%
\pgfpathlineto{\pgfqpoint{4.062800in}{1.610000in}}%
\pgfpathlineto{\pgfqpoint{4.065280in}{2.170000in}}%
\pgfpathlineto{\pgfqpoint{4.067760in}{1.960000in}}%
\pgfpathlineto{\pgfqpoint{4.069000in}{1.960000in}}%
\pgfpathlineto{\pgfqpoint{4.070240in}{2.240000in}}%
\pgfpathlineto{\pgfqpoint{4.072720in}{1.470000in}}%
\pgfpathlineto{\pgfqpoint{4.073960in}{2.135000in}}%
\pgfpathlineto{\pgfqpoint{4.076440in}{1.785000in}}%
\pgfpathlineto{\pgfqpoint{4.077680in}{2.030000in}}%
\pgfpathlineto{\pgfqpoint{4.080160in}{1.820000in}}%
\pgfpathlineto{\pgfqpoint{4.081400in}{1.890000in}}%
\pgfpathlineto{\pgfqpoint{4.082640in}{1.855000in}}%
\pgfpathlineto{\pgfqpoint{4.083880in}{1.750000in}}%
\pgfpathlineto{\pgfqpoint{4.086360in}{1.960000in}}%
\pgfpathlineto{\pgfqpoint{4.088840in}{1.680000in}}%
\pgfpathlineto{\pgfqpoint{4.090080in}{1.785000in}}%
\pgfpathlineto{\pgfqpoint{4.091320in}{1.715000in}}%
\pgfpathlineto{\pgfqpoint{4.092560in}{1.470000in}}%
\pgfpathlineto{\pgfqpoint{4.093800in}{1.575000in}}%
\pgfpathlineto{\pgfqpoint{4.095040in}{2.100000in}}%
\pgfpathlineto{\pgfqpoint{4.096280in}{1.225000in}}%
\pgfpathlineto{\pgfqpoint{4.097520in}{2.030000in}}%
\pgfpathlineto{\pgfqpoint{4.098760in}{2.100000in}}%
\pgfpathlineto{\pgfqpoint{4.100000in}{2.100000in}}%
\pgfpathlineto{\pgfqpoint{4.101240in}{2.555000in}}%
\pgfpathlineto{\pgfqpoint{4.103720in}{1.995000in}}%
\pgfpathlineto{\pgfqpoint{4.106200in}{1.855000in}}%
\pgfpathlineto{\pgfqpoint{4.107440in}{1.855000in}}%
\pgfpathlineto{\pgfqpoint{4.108680in}{1.295000in}}%
\pgfpathlineto{\pgfqpoint{4.111160in}{1.995000in}}%
\pgfpathlineto{\pgfqpoint{4.112400in}{1.820000in}}%
\pgfpathlineto{\pgfqpoint{4.113640in}{2.030000in}}%
\pgfpathlineto{\pgfqpoint{4.116120in}{1.785000in}}%
\pgfpathlineto{\pgfqpoint{4.117360in}{2.170000in}}%
\pgfpathlineto{\pgfqpoint{4.119840in}{1.575000in}}%
\pgfpathlineto{\pgfqpoint{4.121080in}{1.785000in}}%
\pgfpathlineto{\pgfqpoint{4.122320in}{2.380000in}}%
\pgfpathlineto{\pgfqpoint{4.123560in}{2.100000in}}%
\pgfpathlineto{\pgfqpoint{4.124800in}{2.100000in}}%
\pgfpathlineto{\pgfqpoint{4.126040in}{2.205000in}}%
\pgfpathlineto{\pgfqpoint{4.127280in}{1.715000in}}%
\pgfpathlineto{\pgfqpoint{4.128520in}{1.995000in}}%
\pgfpathlineto{\pgfqpoint{4.129760in}{1.680000in}}%
\pgfpathlineto{\pgfqpoint{4.131000in}{2.205000in}}%
\pgfpathlineto{\pgfqpoint{4.132240in}{2.205000in}}%
\pgfpathlineto{\pgfqpoint{4.133480in}{1.855000in}}%
\pgfpathlineto{\pgfqpoint{4.134720in}{2.205000in}}%
\pgfpathlineto{\pgfqpoint{4.135960in}{2.205000in}}%
\pgfpathlineto{\pgfqpoint{4.137200in}{1.995000in}}%
\pgfpathlineto{\pgfqpoint{4.138440in}{2.030000in}}%
\pgfpathlineto{\pgfqpoint{4.139680in}{1.575000in}}%
\pgfpathlineto{\pgfqpoint{4.140920in}{1.855000in}}%
\pgfpathlineto{\pgfqpoint{4.143400in}{1.470000in}}%
\pgfpathlineto{\pgfqpoint{4.144640in}{1.890000in}}%
\pgfpathlineto{\pgfqpoint{4.145880in}{1.890000in}}%
\pgfpathlineto{\pgfqpoint{4.147120in}{1.855000in}}%
\pgfpathlineto{\pgfqpoint{4.148360in}{1.750000in}}%
\pgfpathlineto{\pgfqpoint{4.149600in}{1.785000in}}%
\pgfpathlineto{\pgfqpoint{4.150840in}{2.240000in}}%
\pgfpathlineto{\pgfqpoint{4.154560in}{1.540000in}}%
\pgfpathlineto{\pgfqpoint{4.155800in}{1.960000in}}%
\pgfpathlineto{\pgfqpoint{4.157040in}{1.785000in}}%
\pgfpathlineto{\pgfqpoint{4.158280in}{1.785000in}}%
\pgfpathlineto{\pgfqpoint{4.159520in}{1.855000in}}%
\pgfpathlineto{\pgfqpoint{4.160760in}{2.170000in}}%
\pgfpathlineto{\pgfqpoint{4.162000in}{1.750000in}}%
\pgfpathlineto{\pgfqpoint{4.163240in}{2.240000in}}%
\pgfpathlineto{\pgfqpoint{4.164480in}{2.205000in}}%
\pgfpathlineto{\pgfqpoint{4.166960in}{1.575000in}}%
\pgfpathlineto{\pgfqpoint{4.169440in}{1.960000in}}%
\pgfpathlineto{\pgfqpoint{4.171920in}{2.415000in}}%
\pgfpathlineto{\pgfqpoint{4.173160in}{1.995000in}}%
\pgfpathlineto{\pgfqpoint{4.174400in}{2.065000in}}%
\pgfpathlineto{\pgfqpoint{4.176880in}{1.820000in}}%
\pgfpathlineto{\pgfqpoint{4.178120in}{2.170000in}}%
\pgfpathlineto{\pgfqpoint{4.180600in}{1.785000in}}%
\pgfpathlineto{\pgfqpoint{4.181840in}{1.470000in}}%
\pgfpathlineto{\pgfqpoint{4.183080in}{2.135000in}}%
\pgfpathlineto{\pgfqpoint{4.186800in}{1.330000in}}%
\pgfpathlineto{\pgfqpoint{4.188040in}{2.205000in}}%
\pgfpathlineto{\pgfqpoint{4.190520in}{1.330000in}}%
\pgfpathlineto{\pgfqpoint{4.193000in}{1.960000in}}%
\pgfpathlineto{\pgfqpoint{4.194240in}{1.925000in}}%
\pgfpathlineto{\pgfqpoint{4.195480in}{1.925000in}}%
\pgfpathlineto{\pgfqpoint{4.196720in}{1.890000in}}%
\pgfpathlineto{\pgfqpoint{4.197960in}{2.170000in}}%
\pgfpathlineto{\pgfqpoint{4.199200in}{2.100000in}}%
\pgfpathlineto{\pgfqpoint{4.200440in}{1.645000in}}%
\pgfpathlineto{\pgfqpoint{4.202920in}{1.890000in}}%
\pgfpathlineto{\pgfqpoint{4.204160in}{1.610000in}}%
\pgfpathlineto{\pgfqpoint{4.206640in}{1.890000in}}%
\pgfpathlineto{\pgfqpoint{4.207880in}{2.030000in}}%
\pgfpathlineto{\pgfqpoint{4.209120in}{2.310000in}}%
\pgfpathlineto{\pgfqpoint{4.210360in}{1.330000in}}%
\pgfpathlineto{\pgfqpoint{4.211600in}{1.960000in}}%
\pgfpathlineto{\pgfqpoint{4.212840in}{1.470000in}}%
\pgfpathlineto{\pgfqpoint{4.214080in}{1.575000in}}%
\pgfpathlineto{\pgfqpoint{4.216560in}{2.030000in}}%
\pgfpathlineto{\pgfqpoint{4.217800in}{1.785000in}}%
\pgfpathlineto{\pgfqpoint{4.219040in}{1.855000in}}%
\pgfpathlineto{\pgfqpoint{4.220280in}{1.855000in}}%
\pgfpathlineto{\pgfqpoint{4.221520in}{2.065000in}}%
\pgfpathlineto{\pgfqpoint{4.222760in}{1.855000in}}%
\pgfpathlineto{\pgfqpoint{4.225240in}{1.995000in}}%
\pgfpathlineto{\pgfqpoint{4.226480in}{1.820000in}}%
\pgfpathlineto{\pgfqpoint{4.227720in}{1.925000in}}%
\pgfpathlineto{\pgfqpoint{4.228960in}{2.170000in}}%
\pgfpathlineto{\pgfqpoint{4.231440in}{1.750000in}}%
\pgfpathlineto{\pgfqpoint{4.232680in}{2.345000in}}%
\pgfpathlineto{\pgfqpoint{4.233920in}{1.960000in}}%
\pgfpathlineto{\pgfqpoint{4.235160in}{1.925000in}}%
\pgfpathlineto{\pgfqpoint{4.237640in}{1.925000in}}%
\pgfpathlineto{\pgfqpoint{4.238880in}{2.135000in}}%
\pgfpathlineto{\pgfqpoint{4.240120in}{1.890000in}}%
\pgfpathlineto{\pgfqpoint{4.241360in}{2.065000in}}%
\pgfpathlineto{\pgfqpoint{4.242600in}{1.925000in}}%
\pgfpathlineto{\pgfqpoint{4.245080in}{2.030000in}}%
\pgfpathlineto{\pgfqpoint{4.247560in}{1.960000in}}%
\pgfpathlineto{\pgfqpoint{4.248800in}{2.345000in}}%
\pgfpathlineto{\pgfqpoint{4.250040in}{1.995000in}}%
\pgfpathlineto{\pgfqpoint{4.251280in}{2.135000in}}%
\pgfpathlineto{\pgfqpoint{4.252520in}{2.030000in}}%
\pgfpathlineto{\pgfqpoint{4.253760in}{1.575000in}}%
\pgfpathlineto{\pgfqpoint{4.256240in}{1.960000in}}%
\pgfpathlineto{\pgfqpoint{4.257480in}{1.925000in}}%
\pgfpathlineto{\pgfqpoint{4.258720in}{1.925000in}}%
\pgfpathlineto{\pgfqpoint{4.259960in}{1.960000in}}%
\pgfpathlineto{\pgfqpoint{4.261200in}{1.960000in}}%
\pgfpathlineto{\pgfqpoint{4.263680in}{1.400000in}}%
\pgfpathlineto{\pgfqpoint{4.264920in}{1.505000in}}%
\pgfpathlineto{\pgfqpoint{4.267400in}{2.030000in}}%
\pgfpathlineto{\pgfqpoint{4.268640in}{2.100000in}}%
\pgfpathlineto{\pgfqpoint{4.269880in}{2.345000in}}%
\pgfpathlineto{\pgfqpoint{4.272360in}{1.645000in}}%
\pgfpathlineto{\pgfqpoint{4.273600in}{2.065000in}}%
\pgfpathlineto{\pgfqpoint{4.274840in}{2.065000in}}%
\pgfpathlineto{\pgfqpoint{4.276080in}{2.520000in}}%
\pgfpathlineto{\pgfqpoint{4.277320in}{2.275000in}}%
\pgfpathlineto{\pgfqpoint{4.278560in}{1.785000in}}%
\pgfpathlineto{\pgfqpoint{4.279800in}{2.275000in}}%
\pgfpathlineto{\pgfqpoint{4.283520in}{1.785000in}}%
\pgfpathlineto{\pgfqpoint{4.284760in}{2.170000in}}%
\pgfpathlineto{\pgfqpoint{4.286000in}{2.170000in}}%
\pgfpathlineto{\pgfqpoint{4.287240in}{1.960000in}}%
\pgfpathlineto{\pgfqpoint{4.289720in}{2.205000in}}%
\pgfpathlineto{\pgfqpoint{4.290960in}{2.065000in}}%
\pgfpathlineto{\pgfqpoint{4.292200in}{2.100000in}}%
\pgfpathlineto{\pgfqpoint{4.293440in}{2.275000in}}%
\pgfpathlineto{\pgfqpoint{4.294680in}{1.925000in}}%
\pgfpathlineto{\pgfqpoint{4.295920in}{1.960000in}}%
\pgfpathlineto{\pgfqpoint{4.297160in}{1.960000in}}%
\pgfpathlineto{\pgfqpoint{4.298400in}{2.205000in}}%
\pgfpathlineto{\pgfqpoint{4.300880in}{1.820000in}}%
\pgfpathlineto{\pgfqpoint{4.302120in}{1.680000in}}%
\pgfpathlineto{\pgfqpoint{4.304600in}{1.925000in}}%
\pgfpathlineto{\pgfqpoint{4.305840in}{1.505000in}}%
\pgfpathlineto{\pgfqpoint{4.307080in}{1.680000in}}%
\pgfpathlineto{\pgfqpoint{4.308320in}{2.065000in}}%
\pgfpathlineto{\pgfqpoint{4.309560in}{2.100000in}}%
\pgfpathlineto{\pgfqpoint{4.310800in}{1.925000in}}%
\pgfpathlineto{\pgfqpoint{4.312040in}{2.240000in}}%
\pgfpathlineto{\pgfqpoint{4.313280in}{2.240000in}}%
\pgfpathlineto{\pgfqpoint{4.314520in}{2.170000in}}%
\pgfpathlineto{\pgfqpoint{4.315760in}{2.275000in}}%
\pgfpathlineto{\pgfqpoint{4.317000in}{2.065000in}}%
\pgfpathlineto{\pgfqpoint{4.318240in}{2.065000in}}%
\pgfpathlineto{\pgfqpoint{4.320720in}{1.750000in}}%
\pgfpathlineto{\pgfqpoint{4.321960in}{2.135000in}}%
\pgfpathlineto{\pgfqpoint{4.323200in}{1.995000in}}%
\pgfpathlineto{\pgfqpoint{4.324440in}{2.135000in}}%
\pgfpathlineto{\pgfqpoint{4.325680in}{2.135000in}}%
\pgfpathlineto{\pgfqpoint{4.326920in}{1.750000in}}%
\pgfpathlineto{\pgfqpoint{4.329400in}{2.030000in}}%
\pgfpathlineto{\pgfqpoint{4.330640in}{1.645000in}}%
\pgfpathlineto{\pgfqpoint{4.331880in}{1.785000in}}%
\pgfpathlineto{\pgfqpoint{4.333120in}{1.785000in}}%
\pgfpathlineto{\pgfqpoint{4.334360in}{1.610000in}}%
\pgfpathlineto{\pgfqpoint{4.335600in}{2.065000in}}%
\pgfpathlineto{\pgfqpoint{4.336840in}{2.100000in}}%
\pgfpathlineto{\pgfqpoint{4.339320in}{1.610000in}}%
\pgfpathlineto{\pgfqpoint{4.340560in}{1.855000in}}%
\pgfpathlineto{\pgfqpoint{4.341800in}{1.715000in}}%
\pgfpathlineto{\pgfqpoint{4.343040in}{2.065000in}}%
\pgfpathlineto{\pgfqpoint{4.344280in}{1.540000in}}%
\pgfpathlineto{\pgfqpoint{4.346760in}{1.855000in}}%
\pgfpathlineto{\pgfqpoint{4.348000in}{1.505000in}}%
\pgfpathlineto{\pgfqpoint{4.349240in}{1.470000in}}%
\pgfpathlineto{\pgfqpoint{4.350480in}{1.750000in}}%
\pgfpathlineto{\pgfqpoint{4.351720in}{1.365000in}}%
\pgfpathlineto{\pgfqpoint{4.354200in}{1.785000in}}%
\pgfpathlineto{\pgfqpoint{4.356680in}{1.505000in}}%
\pgfpathlineto{\pgfqpoint{4.357920in}{1.540000in}}%
\pgfpathlineto{\pgfqpoint{4.359160in}{1.715000in}}%
\pgfpathlineto{\pgfqpoint{4.360400in}{1.715000in}}%
\pgfpathlineto{\pgfqpoint{4.361640in}{1.540000in}}%
\pgfpathlineto{\pgfqpoint{4.362880in}{1.890000in}}%
\pgfpathlineto{\pgfqpoint{4.364120in}{1.435000in}}%
\pgfpathlineto{\pgfqpoint{4.366600in}{1.995000in}}%
\pgfpathlineto{\pgfqpoint{4.367840in}{1.820000in}}%
\pgfpathlineto{\pgfqpoint{4.369080in}{1.820000in}}%
\pgfpathlineto{\pgfqpoint{4.371560in}{1.995000in}}%
\pgfpathlineto{\pgfqpoint{4.372800in}{1.925000in}}%
\pgfpathlineto{\pgfqpoint{4.374040in}{1.680000in}}%
\pgfpathlineto{\pgfqpoint{4.376520in}{2.030000in}}%
\pgfpathlineto{\pgfqpoint{4.377760in}{1.925000in}}%
\pgfpathlineto{\pgfqpoint{4.379000in}{1.470000in}}%
\pgfpathlineto{\pgfqpoint{4.381480in}{1.855000in}}%
\pgfpathlineto{\pgfqpoint{4.382720in}{2.310000in}}%
\pgfpathlineto{\pgfqpoint{4.385200in}{1.995000in}}%
\pgfpathlineto{\pgfqpoint{4.386440in}{1.995000in}}%
\pgfpathlineto{\pgfqpoint{4.387680in}{1.435000in}}%
\pgfpathlineto{\pgfqpoint{4.390160in}{2.170000in}}%
\pgfpathlineto{\pgfqpoint{4.391400in}{1.995000in}}%
\pgfpathlineto{\pgfqpoint{4.392640in}{2.030000in}}%
\pgfpathlineto{\pgfqpoint{4.393880in}{1.820000in}}%
\pgfpathlineto{\pgfqpoint{4.397600in}{2.310000in}}%
\pgfpathlineto{\pgfqpoint{4.398840in}{2.205000in}}%
\pgfpathlineto{\pgfqpoint{4.400080in}{1.750000in}}%
\pgfpathlineto{\pgfqpoint{4.402560in}{2.100000in}}%
\pgfpathlineto{\pgfqpoint{4.403800in}{1.960000in}}%
\pgfpathlineto{\pgfqpoint{4.405040in}{2.100000in}}%
\pgfpathlineto{\pgfqpoint{4.406280in}{1.680000in}}%
\pgfpathlineto{\pgfqpoint{4.407520in}{1.995000in}}%
\pgfpathlineto{\pgfqpoint{4.410000in}{1.855000in}}%
\pgfpathlineto{\pgfqpoint{4.411240in}{1.995000in}}%
\pgfpathlineto{\pgfqpoint{4.412480in}{1.540000in}}%
\pgfpathlineto{\pgfqpoint{4.413720in}{1.890000in}}%
\pgfpathlineto{\pgfqpoint{4.416200in}{1.575000in}}%
\pgfpathlineto{\pgfqpoint{4.418680in}{2.065000in}}%
\pgfpathlineto{\pgfqpoint{4.419920in}{1.855000in}}%
\pgfpathlineto{\pgfqpoint{4.421160in}{2.170000in}}%
\pgfpathlineto{\pgfqpoint{4.422400in}{2.100000in}}%
\pgfpathlineto{\pgfqpoint{4.423640in}{2.275000in}}%
\pgfpathlineto{\pgfqpoint{4.426120in}{1.680000in}}%
\pgfpathlineto{\pgfqpoint{4.428600in}{2.100000in}}%
\pgfpathlineto{\pgfqpoint{4.431080in}{1.750000in}}%
\pgfpathlineto{\pgfqpoint{4.432320in}{1.715000in}}%
\pgfpathlineto{\pgfqpoint{4.434800in}{1.995000in}}%
\pgfpathlineto{\pgfqpoint{4.436040in}{1.995000in}}%
\pgfpathlineto{\pgfqpoint{4.437280in}{1.855000in}}%
\pgfpathlineto{\pgfqpoint{4.438520in}{2.065000in}}%
\pgfpathlineto{\pgfqpoint{4.439760in}{1.890000in}}%
\pgfpathlineto{\pgfqpoint{4.441000in}{2.170000in}}%
\pgfpathlineto{\pgfqpoint{4.442240in}{1.995000in}}%
\pgfpathlineto{\pgfqpoint{4.443480in}{2.450000in}}%
\pgfpathlineto{\pgfqpoint{4.444720in}{1.890000in}}%
\pgfpathlineto{\pgfqpoint{4.445960in}{2.065000in}}%
\pgfpathlineto{\pgfqpoint{4.448440in}{1.575000in}}%
\pgfpathlineto{\pgfqpoint{4.449680in}{2.275000in}}%
\pgfpathlineto{\pgfqpoint{4.450920in}{1.890000in}}%
\pgfpathlineto{\pgfqpoint{4.452160in}{1.855000in}}%
\pgfpathlineto{\pgfqpoint{4.453400in}{1.995000in}}%
\pgfpathlineto{\pgfqpoint{4.454640in}{1.610000in}}%
\pgfpathlineto{\pgfqpoint{4.455880in}{1.820000in}}%
\pgfpathlineto{\pgfqpoint{4.457120in}{1.715000in}}%
\pgfpathlineto{\pgfqpoint{4.459600in}{1.820000in}}%
\pgfpathlineto{\pgfqpoint{4.462080in}{1.645000in}}%
\pgfpathlineto{\pgfqpoint{4.463320in}{1.680000in}}%
\pgfpathlineto{\pgfqpoint{4.465800in}{2.205000in}}%
\pgfpathlineto{\pgfqpoint{4.468280in}{1.820000in}}%
\pgfpathlineto{\pgfqpoint{4.469520in}{2.240000in}}%
\pgfpathlineto{\pgfqpoint{4.472000in}{1.610000in}}%
\pgfpathlineto{\pgfqpoint{4.473240in}{2.030000in}}%
\pgfpathlineto{\pgfqpoint{4.476960in}{1.645000in}}%
\pgfpathlineto{\pgfqpoint{4.478200in}{1.785000in}}%
\pgfpathlineto{\pgfqpoint{4.479440in}{1.505000in}}%
\pgfpathlineto{\pgfqpoint{4.480680in}{1.925000in}}%
\pgfpathlineto{\pgfqpoint{4.481920in}{1.855000in}}%
\pgfpathlineto{\pgfqpoint{4.483160in}{2.240000in}}%
\pgfpathlineto{\pgfqpoint{4.484400in}{1.610000in}}%
\pgfpathlineto{\pgfqpoint{4.488120in}{2.135000in}}%
\pgfpathlineto{\pgfqpoint{4.490600in}{1.855000in}}%
\pgfpathlineto{\pgfqpoint{4.493080in}{2.135000in}}%
\pgfpathlineto{\pgfqpoint{4.494320in}{1.785000in}}%
\pgfpathlineto{\pgfqpoint{4.495560in}{2.275000in}}%
\pgfpathlineto{\pgfqpoint{4.498040in}{1.960000in}}%
\pgfpathlineto{\pgfqpoint{4.499280in}{2.030000in}}%
\pgfpathlineto{\pgfqpoint{4.501760in}{1.890000in}}%
\pgfpathlineto{\pgfqpoint{4.503000in}{2.030000in}}%
\pgfpathlineto{\pgfqpoint{4.505480in}{1.435000in}}%
\pgfpathlineto{\pgfqpoint{4.507960in}{1.750000in}}%
\pgfpathlineto{\pgfqpoint{4.509200in}{1.365000in}}%
\pgfpathlineto{\pgfqpoint{4.510440in}{2.065000in}}%
\pgfpathlineto{\pgfqpoint{4.511680in}{1.995000in}}%
\pgfpathlineto{\pgfqpoint{4.512920in}{2.135000in}}%
\pgfpathlineto{\pgfqpoint{4.515400in}{1.925000in}}%
\pgfpathlineto{\pgfqpoint{4.517880in}{2.100000in}}%
\pgfpathlineto{\pgfqpoint{4.519120in}{2.485000in}}%
\pgfpathlineto{\pgfqpoint{4.521600in}{1.890000in}}%
\pgfpathlineto{\pgfqpoint{4.522840in}{2.030000in}}%
\pgfpathlineto{\pgfqpoint{4.524080in}{1.890000in}}%
\pgfpathlineto{\pgfqpoint{4.525320in}{2.345000in}}%
\pgfpathlineto{\pgfqpoint{4.526560in}{1.785000in}}%
\pgfpathlineto{\pgfqpoint{4.527800in}{2.205000in}}%
\pgfpathlineto{\pgfqpoint{4.529040in}{2.205000in}}%
\pgfpathlineto{\pgfqpoint{4.530280in}{1.715000in}}%
\pgfpathlineto{\pgfqpoint{4.532760in}{2.100000in}}%
\pgfpathlineto{\pgfqpoint{4.534000in}{1.890000in}}%
\pgfpathlineto{\pgfqpoint{4.535240in}{2.065000in}}%
\pgfpathlineto{\pgfqpoint{4.536480in}{1.890000in}}%
\pgfpathlineto{\pgfqpoint{4.537720in}{2.100000in}}%
\pgfpathlineto{\pgfqpoint{4.538960in}{1.645000in}}%
\pgfpathlineto{\pgfqpoint{4.540200in}{1.715000in}}%
\pgfpathlineto{\pgfqpoint{4.541440in}{1.715000in}}%
\pgfpathlineto{\pgfqpoint{4.543920in}{2.100000in}}%
\pgfpathlineto{\pgfqpoint{4.545160in}{2.030000in}}%
\pgfpathlineto{\pgfqpoint{4.546400in}{1.785000in}}%
\pgfpathlineto{\pgfqpoint{4.547640in}{1.785000in}}%
\pgfpathlineto{\pgfqpoint{4.548880in}{1.715000in}}%
\pgfpathlineto{\pgfqpoint{4.551360in}{2.310000in}}%
\pgfpathlineto{\pgfqpoint{4.555080in}{1.610000in}}%
\pgfpathlineto{\pgfqpoint{4.556320in}{1.680000in}}%
\pgfpathlineto{\pgfqpoint{4.560040in}{2.065000in}}%
\pgfpathlineto{\pgfqpoint{4.561280in}{2.030000in}}%
\pgfpathlineto{\pgfqpoint{4.562520in}{1.645000in}}%
\pgfpathlineto{\pgfqpoint{4.565000in}{2.520000in}}%
\pgfpathlineto{\pgfqpoint{4.566240in}{1.820000in}}%
\pgfpathlineto{\pgfqpoint{4.567480in}{2.065000in}}%
\pgfpathlineto{\pgfqpoint{4.569960in}{1.820000in}}%
\pgfpathlineto{\pgfqpoint{4.571200in}{1.820000in}}%
\pgfpathlineto{\pgfqpoint{4.572440in}{1.960000in}}%
\pgfpathlineto{\pgfqpoint{4.573680in}{1.925000in}}%
\pgfpathlineto{\pgfqpoint{4.574920in}{2.380000in}}%
\pgfpathlineto{\pgfqpoint{4.576160in}{1.820000in}}%
\pgfpathlineto{\pgfqpoint{4.577400in}{2.030000in}}%
\pgfpathlineto{\pgfqpoint{4.578640in}{1.715000in}}%
\pgfpathlineto{\pgfqpoint{4.579880in}{2.275000in}}%
\pgfpathlineto{\pgfqpoint{4.582360in}{1.785000in}}%
\pgfpathlineto{\pgfqpoint{4.583600in}{1.540000in}}%
\pgfpathlineto{\pgfqpoint{4.586080in}{1.925000in}}%
\pgfpathlineto{\pgfqpoint{4.587320in}{1.820000in}}%
\pgfpathlineto{\pgfqpoint{4.588560in}{2.065000in}}%
\pgfpathlineto{\pgfqpoint{4.589800in}{1.330000in}}%
\pgfpathlineto{\pgfqpoint{4.591040in}{1.715000in}}%
\pgfpathlineto{\pgfqpoint{4.592280in}{1.645000in}}%
\pgfpathlineto{\pgfqpoint{4.593520in}{1.365000in}}%
\pgfpathlineto{\pgfqpoint{4.596000in}{2.135000in}}%
\pgfpathlineto{\pgfqpoint{4.599720in}{1.575000in}}%
\pgfpathlineto{\pgfqpoint{4.600960in}{1.785000in}}%
\pgfpathlineto{\pgfqpoint{4.602200in}{1.785000in}}%
\pgfpathlineto{\pgfqpoint{4.603440in}{1.505000in}}%
\pgfpathlineto{\pgfqpoint{4.604680in}{2.170000in}}%
\pgfpathlineto{\pgfqpoint{4.607160in}{1.505000in}}%
\pgfpathlineto{\pgfqpoint{4.608400in}{1.855000in}}%
\pgfpathlineto{\pgfqpoint{4.610880in}{1.540000in}}%
\pgfpathlineto{\pgfqpoint{4.612120in}{1.470000in}}%
\pgfpathlineto{\pgfqpoint{4.614600in}{2.135000in}}%
\pgfpathlineto{\pgfqpoint{4.615840in}{2.170000in}}%
\pgfpathlineto{\pgfqpoint{4.617080in}{1.890000in}}%
\pgfpathlineto{\pgfqpoint{4.618320in}{2.345000in}}%
\pgfpathlineto{\pgfqpoint{4.619560in}{2.100000in}}%
\pgfpathlineto{\pgfqpoint{4.620800in}{2.520000in}}%
\pgfpathlineto{\pgfqpoint{4.623280in}{1.820000in}}%
\pgfpathlineto{\pgfqpoint{4.625760in}{2.030000in}}%
\pgfpathlineto{\pgfqpoint{4.627000in}{1.925000in}}%
\pgfpathlineto{\pgfqpoint{4.628240in}{2.310000in}}%
\pgfpathlineto{\pgfqpoint{4.629480in}{2.100000in}}%
\pgfpathlineto{\pgfqpoint{4.630720in}{2.205000in}}%
\pgfpathlineto{\pgfqpoint{4.631960in}{1.960000in}}%
\pgfpathlineto{\pgfqpoint{4.633200in}{2.065000in}}%
\pgfpathlineto{\pgfqpoint{4.634440in}{1.855000in}}%
\pgfpathlineto{\pgfqpoint{4.639400in}{2.275000in}}%
\pgfpathlineto{\pgfqpoint{4.641880in}{1.925000in}}%
\pgfpathlineto{\pgfqpoint{4.643120in}{1.890000in}}%
\pgfpathlineto{\pgfqpoint{4.645600in}{1.995000in}}%
\pgfpathlineto{\pgfqpoint{4.646840in}{1.960000in}}%
\pgfpathlineto{\pgfqpoint{4.648080in}{1.750000in}}%
\pgfpathlineto{\pgfqpoint{4.651800in}{2.030000in}}%
\pgfpathlineto{\pgfqpoint{4.653040in}{1.715000in}}%
\pgfpathlineto{\pgfqpoint{4.655520in}{1.995000in}}%
\pgfpathlineto{\pgfqpoint{4.656760in}{1.890000in}}%
\pgfpathlineto{\pgfqpoint{4.659240in}{1.365000in}}%
\pgfpathlineto{\pgfqpoint{4.660480in}{1.470000in}}%
\pgfpathlineto{\pgfqpoint{4.661720in}{1.820000in}}%
\pgfpathlineto{\pgfqpoint{4.662960in}{1.400000in}}%
\pgfpathlineto{\pgfqpoint{4.665440in}{1.645000in}}%
\pgfpathlineto{\pgfqpoint{4.666680in}{1.925000in}}%
\pgfpathlineto{\pgfqpoint{4.669160in}{1.470000in}}%
\pgfpathlineto{\pgfqpoint{4.670400in}{2.030000in}}%
\pgfpathlineto{\pgfqpoint{4.671640in}{2.065000in}}%
\pgfpathlineto{\pgfqpoint{4.672880in}{1.715000in}}%
\pgfpathlineto{\pgfqpoint{4.674120in}{2.100000in}}%
\pgfpathlineto{\pgfqpoint{4.675360in}{1.820000in}}%
\pgfpathlineto{\pgfqpoint{4.676600in}{1.995000in}}%
\pgfpathlineto{\pgfqpoint{4.677840in}{1.855000in}}%
\pgfpathlineto{\pgfqpoint{4.679080in}{2.135000in}}%
\pgfpathlineto{\pgfqpoint{4.680320in}{1.785000in}}%
\pgfpathlineto{\pgfqpoint{4.681560in}{1.820000in}}%
\pgfpathlineto{\pgfqpoint{4.682800in}{2.065000in}}%
\pgfpathlineto{\pgfqpoint{4.684040in}{1.645000in}}%
\pgfpathlineto{\pgfqpoint{4.685280in}{1.715000in}}%
\pgfpathlineto{\pgfqpoint{4.686520in}{2.205000in}}%
\pgfpathlineto{\pgfqpoint{4.687760in}{2.240000in}}%
\pgfpathlineto{\pgfqpoint{4.689000in}{2.240000in}}%
\pgfpathlineto{\pgfqpoint{4.690240in}{2.170000in}}%
\pgfpathlineto{\pgfqpoint{4.692720in}{1.995000in}}%
\pgfpathlineto{\pgfqpoint{4.693960in}{2.135000in}}%
\pgfpathlineto{\pgfqpoint{4.695200in}{1.330000in}}%
\pgfpathlineto{\pgfqpoint{4.696440in}{2.065000in}}%
\pgfpathlineto{\pgfqpoint{4.697680in}{2.030000in}}%
\pgfpathlineto{\pgfqpoint{4.700160in}{1.750000in}}%
\pgfpathlineto{\pgfqpoint{4.702640in}{1.750000in}}%
\pgfpathlineto{\pgfqpoint{4.703880in}{1.820000in}}%
\pgfpathlineto{\pgfqpoint{4.706360in}{1.505000in}}%
\pgfpathlineto{\pgfqpoint{4.707600in}{2.205000in}}%
\pgfpathlineto{\pgfqpoint{4.708840in}{1.680000in}}%
\pgfpathlineto{\pgfqpoint{4.710080in}{1.925000in}}%
\pgfpathlineto{\pgfqpoint{4.712560in}{1.680000in}}%
\pgfpathlineto{\pgfqpoint{4.713800in}{1.505000in}}%
\pgfpathlineto{\pgfqpoint{4.715040in}{1.785000in}}%
\pgfpathlineto{\pgfqpoint{4.716280in}{1.575000in}}%
\pgfpathlineto{\pgfqpoint{4.717520in}{1.890000in}}%
\pgfpathlineto{\pgfqpoint{4.718760in}{1.785000in}}%
\pgfpathlineto{\pgfqpoint{4.720000in}{1.575000in}}%
\pgfpathlineto{\pgfqpoint{4.722480in}{1.960000in}}%
\pgfpathlineto{\pgfqpoint{4.723720in}{1.680000in}}%
\pgfpathlineto{\pgfqpoint{4.724960in}{2.100000in}}%
\pgfpathlineto{\pgfqpoint{4.727440in}{1.610000in}}%
\pgfpathlineto{\pgfqpoint{4.729920in}{1.890000in}}%
\pgfpathlineto{\pgfqpoint{4.731160in}{1.610000in}}%
\pgfpathlineto{\pgfqpoint{4.732400in}{2.030000in}}%
\pgfpathlineto{\pgfqpoint{4.733640in}{1.855000in}}%
\pgfpathlineto{\pgfqpoint{4.734880in}{1.505000in}}%
\pgfpathlineto{\pgfqpoint{4.736120in}{1.575000in}}%
\pgfpathlineto{\pgfqpoint{4.738600in}{2.275000in}}%
\pgfpathlineto{\pgfqpoint{4.739840in}{2.170000in}}%
\pgfpathlineto{\pgfqpoint{4.741080in}{1.715000in}}%
\pgfpathlineto{\pgfqpoint{4.742320in}{2.135000in}}%
\pgfpathlineto{\pgfqpoint{4.743560in}{1.470000in}}%
\pgfpathlineto{\pgfqpoint{4.746040in}{1.855000in}}%
\pgfpathlineto{\pgfqpoint{4.747280in}{1.960000in}}%
\pgfpathlineto{\pgfqpoint{4.748520in}{1.785000in}}%
\pgfpathlineto{\pgfqpoint{4.751000in}{2.100000in}}%
\pgfpathlineto{\pgfqpoint{4.752240in}{1.715000in}}%
\pgfpathlineto{\pgfqpoint{4.753480in}{1.750000in}}%
\pgfpathlineto{\pgfqpoint{4.754720in}{1.925000in}}%
\pgfpathlineto{\pgfqpoint{4.755960in}{1.855000in}}%
\pgfpathlineto{\pgfqpoint{4.759680in}{2.310000in}}%
\pgfpathlineto{\pgfqpoint{4.760920in}{1.855000in}}%
\pgfpathlineto{\pgfqpoint{4.762160in}{1.995000in}}%
\pgfpathlineto{\pgfqpoint{4.763400in}{1.750000in}}%
\pgfpathlineto{\pgfqpoint{4.764640in}{1.855000in}}%
\pgfpathlineto{\pgfqpoint{4.765880in}{1.750000in}}%
\pgfpathlineto{\pgfqpoint{4.768360in}{1.995000in}}%
\pgfpathlineto{\pgfqpoint{4.769600in}{1.995000in}}%
\pgfpathlineto{\pgfqpoint{4.770840in}{2.240000in}}%
\pgfpathlineto{\pgfqpoint{4.772080in}{1.750000in}}%
\pgfpathlineto{\pgfqpoint{4.773320in}{1.750000in}}%
\pgfpathlineto{\pgfqpoint{4.774560in}{1.540000in}}%
\pgfpathlineto{\pgfqpoint{4.775800in}{1.785000in}}%
\pgfpathlineto{\pgfqpoint{4.777040in}{1.470000in}}%
\pgfpathlineto{\pgfqpoint{4.778280in}{1.750000in}}%
\pgfpathlineto{\pgfqpoint{4.779520in}{1.575000in}}%
\pgfpathlineto{\pgfqpoint{4.780760in}{2.135000in}}%
\pgfpathlineto{\pgfqpoint{4.783240in}{1.575000in}}%
\pgfpathlineto{\pgfqpoint{4.784480in}{2.030000in}}%
\pgfpathlineto{\pgfqpoint{4.785720in}{1.330000in}}%
\pgfpathlineto{\pgfqpoint{4.788200in}{2.100000in}}%
\pgfpathlineto{\pgfqpoint{4.790680in}{1.715000in}}%
\pgfpathlineto{\pgfqpoint{4.791920in}{1.960000in}}%
\pgfpathlineto{\pgfqpoint{4.793160in}{1.960000in}}%
\pgfpathlineto{\pgfqpoint{4.794400in}{1.785000in}}%
\pgfpathlineto{\pgfqpoint{4.796880in}{2.205000in}}%
\pgfpathlineto{\pgfqpoint{4.799360in}{1.505000in}}%
\pgfpathlineto{\pgfqpoint{4.800600in}{1.785000in}}%
\pgfpathlineto{\pgfqpoint{4.801840in}{1.505000in}}%
\pgfpathlineto{\pgfqpoint{4.803080in}{1.645000in}}%
\pgfpathlineto{\pgfqpoint{4.804320in}{1.575000in}}%
\pgfpathlineto{\pgfqpoint{4.805560in}{1.960000in}}%
\pgfpathlineto{\pgfqpoint{4.806800in}{1.785000in}}%
\pgfpathlineto{\pgfqpoint{4.809280in}{2.170000in}}%
\pgfpathlineto{\pgfqpoint{4.810520in}{1.820000in}}%
\pgfpathlineto{\pgfqpoint{4.813000in}{2.240000in}}%
\pgfpathlineto{\pgfqpoint{4.814240in}{2.240000in}}%
\pgfpathlineto{\pgfqpoint{4.815480in}{2.030000in}}%
\pgfpathlineto{\pgfqpoint{4.816720in}{2.380000in}}%
\pgfpathlineto{\pgfqpoint{4.819200in}{1.750000in}}%
\pgfpathlineto{\pgfqpoint{4.820440in}{1.890000in}}%
\pgfpathlineto{\pgfqpoint{4.821680in}{2.170000in}}%
\pgfpathlineto{\pgfqpoint{4.822920in}{1.995000in}}%
\pgfpathlineto{\pgfqpoint{4.824160in}{2.065000in}}%
\pgfpathlineto{\pgfqpoint{4.825400in}{1.575000in}}%
\pgfpathlineto{\pgfqpoint{4.826640in}{2.030000in}}%
\pgfpathlineto{\pgfqpoint{4.827880in}{1.610000in}}%
\pgfpathlineto{\pgfqpoint{4.829120in}{1.645000in}}%
\pgfpathlineto{\pgfqpoint{4.830360in}{1.785000in}}%
\pgfpathlineto{\pgfqpoint{4.832840in}{1.575000in}}%
\pgfpathlineto{\pgfqpoint{4.834080in}{1.750000in}}%
\pgfpathlineto{\pgfqpoint{4.835320in}{2.135000in}}%
\pgfpathlineto{\pgfqpoint{4.836560in}{2.170000in}}%
\pgfpathlineto{\pgfqpoint{4.837800in}{1.785000in}}%
\pgfpathlineto{\pgfqpoint{4.839040in}{1.995000in}}%
\pgfpathlineto{\pgfqpoint{4.840280in}{1.750000in}}%
\pgfpathlineto{\pgfqpoint{4.841520in}{1.785000in}}%
\pgfpathlineto{\pgfqpoint{4.842760in}{1.540000in}}%
\pgfpathlineto{\pgfqpoint{4.844000in}{1.645000in}}%
\pgfpathlineto{\pgfqpoint{4.845240in}{1.960000in}}%
\pgfpathlineto{\pgfqpoint{4.846480in}{1.960000in}}%
\pgfpathlineto{\pgfqpoint{4.847720in}{1.645000in}}%
\pgfpathlineto{\pgfqpoint{4.851440in}{1.995000in}}%
\pgfpathlineto{\pgfqpoint{4.853920in}{1.715000in}}%
\pgfpathlineto{\pgfqpoint{4.855160in}{1.890000in}}%
\pgfpathlineto{\pgfqpoint{4.856400in}{1.680000in}}%
\pgfpathlineto{\pgfqpoint{4.858880in}{1.925000in}}%
\pgfpathlineto{\pgfqpoint{4.860120in}{1.610000in}}%
\pgfpathlineto{\pgfqpoint{4.862600in}{1.785000in}}%
\pgfpathlineto{\pgfqpoint{4.863840in}{2.170000in}}%
\pgfpathlineto{\pgfqpoint{4.865080in}{1.715000in}}%
\pgfpathlineto{\pgfqpoint{4.866320in}{1.785000in}}%
\pgfpathlineto{\pgfqpoint{4.867560in}{1.400000in}}%
\pgfpathlineto{\pgfqpoint{4.868800in}{2.100000in}}%
\pgfpathlineto{\pgfqpoint{4.870040in}{2.030000in}}%
\pgfpathlineto{\pgfqpoint{4.871280in}{1.680000in}}%
\pgfpathlineto{\pgfqpoint{4.873760in}{2.065000in}}%
\pgfpathlineto{\pgfqpoint{4.875000in}{1.995000in}}%
\pgfpathlineto{\pgfqpoint{4.877480in}{1.820000in}}%
\pgfpathlineto{\pgfqpoint{4.878720in}{1.785000in}}%
\pgfpathlineto{\pgfqpoint{4.879960in}{1.645000in}}%
\pgfpathlineto{\pgfqpoint{4.881200in}{1.820000in}}%
\pgfpathlineto{\pgfqpoint{4.882440in}{1.680000in}}%
\pgfpathlineto{\pgfqpoint{4.884920in}{1.960000in}}%
\pgfpathlineto{\pgfqpoint{4.886160in}{1.645000in}}%
\pgfpathlineto{\pgfqpoint{4.887400in}{1.960000in}}%
\pgfpathlineto{\pgfqpoint{4.888640in}{1.750000in}}%
\pgfpathlineto{\pgfqpoint{4.889880in}{2.100000in}}%
\pgfpathlineto{\pgfqpoint{4.891120in}{2.100000in}}%
\pgfpathlineto{\pgfqpoint{4.892360in}{1.820000in}}%
\pgfpathlineto{\pgfqpoint{4.894840in}{2.100000in}}%
\pgfpathlineto{\pgfqpoint{4.896080in}{1.890000in}}%
\pgfpathlineto{\pgfqpoint{4.897320in}{2.030000in}}%
\pgfpathlineto{\pgfqpoint{4.898560in}{1.610000in}}%
\pgfpathlineto{\pgfqpoint{4.899800in}{1.925000in}}%
\pgfpathlineto{\pgfqpoint{4.901040in}{1.610000in}}%
\pgfpathlineto{\pgfqpoint{4.902280in}{2.170000in}}%
\pgfpathlineto{\pgfqpoint{4.903520in}{2.030000in}}%
\pgfpathlineto{\pgfqpoint{4.904760in}{2.100000in}}%
\pgfpathlineto{\pgfqpoint{4.906000in}{1.400000in}}%
\pgfpathlineto{\pgfqpoint{4.908480in}{2.100000in}}%
\pgfpathlineto{\pgfqpoint{4.909720in}{2.205000in}}%
\pgfpathlineto{\pgfqpoint{4.910960in}{2.170000in}}%
\pgfpathlineto{\pgfqpoint{4.912200in}{1.820000in}}%
\pgfpathlineto{\pgfqpoint{4.913440in}{2.030000in}}%
\pgfpathlineto{\pgfqpoint{4.914680in}{1.680000in}}%
\pgfpathlineto{\pgfqpoint{4.915920in}{2.030000in}}%
\pgfpathlineto{\pgfqpoint{4.917160in}{1.960000in}}%
\pgfpathlineto{\pgfqpoint{4.918400in}{2.135000in}}%
\pgfpathlineto{\pgfqpoint{4.919640in}{1.575000in}}%
\pgfpathlineto{\pgfqpoint{4.920880in}{1.820000in}}%
\pgfpathlineto{\pgfqpoint{4.923360in}{1.540000in}}%
\pgfpathlineto{\pgfqpoint{4.925840in}{2.100000in}}%
\pgfpathlineto{\pgfqpoint{4.928320in}{1.855000in}}%
\pgfpathlineto{\pgfqpoint{4.929560in}{1.995000in}}%
\pgfpathlineto{\pgfqpoint{4.930800in}{1.470000in}}%
\pgfpathlineto{\pgfqpoint{4.932040in}{2.170000in}}%
\pgfpathlineto{\pgfqpoint{4.933280in}{1.645000in}}%
\pgfpathlineto{\pgfqpoint{4.934520in}{2.275000in}}%
\pgfpathlineto{\pgfqpoint{4.935760in}{1.925000in}}%
\pgfpathlineto{\pgfqpoint{4.937000in}{2.065000in}}%
\pgfpathlineto{\pgfqpoint{4.939480in}{2.065000in}}%
\pgfpathlineto{\pgfqpoint{4.940720in}{1.575000in}}%
\pgfpathlineto{\pgfqpoint{4.943200in}{1.925000in}}%
\pgfpathlineto{\pgfqpoint{4.944440in}{1.750000in}}%
\pgfpathlineto{\pgfqpoint{4.945680in}{1.855000in}}%
\pgfpathlineto{\pgfqpoint{4.946920in}{1.470000in}}%
\pgfpathlineto{\pgfqpoint{4.948160in}{1.470000in}}%
\pgfpathlineto{\pgfqpoint{4.949400in}{1.995000in}}%
\pgfpathlineto{\pgfqpoint{4.950640in}{1.960000in}}%
\pgfpathlineto{\pgfqpoint{4.951880in}{1.960000in}}%
\pgfpathlineto{\pgfqpoint{4.954360in}{1.575000in}}%
\pgfpathlineto{\pgfqpoint{4.956840in}{1.785000in}}%
\pgfpathlineto{\pgfqpoint{4.958080in}{1.680000in}}%
\pgfpathlineto{\pgfqpoint{4.959320in}{1.995000in}}%
\pgfpathlineto{\pgfqpoint{4.961800in}{1.470000in}}%
\pgfpathlineto{\pgfqpoint{4.963040in}{1.400000in}}%
\pgfpathlineto{\pgfqpoint{4.966760in}{2.030000in}}%
\pgfpathlineto{\pgfqpoint{4.968000in}{1.995000in}}%
\pgfpathlineto{\pgfqpoint{4.969240in}{1.960000in}}%
\pgfpathlineto{\pgfqpoint{4.971720in}{2.100000in}}%
\pgfpathlineto{\pgfqpoint{4.972960in}{2.065000in}}%
\pgfpathlineto{\pgfqpoint{4.974200in}{1.750000in}}%
\pgfpathlineto{\pgfqpoint{4.976680in}{2.030000in}}%
\pgfpathlineto{\pgfqpoint{4.977920in}{1.645000in}}%
\pgfpathlineto{\pgfqpoint{4.979160in}{1.645000in}}%
\pgfpathlineto{\pgfqpoint{4.980400in}{2.275000in}}%
\pgfpathlineto{\pgfqpoint{4.981640in}{2.205000in}}%
\pgfpathlineto{\pgfqpoint{4.985360in}{1.505000in}}%
\pgfpathlineto{\pgfqpoint{4.989080in}{1.855000in}}%
\pgfpathlineto{\pgfqpoint{4.990320in}{1.610000in}}%
\pgfpathlineto{\pgfqpoint{4.994040in}{2.240000in}}%
\pgfpathlineto{\pgfqpoint{4.995280in}{1.820000in}}%
\pgfpathlineto{\pgfqpoint{4.996520in}{2.310000in}}%
\pgfpathlineto{\pgfqpoint{4.999000in}{1.960000in}}%
\pgfpathlineto{\pgfqpoint{5.000240in}{2.030000in}}%
\pgfpathlineto{\pgfqpoint{5.001480in}{1.820000in}}%
\pgfpathlineto{\pgfqpoint{5.002720in}{1.820000in}}%
\pgfpathlineto{\pgfqpoint{5.003960in}{2.100000in}}%
\pgfpathlineto{\pgfqpoint{5.005200in}{1.505000in}}%
\pgfpathlineto{\pgfqpoint{5.006440in}{1.505000in}}%
\pgfpathlineto{\pgfqpoint{5.008920in}{2.310000in}}%
\pgfpathlineto{\pgfqpoint{5.010160in}{2.310000in}}%
\pgfpathlineto{\pgfqpoint{5.011400in}{1.890000in}}%
\pgfpathlineto{\pgfqpoint{5.012640in}{2.100000in}}%
\pgfpathlineto{\pgfqpoint{5.013880in}{1.750000in}}%
\pgfpathlineto{\pgfqpoint{5.015120in}{2.030000in}}%
\pgfpathlineto{\pgfqpoint{5.016360in}{1.785000in}}%
\pgfpathlineto{\pgfqpoint{5.018840in}{2.135000in}}%
\pgfpathlineto{\pgfqpoint{5.020080in}{2.345000in}}%
\pgfpathlineto{\pgfqpoint{5.021320in}{1.610000in}}%
\pgfpathlineto{\pgfqpoint{5.022560in}{1.995000in}}%
\pgfpathlineto{\pgfqpoint{5.023800in}{1.890000in}}%
\pgfpathlineto{\pgfqpoint{5.026280in}{2.240000in}}%
\pgfpathlineto{\pgfqpoint{5.027520in}{2.205000in}}%
\pgfpathlineto{\pgfqpoint{5.028760in}{2.275000in}}%
\pgfpathlineto{\pgfqpoint{5.030000in}{1.855000in}}%
\pgfpathlineto{\pgfqpoint{5.031240in}{1.820000in}}%
\pgfpathlineto{\pgfqpoint{5.032480in}{1.925000in}}%
\pgfpathlineto{\pgfqpoint{5.033720in}{1.925000in}}%
\pgfpathlineto{\pgfqpoint{5.034960in}{1.960000in}}%
\pgfpathlineto{\pgfqpoint{5.037440in}{1.715000in}}%
\pgfpathlineto{\pgfqpoint{5.038680in}{2.065000in}}%
\pgfpathlineto{\pgfqpoint{5.039920in}{1.890000in}}%
\pgfpathlineto{\pgfqpoint{5.041160in}{2.135000in}}%
\pgfpathlineto{\pgfqpoint{5.042400in}{1.855000in}}%
\pgfpathlineto{\pgfqpoint{5.043640in}{1.890000in}}%
\pgfpathlineto{\pgfqpoint{5.044880in}{2.030000in}}%
\pgfpathlineto{\pgfqpoint{5.046120in}{1.680000in}}%
\pgfpathlineto{\pgfqpoint{5.048600in}{2.135000in}}%
\pgfpathlineto{\pgfqpoint{5.049840in}{2.170000in}}%
\pgfpathlineto{\pgfqpoint{5.051080in}{1.645000in}}%
\pgfpathlineto{\pgfqpoint{5.052320in}{2.170000in}}%
\pgfpathlineto{\pgfqpoint{5.053560in}{1.785000in}}%
\pgfpathlineto{\pgfqpoint{5.054800in}{2.275000in}}%
\pgfpathlineto{\pgfqpoint{5.056040in}{1.435000in}}%
\pgfpathlineto{\pgfqpoint{5.057280in}{1.925000in}}%
\pgfpathlineto{\pgfqpoint{5.058520in}{1.890000in}}%
\pgfpathlineto{\pgfqpoint{5.059760in}{2.240000in}}%
\pgfpathlineto{\pgfqpoint{5.061000in}{1.960000in}}%
\pgfpathlineto{\pgfqpoint{5.062240in}{2.240000in}}%
\pgfpathlineto{\pgfqpoint{5.063480in}{1.995000in}}%
\pgfpathlineto{\pgfqpoint{5.064720in}{2.275000in}}%
\pgfpathlineto{\pgfqpoint{5.065960in}{1.575000in}}%
\pgfpathlineto{\pgfqpoint{5.068440in}{2.170000in}}%
\pgfpathlineto{\pgfqpoint{5.072160in}{1.645000in}}%
\pgfpathlineto{\pgfqpoint{5.073400in}{1.925000in}}%
\pgfpathlineto{\pgfqpoint{5.074640in}{1.750000in}}%
\pgfpathlineto{\pgfqpoint{5.077120in}{2.205000in}}%
\pgfpathlineto{\pgfqpoint{5.078360in}{2.240000in}}%
\pgfpathlineto{\pgfqpoint{5.080840in}{1.820000in}}%
\pgfpathlineto{\pgfqpoint{5.082080in}{2.135000in}}%
\pgfpathlineto{\pgfqpoint{5.083320in}{1.820000in}}%
\pgfpathlineto{\pgfqpoint{5.084560in}{1.995000in}}%
\pgfpathlineto{\pgfqpoint{5.085800in}{1.575000in}}%
\pgfpathlineto{\pgfqpoint{5.087040in}{2.170000in}}%
\pgfpathlineto{\pgfqpoint{5.088280in}{2.135000in}}%
\pgfpathlineto{\pgfqpoint{5.089520in}{1.750000in}}%
\pgfpathlineto{\pgfqpoint{5.090760in}{2.065000in}}%
\pgfpathlineto{\pgfqpoint{5.092000in}{1.610000in}}%
\pgfpathlineto{\pgfqpoint{5.093240in}{1.925000in}}%
\pgfpathlineto{\pgfqpoint{5.094480in}{1.750000in}}%
\pgfpathlineto{\pgfqpoint{5.095720in}{1.890000in}}%
\pgfpathlineto{\pgfqpoint{5.096960in}{2.310000in}}%
\pgfpathlineto{\pgfqpoint{5.098200in}{1.715000in}}%
\pgfpathlineto{\pgfqpoint{5.099440in}{2.205000in}}%
\pgfpathlineto{\pgfqpoint{5.100680in}{1.820000in}}%
\pgfpathlineto{\pgfqpoint{5.101920in}{1.960000in}}%
\pgfpathlineto{\pgfqpoint{5.104400in}{1.645000in}}%
\pgfpathlineto{\pgfqpoint{5.106880in}{2.205000in}}%
\pgfpathlineto{\pgfqpoint{5.110600in}{1.715000in}}%
\pgfpathlineto{\pgfqpoint{5.113080in}{2.030000in}}%
\pgfpathlineto{\pgfqpoint{5.114320in}{1.925000in}}%
\pgfpathlineto{\pgfqpoint{5.116800in}{2.065000in}}%
\pgfpathlineto{\pgfqpoint{5.119280in}{1.855000in}}%
\pgfpathlineto{\pgfqpoint{5.120520in}{2.275000in}}%
\pgfpathlineto{\pgfqpoint{5.123000in}{2.065000in}}%
\pgfpathlineto{\pgfqpoint{5.124240in}{1.820000in}}%
\pgfpathlineto{\pgfqpoint{5.127960in}{2.170000in}}%
\pgfpathlineto{\pgfqpoint{5.129200in}{2.135000in}}%
\pgfpathlineto{\pgfqpoint{5.130440in}{1.855000in}}%
\pgfpathlineto{\pgfqpoint{5.131680in}{2.380000in}}%
\pgfpathlineto{\pgfqpoint{5.132920in}{1.960000in}}%
\pgfpathlineto{\pgfqpoint{5.134160in}{1.925000in}}%
\pgfpathlineto{\pgfqpoint{5.136640in}{2.380000in}}%
\pgfpathlineto{\pgfqpoint{5.139120in}{2.100000in}}%
\pgfpathlineto{\pgfqpoint{5.141600in}{1.925000in}}%
\pgfpathlineto{\pgfqpoint{5.144080in}{2.275000in}}%
\pgfpathlineto{\pgfqpoint{5.145320in}{2.100000in}}%
\pgfpathlineto{\pgfqpoint{5.147800in}{2.345000in}}%
\pgfpathlineto{\pgfqpoint{5.150280in}{1.890000in}}%
\pgfpathlineto{\pgfqpoint{5.151520in}{1.715000in}}%
\pgfpathlineto{\pgfqpoint{5.152760in}{2.065000in}}%
\pgfpathlineto{\pgfqpoint{5.155240in}{1.750000in}}%
\pgfpathlineto{\pgfqpoint{5.157720in}{1.925000in}}%
\pgfpathlineto{\pgfqpoint{5.158960in}{1.575000in}}%
\pgfpathlineto{\pgfqpoint{5.160200in}{1.645000in}}%
\pgfpathlineto{\pgfqpoint{5.161440in}{1.470000in}}%
\pgfpathlineto{\pgfqpoint{5.162680in}{1.960000in}}%
\pgfpathlineto{\pgfqpoint{5.163920in}{1.750000in}}%
\pgfpathlineto{\pgfqpoint{5.165160in}{1.785000in}}%
\pgfpathlineto{\pgfqpoint{5.166400in}{1.995000in}}%
\pgfpathlineto{\pgfqpoint{5.167640in}{1.960000in}}%
\pgfpathlineto{\pgfqpoint{5.170120in}{1.750000in}}%
\pgfpathlineto{\pgfqpoint{5.171360in}{1.890000in}}%
\pgfpathlineto{\pgfqpoint{5.172600in}{1.785000in}}%
\pgfpathlineto{\pgfqpoint{5.173840in}{1.855000in}}%
\pgfpathlineto{\pgfqpoint{5.175080in}{1.820000in}}%
\pgfpathlineto{\pgfqpoint{5.176320in}{1.820000in}}%
\pgfpathlineto{\pgfqpoint{5.177560in}{1.890000in}}%
\pgfpathlineto{\pgfqpoint{5.181280in}{1.645000in}}%
\pgfpathlineto{\pgfqpoint{5.183760in}{1.925000in}}%
\pgfpathlineto{\pgfqpoint{5.185000in}{1.890000in}}%
\pgfpathlineto{\pgfqpoint{5.186240in}{2.240000in}}%
\pgfpathlineto{\pgfqpoint{5.187480in}{1.820000in}}%
\pgfpathlineto{\pgfqpoint{5.188720in}{2.100000in}}%
\pgfpathlineto{\pgfqpoint{5.189960in}{1.295000in}}%
\pgfpathlineto{\pgfqpoint{5.191200in}{1.750000in}}%
\pgfpathlineto{\pgfqpoint{5.192440in}{1.645000in}}%
\pgfpathlineto{\pgfqpoint{5.194920in}{1.960000in}}%
\pgfpathlineto{\pgfqpoint{5.196160in}{1.610000in}}%
\pgfpathlineto{\pgfqpoint{5.199880in}{1.995000in}}%
\pgfpathlineto{\pgfqpoint{5.201120in}{1.785000in}}%
\pgfpathlineto{\pgfqpoint{5.202360in}{1.820000in}}%
\pgfpathlineto{\pgfqpoint{5.203600in}{1.820000in}}%
\pgfpathlineto{\pgfqpoint{5.204840in}{2.170000in}}%
\pgfpathlineto{\pgfqpoint{5.206080in}{2.065000in}}%
\pgfpathlineto{\pgfqpoint{5.207320in}{1.820000in}}%
\pgfpathlineto{\pgfqpoint{5.208560in}{1.925000in}}%
\pgfpathlineto{\pgfqpoint{5.209800in}{1.855000in}}%
\pgfpathlineto{\pgfqpoint{5.211040in}{2.240000in}}%
\pgfpathlineto{\pgfqpoint{5.214760in}{1.680000in}}%
\pgfpathlineto{\pgfqpoint{5.216000in}{1.715000in}}%
\pgfpathlineto{\pgfqpoint{5.217240in}{1.785000in}}%
\pgfpathlineto{\pgfqpoint{5.218480in}{1.680000in}}%
\pgfpathlineto{\pgfqpoint{5.219720in}{1.715000in}}%
\pgfpathlineto{\pgfqpoint{5.220960in}{1.715000in}}%
\pgfpathlineto{\pgfqpoint{5.222200in}{1.855000in}}%
\pgfpathlineto{\pgfqpoint{5.223440in}{1.820000in}}%
\pgfpathlineto{\pgfqpoint{5.224680in}{1.855000in}}%
\pgfpathlineto{\pgfqpoint{5.225920in}{1.365000in}}%
\pgfpathlineto{\pgfqpoint{5.227160in}{1.890000in}}%
\pgfpathlineto{\pgfqpoint{5.228400in}{1.505000in}}%
\pgfpathlineto{\pgfqpoint{5.229640in}{1.715000in}}%
\pgfpathlineto{\pgfqpoint{5.230880in}{1.680000in}}%
\pgfpathlineto{\pgfqpoint{5.232120in}{1.890000in}}%
\pgfpathlineto{\pgfqpoint{5.233360in}{1.645000in}}%
\pgfpathlineto{\pgfqpoint{5.234600in}{2.030000in}}%
\pgfpathlineto{\pgfqpoint{5.235840in}{1.925000in}}%
\pgfpathlineto{\pgfqpoint{5.237080in}{2.170000in}}%
\pgfpathlineto{\pgfqpoint{5.238320in}{1.855000in}}%
\pgfpathlineto{\pgfqpoint{5.239560in}{2.275000in}}%
\pgfpathlineto{\pgfqpoint{5.240800in}{1.960000in}}%
\pgfpathlineto{\pgfqpoint{5.243280in}{2.240000in}}%
\pgfpathlineto{\pgfqpoint{5.244520in}{2.030000in}}%
\pgfpathlineto{\pgfqpoint{5.245760in}{2.590000in}}%
\pgfpathlineto{\pgfqpoint{5.247000in}{1.995000in}}%
\pgfpathlineto{\pgfqpoint{5.248240in}{1.960000in}}%
\pgfpathlineto{\pgfqpoint{5.249480in}{2.310000in}}%
\pgfpathlineto{\pgfqpoint{5.251960in}{2.030000in}}%
\pgfpathlineto{\pgfqpoint{5.253200in}{2.065000in}}%
\pgfpathlineto{\pgfqpoint{5.255680in}{2.450000in}}%
\pgfpathlineto{\pgfqpoint{5.256920in}{1.680000in}}%
\pgfpathlineto{\pgfqpoint{5.258160in}{1.680000in}}%
\pgfpathlineto{\pgfqpoint{5.259400in}{1.750000in}}%
\pgfpathlineto{\pgfqpoint{5.260640in}{2.275000in}}%
\pgfpathlineto{\pgfqpoint{5.263120in}{1.750000in}}%
\pgfpathlineto{\pgfqpoint{5.268080in}{2.555000in}}%
\pgfpathlineto{\pgfqpoint{5.270560in}{2.240000in}}%
\pgfpathlineto{\pgfqpoint{5.271800in}{2.275000in}}%
\pgfpathlineto{\pgfqpoint{5.273040in}{2.170000in}}%
\pgfpathlineto{\pgfqpoint{5.274280in}{1.855000in}}%
\pgfpathlineto{\pgfqpoint{5.276760in}{2.240000in}}%
\pgfpathlineto{\pgfqpoint{5.278000in}{1.960000in}}%
\pgfpathlineto{\pgfqpoint{5.279240in}{2.065000in}}%
\pgfpathlineto{\pgfqpoint{5.280480in}{1.995000in}}%
\pgfpathlineto{\pgfqpoint{5.281720in}{1.715000in}}%
\pgfpathlineto{\pgfqpoint{5.284200in}{2.030000in}}%
\pgfpathlineto{\pgfqpoint{5.285440in}{1.785000in}}%
\pgfpathlineto{\pgfqpoint{5.286680in}{1.960000in}}%
\pgfpathlineto{\pgfqpoint{5.289160in}{1.645000in}}%
\pgfpathlineto{\pgfqpoint{5.290400in}{2.345000in}}%
\pgfpathlineto{\pgfqpoint{5.291640in}{1.855000in}}%
\pgfpathlineto{\pgfqpoint{5.292880in}{1.820000in}}%
\pgfpathlineto{\pgfqpoint{5.294120in}{1.715000in}}%
\pgfpathlineto{\pgfqpoint{5.295360in}{1.785000in}}%
\pgfpathlineto{\pgfqpoint{5.296600in}{1.750000in}}%
\pgfpathlineto{\pgfqpoint{5.297840in}{1.750000in}}%
\pgfpathlineto{\pgfqpoint{5.299080in}{1.540000in}}%
\pgfpathlineto{\pgfqpoint{5.301560in}{1.925000in}}%
\pgfpathlineto{\pgfqpoint{5.302800in}{1.470000in}}%
\pgfpathlineto{\pgfqpoint{5.305280in}{1.575000in}}%
\pgfpathlineto{\pgfqpoint{5.309000in}{2.380000in}}%
\pgfpathlineto{\pgfqpoint{5.311480in}{1.960000in}}%
\pgfpathlineto{\pgfqpoint{5.312720in}{2.100000in}}%
\pgfpathlineto{\pgfqpoint{5.313960in}{2.065000in}}%
\pgfpathlineto{\pgfqpoint{5.315200in}{2.100000in}}%
\pgfpathlineto{\pgfqpoint{5.316440in}{2.380000in}}%
\pgfpathlineto{\pgfqpoint{5.320160in}{1.400000in}}%
\pgfpathlineto{\pgfqpoint{5.321400in}{2.415000in}}%
\pgfpathlineto{\pgfqpoint{5.325120in}{1.680000in}}%
\pgfpathlineto{\pgfqpoint{5.327600in}{1.925000in}}%
\pgfpathlineto{\pgfqpoint{5.330080in}{2.100000in}}%
\pgfpathlineto{\pgfqpoint{5.331320in}{1.435000in}}%
\pgfpathlineto{\pgfqpoint{5.332560in}{1.890000in}}%
\pgfpathlineto{\pgfqpoint{5.333800in}{1.750000in}}%
\pgfpathlineto{\pgfqpoint{5.335040in}{2.065000in}}%
\pgfpathlineto{\pgfqpoint{5.336280in}{1.925000in}}%
\pgfpathlineto{\pgfqpoint{5.338760in}{2.135000in}}%
\pgfpathlineto{\pgfqpoint{5.342480in}{1.785000in}}%
\pgfpathlineto{\pgfqpoint{5.343720in}{2.275000in}}%
\pgfpathlineto{\pgfqpoint{5.344960in}{1.785000in}}%
\pgfpathlineto{\pgfqpoint{5.346200in}{1.890000in}}%
\pgfpathlineto{\pgfqpoint{5.347440in}{1.330000in}}%
\pgfpathlineto{\pgfqpoint{5.349920in}{2.275000in}}%
\pgfpathlineto{\pgfqpoint{5.351160in}{2.170000in}}%
\pgfpathlineto{\pgfqpoint{5.353640in}{1.855000in}}%
\pgfpathlineto{\pgfqpoint{5.354880in}{2.310000in}}%
\pgfpathlineto{\pgfqpoint{5.357360in}{1.820000in}}%
\pgfpathlineto{\pgfqpoint{5.358600in}{2.450000in}}%
\pgfpathlineto{\pgfqpoint{5.359840in}{1.610000in}}%
\pgfpathlineto{\pgfqpoint{5.361080in}{1.540000in}}%
\pgfpathlineto{\pgfqpoint{5.362320in}{1.820000in}}%
\pgfpathlineto{\pgfqpoint{5.363560in}{1.785000in}}%
\pgfpathlineto{\pgfqpoint{5.364800in}{1.890000in}}%
\pgfpathlineto{\pgfqpoint{5.366040in}{1.610000in}}%
\pgfpathlineto{\pgfqpoint{5.368520in}{2.170000in}}%
\pgfpathlineto{\pgfqpoint{5.371000in}{1.715000in}}%
\pgfpathlineto{\pgfqpoint{5.373480in}{1.960000in}}%
\pgfpathlineto{\pgfqpoint{5.374720in}{2.030000in}}%
\pgfpathlineto{\pgfqpoint{5.377200in}{1.715000in}}%
\pgfpathlineto{\pgfqpoint{5.378440in}{1.785000in}}%
\pgfpathlineto{\pgfqpoint{5.379680in}{1.715000in}}%
\pgfpathlineto{\pgfqpoint{5.380920in}{1.785000in}}%
\pgfpathlineto{\pgfqpoint{5.382160in}{1.575000in}}%
\pgfpathlineto{\pgfqpoint{5.384640in}{2.240000in}}%
\pgfpathlineto{\pgfqpoint{5.385880in}{2.275000in}}%
\pgfpathlineto{\pgfqpoint{5.387120in}{2.170000in}}%
\pgfpathlineto{\pgfqpoint{5.388360in}{1.855000in}}%
\pgfpathlineto{\pgfqpoint{5.389600in}{2.555000in}}%
\pgfpathlineto{\pgfqpoint{5.390840in}{2.450000in}}%
\pgfpathlineto{\pgfqpoint{5.392080in}{2.450000in}}%
\pgfpathlineto{\pgfqpoint{5.393320in}{2.345000in}}%
\pgfpathlineto{\pgfqpoint{5.394560in}{1.925000in}}%
\pgfpathlineto{\pgfqpoint{5.395800in}{2.100000in}}%
\pgfpathlineto{\pgfqpoint{5.398280in}{1.715000in}}%
\pgfpathlineto{\pgfqpoint{5.400760in}{2.205000in}}%
\pgfpathlineto{\pgfqpoint{5.402000in}{1.820000in}}%
\pgfpathlineto{\pgfqpoint{5.405720in}{2.030000in}}%
\pgfpathlineto{\pgfqpoint{5.408200in}{1.820000in}}%
\pgfpathlineto{\pgfqpoint{5.409440in}{1.820000in}}%
\pgfpathlineto{\pgfqpoint{5.410680in}{1.505000in}}%
\pgfpathlineto{\pgfqpoint{5.411920in}{2.170000in}}%
\pgfpathlineto{\pgfqpoint{5.413160in}{2.170000in}}%
\pgfpathlineto{\pgfqpoint{5.414400in}{1.715000in}}%
\pgfpathlineto{\pgfqpoint{5.415640in}{1.680000in}}%
\pgfpathlineto{\pgfqpoint{5.416880in}{1.610000in}}%
\pgfpathlineto{\pgfqpoint{5.418120in}{1.715000in}}%
\pgfpathlineto{\pgfqpoint{5.419360in}{1.715000in}}%
\pgfpathlineto{\pgfqpoint{5.420600in}{1.470000in}}%
\pgfpathlineto{\pgfqpoint{5.423080in}{2.135000in}}%
\pgfpathlineto{\pgfqpoint{5.424320in}{1.925000in}}%
\pgfpathlineto{\pgfqpoint{5.425560in}{2.100000in}}%
\pgfpathlineto{\pgfqpoint{5.426800in}{1.960000in}}%
\pgfpathlineto{\pgfqpoint{5.428040in}{2.030000in}}%
\pgfpathlineto{\pgfqpoint{5.429280in}{1.890000in}}%
\pgfpathlineto{\pgfqpoint{5.430520in}{1.890000in}}%
\pgfpathlineto{\pgfqpoint{5.431760in}{2.240000in}}%
\pgfpathlineto{\pgfqpoint{5.434240in}{1.890000in}}%
\pgfpathlineto{\pgfqpoint{5.435480in}{2.205000in}}%
\pgfpathlineto{\pgfqpoint{5.436720in}{2.100000in}}%
\pgfpathlineto{\pgfqpoint{5.437960in}{2.240000in}}%
\pgfpathlineto{\pgfqpoint{5.439200in}{1.995000in}}%
\pgfpathlineto{\pgfqpoint{5.440440in}{2.450000in}}%
\pgfpathlineto{\pgfqpoint{5.442920in}{1.995000in}}%
\pgfpathlineto{\pgfqpoint{5.444160in}{2.345000in}}%
\pgfpathlineto{\pgfqpoint{5.445400in}{1.890000in}}%
\pgfpathlineto{\pgfqpoint{5.446640in}{2.065000in}}%
\pgfpathlineto{\pgfqpoint{5.447880in}{1.820000in}}%
\pgfpathlineto{\pgfqpoint{5.449120in}{1.995000in}}%
\pgfpathlineto{\pgfqpoint{5.450360in}{1.680000in}}%
\pgfpathlineto{\pgfqpoint{5.451600in}{1.680000in}}%
\pgfpathlineto{\pgfqpoint{5.454080in}{2.100000in}}%
\pgfpathlineto{\pgfqpoint{5.459040in}{1.645000in}}%
\pgfpathlineto{\pgfqpoint{5.460280in}{2.065000in}}%
\pgfpathlineto{\pgfqpoint{5.461520in}{2.065000in}}%
\pgfpathlineto{\pgfqpoint{5.462760in}{1.995000in}}%
\pgfpathlineto{\pgfqpoint{5.465240in}{1.610000in}}%
\pgfpathlineto{\pgfqpoint{5.467720in}{1.470000in}}%
\pgfpathlineto{\pgfqpoint{5.468960in}{1.995000in}}%
\pgfpathlineto{\pgfqpoint{5.470200in}{1.645000in}}%
\pgfpathlineto{\pgfqpoint{5.472680in}{1.820000in}}%
\pgfpathlineto{\pgfqpoint{5.473920in}{2.205000in}}%
\pgfpathlineto{\pgfqpoint{5.475160in}{1.820000in}}%
\pgfpathlineto{\pgfqpoint{5.477640in}{2.275000in}}%
\pgfpathlineto{\pgfqpoint{5.478880in}{1.925000in}}%
\pgfpathlineto{\pgfqpoint{5.480120in}{1.960000in}}%
\pgfpathlineto{\pgfqpoint{5.481360in}{1.435000in}}%
\pgfpathlineto{\pgfqpoint{5.482600in}{2.170000in}}%
\pgfpathlineto{\pgfqpoint{5.483840in}{1.820000in}}%
\pgfpathlineto{\pgfqpoint{5.485080in}{2.100000in}}%
\pgfpathlineto{\pgfqpoint{5.487560in}{1.470000in}}%
\pgfpathlineto{\pgfqpoint{5.488800in}{1.645000in}}%
\pgfpathlineto{\pgfqpoint{5.490040in}{1.365000in}}%
\pgfpathlineto{\pgfqpoint{5.491280in}{2.135000in}}%
\pgfpathlineto{\pgfqpoint{5.493760in}{1.435000in}}%
\pgfpathlineto{\pgfqpoint{5.495000in}{1.820000in}}%
\pgfpathlineto{\pgfqpoint{5.496240in}{1.680000in}}%
\pgfpathlineto{\pgfqpoint{5.497480in}{2.135000in}}%
\pgfpathlineto{\pgfqpoint{5.498720in}{1.715000in}}%
\pgfpathlineto{\pgfqpoint{5.499960in}{2.205000in}}%
\pgfpathlineto{\pgfqpoint{5.501200in}{2.135000in}}%
\pgfpathlineto{\pgfqpoint{5.502440in}{1.680000in}}%
\pgfpathlineto{\pgfqpoint{5.503680in}{1.715000in}}%
\pgfpathlineto{\pgfqpoint{5.504920in}{1.785000in}}%
\pgfpathlineto{\pgfqpoint{5.506160in}{1.435000in}}%
\pgfpathlineto{\pgfqpoint{5.507400in}{1.785000in}}%
\pgfpathlineto{\pgfqpoint{5.508640in}{1.645000in}}%
\pgfpathlineto{\pgfqpoint{5.509880in}{1.680000in}}%
\pgfpathlineto{\pgfqpoint{5.511120in}{1.540000in}}%
\pgfpathlineto{\pgfqpoint{5.513600in}{1.960000in}}%
\pgfpathlineto{\pgfqpoint{5.514840in}{1.785000in}}%
\pgfpathlineto{\pgfqpoint{5.516080in}{2.135000in}}%
\pgfpathlineto{\pgfqpoint{5.517320in}{1.785000in}}%
\pgfpathlineto{\pgfqpoint{5.518560in}{2.170000in}}%
\pgfpathlineto{\pgfqpoint{5.519800in}{1.435000in}}%
\pgfpathlineto{\pgfqpoint{5.521040in}{2.135000in}}%
\pgfpathlineto{\pgfqpoint{5.522280in}{2.065000in}}%
\pgfpathlineto{\pgfqpoint{5.523520in}{1.925000in}}%
\pgfpathlineto{\pgfqpoint{5.526000in}{1.330000in}}%
\pgfpathlineto{\pgfqpoint{5.527240in}{2.135000in}}%
\pgfpathlineto{\pgfqpoint{5.528480in}{1.750000in}}%
\pgfpathlineto{\pgfqpoint{5.529720in}{1.715000in}}%
\pgfpathlineto{\pgfqpoint{5.530960in}{1.610000in}}%
\pgfpathlineto{\pgfqpoint{5.532200in}{1.960000in}}%
\pgfpathlineto{\pgfqpoint{5.533440in}{1.925000in}}%
\pgfpathlineto{\pgfqpoint{5.534680in}{1.715000in}}%
\pgfpathlineto{\pgfqpoint{5.537160in}{2.205000in}}%
\pgfpathlineto{\pgfqpoint{5.539640in}{1.890000in}}%
\pgfpathlineto{\pgfqpoint{5.540880in}{1.960000in}}%
\pgfpathlineto{\pgfqpoint{5.542120in}{1.645000in}}%
\pgfpathlineto{\pgfqpoint{5.543360in}{2.100000in}}%
\pgfpathlineto{\pgfqpoint{5.544600in}{2.135000in}}%
\pgfpathlineto{\pgfqpoint{5.547080in}{1.610000in}}%
\pgfpathlineto{\pgfqpoint{5.548320in}{2.310000in}}%
\pgfpathlineto{\pgfqpoint{5.549560in}{2.240000in}}%
\pgfpathlineto{\pgfqpoint{5.550800in}{1.960000in}}%
\pgfpathlineto{\pgfqpoint{5.552040in}{1.995000in}}%
\pgfpathlineto{\pgfqpoint{5.553280in}{2.135000in}}%
\pgfpathlineto{\pgfqpoint{5.554520in}{1.610000in}}%
\pgfpathlineto{\pgfqpoint{5.555760in}{1.925000in}}%
\pgfpathlineto{\pgfqpoint{5.557000in}{1.575000in}}%
\pgfpathlineto{\pgfqpoint{5.558240in}{1.820000in}}%
\pgfpathlineto{\pgfqpoint{5.559480in}{1.470000in}}%
\pgfpathlineto{\pgfqpoint{5.561960in}{2.240000in}}%
\pgfpathlineto{\pgfqpoint{5.563200in}{1.960000in}}%
\pgfpathlineto{\pgfqpoint{5.564440in}{2.205000in}}%
\pgfpathlineto{\pgfqpoint{5.565680in}{1.540000in}}%
\pgfpathlineto{\pgfqpoint{5.569400in}{2.135000in}}%
\pgfpathlineto{\pgfqpoint{5.570640in}{1.820000in}}%
\pgfpathlineto{\pgfqpoint{5.571880in}{1.925000in}}%
\pgfpathlineto{\pgfqpoint{5.574360in}{2.240000in}}%
\pgfpathlineto{\pgfqpoint{5.575600in}{2.415000in}}%
\pgfpathlineto{\pgfqpoint{5.576840in}{2.030000in}}%
\pgfpathlineto{\pgfqpoint{5.579320in}{2.485000in}}%
\pgfpathlineto{\pgfqpoint{5.581800in}{1.855000in}}%
\pgfpathlineto{\pgfqpoint{5.583040in}{1.785000in}}%
\pgfpathlineto{\pgfqpoint{5.584280in}{2.135000in}}%
\pgfpathlineto{\pgfqpoint{5.586760in}{1.820000in}}%
\pgfpathlineto{\pgfqpoint{5.589240in}{2.170000in}}%
\pgfpathlineto{\pgfqpoint{5.590480in}{1.645000in}}%
\pgfpathlineto{\pgfqpoint{5.591720in}{2.100000in}}%
\pgfpathlineto{\pgfqpoint{5.594200in}{1.960000in}}%
\pgfpathlineto{\pgfqpoint{5.595440in}{2.100000in}}%
\pgfpathlineto{\pgfqpoint{5.596680in}{1.750000in}}%
\pgfpathlineto{\pgfqpoint{5.597920in}{1.925000in}}%
\pgfpathlineto{\pgfqpoint{5.599160in}{2.310000in}}%
\pgfpathlineto{\pgfqpoint{5.601640in}{2.065000in}}%
\pgfpathlineto{\pgfqpoint{5.602880in}{1.995000in}}%
\pgfpathlineto{\pgfqpoint{5.604120in}{2.170000in}}%
\pgfpathlineto{\pgfqpoint{5.605360in}{1.890000in}}%
\pgfpathlineto{\pgfqpoint{5.606600in}{1.890000in}}%
\pgfpathlineto{\pgfqpoint{5.607840in}{1.680000in}}%
\pgfpathlineto{\pgfqpoint{5.609080in}{2.030000in}}%
\pgfpathlineto{\pgfqpoint{5.611560in}{1.855000in}}%
\pgfpathlineto{\pgfqpoint{5.612800in}{1.890000in}}%
\pgfpathlineto{\pgfqpoint{5.614040in}{1.960000in}}%
\pgfpathlineto{\pgfqpoint{5.615280in}{1.925000in}}%
\pgfpathlineto{\pgfqpoint{5.616520in}{2.240000in}}%
\pgfpathlineto{\pgfqpoint{5.619000in}{1.855000in}}%
\pgfpathlineto{\pgfqpoint{5.621480in}{2.170000in}}%
\pgfpathlineto{\pgfqpoint{5.622720in}{1.890000in}}%
\pgfpathlineto{\pgfqpoint{5.623960in}{2.170000in}}%
\pgfpathlineto{\pgfqpoint{5.625200in}{2.135000in}}%
\pgfpathlineto{\pgfqpoint{5.626440in}{1.575000in}}%
\pgfpathlineto{\pgfqpoint{5.627680in}{2.065000in}}%
\pgfpathlineto{\pgfqpoint{5.628920in}{1.890000in}}%
\pgfpathlineto{\pgfqpoint{5.630160in}{2.135000in}}%
\pgfpathlineto{\pgfqpoint{5.631400in}{1.610000in}}%
\pgfpathlineto{\pgfqpoint{5.632640in}{1.680000in}}%
\pgfpathlineto{\pgfqpoint{5.635120in}{2.135000in}}%
\pgfpathlineto{\pgfqpoint{5.637600in}{1.540000in}}%
\pgfpathlineto{\pgfqpoint{5.640080in}{1.995000in}}%
\pgfpathlineto{\pgfqpoint{5.641320in}{1.960000in}}%
\pgfpathlineto{\pgfqpoint{5.642560in}{1.610000in}}%
\pgfpathlineto{\pgfqpoint{5.643800in}{1.680000in}}%
\pgfpathlineto{\pgfqpoint{5.645040in}{1.995000in}}%
\pgfpathlineto{\pgfqpoint{5.646280in}{1.435000in}}%
\pgfpathlineto{\pgfqpoint{5.648760in}{2.205000in}}%
\pgfpathlineto{\pgfqpoint{5.650000in}{2.170000in}}%
\pgfpathlineto{\pgfqpoint{5.651240in}{2.310000in}}%
\pgfpathlineto{\pgfqpoint{5.653720in}{1.925000in}}%
\pgfpathlineto{\pgfqpoint{5.654960in}{2.275000in}}%
\pgfpathlineto{\pgfqpoint{5.656200in}{2.135000in}}%
\pgfpathlineto{\pgfqpoint{5.657440in}{2.555000in}}%
\pgfpathlineto{\pgfqpoint{5.658680in}{2.030000in}}%
\pgfpathlineto{\pgfqpoint{5.659920in}{1.995000in}}%
\pgfpathlineto{\pgfqpoint{5.663640in}{1.540000in}}%
\pgfpathlineto{\pgfqpoint{5.666120in}{2.170000in}}%
\pgfpathlineto{\pgfqpoint{5.668600in}{1.645000in}}%
\pgfpathlineto{\pgfqpoint{5.669840in}{1.820000in}}%
\pgfpathlineto{\pgfqpoint{5.671080in}{1.820000in}}%
\pgfpathlineto{\pgfqpoint{5.672320in}{2.030000in}}%
\pgfpathlineto{\pgfqpoint{5.673560in}{1.645000in}}%
\pgfpathlineto{\pgfqpoint{5.674800in}{1.995000in}}%
\pgfpathlineto{\pgfqpoint{5.676040in}{1.750000in}}%
\pgfpathlineto{\pgfqpoint{5.677280in}{1.785000in}}%
\pgfpathlineto{\pgfqpoint{5.678520in}{1.470000in}}%
\pgfpathlineto{\pgfqpoint{5.681000in}{1.820000in}}%
\pgfpathlineto{\pgfqpoint{5.682240in}{1.645000in}}%
\pgfpathlineto{\pgfqpoint{5.685960in}{2.135000in}}%
\pgfpathlineto{\pgfqpoint{5.687200in}{1.995000in}}%
\pgfpathlineto{\pgfqpoint{5.688440in}{1.400000in}}%
\pgfpathlineto{\pgfqpoint{5.689680in}{1.925000in}}%
\pgfpathlineto{\pgfqpoint{5.690920in}{1.225000in}}%
\pgfpathlineto{\pgfqpoint{5.692160in}{1.960000in}}%
\pgfpathlineto{\pgfqpoint{5.693400in}{1.855000in}}%
\pgfpathlineto{\pgfqpoint{5.694640in}{1.855000in}}%
\pgfpathlineto{\pgfqpoint{5.697120in}{2.345000in}}%
\pgfpathlineto{\pgfqpoint{5.699600in}{1.435000in}}%
\pgfpathlineto{\pgfqpoint{5.700840in}{1.855000in}}%
\pgfpathlineto{\pgfqpoint{5.702080in}{1.680000in}}%
\pgfpathlineto{\pgfqpoint{5.703320in}{1.820000in}}%
\pgfpathlineto{\pgfqpoint{5.704560in}{1.645000in}}%
\pgfpathlineto{\pgfqpoint{5.705800in}{1.750000in}}%
\pgfpathlineto{\pgfqpoint{5.707040in}{2.065000in}}%
\pgfpathlineto{\pgfqpoint{5.708280in}{1.890000in}}%
\pgfpathlineto{\pgfqpoint{5.709520in}{1.890000in}}%
\pgfpathlineto{\pgfqpoint{5.710760in}{1.750000in}}%
\pgfpathlineto{\pgfqpoint{5.713240in}{2.100000in}}%
\pgfpathlineto{\pgfqpoint{5.714480in}{1.365000in}}%
\pgfpathlineto{\pgfqpoint{5.715720in}{1.855000in}}%
\pgfpathlineto{\pgfqpoint{5.716960in}{1.505000in}}%
\pgfpathlineto{\pgfqpoint{5.718200in}{1.680000in}}%
\pgfpathlineto{\pgfqpoint{5.719440in}{2.345000in}}%
\pgfpathlineto{\pgfqpoint{5.721920in}{1.330000in}}%
\pgfpathlineto{\pgfqpoint{5.723160in}{1.645000in}}%
\pgfpathlineto{\pgfqpoint{5.724400in}{1.330000in}}%
\pgfpathlineto{\pgfqpoint{5.726880in}{1.890000in}}%
\pgfpathlineto{\pgfqpoint{5.728120in}{1.785000in}}%
\pgfpathlineto{\pgfqpoint{5.729360in}{1.820000in}}%
\pgfpathlineto{\pgfqpoint{5.730600in}{1.645000in}}%
\pgfpathlineto{\pgfqpoint{5.731840in}{1.715000in}}%
\pgfpathlineto{\pgfqpoint{5.733080in}{1.890000in}}%
\pgfpathlineto{\pgfqpoint{5.734320in}{1.715000in}}%
\pgfpathlineto{\pgfqpoint{5.736800in}{2.310000in}}%
\pgfpathlineto{\pgfqpoint{5.738040in}{1.960000in}}%
\pgfpathlineto{\pgfqpoint{5.739280in}{2.065000in}}%
\pgfpathlineto{\pgfqpoint{5.741760in}{1.750000in}}%
\pgfpathlineto{\pgfqpoint{5.743000in}{1.750000in}}%
\pgfpathlineto{\pgfqpoint{5.744240in}{2.170000in}}%
\pgfpathlineto{\pgfqpoint{5.746720in}{1.435000in}}%
\pgfpathlineto{\pgfqpoint{5.750440in}{2.170000in}}%
\pgfpathlineto{\pgfqpoint{5.751680in}{1.820000in}}%
\pgfpathlineto{\pgfqpoint{5.754160in}{2.240000in}}%
\pgfpathlineto{\pgfqpoint{5.756640in}{1.645000in}}%
\pgfpathlineto{\pgfqpoint{5.757880in}{1.680000in}}%
\pgfpathlineto{\pgfqpoint{5.760360in}{1.960000in}}%
\pgfpathlineto{\pgfqpoint{5.761600in}{1.960000in}}%
\pgfpathlineto{\pgfqpoint{5.762840in}{1.995000in}}%
\pgfpathlineto{\pgfqpoint{5.764080in}{1.925000in}}%
\pgfpathlineto{\pgfqpoint{5.765320in}{1.925000in}}%
\pgfpathlineto{\pgfqpoint{5.766560in}{1.715000in}}%
\pgfpathlineto{\pgfqpoint{5.767800in}{1.960000in}}%
\pgfpathlineto{\pgfqpoint{5.770280in}{1.610000in}}%
\pgfpathlineto{\pgfqpoint{5.771520in}{1.785000in}}%
\pgfpathlineto{\pgfqpoint{5.772760in}{1.750000in}}%
\pgfpathlineto{\pgfqpoint{5.774000in}{1.995000in}}%
\pgfpathlineto{\pgfqpoint{5.775240in}{1.855000in}}%
\pgfpathlineto{\pgfqpoint{5.776480in}{2.065000in}}%
\pgfpathlineto{\pgfqpoint{5.778960in}{1.750000in}}%
\pgfpathlineto{\pgfqpoint{5.780200in}{1.960000in}}%
\pgfpathlineto{\pgfqpoint{5.785160in}{1.540000in}}%
\pgfpathlineto{\pgfqpoint{5.786400in}{1.575000in}}%
\pgfpathlineto{\pgfqpoint{5.787640in}{1.435000in}}%
\pgfpathlineto{\pgfqpoint{5.788880in}{1.890000in}}%
\pgfpathlineto{\pgfqpoint{5.791360in}{1.610000in}}%
\pgfpathlineto{\pgfqpoint{5.795080in}{2.520000in}}%
\pgfpathlineto{\pgfqpoint{5.796320in}{1.680000in}}%
\pgfpathlineto{\pgfqpoint{5.797560in}{1.785000in}}%
\pgfpathlineto{\pgfqpoint{5.798800in}{2.275000in}}%
\pgfpathlineto{\pgfqpoint{5.800040in}{2.170000in}}%
\pgfpathlineto{\pgfqpoint{5.801280in}{1.855000in}}%
\pgfpathlineto{\pgfqpoint{5.803760in}{2.170000in}}%
\pgfpathlineto{\pgfqpoint{5.807480in}{1.645000in}}%
\pgfpathlineto{\pgfqpoint{5.808720in}{1.750000in}}%
\pgfpathlineto{\pgfqpoint{5.809960in}{1.680000in}}%
\pgfpathlineto{\pgfqpoint{5.811200in}{1.960000in}}%
\pgfpathlineto{\pgfqpoint{5.812440in}{1.540000in}}%
\pgfpathlineto{\pgfqpoint{5.813680in}{1.855000in}}%
\pgfpathlineto{\pgfqpoint{5.814920in}{1.785000in}}%
\pgfpathlineto{\pgfqpoint{5.816160in}{1.400000in}}%
\pgfpathlineto{\pgfqpoint{5.817400in}{1.645000in}}%
\pgfpathlineto{\pgfqpoint{5.818640in}{1.610000in}}%
\pgfpathlineto{\pgfqpoint{5.819880in}{1.925000in}}%
\pgfpathlineto{\pgfqpoint{5.822360in}{1.610000in}}%
\pgfpathlineto{\pgfqpoint{5.823600in}{1.680000in}}%
\pgfpathlineto{\pgfqpoint{5.824840in}{1.470000in}}%
\pgfpathlineto{\pgfqpoint{5.827320in}{2.100000in}}%
\pgfpathlineto{\pgfqpoint{5.828560in}{2.135000in}}%
\pgfpathlineto{\pgfqpoint{5.829800in}{1.925000in}}%
\pgfpathlineto{\pgfqpoint{5.831040in}{1.295000in}}%
\pgfpathlineto{\pgfqpoint{5.832280in}{1.995000in}}%
\pgfpathlineto{\pgfqpoint{5.833520in}{1.680000in}}%
\pgfpathlineto{\pgfqpoint{5.834760in}{1.750000in}}%
\pgfpathlineto{\pgfqpoint{5.836000in}{2.240000in}}%
\pgfpathlineto{\pgfqpoint{5.837240in}{2.135000in}}%
\pgfpathlineto{\pgfqpoint{5.838480in}{1.925000in}}%
\pgfpathlineto{\pgfqpoint{5.839720in}{2.415000in}}%
\pgfpathlineto{\pgfqpoint{5.842200in}{1.680000in}}%
\pgfpathlineto{\pgfqpoint{5.843440in}{1.610000in}}%
\pgfpathlineto{\pgfqpoint{5.844680in}{2.030000in}}%
\pgfpathlineto{\pgfqpoint{5.845920in}{1.645000in}}%
\pgfpathlineto{\pgfqpoint{5.847160in}{1.750000in}}%
\pgfpathlineto{\pgfqpoint{5.848400in}{1.715000in}}%
\pgfpathlineto{\pgfqpoint{5.849640in}{2.205000in}}%
\pgfpathlineto{\pgfqpoint{5.850880in}{2.240000in}}%
\pgfpathlineto{\pgfqpoint{5.852120in}{2.135000in}}%
\pgfpathlineto{\pgfqpoint{5.854600in}{1.645000in}}%
\pgfpathlineto{\pgfqpoint{5.855840in}{2.065000in}}%
\pgfpathlineto{\pgfqpoint{5.857080in}{1.855000in}}%
\pgfpathlineto{\pgfqpoint{5.858320in}{1.890000in}}%
\pgfpathlineto{\pgfqpoint{5.859560in}{1.960000in}}%
\pgfpathlineto{\pgfqpoint{5.860800in}{1.925000in}}%
\pgfpathlineto{\pgfqpoint{5.862040in}{1.960000in}}%
\pgfpathlineto{\pgfqpoint{5.863280in}{1.785000in}}%
\pgfpathlineto{\pgfqpoint{5.864520in}{1.960000in}}%
\pgfpathlineto{\pgfqpoint{5.865760in}{1.750000in}}%
\pgfpathlineto{\pgfqpoint{5.867000in}{1.750000in}}%
\pgfpathlineto{\pgfqpoint{5.868240in}{1.470000in}}%
\pgfpathlineto{\pgfqpoint{5.870720in}{2.100000in}}%
\pgfpathlineto{\pgfqpoint{5.873200in}{1.715000in}}%
\pgfpathlineto{\pgfqpoint{5.875680in}{2.135000in}}%
\pgfpathlineto{\pgfqpoint{5.876920in}{1.645000in}}%
\pgfpathlineto{\pgfqpoint{5.878160in}{1.890000in}}%
\pgfpathlineto{\pgfqpoint{5.879400in}{1.855000in}}%
\pgfpathlineto{\pgfqpoint{5.880640in}{2.240000in}}%
\pgfpathlineto{\pgfqpoint{5.881880in}{1.995000in}}%
\pgfpathlineto{\pgfqpoint{5.883120in}{2.030000in}}%
\pgfpathlineto{\pgfqpoint{5.884360in}{1.995000in}}%
\pgfpathlineto{\pgfqpoint{5.885600in}{2.065000in}}%
\pgfpathlineto{\pgfqpoint{5.886840in}{1.715000in}}%
\pgfpathlineto{\pgfqpoint{5.888080in}{2.065000in}}%
\pgfpathlineto{\pgfqpoint{5.889320in}{2.030000in}}%
\pgfpathlineto{\pgfqpoint{5.891800in}{1.645000in}}%
\pgfpathlineto{\pgfqpoint{5.894280in}{2.205000in}}%
\pgfpathlineto{\pgfqpoint{5.895520in}{1.715000in}}%
\pgfpathlineto{\pgfqpoint{5.896760in}{2.065000in}}%
\pgfpathlineto{\pgfqpoint{5.898000in}{2.065000in}}%
\pgfpathlineto{\pgfqpoint{5.899240in}{1.715000in}}%
\pgfpathlineto{\pgfqpoint{5.900480in}{1.750000in}}%
\pgfpathlineto{\pgfqpoint{5.901720in}{1.365000in}}%
\pgfpathlineto{\pgfqpoint{5.905440in}{2.100000in}}%
\pgfpathlineto{\pgfqpoint{5.906680in}{1.855000in}}%
\pgfpathlineto{\pgfqpoint{5.907920in}{2.065000in}}%
\pgfpathlineto{\pgfqpoint{5.909160in}{1.925000in}}%
\pgfpathlineto{\pgfqpoint{5.910400in}{2.135000in}}%
\pgfpathlineto{\pgfqpoint{5.912880in}{1.715000in}}%
\pgfpathlineto{\pgfqpoint{5.914120in}{1.785000in}}%
\pgfpathlineto{\pgfqpoint{5.916600in}{2.170000in}}%
\pgfpathlineto{\pgfqpoint{5.919080in}{1.750000in}}%
\pgfpathlineto{\pgfqpoint{5.920320in}{2.170000in}}%
\pgfpathlineto{\pgfqpoint{5.921560in}{1.750000in}}%
\pgfpathlineto{\pgfqpoint{5.924040in}{2.100000in}}%
\pgfpathlineto{\pgfqpoint{5.926520in}{1.505000in}}%
\pgfpathlineto{\pgfqpoint{5.927760in}{1.610000in}}%
\pgfpathlineto{\pgfqpoint{5.929000in}{2.030000in}}%
\pgfpathlineto{\pgfqpoint{5.930240in}{2.065000in}}%
\pgfpathlineto{\pgfqpoint{5.931480in}{2.170000in}}%
\pgfpathlineto{\pgfqpoint{5.933960in}{1.680000in}}%
\pgfpathlineto{\pgfqpoint{5.935200in}{1.715000in}}%
\pgfpathlineto{\pgfqpoint{5.936440in}{2.065000in}}%
\pgfpathlineto{\pgfqpoint{5.937680in}{1.890000in}}%
\pgfpathlineto{\pgfqpoint{5.938920in}{1.925000in}}%
\pgfpathlineto{\pgfqpoint{5.940160in}{2.030000in}}%
\pgfpathlineto{\pgfqpoint{5.941400in}{1.995000in}}%
\pgfpathlineto{\pgfqpoint{5.942640in}{2.065000in}}%
\pgfpathlineto{\pgfqpoint{5.943880in}{1.785000in}}%
\pgfpathlineto{\pgfqpoint{5.945120in}{2.135000in}}%
\pgfpathlineto{\pgfqpoint{5.946360in}{1.645000in}}%
\pgfpathlineto{\pgfqpoint{5.951320in}{2.205000in}}%
\pgfpathlineto{\pgfqpoint{5.953800in}{1.610000in}}%
\pgfpathlineto{\pgfqpoint{5.955040in}{1.400000in}}%
\pgfpathlineto{\pgfqpoint{5.956280in}{1.645000in}}%
\pgfpathlineto{\pgfqpoint{5.957520in}{1.190000in}}%
\pgfpathlineto{\pgfqpoint{5.958760in}{1.610000in}}%
\pgfpathlineto{\pgfqpoint{5.960000in}{1.365000in}}%
\pgfpathlineto{\pgfqpoint{5.962480in}{2.135000in}}%
\pgfpathlineto{\pgfqpoint{5.963720in}{1.855000in}}%
\pgfpathlineto{\pgfqpoint{5.964960in}{1.995000in}}%
\pgfpathlineto{\pgfqpoint{5.966200in}{1.645000in}}%
\pgfpathlineto{\pgfqpoint{5.967440in}{1.785000in}}%
\pgfpathlineto{\pgfqpoint{5.968680in}{1.435000in}}%
\pgfpathlineto{\pgfqpoint{5.969920in}{1.470000in}}%
\pgfpathlineto{\pgfqpoint{5.972400in}{1.925000in}}%
\pgfpathlineto{\pgfqpoint{5.973640in}{1.435000in}}%
\pgfpathlineto{\pgfqpoint{5.974880in}{1.960000in}}%
\pgfpathlineto{\pgfqpoint{5.977360in}{1.470000in}}%
\pgfpathlineto{\pgfqpoint{5.978600in}{2.100000in}}%
\pgfpathlineto{\pgfqpoint{5.981080in}{1.925000in}}%
\pgfpathlineto{\pgfqpoint{5.982320in}{1.680000in}}%
\pgfpathlineto{\pgfqpoint{5.984800in}{1.925000in}}%
\pgfpathlineto{\pgfqpoint{5.986040in}{1.505000in}}%
\pgfpathlineto{\pgfqpoint{5.987280in}{1.540000in}}%
\pgfpathlineto{\pgfqpoint{5.988520in}{2.135000in}}%
\pgfpathlineto{\pgfqpoint{5.991000in}{1.575000in}}%
\pgfpathlineto{\pgfqpoint{5.992240in}{1.925000in}}%
\pgfpathlineto{\pgfqpoint{5.993480in}{1.435000in}}%
\pgfpathlineto{\pgfqpoint{5.995960in}{1.995000in}}%
\pgfpathlineto{\pgfqpoint{5.997200in}{1.925000in}}%
\pgfpathlineto{\pgfqpoint{5.998440in}{1.785000in}}%
\pgfpathlineto{\pgfqpoint{5.999680in}{2.030000in}}%
\pgfpathlineto{\pgfqpoint{6.000920in}{2.590000in}}%
\pgfpathlineto{\pgfqpoint{6.002160in}{1.645000in}}%
\pgfpathlineto{\pgfqpoint{6.003400in}{2.065000in}}%
\pgfpathlineto{\pgfqpoint{6.004640in}{1.995000in}}%
\pgfpathlineto{\pgfqpoint{6.005880in}{1.855000in}}%
\pgfpathlineto{\pgfqpoint{6.007120in}{2.065000in}}%
\pgfpathlineto{\pgfqpoint{6.009600in}{1.820000in}}%
\pgfpathlineto{\pgfqpoint{6.010840in}{1.505000in}}%
\pgfpathlineto{\pgfqpoint{6.012080in}{2.100000in}}%
\pgfpathlineto{\pgfqpoint{6.014560in}{1.855000in}}%
\pgfpathlineto{\pgfqpoint{6.015800in}{2.100000in}}%
\pgfpathlineto{\pgfqpoint{6.017040in}{1.750000in}}%
\pgfpathlineto{\pgfqpoint{6.018280in}{1.820000in}}%
\pgfpathlineto{\pgfqpoint{6.019520in}{2.240000in}}%
\pgfpathlineto{\pgfqpoint{6.020760in}{2.170000in}}%
\pgfpathlineto{\pgfqpoint{6.022000in}{2.030000in}}%
\pgfpathlineto{\pgfqpoint{6.023240in}{1.610000in}}%
\pgfpathlineto{\pgfqpoint{6.024480in}{1.960000in}}%
\pgfpathlineto{\pgfqpoint{6.025720in}{1.680000in}}%
\pgfpathlineto{\pgfqpoint{6.026960in}{1.750000in}}%
\pgfpathlineto{\pgfqpoint{6.029440in}{2.135000in}}%
\pgfpathlineto{\pgfqpoint{6.030680in}{2.240000in}}%
\pgfpathlineto{\pgfqpoint{6.031920in}{1.890000in}}%
\pgfpathlineto{\pgfqpoint{6.033160in}{1.960000in}}%
\pgfpathlineto{\pgfqpoint{6.035640in}{1.505000in}}%
\pgfpathlineto{\pgfqpoint{6.039360in}{2.170000in}}%
\pgfpathlineto{\pgfqpoint{6.040600in}{1.890000in}}%
\pgfpathlineto{\pgfqpoint{6.041840in}{2.030000in}}%
\pgfpathlineto{\pgfqpoint{6.043080in}{1.680000in}}%
\pgfpathlineto{\pgfqpoint{6.045560in}{2.100000in}}%
\pgfpathlineto{\pgfqpoint{6.046800in}{1.890000in}}%
\pgfpathlineto{\pgfqpoint{6.048040in}{2.380000in}}%
\pgfpathlineto{\pgfqpoint{6.049280in}{2.135000in}}%
\pgfpathlineto{\pgfqpoint{6.050520in}{2.170000in}}%
\pgfpathlineto{\pgfqpoint{6.053000in}{1.995000in}}%
\pgfpathlineto{\pgfqpoint{6.054240in}{1.680000in}}%
\pgfpathlineto{\pgfqpoint{6.055480in}{1.820000in}}%
\pgfpathlineto{\pgfqpoint{6.056720in}{1.820000in}}%
\pgfpathlineto{\pgfqpoint{6.057960in}{2.135000in}}%
\pgfpathlineto{\pgfqpoint{6.059200in}{2.135000in}}%
\pgfpathlineto{\pgfqpoint{6.060440in}{2.065000in}}%
\pgfpathlineto{\pgfqpoint{6.061680in}{1.890000in}}%
\pgfpathlineto{\pgfqpoint{6.064160in}{2.030000in}}%
\pgfpathlineto{\pgfqpoint{6.065400in}{1.960000in}}%
\pgfpathlineto{\pgfqpoint{6.066640in}{1.540000in}}%
\pgfpathlineto{\pgfqpoint{6.067880in}{1.645000in}}%
\pgfpathlineto{\pgfqpoint{6.069120in}{1.435000in}}%
\pgfpathlineto{\pgfqpoint{6.070360in}{1.680000in}}%
\pgfpathlineto{\pgfqpoint{6.071600in}{1.575000in}}%
\pgfpathlineto{\pgfqpoint{6.072840in}{2.240000in}}%
\pgfpathlineto{\pgfqpoint{6.075320in}{1.540000in}}%
\pgfpathlineto{\pgfqpoint{6.076560in}{1.925000in}}%
\pgfpathlineto{\pgfqpoint{6.077800in}{1.400000in}}%
\pgfpathlineto{\pgfqpoint{6.079040in}{1.890000in}}%
\pgfpathlineto{\pgfqpoint{6.082760in}{1.365000in}}%
\pgfpathlineto{\pgfqpoint{6.085240in}{1.925000in}}%
\pgfpathlineto{\pgfqpoint{6.086480in}{1.680000in}}%
\pgfpathlineto{\pgfqpoint{6.087720in}{1.680000in}}%
\pgfpathlineto{\pgfqpoint{6.088960in}{1.785000in}}%
\pgfpathlineto{\pgfqpoint{6.090200in}{1.750000in}}%
\pgfpathlineto{\pgfqpoint{6.091440in}{1.680000in}}%
\pgfpathlineto{\pgfqpoint{6.092680in}{1.785000in}}%
\pgfpathlineto{\pgfqpoint{6.093920in}{2.240000in}}%
\pgfpathlineto{\pgfqpoint{6.095160in}{1.855000in}}%
\pgfpathlineto{\pgfqpoint{6.096400in}{1.890000in}}%
\pgfpathlineto{\pgfqpoint{6.097640in}{1.610000in}}%
\pgfpathlineto{\pgfqpoint{6.098880in}{1.750000in}}%
\pgfpathlineto{\pgfqpoint{6.100120in}{1.680000in}}%
\pgfpathlineto{\pgfqpoint{6.102600in}{1.960000in}}%
\pgfpathlineto{\pgfqpoint{6.103840in}{1.855000in}}%
\pgfpathlineto{\pgfqpoint{6.105080in}{1.960000in}}%
\pgfpathlineto{\pgfqpoint{6.106320in}{1.960000in}}%
\pgfpathlineto{\pgfqpoint{6.107560in}{2.135000in}}%
\pgfpathlineto{\pgfqpoint{6.108800in}{1.855000in}}%
\pgfpathlineto{\pgfqpoint{6.110040in}{2.170000in}}%
\pgfpathlineto{\pgfqpoint{6.112520in}{1.785000in}}%
\pgfpathlineto{\pgfqpoint{6.113760in}{2.030000in}}%
\pgfpathlineto{\pgfqpoint{6.115000in}{1.925000in}}%
\pgfpathlineto{\pgfqpoint{6.116240in}{1.995000in}}%
\pgfpathlineto{\pgfqpoint{6.117480in}{1.785000in}}%
\pgfpathlineto{\pgfqpoint{6.118720in}{1.925000in}}%
\pgfpathlineto{\pgfqpoint{6.119960in}{2.345000in}}%
\pgfpathlineto{\pgfqpoint{6.121200in}{1.680000in}}%
\pgfpathlineto{\pgfqpoint{6.123680in}{2.030000in}}%
\pgfpathlineto{\pgfqpoint{6.124920in}{2.030000in}}%
\pgfpathlineto{\pgfqpoint{6.126160in}{1.995000in}}%
\pgfpathlineto{\pgfqpoint{6.127400in}{1.715000in}}%
\pgfpathlineto{\pgfqpoint{6.128640in}{1.785000in}}%
\pgfpathlineto{\pgfqpoint{6.129880in}{2.030000in}}%
\pgfpathlineto{\pgfqpoint{6.131120in}{1.960000in}}%
\pgfpathlineto{\pgfqpoint{6.132360in}{1.995000in}}%
\pgfpathlineto{\pgfqpoint{6.133600in}{1.995000in}}%
\pgfpathlineto{\pgfqpoint{6.134840in}{1.855000in}}%
\pgfpathlineto{\pgfqpoint{6.136080in}{2.135000in}}%
\pgfpathlineto{\pgfqpoint{6.137320in}{1.470000in}}%
\pgfpathlineto{\pgfqpoint{6.138560in}{1.750000in}}%
\pgfpathlineto{\pgfqpoint{6.139800in}{1.680000in}}%
\pgfpathlineto{\pgfqpoint{6.143520in}{2.240000in}}%
\pgfpathlineto{\pgfqpoint{6.144760in}{2.135000in}}%
\pgfpathlineto{\pgfqpoint{6.146000in}{2.205000in}}%
\pgfpathlineto{\pgfqpoint{6.147240in}{1.820000in}}%
\pgfpathlineto{\pgfqpoint{6.149720in}{2.170000in}}%
\pgfpathlineto{\pgfqpoint{6.150960in}{2.135000in}}%
\pgfpathlineto{\pgfqpoint{6.152200in}{2.205000in}}%
\pgfpathlineto{\pgfqpoint{6.153440in}{2.170000in}}%
\pgfpathlineto{\pgfqpoint{6.154680in}{2.170000in}}%
\pgfpathlineto{\pgfqpoint{6.155920in}{2.135000in}}%
\pgfpathlineto{\pgfqpoint{6.157160in}{2.170000in}}%
\pgfpathlineto{\pgfqpoint{6.158400in}{1.995000in}}%
\pgfpathlineto{\pgfqpoint{6.159640in}{2.205000in}}%
\pgfpathlineto{\pgfqpoint{6.160880in}{1.645000in}}%
\pgfpathlineto{\pgfqpoint{6.163360in}{2.310000in}}%
\pgfpathlineto{\pgfqpoint{6.164600in}{1.995000in}}%
\pgfpathlineto{\pgfqpoint{6.165840in}{2.310000in}}%
\pgfpathlineto{\pgfqpoint{6.167080in}{1.785000in}}%
\pgfpathlineto{\pgfqpoint{6.168320in}{1.855000in}}%
\pgfpathlineto{\pgfqpoint{6.169560in}{1.820000in}}%
\pgfpathlineto{\pgfqpoint{6.170800in}{2.065000in}}%
\pgfpathlineto{\pgfqpoint{6.172040in}{2.030000in}}%
\pgfpathlineto{\pgfqpoint{6.173280in}{2.135000in}}%
\pgfpathlineto{\pgfqpoint{6.174520in}{1.820000in}}%
\pgfpathlineto{\pgfqpoint{6.175760in}{2.030000in}}%
\pgfpathlineto{\pgfqpoint{6.177000in}{1.855000in}}%
\pgfpathlineto{\pgfqpoint{6.179480in}{2.345000in}}%
\pgfpathlineto{\pgfqpoint{6.180720in}{1.890000in}}%
\pgfpathlineto{\pgfqpoint{6.181960in}{1.960000in}}%
\pgfpathlineto{\pgfqpoint{6.183200in}{1.820000in}}%
\pgfpathlineto{\pgfqpoint{6.184440in}{1.995000in}}%
\pgfpathlineto{\pgfqpoint{6.185680in}{1.890000in}}%
\pgfpathlineto{\pgfqpoint{6.186920in}{1.925000in}}%
\pgfpathlineto{\pgfqpoint{6.188160in}{1.505000in}}%
\pgfpathlineto{\pgfqpoint{6.189400in}{1.715000in}}%
\pgfpathlineto{\pgfqpoint{6.190640in}{1.540000in}}%
\pgfpathlineto{\pgfqpoint{6.195600in}{1.960000in}}%
\pgfpathlineto{\pgfqpoint{6.196840in}{1.750000in}}%
\pgfpathlineto{\pgfqpoint{6.198080in}{2.205000in}}%
\pgfpathlineto{\pgfqpoint{6.200560in}{1.680000in}}%
\pgfpathlineto{\pgfqpoint{6.201800in}{2.030000in}}%
\pgfpathlineto{\pgfqpoint{6.204280in}{1.575000in}}%
\pgfpathlineto{\pgfqpoint{6.206760in}{2.065000in}}%
\pgfpathlineto{\pgfqpoint{6.208000in}{2.100000in}}%
\pgfpathlineto{\pgfqpoint{6.210480in}{1.785000in}}%
\pgfpathlineto{\pgfqpoint{6.211720in}{1.680000in}}%
\pgfpathlineto{\pgfqpoint{6.212960in}{2.030000in}}%
\pgfpathlineto{\pgfqpoint{6.215440in}{1.435000in}}%
\pgfpathlineto{\pgfqpoint{6.217920in}{1.925000in}}%
\pgfpathlineto{\pgfqpoint{6.219160in}{1.225000in}}%
\pgfpathlineto{\pgfqpoint{6.222880in}{2.065000in}}%
\pgfpathlineto{\pgfqpoint{6.224120in}{1.785000in}}%
\pgfpathlineto{\pgfqpoint{6.225360in}{1.925000in}}%
\pgfpathlineto{\pgfqpoint{6.226600in}{1.575000in}}%
\pgfpathlineto{\pgfqpoint{6.229080in}{1.960000in}}%
\pgfpathlineto{\pgfqpoint{6.231560in}{1.680000in}}%
\pgfpathlineto{\pgfqpoint{6.234040in}{2.100000in}}%
\pgfpathlineto{\pgfqpoint{6.235280in}{1.855000in}}%
\pgfpathlineto{\pgfqpoint{6.236520in}{2.100000in}}%
\pgfpathlineto{\pgfqpoint{6.237760in}{1.995000in}}%
\pgfpathlineto{\pgfqpoint{6.239000in}{2.100000in}}%
\pgfpathlineto{\pgfqpoint{6.240240in}{1.890000in}}%
\pgfpathlineto{\pgfqpoint{6.241480in}{1.435000in}}%
\pgfpathlineto{\pgfqpoint{6.247680in}{1.925000in}}%
\pgfpathlineto{\pgfqpoint{6.248920in}{1.680000in}}%
\pgfpathlineto{\pgfqpoint{6.251400in}{2.100000in}}%
\pgfpathlineto{\pgfqpoint{6.252640in}{1.680000in}}%
\pgfpathlineto{\pgfqpoint{6.256360in}{2.485000in}}%
\pgfpathlineto{\pgfqpoint{6.257600in}{2.170000in}}%
\pgfpathlineto{\pgfqpoint{6.258840in}{2.170000in}}%
\pgfpathlineto{\pgfqpoint{6.261320in}{1.995000in}}%
\pgfpathlineto{\pgfqpoint{6.262560in}{2.030000in}}%
\pgfpathlineto{\pgfqpoint{6.263800in}{1.820000in}}%
\pgfpathlineto{\pgfqpoint{6.265040in}{2.030000in}}%
\pgfpathlineto{\pgfqpoint{6.266280in}{1.785000in}}%
\pgfpathlineto{\pgfqpoint{6.267520in}{1.890000in}}%
\pgfpathlineto{\pgfqpoint{6.268760in}{2.135000in}}%
\pgfpathlineto{\pgfqpoint{6.270000in}{1.890000in}}%
\pgfpathlineto{\pgfqpoint{6.271240in}{2.065000in}}%
\pgfpathlineto{\pgfqpoint{6.272480in}{1.435000in}}%
\pgfpathlineto{\pgfqpoint{6.273720in}{2.100000in}}%
\pgfpathlineto{\pgfqpoint{6.276200in}{1.785000in}}%
\pgfpathlineto{\pgfqpoint{6.277440in}{1.995000in}}%
\pgfpathlineto{\pgfqpoint{6.278680in}{1.505000in}}%
\pgfpathlineto{\pgfqpoint{6.281160in}{2.030000in}}%
\pgfpathlineto{\pgfqpoint{6.282400in}{1.505000in}}%
\pgfpathlineto{\pgfqpoint{6.283640in}{2.065000in}}%
\pgfpathlineto{\pgfqpoint{6.284880in}{1.750000in}}%
\pgfpathlineto{\pgfqpoint{6.286120in}{2.065000in}}%
\pgfpathlineto{\pgfqpoint{6.287360in}{2.065000in}}%
\pgfpathlineto{\pgfqpoint{6.288600in}{2.450000in}}%
\pgfpathlineto{\pgfqpoint{6.291080in}{1.855000in}}%
\pgfpathlineto{\pgfqpoint{6.292320in}{1.890000in}}%
\pgfpathlineto{\pgfqpoint{6.293560in}{2.030000in}}%
\pgfpathlineto{\pgfqpoint{6.296040in}{1.855000in}}%
\pgfpathlineto{\pgfqpoint{6.297280in}{1.750000in}}%
\pgfpathlineto{\pgfqpoint{6.298520in}{2.030000in}}%
\pgfpathlineto{\pgfqpoint{6.299760in}{1.575000in}}%
\pgfpathlineto{\pgfqpoint{6.301000in}{1.855000in}}%
\pgfpathlineto{\pgfqpoint{6.302240in}{1.855000in}}%
\pgfpathlineto{\pgfqpoint{6.304720in}{1.680000in}}%
\pgfpathlineto{\pgfqpoint{6.305960in}{1.680000in}}%
\pgfpathlineto{\pgfqpoint{6.307200in}{1.435000in}}%
\pgfpathlineto{\pgfqpoint{6.308440in}{1.680000in}}%
\pgfpathlineto{\pgfqpoint{6.309680in}{1.540000in}}%
\pgfpathlineto{\pgfqpoint{6.312160in}{1.750000in}}%
\pgfpathlineto{\pgfqpoint{6.313400in}{1.750000in}}%
\pgfpathlineto{\pgfqpoint{6.314640in}{2.135000in}}%
\pgfpathlineto{\pgfqpoint{6.315880in}{1.680000in}}%
\pgfpathlineto{\pgfqpoint{6.318360in}{1.680000in}}%
\pgfpathlineto{\pgfqpoint{6.319600in}{2.030000in}}%
\pgfpathlineto{\pgfqpoint{6.320840in}{1.995000in}}%
\pgfpathlineto{\pgfqpoint{6.322080in}{1.995000in}}%
\pgfpathlineto{\pgfqpoint{6.323320in}{2.275000in}}%
\pgfpathlineto{\pgfqpoint{6.324560in}{1.680000in}}%
\pgfpathlineto{\pgfqpoint{6.325800in}{1.680000in}}%
\pgfpathlineto{\pgfqpoint{6.327040in}{1.890000in}}%
\pgfpathlineto{\pgfqpoint{6.328280in}{1.785000in}}%
\pgfpathlineto{\pgfqpoint{6.329520in}{1.470000in}}%
\pgfpathlineto{\pgfqpoint{6.330760in}{1.750000in}}%
\pgfpathlineto{\pgfqpoint{6.332000in}{1.680000in}}%
\pgfpathlineto{\pgfqpoint{6.333240in}{2.135000in}}%
\pgfpathlineto{\pgfqpoint{6.334480in}{1.960000in}}%
\pgfpathlineto{\pgfqpoint{6.335720in}{2.170000in}}%
\pgfpathlineto{\pgfqpoint{6.336960in}{2.065000in}}%
\pgfpathlineto{\pgfqpoint{6.338200in}{1.645000in}}%
\pgfpathlineto{\pgfqpoint{6.339440in}{2.170000in}}%
\pgfpathlineto{\pgfqpoint{6.340680in}{1.505000in}}%
\pgfpathlineto{\pgfqpoint{6.341920in}{1.960000in}}%
\pgfpathlineto{\pgfqpoint{6.344400in}{1.680000in}}%
\pgfpathlineto{\pgfqpoint{6.345640in}{1.890000in}}%
\pgfpathlineto{\pgfqpoint{6.346880in}{2.310000in}}%
\pgfpathlineto{\pgfqpoint{6.349360in}{2.030000in}}%
\pgfpathlineto{\pgfqpoint{6.353080in}{1.645000in}}%
\pgfpathlineto{\pgfqpoint{6.354320in}{1.890000in}}%
\pgfpathlineto{\pgfqpoint{6.355560in}{1.855000in}}%
\pgfpathlineto{\pgfqpoint{6.356800in}{1.645000in}}%
\pgfpathlineto{\pgfqpoint{6.358040in}{2.065000in}}%
\pgfpathlineto{\pgfqpoint{6.359280in}{1.645000in}}%
\pgfpathlineto{\pgfqpoint{6.363000in}{2.135000in}}%
\pgfpathlineto{\pgfqpoint{6.364240in}{1.995000in}}%
\pgfpathlineto{\pgfqpoint{6.365480in}{2.030000in}}%
\pgfpathlineto{\pgfqpoint{6.366720in}{1.960000in}}%
\pgfpathlineto{\pgfqpoint{6.367960in}{2.065000in}}%
\pgfpathlineto{\pgfqpoint{6.369200in}{1.925000in}}%
\pgfpathlineto{\pgfqpoint{6.370440in}{2.135000in}}%
\pgfpathlineto{\pgfqpoint{6.371680in}{2.065000in}}%
\pgfpathlineto{\pgfqpoint{6.372920in}{2.170000in}}%
\pgfpathlineto{\pgfqpoint{6.374160in}{2.065000in}}%
\pgfpathlineto{\pgfqpoint{6.376640in}{1.540000in}}%
\pgfpathlineto{\pgfqpoint{6.377880in}{1.925000in}}%
\pgfpathlineto{\pgfqpoint{6.379120in}{1.575000in}}%
\pgfpathlineto{\pgfqpoint{6.381600in}{1.925000in}}%
\pgfpathlineto{\pgfqpoint{6.382840in}{1.890000in}}%
\pgfpathlineto{\pgfqpoint{6.384080in}{1.540000in}}%
\pgfpathlineto{\pgfqpoint{6.387800in}{1.925000in}}%
\pgfpathlineto{\pgfqpoint{6.389040in}{1.925000in}}%
\pgfpathlineto{\pgfqpoint{6.390280in}{2.170000in}}%
\pgfpathlineto{\pgfqpoint{6.392760in}{1.680000in}}%
\pgfpathlineto{\pgfqpoint{6.395240in}{1.855000in}}%
\pgfpathlineto{\pgfqpoint{6.396480in}{1.715000in}}%
\pgfpathlineto{\pgfqpoint{6.397720in}{1.890000in}}%
\pgfpathlineto{\pgfqpoint{6.398960in}{1.680000in}}%
\pgfpathlineto{\pgfqpoint{6.400200in}{1.925000in}}%
\pgfpathlineto{\pgfqpoint{6.401440in}{1.715000in}}%
\pgfpathlineto{\pgfqpoint{6.402680in}{2.170000in}}%
\pgfpathlineto{\pgfqpoint{6.403920in}{1.295000in}}%
\pgfpathlineto{\pgfqpoint{6.405160in}{1.855000in}}%
\pgfpathlineto{\pgfqpoint{6.407640in}{1.680000in}}%
\pgfpathlineto{\pgfqpoint{6.408880in}{1.715000in}}%
\pgfpathlineto{\pgfqpoint{6.410120in}{1.925000in}}%
\pgfpathlineto{\pgfqpoint{6.411360in}{1.190000in}}%
\pgfpathlineto{\pgfqpoint{6.413840in}{1.890000in}}%
\pgfpathlineto{\pgfqpoint{6.415080in}{1.890000in}}%
\pgfpathlineto{\pgfqpoint{6.416320in}{1.680000in}}%
\pgfpathlineto{\pgfqpoint{6.417560in}{1.715000in}}%
\pgfpathlineto{\pgfqpoint{6.420040in}{1.330000in}}%
\pgfpathlineto{\pgfqpoint{6.423760in}{2.135000in}}%
\pgfpathlineto{\pgfqpoint{6.426240in}{1.785000in}}%
\pgfpathlineto{\pgfqpoint{6.427480in}{2.415000in}}%
\pgfpathlineto{\pgfqpoint{6.428720in}{1.785000in}}%
\pgfpathlineto{\pgfqpoint{6.429960in}{1.855000in}}%
\pgfpathlineto{\pgfqpoint{6.431200in}{1.750000in}}%
\pgfpathlineto{\pgfqpoint{6.432440in}{2.065000in}}%
\pgfpathlineto{\pgfqpoint{6.433680in}{1.890000in}}%
\pgfpathlineto{\pgfqpoint{6.434920in}{2.065000in}}%
\pgfpathlineto{\pgfqpoint{6.436160in}{2.730000in}}%
\pgfpathlineto{\pgfqpoint{6.437400in}{2.730000in}}%
\pgfpathlineto{\pgfqpoint{6.439880in}{2.135000in}}%
\pgfpathlineto{\pgfqpoint{6.441120in}{2.170000in}}%
\pgfpathlineto{\pgfqpoint{6.442360in}{2.065000in}}%
\pgfpathlineto{\pgfqpoint{6.443600in}{2.065000in}}%
\pgfpathlineto{\pgfqpoint{6.444840in}{1.890000in}}%
\pgfpathlineto{\pgfqpoint{6.447320in}{2.205000in}}%
\pgfpathlineto{\pgfqpoint{6.448560in}{2.240000in}}%
\pgfpathlineto{\pgfqpoint{6.449800in}{2.205000in}}%
\pgfpathlineto{\pgfqpoint{6.451040in}{2.240000in}}%
\pgfpathlineto{\pgfqpoint{6.452280in}{2.380000in}}%
\pgfpathlineto{\pgfqpoint{6.453520in}{2.380000in}}%
\pgfpathlineto{\pgfqpoint{6.454760in}{1.925000in}}%
\pgfpathlineto{\pgfqpoint{6.456000in}{2.450000in}}%
\pgfpathlineto{\pgfqpoint{6.457240in}{1.715000in}}%
\pgfpathlineto{\pgfqpoint{6.458480in}{1.645000in}}%
\pgfpathlineto{\pgfqpoint{6.459720in}{2.030000in}}%
\pgfpathlineto{\pgfqpoint{6.462200in}{1.855000in}}%
\pgfpathlineto{\pgfqpoint{6.464680in}{2.030000in}}%
\pgfpathlineto{\pgfqpoint{6.465920in}{1.890000in}}%
\pgfpathlineto{\pgfqpoint{6.467160in}{1.925000in}}%
\pgfpathlineto{\pgfqpoint{6.468400in}{2.135000in}}%
\pgfpathlineto{\pgfqpoint{6.469640in}{1.715000in}}%
\pgfpathlineto{\pgfqpoint{6.470880in}{1.715000in}}%
\pgfpathlineto{\pgfqpoint{6.472120in}{1.995000in}}%
\pgfpathlineto{\pgfqpoint{6.473360in}{1.960000in}}%
\pgfpathlineto{\pgfqpoint{6.474600in}{2.240000in}}%
\pgfpathlineto{\pgfqpoint{6.475840in}{1.925000in}}%
\pgfpathlineto{\pgfqpoint{6.478320in}{2.205000in}}%
\pgfpathlineto{\pgfqpoint{6.479560in}{2.135000in}}%
\pgfpathlineto{\pgfqpoint{6.480800in}{2.310000in}}%
\pgfpathlineto{\pgfqpoint{6.482040in}{1.960000in}}%
\pgfpathlineto{\pgfqpoint{6.483280in}{2.030000in}}%
\pgfpathlineto{\pgfqpoint{6.484520in}{2.240000in}}%
\pgfpathlineto{\pgfqpoint{6.485760in}{2.100000in}}%
\pgfpathlineto{\pgfqpoint{6.487000in}{1.750000in}}%
\pgfpathlineto{\pgfqpoint{6.488240in}{2.100000in}}%
\pgfpathlineto{\pgfqpoint{6.489480in}{2.135000in}}%
\pgfpathlineto{\pgfqpoint{6.490720in}{2.240000in}}%
\pgfpathlineto{\pgfqpoint{6.493200in}{1.575000in}}%
\pgfpathlineto{\pgfqpoint{6.494440in}{1.715000in}}%
\pgfpathlineto{\pgfqpoint{6.495680in}{2.065000in}}%
\pgfpathlineto{\pgfqpoint{6.496920in}{1.645000in}}%
\pgfpathlineto{\pgfqpoint{6.498160in}{1.785000in}}%
\pgfpathlineto{\pgfqpoint{6.499400in}{1.715000in}}%
\pgfpathlineto{\pgfqpoint{6.500640in}{1.715000in}}%
\pgfpathlineto{\pgfqpoint{6.501880in}{1.785000in}}%
\pgfpathlineto{\pgfqpoint{6.503120in}{1.470000in}}%
\pgfpathlineto{\pgfqpoint{6.504360in}{1.540000in}}%
\pgfpathlineto{\pgfqpoint{6.505600in}{1.925000in}}%
\pgfpathlineto{\pgfqpoint{6.506840in}{1.610000in}}%
\pgfpathlineto{\pgfqpoint{6.509320in}{2.065000in}}%
\pgfpathlineto{\pgfqpoint{6.510560in}{2.065000in}}%
\pgfpathlineto{\pgfqpoint{6.511800in}{1.890000in}}%
\pgfpathlineto{\pgfqpoint{6.513040in}{1.995000in}}%
\pgfpathlineto{\pgfqpoint{6.515520in}{2.345000in}}%
\pgfpathlineto{\pgfqpoint{6.516760in}{1.890000in}}%
\pgfpathlineto{\pgfqpoint{6.518000in}{1.855000in}}%
\pgfpathlineto{\pgfqpoint{6.519240in}{1.260000in}}%
\pgfpathlineto{\pgfqpoint{6.521720in}{2.100000in}}%
\pgfpathlineto{\pgfqpoint{6.525440in}{1.855000in}}%
\pgfpathlineto{\pgfqpoint{6.526680in}{2.135000in}}%
\pgfpathlineto{\pgfqpoint{6.529160in}{1.960000in}}%
\pgfpathlineto{\pgfqpoint{6.530400in}{1.750000in}}%
\pgfpathlineto{\pgfqpoint{6.531640in}{1.960000in}}%
\pgfpathlineto{\pgfqpoint{6.532880in}{1.680000in}}%
\pgfpathlineto{\pgfqpoint{6.534120in}{1.680000in}}%
\pgfpathlineto{\pgfqpoint{6.536600in}{1.995000in}}%
\pgfpathlineto{\pgfqpoint{6.537840in}{1.855000in}}%
\pgfpathlineto{\pgfqpoint{6.539080in}{2.485000in}}%
\pgfpathlineto{\pgfqpoint{6.540320in}{2.240000in}}%
\pgfpathlineto{\pgfqpoint{6.541560in}{1.750000in}}%
\pgfpathlineto{\pgfqpoint{6.542800in}{1.890000in}}%
\pgfpathlineto{\pgfqpoint{6.544040in}{1.575000in}}%
\pgfpathlineto{\pgfqpoint{6.545280in}{2.415000in}}%
\pgfpathlineto{\pgfqpoint{6.546520in}{1.645000in}}%
\pgfpathlineto{\pgfqpoint{6.547760in}{2.030000in}}%
\pgfpathlineto{\pgfqpoint{6.549000in}{1.820000in}}%
\pgfpathlineto{\pgfqpoint{6.550240in}{2.065000in}}%
\pgfpathlineto{\pgfqpoint{6.551480in}{1.470000in}}%
\pgfpathlineto{\pgfqpoint{6.555200in}{2.065000in}}%
\pgfpathlineto{\pgfqpoint{6.556440in}{1.995000in}}%
\pgfpathlineto{\pgfqpoint{6.557680in}{1.680000in}}%
\pgfpathlineto{\pgfqpoint{6.562640in}{2.065000in}}%
\pgfpathlineto{\pgfqpoint{6.563880in}{2.100000in}}%
\pgfpathlineto{\pgfqpoint{6.568840in}{1.540000in}}%
\pgfpathlineto{\pgfqpoint{6.570080in}{1.820000in}}%
\pgfpathlineto{\pgfqpoint{6.571320in}{1.750000in}}%
\pgfpathlineto{\pgfqpoint{6.572560in}{1.260000in}}%
\pgfpathlineto{\pgfqpoint{6.573800in}{1.855000in}}%
\pgfpathlineto{\pgfqpoint{6.575040in}{1.855000in}}%
\pgfpathlineto{\pgfqpoint{6.576280in}{1.820000in}}%
\pgfpathlineto{\pgfqpoint{6.577520in}{1.120000in}}%
\pgfpathlineto{\pgfqpoint{6.578760in}{1.960000in}}%
\pgfpathlineto{\pgfqpoint{6.580000in}{1.610000in}}%
\pgfpathlineto{\pgfqpoint{6.581240in}{1.960000in}}%
\pgfpathlineto{\pgfqpoint{6.582480in}{1.925000in}}%
\pgfpathlineto{\pgfqpoint{6.583720in}{1.925000in}}%
\pgfpathlineto{\pgfqpoint{6.584960in}{2.450000in}}%
\pgfpathlineto{\pgfqpoint{6.586200in}{1.750000in}}%
\pgfpathlineto{\pgfqpoint{6.587440in}{1.890000in}}%
\pgfpathlineto{\pgfqpoint{6.589920in}{1.890000in}}%
\pgfpathlineto{\pgfqpoint{6.591160in}{1.715000in}}%
\pgfpathlineto{\pgfqpoint{6.592400in}{1.890000in}}%
\pgfpathlineto{\pgfqpoint{6.593640in}{1.855000in}}%
\pgfpathlineto{\pgfqpoint{6.596120in}{1.925000in}}%
\pgfpathlineto{\pgfqpoint{6.597360in}{1.750000in}}%
\pgfpathlineto{\pgfqpoint{6.598600in}{1.785000in}}%
\pgfpathlineto{\pgfqpoint{6.599840in}{2.135000in}}%
\pgfpathlineto{\pgfqpoint{6.602320in}{1.715000in}}%
\pgfpathlineto{\pgfqpoint{6.603560in}{1.925000in}}%
\pgfpathlineto{\pgfqpoint{6.604800in}{1.890000in}}%
\pgfpathlineto{\pgfqpoint{6.606040in}{1.925000in}}%
\pgfpathlineto{\pgfqpoint{6.607280in}{1.890000in}}%
\pgfpathlineto{\pgfqpoint{6.608520in}{1.890000in}}%
\pgfpathlineto{\pgfqpoint{6.609760in}{1.960000in}}%
\pgfpathlineto{\pgfqpoint{6.612240in}{1.960000in}}%
\pgfpathlineto{\pgfqpoint{6.613480in}{1.855000in}}%
\pgfpathlineto{\pgfqpoint{6.614720in}{2.275000in}}%
\pgfpathlineto{\pgfqpoint{6.615960in}{2.205000in}}%
\pgfpathlineto{\pgfqpoint{6.617200in}{1.575000in}}%
\pgfpathlineto{\pgfqpoint{6.619680in}{2.100000in}}%
\pgfpathlineto{\pgfqpoint{6.620920in}{1.575000in}}%
\pgfpathlineto{\pgfqpoint{6.622160in}{1.750000in}}%
\pgfpathlineto{\pgfqpoint{6.623400in}{2.100000in}}%
\pgfpathlineto{\pgfqpoint{6.625880in}{1.785000in}}%
\pgfpathlineto{\pgfqpoint{6.627120in}{1.995000in}}%
\pgfpathlineto{\pgfqpoint{6.628360in}{1.435000in}}%
\pgfpathlineto{\pgfqpoint{6.630840in}{2.205000in}}%
\pgfpathlineto{\pgfqpoint{6.632080in}{2.100000in}}%
\pgfpathlineto{\pgfqpoint{6.633320in}{2.170000in}}%
\pgfpathlineto{\pgfqpoint{6.634560in}{1.890000in}}%
\pgfpathlineto{\pgfqpoint{6.635800in}{1.925000in}}%
\pgfpathlineto{\pgfqpoint{6.637040in}{1.855000in}}%
\pgfpathlineto{\pgfqpoint{6.638280in}{1.960000in}}%
\pgfpathlineto{\pgfqpoint{6.639520in}{2.170000in}}%
\pgfpathlineto{\pgfqpoint{6.640760in}{1.995000in}}%
\pgfpathlineto{\pgfqpoint{6.642000in}{1.645000in}}%
\pgfpathlineto{\pgfqpoint{6.644480in}{1.995000in}}%
\pgfpathlineto{\pgfqpoint{6.645720in}{2.030000in}}%
\pgfpathlineto{\pgfqpoint{6.646960in}{1.960000in}}%
\pgfpathlineto{\pgfqpoint{6.648200in}{1.995000in}}%
\pgfpathlineto{\pgfqpoint{6.649440in}{1.995000in}}%
\pgfpathlineto{\pgfqpoint{6.650680in}{2.170000in}}%
\pgfpathlineto{\pgfqpoint{6.651920in}{1.995000in}}%
\pgfpathlineto{\pgfqpoint{6.653160in}{2.030000in}}%
\pgfpathlineto{\pgfqpoint{6.654400in}{1.610000in}}%
\pgfpathlineto{\pgfqpoint{6.656880in}{2.240000in}}%
\pgfpathlineto{\pgfqpoint{6.659360in}{1.995000in}}%
\pgfpathlineto{\pgfqpoint{6.660600in}{1.960000in}}%
\pgfpathlineto{\pgfqpoint{6.661840in}{2.240000in}}%
\pgfpathlineto{\pgfqpoint{6.663080in}{1.610000in}}%
\pgfpathlineto{\pgfqpoint{6.664320in}{1.960000in}}%
\pgfpathlineto{\pgfqpoint{6.665560in}{1.960000in}}%
\pgfpathlineto{\pgfqpoint{6.666800in}{1.890000in}}%
\pgfpathlineto{\pgfqpoint{6.668040in}{2.065000in}}%
\pgfpathlineto{\pgfqpoint{6.669280in}{2.065000in}}%
\pgfpathlineto{\pgfqpoint{6.670520in}{1.750000in}}%
\pgfpathlineto{\pgfqpoint{6.671760in}{1.890000in}}%
\pgfpathlineto{\pgfqpoint{6.673000in}{1.820000in}}%
\pgfpathlineto{\pgfqpoint{6.674240in}{1.855000in}}%
\pgfpathlineto{\pgfqpoint{6.675480in}{1.645000in}}%
\pgfpathlineto{\pgfqpoint{6.677960in}{2.100000in}}%
\pgfpathlineto{\pgfqpoint{6.679200in}{1.610000in}}%
\pgfpathlineto{\pgfqpoint{6.680440in}{1.610000in}}%
\pgfpathlineto{\pgfqpoint{6.681680in}{1.715000in}}%
\pgfpathlineto{\pgfqpoint{6.684160in}{1.610000in}}%
\pgfpathlineto{\pgfqpoint{6.686640in}{1.750000in}}%
\pgfpathlineto{\pgfqpoint{6.687880in}{1.470000in}}%
\pgfpathlineto{\pgfqpoint{6.690360in}{1.820000in}}%
\pgfpathlineto{\pgfqpoint{6.691600in}{1.785000in}}%
\pgfpathlineto{\pgfqpoint{6.692840in}{1.715000in}}%
\pgfpathlineto{\pgfqpoint{6.694080in}{1.890000in}}%
\pgfpathlineto{\pgfqpoint{6.695320in}{1.820000in}}%
\pgfpathlineto{\pgfqpoint{6.696560in}{1.855000in}}%
\pgfpathlineto{\pgfqpoint{6.697800in}{2.240000in}}%
\pgfpathlineto{\pgfqpoint{6.699040in}{1.995000in}}%
\pgfpathlineto{\pgfqpoint{6.700280in}{2.065000in}}%
\pgfpathlineto{\pgfqpoint{6.701520in}{2.065000in}}%
\pgfpathlineto{\pgfqpoint{6.702760in}{1.750000in}}%
\pgfpathlineto{\pgfqpoint{6.704000in}{2.065000in}}%
\pgfpathlineto{\pgfqpoint{6.705240in}{1.785000in}}%
\pgfpathlineto{\pgfqpoint{6.706480in}{2.205000in}}%
\pgfpathlineto{\pgfqpoint{6.707720in}{2.205000in}}%
\pgfpathlineto{\pgfqpoint{6.708960in}{1.890000in}}%
\pgfpathlineto{\pgfqpoint{6.710200in}{1.925000in}}%
\pgfpathlineto{\pgfqpoint{6.711440in}{2.065000in}}%
\pgfpathlineto{\pgfqpoint{6.712680in}{1.610000in}}%
\pgfpathlineto{\pgfqpoint{6.713920in}{1.855000in}}%
\pgfpathlineto{\pgfqpoint{6.715160in}{1.750000in}}%
\pgfpathlineto{\pgfqpoint{6.717640in}{1.995000in}}%
\pgfpathlineto{\pgfqpoint{6.718880in}{1.505000in}}%
\pgfpathlineto{\pgfqpoint{6.721360in}{2.100000in}}%
\pgfpathlineto{\pgfqpoint{6.722600in}{2.205000in}}%
\pgfpathlineto{\pgfqpoint{6.723840in}{1.785000in}}%
\pgfpathlineto{\pgfqpoint{6.726320in}{1.925000in}}%
\pgfpathlineto{\pgfqpoint{6.727560in}{1.855000in}}%
\pgfpathlineto{\pgfqpoint{6.728800in}{1.995000in}}%
\pgfpathlineto{\pgfqpoint{6.730040in}{1.855000in}}%
\pgfpathlineto{\pgfqpoint{6.731280in}{1.960000in}}%
\pgfpathlineto{\pgfqpoint{6.732520in}{1.925000in}}%
\pgfpathlineto{\pgfqpoint{6.733760in}{1.540000in}}%
\pgfpathlineto{\pgfqpoint{6.735000in}{1.820000in}}%
\pgfpathlineto{\pgfqpoint{6.736240in}{1.610000in}}%
\pgfpathlineto{\pgfqpoint{6.737480in}{1.645000in}}%
\pgfpathlineto{\pgfqpoint{6.738720in}{2.170000in}}%
\pgfpathlineto{\pgfqpoint{6.739960in}{2.100000in}}%
\pgfpathlineto{\pgfqpoint{6.741200in}{2.100000in}}%
\pgfpathlineto{\pgfqpoint{6.743680in}{1.575000in}}%
\pgfpathlineto{\pgfqpoint{6.744920in}{1.925000in}}%
\pgfpathlineto{\pgfqpoint{6.747400in}{1.645000in}}%
\pgfpathlineto{\pgfqpoint{6.748640in}{1.820000in}}%
\pgfpathlineto{\pgfqpoint{6.749880in}{1.610000in}}%
\pgfpathlineto{\pgfqpoint{6.751120in}{1.680000in}}%
\pgfpathlineto{\pgfqpoint{6.752360in}{1.575000in}}%
\pgfpathlineto{\pgfqpoint{6.753600in}{1.960000in}}%
\pgfpathlineto{\pgfqpoint{6.754840in}{1.995000in}}%
\pgfpathlineto{\pgfqpoint{6.756080in}{1.610000in}}%
\pgfpathlineto{\pgfqpoint{6.757320in}{2.065000in}}%
\pgfpathlineto{\pgfqpoint{6.758560in}{1.400000in}}%
\pgfpathlineto{\pgfqpoint{6.759800in}{1.645000in}}%
\pgfpathlineto{\pgfqpoint{6.761040in}{1.610000in}}%
\pgfpathlineto{\pgfqpoint{6.762280in}{1.610000in}}%
\pgfpathlineto{\pgfqpoint{6.763520in}{1.365000in}}%
\pgfpathlineto{\pgfqpoint{6.764760in}{1.960000in}}%
\pgfpathlineto{\pgfqpoint{6.766000in}{1.680000in}}%
\pgfpathlineto{\pgfqpoint{6.767240in}{1.890000in}}%
\pgfpathlineto{\pgfqpoint{6.768480in}{1.610000in}}%
\pgfpathlineto{\pgfqpoint{6.770960in}{1.855000in}}%
\pgfpathlineto{\pgfqpoint{6.772200in}{1.505000in}}%
\pgfpathlineto{\pgfqpoint{6.773440in}{1.610000in}}%
\pgfpathlineto{\pgfqpoint{6.774680in}{2.065000in}}%
\pgfpathlineto{\pgfqpoint{6.775920in}{1.645000in}}%
\pgfpathlineto{\pgfqpoint{6.777160in}{1.715000in}}%
\pgfpathlineto{\pgfqpoint{6.779640in}{2.065000in}}%
\pgfpathlineto{\pgfqpoint{6.780880in}{1.470000in}}%
\pgfpathlineto{\pgfqpoint{6.782120in}{1.680000in}}%
\pgfpathlineto{\pgfqpoint{6.783360in}{1.540000in}}%
\pgfpathlineto{\pgfqpoint{6.784600in}{1.610000in}}%
\pgfpathlineto{\pgfqpoint{6.785840in}{1.820000in}}%
\pgfpathlineto{\pgfqpoint{6.787080in}{1.540000in}}%
\pgfpathlineto{\pgfqpoint{6.788320in}{1.890000in}}%
\pgfpathlineto{\pgfqpoint{6.789560in}{1.715000in}}%
\pgfpathlineto{\pgfqpoint{6.790800in}{1.365000in}}%
\pgfpathlineto{\pgfqpoint{6.793280in}{1.995000in}}%
\pgfpathlineto{\pgfqpoint{6.794520in}{1.540000in}}%
\pgfpathlineto{\pgfqpoint{6.795760in}{2.345000in}}%
\pgfpathlineto{\pgfqpoint{6.797000in}{1.085000in}}%
\pgfpathlineto{\pgfqpoint{6.798240in}{1.260000in}}%
\pgfpathlineto{\pgfqpoint{6.799480in}{1.890000in}}%
\pgfpathlineto{\pgfqpoint{6.801960in}{1.715000in}}%
\pgfpathlineto{\pgfqpoint{6.803200in}{1.890000in}}%
\pgfpathlineto{\pgfqpoint{6.804440in}{1.820000in}}%
\pgfpathlineto{\pgfqpoint{6.805680in}{2.240000in}}%
\pgfpathlineto{\pgfqpoint{6.808160in}{1.715000in}}%
\pgfpathlineto{\pgfqpoint{6.809400in}{1.960000in}}%
\pgfpathlineto{\pgfqpoint{6.810640in}{1.680000in}}%
\pgfpathlineto{\pgfqpoint{6.813120in}{2.310000in}}%
\pgfpathlineto{\pgfqpoint{6.814360in}{1.715000in}}%
\pgfpathlineto{\pgfqpoint{6.815600in}{2.205000in}}%
\pgfpathlineto{\pgfqpoint{6.816840in}{2.065000in}}%
\pgfpathlineto{\pgfqpoint{6.818080in}{2.100000in}}%
\pgfpathlineto{\pgfqpoint{6.819320in}{1.750000in}}%
\pgfpathlineto{\pgfqpoint{6.820560in}{1.995000in}}%
\pgfpathlineto{\pgfqpoint{6.821800in}{1.610000in}}%
\pgfpathlineto{\pgfqpoint{6.823040in}{2.205000in}}%
\pgfpathlineto{\pgfqpoint{6.824280in}{1.575000in}}%
\pgfpathlineto{\pgfqpoint{6.825520in}{1.750000in}}%
\pgfpathlineto{\pgfqpoint{6.826760in}{1.610000in}}%
\pgfpathlineto{\pgfqpoint{6.830480in}{2.345000in}}%
\pgfpathlineto{\pgfqpoint{6.831720in}{1.855000in}}%
\pgfpathlineto{\pgfqpoint{6.832960in}{2.100000in}}%
\pgfpathlineto{\pgfqpoint{6.834200in}{1.540000in}}%
\pgfpathlineto{\pgfqpoint{6.835440in}{2.135000in}}%
\pgfpathlineto{\pgfqpoint{6.836680in}{2.135000in}}%
\pgfpathlineto{\pgfqpoint{6.837920in}{2.275000in}}%
\pgfpathlineto{\pgfqpoint{6.839160in}{2.030000in}}%
\pgfpathlineto{\pgfqpoint{6.841640in}{2.345000in}}%
\pgfpathlineto{\pgfqpoint{6.844120in}{1.645000in}}%
\pgfpathlineto{\pgfqpoint{6.845360in}{1.960000in}}%
\pgfpathlineto{\pgfqpoint{6.846600in}{1.575000in}}%
\pgfpathlineto{\pgfqpoint{6.847840in}{2.100000in}}%
\pgfpathlineto{\pgfqpoint{6.849080in}{1.785000in}}%
\pgfpathlineto{\pgfqpoint{6.850320in}{2.030000in}}%
\pgfpathlineto{\pgfqpoint{6.851560in}{1.820000in}}%
\pgfpathlineto{\pgfqpoint{6.852800in}{1.820000in}}%
\pgfpathlineto{\pgfqpoint{6.854040in}{1.645000in}}%
\pgfpathlineto{\pgfqpoint{6.856520in}{1.995000in}}%
\pgfpathlineto{\pgfqpoint{6.857760in}{1.960000in}}%
\pgfpathlineto{\pgfqpoint{6.859000in}{2.345000in}}%
\pgfpathlineto{\pgfqpoint{6.861480in}{1.995000in}}%
\pgfpathlineto{\pgfqpoint{6.862720in}{1.855000in}}%
\pgfpathlineto{\pgfqpoint{6.863960in}{2.065000in}}%
\pgfpathlineto{\pgfqpoint{6.865200in}{1.855000in}}%
\pgfpathlineto{\pgfqpoint{6.866440in}{2.100000in}}%
\pgfpathlineto{\pgfqpoint{6.867680in}{1.785000in}}%
\pgfpathlineto{\pgfqpoint{6.868920in}{2.065000in}}%
\pgfpathlineto{\pgfqpoint{6.870160in}{1.715000in}}%
\pgfpathlineto{\pgfqpoint{6.871400in}{1.750000in}}%
\pgfpathlineto{\pgfqpoint{6.872640in}{1.820000in}}%
\pgfpathlineto{\pgfqpoint{6.873880in}{1.960000in}}%
\pgfpathlineto{\pgfqpoint{6.875120in}{1.505000in}}%
\pgfpathlineto{\pgfqpoint{6.876360in}{2.100000in}}%
\pgfpathlineto{\pgfqpoint{6.877600in}{1.960000in}}%
\pgfpathlineto{\pgfqpoint{6.878840in}{2.205000in}}%
\pgfpathlineto{\pgfqpoint{6.880080in}{1.295000in}}%
\pgfpathlineto{\pgfqpoint{6.881320in}{2.030000in}}%
\pgfpathlineto{\pgfqpoint{6.882560in}{1.925000in}}%
\pgfpathlineto{\pgfqpoint{6.883800in}{1.715000in}}%
\pgfpathlineto{\pgfqpoint{6.885040in}{1.820000in}}%
\pgfpathlineto{\pgfqpoint{6.886280in}{1.750000in}}%
\pgfpathlineto{\pgfqpoint{6.887520in}{2.065000in}}%
\pgfpathlineto{\pgfqpoint{6.888760in}{1.960000in}}%
\pgfpathlineto{\pgfqpoint{6.891240in}{1.505000in}}%
\pgfpathlineto{\pgfqpoint{6.892480in}{1.715000in}}%
\pgfpathlineto{\pgfqpoint{6.893720in}{1.645000in}}%
\pgfpathlineto{\pgfqpoint{6.894960in}{1.505000in}}%
\pgfpathlineto{\pgfqpoint{6.896200in}{1.960000in}}%
\pgfpathlineto{\pgfqpoint{6.898680in}{1.470000in}}%
\pgfpathlineto{\pgfqpoint{6.899920in}{2.205000in}}%
\pgfpathlineto{\pgfqpoint{6.901160in}{1.715000in}}%
\pgfpathlineto{\pgfqpoint{6.902400in}{1.960000in}}%
\pgfpathlineto{\pgfqpoint{6.904880in}{1.575000in}}%
\pgfpathlineto{\pgfqpoint{6.907360in}{2.345000in}}%
\pgfpathlineto{\pgfqpoint{6.908600in}{2.345000in}}%
\pgfpathlineto{\pgfqpoint{6.912320in}{1.820000in}}%
\pgfpathlineto{\pgfqpoint{6.913560in}{1.890000in}}%
\pgfpathlineto{\pgfqpoint{6.914800in}{2.205000in}}%
\pgfpathlineto{\pgfqpoint{6.916040in}{1.750000in}}%
\pgfpathlineto{\pgfqpoint{6.917280in}{1.890000in}}%
\pgfpathlineto{\pgfqpoint{6.918520in}{2.205000in}}%
\pgfpathlineto{\pgfqpoint{6.919760in}{1.785000in}}%
\pgfpathlineto{\pgfqpoint{6.922240in}{2.100000in}}%
\pgfpathlineto{\pgfqpoint{6.923480in}{2.275000in}}%
\pgfpathlineto{\pgfqpoint{6.925960in}{2.065000in}}%
\pgfpathlineto{\pgfqpoint{6.927200in}{2.205000in}}%
\pgfpathlineto{\pgfqpoint{6.928440in}{1.715000in}}%
\pgfpathlineto{\pgfqpoint{6.929680in}{1.750000in}}%
\pgfpathlineto{\pgfqpoint{6.932160in}{1.890000in}}%
\pgfpathlineto{\pgfqpoint{6.934640in}{1.610000in}}%
\pgfpathlineto{\pgfqpoint{6.935880in}{2.135000in}}%
\pgfpathlineto{\pgfqpoint{6.937120in}{2.135000in}}%
\pgfpathlineto{\pgfqpoint{6.939600in}{1.680000in}}%
\pgfpathlineto{\pgfqpoint{6.940840in}{2.170000in}}%
\pgfpathlineto{\pgfqpoint{6.942080in}{2.065000in}}%
\pgfpathlineto{\pgfqpoint{6.943320in}{1.435000in}}%
\pgfpathlineto{\pgfqpoint{6.944560in}{1.750000in}}%
\pgfpathlineto{\pgfqpoint{6.945800in}{1.680000in}}%
\pgfpathlineto{\pgfqpoint{6.947040in}{2.345000in}}%
\pgfpathlineto{\pgfqpoint{6.949520in}{1.750000in}}%
\pgfpathlineto{\pgfqpoint{6.950760in}{1.925000in}}%
\pgfpathlineto{\pgfqpoint{6.952000in}{1.645000in}}%
\pgfpathlineto{\pgfqpoint{6.953240in}{1.925000in}}%
\pgfpathlineto{\pgfqpoint{6.954480in}{1.925000in}}%
\pgfpathlineto{\pgfqpoint{6.955720in}{2.240000in}}%
\pgfpathlineto{\pgfqpoint{6.956960in}{1.750000in}}%
\pgfpathlineto{\pgfqpoint{6.958200in}{1.785000in}}%
\pgfpathlineto{\pgfqpoint{6.960680in}{2.135000in}}%
\pgfpathlineto{\pgfqpoint{6.961920in}{2.275000in}}%
\pgfpathlineto{\pgfqpoint{6.963160in}{1.715000in}}%
\pgfpathlineto{\pgfqpoint{6.964400in}{2.030000in}}%
\pgfpathlineto{\pgfqpoint{6.965640in}{1.960000in}}%
\pgfpathlineto{\pgfqpoint{6.966880in}{2.065000in}}%
\pgfpathlineto{\pgfqpoint{6.968120in}{1.820000in}}%
\pgfpathlineto{\pgfqpoint{6.970600in}{1.995000in}}%
\pgfpathlineto{\pgfqpoint{6.971840in}{1.925000in}}%
\pgfpathlineto{\pgfqpoint{6.973080in}{2.065000in}}%
\pgfpathlineto{\pgfqpoint{6.974320in}{1.715000in}}%
\pgfpathlineto{\pgfqpoint{6.975560in}{2.135000in}}%
\pgfpathlineto{\pgfqpoint{6.976800in}{2.065000in}}%
\pgfpathlineto{\pgfqpoint{6.978040in}{2.275000in}}%
\pgfpathlineto{\pgfqpoint{6.979280in}{2.170000in}}%
\pgfpathlineto{\pgfqpoint{6.980520in}{1.715000in}}%
\pgfpathlineto{\pgfqpoint{6.981760in}{2.170000in}}%
\pgfpathlineto{\pgfqpoint{6.983000in}{2.065000in}}%
\pgfpathlineto{\pgfqpoint{6.984240in}{1.785000in}}%
\pgfpathlineto{\pgfqpoint{6.985480in}{1.820000in}}%
\pgfpathlineto{\pgfqpoint{6.987960in}{1.470000in}}%
\pgfpathlineto{\pgfqpoint{6.989200in}{1.505000in}}%
\pgfpathlineto{\pgfqpoint{6.990440in}{1.890000in}}%
\pgfpathlineto{\pgfqpoint{6.991680in}{1.680000in}}%
\pgfpathlineto{\pgfqpoint{6.992920in}{1.855000in}}%
\pgfpathlineto{\pgfqpoint{6.994160in}{1.505000in}}%
\pgfpathlineto{\pgfqpoint{6.996640in}{1.995000in}}%
\pgfpathlineto{\pgfqpoint{6.997880in}{1.855000in}}%
\pgfpathlineto{\pgfqpoint{6.999120in}{1.890000in}}%
\pgfpathlineto{\pgfqpoint{7.000360in}{1.820000in}}%
\pgfpathlineto{\pgfqpoint{7.001600in}{2.100000in}}%
\pgfpathlineto{\pgfqpoint{7.002840in}{1.680000in}}%
\pgfpathlineto{\pgfqpoint{7.004080in}{1.995000in}}%
\pgfpathlineto{\pgfqpoint{7.005320in}{1.785000in}}%
\pgfpathlineto{\pgfqpoint{7.006560in}{2.065000in}}%
\pgfpathlineto{\pgfqpoint{7.009040in}{1.505000in}}%
\pgfpathlineto{\pgfqpoint{7.011520in}{1.680000in}}%
\pgfpathlineto{\pgfqpoint{7.012760in}{1.610000in}}%
\pgfpathlineto{\pgfqpoint{7.014000in}{1.645000in}}%
\pgfpathlineto{\pgfqpoint{7.015240in}{1.960000in}}%
\pgfpathlineto{\pgfqpoint{7.016480in}{1.855000in}}%
\pgfpathlineto{\pgfqpoint{7.017720in}{1.960000in}}%
\pgfpathlineto{\pgfqpoint{7.018960in}{1.750000in}}%
\pgfpathlineto{\pgfqpoint{7.020200in}{2.100000in}}%
\pgfpathlineto{\pgfqpoint{7.021440in}{1.750000in}}%
\pgfpathlineto{\pgfqpoint{7.022680in}{1.855000in}}%
\pgfpathlineto{\pgfqpoint{7.023920in}{1.715000in}}%
\pgfpathlineto{\pgfqpoint{7.025160in}{1.820000in}}%
\pgfpathlineto{\pgfqpoint{7.026400in}{1.750000in}}%
\pgfpathlineto{\pgfqpoint{7.027640in}{1.330000in}}%
\pgfpathlineto{\pgfqpoint{7.028880in}{1.995000in}}%
\pgfpathlineto{\pgfqpoint{7.030120in}{1.960000in}}%
\pgfpathlineto{\pgfqpoint{7.031360in}{2.135000in}}%
\pgfpathlineto{\pgfqpoint{7.032600in}{2.030000in}}%
\pgfpathlineto{\pgfqpoint{7.033840in}{1.680000in}}%
\pgfpathlineto{\pgfqpoint{7.035080in}{1.680000in}}%
\pgfpathlineto{\pgfqpoint{7.036320in}{1.435000in}}%
\pgfpathlineto{\pgfqpoint{7.038800in}{2.135000in}}%
\pgfpathlineto{\pgfqpoint{7.040040in}{1.785000in}}%
\pgfpathlineto{\pgfqpoint{7.041280in}{2.170000in}}%
\pgfpathlineto{\pgfqpoint{7.042520in}{1.960000in}}%
\pgfpathlineto{\pgfqpoint{7.043760in}{2.030000in}}%
\pgfpathlineto{\pgfqpoint{7.045000in}{2.240000in}}%
\pgfpathlineto{\pgfqpoint{7.046240in}{1.505000in}}%
\pgfpathlineto{\pgfqpoint{7.048720in}{1.995000in}}%
\pgfpathlineto{\pgfqpoint{7.049960in}{1.750000in}}%
\pgfpathlineto{\pgfqpoint{7.051200in}{2.100000in}}%
\pgfpathlineto{\pgfqpoint{7.052440in}{1.960000in}}%
\pgfpathlineto{\pgfqpoint{7.053680in}{1.575000in}}%
\pgfpathlineto{\pgfqpoint{7.054920in}{2.100000in}}%
\pgfpathlineto{\pgfqpoint{7.056160in}{1.540000in}}%
\pgfpathlineto{\pgfqpoint{7.057400in}{1.995000in}}%
\pgfpathlineto{\pgfqpoint{7.058640in}{1.575000in}}%
\pgfpathlineto{\pgfqpoint{7.059880in}{2.065000in}}%
\pgfpathlineto{\pgfqpoint{7.061120in}{1.960000in}}%
\pgfpathlineto{\pgfqpoint{7.062360in}{2.135000in}}%
\pgfpathlineto{\pgfqpoint{7.063600in}{1.855000in}}%
\pgfpathlineto{\pgfqpoint{7.064840in}{2.100000in}}%
\pgfpathlineto{\pgfqpoint{7.066080in}{1.750000in}}%
\pgfpathlineto{\pgfqpoint{7.068560in}{2.030000in}}%
\pgfpathlineto{\pgfqpoint{7.069800in}{2.100000in}}%
\pgfpathlineto{\pgfqpoint{7.071040in}{2.100000in}}%
\pgfpathlineto{\pgfqpoint{7.072280in}{1.855000in}}%
\pgfpathlineto{\pgfqpoint{7.073520in}{2.275000in}}%
\pgfpathlineto{\pgfqpoint{7.077240in}{1.505000in}}%
\pgfpathlineto{\pgfqpoint{7.079720in}{1.715000in}}%
\pgfpathlineto{\pgfqpoint{7.080960in}{1.505000in}}%
\pgfpathlineto{\pgfqpoint{7.084680in}{2.100000in}}%
\pgfpathlineto{\pgfqpoint{7.087160in}{1.435000in}}%
\pgfpathlineto{\pgfqpoint{7.088400in}{1.680000in}}%
\pgfpathlineto{\pgfqpoint{7.089640in}{1.505000in}}%
\pgfpathlineto{\pgfqpoint{7.090880in}{2.135000in}}%
\pgfpathlineto{\pgfqpoint{7.093360in}{1.645000in}}%
\pgfpathlineto{\pgfqpoint{7.095840in}{1.890000in}}%
\pgfpathlineto{\pgfqpoint{7.098320in}{1.715000in}}%
\pgfpathlineto{\pgfqpoint{7.099560in}{2.170000in}}%
\pgfpathlineto{\pgfqpoint{7.100800in}{2.065000in}}%
\pgfpathlineto{\pgfqpoint{7.102040in}{1.750000in}}%
\pgfpathlineto{\pgfqpoint{7.104520in}{1.960000in}}%
\pgfpathlineto{\pgfqpoint{7.105760in}{1.470000in}}%
\pgfpathlineto{\pgfqpoint{7.107000in}{1.960000in}}%
\pgfpathlineto{\pgfqpoint{7.108240in}{1.610000in}}%
\pgfpathlineto{\pgfqpoint{7.110720in}{2.205000in}}%
\pgfpathlineto{\pgfqpoint{7.113200in}{1.750000in}}%
\pgfpathlineto{\pgfqpoint{7.114440in}{1.890000in}}%
\pgfpathlineto{\pgfqpoint{7.116920in}{1.435000in}}%
\pgfpathlineto{\pgfqpoint{7.118160in}{1.960000in}}%
\pgfpathlineto{\pgfqpoint{7.120640in}{1.785000in}}%
\pgfpathlineto{\pgfqpoint{7.121880in}{1.960000in}}%
\pgfpathlineto{\pgfqpoint{7.123120in}{1.645000in}}%
\pgfpathlineto{\pgfqpoint{7.124360in}{1.890000in}}%
\pgfpathlineto{\pgfqpoint{7.125600in}{1.820000in}}%
\pgfpathlineto{\pgfqpoint{7.126840in}{1.680000in}}%
\pgfpathlineto{\pgfqpoint{7.129320in}{1.820000in}}%
\pgfpathlineto{\pgfqpoint{7.130560in}{1.890000in}}%
\pgfpathlineto{\pgfqpoint{7.131800in}{2.135000in}}%
\pgfpathlineto{\pgfqpoint{7.133040in}{1.610000in}}%
\pgfpathlineto{\pgfqpoint{7.134280in}{1.820000in}}%
\pgfpathlineto{\pgfqpoint{7.135520in}{1.680000in}}%
\pgfpathlineto{\pgfqpoint{7.138000in}{1.890000in}}%
\pgfpathlineto{\pgfqpoint{7.139240in}{1.925000in}}%
\pgfpathlineto{\pgfqpoint{7.140480in}{1.575000in}}%
\pgfpathlineto{\pgfqpoint{7.141720in}{1.610000in}}%
\pgfpathlineto{\pgfqpoint{7.142960in}{2.100000in}}%
\pgfpathlineto{\pgfqpoint{7.144200in}{1.995000in}}%
\pgfpathlineto{\pgfqpoint{7.145440in}{2.030000in}}%
\pgfpathlineto{\pgfqpoint{7.146680in}{1.995000in}}%
\pgfpathlineto{\pgfqpoint{7.147920in}{2.030000in}}%
\pgfpathlineto{\pgfqpoint{7.149160in}{2.100000in}}%
\pgfpathlineto{\pgfqpoint{7.150400in}{1.855000in}}%
\pgfpathlineto{\pgfqpoint{7.151640in}{1.995000in}}%
\pgfpathlineto{\pgfqpoint{7.152880in}{1.890000in}}%
\pgfpathlineto{\pgfqpoint{7.155360in}{2.310000in}}%
\pgfpathlineto{\pgfqpoint{7.156600in}{2.240000in}}%
\pgfpathlineto{\pgfqpoint{7.159080in}{1.645000in}}%
\pgfpathlineto{\pgfqpoint{7.160320in}{1.925000in}}%
\pgfpathlineto{\pgfqpoint{7.161560in}{1.890000in}}%
\pgfpathlineto{\pgfqpoint{7.162800in}{2.205000in}}%
\pgfpathlineto{\pgfqpoint{7.164040in}{1.820000in}}%
\pgfpathlineto{\pgfqpoint{7.165280in}{2.415000in}}%
\pgfpathlineto{\pgfqpoint{7.167760in}{1.995000in}}%
\pgfpathlineto{\pgfqpoint{7.169000in}{1.960000in}}%
\pgfpathlineto{\pgfqpoint{7.170240in}{2.170000in}}%
\pgfpathlineto{\pgfqpoint{7.173960in}{1.890000in}}%
\pgfpathlineto{\pgfqpoint{7.175200in}{1.925000in}}%
\pgfpathlineto{\pgfqpoint{7.176440in}{1.855000in}}%
\pgfpathlineto{\pgfqpoint{7.178920in}{1.960000in}}%
\pgfpathlineto{\pgfqpoint{7.180160in}{1.995000in}}%
\pgfpathlineto{\pgfqpoint{7.181400in}{1.960000in}}%
\pgfpathlineto{\pgfqpoint{7.182640in}{1.540000in}}%
\pgfpathlineto{\pgfqpoint{7.183880in}{2.065000in}}%
\pgfpathlineto{\pgfqpoint{7.186360in}{1.785000in}}%
\pgfpathlineto{\pgfqpoint{7.187600in}{2.135000in}}%
\pgfpathlineto{\pgfqpoint{7.188840in}{2.030000in}}%
\pgfpathlineto{\pgfqpoint{7.190080in}{2.100000in}}%
\pgfpathlineto{\pgfqpoint{7.191320in}{1.995000in}}%
\pgfpathlineto{\pgfqpoint{7.192560in}{2.100000in}}%
\pgfpathlineto{\pgfqpoint{7.193800in}{2.100000in}}%
\pgfpathlineto{\pgfqpoint{7.195040in}{1.855000in}}%
\pgfpathlineto{\pgfqpoint{7.196280in}{1.855000in}}%
\pgfpathlineto{\pgfqpoint{7.197520in}{1.680000in}}%
\pgfpathlineto{\pgfqpoint{7.200000in}{2.065000in}}%
\pgfpathlineto{\pgfqpoint{7.200000in}{2.065000in}}%
\pgfusepath{stroke}%
\end{pgfscope}%
\begin{pgfscope}%
\pgfpathrectangle{\pgfqpoint{1.000000in}{0.350000in}}{\pgfqpoint{6.200000in}{2.800000in}} %
\pgfusepath{clip}%
\pgfsetrectcap%
\pgfsetroundjoin%
\pgfsetlinewidth{1.003750pt}%
\definecolor{currentstroke}{rgb}{0.000000,0.500000,0.000000}%
\pgfsetstrokecolor{currentstroke}%
\pgfsetdash{}{0pt}%
\pgfpathmoveto{\pgfqpoint{1.000000in}{2.415000in}}%
\pgfpathlineto{\pgfqpoint{1.001240in}{2.205000in}}%
\pgfpathlineto{\pgfqpoint{1.002480in}{2.450000in}}%
\pgfpathlineto{\pgfqpoint{1.003720in}{2.450000in}}%
\pgfpathlineto{\pgfqpoint{1.004960in}{2.275000in}}%
\pgfpathlineto{\pgfqpoint{1.007440in}{2.345000in}}%
\pgfpathlineto{\pgfqpoint{1.008680in}{2.275000in}}%
\pgfpathlineto{\pgfqpoint{1.009920in}{2.345000in}}%
\pgfpathlineto{\pgfqpoint{1.012400in}{2.170000in}}%
\pgfpathlineto{\pgfqpoint{1.013640in}{2.380000in}}%
\pgfpathlineto{\pgfqpoint{1.014880in}{2.240000in}}%
\pgfpathlineto{\pgfqpoint{1.016120in}{2.625000in}}%
\pgfpathlineto{\pgfqpoint{1.017360in}{2.590000in}}%
\pgfpathlineto{\pgfqpoint{1.018600in}{2.345000in}}%
\pgfpathlineto{\pgfqpoint{1.019840in}{2.590000in}}%
\pgfpathlineto{\pgfqpoint{1.021080in}{2.275000in}}%
\pgfpathlineto{\pgfqpoint{1.022320in}{2.415000in}}%
\pgfpathlineto{\pgfqpoint{1.024800in}{2.345000in}}%
\pgfpathlineto{\pgfqpoint{1.026040in}{2.380000in}}%
\pgfpathlineto{\pgfqpoint{1.027280in}{2.345000in}}%
\pgfpathlineto{\pgfqpoint{1.028520in}{2.520000in}}%
\pgfpathlineto{\pgfqpoint{1.029760in}{2.275000in}}%
\pgfpathlineto{\pgfqpoint{1.032240in}{2.695000in}}%
\pgfpathlineto{\pgfqpoint{1.034720in}{1.925000in}}%
\pgfpathlineto{\pgfqpoint{1.035960in}{2.590000in}}%
\pgfpathlineto{\pgfqpoint{1.037200in}{2.555000in}}%
\pgfpathlineto{\pgfqpoint{1.038440in}{2.590000in}}%
\pgfpathlineto{\pgfqpoint{1.039680in}{2.240000in}}%
\pgfpathlineto{\pgfqpoint{1.040920in}{2.380000in}}%
\pgfpathlineto{\pgfqpoint{1.042160in}{2.310000in}}%
\pgfpathlineto{\pgfqpoint{1.045880in}{2.765000in}}%
\pgfpathlineto{\pgfqpoint{1.047120in}{2.240000in}}%
\pgfpathlineto{\pgfqpoint{1.049600in}{2.485000in}}%
\pgfpathlineto{\pgfqpoint{1.052080in}{2.240000in}}%
\pgfpathlineto{\pgfqpoint{1.053320in}{2.240000in}}%
\pgfpathlineto{\pgfqpoint{1.054560in}{2.660000in}}%
\pgfpathlineto{\pgfqpoint{1.055800in}{2.520000in}}%
\pgfpathlineto{\pgfqpoint{1.057040in}{2.625000in}}%
\pgfpathlineto{\pgfqpoint{1.058280in}{2.520000in}}%
\pgfpathlineto{\pgfqpoint{1.059520in}{2.555000in}}%
\pgfpathlineto{\pgfqpoint{1.060760in}{1.995000in}}%
\pgfpathlineto{\pgfqpoint{1.063240in}{2.380000in}}%
\pgfpathlineto{\pgfqpoint{1.064480in}{2.100000in}}%
\pgfpathlineto{\pgfqpoint{1.066960in}{2.415000in}}%
\pgfpathlineto{\pgfqpoint{1.068200in}{2.485000in}}%
\pgfpathlineto{\pgfqpoint{1.069440in}{2.730000in}}%
\pgfpathlineto{\pgfqpoint{1.070680in}{2.590000in}}%
\pgfpathlineto{\pgfqpoint{1.071920in}{2.660000in}}%
\pgfpathlineto{\pgfqpoint{1.074400in}{2.065000in}}%
\pgfpathlineto{\pgfqpoint{1.075640in}{2.205000in}}%
\pgfpathlineto{\pgfqpoint{1.078120in}{2.555000in}}%
\pgfpathlineto{\pgfqpoint{1.079360in}{2.625000in}}%
\pgfpathlineto{\pgfqpoint{1.080600in}{2.590000in}}%
\pgfpathlineto{\pgfqpoint{1.081840in}{2.170000in}}%
\pgfpathlineto{\pgfqpoint{1.083080in}{2.695000in}}%
\pgfpathlineto{\pgfqpoint{1.085560in}{2.205000in}}%
\pgfpathlineto{\pgfqpoint{1.088040in}{2.625000in}}%
\pgfpathlineto{\pgfqpoint{1.089280in}{2.240000in}}%
\pgfpathlineto{\pgfqpoint{1.090520in}{2.590000in}}%
\pgfpathlineto{\pgfqpoint{1.091760in}{2.625000in}}%
\pgfpathlineto{\pgfqpoint{1.093000in}{2.555000in}}%
\pgfpathlineto{\pgfqpoint{1.094240in}{2.310000in}}%
\pgfpathlineto{\pgfqpoint{1.095480in}{2.415000in}}%
\pgfpathlineto{\pgfqpoint{1.096720in}{2.345000in}}%
\pgfpathlineto{\pgfqpoint{1.097960in}{2.415000in}}%
\pgfpathlineto{\pgfqpoint{1.099200in}{1.925000in}}%
\pgfpathlineto{\pgfqpoint{1.100440in}{2.485000in}}%
\pgfpathlineto{\pgfqpoint{1.101680in}{2.100000in}}%
\pgfpathlineto{\pgfqpoint{1.102920in}{2.345000in}}%
\pgfpathlineto{\pgfqpoint{1.104160in}{2.275000in}}%
\pgfpathlineto{\pgfqpoint{1.105400in}{2.450000in}}%
\pgfpathlineto{\pgfqpoint{1.106640in}{2.345000in}}%
\pgfpathlineto{\pgfqpoint{1.107880in}{2.415000in}}%
\pgfpathlineto{\pgfqpoint{1.109120in}{2.415000in}}%
\pgfpathlineto{\pgfqpoint{1.110360in}{2.485000in}}%
\pgfpathlineto{\pgfqpoint{1.111600in}{2.205000in}}%
\pgfpathlineto{\pgfqpoint{1.112840in}{2.310000in}}%
\pgfpathlineto{\pgfqpoint{1.114080in}{2.205000in}}%
\pgfpathlineto{\pgfqpoint{1.115320in}{2.380000in}}%
\pgfpathlineto{\pgfqpoint{1.116560in}{2.765000in}}%
\pgfpathlineto{\pgfqpoint{1.119040in}{2.555000in}}%
\pgfpathlineto{\pgfqpoint{1.120280in}{2.730000in}}%
\pgfpathlineto{\pgfqpoint{1.121520in}{2.135000in}}%
\pgfpathlineto{\pgfqpoint{1.122760in}{2.275000in}}%
\pgfpathlineto{\pgfqpoint{1.124000in}{2.240000in}}%
\pgfpathlineto{\pgfqpoint{1.125240in}{2.450000in}}%
\pgfpathlineto{\pgfqpoint{1.126480in}{2.275000in}}%
\pgfpathlineto{\pgfqpoint{1.130200in}{2.940000in}}%
\pgfpathlineto{\pgfqpoint{1.131440in}{2.520000in}}%
\pgfpathlineto{\pgfqpoint{1.132680in}{2.485000in}}%
\pgfpathlineto{\pgfqpoint{1.133920in}{2.275000in}}%
\pgfpathlineto{\pgfqpoint{1.138880in}{2.555000in}}%
\pgfpathlineto{\pgfqpoint{1.140120in}{2.205000in}}%
\pgfpathlineto{\pgfqpoint{1.141360in}{2.205000in}}%
\pgfpathlineto{\pgfqpoint{1.142600in}{2.590000in}}%
\pgfpathlineto{\pgfqpoint{1.143840in}{2.590000in}}%
\pgfpathlineto{\pgfqpoint{1.145080in}{2.240000in}}%
\pgfpathlineto{\pgfqpoint{1.146320in}{2.765000in}}%
\pgfpathlineto{\pgfqpoint{1.151280in}{2.030000in}}%
\pgfpathlineto{\pgfqpoint{1.152520in}{2.240000in}}%
\pgfpathlineto{\pgfqpoint{1.153760in}{2.765000in}}%
\pgfpathlineto{\pgfqpoint{1.157480in}{2.030000in}}%
\pgfpathlineto{\pgfqpoint{1.158720in}{2.380000in}}%
\pgfpathlineto{\pgfqpoint{1.159960in}{2.205000in}}%
\pgfpathlineto{\pgfqpoint{1.162440in}{2.765000in}}%
\pgfpathlineto{\pgfqpoint{1.164920in}{2.170000in}}%
\pgfpathlineto{\pgfqpoint{1.167400in}{2.590000in}}%
\pgfpathlineto{\pgfqpoint{1.168640in}{2.205000in}}%
\pgfpathlineto{\pgfqpoint{1.169880in}{2.170000in}}%
\pgfpathlineto{\pgfqpoint{1.171120in}{2.800000in}}%
\pgfpathlineto{\pgfqpoint{1.172360in}{2.625000in}}%
\pgfpathlineto{\pgfqpoint{1.173600in}{2.205000in}}%
\pgfpathlineto{\pgfqpoint{1.174840in}{2.310000in}}%
\pgfpathlineto{\pgfqpoint{1.176080in}{2.765000in}}%
\pgfpathlineto{\pgfqpoint{1.177320in}{2.170000in}}%
\pgfpathlineto{\pgfqpoint{1.178560in}{2.240000in}}%
\pgfpathlineto{\pgfqpoint{1.181040in}{2.660000in}}%
\pgfpathlineto{\pgfqpoint{1.182280in}{2.240000in}}%
\pgfpathlineto{\pgfqpoint{1.183520in}{2.240000in}}%
\pgfpathlineto{\pgfqpoint{1.184760in}{2.205000in}}%
\pgfpathlineto{\pgfqpoint{1.186000in}{1.995000in}}%
\pgfpathlineto{\pgfqpoint{1.187240in}{1.995000in}}%
\pgfpathlineto{\pgfqpoint{1.189720in}{2.345000in}}%
\pgfpathlineto{\pgfqpoint{1.190960in}{2.345000in}}%
\pgfpathlineto{\pgfqpoint{1.192200in}{2.135000in}}%
\pgfpathlineto{\pgfqpoint{1.193440in}{2.695000in}}%
\pgfpathlineto{\pgfqpoint{1.194680in}{2.555000in}}%
\pgfpathlineto{\pgfqpoint{1.195920in}{2.555000in}}%
\pgfpathlineto{\pgfqpoint{1.197160in}{2.380000in}}%
\pgfpathlineto{\pgfqpoint{1.198400in}{2.800000in}}%
\pgfpathlineto{\pgfqpoint{1.199640in}{2.835000in}}%
\pgfpathlineto{\pgfqpoint{1.200880in}{2.520000in}}%
\pgfpathlineto{\pgfqpoint{1.202120in}{2.730000in}}%
\pgfpathlineto{\pgfqpoint{1.203360in}{2.695000in}}%
\pgfpathlineto{\pgfqpoint{1.205840in}{2.100000in}}%
\pgfpathlineto{\pgfqpoint{1.207080in}{2.135000in}}%
\pgfpathlineto{\pgfqpoint{1.210800in}{2.590000in}}%
\pgfpathlineto{\pgfqpoint{1.213280in}{2.345000in}}%
\pgfpathlineto{\pgfqpoint{1.215760in}{2.905000in}}%
\pgfpathlineto{\pgfqpoint{1.217000in}{2.590000in}}%
\pgfpathlineto{\pgfqpoint{1.218240in}{2.835000in}}%
\pgfpathlineto{\pgfqpoint{1.219480in}{2.730000in}}%
\pgfpathlineto{\pgfqpoint{1.220720in}{2.940000in}}%
\pgfpathlineto{\pgfqpoint{1.221960in}{2.450000in}}%
\pgfpathlineto{\pgfqpoint{1.224440in}{2.520000in}}%
\pgfpathlineto{\pgfqpoint{1.225680in}{2.520000in}}%
\pgfpathlineto{\pgfqpoint{1.226920in}{1.855000in}}%
\pgfpathlineto{\pgfqpoint{1.228160in}{2.485000in}}%
\pgfpathlineto{\pgfqpoint{1.229400in}{2.100000in}}%
\pgfpathlineto{\pgfqpoint{1.230640in}{2.380000in}}%
\pgfpathlineto{\pgfqpoint{1.233120in}{2.170000in}}%
\pgfpathlineto{\pgfqpoint{1.234360in}{2.520000in}}%
\pgfpathlineto{\pgfqpoint{1.235600in}{2.415000in}}%
\pgfpathlineto{\pgfqpoint{1.236840in}{2.485000in}}%
\pgfpathlineto{\pgfqpoint{1.238080in}{2.205000in}}%
\pgfpathlineto{\pgfqpoint{1.239320in}{2.660000in}}%
\pgfpathlineto{\pgfqpoint{1.240560in}{2.345000in}}%
\pgfpathlineto{\pgfqpoint{1.241800in}{2.590000in}}%
\pgfpathlineto{\pgfqpoint{1.243040in}{2.520000in}}%
\pgfpathlineto{\pgfqpoint{1.244280in}{2.555000in}}%
\pgfpathlineto{\pgfqpoint{1.245520in}{2.345000in}}%
\pgfpathlineto{\pgfqpoint{1.246760in}{2.555000in}}%
\pgfpathlineto{\pgfqpoint{1.248000in}{2.415000in}}%
\pgfpathlineto{\pgfqpoint{1.249240in}{2.520000in}}%
\pgfpathlineto{\pgfqpoint{1.250480in}{2.345000in}}%
\pgfpathlineto{\pgfqpoint{1.251720in}{2.450000in}}%
\pgfpathlineto{\pgfqpoint{1.252960in}{2.835000in}}%
\pgfpathlineto{\pgfqpoint{1.255440in}{2.450000in}}%
\pgfpathlineto{\pgfqpoint{1.256680in}{2.520000in}}%
\pgfpathlineto{\pgfqpoint{1.257920in}{2.415000in}}%
\pgfpathlineto{\pgfqpoint{1.259160in}{2.100000in}}%
\pgfpathlineto{\pgfqpoint{1.261640in}{2.520000in}}%
\pgfpathlineto{\pgfqpoint{1.262880in}{2.520000in}}%
\pgfpathlineto{\pgfqpoint{1.264120in}{2.345000in}}%
\pgfpathlineto{\pgfqpoint{1.265360in}{2.765000in}}%
\pgfpathlineto{\pgfqpoint{1.266600in}{2.345000in}}%
\pgfpathlineto{\pgfqpoint{1.267840in}{2.345000in}}%
\pgfpathlineto{\pgfqpoint{1.269080in}{2.380000in}}%
\pgfpathlineto{\pgfqpoint{1.270320in}{2.205000in}}%
\pgfpathlineto{\pgfqpoint{1.271560in}{2.310000in}}%
\pgfpathlineto{\pgfqpoint{1.272800in}{2.205000in}}%
\pgfpathlineto{\pgfqpoint{1.274040in}{2.240000in}}%
\pgfpathlineto{\pgfqpoint{1.276520in}{2.590000in}}%
\pgfpathlineto{\pgfqpoint{1.277760in}{2.765000in}}%
\pgfpathlineto{\pgfqpoint{1.279000in}{2.310000in}}%
\pgfpathlineto{\pgfqpoint{1.280240in}{2.450000in}}%
\pgfpathlineto{\pgfqpoint{1.281480in}{2.345000in}}%
\pgfpathlineto{\pgfqpoint{1.282720in}{2.345000in}}%
\pgfpathlineto{\pgfqpoint{1.283960in}{2.485000in}}%
\pgfpathlineto{\pgfqpoint{1.285200in}{2.380000in}}%
\pgfpathlineto{\pgfqpoint{1.286440in}{2.415000in}}%
\pgfpathlineto{\pgfqpoint{1.287680in}{2.415000in}}%
\pgfpathlineto{\pgfqpoint{1.288920in}{2.555000in}}%
\pgfpathlineto{\pgfqpoint{1.290160in}{2.275000in}}%
\pgfpathlineto{\pgfqpoint{1.291400in}{2.660000in}}%
\pgfpathlineto{\pgfqpoint{1.292640in}{2.415000in}}%
\pgfpathlineto{\pgfqpoint{1.293880in}{2.555000in}}%
\pgfpathlineto{\pgfqpoint{1.296360in}{2.205000in}}%
\pgfpathlineto{\pgfqpoint{1.297600in}{2.240000in}}%
\pgfpathlineto{\pgfqpoint{1.298840in}{2.030000in}}%
\pgfpathlineto{\pgfqpoint{1.300080in}{2.205000in}}%
\pgfpathlineto{\pgfqpoint{1.301320in}{2.695000in}}%
\pgfpathlineto{\pgfqpoint{1.302560in}{2.660000in}}%
\pgfpathlineto{\pgfqpoint{1.303800in}{2.555000in}}%
\pgfpathlineto{\pgfqpoint{1.305040in}{2.800000in}}%
\pgfpathlineto{\pgfqpoint{1.306280in}{2.485000in}}%
\pgfpathlineto{\pgfqpoint{1.307520in}{2.695000in}}%
\pgfpathlineto{\pgfqpoint{1.308760in}{2.520000in}}%
\pgfpathlineto{\pgfqpoint{1.310000in}{2.695000in}}%
\pgfpathlineto{\pgfqpoint{1.311240in}{2.345000in}}%
\pgfpathlineto{\pgfqpoint{1.314960in}{2.870000in}}%
\pgfpathlineto{\pgfqpoint{1.318680in}{2.520000in}}%
\pgfpathlineto{\pgfqpoint{1.319920in}{2.590000in}}%
\pgfpathlineto{\pgfqpoint{1.321160in}{2.310000in}}%
\pgfpathlineto{\pgfqpoint{1.322400in}{2.380000in}}%
\pgfpathlineto{\pgfqpoint{1.323640in}{2.555000in}}%
\pgfpathlineto{\pgfqpoint{1.324880in}{2.415000in}}%
\pgfpathlineto{\pgfqpoint{1.327360in}{2.730000in}}%
\pgfpathlineto{\pgfqpoint{1.328600in}{2.555000in}}%
\pgfpathlineto{\pgfqpoint{1.329840in}{2.800000in}}%
\pgfpathlineto{\pgfqpoint{1.331080in}{2.555000in}}%
\pgfpathlineto{\pgfqpoint{1.332320in}{2.695000in}}%
\pgfpathlineto{\pgfqpoint{1.333560in}{2.555000in}}%
\pgfpathlineto{\pgfqpoint{1.334800in}{2.695000in}}%
\pgfpathlineto{\pgfqpoint{1.336040in}{2.975000in}}%
\pgfpathlineto{\pgfqpoint{1.337280in}{2.485000in}}%
\pgfpathlineto{\pgfqpoint{1.339760in}{2.765000in}}%
\pgfpathlineto{\pgfqpoint{1.342240in}{2.240000in}}%
\pgfpathlineto{\pgfqpoint{1.343480in}{2.695000in}}%
\pgfpathlineto{\pgfqpoint{1.344720in}{2.695000in}}%
\pgfpathlineto{\pgfqpoint{1.345960in}{2.345000in}}%
\pgfpathlineto{\pgfqpoint{1.348440in}{2.730000in}}%
\pgfpathlineto{\pgfqpoint{1.349680in}{2.520000in}}%
\pgfpathlineto{\pgfqpoint{1.350920in}{2.520000in}}%
\pgfpathlineto{\pgfqpoint{1.352160in}{2.800000in}}%
\pgfpathlineto{\pgfqpoint{1.353400in}{2.485000in}}%
\pgfpathlineto{\pgfqpoint{1.354640in}{2.765000in}}%
\pgfpathlineto{\pgfqpoint{1.357120in}{2.205000in}}%
\pgfpathlineto{\pgfqpoint{1.358360in}{2.205000in}}%
\pgfpathlineto{\pgfqpoint{1.360840in}{2.520000in}}%
\pgfpathlineto{\pgfqpoint{1.363320in}{2.065000in}}%
\pgfpathlineto{\pgfqpoint{1.364560in}{2.590000in}}%
\pgfpathlineto{\pgfqpoint{1.367040in}{2.275000in}}%
\pgfpathlineto{\pgfqpoint{1.368280in}{2.450000in}}%
\pgfpathlineto{\pgfqpoint{1.369520in}{2.310000in}}%
\pgfpathlineto{\pgfqpoint{1.370760in}{2.520000in}}%
\pgfpathlineto{\pgfqpoint{1.372000in}{2.100000in}}%
\pgfpathlineto{\pgfqpoint{1.376960in}{2.695000in}}%
\pgfpathlineto{\pgfqpoint{1.378200in}{2.065000in}}%
\pgfpathlineto{\pgfqpoint{1.379440in}{2.485000in}}%
\pgfpathlineto{\pgfqpoint{1.381920in}{2.135000in}}%
\pgfpathlineto{\pgfqpoint{1.383160in}{2.135000in}}%
\pgfpathlineto{\pgfqpoint{1.384400in}{1.855000in}}%
\pgfpathlineto{\pgfqpoint{1.385640in}{2.485000in}}%
\pgfpathlineto{\pgfqpoint{1.386880in}{2.415000in}}%
\pgfpathlineto{\pgfqpoint{1.388120in}{2.520000in}}%
\pgfpathlineto{\pgfqpoint{1.389360in}{2.170000in}}%
\pgfpathlineto{\pgfqpoint{1.390600in}{2.800000in}}%
\pgfpathlineto{\pgfqpoint{1.393080in}{2.555000in}}%
\pgfpathlineto{\pgfqpoint{1.394320in}{2.590000in}}%
\pgfpathlineto{\pgfqpoint{1.395560in}{2.485000in}}%
\pgfpathlineto{\pgfqpoint{1.396800in}{2.135000in}}%
\pgfpathlineto{\pgfqpoint{1.398040in}{2.170000in}}%
\pgfpathlineto{\pgfqpoint{1.399280in}{2.590000in}}%
\pgfpathlineto{\pgfqpoint{1.401760in}{2.135000in}}%
\pgfpathlineto{\pgfqpoint{1.404240in}{2.520000in}}%
\pgfpathlineto{\pgfqpoint{1.406720in}{2.450000in}}%
\pgfpathlineto{\pgfqpoint{1.407960in}{2.695000in}}%
\pgfpathlineto{\pgfqpoint{1.409200in}{2.695000in}}%
\pgfpathlineto{\pgfqpoint{1.410440in}{2.625000in}}%
\pgfpathlineto{\pgfqpoint{1.411680in}{2.625000in}}%
\pgfpathlineto{\pgfqpoint{1.412920in}{2.415000in}}%
\pgfpathlineto{\pgfqpoint{1.414160in}{2.765000in}}%
\pgfpathlineto{\pgfqpoint{1.416640in}{2.205000in}}%
\pgfpathlineto{\pgfqpoint{1.417880in}{2.380000in}}%
\pgfpathlineto{\pgfqpoint{1.419120in}{2.940000in}}%
\pgfpathlineto{\pgfqpoint{1.421600in}{2.100000in}}%
\pgfpathlineto{\pgfqpoint{1.425320in}{2.765000in}}%
\pgfpathlineto{\pgfqpoint{1.426560in}{2.555000in}}%
\pgfpathlineto{\pgfqpoint{1.427800in}{2.590000in}}%
\pgfpathlineto{\pgfqpoint{1.429040in}{3.010000in}}%
\pgfpathlineto{\pgfqpoint{1.431520in}{2.275000in}}%
\pgfpathlineto{\pgfqpoint{1.432760in}{2.450000in}}%
\pgfpathlineto{\pgfqpoint{1.434000in}{2.030000in}}%
\pgfpathlineto{\pgfqpoint{1.435240in}{2.380000in}}%
\pgfpathlineto{\pgfqpoint{1.436480in}{2.205000in}}%
\pgfpathlineto{\pgfqpoint{1.438960in}{2.555000in}}%
\pgfpathlineto{\pgfqpoint{1.441440in}{1.960000in}}%
\pgfpathlineto{\pgfqpoint{1.443920in}{2.380000in}}%
\pgfpathlineto{\pgfqpoint{1.445160in}{2.450000in}}%
\pgfpathlineto{\pgfqpoint{1.446400in}{2.170000in}}%
\pgfpathlineto{\pgfqpoint{1.447640in}{2.415000in}}%
\pgfpathlineto{\pgfqpoint{1.448880in}{2.065000in}}%
\pgfpathlineto{\pgfqpoint{1.450120in}{2.065000in}}%
\pgfpathlineto{\pgfqpoint{1.452600in}{2.310000in}}%
\pgfpathlineto{\pgfqpoint{1.455080in}{1.820000in}}%
\pgfpathlineto{\pgfqpoint{1.456320in}{1.995000in}}%
\pgfpathlineto{\pgfqpoint{1.457560in}{2.345000in}}%
\pgfpathlineto{\pgfqpoint{1.458800in}{2.275000in}}%
\pgfpathlineto{\pgfqpoint{1.460040in}{2.135000in}}%
\pgfpathlineto{\pgfqpoint{1.461280in}{2.275000in}}%
\pgfpathlineto{\pgfqpoint{1.462520in}{2.100000in}}%
\pgfpathlineto{\pgfqpoint{1.463760in}{2.275000in}}%
\pgfpathlineto{\pgfqpoint{1.465000in}{2.135000in}}%
\pgfpathlineto{\pgfqpoint{1.466240in}{2.205000in}}%
\pgfpathlineto{\pgfqpoint{1.467480in}{2.205000in}}%
\pgfpathlineto{\pgfqpoint{1.468720in}{2.380000in}}%
\pgfpathlineto{\pgfqpoint{1.469960in}{2.310000in}}%
\pgfpathlineto{\pgfqpoint{1.471200in}{2.485000in}}%
\pgfpathlineto{\pgfqpoint{1.472440in}{2.450000in}}%
\pgfpathlineto{\pgfqpoint{1.473680in}{2.485000in}}%
\pgfpathlineto{\pgfqpoint{1.474920in}{2.625000in}}%
\pgfpathlineto{\pgfqpoint{1.476160in}{2.275000in}}%
\pgfpathlineto{\pgfqpoint{1.477400in}{2.275000in}}%
\pgfpathlineto{\pgfqpoint{1.478640in}{2.660000in}}%
\pgfpathlineto{\pgfqpoint{1.481120in}{2.310000in}}%
\pgfpathlineto{\pgfqpoint{1.483600in}{2.030000in}}%
\pgfpathlineto{\pgfqpoint{1.486080in}{2.415000in}}%
\pgfpathlineto{\pgfqpoint{1.488560in}{2.345000in}}%
\pgfpathlineto{\pgfqpoint{1.491040in}{2.625000in}}%
\pgfpathlineto{\pgfqpoint{1.492280in}{2.590000in}}%
\pgfpathlineto{\pgfqpoint{1.494760in}{1.925000in}}%
\pgfpathlineto{\pgfqpoint{1.497240in}{2.170000in}}%
\pgfpathlineto{\pgfqpoint{1.498480in}{2.065000in}}%
\pgfpathlineto{\pgfqpoint{1.500960in}{2.345000in}}%
\pgfpathlineto{\pgfqpoint{1.503440in}{2.345000in}}%
\pgfpathlineto{\pgfqpoint{1.504680in}{2.590000in}}%
\pgfpathlineto{\pgfqpoint{1.505920in}{2.555000in}}%
\pgfpathlineto{\pgfqpoint{1.507160in}{2.555000in}}%
\pgfpathlineto{\pgfqpoint{1.509640in}{2.380000in}}%
\pgfpathlineto{\pgfqpoint{1.512120in}{2.555000in}}%
\pgfpathlineto{\pgfqpoint{1.513360in}{2.275000in}}%
\pgfpathlineto{\pgfqpoint{1.514600in}{2.590000in}}%
\pgfpathlineto{\pgfqpoint{1.517080in}{2.240000in}}%
\pgfpathlineto{\pgfqpoint{1.518320in}{2.345000in}}%
\pgfpathlineto{\pgfqpoint{1.520800in}{2.170000in}}%
\pgfpathlineto{\pgfqpoint{1.523280in}{2.730000in}}%
\pgfpathlineto{\pgfqpoint{1.524520in}{2.275000in}}%
\pgfpathlineto{\pgfqpoint{1.525760in}{2.240000in}}%
\pgfpathlineto{\pgfqpoint{1.527000in}{2.310000in}}%
\pgfpathlineto{\pgfqpoint{1.528240in}{2.275000in}}%
\pgfpathlineto{\pgfqpoint{1.529480in}{2.415000in}}%
\pgfpathlineto{\pgfqpoint{1.531960in}{2.415000in}}%
\pgfpathlineto{\pgfqpoint{1.533200in}{2.240000in}}%
\pgfpathlineto{\pgfqpoint{1.534440in}{2.415000in}}%
\pgfpathlineto{\pgfqpoint{1.535680in}{2.415000in}}%
\pgfpathlineto{\pgfqpoint{1.536920in}{2.345000in}}%
\pgfpathlineto{\pgfqpoint{1.538160in}{2.485000in}}%
\pgfpathlineto{\pgfqpoint{1.539400in}{2.450000in}}%
\pgfpathlineto{\pgfqpoint{1.540640in}{2.310000in}}%
\pgfpathlineto{\pgfqpoint{1.541880in}{2.310000in}}%
\pgfpathlineto{\pgfqpoint{1.543120in}{2.135000in}}%
\pgfpathlineto{\pgfqpoint{1.545600in}{2.660000in}}%
\pgfpathlineto{\pgfqpoint{1.546840in}{2.135000in}}%
\pgfpathlineto{\pgfqpoint{1.548080in}{2.555000in}}%
\pgfpathlineto{\pgfqpoint{1.549320in}{2.345000in}}%
\pgfpathlineto{\pgfqpoint{1.550560in}{2.450000in}}%
\pgfpathlineto{\pgfqpoint{1.553040in}{2.100000in}}%
\pgfpathlineto{\pgfqpoint{1.555520in}{2.695000in}}%
\pgfpathlineto{\pgfqpoint{1.556760in}{2.590000in}}%
\pgfpathlineto{\pgfqpoint{1.558000in}{2.135000in}}%
\pgfpathlineto{\pgfqpoint{1.559240in}{2.520000in}}%
\pgfpathlineto{\pgfqpoint{1.561720in}{2.275000in}}%
\pgfpathlineto{\pgfqpoint{1.564200in}{2.520000in}}%
\pgfpathlineto{\pgfqpoint{1.565440in}{2.415000in}}%
\pgfpathlineto{\pgfqpoint{1.566680in}{2.415000in}}%
\pgfpathlineto{\pgfqpoint{1.567920in}{2.835000in}}%
\pgfpathlineto{\pgfqpoint{1.569160in}{2.450000in}}%
\pgfpathlineto{\pgfqpoint{1.570400in}{2.520000in}}%
\pgfpathlineto{\pgfqpoint{1.571640in}{2.660000in}}%
\pgfpathlineto{\pgfqpoint{1.572880in}{2.520000in}}%
\pgfpathlineto{\pgfqpoint{1.574120in}{2.240000in}}%
\pgfpathlineto{\pgfqpoint{1.576600in}{2.485000in}}%
\pgfpathlineto{\pgfqpoint{1.577840in}{2.380000in}}%
\pgfpathlineto{\pgfqpoint{1.579080in}{2.415000in}}%
\pgfpathlineto{\pgfqpoint{1.581560in}{2.660000in}}%
\pgfpathlineto{\pgfqpoint{1.582800in}{2.485000in}}%
\pgfpathlineto{\pgfqpoint{1.584040in}{2.590000in}}%
\pgfpathlineto{\pgfqpoint{1.585280in}{2.590000in}}%
\pgfpathlineto{\pgfqpoint{1.587760in}{2.380000in}}%
\pgfpathlineto{\pgfqpoint{1.589000in}{2.450000in}}%
\pgfpathlineto{\pgfqpoint{1.590240in}{2.380000in}}%
\pgfpathlineto{\pgfqpoint{1.591480in}{2.205000in}}%
\pgfpathlineto{\pgfqpoint{1.592720in}{2.275000in}}%
\pgfpathlineto{\pgfqpoint{1.593960in}{2.275000in}}%
\pgfpathlineto{\pgfqpoint{1.595200in}{2.310000in}}%
\pgfpathlineto{\pgfqpoint{1.597680in}{2.660000in}}%
\pgfpathlineto{\pgfqpoint{1.600160in}{2.205000in}}%
\pgfpathlineto{\pgfqpoint{1.601400in}{2.415000in}}%
\pgfpathlineto{\pgfqpoint{1.602640in}{2.275000in}}%
\pgfpathlineto{\pgfqpoint{1.603880in}{2.520000in}}%
\pgfpathlineto{\pgfqpoint{1.605120in}{2.415000in}}%
\pgfpathlineto{\pgfqpoint{1.606360in}{2.625000in}}%
\pgfpathlineto{\pgfqpoint{1.607600in}{2.380000in}}%
\pgfpathlineto{\pgfqpoint{1.608840in}{2.800000in}}%
\pgfpathlineto{\pgfqpoint{1.610080in}{1.960000in}}%
\pgfpathlineto{\pgfqpoint{1.612560in}{2.520000in}}%
\pgfpathlineto{\pgfqpoint{1.613800in}{2.275000in}}%
\pgfpathlineto{\pgfqpoint{1.615040in}{2.310000in}}%
\pgfpathlineto{\pgfqpoint{1.617520in}{2.135000in}}%
\pgfpathlineto{\pgfqpoint{1.618760in}{2.345000in}}%
\pgfpathlineto{\pgfqpoint{1.620000in}{2.310000in}}%
\pgfpathlineto{\pgfqpoint{1.621240in}{2.345000in}}%
\pgfpathlineto{\pgfqpoint{1.622480in}{2.520000in}}%
\pgfpathlineto{\pgfqpoint{1.624960in}{1.890000in}}%
\pgfpathlineto{\pgfqpoint{1.627440in}{2.590000in}}%
\pgfpathlineto{\pgfqpoint{1.629920in}{2.240000in}}%
\pgfpathlineto{\pgfqpoint{1.631160in}{2.310000in}}%
\pgfpathlineto{\pgfqpoint{1.632400in}{2.170000in}}%
\pgfpathlineto{\pgfqpoint{1.634880in}{2.835000in}}%
\pgfpathlineto{\pgfqpoint{1.636120in}{2.730000in}}%
\pgfpathlineto{\pgfqpoint{1.637360in}{2.240000in}}%
\pgfpathlineto{\pgfqpoint{1.638600in}{2.310000in}}%
\pgfpathlineto{\pgfqpoint{1.639840in}{2.170000in}}%
\pgfpathlineto{\pgfqpoint{1.642320in}{2.450000in}}%
\pgfpathlineto{\pgfqpoint{1.643560in}{2.240000in}}%
\pgfpathlineto{\pgfqpoint{1.647280in}{2.730000in}}%
\pgfpathlineto{\pgfqpoint{1.649760in}{2.345000in}}%
\pgfpathlineto{\pgfqpoint{1.651000in}{2.835000in}}%
\pgfpathlineto{\pgfqpoint{1.652240in}{2.415000in}}%
\pgfpathlineto{\pgfqpoint{1.653480in}{2.520000in}}%
\pgfpathlineto{\pgfqpoint{1.654720in}{2.310000in}}%
\pgfpathlineto{\pgfqpoint{1.655960in}{2.380000in}}%
\pgfpathlineto{\pgfqpoint{1.657200in}{2.730000in}}%
\pgfpathlineto{\pgfqpoint{1.660920in}{2.450000in}}%
\pgfpathlineto{\pgfqpoint{1.662160in}{2.415000in}}%
\pgfpathlineto{\pgfqpoint{1.663400in}{2.555000in}}%
\pgfpathlineto{\pgfqpoint{1.664640in}{2.555000in}}%
\pgfpathlineto{\pgfqpoint{1.667120in}{2.275000in}}%
\pgfpathlineto{\pgfqpoint{1.669600in}{2.555000in}}%
\pgfpathlineto{\pgfqpoint{1.672080in}{2.415000in}}%
\pgfpathlineto{\pgfqpoint{1.673320in}{2.170000in}}%
\pgfpathlineto{\pgfqpoint{1.674560in}{2.590000in}}%
\pgfpathlineto{\pgfqpoint{1.675800in}{2.345000in}}%
\pgfpathlineto{\pgfqpoint{1.677040in}{2.450000in}}%
\pgfpathlineto{\pgfqpoint{1.678280in}{2.870000in}}%
\pgfpathlineto{\pgfqpoint{1.679520in}{2.800000in}}%
\pgfpathlineto{\pgfqpoint{1.680760in}{2.380000in}}%
\pgfpathlineto{\pgfqpoint{1.683240in}{2.590000in}}%
\pgfpathlineto{\pgfqpoint{1.684480in}{2.520000in}}%
\pgfpathlineto{\pgfqpoint{1.685720in}{2.590000in}}%
\pgfpathlineto{\pgfqpoint{1.686960in}{2.205000in}}%
\pgfpathlineto{\pgfqpoint{1.688200in}{2.170000in}}%
\pgfpathlineto{\pgfqpoint{1.689440in}{2.205000in}}%
\pgfpathlineto{\pgfqpoint{1.690680in}{2.590000in}}%
\pgfpathlineto{\pgfqpoint{1.693160in}{2.170000in}}%
\pgfpathlineto{\pgfqpoint{1.694400in}{2.205000in}}%
\pgfpathlineto{\pgfqpoint{1.695640in}{2.030000in}}%
\pgfpathlineto{\pgfqpoint{1.698120in}{2.555000in}}%
\pgfpathlineto{\pgfqpoint{1.699360in}{2.520000in}}%
\pgfpathlineto{\pgfqpoint{1.701840in}{2.030000in}}%
\pgfpathlineto{\pgfqpoint{1.703080in}{2.485000in}}%
\pgfpathlineto{\pgfqpoint{1.704320in}{2.520000in}}%
\pgfpathlineto{\pgfqpoint{1.705560in}{2.590000in}}%
\pgfpathlineto{\pgfqpoint{1.706800in}{2.170000in}}%
\pgfpathlineto{\pgfqpoint{1.708040in}{2.345000in}}%
\pgfpathlineto{\pgfqpoint{1.709280in}{2.275000in}}%
\pgfpathlineto{\pgfqpoint{1.710520in}{2.065000in}}%
\pgfpathlineto{\pgfqpoint{1.711760in}{2.730000in}}%
\pgfpathlineto{\pgfqpoint{1.713000in}{2.275000in}}%
\pgfpathlineto{\pgfqpoint{1.714240in}{2.380000in}}%
\pgfpathlineto{\pgfqpoint{1.716720in}{2.065000in}}%
\pgfpathlineto{\pgfqpoint{1.719200in}{2.520000in}}%
\pgfpathlineto{\pgfqpoint{1.720440in}{2.100000in}}%
\pgfpathlineto{\pgfqpoint{1.722920in}{2.660000in}}%
\pgfpathlineto{\pgfqpoint{1.724160in}{2.485000in}}%
\pgfpathlineto{\pgfqpoint{1.725400in}{2.730000in}}%
\pgfpathlineto{\pgfqpoint{1.726640in}{2.660000in}}%
\pgfpathlineto{\pgfqpoint{1.729120in}{2.275000in}}%
\pgfpathlineto{\pgfqpoint{1.731600in}{2.415000in}}%
\pgfpathlineto{\pgfqpoint{1.732840in}{2.415000in}}%
\pgfpathlineto{\pgfqpoint{1.734080in}{2.170000in}}%
\pgfpathlineto{\pgfqpoint{1.735320in}{2.485000in}}%
\pgfpathlineto{\pgfqpoint{1.736560in}{2.450000in}}%
\pgfpathlineto{\pgfqpoint{1.737800in}{2.485000in}}%
\pgfpathlineto{\pgfqpoint{1.739040in}{2.765000in}}%
\pgfpathlineto{\pgfqpoint{1.740280in}{2.415000in}}%
\pgfpathlineto{\pgfqpoint{1.741520in}{2.520000in}}%
\pgfpathlineto{\pgfqpoint{1.742760in}{2.065000in}}%
\pgfpathlineto{\pgfqpoint{1.745240in}{2.450000in}}%
\pgfpathlineto{\pgfqpoint{1.746480in}{2.415000in}}%
\pgfpathlineto{\pgfqpoint{1.747720in}{2.100000in}}%
\pgfpathlineto{\pgfqpoint{1.748960in}{2.520000in}}%
\pgfpathlineto{\pgfqpoint{1.750200in}{2.205000in}}%
\pgfpathlineto{\pgfqpoint{1.753920in}{2.765000in}}%
\pgfpathlineto{\pgfqpoint{1.755160in}{2.520000in}}%
\pgfpathlineto{\pgfqpoint{1.757640in}{2.765000in}}%
\pgfpathlineto{\pgfqpoint{1.758880in}{2.485000in}}%
\pgfpathlineto{\pgfqpoint{1.761360in}{2.835000in}}%
\pgfpathlineto{\pgfqpoint{1.763840in}{2.555000in}}%
\pgfpathlineto{\pgfqpoint{1.765080in}{2.555000in}}%
\pgfpathlineto{\pgfqpoint{1.767560in}{2.170000in}}%
\pgfpathlineto{\pgfqpoint{1.768800in}{2.730000in}}%
\pgfpathlineto{\pgfqpoint{1.771280in}{2.345000in}}%
\pgfpathlineto{\pgfqpoint{1.772520in}{2.660000in}}%
\pgfpathlineto{\pgfqpoint{1.773760in}{2.590000in}}%
\pgfpathlineto{\pgfqpoint{1.775000in}{2.590000in}}%
\pgfpathlineto{\pgfqpoint{1.776240in}{1.960000in}}%
\pgfpathlineto{\pgfqpoint{1.777480in}{2.555000in}}%
\pgfpathlineto{\pgfqpoint{1.778720in}{2.415000in}}%
\pgfpathlineto{\pgfqpoint{1.779960in}{2.555000in}}%
\pgfpathlineto{\pgfqpoint{1.781200in}{2.485000in}}%
\pgfpathlineto{\pgfqpoint{1.782440in}{2.730000in}}%
\pgfpathlineto{\pgfqpoint{1.783680in}{2.380000in}}%
\pgfpathlineto{\pgfqpoint{1.784920in}{2.485000in}}%
\pgfpathlineto{\pgfqpoint{1.786160in}{2.310000in}}%
\pgfpathlineto{\pgfqpoint{1.788640in}{2.590000in}}%
\pgfpathlineto{\pgfqpoint{1.791120in}{2.275000in}}%
\pgfpathlineto{\pgfqpoint{1.792360in}{2.310000in}}%
\pgfpathlineto{\pgfqpoint{1.794840in}{2.730000in}}%
\pgfpathlineto{\pgfqpoint{1.796080in}{1.995000in}}%
\pgfpathlineto{\pgfqpoint{1.797320in}{2.555000in}}%
\pgfpathlineto{\pgfqpoint{1.798560in}{2.590000in}}%
\pgfpathlineto{\pgfqpoint{1.801040in}{2.100000in}}%
\pgfpathlineto{\pgfqpoint{1.802280in}{2.135000in}}%
\pgfpathlineto{\pgfqpoint{1.803520in}{2.625000in}}%
\pgfpathlineto{\pgfqpoint{1.804760in}{2.415000in}}%
\pgfpathlineto{\pgfqpoint{1.806000in}{2.555000in}}%
\pgfpathlineto{\pgfqpoint{1.807240in}{2.170000in}}%
\pgfpathlineto{\pgfqpoint{1.808480in}{2.205000in}}%
\pgfpathlineto{\pgfqpoint{1.809720in}{2.170000in}}%
\pgfpathlineto{\pgfqpoint{1.812200in}{2.870000in}}%
\pgfpathlineto{\pgfqpoint{1.814680in}{2.310000in}}%
\pgfpathlineto{\pgfqpoint{1.815920in}{2.625000in}}%
\pgfpathlineto{\pgfqpoint{1.817160in}{2.065000in}}%
\pgfpathlineto{\pgfqpoint{1.819640in}{2.415000in}}%
\pgfpathlineto{\pgfqpoint{1.820880in}{2.380000in}}%
\pgfpathlineto{\pgfqpoint{1.822120in}{2.415000in}}%
\pgfpathlineto{\pgfqpoint{1.823360in}{2.380000in}}%
\pgfpathlineto{\pgfqpoint{1.824600in}{2.555000in}}%
\pgfpathlineto{\pgfqpoint{1.825840in}{2.310000in}}%
\pgfpathlineto{\pgfqpoint{1.827080in}{2.415000in}}%
\pgfpathlineto{\pgfqpoint{1.828320in}{2.205000in}}%
\pgfpathlineto{\pgfqpoint{1.830800in}{2.765000in}}%
\pgfpathlineto{\pgfqpoint{1.834520in}{2.240000in}}%
\pgfpathlineto{\pgfqpoint{1.835760in}{2.380000in}}%
\pgfpathlineto{\pgfqpoint{1.837000in}{2.345000in}}%
\pgfpathlineto{\pgfqpoint{1.839480in}{2.205000in}}%
\pgfpathlineto{\pgfqpoint{1.841960in}{2.380000in}}%
\pgfpathlineto{\pgfqpoint{1.843200in}{2.415000in}}%
\pgfpathlineto{\pgfqpoint{1.844440in}{2.765000in}}%
\pgfpathlineto{\pgfqpoint{1.845680in}{2.590000in}}%
\pgfpathlineto{\pgfqpoint{1.846920in}{2.135000in}}%
\pgfpathlineto{\pgfqpoint{1.848160in}{2.205000in}}%
\pgfpathlineto{\pgfqpoint{1.849400in}{2.590000in}}%
\pgfpathlineto{\pgfqpoint{1.850640in}{2.380000in}}%
\pgfpathlineto{\pgfqpoint{1.851880in}{3.045000in}}%
\pgfpathlineto{\pgfqpoint{1.853120in}{2.310000in}}%
\pgfpathlineto{\pgfqpoint{1.854360in}{2.415000in}}%
\pgfpathlineto{\pgfqpoint{1.855600in}{2.065000in}}%
\pgfpathlineto{\pgfqpoint{1.856840in}{2.275000in}}%
\pgfpathlineto{\pgfqpoint{1.858080in}{2.100000in}}%
\pgfpathlineto{\pgfqpoint{1.859320in}{2.485000in}}%
\pgfpathlineto{\pgfqpoint{1.860560in}{2.310000in}}%
\pgfpathlineto{\pgfqpoint{1.861800in}{1.960000in}}%
\pgfpathlineto{\pgfqpoint{1.863040in}{2.310000in}}%
\pgfpathlineto{\pgfqpoint{1.864280in}{2.240000in}}%
\pgfpathlineto{\pgfqpoint{1.865520in}{2.590000in}}%
\pgfpathlineto{\pgfqpoint{1.866760in}{2.170000in}}%
\pgfpathlineto{\pgfqpoint{1.869240in}{2.450000in}}%
\pgfpathlineto{\pgfqpoint{1.871720in}{2.135000in}}%
\pgfpathlineto{\pgfqpoint{1.872960in}{2.170000in}}%
\pgfpathlineto{\pgfqpoint{1.874200in}{1.960000in}}%
\pgfpathlineto{\pgfqpoint{1.875440in}{2.310000in}}%
\pgfpathlineto{\pgfqpoint{1.876680in}{1.925000in}}%
\pgfpathlineto{\pgfqpoint{1.877920in}{1.925000in}}%
\pgfpathlineto{\pgfqpoint{1.879160in}{2.450000in}}%
\pgfpathlineto{\pgfqpoint{1.881640in}{2.100000in}}%
\pgfpathlineto{\pgfqpoint{1.882880in}{2.380000in}}%
\pgfpathlineto{\pgfqpoint{1.887840in}{2.065000in}}%
\pgfpathlineto{\pgfqpoint{1.889080in}{2.275000in}}%
\pgfpathlineto{\pgfqpoint{1.890320in}{2.205000in}}%
\pgfpathlineto{\pgfqpoint{1.891560in}{2.345000in}}%
\pgfpathlineto{\pgfqpoint{1.894040in}{2.135000in}}%
\pgfpathlineto{\pgfqpoint{1.895280in}{2.205000in}}%
\pgfpathlineto{\pgfqpoint{1.896520in}{2.205000in}}%
\pgfpathlineto{\pgfqpoint{1.899000in}{1.855000in}}%
\pgfpathlineto{\pgfqpoint{1.900240in}{2.485000in}}%
\pgfpathlineto{\pgfqpoint{1.901480in}{2.450000in}}%
\pgfpathlineto{\pgfqpoint{1.902720in}{2.520000in}}%
\pgfpathlineto{\pgfqpoint{1.906440in}{2.100000in}}%
\pgfpathlineto{\pgfqpoint{1.907680in}{2.415000in}}%
\pgfpathlineto{\pgfqpoint{1.908920in}{2.205000in}}%
\pgfpathlineto{\pgfqpoint{1.910160in}{2.625000in}}%
\pgfpathlineto{\pgfqpoint{1.911400in}{2.345000in}}%
\pgfpathlineto{\pgfqpoint{1.912640in}{2.380000in}}%
\pgfpathlineto{\pgfqpoint{1.913880in}{2.660000in}}%
\pgfpathlineto{\pgfqpoint{1.915120in}{2.555000in}}%
\pgfpathlineto{\pgfqpoint{1.916360in}{2.345000in}}%
\pgfpathlineto{\pgfqpoint{1.917600in}{2.485000in}}%
\pgfpathlineto{\pgfqpoint{1.920080in}{2.205000in}}%
\pgfpathlineto{\pgfqpoint{1.921320in}{2.275000in}}%
\pgfpathlineto{\pgfqpoint{1.922560in}{2.415000in}}%
\pgfpathlineto{\pgfqpoint{1.925040in}{2.065000in}}%
\pgfpathlineto{\pgfqpoint{1.926280in}{2.030000in}}%
\pgfpathlineto{\pgfqpoint{1.927520in}{1.960000in}}%
\pgfpathlineto{\pgfqpoint{1.930000in}{2.205000in}}%
\pgfpathlineto{\pgfqpoint{1.931240in}{2.730000in}}%
\pgfpathlineto{\pgfqpoint{1.932480in}{2.625000in}}%
\pgfpathlineto{\pgfqpoint{1.933720in}{2.660000in}}%
\pgfpathlineto{\pgfqpoint{1.934960in}{2.625000in}}%
\pgfpathlineto{\pgfqpoint{1.936200in}{2.555000in}}%
\pgfpathlineto{\pgfqpoint{1.937440in}{2.310000in}}%
\pgfpathlineto{\pgfqpoint{1.938680in}{2.660000in}}%
\pgfpathlineto{\pgfqpoint{1.939920in}{2.625000in}}%
\pgfpathlineto{\pgfqpoint{1.941160in}{2.275000in}}%
\pgfpathlineto{\pgfqpoint{1.942400in}{2.275000in}}%
\pgfpathlineto{\pgfqpoint{1.943640in}{2.625000in}}%
\pgfpathlineto{\pgfqpoint{1.944880in}{2.450000in}}%
\pgfpathlineto{\pgfqpoint{1.946120in}{2.520000in}}%
\pgfpathlineto{\pgfqpoint{1.947360in}{2.485000in}}%
\pgfpathlineto{\pgfqpoint{1.948600in}{2.415000in}}%
\pgfpathlineto{\pgfqpoint{1.949840in}{2.065000in}}%
\pgfpathlineto{\pgfqpoint{1.951080in}{2.450000in}}%
\pgfpathlineto{\pgfqpoint{1.952320in}{2.415000in}}%
\pgfpathlineto{\pgfqpoint{1.953560in}{2.310000in}}%
\pgfpathlineto{\pgfqpoint{1.956040in}{2.555000in}}%
\pgfpathlineto{\pgfqpoint{1.958520in}{2.135000in}}%
\pgfpathlineto{\pgfqpoint{1.959760in}{2.135000in}}%
\pgfpathlineto{\pgfqpoint{1.961000in}{2.415000in}}%
\pgfpathlineto{\pgfqpoint{1.962240in}{2.030000in}}%
\pgfpathlineto{\pgfqpoint{1.963480in}{2.205000in}}%
\pgfpathlineto{\pgfqpoint{1.964720in}{2.870000in}}%
\pgfpathlineto{\pgfqpoint{1.967200in}{2.380000in}}%
\pgfpathlineto{\pgfqpoint{1.968440in}{2.415000in}}%
\pgfpathlineto{\pgfqpoint{1.969680in}{2.170000in}}%
\pgfpathlineto{\pgfqpoint{1.970920in}{2.380000in}}%
\pgfpathlineto{\pgfqpoint{1.973400in}{2.170000in}}%
\pgfpathlineto{\pgfqpoint{1.975880in}{2.625000in}}%
\pgfpathlineto{\pgfqpoint{1.977120in}{2.380000in}}%
\pgfpathlineto{\pgfqpoint{1.978360in}{2.590000in}}%
\pgfpathlineto{\pgfqpoint{1.979600in}{2.380000in}}%
\pgfpathlineto{\pgfqpoint{1.982080in}{2.660000in}}%
\pgfpathlineto{\pgfqpoint{1.983320in}{2.660000in}}%
\pgfpathlineto{\pgfqpoint{1.984560in}{2.590000in}}%
\pgfpathlineto{\pgfqpoint{1.985800in}{2.450000in}}%
\pgfpathlineto{\pgfqpoint{1.990760in}{2.450000in}}%
\pgfpathlineto{\pgfqpoint{1.992000in}{2.380000in}}%
\pgfpathlineto{\pgfqpoint{1.994480in}{2.625000in}}%
\pgfpathlineto{\pgfqpoint{1.995720in}{2.695000in}}%
\pgfpathlineto{\pgfqpoint{1.996960in}{2.310000in}}%
\pgfpathlineto{\pgfqpoint{1.998200in}{2.310000in}}%
\pgfpathlineto{\pgfqpoint{1.999440in}{2.030000in}}%
\pgfpathlineto{\pgfqpoint{2.000680in}{2.660000in}}%
\pgfpathlineto{\pgfqpoint{2.001920in}{2.450000in}}%
\pgfpathlineto{\pgfqpoint{2.003160in}{2.765000in}}%
\pgfpathlineto{\pgfqpoint{2.004400in}{2.450000in}}%
\pgfpathlineto{\pgfqpoint{2.005640in}{2.660000in}}%
\pgfpathlineto{\pgfqpoint{2.006880in}{2.170000in}}%
\pgfpathlineto{\pgfqpoint{2.008120in}{2.170000in}}%
\pgfpathlineto{\pgfqpoint{2.009360in}{2.310000in}}%
\pgfpathlineto{\pgfqpoint{2.010600in}{2.170000in}}%
\pgfpathlineto{\pgfqpoint{2.011840in}{2.485000in}}%
\pgfpathlineto{\pgfqpoint{2.013080in}{2.310000in}}%
\pgfpathlineto{\pgfqpoint{2.014320in}{2.555000in}}%
\pgfpathlineto{\pgfqpoint{2.016800in}{2.100000in}}%
\pgfpathlineto{\pgfqpoint{2.018040in}{2.520000in}}%
\pgfpathlineto{\pgfqpoint{2.019280in}{2.135000in}}%
\pgfpathlineto{\pgfqpoint{2.020520in}{2.205000in}}%
\pgfpathlineto{\pgfqpoint{2.021760in}{2.345000in}}%
\pgfpathlineto{\pgfqpoint{2.023000in}{2.345000in}}%
\pgfpathlineto{\pgfqpoint{2.024240in}{2.205000in}}%
\pgfpathlineto{\pgfqpoint{2.025480in}{2.590000in}}%
\pgfpathlineto{\pgfqpoint{2.026720in}{2.415000in}}%
\pgfpathlineto{\pgfqpoint{2.027960in}{2.450000in}}%
\pgfpathlineto{\pgfqpoint{2.029200in}{2.555000in}}%
\pgfpathlineto{\pgfqpoint{2.030440in}{2.415000in}}%
\pgfpathlineto{\pgfqpoint{2.031680in}{2.520000in}}%
\pgfpathlineto{\pgfqpoint{2.032920in}{2.345000in}}%
\pgfpathlineto{\pgfqpoint{2.034160in}{2.555000in}}%
\pgfpathlineto{\pgfqpoint{2.035400in}{2.240000in}}%
\pgfpathlineto{\pgfqpoint{2.036640in}{2.450000in}}%
\pgfpathlineto{\pgfqpoint{2.039120in}{2.205000in}}%
\pgfpathlineto{\pgfqpoint{2.040360in}{2.310000in}}%
\pgfpathlineto{\pgfqpoint{2.041600in}{2.520000in}}%
\pgfpathlineto{\pgfqpoint{2.042840in}{2.345000in}}%
\pgfpathlineto{\pgfqpoint{2.044080in}{2.380000in}}%
\pgfpathlineto{\pgfqpoint{2.045320in}{2.380000in}}%
\pgfpathlineto{\pgfqpoint{2.046560in}{2.345000in}}%
\pgfpathlineto{\pgfqpoint{2.047800in}{2.555000in}}%
\pgfpathlineto{\pgfqpoint{2.049040in}{2.415000in}}%
\pgfpathlineto{\pgfqpoint{2.050280in}{2.555000in}}%
\pgfpathlineto{\pgfqpoint{2.051520in}{2.030000in}}%
\pgfpathlineto{\pgfqpoint{2.055240in}{2.555000in}}%
\pgfpathlineto{\pgfqpoint{2.057720in}{2.135000in}}%
\pgfpathlineto{\pgfqpoint{2.061440in}{2.520000in}}%
\pgfpathlineto{\pgfqpoint{2.062680in}{2.835000in}}%
\pgfpathlineto{\pgfqpoint{2.063920in}{2.275000in}}%
\pgfpathlineto{\pgfqpoint{2.066400in}{2.450000in}}%
\pgfpathlineto{\pgfqpoint{2.067640in}{1.995000in}}%
\pgfpathlineto{\pgfqpoint{2.070120in}{2.485000in}}%
\pgfpathlineto{\pgfqpoint{2.071360in}{2.065000in}}%
\pgfpathlineto{\pgfqpoint{2.073840in}{2.555000in}}%
\pgfpathlineto{\pgfqpoint{2.075080in}{2.590000in}}%
\pgfpathlineto{\pgfqpoint{2.076320in}{2.415000in}}%
\pgfpathlineto{\pgfqpoint{2.077560in}{2.415000in}}%
\pgfpathlineto{\pgfqpoint{2.078800in}{2.590000in}}%
\pgfpathlineto{\pgfqpoint{2.080040in}{2.520000in}}%
\pgfpathlineto{\pgfqpoint{2.081280in}{2.275000in}}%
\pgfpathlineto{\pgfqpoint{2.082520in}{2.380000in}}%
\pgfpathlineto{\pgfqpoint{2.083760in}{2.695000in}}%
\pgfpathlineto{\pgfqpoint{2.085000in}{2.240000in}}%
\pgfpathlineto{\pgfqpoint{2.086240in}{2.800000in}}%
\pgfpathlineto{\pgfqpoint{2.087480in}{2.660000in}}%
\pgfpathlineto{\pgfqpoint{2.088720in}{2.205000in}}%
\pgfpathlineto{\pgfqpoint{2.089960in}{2.240000in}}%
\pgfpathlineto{\pgfqpoint{2.091200in}{2.590000in}}%
\pgfpathlineto{\pgfqpoint{2.092440in}{2.555000in}}%
\pgfpathlineto{\pgfqpoint{2.093680in}{2.695000in}}%
\pgfpathlineto{\pgfqpoint{2.094920in}{2.695000in}}%
\pgfpathlineto{\pgfqpoint{2.096160in}{2.135000in}}%
\pgfpathlineto{\pgfqpoint{2.098640in}{2.450000in}}%
\pgfpathlineto{\pgfqpoint{2.099880in}{2.555000in}}%
\pgfpathlineto{\pgfqpoint{2.101120in}{2.275000in}}%
\pgfpathlineto{\pgfqpoint{2.102360in}{2.660000in}}%
\pgfpathlineto{\pgfqpoint{2.103600in}{2.415000in}}%
\pgfpathlineto{\pgfqpoint{2.104840in}{2.590000in}}%
\pgfpathlineto{\pgfqpoint{2.106080in}{2.415000in}}%
\pgfpathlineto{\pgfqpoint{2.107320in}{2.415000in}}%
\pgfpathlineto{\pgfqpoint{2.108560in}{2.275000in}}%
\pgfpathlineto{\pgfqpoint{2.109800in}{2.520000in}}%
\pgfpathlineto{\pgfqpoint{2.111040in}{2.415000in}}%
\pgfpathlineto{\pgfqpoint{2.112280in}{1.995000in}}%
\pgfpathlineto{\pgfqpoint{2.114760in}{2.520000in}}%
\pgfpathlineto{\pgfqpoint{2.118480in}{2.345000in}}%
\pgfpathlineto{\pgfqpoint{2.119720in}{2.415000in}}%
\pgfpathlineto{\pgfqpoint{2.120960in}{2.415000in}}%
\pgfpathlineto{\pgfqpoint{2.122200in}{2.660000in}}%
\pgfpathlineto{\pgfqpoint{2.123440in}{2.555000in}}%
\pgfpathlineto{\pgfqpoint{2.124680in}{2.345000in}}%
\pgfpathlineto{\pgfqpoint{2.125920in}{2.590000in}}%
\pgfpathlineto{\pgfqpoint{2.127160in}{2.240000in}}%
\pgfpathlineto{\pgfqpoint{2.128400in}{2.275000in}}%
\pgfpathlineto{\pgfqpoint{2.129640in}{2.240000in}}%
\pgfpathlineto{\pgfqpoint{2.130880in}{2.030000in}}%
\pgfpathlineto{\pgfqpoint{2.133360in}{2.555000in}}%
\pgfpathlineto{\pgfqpoint{2.134600in}{2.310000in}}%
\pgfpathlineto{\pgfqpoint{2.135840in}{2.625000in}}%
\pgfpathlineto{\pgfqpoint{2.138320in}{2.275000in}}%
\pgfpathlineto{\pgfqpoint{2.140800in}{2.800000in}}%
\pgfpathlineto{\pgfqpoint{2.142040in}{2.065000in}}%
\pgfpathlineto{\pgfqpoint{2.143280in}{2.030000in}}%
\pgfpathlineto{\pgfqpoint{2.144520in}{2.415000in}}%
\pgfpathlineto{\pgfqpoint{2.145760in}{2.345000in}}%
\pgfpathlineto{\pgfqpoint{2.147000in}{2.415000in}}%
\pgfpathlineto{\pgfqpoint{2.148240in}{2.590000in}}%
\pgfpathlineto{\pgfqpoint{2.149480in}{2.520000in}}%
\pgfpathlineto{\pgfqpoint{2.150720in}{2.275000in}}%
\pgfpathlineto{\pgfqpoint{2.153200in}{2.590000in}}%
\pgfpathlineto{\pgfqpoint{2.154440in}{2.800000in}}%
\pgfpathlineto{\pgfqpoint{2.155680in}{2.555000in}}%
\pgfpathlineto{\pgfqpoint{2.156920in}{2.660000in}}%
\pgfpathlineto{\pgfqpoint{2.158160in}{2.450000in}}%
\pgfpathlineto{\pgfqpoint{2.159400in}{2.555000in}}%
\pgfpathlineto{\pgfqpoint{2.160640in}{2.310000in}}%
\pgfpathlineto{\pgfqpoint{2.161880in}{2.415000in}}%
\pgfpathlineto{\pgfqpoint{2.163120in}{2.415000in}}%
\pgfpathlineto{\pgfqpoint{2.164360in}{2.800000in}}%
\pgfpathlineto{\pgfqpoint{2.165600in}{2.695000in}}%
\pgfpathlineto{\pgfqpoint{2.166840in}{2.695000in}}%
\pgfpathlineto{\pgfqpoint{2.170560in}{2.310000in}}%
\pgfpathlineto{\pgfqpoint{2.171800in}{2.870000in}}%
\pgfpathlineto{\pgfqpoint{2.173040in}{2.800000in}}%
\pgfpathlineto{\pgfqpoint{2.174280in}{2.555000in}}%
\pgfpathlineto{\pgfqpoint{2.175520in}{2.870000in}}%
\pgfpathlineto{\pgfqpoint{2.176760in}{2.765000in}}%
\pgfpathlineto{\pgfqpoint{2.178000in}{2.765000in}}%
\pgfpathlineto{\pgfqpoint{2.179240in}{2.660000in}}%
\pgfpathlineto{\pgfqpoint{2.180480in}{2.730000in}}%
\pgfpathlineto{\pgfqpoint{2.181720in}{2.660000in}}%
\pgfpathlineto{\pgfqpoint{2.184200in}{2.030000in}}%
\pgfpathlineto{\pgfqpoint{2.185440in}{2.555000in}}%
\pgfpathlineto{\pgfqpoint{2.186680in}{2.555000in}}%
\pgfpathlineto{\pgfqpoint{2.187920in}{2.240000in}}%
\pgfpathlineto{\pgfqpoint{2.189160in}{2.310000in}}%
\pgfpathlineto{\pgfqpoint{2.190400in}{2.730000in}}%
\pgfpathlineto{\pgfqpoint{2.192880in}{2.275000in}}%
\pgfpathlineto{\pgfqpoint{2.196600in}{2.730000in}}%
\pgfpathlineto{\pgfqpoint{2.199080in}{2.205000in}}%
\pgfpathlineto{\pgfqpoint{2.201560in}{2.695000in}}%
\pgfpathlineto{\pgfqpoint{2.202800in}{2.660000in}}%
\pgfpathlineto{\pgfqpoint{2.204040in}{2.765000in}}%
\pgfpathlineto{\pgfqpoint{2.206520in}{2.275000in}}%
\pgfpathlineto{\pgfqpoint{2.207760in}{2.520000in}}%
\pgfpathlineto{\pgfqpoint{2.209000in}{2.415000in}}%
\pgfpathlineto{\pgfqpoint{2.210240in}{2.135000in}}%
\pgfpathlineto{\pgfqpoint{2.212720in}{2.345000in}}%
\pgfpathlineto{\pgfqpoint{2.213960in}{2.240000in}}%
\pgfpathlineto{\pgfqpoint{2.217680in}{2.625000in}}%
\pgfpathlineto{\pgfqpoint{2.218920in}{2.345000in}}%
\pgfpathlineto{\pgfqpoint{2.220160in}{2.555000in}}%
\pgfpathlineto{\pgfqpoint{2.222640in}{2.275000in}}%
\pgfpathlineto{\pgfqpoint{2.223880in}{2.380000in}}%
\pgfpathlineto{\pgfqpoint{2.225120in}{2.345000in}}%
\pgfpathlineto{\pgfqpoint{2.226360in}{2.380000in}}%
\pgfpathlineto{\pgfqpoint{2.227600in}{2.100000in}}%
\pgfpathlineto{\pgfqpoint{2.228840in}{2.240000in}}%
\pgfpathlineto{\pgfqpoint{2.230080in}{2.905000in}}%
\pgfpathlineto{\pgfqpoint{2.232560in}{2.310000in}}%
\pgfpathlineto{\pgfqpoint{2.233800in}{2.485000in}}%
\pgfpathlineto{\pgfqpoint{2.235040in}{2.975000in}}%
\pgfpathlineto{\pgfqpoint{2.237520in}{2.450000in}}%
\pgfpathlineto{\pgfqpoint{2.238760in}{2.765000in}}%
\pgfpathlineto{\pgfqpoint{2.240000in}{1.960000in}}%
\pgfpathlineto{\pgfqpoint{2.242480in}{2.520000in}}%
\pgfpathlineto{\pgfqpoint{2.244960in}{2.765000in}}%
\pgfpathlineto{\pgfqpoint{2.247440in}{2.345000in}}%
\pgfpathlineto{\pgfqpoint{2.248680in}{2.345000in}}%
\pgfpathlineto{\pgfqpoint{2.249920in}{2.275000in}}%
\pgfpathlineto{\pgfqpoint{2.251160in}{2.765000in}}%
\pgfpathlineto{\pgfqpoint{2.252400in}{2.485000in}}%
\pgfpathlineto{\pgfqpoint{2.253640in}{2.485000in}}%
\pgfpathlineto{\pgfqpoint{2.254880in}{2.030000in}}%
\pgfpathlineto{\pgfqpoint{2.257360in}{2.765000in}}%
\pgfpathlineto{\pgfqpoint{2.259840in}{2.170000in}}%
\pgfpathlineto{\pgfqpoint{2.261080in}{2.800000in}}%
\pgfpathlineto{\pgfqpoint{2.263560in}{2.415000in}}%
\pgfpathlineto{\pgfqpoint{2.264800in}{2.800000in}}%
\pgfpathlineto{\pgfqpoint{2.266040in}{2.345000in}}%
\pgfpathlineto{\pgfqpoint{2.269760in}{2.590000in}}%
\pgfpathlineto{\pgfqpoint{2.271000in}{2.380000in}}%
\pgfpathlineto{\pgfqpoint{2.272240in}{2.520000in}}%
\pgfpathlineto{\pgfqpoint{2.273480in}{2.520000in}}%
\pgfpathlineto{\pgfqpoint{2.274720in}{2.695000in}}%
\pgfpathlineto{\pgfqpoint{2.275960in}{2.520000in}}%
\pgfpathlineto{\pgfqpoint{2.277200in}{2.590000in}}%
\pgfpathlineto{\pgfqpoint{2.279680in}{2.345000in}}%
\pgfpathlineto{\pgfqpoint{2.280920in}{2.590000in}}%
\pgfpathlineto{\pgfqpoint{2.282160in}{2.590000in}}%
\pgfpathlineto{\pgfqpoint{2.283400in}{2.100000in}}%
\pgfpathlineto{\pgfqpoint{2.285880in}{2.765000in}}%
\pgfpathlineto{\pgfqpoint{2.287120in}{2.555000in}}%
\pgfpathlineto{\pgfqpoint{2.289600in}{2.555000in}}%
\pgfpathlineto{\pgfqpoint{2.290840in}{2.800000in}}%
\pgfpathlineto{\pgfqpoint{2.292080in}{2.100000in}}%
\pgfpathlineto{\pgfqpoint{2.294560in}{2.835000in}}%
\pgfpathlineto{\pgfqpoint{2.295800in}{2.380000in}}%
\pgfpathlineto{\pgfqpoint{2.297040in}{2.380000in}}%
\pgfpathlineto{\pgfqpoint{2.298280in}{2.415000in}}%
\pgfpathlineto{\pgfqpoint{2.299520in}{2.205000in}}%
\pgfpathlineto{\pgfqpoint{2.300760in}{2.625000in}}%
\pgfpathlineto{\pgfqpoint{2.302000in}{2.170000in}}%
\pgfpathlineto{\pgfqpoint{2.304480in}{3.045000in}}%
\pgfpathlineto{\pgfqpoint{2.306960in}{2.415000in}}%
\pgfpathlineto{\pgfqpoint{2.308200in}{2.345000in}}%
\pgfpathlineto{\pgfqpoint{2.309440in}{2.345000in}}%
\pgfpathlineto{\pgfqpoint{2.310680in}{2.380000in}}%
\pgfpathlineto{\pgfqpoint{2.311920in}{2.380000in}}%
\pgfpathlineto{\pgfqpoint{2.313160in}{2.450000in}}%
\pgfpathlineto{\pgfqpoint{2.314400in}{1.960000in}}%
\pgfpathlineto{\pgfqpoint{2.315640in}{2.310000in}}%
\pgfpathlineto{\pgfqpoint{2.316880in}{2.310000in}}%
\pgfpathlineto{\pgfqpoint{2.318120in}{2.485000in}}%
\pgfpathlineto{\pgfqpoint{2.319360in}{2.415000in}}%
\pgfpathlineto{\pgfqpoint{2.320600in}{2.135000in}}%
\pgfpathlineto{\pgfqpoint{2.321840in}{2.625000in}}%
\pgfpathlineto{\pgfqpoint{2.323080in}{2.590000in}}%
\pgfpathlineto{\pgfqpoint{2.324320in}{2.380000in}}%
\pgfpathlineto{\pgfqpoint{2.326800in}{2.590000in}}%
\pgfpathlineto{\pgfqpoint{2.329280in}{2.310000in}}%
\pgfpathlineto{\pgfqpoint{2.330520in}{2.485000in}}%
\pgfpathlineto{\pgfqpoint{2.331760in}{2.240000in}}%
\pgfpathlineto{\pgfqpoint{2.333000in}{2.485000in}}%
\pgfpathlineto{\pgfqpoint{2.334240in}{2.415000in}}%
\pgfpathlineto{\pgfqpoint{2.335480in}{2.485000in}}%
\pgfpathlineto{\pgfqpoint{2.336720in}{2.450000in}}%
\pgfpathlineto{\pgfqpoint{2.337960in}{2.135000in}}%
\pgfpathlineto{\pgfqpoint{2.339200in}{2.380000in}}%
\pgfpathlineto{\pgfqpoint{2.340440in}{2.345000in}}%
\pgfpathlineto{\pgfqpoint{2.342920in}{2.065000in}}%
\pgfpathlineto{\pgfqpoint{2.344160in}{2.450000in}}%
\pgfpathlineto{\pgfqpoint{2.345400in}{2.205000in}}%
\pgfpathlineto{\pgfqpoint{2.346640in}{2.555000in}}%
\pgfpathlineto{\pgfqpoint{2.347880in}{2.450000in}}%
\pgfpathlineto{\pgfqpoint{2.349120in}{2.240000in}}%
\pgfpathlineto{\pgfqpoint{2.351600in}{2.590000in}}%
\pgfpathlineto{\pgfqpoint{2.352840in}{2.450000in}}%
\pgfpathlineto{\pgfqpoint{2.354080in}{2.695000in}}%
\pgfpathlineto{\pgfqpoint{2.356560in}{2.275000in}}%
\pgfpathlineto{\pgfqpoint{2.359040in}{2.450000in}}%
\pgfpathlineto{\pgfqpoint{2.360280in}{2.275000in}}%
\pgfpathlineto{\pgfqpoint{2.361520in}{2.275000in}}%
\pgfpathlineto{\pgfqpoint{2.362760in}{2.625000in}}%
\pgfpathlineto{\pgfqpoint{2.364000in}{2.275000in}}%
\pgfpathlineto{\pgfqpoint{2.365240in}{2.275000in}}%
\pgfpathlineto{\pgfqpoint{2.366480in}{2.660000in}}%
\pgfpathlineto{\pgfqpoint{2.368960in}{2.310000in}}%
\pgfpathlineto{\pgfqpoint{2.370200in}{2.660000in}}%
\pgfpathlineto{\pgfqpoint{2.371440in}{2.345000in}}%
\pgfpathlineto{\pgfqpoint{2.372680in}{2.450000in}}%
\pgfpathlineto{\pgfqpoint{2.375160in}{2.275000in}}%
\pgfpathlineto{\pgfqpoint{2.377640in}{2.555000in}}%
\pgfpathlineto{\pgfqpoint{2.380120in}{2.310000in}}%
\pgfpathlineto{\pgfqpoint{2.381360in}{2.275000in}}%
\pgfpathlineto{\pgfqpoint{2.382600in}{2.170000in}}%
\pgfpathlineto{\pgfqpoint{2.385080in}{2.555000in}}%
\pgfpathlineto{\pgfqpoint{2.386320in}{2.590000in}}%
\pgfpathlineto{\pgfqpoint{2.388800in}{2.170000in}}%
\pgfpathlineto{\pgfqpoint{2.392520in}{2.590000in}}%
\pgfpathlineto{\pgfqpoint{2.395000in}{2.240000in}}%
\pgfpathlineto{\pgfqpoint{2.396240in}{2.415000in}}%
\pgfpathlineto{\pgfqpoint{2.397480in}{2.205000in}}%
\pgfpathlineto{\pgfqpoint{2.398720in}{2.520000in}}%
\pgfpathlineto{\pgfqpoint{2.399960in}{2.520000in}}%
\pgfpathlineto{\pgfqpoint{2.401200in}{2.345000in}}%
\pgfpathlineto{\pgfqpoint{2.402440in}{2.415000in}}%
\pgfpathlineto{\pgfqpoint{2.403680in}{2.345000in}}%
\pgfpathlineto{\pgfqpoint{2.404920in}{2.415000in}}%
\pgfpathlineto{\pgfqpoint{2.406160in}{2.800000in}}%
\pgfpathlineto{\pgfqpoint{2.408640in}{2.275000in}}%
\pgfpathlineto{\pgfqpoint{2.411120in}{2.380000in}}%
\pgfpathlineto{\pgfqpoint{2.412360in}{2.310000in}}%
\pgfpathlineto{\pgfqpoint{2.413600in}{2.310000in}}%
\pgfpathlineto{\pgfqpoint{2.414840in}{2.555000in}}%
\pgfpathlineto{\pgfqpoint{2.416080in}{2.520000in}}%
\pgfpathlineto{\pgfqpoint{2.417320in}{2.170000in}}%
\pgfpathlineto{\pgfqpoint{2.418560in}{2.730000in}}%
\pgfpathlineto{\pgfqpoint{2.419800in}{2.730000in}}%
\pgfpathlineto{\pgfqpoint{2.421040in}{2.450000in}}%
\pgfpathlineto{\pgfqpoint{2.422280in}{2.555000in}}%
\pgfpathlineto{\pgfqpoint{2.424760in}{2.240000in}}%
\pgfpathlineto{\pgfqpoint{2.426000in}{2.345000in}}%
\pgfpathlineto{\pgfqpoint{2.427240in}{2.660000in}}%
\pgfpathlineto{\pgfqpoint{2.429720in}{2.380000in}}%
\pgfpathlineto{\pgfqpoint{2.430960in}{2.695000in}}%
\pgfpathlineto{\pgfqpoint{2.432200in}{2.135000in}}%
\pgfpathlineto{\pgfqpoint{2.433440in}{2.625000in}}%
\pgfpathlineto{\pgfqpoint{2.434680in}{2.100000in}}%
\pgfpathlineto{\pgfqpoint{2.435920in}{2.380000in}}%
\pgfpathlineto{\pgfqpoint{2.440880in}{1.890000in}}%
\pgfpathlineto{\pgfqpoint{2.443360in}{2.275000in}}%
\pgfpathlineto{\pgfqpoint{2.444600in}{2.100000in}}%
\pgfpathlineto{\pgfqpoint{2.445840in}{2.450000in}}%
\pgfpathlineto{\pgfqpoint{2.447080in}{2.450000in}}%
\pgfpathlineto{\pgfqpoint{2.449560in}{1.995000in}}%
\pgfpathlineto{\pgfqpoint{2.450800in}{2.730000in}}%
\pgfpathlineto{\pgfqpoint{2.452040in}{2.345000in}}%
\pgfpathlineto{\pgfqpoint{2.453280in}{2.380000in}}%
\pgfpathlineto{\pgfqpoint{2.454520in}{2.275000in}}%
\pgfpathlineto{\pgfqpoint{2.457000in}{2.800000in}}%
\pgfpathlineto{\pgfqpoint{2.459480in}{2.310000in}}%
\pgfpathlineto{\pgfqpoint{2.461960in}{2.660000in}}%
\pgfpathlineto{\pgfqpoint{2.463200in}{2.660000in}}%
\pgfpathlineto{\pgfqpoint{2.464440in}{2.765000in}}%
\pgfpathlineto{\pgfqpoint{2.465680in}{2.485000in}}%
\pgfpathlineto{\pgfqpoint{2.466920in}{2.695000in}}%
\pgfpathlineto{\pgfqpoint{2.468160in}{2.240000in}}%
\pgfpathlineto{\pgfqpoint{2.470640in}{2.695000in}}%
\pgfpathlineto{\pgfqpoint{2.473120in}{2.240000in}}%
\pgfpathlineto{\pgfqpoint{2.474360in}{2.765000in}}%
\pgfpathlineto{\pgfqpoint{2.476840in}{2.030000in}}%
\pgfpathlineto{\pgfqpoint{2.478080in}{2.170000in}}%
\pgfpathlineto{\pgfqpoint{2.479320in}{2.485000in}}%
\pgfpathlineto{\pgfqpoint{2.480560in}{2.275000in}}%
\pgfpathlineto{\pgfqpoint{2.483040in}{2.485000in}}%
\pgfpathlineto{\pgfqpoint{2.484280in}{2.240000in}}%
\pgfpathlineto{\pgfqpoint{2.485520in}{2.345000in}}%
\pgfpathlineto{\pgfqpoint{2.486760in}{2.625000in}}%
\pgfpathlineto{\pgfqpoint{2.488000in}{2.625000in}}%
\pgfpathlineto{\pgfqpoint{2.489240in}{2.450000in}}%
\pgfpathlineto{\pgfqpoint{2.490480in}{2.590000in}}%
\pgfpathlineto{\pgfqpoint{2.491720in}{2.555000in}}%
\pgfpathlineto{\pgfqpoint{2.495440in}{1.925000in}}%
\pgfpathlineto{\pgfqpoint{2.496680in}{2.590000in}}%
\pgfpathlineto{\pgfqpoint{2.497920in}{2.205000in}}%
\pgfpathlineto{\pgfqpoint{2.499160in}{2.660000in}}%
\pgfpathlineto{\pgfqpoint{2.502880in}{1.925000in}}%
\pgfpathlineto{\pgfqpoint{2.505360in}{2.520000in}}%
\pgfpathlineto{\pgfqpoint{2.506600in}{2.415000in}}%
\pgfpathlineto{\pgfqpoint{2.507840in}{2.205000in}}%
\pgfpathlineto{\pgfqpoint{2.509080in}{2.730000in}}%
\pgfpathlineto{\pgfqpoint{2.510320in}{2.625000in}}%
\pgfpathlineto{\pgfqpoint{2.511560in}{2.310000in}}%
\pgfpathlineto{\pgfqpoint{2.512800in}{2.485000in}}%
\pgfpathlineto{\pgfqpoint{2.514040in}{1.890000in}}%
\pgfpathlineto{\pgfqpoint{2.516520in}{2.415000in}}%
\pgfpathlineto{\pgfqpoint{2.517760in}{2.450000in}}%
\pgfpathlineto{\pgfqpoint{2.519000in}{2.240000in}}%
\pgfpathlineto{\pgfqpoint{2.521480in}{2.765000in}}%
\pgfpathlineto{\pgfqpoint{2.522720in}{2.450000in}}%
\pgfpathlineto{\pgfqpoint{2.523960in}{2.450000in}}%
\pgfpathlineto{\pgfqpoint{2.527680in}{1.995000in}}%
\pgfpathlineto{\pgfqpoint{2.528920in}{2.625000in}}%
\pgfpathlineto{\pgfqpoint{2.530160in}{2.485000in}}%
\pgfpathlineto{\pgfqpoint{2.531400in}{2.520000in}}%
\pgfpathlineto{\pgfqpoint{2.532640in}{2.310000in}}%
\pgfpathlineto{\pgfqpoint{2.535120in}{2.520000in}}%
\pgfpathlineto{\pgfqpoint{2.536360in}{2.310000in}}%
\pgfpathlineto{\pgfqpoint{2.537600in}{2.590000in}}%
\pgfpathlineto{\pgfqpoint{2.540080in}{2.240000in}}%
\pgfpathlineto{\pgfqpoint{2.541320in}{2.450000in}}%
\pgfpathlineto{\pgfqpoint{2.542560in}{2.240000in}}%
\pgfpathlineto{\pgfqpoint{2.543800in}{2.450000in}}%
\pgfpathlineto{\pgfqpoint{2.545040in}{2.345000in}}%
\pgfpathlineto{\pgfqpoint{2.546280in}{2.555000in}}%
\pgfpathlineto{\pgfqpoint{2.547520in}{2.485000in}}%
\pgfpathlineto{\pgfqpoint{2.548760in}{2.940000in}}%
\pgfpathlineto{\pgfqpoint{2.550000in}{2.695000in}}%
\pgfpathlineto{\pgfqpoint{2.551240in}{2.695000in}}%
\pgfpathlineto{\pgfqpoint{2.552480in}{2.345000in}}%
\pgfpathlineto{\pgfqpoint{2.553720in}{2.450000in}}%
\pgfpathlineto{\pgfqpoint{2.554960in}{2.660000in}}%
\pgfpathlineto{\pgfqpoint{2.556200in}{2.135000in}}%
\pgfpathlineto{\pgfqpoint{2.557440in}{2.170000in}}%
\pgfpathlineto{\pgfqpoint{2.558680in}{2.450000in}}%
\pgfpathlineto{\pgfqpoint{2.559920in}{2.065000in}}%
\pgfpathlineto{\pgfqpoint{2.562400in}{2.555000in}}%
\pgfpathlineto{\pgfqpoint{2.563640in}{2.100000in}}%
\pgfpathlineto{\pgfqpoint{2.564880in}{2.135000in}}%
\pgfpathlineto{\pgfqpoint{2.567360in}{2.695000in}}%
\pgfpathlineto{\pgfqpoint{2.568600in}{2.030000in}}%
\pgfpathlineto{\pgfqpoint{2.571080in}{2.660000in}}%
\pgfpathlineto{\pgfqpoint{2.573560in}{2.240000in}}%
\pgfpathlineto{\pgfqpoint{2.576040in}{2.625000in}}%
\pgfpathlineto{\pgfqpoint{2.578520in}{2.765000in}}%
\pgfpathlineto{\pgfqpoint{2.579760in}{2.450000in}}%
\pgfpathlineto{\pgfqpoint{2.581000in}{2.555000in}}%
\pgfpathlineto{\pgfqpoint{2.582240in}{2.065000in}}%
\pgfpathlineto{\pgfqpoint{2.584720in}{2.625000in}}%
\pgfpathlineto{\pgfqpoint{2.585960in}{2.590000in}}%
\pgfpathlineto{\pgfqpoint{2.587200in}{2.660000in}}%
\pgfpathlineto{\pgfqpoint{2.589680in}{2.450000in}}%
\pgfpathlineto{\pgfqpoint{2.590920in}{2.555000in}}%
\pgfpathlineto{\pgfqpoint{2.592160in}{2.380000in}}%
\pgfpathlineto{\pgfqpoint{2.593400in}{3.115000in}}%
\pgfpathlineto{\pgfqpoint{2.595880in}{2.555000in}}%
\pgfpathlineto{\pgfqpoint{2.597120in}{2.765000in}}%
\pgfpathlineto{\pgfqpoint{2.598360in}{2.765000in}}%
\pgfpathlineto{\pgfqpoint{2.599600in}{2.660000in}}%
\pgfpathlineto{\pgfqpoint{2.602080in}{2.275000in}}%
\pgfpathlineto{\pgfqpoint{2.605800in}{2.590000in}}%
\pgfpathlineto{\pgfqpoint{2.607040in}{2.485000in}}%
\pgfpathlineto{\pgfqpoint{2.608280in}{2.625000in}}%
\pgfpathlineto{\pgfqpoint{2.609520in}{2.275000in}}%
\pgfpathlineto{\pgfqpoint{2.610760in}{2.625000in}}%
\pgfpathlineto{\pgfqpoint{2.613240in}{2.310000in}}%
\pgfpathlineto{\pgfqpoint{2.614480in}{2.555000in}}%
\pgfpathlineto{\pgfqpoint{2.615720in}{2.415000in}}%
\pgfpathlineto{\pgfqpoint{2.616960in}{2.485000in}}%
\pgfpathlineto{\pgfqpoint{2.618200in}{2.240000in}}%
\pgfpathlineto{\pgfqpoint{2.621920in}{2.555000in}}%
\pgfpathlineto{\pgfqpoint{2.623160in}{2.450000in}}%
\pgfpathlineto{\pgfqpoint{2.624400in}{2.450000in}}%
\pgfpathlineto{\pgfqpoint{2.625640in}{2.800000in}}%
\pgfpathlineto{\pgfqpoint{2.626880in}{2.415000in}}%
\pgfpathlineto{\pgfqpoint{2.628120in}{2.695000in}}%
\pgfpathlineto{\pgfqpoint{2.629360in}{2.135000in}}%
\pgfpathlineto{\pgfqpoint{2.631840in}{2.765000in}}%
\pgfpathlineto{\pgfqpoint{2.634320in}{2.380000in}}%
\pgfpathlineto{\pgfqpoint{2.635560in}{2.800000in}}%
\pgfpathlineto{\pgfqpoint{2.638040in}{2.380000in}}%
\pgfpathlineto{\pgfqpoint{2.639280in}{2.835000in}}%
\pgfpathlineto{\pgfqpoint{2.640520in}{2.625000in}}%
\pgfpathlineto{\pgfqpoint{2.641760in}{1.960000in}}%
\pgfpathlineto{\pgfqpoint{2.643000in}{2.065000in}}%
\pgfpathlineto{\pgfqpoint{2.645480in}{2.835000in}}%
\pgfpathlineto{\pgfqpoint{2.646720in}{2.485000in}}%
\pgfpathlineto{\pgfqpoint{2.647960in}{2.590000in}}%
\pgfpathlineto{\pgfqpoint{2.649200in}{2.170000in}}%
\pgfpathlineto{\pgfqpoint{2.650440in}{2.345000in}}%
\pgfpathlineto{\pgfqpoint{2.651680in}{2.345000in}}%
\pgfpathlineto{\pgfqpoint{2.654160in}{2.730000in}}%
\pgfpathlineto{\pgfqpoint{2.655400in}{2.660000in}}%
\pgfpathlineto{\pgfqpoint{2.657880in}{2.100000in}}%
\pgfpathlineto{\pgfqpoint{2.659120in}{2.625000in}}%
\pgfpathlineto{\pgfqpoint{2.661600in}{2.380000in}}%
\pgfpathlineto{\pgfqpoint{2.664080in}{2.555000in}}%
\pgfpathlineto{\pgfqpoint{2.665320in}{2.240000in}}%
\pgfpathlineto{\pgfqpoint{2.666560in}{2.380000in}}%
\pgfpathlineto{\pgfqpoint{2.667800in}{2.310000in}}%
\pgfpathlineto{\pgfqpoint{2.669040in}{2.870000in}}%
\pgfpathlineto{\pgfqpoint{2.670280in}{2.345000in}}%
\pgfpathlineto{\pgfqpoint{2.672760in}{2.590000in}}%
\pgfpathlineto{\pgfqpoint{2.674000in}{2.520000in}}%
\pgfpathlineto{\pgfqpoint{2.675240in}{2.625000in}}%
\pgfpathlineto{\pgfqpoint{2.676480in}{2.240000in}}%
\pgfpathlineto{\pgfqpoint{2.677720in}{2.555000in}}%
\pgfpathlineto{\pgfqpoint{2.680200in}{2.205000in}}%
\pgfpathlineto{\pgfqpoint{2.681440in}{2.765000in}}%
\pgfpathlineto{\pgfqpoint{2.683920in}{2.275000in}}%
\pgfpathlineto{\pgfqpoint{2.685160in}{2.625000in}}%
\pgfpathlineto{\pgfqpoint{2.686400in}{2.485000in}}%
\pgfpathlineto{\pgfqpoint{2.687640in}{2.100000in}}%
\pgfpathlineto{\pgfqpoint{2.688880in}{2.100000in}}%
\pgfpathlineto{\pgfqpoint{2.691360in}{2.555000in}}%
\pgfpathlineto{\pgfqpoint{2.692600in}{2.135000in}}%
\pgfpathlineto{\pgfqpoint{2.693840in}{2.170000in}}%
\pgfpathlineto{\pgfqpoint{2.695080in}{2.555000in}}%
\pgfpathlineto{\pgfqpoint{2.696320in}{2.520000in}}%
\pgfpathlineto{\pgfqpoint{2.697560in}{2.345000in}}%
\pgfpathlineto{\pgfqpoint{2.698800in}{2.695000in}}%
\pgfpathlineto{\pgfqpoint{2.701280in}{2.380000in}}%
\pgfpathlineto{\pgfqpoint{2.702520in}{2.380000in}}%
\pgfpathlineto{\pgfqpoint{2.705000in}{2.800000in}}%
\pgfpathlineto{\pgfqpoint{2.708720in}{2.310000in}}%
\pgfpathlineto{\pgfqpoint{2.709960in}{2.835000in}}%
\pgfpathlineto{\pgfqpoint{2.713680in}{2.170000in}}%
\pgfpathlineto{\pgfqpoint{2.714920in}{2.450000in}}%
\pgfpathlineto{\pgfqpoint{2.716160in}{2.345000in}}%
\pgfpathlineto{\pgfqpoint{2.717400in}{1.995000in}}%
\pgfpathlineto{\pgfqpoint{2.719880in}{2.800000in}}%
\pgfpathlineto{\pgfqpoint{2.721120in}{2.205000in}}%
\pgfpathlineto{\pgfqpoint{2.723600in}{2.625000in}}%
\pgfpathlineto{\pgfqpoint{2.724840in}{2.485000in}}%
\pgfpathlineto{\pgfqpoint{2.726080in}{2.485000in}}%
\pgfpathlineto{\pgfqpoint{2.727320in}{2.415000in}}%
\pgfpathlineto{\pgfqpoint{2.728560in}{2.170000in}}%
\pgfpathlineto{\pgfqpoint{2.731040in}{2.625000in}}%
\pgfpathlineto{\pgfqpoint{2.732280in}{2.415000in}}%
\pgfpathlineto{\pgfqpoint{2.733520in}{2.450000in}}%
\pgfpathlineto{\pgfqpoint{2.734760in}{2.205000in}}%
\pgfpathlineto{\pgfqpoint{2.736000in}{2.275000in}}%
\pgfpathlineto{\pgfqpoint{2.737240in}{2.730000in}}%
\pgfpathlineto{\pgfqpoint{2.738480in}{2.205000in}}%
\pgfpathlineto{\pgfqpoint{2.739720in}{2.485000in}}%
\pgfpathlineto{\pgfqpoint{2.740960in}{2.240000in}}%
\pgfpathlineto{\pgfqpoint{2.742200in}{2.345000in}}%
\pgfpathlineto{\pgfqpoint{2.743440in}{2.660000in}}%
\pgfpathlineto{\pgfqpoint{2.744680in}{2.520000in}}%
\pgfpathlineto{\pgfqpoint{2.745920in}{2.520000in}}%
\pgfpathlineto{\pgfqpoint{2.747160in}{2.590000in}}%
\pgfpathlineto{\pgfqpoint{2.749640in}{2.240000in}}%
\pgfpathlineto{\pgfqpoint{2.750880in}{1.960000in}}%
\pgfpathlineto{\pgfqpoint{2.752120in}{2.485000in}}%
\pgfpathlineto{\pgfqpoint{2.753360in}{2.030000in}}%
\pgfpathlineto{\pgfqpoint{2.754600in}{2.275000in}}%
\pgfpathlineto{\pgfqpoint{2.755840in}{2.275000in}}%
\pgfpathlineto{\pgfqpoint{2.757080in}{2.590000in}}%
\pgfpathlineto{\pgfqpoint{2.758320in}{2.135000in}}%
\pgfpathlineto{\pgfqpoint{2.759560in}{2.555000in}}%
\pgfpathlineto{\pgfqpoint{2.760800in}{2.345000in}}%
\pgfpathlineto{\pgfqpoint{2.762040in}{2.415000in}}%
\pgfpathlineto{\pgfqpoint{2.763280in}{2.345000in}}%
\pgfpathlineto{\pgfqpoint{2.764520in}{2.415000in}}%
\pgfpathlineto{\pgfqpoint{2.765760in}{2.275000in}}%
\pgfpathlineto{\pgfqpoint{2.767000in}{2.520000in}}%
\pgfpathlineto{\pgfqpoint{2.769480in}{2.240000in}}%
\pgfpathlineto{\pgfqpoint{2.770720in}{2.135000in}}%
\pgfpathlineto{\pgfqpoint{2.771960in}{2.205000in}}%
\pgfpathlineto{\pgfqpoint{2.773200in}{2.625000in}}%
\pgfpathlineto{\pgfqpoint{2.774440in}{2.555000in}}%
\pgfpathlineto{\pgfqpoint{2.776920in}{2.240000in}}%
\pgfpathlineto{\pgfqpoint{2.778160in}{2.660000in}}%
\pgfpathlineto{\pgfqpoint{2.779400in}{2.240000in}}%
\pgfpathlineto{\pgfqpoint{2.781880in}{2.975000in}}%
\pgfpathlineto{\pgfqpoint{2.785600in}{1.890000in}}%
\pgfpathlineto{\pgfqpoint{2.786840in}{2.520000in}}%
\pgfpathlineto{\pgfqpoint{2.790560in}{2.100000in}}%
\pgfpathlineto{\pgfqpoint{2.793040in}{2.520000in}}%
\pgfpathlineto{\pgfqpoint{2.794280in}{2.240000in}}%
\pgfpathlineto{\pgfqpoint{2.795520in}{2.240000in}}%
\pgfpathlineto{\pgfqpoint{2.798000in}{2.590000in}}%
\pgfpathlineto{\pgfqpoint{2.799240in}{2.380000in}}%
\pgfpathlineto{\pgfqpoint{2.800480in}{2.415000in}}%
\pgfpathlineto{\pgfqpoint{2.801720in}{2.555000in}}%
\pgfpathlineto{\pgfqpoint{2.802960in}{2.170000in}}%
\pgfpathlineto{\pgfqpoint{2.804200in}{2.765000in}}%
\pgfpathlineto{\pgfqpoint{2.805440in}{2.660000in}}%
\pgfpathlineto{\pgfqpoint{2.807920in}{2.660000in}}%
\pgfpathlineto{\pgfqpoint{2.809160in}{2.590000in}}%
\pgfpathlineto{\pgfqpoint{2.811640in}{2.380000in}}%
\pgfpathlineto{\pgfqpoint{2.812880in}{2.695000in}}%
\pgfpathlineto{\pgfqpoint{2.816600in}{2.065000in}}%
\pgfpathlineto{\pgfqpoint{2.817840in}{2.765000in}}%
\pgfpathlineto{\pgfqpoint{2.819080in}{2.520000in}}%
\pgfpathlineto{\pgfqpoint{2.820320in}{2.520000in}}%
\pgfpathlineto{\pgfqpoint{2.821560in}{2.415000in}}%
\pgfpathlineto{\pgfqpoint{2.824040in}{2.590000in}}%
\pgfpathlineto{\pgfqpoint{2.825280in}{2.135000in}}%
\pgfpathlineto{\pgfqpoint{2.827760in}{2.520000in}}%
\pgfpathlineto{\pgfqpoint{2.830240in}{2.275000in}}%
\pgfpathlineto{\pgfqpoint{2.831480in}{2.590000in}}%
\pgfpathlineto{\pgfqpoint{2.832720in}{2.450000in}}%
\pgfpathlineto{\pgfqpoint{2.833960in}{2.660000in}}%
\pgfpathlineto{\pgfqpoint{2.835200in}{2.275000in}}%
\pgfpathlineto{\pgfqpoint{2.836440in}{2.765000in}}%
\pgfpathlineto{\pgfqpoint{2.837680in}{2.275000in}}%
\pgfpathlineto{\pgfqpoint{2.838920in}{2.275000in}}%
\pgfpathlineto{\pgfqpoint{2.840160in}{1.820000in}}%
\pgfpathlineto{\pgfqpoint{2.842640in}{2.590000in}}%
\pgfpathlineto{\pgfqpoint{2.845120in}{2.310000in}}%
\pgfpathlineto{\pgfqpoint{2.846360in}{2.555000in}}%
\pgfpathlineto{\pgfqpoint{2.847600in}{1.995000in}}%
\pgfpathlineto{\pgfqpoint{2.848840in}{2.205000in}}%
\pgfpathlineto{\pgfqpoint{2.851320in}{2.765000in}}%
\pgfpathlineto{\pgfqpoint{2.852560in}{2.590000in}}%
\pgfpathlineto{\pgfqpoint{2.853800in}{2.905000in}}%
\pgfpathlineto{\pgfqpoint{2.855040in}{2.135000in}}%
\pgfpathlineto{\pgfqpoint{2.856280in}{2.940000in}}%
\pgfpathlineto{\pgfqpoint{2.857520in}{2.310000in}}%
\pgfpathlineto{\pgfqpoint{2.860000in}{2.695000in}}%
\pgfpathlineto{\pgfqpoint{2.861240in}{2.835000in}}%
\pgfpathlineto{\pgfqpoint{2.862480in}{2.555000in}}%
\pgfpathlineto{\pgfqpoint{2.863720in}{2.730000in}}%
\pgfpathlineto{\pgfqpoint{2.864960in}{2.450000in}}%
\pgfpathlineto{\pgfqpoint{2.866200in}{2.730000in}}%
\pgfpathlineto{\pgfqpoint{2.867440in}{2.660000in}}%
\pgfpathlineto{\pgfqpoint{2.868680in}{2.310000in}}%
\pgfpathlineto{\pgfqpoint{2.871160in}{2.695000in}}%
\pgfpathlineto{\pgfqpoint{2.872400in}{2.520000in}}%
\pgfpathlineto{\pgfqpoint{2.873640in}{2.765000in}}%
\pgfpathlineto{\pgfqpoint{2.874880in}{2.485000in}}%
\pgfpathlineto{\pgfqpoint{2.876120in}{2.555000in}}%
\pgfpathlineto{\pgfqpoint{2.877360in}{2.695000in}}%
\pgfpathlineto{\pgfqpoint{2.878600in}{2.030000in}}%
\pgfpathlineto{\pgfqpoint{2.879840in}{2.590000in}}%
\pgfpathlineto{\pgfqpoint{2.881080in}{2.485000in}}%
\pgfpathlineto{\pgfqpoint{2.882320in}{2.100000in}}%
\pgfpathlineto{\pgfqpoint{2.883560in}{2.625000in}}%
\pgfpathlineto{\pgfqpoint{2.886040in}{2.135000in}}%
\pgfpathlineto{\pgfqpoint{2.888520in}{2.310000in}}%
\pgfpathlineto{\pgfqpoint{2.891000in}{2.485000in}}%
\pgfpathlineto{\pgfqpoint{2.892240in}{2.415000in}}%
\pgfpathlineto{\pgfqpoint{2.893480in}{2.415000in}}%
\pgfpathlineto{\pgfqpoint{2.894720in}{2.625000in}}%
\pgfpathlineto{\pgfqpoint{2.897200in}{1.995000in}}%
\pgfpathlineto{\pgfqpoint{2.902160in}{2.835000in}}%
\pgfpathlineto{\pgfqpoint{2.904640in}{2.345000in}}%
\pgfpathlineto{\pgfqpoint{2.905880in}{2.310000in}}%
\pgfpathlineto{\pgfqpoint{2.907120in}{2.030000in}}%
\pgfpathlineto{\pgfqpoint{2.908360in}{2.415000in}}%
\pgfpathlineto{\pgfqpoint{2.909600in}{2.415000in}}%
\pgfpathlineto{\pgfqpoint{2.910840in}{2.310000in}}%
\pgfpathlineto{\pgfqpoint{2.912080in}{2.345000in}}%
\pgfpathlineto{\pgfqpoint{2.913320in}{2.590000in}}%
\pgfpathlineto{\pgfqpoint{2.914560in}{2.590000in}}%
\pgfpathlineto{\pgfqpoint{2.915800in}{2.555000in}}%
\pgfpathlineto{\pgfqpoint{2.917040in}{2.590000in}}%
\pgfpathlineto{\pgfqpoint{2.918280in}{2.415000in}}%
\pgfpathlineto{\pgfqpoint{2.919520in}{2.520000in}}%
\pgfpathlineto{\pgfqpoint{2.920760in}{2.345000in}}%
\pgfpathlineto{\pgfqpoint{2.922000in}{2.520000in}}%
\pgfpathlineto{\pgfqpoint{2.923240in}{2.870000in}}%
\pgfpathlineto{\pgfqpoint{2.925720in}{2.275000in}}%
\pgfpathlineto{\pgfqpoint{2.926960in}{2.275000in}}%
\pgfpathlineto{\pgfqpoint{2.928200in}{2.590000in}}%
\pgfpathlineto{\pgfqpoint{2.929440in}{2.205000in}}%
\pgfpathlineto{\pgfqpoint{2.930680in}{2.205000in}}%
\pgfpathlineto{\pgfqpoint{2.931920in}{2.310000in}}%
\pgfpathlineto{\pgfqpoint{2.933160in}{2.625000in}}%
\pgfpathlineto{\pgfqpoint{2.934400in}{2.590000in}}%
\pgfpathlineto{\pgfqpoint{2.935640in}{2.590000in}}%
\pgfpathlineto{\pgfqpoint{2.936880in}{2.835000in}}%
\pgfpathlineto{\pgfqpoint{2.938120in}{2.590000in}}%
\pgfpathlineto{\pgfqpoint{2.939360in}{2.590000in}}%
\pgfpathlineto{\pgfqpoint{2.940600in}{2.485000in}}%
\pgfpathlineto{\pgfqpoint{2.941840in}{2.520000in}}%
\pgfpathlineto{\pgfqpoint{2.943080in}{2.275000in}}%
\pgfpathlineto{\pgfqpoint{2.944320in}{2.660000in}}%
\pgfpathlineto{\pgfqpoint{2.945560in}{2.100000in}}%
\pgfpathlineto{\pgfqpoint{2.948040in}{2.520000in}}%
\pgfpathlineto{\pgfqpoint{2.949280in}{2.520000in}}%
\pgfpathlineto{\pgfqpoint{2.950520in}{2.485000in}}%
\pgfpathlineto{\pgfqpoint{2.951760in}{2.170000in}}%
\pgfpathlineto{\pgfqpoint{2.953000in}{2.590000in}}%
\pgfpathlineto{\pgfqpoint{2.954240in}{2.555000in}}%
\pgfpathlineto{\pgfqpoint{2.955480in}{2.625000in}}%
\pgfpathlineto{\pgfqpoint{2.957960in}{2.345000in}}%
\pgfpathlineto{\pgfqpoint{2.959200in}{2.275000in}}%
\pgfpathlineto{\pgfqpoint{2.960440in}{2.380000in}}%
\pgfpathlineto{\pgfqpoint{2.961680in}{2.240000in}}%
\pgfpathlineto{\pgfqpoint{2.962920in}{2.625000in}}%
\pgfpathlineto{\pgfqpoint{2.965400in}{2.485000in}}%
\pgfpathlineto{\pgfqpoint{2.966640in}{2.450000in}}%
\pgfpathlineto{\pgfqpoint{2.969120in}{1.890000in}}%
\pgfpathlineto{\pgfqpoint{2.970360in}{2.135000in}}%
\pgfpathlineto{\pgfqpoint{2.971600in}{2.625000in}}%
\pgfpathlineto{\pgfqpoint{2.972840in}{2.065000in}}%
\pgfpathlineto{\pgfqpoint{2.974080in}{2.625000in}}%
\pgfpathlineto{\pgfqpoint{2.975320in}{2.030000in}}%
\pgfpathlineto{\pgfqpoint{2.976560in}{2.660000in}}%
\pgfpathlineto{\pgfqpoint{2.977800in}{2.485000in}}%
\pgfpathlineto{\pgfqpoint{2.979040in}{1.995000in}}%
\pgfpathlineto{\pgfqpoint{2.981520in}{2.555000in}}%
\pgfpathlineto{\pgfqpoint{2.984000in}{2.765000in}}%
\pgfpathlineto{\pgfqpoint{2.985240in}{2.695000in}}%
\pgfpathlineto{\pgfqpoint{2.986480in}{2.485000in}}%
\pgfpathlineto{\pgfqpoint{2.987720in}{2.660000in}}%
\pgfpathlineto{\pgfqpoint{2.988960in}{2.590000in}}%
\pgfpathlineto{\pgfqpoint{2.990200in}{2.415000in}}%
\pgfpathlineto{\pgfqpoint{2.991440in}{2.590000in}}%
\pgfpathlineto{\pgfqpoint{2.992680in}{2.450000in}}%
\pgfpathlineto{\pgfqpoint{2.995160in}{2.800000in}}%
\pgfpathlineto{\pgfqpoint{2.996400in}{2.555000in}}%
\pgfpathlineto{\pgfqpoint{2.997640in}{2.555000in}}%
\pgfpathlineto{\pgfqpoint{2.998880in}{2.380000in}}%
\pgfpathlineto{\pgfqpoint{3.000120in}{2.660000in}}%
\pgfpathlineto{\pgfqpoint{3.001360in}{2.310000in}}%
\pgfpathlineto{\pgfqpoint{3.002600in}{2.345000in}}%
\pgfpathlineto{\pgfqpoint{3.003840in}{2.660000in}}%
\pgfpathlineto{\pgfqpoint{3.005080in}{2.275000in}}%
\pgfpathlineto{\pgfqpoint{3.006320in}{2.240000in}}%
\pgfpathlineto{\pgfqpoint{3.008800in}{2.695000in}}%
\pgfpathlineto{\pgfqpoint{3.011280in}{2.240000in}}%
\pgfpathlineto{\pgfqpoint{3.012520in}{2.205000in}}%
\pgfpathlineto{\pgfqpoint{3.013760in}{2.450000in}}%
\pgfpathlineto{\pgfqpoint{3.015000in}{2.345000in}}%
\pgfpathlineto{\pgfqpoint{3.016240in}{1.890000in}}%
\pgfpathlineto{\pgfqpoint{3.017480in}{2.485000in}}%
\pgfpathlineto{\pgfqpoint{3.018720in}{2.450000in}}%
\pgfpathlineto{\pgfqpoint{3.019960in}{1.645000in}}%
\pgfpathlineto{\pgfqpoint{3.021200in}{1.995000in}}%
\pgfpathlineto{\pgfqpoint{3.022440in}{1.855000in}}%
\pgfpathlineto{\pgfqpoint{3.024920in}{2.555000in}}%
\pgfpathlineto{\pgfqpoint{3.029880in}{1.995000in}}%
\pgfpathlineto{\pgfqpoint{3.032360in}{2.450000in}}%
\pgfpathlineto{\pgfqpoint{3.034840in}{2.135000in}}%
\pgfpathlineto{\pgfqpoint{3.036080in}{2.590000in}}%
\pgfpathlineto{\pgfqpoint{3.037320in}{2.485000in}}%
\pgfpathlineto{\pgfqpoint{3.038560in}{2.275000in}}%
\pgfpathlineto{\pgfqpoint{3.039800in}{1.855000in}}%
\pgfpathlineto{\pgfqpoint{3.041040in}{2.555000in}}%
\pgfpathlineto{\pgfqpoint{3.042280in}{2.520000in}}%
\pgfpathlineto{\pgfqpoint{3.043520in}{2.660000in}}%
\pgfpathlineto{\pgfqpoint{3.044760in}{2.380000in}}%
\pgfpathlineto{\pgfqpoint{3.046000in}{2.835000in}}%
\pgfpathlineto{\pgfqpoint{3.048480in}{2.450000in}}%
\pgfpathlineto{\pgfqpoint{3.053440in}{2.835000in}}%
\pgfpathlineto{\pgfqpoint{3.054680in}{2.275000in}}%
\pgfpathlineto{\pgfqpoint{3.055920in}{2.380000in}}%
\pgfpathlineto{\pgfqpoint{3.057160in}{2.205000in}}%
\pgfpathlineto{\pgfqpoint{3.059640in}{2.660000in}}%
\pgfpathlineto{\pgfqpoint{3.062120in}{2.275000in}}%
\pgfpathlineto{\pgfqpoint{3.063360in}{2.310000in}}%
\pgfpathlineto{\pgfqpoint{3.064600in}{2.485000in}}%
\pgfpathlineto{\pgfqpoint{3.067080in}{2.205000in}}%
\pgfpathlineto{\pgfqpoint{3.069560in}{2.380000in}}%
\pgfpathlineto{\pgfqpoint{3.070800in}{2.275000in}}%
\pgfpathlineto{\pgfqpoint{3.073280in}{2.485000in}}%
\pgfpathlineto{\pgfqpoint{3.075760in}{2.275000in}}%
\pgfpathlineto{\pgfqpoint{3.077000in}{2.555000in}}%
\pgfpathlineto{\pgfqpoint{3.078240in}{2.205000in}}%
\pgfpathlineto{\pgfqpoint{3.079480in}{2.625000in}}%
\pgfpathlineto{\pgfqpoint{3.080720in}{2.485000in}}%
\pgfpathlineto{\pgfqpoint{3.081960in}{2.100000in}}%
\pgfpathlineto{\pgfqpoint{3.083200in}{2.065000in}}%
\pgfpathlineto{\pgfqpoint{3.084440in}{2.450000in}}%
\pgfpathlineto{\pgfqpoint{3.088160in}{2.065000in}}%
\pgfpathlineto{\pgfqpoint{3.089400in}{2.135000in}}%
\pgfpathlineto{\pgfqpoint{3.090640in}{2.135000in}}%
\pgfpathlineto{\pgfqpoint{3.091880in}{2.170000in}}%
\pgfpathlineto{\pgfqpoint{3.093120in}{1.960000in}}%
\pgfpathlineto{\pgfqpoint{3.095600in}{2.590000in}}%
\pgfpathlineto{\pgfqpoint{3.096840in}{2.450000in}}%
\pgfpathlineto{\pgfqpoint{3.098080in}{1.960000in}}%
\pgfpathlineto{\pgfqpoint{3.099320in}{2.555000in}}%
\pgfpathlineto{\pgfqpoint{3.100560in}{2.450000in}}%
\pgfpathlineto{\pgfqpoint{3.101800in}{1.785000in}}%
\pgfpathlineto{\pgfqpoint{3.104280in}{2.345000in}}%
\pgfpathlineto{\pgfqpoint{3.105520in}{2.240000in}}%
\pgfpathlineto{\pgfqpoint{3.106760in}{2.310000in}}%
\pgfpathlineto{\pgfqpoint{3.108000in}{2.310000in}}%
\pgfpathlineto{\pgfqpoint{3.109240in}{2.275000in}}%
\pgfpathlineto{\pgfqpoint{3.110480in}{2.065000in}}%
\pgfpathlineto{\pgfqpoint{3.111720in}{2.240000in}}%
\pgfpathlineto{\pgfqpoint{3.112960in}{2.205000in}}%
\pgfpathlineto{\pgfqpoint{3.114200in}{2.555000in}}%
\pgfpathlineto{\pgfqpoint{3.115440in}{2.450000in}}%
\pgfpathlineto{\pgfqpoint{3.116680in}{2.520000in}}%
\pgfpathlineto{\pgfqpoint{3.117920in}{2.380000in}}%
\pgfpathlineto{\pgfqpoint{3.119160in}{2.520000in}}%
\pgfpathlineto{\pgfqpoint{3.120400in}{2.240000in}}%
\pgfpathlineto{\pgfqpoint{3.121640in}{2.310000in}}%
\pgfpathlineto{\pgfqpoint{3.122880in}{2.065000in}}%
\pgfpathlineto{\pgfqpoint{3.124120in}{2.380000in}}%
\pgfpathlineto{\pgfqpoint{3.125360in}{2.275000in}}%
\pgfpathlineto{\pgfqpoint{3.126600in}{2.625000in}}%
\pgfpathlineto{\pgfqpoint{3.127840in}{2.625000in}}%
\pgfpathlineto{\pgfqpoint{3.129080in}{2.730000in}}%
\pgfpathlineto{\pgfqpoint{3.130320in}{2.205000in}}%
\pgfpathlineto{\pgfqpoint{3.131560in}{2.415000in}}%
\pgfpathlineto{\pgfqpoint{3.132800in}{2.380000in}}%
\pgfpathlineto{\pgfqpoint{3.134040in}{2.205000in}}%
\pgfpathlineto{\pgfqpoint{3.135280in}{2.870000in}}%
\pgfpathlineto{\pgfqpoint{3.136520in}{2.170000in}}%
\pgfpathlineto{\pgfqpoint{3.137760in}{2.170000in}}%
\pgfpathlineto{\pgfqpoint{3.139000in}{2.205000in}}%
\pgfpathlineto{\pgfqpoint{3.140240in}{2.345000in}}%
\pgfpathlineto{\pgfqpoint{3.141480in}{2.765000in}}%
\pgfpathlineto{\pgfqpoint{3.142720in}{2.240000in}}%
\pgfpathlineto{\pgfqpoint{3.143960in}{2.205000in}}%
\pgfpathlineto{\pgfqpoint{3.145200in}{2.275000in}}%
\pgfpathlineto{\pgfqpoint{3.146440in}{2.625000in}}%
\pgfpathlineto{\pgfqpoint{3.147680in}{2.135000in}}%
\pgfpathlineto{\pgfqpoint{3.148920in}{2.800000in}}%
\pgfpathlineto{\pgfqpoint{3.151400in}{2.310000in}}%
\pgfpathlineto{\pgfqpoint{3.153880in}{2.485000in}}%
\pgfpathlineto{\pgfqpoint{3.155120in}{2.765000in}}%
\pgfpathlineto{\pgfqpoint{3.156360in}{2.275000in}}%
\pgfpathlineto{\pgfqpoint{3.157600in}{2.240000in}}%
\pgfpathlineto{\pgfqpoint{3.158840in}{2.345000in}}%
\pgfpathlineto{\pgfqpoint{3.160080in}{2.590000in}}%
\pgfpathlineto{\pgfqpoint{3.161320in}{2.450000in}}%
\pgfpathlineto{\pgfqpoint{3.162560in}{2.660000in}}%
\pgfpathlineto{\pgfqpoint{3.163800in}{2.240000in}}%
\pgfpathlineto{\pgfqpoint{3.165040in}{2.590000in}}%
\pgfpathlineto{\pgfqpoint{3.166280in}{2.555000in}}%
\pgfpathlineto{\pgfqpoint{3.170000in}{1.855000in}}%
\pgfpathlineto{\pgfqpoint{3.171240in}{2.450000in}}%
\pgfpathlineto{\pgfqpoint{3.172480in}{2.450000in}}%
\pgfpathlineto{\pgfqpoint{3.173720in}{2.345000in}}%
\pgfpathlineto{\pgfqpoint{3.174960in}{2.695000in}}%
\pgfpathlineto{\pgfqpoint{3.176200in}{2.380000in}}%
\pgfpathlineto{\pgfqpoint{3.177440in}{2.695000in}}%
\pgfpathlineto{\pgfqpoint{3.178680in}{2.275000in}}%
\pgfpathlineto{\pgfqpoint{3.179920in}{2.310000in}}%
\pgfpathlineto{\pgfqpoint{3.182400in}{2.170000in}}%
\pgfpathlineto{\pgfqpoint{3.184880in}{2.555000in}}%
\pgfpathlineto{\pgfqpoint{3.186120in}{2.485000in}}%
\pgfpathlineto{\pgfqpoint{3.187360in}{2.485000in}}%
\pgfpathlineto{\pgfqpoint{3.188600in}{2.275000in}}%
\pgfpathlineto{\pgfqpoint{3.189840in}{2.485000in}}%
\pgfpathlineto{\pgfqpoint{3.191080in}{2.240000in}}%
\pgfpathlineto{\pgfqpoint{3.192320in}{2.415000in}}%
\pgfpathlineto{\pgfqpoint{3.193560in}{2.065000in}}%
\pgfpathlineto{\pgfqpoint{3.194800in}{2.135000in}}%
\pgfpathlineto{\pgfqpoint{3.196040in}{2.100000in}}%
\pgfpathlineto{\pgfqpoint{3.198520in}{2.555000in}}%
\pgfpathlineto{\pgfqpoint{3.199760in}{2.380000in}}%
\pgfpathlineto{\pgfqpoint{3.201000in}{2.030000in}}%
\pgfpathlineto{\pgfqpoint{3.202240in}{2.205000in}}%
\pgfpathlineto{\pgfqpoint{3.204720in}{1.925000in}}%
\pgfpathlineto{\pgfqpoint{3.205960in}{1.925000in}}%
\pgfpathlineto{\pgfqpoint{3.207200in}{2.065000in}}%
\pgfpathlineto{\pgfqpoint{3.208440in}{2.415000in}}%
\pgfpathlineto{\pgfqpoint{3.209680in}{2.450000in}}%
\pgfpathlineto{\pgfqpoint{3.210920in}{2.590000in}}%
\pgfpathlineto{\pgfqpoint{3.212160in}{2.135000in}}%
\pgfpathlineto{\pgfqpoint{3.213400in}{2.380000in}}%
\pgfpathlineto{\pgfqpoint{3.214640in}{2.310000in}}%
\pgfpathlineto{\pgfqpoint{3.215880in}{1.960000in}}%
\pgfpathlineto{\pgfqpoint{3.217120in}{2.170000in}}%
\pgfpathlineto{\pgfqpoint{3.219600in}{1.960000in}}%
\pgfpathlineto{\pgfqpoint{3.220840in}{2.170000in}}%
\pgfpathlineto{\pgfqpoint{3.222080in}{1.785000in}}%
\pgfpathlineto{\pgfqpoint{3.224560in}{2.450000in}}%
\pgfpathlineto{\pgfqpoint{3.225800in}{2.485000in}}%
\pgfpathlineto{\pgfqpoint{3.228280in}{2.135000in}}%
\pgfpathlineto{\pgfqpoint{3.230760in}{2.660000in}}%
\pgfpathlineto{\pgfqpoint{3.232000in}{2.590000in}}%
\pgfpathlineto{\pgfqpoint{3.233240in}{2.240000in}}%
\pgfpathlineto{\pgfqpoint{3.234480in}{2.345000in}}%
\pgfpathlineto{\pgfqpoint{3.235720in}{2.135000in}}%
\pgfpathlineto{\pgfqpoint{3.238200in}{2.520000in}}%
\pgfpathlineto{\pgfqpoint{3.239440in}{2.030000in}}%
\pgfpathlineto{\pgfqpoint{3.240680in}{2.625000in}}%
\pgfpathlineto{\pgfqpoint{3.243160in}{1.995000in}}%
\pgfpathlineto{\pgfqpoint{3.244400in}{2.345000in}}%
\pgfpathlineto{\pgfqpoint{3.245640in}{2.135000in}}%
\pgfpathlineto{\pgfqpoint{3.246880in}{2.520000in}}%
\pgfpathlineto{\pgfqpoint{3.248120in}{2.520000in}}%
\pgfpathlineto{\pgfqpoint{3.249360in}{2.205000in}}%
\pgfpathlineto{\pgfqpoint{3.250600in}{2.275000in}}%
\pgfpathlineto{\pgfqpoint{3.251840in}{2.415000in}}%
\pgfpathlineto{\pgfqpoint{3.253080in}{2.205000in}}%
\pgfpathlineto{\pgfqpoint{3.254320in}{2.555000in}}%
\pgfpathlineto{\pgfqpoint{3.255560in}{2.520000in}}%
\pgfpathlineto{\pgfqpoint{3.256800in}{2.275000in}}%
\pgfpathlineto{\pgfqpoint{3.258040in}{2.485000in}}%
\pgfpathlineto{\pgfqpoint{3.259280in}{2.485000in}}%
\pgfpathlineto{\pgfqpoint{3.260520in}{2.590000in}}%
\pgfpathlineto{\pgfqpoint{3.263000in}{2.345000in}}%
\pgfpathlineto{\pgfqpoint{3.264240in}{2.450000in}}%
\pgfpathlineto{\pgfqpoint{3.265480in}{2.135000in}}%
\pgfpathlineto{\pgfqpoint{3.267960in}{2.450000in}}%
\pgfpathlineto{\pgfqpoint{3.269200in}{2.380000in}}%
\pgfpathlineto{\pgfqpoint{3.270440in}{2.380000in}}%
\pgfpathlineto{\pgfqpoint{3.271680in}{2.485000in}}%
\pgfpathlineto{\pgfqpoint{3.272920in}{2.415000in}}%
\pgfpathlineto{\pgfqpoint{3.274160in}{2.205000in}}%
\pgfpathlineto{\pgfqpoint{3.275400in}{2.275000in}}%
\pgfpathlineto{\pgfqpoint{3.276640in}{1.890000in}}%
\pgfpathlineto{\pgfqpoint{3.279120in}{2.625000in}}%
\pgfpathlineto{\pgfqpoint{3.280360in}{2.380000in}}%
\pgfpathlineto{\pgfqpoint{3.284080in}{2.695000in}}%
\pgfpathlineto{\pgfqpoint{3.285320in}{2.555000in}}%
\pgfpathlineto{\pgfqpoint{3.286560in}{2.205000in}}%
\pgfpathlineto{\pgfqpoint{3.287800in}{2.380000in}}%
\pgfpathlineto{\pgfqpoint{3.289040in}{2.135000in}}%
\pgfpathlineto{\pgfqpoint{3.290280in}{2.415000in}}%
\pgfpathlineto{\pgfqpoint{3.291520in}{2.135000in}}%
\pgfpathlineto{\pgfqpoint{3.292760in}{2.205000in}}%
\pgfpathlineto{\pgfqpoint{3.294000in}{2.625000in}}%
\pgfpathlineto{\pgfqpoint{3.295240in}{2.625000in}}%
\pgfpathlineto{\pgfqpoint{3.296480in}{2.380000in}}%
\pgfpathlineto{\pgfqpoint{3.298960in}{2.625000in}}%
\pgfpathlineto{\pgfqpoint{3.300200in}{2.415000in}}%
\pgfpathlineto{\pgfqpoint{3.301440in}{2.485000in}}%
\pgfpathlineto{\pgfqpoint{3.303920in}{2.170000in}}%
\pgfpathlineto{\pgfqpoint{3.305160in}{2.450000in}}%
\pgfpathlineto{\pgfqpoint{3.306400in}{2.310000in}}%
\pgfpathlineto{\pgfqpoint{3.307640in}{2.730000in}}%
\pgfpathlineto{\pgfqpoint{3.308880in}{2.065000in}}%
\pgfpathlineto{\pgfqpoint{3.310120in}{2.380000in}}%
\pgfpathlineto{\pgfqpoint{3.311360in}{2.310000in}}%
\pgfpathlineto{\pgfqpoint{3.312600in}{2.170000in}}%
\pgfpathlineto{\pgfqpoint{3.313840in}{2.345000in}}%
\pgfpathlineto{\pgfqpoint{3.315080in}{2.275000in}}%
\pgfpathlineto{\pgfqpoint{3.317560in}{2.520000in}}%
\pgfpathlineto{\pgfqpoint{3.318800in}{2.520000in}}%
\pgfpathlineto{\pgfqpoint{3.320040in}{2.485000in}}%
\pgfpathlineto{\pgfqpoint{3.321280in}{2.555000in}}%
\pgfpathlineto{\pgfqpoint{3.322520in}{2.450000in}}%
\pgfpathlineto{\pgfqpoint{3.323760in}{2.730000in}}%
\pgfpathlineto{\pgfqpoint{3.326240in}{2.520000in}}%
\pgfpathlineto{\pgfqpoint{3.327480in}{2.555000in}}%
\pgfpathlineto{\pgfqpoint{3.328720in}{2.275000in}}%
\pgfpathlineto{\pgfqpoint{3.329960in}{2.310000in}}%
\pgfpathlineto{\pgfqpoint{3.331200in}{2.100000in}}%
\pgfpathlineto{\pgfqpoint{3.333680in}{2.590000in}}%
\pgfpathlineto{\pgfqpoint{3.334920in}{2.170000in}}%
\pgfpathlineto{\pgfqpoint{3.336160in}{2.660000in}}%
\pgfpathlineto{\pgfqpoint{3.337400in}{2.380000in}}%
\pgfpathlineto{\pgfqpoint{3.338640in}{2.695000in}}%
\pgfpathlineto{\pgfqpoint{3.342360in}{2.205000in}}%
\pgfpathlineto{\pgfqpoint{3.343600in}{2.275000in}}%
\pgfpathlineto{\pgfqpoint{3.344840in}{2.695000in}}%
\pgfpathlineto{\pgfqpoint{3.348560in}{2.135000in}}%
\pgfpathlineto{\pgfqpoint{3.349800in}{2.380000in}}%
\pgfpathlineto{\pgfqpoint{3.352280in}{2.380000in}}%
\pgfpathlineto{\pgfqpoint{3.353520in}{2.310000in}}%
\pgfpathlineto{\pgfqpoint{3.354760in}{2.520000in}}%
\pgfpathlineto{\pgfqpoint{3.356000in}{2.345000in}}%
\pgfpathlineto{\pgfqpoint{3.357240in}{2.765000in}}%
\pgfpathlineto{\pgfqpoint{3.358480in}{2.240000in}}%
\pgfpathlineto{\pgfqpoint{3.360960in}{2.485000in}}%
\pgfpathlineto{\pgfqpoint{3.362200in}{2.205000in}}%
\pgfpathlineto{\pgfqpoint{3.363440in}{2.660000in}}%
\pgfpathlineto{\pgfqpoint{3.364680in}{2.170000in}}%
\pgfpathlineto{\pgfqpoint{3.368400in}{2.590000in}}%
\pgfpathlineto{\pgfqpoint{3.369640in}{2.275000in}}%
\pgfpathlineto{\pgfqpoint{3.370880in}{2.415000in}}%
\pgfpathlineto{\pgfqpoint{3.372120in}{2.135000in}}%
\pgfpathlineto{\pgfqpoint{3.373360in}{2.485000in}}%
\pgfpathlineto{\pgfqpoint{3.374600in}{2.170000in}}%
\pgfpathlineto{\pgfqpoint{3.375840in}{2.275000in}}%
\pgfpathlineto{\pgfqpoint{3.377080in}{2.135000in}}%
\pgfpathlineto{\pgfqpoint{3.378320in}{2.345000in}}%
\pgfpathlineto{\pgfqpoint{3.379560in}{2.100000in}}%
\pgfpathlineto{\pgfqpoint{3.380800in}{2.100000in}}%
\pgfpathlineto{\pgfqpoint{3.382040in}{2.065000in}}%
\pgfpathlineto{\pgfqpoint{3.383280in}{1.925000in}}%
\pgfpathlineto{\pgfqpoint{3.384520in}{2.555000in}}%
\pgfpathlineto{\pgfqpoint{3.385760in}{2.310000in}}%
\pgfpathlineto{\pgfqpoint{3.387000in}{2.310000in}}%
\pgfpathlineto{\pgfqpoint{3.388240in}{2.170000in}}%
\pgfpathlineto{\pgfqpoint{3.389480in}{2.590000in}}%
\pgfpathlineto{\pgfqpoint{3.390720in}{2.415000in}}%
\pgfpathlineto{\pgfqpoint{3.391960in}{2.485000in}}%
\pgfpathlineto{\pgfqpoint{3.393200in}{2.625000in}}%
\pgfpathlineto{\pgfqpoint{3.396920in}{2.170000in}}%
\pgfpathlineto{\pgfqpoint{3.398160in}{2.240000in}}%
\pgfpathlineto{\pgfqpoint{3.399400in}{2.205000in}}%
\pgfpathlineto{\pgfqpoint{3.400640in}{2.625000in}}%
\pgfpathlineto{\pgfqpoint{3.401880in}{2.415000in}}%
\pgfpathlineto{\pgfqpoint{3.403120in}{2.450000in}}%
\pgfpathlineto{\pgfqpoint{3.404360in}{2.450000in}}%
\pgfpathlineto{\pgfqpoint{3.405600in}{2.380000in}}%
\pgfpathlineto{\pgfqpoint{3.406840in}{2.450000in}}%
\pgfpathlineto{\pgfqpoint{3.408080in}{2.415000in}}%
\pgfpathlineto{\pgfqpoint{3.409320in}{2.485000in}}%
\pgfpathlineto{\pgfqpoint{3.410560in}{2.240000in}}%
\pgfpathlineto{\pgfqpoint{3.411800in}{2.660000in}}%
\pgfpathlineto{\pgfqpoint{3.413040in}{2.240000in}}%
\pgfpathlineto{\pgfqpoint{3.414280in}{2.275000in}}%
\pgfpathlineto{\pgfqpoint{3.415520in}{2.730000in}}%
\pgfpathlineto{\pgfqpoint{3.416760in}{2.415000in}}%
\pgfpathlineto{\pgfqpoint{3.418000in}{2.765000in}}%
\pgfpathlineto{\pgfqpoint{3.419240in}{2.590000in}}%
\pgfpathlineto{\pgfqpoint{3.421720in}{2.695000in}}%
\pgfpathlineto{\pgfqpoint{3.424200in}{2.380000in}}%
\pgfpathlineto{\pgfqpoint{3.425440in}{2.135000in}}%
\pgfpathlineto{\pgfqpoint{3.426680in}{2.135000in}}%
\pgfpathlineto{\pgfqpoint{3.427920in}{2.485000in}}%
\pgfpathlineto{\pgfqpoint{3.429160in}{2.240000in}}%
\pgfpathlineto{\pgfqpoint{3.431640in}{2.380000in}}%
\pgfpathlineto{\pgfqpoint{3.432880in}{2.310000in}}%
\pgfpathlineto{\pgfqpoint{3.434120in}{2.555000in}}%
\pgfpathlineto{\pgfqpoint{3.436600in}{2.205000in}}%
\pgfpathlineto{\pgfqpoint{3.437840in}{1.925000in}}%
\pgfpathlineto{\pgfqpoint{3.439080in}{2.415000in}}%
\pgfpathlineto{\pgfqpoint{3.440320in}{2.345000in}}%
\pgfpathlineto{\pgfqpoint{3.441560in}{2.695000in}}%
\pgfpathlineto{\pgfqpoint{3.442800in}{2.415000in}}%
\pgfpathlineto{\pgfqpoint{3.444040in}{2.730000in}}%
\pgfpathlineto{\pgfqpoint{3.446520in}{2.730000in}}%
\pgfpathlineto{\pgfqpoint{3.447760in}{2.520000in}}%
\pgfpathlineto{\pgfqpoint{3.449000in}{2.905000in}}%
\pgfpathlineto{\pgfqpoint{3.451480in}{2.415000in}}%
\pgfpathlineto{\pgfqpoint{3.453960in}{2.625000in}}%
\pgfpathlineto{\pgfqpoint{3.455200in}{2.275000in}}%
\pgfpathlineto{\pgfqpoint{3.457680in}{2.870000in}}%
\pgfpathlineto{\pgfqpoint{3.458920in}{2.765000in}}%
\pgfpathlineto{\pgfqpoint{3.461400in}{2.345000in}}%
\pgfpathlineto{\pgfqpoint{3.462640in}{2.520000in}}%
\pgfpathlineto{\pgfqpoint{3.465120in}{2.240000in}}%
\pgfpathlineto{\pgfqpoint{3.466360in}{2.520000in}}%
\pgfpathlineto{\pgfqpoint{3.467600in}{2.240000in}}%
\pgfpathlineto{\pgfqpoint{3.468840in}{2.380000in}}%
\pgfpathlineto{\pgfqpoint{3.470080in}{2.905000in}}%
\pgfpathlineto{\pgfqpoint{3.473800in}{2.345000in}}%
\pgfpathlineto{\pgfqpoint{3.475040in}{2.590000in}}%
\pgfpathlineto{\pgfqpoint{3.478760in}{2.030000in}}%
\pgfpathlineto{\pgfqpoint{3.480000in}{2.590000in}}%
\pgfpathlineto{\pgfqpoint{3.481240in}{2.380000in}}%
\pgfpathlineto{\pgfqpoint{3.482480in}{2.380000in}}%
\pgfpathlineto{\pgfqpoint{3.483720in}{2.835000in}}%
\pgfpathlineto{\pgfqpoint{3.484960in}{2.660000in}}%
\pgfpathlineto{\pgfqpoint{3.486200in}{2.905000in}}%
\pgfpathlineto{\pgfqpoint{3.487440in}{2.450000in}}%
\pgfpathlineto{\pgfqpoint{3.488680in}{2.450000in}}%
\pgfpathlineto{\pgfqpoint{3.489920in}{2.275000in}}%
\pgfpathlineto{\pgfqpoint{3.491160in}{2.555000in}}%
\pgfpathlineto{\pgfqpoint{3.493640in}{2.310000in}}%
\pgfpathlineto{\pgfqpoint{3.494880in}{2.380000in}}%
\pgfpathlineto{\pgfqpoint{3.496120in}{2.590000in}}%
\pgfpathlineto{\pgfqpoint{3.497360in}{2.310000in}}%
\pgfpathlineto{\pgfqpoint{3.498600in}{2.625000in}}%
\pgfpathlineto{\pgfqpoint{3.499840in}{2.380000in}}%
\pgfpathlineto{\pgfqpoint{3.502320in}{2.695000in}}%
\pgfpathlineto{\pgfqpoint{3.503560in}{2.520000in}}%
\pgfpathlineto{\pgfqpoint{3.504800in}{2.520000in}}%
\pgfpathlineto{\pgfqpoint{3.506040in}{2.415000in}}%
\pgfpathlineto{\pgfqpoint{3.507280in}{2.555000in}}%
\pgfpathlineto{\pgfqpoint{3.508520in}{2.310000in}}%
\pgfpathlineto{\pgfqpoint{3.509760in}{2.310000in}}%
\pgfpathlineto{\pgfqpoint{3.511000in}{2.625000in}}%
\pgfpathlineto{\pgfqpoint{3.513480in}{2.170000in}}%
\pgfpathlineto{\pgfqpoint{3.514720in}{2.905000in}}%
\pgfpathlineto{\pgfqpoint{3.515960in}{2.100000in}}%
\pgfpathlineto{\pgfqpoint{3.517200in}{2.555000in}}%
\pgfpathlineto{\pgfqpoint{3.518440in}{2.170000in}}%
\pgfpathlineto{\pgfqpoint{3.520920in}{2.590000in}}%
\pgfpathlineto{\pgfqpoint{3.522160in}{2.240000in}}%
\pgfpathlineto{\pgfqpoint{3.523400in}{2.275000in}}%
\pgfpathlineto{\pgfqpoint{3.524640in}{2.555000in}}%
\pgfpathlineto{\pgfqpoint{3.525880in}{2.520000in}}%
\pgfpathlineto{\pgfqpoint{3.527120in}{2.695000in}}%
\pgfpathlineto{\pgfqpoint{3.528360in}{2.625000in}}%
\pgfpathlineto{\pgfqpoint{3.529600in}{2.345000in}}%
\pgfpathlineto{\pgfqpoint{3.532080in}{2.590000in}}%
\pgfpathlineto{\pgfqpoint{3.533320in}{2.590000in}}%
\pgfpathlineto{\pgfqpoint{3.534560in}{2.415000in}}%
\pgfpathlineto{\pgfqpoint{3.535800in}{2.660000in}}%
\pgfpathlineto{\pgfqpoint{3.537040in}{2.205000in}}%
\pgfpathlineto{\pgfqpoint{3.538280in}{2.730000in}}%
\pgfpathlineto{\pgfqpoint{3.539520in}{2.240000in}}%
\pgfpathlineto{\pgfqpoint{3.540760in}{2.555000in}}%
\pgfpathlineto{\pgfqpoint{3.542000in}{2.450000in}}%
\pgfpathlineto{\pgfqpoint{3.543240in}{2.205000in}}%
\pgfpathlineto{\pgfqpoint{3.545720in}{2.625000in}}%
\pgfpathlineto{\pgfqpoint{3.548200in}{2.205000in}}%
\pgfpathlineto{\pgfqpoint{3.549440in}{2.275000in}}%
\pgfpathlineto{\pgfqpoint{3.550680in}{2.590000in}}%
\pgfpathlineto{\pgfqpoint{3.553160in}{2.100000in}}%
\pgfpathlineto{\pgfqpoint{3.554400in}{2.905000in}}%
\pgfpathlineto{\pgfqpoint{3.555640in}{2.240000in}}%
\pgfpathlineto{\pgfqpoint{3.556880in}{2.485000in}}%
\pgfpathlineto{\pgfqpoint{3.558120in}{2.135000in}}%
\pgfpathlineto{\pgfqpoint{3.559360in}{2.100000in}}%
\pgfpathlineto{\pgfqpoint{3.560600in}{2.135000in}}%
\pgfpathlineto{\pgfqpoint{3.561840in}{2.625000in}}%
\pgfpathlineto{\pgfqpoint{3.564320in}{2.520000in}}%
\pgfpathlineto{\pgfqpoint{3.566800in}{2.275000in}}%
\pgfpathlineto{\pgfqpoint{3.568040in}{2.170000in}}%
\pgfpathlineto{\pgfqpoint{3.569280in}{2.485000in}}%
\pgfpathlineto{\pgfqpoint{3.571760in}{2.310000in}}%
\pgfpathlineto{\pgfqpoint{3.573000in}{2.170000in}}%
\pgfpathlineto{\pgfqpoint{3.575480in}{2.625000in}}%
\pgfpathlineto{\pgfqpoint{3.576720in}{2.065000in}}%
\pgfpathlineto{\pgfqpoint{3.577960in}{2.625000in}}%
\pgfpathlineto{\pgfqpoint{3.579200in}{2.660000in}}%
\pgfpathlineto{\pgfqpoint{3.580440in}{2.625000in}}%
\pgfpathlineto{\pgfqpoint{3.581680in}{2.415000in}}%
\pgfpathlineto{\pgfqpoint{3.582920in}{2.590000in}}%
\pgfpathlineto{\pgfqpoint{3.584160in}{2.450000in}}%
\pgfpathlineto{\pgfqpoint{3.585400in}{3.010000in}}%
\pgfpathlineto{\pgfqpoint{3.586640in}{2.345000in}}%
\pgfpathlineto{\pgfqpoint{3.587880in}{2.520000in}}%
\pgfpathlineto{\pgfqpoint{3.589120in}{2.485000in}}%
\pgfpathlineto{\pgfqpoint{3.590360in}{2.695000in}}%
\pgfpathlineto{\pgfqpoint{3.594080in}{1.855000in}}%
\pgfpathlineto{\pgfqpoint{3.595320in}{2.205000in}}%
\pgfpathlineto{\pgfqpoint{3.596560in}{2.170000in}}%
\pgfpathlineto{\pgfqpoint{3.597800in}{2.380000in}}%
\pgfpathlineto{\pgfqpoint{3.599040in}{2.170000in}}%
\pgfpathlineto{\pgfqpoint{3.600280in}{2.590000in}}%
\pgfpathlineto{\pgfqpoint{3.601520in}{2.555000in}}%
\pgfpathlineto{\pgfqpoint{3.602760in}{2.100000in}}%
\pgfpathlineto{\pgfqpoint{3.605240in}{2.275000in}}%
\pgfpathlineto{\pgfqpoint{3.607720in}{1.960000in}}%
\pgfpathlineto{\pgfqpoint{3.608960in}{2.310000in}}%
\pgfpathlineto{\pgfqpoint{3.610200in}{2.345000in}}%
\pgfpathlineto{\pgfqpoint{3.611440in}{2.275000in}}%
\pgfpathlineto{\pgfqpoint{3.612680in}{2.275000in}}%
\pgfpathlineto{\pgfqpoint{3.613920in}{2.380000in}}%
\pgfpathlineto{\pgfqpoint{3.615160in}{2.205000in}}%
\pgfpathlineto{\pgfqpoint{3.616400in}{2.345000in}}%
\pgfpathlineto{\pgfqpoint{3.617640in}{2.345000in}}%
\pgfpathlineto{\pgfqpoint{3.618880in}{2.415000in}}%
\pgfpathlineto{\pgfqpoint{3.620120in}{1.925000in}}%
\pgfpathlineto{\pgfqpoint{3.622600in}{2.415000in}}%
\pgfpathlineto{\pgfqpoint{3.623840in}{2.240000in}}%
\pgfpathlineto{\pgfqpoint{3.625080in}{2.835000in}}%
\pgfpathlineto{\pgfqpoint{3.626320in}{2.555000in}}%
\pgfpathlineto{\pgfqpoint{3.627560in}{1.960000in}}%
\pgfpathlineto{\pgfqpoint{3.628800in}{2.520000in}}%
\pgfpathlineto{\pgfqpoint{3.631280in}{1.995000in}}%
\pgfpathlineto{\pgfqpoint{3.632520in}{2.275000in}}%
\pgfpathlineto{\pgfqpoint{3.633760in}{2.275000in}}%
\pgfpathlineto{\pgfqpoint{3.635000in}{2.485000in}}%
\pgfpathlineto{\pgfqpoint{3.637480in}{2.310000in}}%
\pgfpathlineto{\pgfqpoint{3.638720in}{2.170000in}}%
\pgfpathlineto{\pgfqpoint{3.639960in}{2.660000in}}%
\pgfpathlineto{\pgfqpoint{3.641200in}{2.625000in}}%
\pgfpathlineto{\pgfqpoint{3.643680in}{2.345000in}}%
\pgfpathlineto{\pgfqpoint{3.644920in}{2.450000in}}%
\pgfpathlineto{\pgfqpoint{3.646160in}{2.835000in}}%
\pgfpathlineto{\pgfqpoint{3.647400in}{2.345000in}}%
\pgfpathlineto{\pgfqpoint{3.648640in}{2.660000in}}%
\pgfpathlineto{\pgfqpoint{3.649880in}{2.625000in}}%
\pgfpathlineto{\pgfqpoint{3.651120in}{2.310000in}}%
\pgfpathlineto{\pgfqpoint{3.652360in}{2.345000in}}%
\pgfpathlineto{\pgfqpoint{3.653600in}{2.450000in}}%
\pgfpathlineto{\pgfqpoint{3.654840in}{2.730000in}}%
\pgfpathlineto{\pgfqpoint{3.656080in}{2.135000in}}%
\pgfpathlineto{\pgfqpoint{3.657320in}{2.660000in}}%
\pgfpathlineto{\pgfqpoint{3.658560in}{2.660000in}}%
\pgfpathlineto{\pgfqpoint{3.659800in}{2.205000in}}%
\pgfpathlineto{\pgfqpoint{3.661040in}{2.485000in}}%
\pgfpathlineto{\pgfqpoint{3.662280in}{2.450000in}}%
\pgfpathlineto{\pgfqpoint{3.663520in}{2.485000in}}%
\pgfpathlineto{\pgfqpoint{3.666000in}{2.275000in}}%
\pgfpathlineto{\pgfqpoint{3.669720in}{2.695000in}}%
\pgfpathlineto{\pgfqpoint{3.670960in}{2.170000in}}%
\pgfpathlineto{\pgfqpoint{3.673440in}{2.520000in}}%
\pgfpathlineto{\pgfqpoint{3.674680in}{2.625000in}}%
\pgfpathlineto{\pgfqpoint{3.675920in}{2.310000in}}%
\pgfpathlineto{\pgfqpoint{3.677160in}{2.345000in}}%
\pgfpathlineto{\pgfqpoint{3.678400in}{2.100000in}}%
\pgfpathlineto{\pgfqpoint{3.680880in}{2.625000in}}%
\pgfpathlineto{\pgfqpoint{3.682120in}{2.415000in}}%
\pgfpathlineto{\pgfqpoint{3.683360in}{2.625000in}}%
\pgfpathlineto{\pgfqpoint{3.684600in}{1.925000in}}%
\pgfpathlineto{\pgfqpoint{3.685840in}{2.625000in}}%
\pgfpathlineto{\pgfqpoint{3.687080in}{2.695000in}}%
\pgfpathlineto{\pgfqpoint{3.689560in}{2.520000in}}%
\pgfpathlineto{\pgfqpoint{3.690800in}{2.590000in}}%
\pgfpathlineto{\pgfqpoint{3.692040in}{2.450000in}}%
\pgfpathlineto{\pgfqpoint{3.693280in}{2.450000in}}%
\pgfpathlineto{\pgfqpoint{3.694520in}{2.415000in}}%
\pgfpathlineto{\pgfqpoint{3.698240in}{2.835000in}}%
\pgfpathlineto{\pgfqpoint{3.699480in}{2.590000in}}%
\pgfpathlineto{\pgfqpoint{3.700720in}{2.905000in}}%
\pgfpathlineto{\pgfqpoint{3.701960in}{2.345000in}}%
\pgfpathlineto{\pgfqpoint{3.703200in}{2.625000in}}%
\pgfpathlineto{\pgfqpoint{3.704440in}{2.555000in}}%
\pgfpathlineto{\pgfqpoint{3.705680in}{2.310000in}}%
\pgfpathlineto{\pgfqpoint{3.706920in}{2.310000in}}%
\pgfpathlineto{\pgfqpoint{3.708160in}{2.485000in}}%
\pgfpathlineto{\pgfqpoint{3.709400in}{2.380000in}}%
\pgfpathlineto{\pgfqpoint{3.710640in}{2.730000in}}%
\pgfpathlineto{\pgfqpoint{3.711880in}{2.380000in}}%
\pgfpathlineto{\pgfqpoint{3.713120in}{2.625000in}}%
\pgfpathlineto{\pgfqpoint{3.714360in}{2.590000in}}%
\pgfpathlineto{\pgfqpoint{3.715600in}{2.030000in}}%
\pgfpathlineto{\pgfqpoint{3.718080in}{2.485000in}}%
\pgfpathlineto{\pgfqpoint{3.719320in}{2.450000in}}%
\pgfpathlineto{\pgfqpoint{3.720560in}{2.485000in}}%
\pgfpathlineto{\pgfqpoint{3.721800in}{2.590000in}}%
\pgfpathlineto{\pgfqpoint{3.724280in}{2.520000in}}%
\pgfpathlineto{\pgfqpoint{3.726760in}{2.660000in}}%
\pgfpathlineto{\pgfqpoint{3.728000in}{2.345000in}}%
\pgfpathlineto{\pgfqpoint{3.729240in}{2.625000in}}%
\pgfpathlineto{\pgfqpoint{3.731720in}{2.205000in}}%
\pgfpathlineto{\pgfqpoint{3.732960in}{2.555000in}}%
\pgfpathlineto{\pgfqpoint{3.734200in}{2.135000in}}%
\pgfpathlineto{\pgfqpoint{3.736680in}{2.835000in}}%
\pgfpathlineto{\pgfqpoint{3.739160in}{2.030000in}}%
\pgfpathlineto{\pgfqpoint{3.740400in}{2.555000in}}%
\pgfpathlineto{\pgfqpoint{3.742880in}{2.205000in}}%
\pgfpathlineto{\pgfqpoint{3.744120in}{2.030000in}}%
\pgfpathlineto{\pgfqpoint{3.745360in}{2.450000in}}%
\pgfpathlineto{\pgfqpoint{3.746600in}{2.205000in}}%
\pgfpathlineto{\pgfqpoint{3.747840in}{2.520000in}}%
\pgfpathlineto{\pgfqpoint{3.749080in}{2.380000in}}%
\pgfpathlineto{\pgfqpoint{3.750320in}{2.520000in}}%
\pgfpathlineto{\pgfqpoint{3.751560in}{2.345000in}}%
\pgfpathlineto{\pgfqpoint{3.752800in}{2.660000in}}%
\pgfpathlineto{\pgfqpoint{3.755280in}{2.240000in}}%
\pgfpathlineto{\pgfqpoint{3.756520in}{2.590000in}}%
\pgfpathlineto{\pgfqpoint{3.757760in}{2.520000in}}%
\pgfpathlineto{\pgfqpoint{3.759000in}{2.590000in}}%
\pgfpathlineto{\pgfqpoint{3.761480in}{2.240000in}}%
\pgfpathlineto{\pgfqpoint{3.762720in}{2.240000in}}%
\pgfpathlineto{\pgfqpoint{3.765200in}{2.520000in}}%
\pgfpathlineto{\pgfqpoint{3.767680in}{2.240000in}}%
\pgfpathlineto{\pgfqpoint{3.768920in}{2.555000in}}%
\pgfpathlineto{\pgfqpoint{3.770160in}{2.485000in}}%
\pgfpathlineto{\pgfqpoint{3.771400in}{2.520000in}}%
\pgfpathlineto{\pgfqpoint{3.772640in}{2.100000in}}%
\pgfpathlineto{\pgfqpoint{3.773880in}{2.485000in}}%
\pgfpathlineto{\pgfqpoint{3.775120in}{2.170000in}}%
\pgfpathlineto{\pgfqpoint{3.776360in}{2.415000in}}%
\pgfpathlineto{\pgfqpoint{3.777600in}{2.345000in}}%
\pgfpathlineto{\pgfqpoint{3.778840in}{2.555000in}}%
\pgfpathlineto{\pgfqpoint{3.780080in}{2.240000in}}%
\pgfpathlineto{\pgfqpoint{3.781320in}{2.520000in}}%
\pgfpathlineto{\pgfqpoint{3.783800in}{2.345000in}}%
\pgfpathlineto{\pgfqpoint{3.786280in}{2.205000in}}%
\pgfpathlineto{\pgfqpoint{3.788760in}{2.485000in}}%
\pgfpathlineto{\pgfqpoint{3.790000in}{2.205000in}}%
\pgfpathlineto{\pgfqpoint{3.791240in}{2.520000in}}%
\pgfpathlineto{\pgfqpoint{3.793720in}{2.310000in}}%
\pgfpathlineto{\pgfqpoint{3.796200in}{2.625000in}}%
\pgfpathlineto{\pgfqpoint{3.797440in}{2.625000in}}%
\pgfpathlineto{\pgfqpoint{3.798680in}{2.275000in}}%
\pgfpathlineto{\pgfqpoint{3.799920in}{2.415000in}}%
\pgfpathlineto{\pgfqpoint{3.801160in}{2.275000in}}%
\pgfpathlineto{\pgfqpoint{3.802400in}{2.275000in}}%
\pgfpathlineto{\pgfqpoint{3.803640in}{2.205000in}}%
\pgfpathlineto{\pgfqpoint{3.804880in}{2.240000in}}%
\pgfpathlineto{\pgfqpoint{3.806120in}{2.310000in}}%
\pgfpathlineto{\pgfqpoint{3.808600in}{2.555000in}}%
\pgfpathlineto{\pgfqpoint{3.809840in}{2.520000in}}%
\pgfpathlineto{\pgfqpoint{3.811080in}{2.660000in}}%
\pgfpathlineto{\pgfqpoint{3.813560in}{2.415000in}}%
\pgfpathlineto{\pgfqpoint{3.814800in}{2.345000in}}%
\pgfpathlineto{\pgfqpoint{3.816040in}{2.590000in}}%
\pgfpathlineto{\pgfqpoint{3.818520in}{2.345000in}}%
\pgfpathlineto{\pgfqpoint{3.819760in}{2.310000in}}%
\pgfpathlineto{\pgfqpoint{3.822240in}{2.450000in}}%
\pgfpathlineto{\pgfqpoint{3.823480in}{2.520000in}}%
\pgfpathlineto{\pgfqpoint{3.824720in}{2.450000in}}%
\pgfpathlineto{\pgfqpoint{3.825960in}{2.765000in}}%
\pgfpathlineto{\pgfqpoint{3.827200in}{2.590000in}}%
\pgfpathlineto{\pgfqpoint{3.828440in}{2.205000in}}%
\pgfpathlineto{\pgfqpoint{3.829680in}{2.660000in}}%
\pgfpathlineto{\pgfqpoint{3.830920in}{2.590000in}}%
\pgfpathlineto{\pgfqpoint{3.832160in}{2.310000in}}%
\pgfpathlineto{\pgfqpoint{3.834640in}{2.625000in}}%
\pgfpathlineto{\pgfqpoint{3.835880in}{2.905000in}}%
\pgfpathlineto{\pgfqpoint{3.837120in}{2.555000in}}%
\pgfpathlineto{\pgfqpoint{3.838360in}{2.835000in}}%
\pgfpathlineto{\pgfqpoint{3.840840in}{2.310000in}}%
\pgfpathlineto{\pgfqpoint{3.842080in}{2.275000in}}%
\pgfpathlineto{\pgfqpoint{3.843320in}{2.135000in}}%
\pgfpathlineto{\pgfqpoint{3.844560in}{2.555000in}}%
\pgfpathlineto{\pgfqpoint{3.845800in}{2.240000in}}%
\pgfpathlineto{\pgfqpoint{3.847040in}{2.310000in}}%
\pgfpathlineto{\pgfqpoint{3.848280in}{2.800000in}}%
\pgfpathlineto{\pgfqpoint{3.850760in}{2.380000in}}%
\pgfpathlineto{\pgfqpoint{3.852000in}{2.415000in}}%
\pgfpathlineto{\pgfqpoint{3.853240in}{2.625000in}}%
\pgfpathlineto{\pgfqpoint{3.854480in}{2.345000in}}%
\pgfpathlineto{\pgfqpoint{3.855720in}{2.625000in}}%
\pgfpathlineto{\pgfqpoint{3.859440in}{2.625000in}}%
\pgfpathlineto{\pgfqpoint{3.861920in}{2.450000in}}%
\pgfpathlineto{\pgfqpoint{3.863160in}{2.590000in}}%
\pgfpathlineto{\pgfqpoint{3.864400in}{2.520000in}}%
\pgfpathlineto{\pgfqpoint{3.865640in}{2.800000in}}%
\pgfpathlineto{\pgfqpoint{3.866880in}{2.695000in}}%
\pgfpathlineto{\pgfqpoint{3.868120in}{2.240000in}}%
\pgfpathlineto{\pgfqpoint{3.870600in}{2.485000in}}%
\pgfpathlineto{\pgfqpoint{3.871840in}{2.205000in}}%
\pgfpathlineto{\pgfqpoint{3.873080in}{2.555000in}}%
\pgfpathlineto{\pgfqpoint{3.874320in}{2.590000in}}%
\pgfpathlineto{\pgfqpoint{3.875560in}{2.590000in}}%
\pgfpathlineto{\pgfqpoint{3.876800in}{2.205000in}}%
\pgfpathlineto{\pgfqpoint{3.878040in}{2.170000in}}%
\pgfpathlineto{\pgfqpoint{3.880520in}{2.590000in}}%
\pgfpathlineto{\pgfqpoint{3.881760in}{2.415000in}}%
\pgfpathlineto{\pgfqpoint{3.883000in}{2.415000in}}%
\pgfpathlineto{\pgfqpoint{3.884240in}{2.590000in}}%
\pgfpathlineto{\pgfqpoint{3.885480in}{2.380000in}}%
\pgfpathlineto{\pgfqpoint{3.886720in}{2.555000in}}%
\pgfpathlineto{\pgfqpoint{3.887960in}{2.485000in}}%
\pgfpathlineto{\pgfqpoint{3.890440in}{2.660000in}}%
\pgfpathlineto{\pgfqpoint{3.892920in}{2.240000in}}%
\pgfpathlineto{\pgfqpoint{3.894160in}{2.940000in}}%
\pgfpathlineto{\pgfqpoint{3.895400in}{2.345000in}}%
\pgfpathlineto{\pgfqpoint{3.896640in}{2.345000in}}%
\pgfpathlineto{\pgfqpoint{3.897880in}{2.485000in}}%
\pgfpathlineto{\pgfqpoint{3.899120in}{2.030000in}}%
\pgfpathlineto{\pgfqpoint{3.900360in}{2.450000in}}%
\pgfpathlineto{\pgfqpoint{3.901600in}{2.380000in}}%
\pgfpathlineto{\pgfqpoint{3.902840in}{2.625000in}}%
\pgfpathlineto{\pgfqpoint{3.904080in}{2.345000in}}%
\pgfpathlineto{\pgfqpoint{3.905320in}{2.415000in}}%
\pgfpathlineto{\pgfqpoint{3.907800in}{1.995000in}}%
\pgfpathlineto{\pgfqpoint{3.910280in}{2.485000in}}%
\pgfpathlineto{\pgfqpoint{3.911520in}{2.450000in}}%
\pgfpathlineto{\pgfqpoint{3.912760in}{2.485000in}}%
\pgfpathlineto{\pgfqpoint{3.914000in}{2.345000in}}%
\pgfpathlineto{\pgfqpoint{3.915240in}{2.520000in}}%
\pgfpathlineto{\pgfqpoint{3.917720in}{2.275000in}}%
\pgfpathlineto{\pgfqpoint{3.918960in}{2.835000in}}%
\pgfpathlineto{\pgfqpoint{3.921440in}{2.240000in}}%
\pgfpathlineto{\pgfqpoint{3.922680in}{2.380000in}}%
\pgfpathlineto{\pgfqpoint{3.923920in}{2.345000in}}%
\pgfpathlineto{\pgfqpoint{3.926400in}{2.870000in}}%
\pgfpathlineto{\pgfqpoint{3.928880in}{2.450000in}}%
\pgfpathlineto{\pgfqpoint{3.930120in}{2.450000in}}%
\pgfpathlineto{\pgfqpoint{3.931360in}{2.380000in}}%
\pgfpathlineto{\pgfqpoint{3.932600in}{2.730000in}}%
\pgfpathlineto{\pgfqpoint{3.933840in}{2.695000in}}%
\pgfpathlineto{\pgfqpoint{3.935080in}{2.695000in}}%
\pgfpathlineto{\pgfqpoint{3.936320in}{2.380000in}}%
\pgfpathlineto{\pgfqpoint{3.938800in}{2.520000in}}%
\pgfpathlineto{\pgfqpoint{3.940040in}{2.240000in}}%
\pgfpathlineto{\pgfqpoint{3.941280in}{2.730000in}}%
\pgfpathlineto{\pgfqpoint{3.943760in}{2.730000in}}%
\pgfpathlineto{\pgfqpoint{3.945000in}{2.835000in}}%
\pgfpathlineto{\pgfqpoint{3.949960in}{2.170000in}}%
\pgfpathlineto{\pgfqpoint{3.951200in}{2.660000in}}%
\pgfpathlineto{\pgfqpoint{3.952440in}{2.520000in}}%
\pgfpathlineto{\pgfqpoint{3.953680in}{2.730000in}}%
\pgfpathlineto{\pgfqpoint{3.956160in}{2.310000in}}%
\pgfpathlineto{\pgfqpoint{3.959880in}{2.800000in}}%
\pgfpathlineto{\pgfqpoint{3.962360in}{2.450000in}}%
\pgfpathlineto{\pgfqpoint{3.963600in}{2.485000in}}%
\pgfpathlineto{\pgfqpoint{3.966080in}{2.065000in}}%
\pgfpathlineto{\pgfqpoint{3.967320in}{1.995000in}}%
\pgfpathlineto{\pgfqpoint{3.968560in}{2.730000in}}%
\pgfpathlineto{\pgfqpoint{3.969800in}{2.135000in}}%
\pgfpathlineto{\pgfqpoint{3.971040in}{2.170000in}}%
\pgfpathlineto{\pgfqpoint{3.972280in}{2.275000in}}%
\pgfpathlineto{\pgfqpoint{3.973520in}{2.660000in}}%
\pgfpathlineto{\pgfqpoint{3.974760in}{2.520000in}}%
\pgfpathlineto{\pgfqpoint{3.976000in}{2.240000in}}%
\pgfpathlineto{\pgfqpoint{3.977240in}{2.310000in}}%
\pgfpathlineto{\pgfqpoint{3.978480in}{2.240000in}}%
\pgfpathlineto{\pgfqpoint{3.979720in}{2.590000in}}%
\pgfpathlineto{\pgfqpoint{3.982200in}{2.415000in}}%
\pgfpathlineto{\pgfqpoint{3.983440in}{2.555000in}}%
\pgfpathlineto{\pgfqpoint{3.984680in}{2.555000in}}%
\pgfpathlineto{\pgfqpoint{3.985920in}{2.135000in}}%
\pgfpathlineto{\pgfqpoint{3.987160in}{2.380000in}}%
\pgfpathlineto{\pgfqpoint{3.989640in}{2.135000in}}%
\pgfpathlineto{\pgfqpoint{3.990880in}{1.995000in}}%
\pgfpathlineto{\pgfqpoint{3.992120in}{2.590000in}}%
\pgfpathlineto{\pgfqpoint{3.993360in}{2.625000in}}%
\pgfpathlineto{\pgfqpoint{3.994600in}{1.925000in}}%
\pgfpathlineto{\pgfqpoint{3.997080in}{2.275000in}}%
\pgfpathlineto{\pgfqpoint{3.998320in}{2.065000in}}%
\pgfpathlineto{\pgfqpoint{4.000800in}{2.520000in}}%
\pgfpathlineto{\pgfqpoint{4.002040in}{2.100000in}}%
\pgfpathlineto{\pgfqpoint{4.003280in}{2.135000in}}%
\pgfpathlineto{\pgfqpoint{4.004520in}{2.205000in}}%
\pgfpathlineto{\pgfqpoint{4.007000in}{2.555000in}}%
\pgfpathlineto{\pgfqpoint{4.009480in}{2.380000in}}%
\pgfpathlineto{\pgfqpoint{4.010720in}{2.310000in}}%
\pgfpathlineto{\pgfqpoint{4.011960in}{2.450000in}}%
\pgfpathlineto{\pgfqpoint{4.013200in}{1.855000in}}%
\pgfpathlineto{\pgfqpoint{4.014440in}{1.855000in}}%
\pgfpathlineto{\pgfqpoint{4.015680in}{1.925000in}}%
\pgfpathlineto{\pgfqpoint{4.016920in}{1.785000in}}%
\pgfpathlineto{\pgfqpoint{4.019400in}{2.065000in}}%
\pgfpathlineto{\pgfqpoint{4.020640in}{1.960000in}}%
\pgfpathlineto{\pgfqpoint{4.021880in}{1.995000in}}%
\pgfpathlineto{\pgfqpoint{4.023120in}{2.065000in}}%
\pgfpathlineto{\pgfqpoint{4.024360in}{2.450000in}}%
\pgfpathlineto{\pgfqpoint{4.025600in}{1.820000in}}%
\pgfpathlineto{\pgfqpoint{4.026840in}{2.100000in}}%
\pgfpathlineto{\pgfqpoint{4.028080in}{2.065000in}}%
\pgfpathlineto{\pgfqpoint{4.030560in}{2.485000in}}%
\pgfpathlineto{\pgfqpoint{4.031800in}{2.450000in}}%
\pgfpathlineto{\pgfqpoint{4.033040in}{2.450000in}}%
\pgfpathlineto{\pgfqpoint{4.034280in}{2.240000in}}%
\pgfpathlineto{\pgfqpoint{4.035520in}{2.485000in}}%
\pgfpathlineto{\pgfqpoint{4.036760in}{1.890000in}}%
\pgfpathlineto{\pgfqpoint{4.038000in}{2.380000in}}%
\pgfpathlineto{\pgfqpoint{4.039240in}{1.995000in}}%
\pgfpathlineto{\pgfqpoint{4.040480in}{2.555000in}}%
\pgfpathlineto{\pgfqpoint{4.041720in}{2.485000in}}%
\pgfpathlineto{\pgfqpoint{4.042960in}{1.855000in}}%
\pgfpathlineto{\pgfqpoint{4.044200in}{2.415000in}}%
\pgfpathlineto{\pgfqpoint{4.045440in}{1.820000in}}%
\pgfpathlineto{\pgfqpoint{4.046680in}{2.555000in}}%
\pgfpathlineto{\pgfqpoint{4.047920in}{2.100000in}}%
\pgfpathlineto{\pgfqpoint{4.049160in}{2.310000in}}%
\pgfpathlineto{\pgfqpoint{4.051640in}{2.135000in}}%
\pgfpathlineto{\pgfqpoint{4.052880in}{2.590000in}}%
\pgfpathlineto{\pgfqpoint{4.054120in}{2.310000in}}%
\pgfpathlineto{\pgfqpoint{4.055360in}{2.450000in}}%
\pgfpathlineto{\pgfqpoint{4.056600in}{2.765000in}}%
\pgfpathlineto{\pgfqpoint{4.057840in}{2.485000in}}%
\pgfpathlineto{\pgfqpoint{4.059080in}{2.695000in}}%
\pgfpathlineto{\pgfqpoint{4.060320in}{2.380000in}}%
\pgfpathlineto{\pgfqpoint{4.061560in}{2.590000in}}%
\pgfpathlineto{\pgfqpoint{4.062800in}{2.415000in}}%
\pgfpathlineto{\pgfqpoint{4.064040in}{2.485000in}}%
\pgfpathlineto{\pgfqpoint{4.065280in}{2.415000in}}%
\pgfpathlineto{\pgfqpoint{4.066520in}{2.555000in}}%
\pgfpathlineto{\pgfqpoint{4.067760in}{2.380000in}}%
\pgfpathlineto{\pgfqpoint{4.069000in}{2.590000in}}%
\pgfpathlineto{\pgfqpoint{4.071480in}{2.415000in}}%
\pgfpathlineto{\pgfqpoint{4.072720in}{2.205000in}}%
\pgfpathlineto{\pgfqpoint{4.075200in}{2.310000in}}%
\pgfpathlineto{\pgfqpoint{4.076440in}{2.800000in}}%
\pgfpathlineto{\pgfqpoint{4.077680in}{2.695000in}}%
\pgfpathlineto{\pgfqpoint{4.078920in}{2.275000in}}%
\pgfpathlineto{\pgfqpoint{4.080160in}{2.240000in}}%
\pgfpathlineto{\pgfqpoint{4.081400in}{2.520000in}}%
\pgfpathlineto{\pgfqpoint{4.082640in}{1.925000in}}%
\pgfpathlineto{\pgfqpoint{4.083880in}{2.555000in}}%
\pgfpathlineto{\pgfqpoint{4.085120in}{2.520000in}}%
\pgfpathlineto{\pgfqpoint{4.087600in}{2.380000in}}%
\pgfpathlineto{\pgfqpoint{4.088840in}{2.450000in}}%
\pgfpathlineto{\pgfqpoint{4.090080in}{2.170000in}}%
\pgfpathlineto{\pgfqpoint{4.091320in}{2.590000in}}%
\pgfpathlineto{\pgfqpoint{4.093800in}{1.820000in}}%
\pgfpathlineto{\pgfqpoint{4.096280in}{2.555000in}}%
\pgfpathlineto{\pgfqpoint{4.097520in}{2.310000in}}%
\pgfpathlineto{\pgfqpoint{4.100000in}{2.415000in}}%
\pgfpathlineto{\pgfqpoint{4.101240in}{2.170000in}}%
\pgfpathlineto{\pgfqpoint{4.102480in}{2.275000in}}%
\pgfpathlineto{\pgfqpoint{4.103720in}{2.275000in}}%
\pgfpathlineto{\pgfqpoint{4.106200in}{2.415000in}}%
\pgfpathlineto{\pgfqpoint{4.108680in}{2.345000in}}%
\pgfpathlineto{\pgfqpoint{4.109920in}{2.275000in}}%
\pgfpathlineto{\pgfqpoint{4.111160in}{2.345000in}}%
\pgfpathlineto{\pgfqpoint{4.112400in}{2.520000in}}%
\pgfpathlineto{\pgfqpoint{4.113640in}{2.065000in}}%
\pgfpathlineto{\pgfqpoint{4.114880in}{2.345000in}}%
\pgfpathlineto{\pgfqpoint{4.116120in}{2.310000in}}%
\pgfpathlineto{\pgfqpoint{4.117360in}{2.450000in}}%
\pgfpathlineto{\pgfqpoint{4.118600in}{2.450000in}}%
\pgfpathlineto{\pgfqpoint{4.119840in}{2.485000in}}%
\pgfpathlineto{\pgfqpoint{4.121080in}{2.310000in}}%
\pgfpathlineto{\pgfqpoint{4.124800in}{2.555000in}}%
\pgfpathlineto{\pgfqpoint{4.126040in}{2.205000in}}%
\pgfpathlineto{\pgfqpoint{4.128520in}{2.520000in}}%
\pgfpathlineto{\pgfqpoint{4.129760in}{2.240000in}}%
\pgfpathlineto{\pgfqpoint{4.132240in}{2.555000in}}%
\pgfpathlineto{\pgfqpoint{4.133480in}{2.415000in}}%
\pgfpathlineto{\pgfqpoint{4.134720in}{2.065000in}}%
\pgfpathlineto{\pgfqpoint{4.137200in}{2.625000in}}%
\pgfpathlineto{\pgfqpoint{4.138440in}{2.485000in}}%
\pgfpathlineto{\pgfqpoint{4.139680in}{2.800000in}}%
\pgfpathlineto{\pgfqpoint{4.140920in}{2.380000in}}%
\pgfpathlineto{\pgfqpoint{4.142160in}{2.345000in}}%
\pgfpathlineto{\pgfqpoint{4.143400in}{2.485000in}}%
\pgfpathlineto{\pgfqpoint{4.144640in}{2.485000in}}%
\pgfpathlineto{\pgfqpoint{4.145880in}{2.450000in}}%
\pgfpathlineto{\pgfqpoint{4.147120in}{2.695000in}}%
\pgfpathlineto{\pgfqpoint{4.148360in}{2.625000in}}%
\pgfpathlineto{\pgfqpoint{4.150840in}{2.765000in}}%
\pgfpathlineto{\pgfqpoint{4.152080in}{2.415000in}}%
\pgfpathlineto{\pgfqpoint{4.153320in}{2.555000in}}%
\pgfpathlineto{\pgfqpoint{4.154560in}{2.170000in}}%
\pgfpathlineto{\pgfqpoint{4.157040in}{2.415000in}}%
\pgfpathlineto{\pgfqpoint{4.158280in}{2.380000in}}%
\pgfpathlineto{\pgfqpoint{4.159520in}{1.925000in}}%
\pgfpathlineto{\pgfqpoint{4.160760in}{2.275000in}}%
\pgfpathlineto{\pgfqpoint{4.163240in}{2.275000in}}%
\pgfpathlineto{\pgfqpoint{4.164480in}{2.450000in}}%
\pgfpathlineto{\pgfqpoint{4.165720in}{2.205000in}}%
\pgfpathlineto{\pgfqpoint{4.166960in}{2.345000in}}%
\pgfpathlineto{\pgfqpoint{4.168200in}{2.310000in}}%
\pgfpathlineto{\pgfqpoint{4.170680in}{2.415000in}}%
\pgfpathlineto{\pgfqpoint{4.171920in}{2.590000in}}%
\pgfpathlineto{\pgfqpoint{4.174400in}{2.030000in}}%
\pgfpathlineto{\pgfqpoint{4.175640in}{2.170000in}}%
\pgfpathlineto{\pgfqpoint{4.176880in}{2.765000in}}%
\pgfpathlineto{\pgfqpoint{4.179360in}{2.030000in}}%
\pgfpathlineto{\pgfqpoint{4.180600in}{2.730000in}}%
\pgfpathlineto{\pgfqpoint{4.181840in}{2.205000in}}%
\pgfpathlineto{\pgfqpoint{4.183080in}{2.380000in}}%
\pgfpathlineto{\pgfqpoint{4.184320in}{2.100000in}}%
\pgfpathlineto{\pgfqpoint{4.185560in}{2.555000in}}%
\pgfpathlineto{\pgfqpoint{4.186800in}{2.485000in}}%
\pgfpathlineto{\pgfqpoint{4.188040in}{2.135000in}}%
\pgfpathlineto{\pgfqpoint{4.189280in}{2.660000in}}%
\pgfpathlineto{\pgfqpoint{4.190520in}{2.520000in}}%
\pgfpathlineto{\pgfqpoint{4.191760in}{2.835000in}}%
\pgfpathlineto{\pgfqpoint{4.193000in}{2.520000in}}%
\pgfpathlineto{\pgfqpoint{4.194240in}{2.870000in}}%
\pgfpathlineto{\pgfqpoint{4.195480in}{2.450000in}}%
\pgfpathlineto{\pgfqpoint{4.196720in}{2.660000in}}%
\pgfpathlineto{\pgfqpoint{4.197960in}{2.590000in}}%
\pgfpathlineto{\pgfqpoint{4.199200in}{2.835000in}}%
\pgfpathlineto{\pgfqpoint{4.200440in}{2.520000in}}%
\pgfpathlineto{\pgfqpoint{4.201680in}{2.555000in}}%
\pgfpathlineto{\pgfqpoint{4.202920in}{2.205000in}}%
\pgfpathlineto{\pgfqpoint{4.204160in}{2.625000in}}%
\pgfpathlineto{\pgfqpoint{4.205400in}{2.205000in}}%
\pgfpathlineto{\pgfqpoint{4.207880in}{2.800000in}}%
\pgfpathlineto{\pgfqpoint{4.209120in}{2.730000in}}%
\pgfpathlineto{\pgfqpoint{4.210360in}{2.905000in}}%
\pgfpathlineto{\pgfqpoint{4.212840in}{2.625000in}}%
\pgfpathlineto{\pgfqpoint{4.214080in}{2.660000in}}%
\pgfpathlineto{\pgfqpoint{4.217800in}{2.240000in}}%
\pgfpathlineto{\pgfqpoint{4.219040in}{2.275000in}}%
\pgfpathlineto{\pgfqpoint{4.220280in}{2.345000in}}%
\pgfpathlineto{\pgfqpoint{4.221520in}{2.345000in}}%
\pgfpathlineto{\pgfqpoint{4.222760in}{2.380000in}}%
\pgfpathlineto{\pgfqpoint{4.224000in}{1.960000in}}%
\pgfpathlineto{\pgfqpoint{4.226480in}{2.415000in}}%
\pgfpathlineto{\pgfqpoint{4.227720in}{2.275000in}}%
\pgfpathlineto{\pgfqpoint{4.232680in}{2.695000in}}%
\pgfpathlineto{\pgfqpoint{4.233920in}{2.345000in}}%
\pgfpathlineto{\pgfqpoint{4.235160in}{2.870000in}}%
\pgfpathlineto{\pgfqpoint{4.237640in}{2.415000in}}%
\pgfpathlineto{\pgfqpoint{4.238880in}{2.590000in}}%
\pgfpathlineto{\pgfqpoint{4.240120in}{1.925000in}}%
\pgfpathlineto{\pgfqpoint{4.242600in}{2.520000in}}%
\pgfpathlineto{\pgfqpoint{4.243840in}{2.520000in}}%
\pgfpathlineto{\pgfqpoint{4.245080in}{2.135000in}}%
\pgfpathlineto{\pgfqpoint{4.248800in}{2.940000in}}%
\pgfpathlineto{\pgfqpoint{4.250040in}{2.835000in}}%
\pgfpathlineto{\pgfqpoint{4.251280in}{2.835000in}}%
\pgfpathlineto{\pgfqpoint{4.252520in}{2.345000in}}%
\pgfpathlineto{\pgfqpoint{4.255000in}{2.765000in}}%
\pgfpathlineto{\pgfqpoint{4.257480in}{2.065000in}}%
\pgfpathlineto{\pgfqpoint{4.258720in}{2.485000in}}%
\pgfpathlineto{\pgfqpoint{4.259960in}{1.995000in}}%
\pgfpathlineto{\pgfqpoint{4.262440in}{2.625000in}}%
\pgfpathlineto{\pgfqpoint{4.263680in}{2.695000in}}%
\pgfpathlineto{\pgfqpoint{4.264920in}{2.695000in}}%
\pgfpathlineto{\pgfqpoint{4.267400in}{2.415000in}}%
\pgfpathlineto{\pgfqpoint{4.268640in}{2.555000in}}%
\pgfpathlineto{\pgfqpoint{4.269880in}{2.485000in}}%
\pgfpathlineto{\pgfqpoint{4.271120in}{2.170000in}}%
\pgfpathlineto{\pgfqpoint{4.272360in}{2.520000in}}%
\pgfpathlineto{\pgfqpoint{4.273600in}{2.485000in}}%
\pgfpathlineto{\pgfqpoint{4.276080in}{1.750000in}}%
\pgfpathlineto{\pgfqpoint{4.278560in}{2.695000in}}%
\pgfpathlineto{\pgfqpoint{4.281040in}{2.380000in}}%
\pgfpathlineto{\pgfqpoint{4.282280in}{2.450000in}}%
\pgfpathlineto{\pgfqpoint{4.283520in}{2.590000in}}%
\pgfpathlineto{\pgfqpoint{4.284760in}{2.380000in}}%
\pgfpathlineto{\pgfqpoint{4.286000in}{2.450000in}}%
\pgfpathlineto{\pgfqpoint{4.287240in}{2.380000in}}%
\pgfpathlineto{\pgfqpoint{4.288480in}{2.555000in}}%
\pgfpathlineto{\pgfqpoint{4.289720in}{2.205000in}}%
\pgfpathlineto{\pgfqpoint{4.292200in}{2.730000in}}%
\pgfpathlineto{\pgfqpoint{4.293440in}{2.415000in}}%
\pgfpathlineto{\pgfqpoint{4.294680in}{2.555000in}}%
\pgfpathlineto{\pgfqpoint{4.295920in}{2.240000in}}%
\pgfpathlineto{\pgfqpoint{4.297160in}{2.345000in}}%
\pgfpathlineto{\pgfqpoint{4.298400in}{2.590000in}}%
\pgfpathlineto{\pgfqpoint{4.300880in}{2.415000in}}%
\pgfpathlineto{\pgfqpoint{4.303360in}{2.695000in}}%
\pgfpathlineto{\pgfqpoint{4.304600in}{2.765000in}}%
\pgfpathlineto{\pgfqpoint{4.305840in}{2.380000in}}%
\pgfpathlineto{\pgfqpoint{4.307080in}{2.345000in}}%
\pgfpathlineto{\pgfqpoint{4.308320in}{2.205000in}}%
\pgfpathlineto{\pgfqpoint{4.309560in}{2.520000in}}%
\pgfpathlineto{\pgfqpoint{4.314520in}{1.715000in}}%
\pgfpathlineto{\pgfqpoint{4.317000in}{2.450000in}}%
\pgfpathlineto{\pgfqpoint{4.318240in}{2.625000in}}%
\pgfpathlineto{\pgfqpoint{4.321960in}{1.925000in}}%
\pgfpathlineto{\pgfqpoint{4.324440in}{2.240000in}}%
\pgfpathlineto{\pgfqpoint{4.325680in}{2.380000in}}%
\pgfpathlineto{\pgfqpoint{4.326920in}{2.380000in}}%
\pgfpathlineto{\pgfqpoint{4.328160in}{2.590000in}}%
\pgfpathlineto{\pgfqpoint{4.329400in}{2.415000in}}%
\pgfpathlineto{\pgfqpoint{4.330640in}{2.065000in}}%
\pgfpathlineto{\pgfqpoint{4.333120in}{2.625000in}}%
\pgfpathlineto{\pgfqpoint{4.334360in}{2.415000in}}%
\pgfpathlineto{\pgfqpoint{4.335600in}{2.450000in}}%
\pgfpathlineto{\pgfqpoint{4.338080in}{2.660000in}}%
\pgfpathlineto{\pgfqpoint{4.339320in}{2.625000in}}%
\pgfpathlineto{\pgfqpoint{4.340560in}{2.520000in}}%
\pgfpathlineto{\pgfqpoint{4.341800in}{2.695000in}}%
\pgfpathlineto{\pgfqpoint{4.343040in}{2.205000in}}%
\pgfpathlineto{\pgfqpoint{4.344280in}{2.555000in}}%
\pgfpathlineto{\pgfqpoint{4.345520in}{2.590000in}}%
\pgfpathlineto{\pgfqpoint{4.346760in}{2.730000in}}%
\pgfpathlineto{\pgfqpoint{4.348000in}{2.695000in}}%
\pgfpathlineto{\pgfqpoint{4.349240in}{2.590000in}}%
\pgfpathlineto{\pgfqpoint{4.350480in}{2.730000in}}%
\pgfpathlineto{\pgfqpoint{4.354200in}{2.240000in}}%
\pgfpathlineto{\pgfqpoint{4.355440in}{2.520000in}}%
\pgfpathlineto{\pgfqpoint{4.357920in}{2.415000in}}%
\pgfpathlineto{\pgfqpoint{4.359160in}{2.835000in}}%
\pgfpathlineto{\pgfqpoint{4.360400in}{2.555000in}}%
\pgfpathlineto{\pgfqpoint{4.361640in}{2.765000in}}%
\pgfpathlineto{\pgfqpoint{4.362880in}{2.415000in}}%
\pgfpathlineto{\pgfqpoint{4.364120in}{2.625000in}}%
\pgfpathlineto{\pgfqpoint{4.365360in}{2.100000in}}%
\pgfpathlineto{\pgfqpoint{4.366600in}{2.625000in}}%
\pgfpathlineto{\pgfqpoint{4.367840in}{2.590000in}}%
\pgfpathlineto{\pgfqpoint{4.370320in}{2.135000in}}%
\pgfpathlineto{\pgfqpoint{4.371560in}{2.065000in}}%
\pgfpathlineto{\pgfqpoint{4.374040in}{2.170000in}}%
\pgfpathlineto{\pgfqpoint{4.375280in}{2.380000in}}%
\pgfpathlineto{\pgfqpoint{4.376520in}{2.380000in}}%
\pgfpathlineto{\pgfqpoint{4.377760in}{2.275000in}}%
\pgfpathlineto{\pgfqpoint{4.379000in}{1.995000in}}%
\pgfpathlineto{\pgfqpoint{4.381480in}{2.520000in}}%
\pgfpathlineto{\pgfqpoint{4.382720in}{1.890000in}}%
\pgfpathlineto{\pgfqpoint{4.383960in}{2.695000in}}%
\pgfpathlineto{\pgfqpoint{4.386440in}{2.205000in}}%
\pgfpathlineto{\pgfqpoint{4.388920in}{2.100000in}}%
\pgfpathlineto{\pgfqpoint{4.390160in}{2.170000in}}%
\pgfpathlineto{\pgfqpoint{4.391400in}{1.925000in}}%
\pgfpathlineto{\pgfqpoint{4.392640in}{2.380000in}}%
\pgfpathlineto{\pgfqpoint{4.393880in}{1.855000in}}%
\pgfpathlineto{\pgfqpoint{4.396360in}{2.065000in}}%
\pgfpathlineto{\pgfqpoint{4.397600in}{2.065000in}}%
\pgfpathlineto{\pgfqpoint{4.400080in}{1.855000in}}%
\pgfpathlineto{\pgfqpoint{4.401320in}{2.485000in}}%
\pgfpathlineto{\pgfqpoint{4.402560in}{2.205000in}}%
\pgfpathlineto{\pgfqpoint{4.403800in}{2.310000in}}%
\pgfpathlineto{\pgfqpoint{4.406280in}{2.170000in}}%
\pgfpathlineto{\pgfqpoint{4.407520in}{2.520000in}}%
\pgfpathlineto{\pgfqpoint{4.408760in}{2.170000in}}%
\pgfpathlineto{\pgfqpoint{4.411240in}{2.450000in}}%
\pgfpathlineto{\pgfqpoint{4.412480in}{2.170000in}}%
\pgfpathlineto{\pgfqpoint{4.414960in}{2.765000in}}%
\pgfpathlineto{\pgfqpoint{4.418680in}{2.275000in}}%
\pgfpathlineto{\pgfqpoint{4.419920in}{2.310000in}}%
\pgfpathlineto{\pgfqpoint{4.421160in}{2.275000in}}%
\pgfpathlineto{\pgfqpoint{4.422400in}{2.275000in}}%
\pgfpathlineto{\pgfqpoint{4.423640in}{2.555000in}}%
\pgfpathlineto{\pgfqpoint{4.424880in}{2.170000in}}%
\pgfpathlineto{\pgfqpoint{4.426120in}{2.240000in}}%
\pgfpathlineto{\pgfqpoint{4.427360in}{2.555000in}}%
\pgfpathlineto{\pgfqpoint{4.428600in}{2.485000in}}%
\pgfpathlineto{\pgfqpoint{4.429840in}{2.625000in}}%
\pgfpathlineto{\pgfqpoint{4.431080in}{2.555000in}}%
\pgfpathlineto{\pgfqpoint{4.432320in}{2.275000in}}%
\pgfpathlineto{\pgfqpoint{4.436040in}{2.660000in}}%
\pgfpathlineto{\pgfqpoint{4.437280in}{2.345000in}}%
\pgfpathlineto{\pgfqpoint{4.438520in}{2.625000in}}%
\pgfpathlineto{\pgfqpoint{4.439760in}{2.590000in}}%
\pgfpathlineto{\pgfqpoint{4.441000in}{2.905000in}}%
\pgfpathlineto{\pgfqpoint{4.442240in}{2.450000in}}%
\pgfpathlineto{\pgfqpoint{4.444720in}{2.730000in}}%
\pgfpathlineto{\pgfqpoint{4.445960in}{2.520000in}}%
\pgfpathlineto{\pgfqpoint{4.447200in}{2.555000in}}%
\pgfpathlineto{\pgfqpoint{4.448440in}{2.520000in}}%
\pgfpathlineto{\pgfqpoint{4.449680in}{2.450000in}}%
\pgfpathlineto{\pgfqpoint{4.450920in}{2.485000in}}%
\pgfpathlineto{\pgfqpoint{4.452160in}{2.765000in}}%
\pgfpathlineto{\pgfqpoint{4.453400in}{2.730000in}}%
\pgfpathlineto{\pgfqpoint{4.454640in}{2.485000in}}%
\pgfpathlineto{\pgfqpoint{4.455880in}{2.485000in}}%
\pgfpathlineto{\pgfqpoint{4.457120in}{2.135000in}}%
\pgfpathlineto{\pgfqpoint{4.458360in}{2.660000in}}%
\pgfpathlineto{\pgfqpoint{4.459600in}{2.590000in}}%
\pgfpathlineto{\pgfqpoint{4.462080in}{2.590000in}}%
\pgfpathlineto{\pgfqpoint{4.464560in}{2.345000in}}%
\pgfpathlineto{\pgfqpoint{4.465800in}{2.730000in}}%
\pgfpathlineto{\pgfqpoint{4.467040in}{2.170000in}}%
\pgfpathlineto{\pgfqpoint{4.469520in}{2.660000in}}%
\pgfpathlineto{\pgfqpoint{4.470760in}{2.135000in}}%
\pgfpathlineto{\pgfqpoint{4.473240in}{2.730000in}}%
\pgfpathlineto{\pgfqpoint{4.474480in}{2.380000in}}%
\pgfpathlineto{\pgfqpoint{4.475720in}{2.345000in}}%
\pgfpathlineto{\pgfqpoint{4.476960in}{2.590000in}}%
\pgfpathlineto{\pgfqpoint{4.478200in}{2.170000in}}%
\pgfpathlineto{\pgfqpoint{4.481920in}{2.520000in}}%
\pgfpathlineto{\pgfqpoint{4.483160in}{2.275000in}}%
\pgfpathlineto{\pgfqpoint{4.485640in}{2.450000in}}%
\pgfpathlineto{\pgfqpoint{4.486880in}{2.275000in}}%
\pgfpathlineto{\pgfqpoint{4.488120in}{2.765000in}}%
\pgfpathlineto{\pgfqpoint{4.489360in}{2.695000in}}%
\pgfpathlineto{\pgfqpoint{4.490600in}{2.800000in}}%
\pgfpathlineto{\pgfqpoint{4.491840in}{2.555000in}}%
\pgfpathlineto{\pgfqpoint{4.493080in}{2.695000in}}%
\pgfpathlineto{\pgfqpoint{4.494320in}{2.520000in}}%
\pgfpathlineto{\pgfqpoint{4.495560in}{2.030000in}}%
\pgfpathlineto{\pgfqpoint{4.498040in}{2.170000in}}%
\pgfpathlineto{\pgfqpoint{4.499280in}{2.170000in}}%
\pgfpathlineto{\pgfqpoint{4.501760in}{2.380000in}}%
\pgfpathlineto{\pgfqpoint{4.503000in}{2.205000in}}%
\pgfpathlineto{\pgfqpoint{4.504240in}{2.450000in}}%
\pgfpathlineto{\pgfqpoint{4.505480in}{2.380000in}}%
\pgfpathlineto{\pgfqpoint{4.506720in}{2.100000in}}%
\pgfpathlineto{\pgfqpoint{4.507960in}{2.135000in}}%
\pgfpathlineto{\pgfqpoint{4.509200in}{2.730000in}}%
\pgfpathlineto{\pgfqpoint{4.511680in}{2.345000in}}%
\pgfpathlineto{\pgfqpoint{4.512920in}{2.555000in}}%
\pgfpathlineto{\pgfqpoint{4.514160in}{2.520000in}}%
\pgfpathlineto{\pgfqpoint{4.515400in}{2.450000in}}%
\pgfpathlineto{\pgfqpoint{4.516640in}{2.590000in}}%
\pgfpathlineto{\pgfqpoint{4.517880in}{2.415000in}}%
\pgfpathlineto{\pgfqpoint{4.519120in}{2.625000in}}%
\pgfpathlineto{\pgfqpoint{4.520360in}{2.100000in}}%
\pgfpathlineto{\pgfqpoint{4.524080in}{2.485000in}}%
\pgfpathlineto{\pgfqpoint{4.525320in}{2.345000in}}%
\pgfpathlineto{\pgfqpoint{4.526560in}{2.625000in}}%
\pgfpathlineto{\pgfqpoint{4.529040in}{2.240000in}}%
\pgfpathlineto{\pgfqpoint{4.530280in}{2.625000in}}%
\pgfpathlineto{\pgfqpoint{4.531520in}{2.625000in}}%
\pgfpathlineto{\pgfqpoint{4.534000in}{2.310000in}}%
\pgfpathlineto{\pgfqpoint{4.535240in}{2.345000in}}%
\pgfpathlineto{\pgfqpoint{4.536480in}{2.275000in}}%
\pgfpathlineto{\pgfqpoint{4.537720in}{2.345000in}}%
\pgfpathlineto{\pgfqpoint{4.538960in}{2.205000in}}%
\pgfpathlineto{\pgfqpoint{4.542680in}{2.555000in}}%
\pgfpathlineto{\pgfqpoint{4.543920in}{1.995000in}}%
\pgfpathlineto{\pgfqpoint{4.545160in}{2.520000in}}%
\pgfpathlineto{\pgfqpoint{4.546400in}{2.310000in}}%
\pgfpathlineto{\pgfqpoint{4.547640in}{2.695000in}}%
\pgfpathlineto{\pgfqpoint{4.550120in}{2.380000in}}%
\pgfpathlineto{\pgfqpoint{4.551360in}{2.485000in}}%
\pgfpathlineto{\pgfqpoint{4.552600in}{2.240000in}}%
\pgfpathlineto{\pgfqpoint{4.553840in}{2.765000in}}%
\pgfpathlineto{\pgfqpoint{4.555080in}{2.485000in}}%
\pgfpathlineto{\pgfqpoint{4.556320in}{2.485000in}}%
\pgfpathlineto{\pgfqpoint{4.557560in}{2.275000in}}%
\pgfpathlineto{\pgfqpoint{4.558800in}{2.940000in}}%
\pgfpathlineto{\pgfqpoint{4.561280in}{2.450000in}}%
\pgfpathlineto{\pgfqpoint{4.562520in}{2.520000in}}%
\pgfpathlineto{\pgfqpoint{4.565000in}{2.100000in}}%
\pgfpathlineto{\pgfqpoint{4.566240in}{2.170000in}}%
\pgfpathlineto{\pgfqpoint{4.567480in}{2.450000in}}%
\pgfpathlineto{\pgfqpoint{4.568720in}{2.415000in}}%
\pgfpathlineto{\pgfqpoint{4.569960in}{1.960000in}}%
\pgfpathlineto{\pgfqpoint{4.571200in}{2.275000in}}%
\pgfpathlineto{\pgfqpoint{4.572440in}{2.205000in}}%
\pgfpathlineto{\pgfqpoint{4.573680in}{1.925000in}}%
\pgfpathlineto{\pgfqpoint{4.574920in}{2.240000in}}%
\pgfpathlineto{\pgfqpoint{4.577400in}{1.960000in}}%
\pgfpathlineto{\pgfqpoint{4.579880in}{2.380000in}}%
\pgfpathlineto{\pgfqpoint{4.581120in}{2.170000in}}%
\pgfpathlineto{\pgfqpoint{4.582360in}{2.520000in}}%
\pgfpathlineto{\pgfqpoint{4.584840in}{2.030000in}}%
\pgfpathlineto{\pgfqpoint{4.586080in}{2.065000in}}%
\pgfpathlineto{\pgfqpoint{4.587320in}{2.065000in}}%
\pgfpathlineto{\pgfqpoint{4.588560in}{1.925000in}}%
\pgfpathlineto{\pgfqpoint{4.589800in}{2.310000in}}%
\pgfpathlineto{\pgfqpoint{4.591040in}{2.240000in}}%
\pgfpathlineto{\pgfqpoint{4.592280in}{1.995000in}}%
\pgfpathlineto{\pgfqpoint{4.597240in}{2.485000in}}%
\pgfpathlineto{\pgfqpoint{4.599720in}{2.135000in}}%
\pgfpathlineto{\pgfqpoint{4.600960in}{2.415000in}}%
\pgfpathlineto{\pgfqpoint{4.602200in}{2.205000in}}%
\pgfpathlineto{\pgfqpoint{4.603440in}{2.590000in}}%
\pgfpathlineto{\pgfqpoint{4.605920in}{2.450000in}}%
\pgfpathlineto{\pgfqpoint{4.607160in}{2.485000in}}%
\pgfpathlineto{\pgfqpoint{4.608400in}{2.415000in}}%
\pgfpathlineto{\pgfqpoint{4.609640in}{2.555000in}}%
\pgfpathlineto{\pgfqpoint{4.610880in}{2.170000in}}%
\pgfpathlineto{\pgfqpoint{4.612120in}{2.310000in}}%
\pgfpathlineto{\pgfqpoint{4.613360in}{2.135000in}}%
\pgfpathlineto{\pgfqpoint{4.614600in}{2.380000in}}%
\pgfpathlineto{\pgfqpoint{4.615840in}{2.275000in}}%
\pgfpathlineto{\pgfqpoint{4.617080in}{2.065000in}}%
\pgfpathlineto{\pgfqpoint{4.618320in}{2.415000in}}%
\pgfpathlineto{\pgfqpoint{4.620800in}{1.855000in}}%
\pgfpathlineto{\pgfqpoint{4.622040in}{2.275000in}}%
\pgfpathlineto{\pgfqpoint{4.623280in}{2.100000in}}%
\pgfpathlineto{\pgfqpoint{4.624520in}{2.170000in}}%
\pgfpathlineto{\pgfqpoint{4.625760in}{2.555000in}}%
\pgfpathlineto{\pgfqpoint{4.627000in}{2.485000in}}%
\pgfpathlineto{\pgfqpoint{4.629480in}{2.205000in}}%
\pgfpathlineto{\pgfqpoint{4.630720in}{2.520000in}}%
\pgfpathlineto{\pgfqpoint{4.631960in}{2.415000in}}%
\pgfpathlineto{\pgfqpoint{4.634440in}{2.555000in}}%
\pgfpathlineto{\pgfqpoint{4.635680in}{2.555000in}}%
\pgfpathlineto{\pgfqpoint{4.636920in}{2.205000in}}%
\pgfpathlineto{\pgfqpoint{4.638160in}{2.485000in}}%
\pgfpathlineto{\pgfqpoint{4.639400in}{2.275000in}}%
\pgfpathlineto{\pgfqpoint{4.640640in}{2.345000in}}%
\pgfpathlineto{\pgfqpoint{4.641880in}{2.345000in}}%
\pgfpathlineto{\pgfqpoint{4.644360in}{2.485000in}}%
\pgfpathlineto{\pgfqpoint{4.645600in}{2.625000in}}%
\pgfpathlineto{\pgfqpoint{4.648080in}{2.030000in}}%
\pgfpathlineto{\pgfqpoint{4.650560in}{2.625000in}}%
\pgfpathlineto{\pgfqpoint{4.651800in}{2.520000in}}%
\pgfpathlineto{\pgfqpoint{4.653040in}{2.765000in}}%
\pgfpathlineto{\pgfqpoint{4.655520in}{2.065000in}}%
\pgfpathlineto{\pgfqpoint{4.656760in}{2.380000in}}%
\pgfpathlineto{\pgfqpoint{4.659240in}{2.030000in}}%
\pgfpathlineto{\pgfqpoint{4.660480in}{2.485000in}}%
\pgfpathlineto{\pgfqpoint{4.661720in}{2.205000in}}%
\pgfpathlineto{\pgfqpoint{4.665440in}{2.870000in}}%
\pgfpathlineto{\pgfqpoint{4.666680in}{2.275000in}}%
\pgfpathlineto{\pgfqpoint{4.667920in}{2.590000in}}%
\pgfpathlineto{\pgfqpoint{4.669160in}{2.450000in}}%
\pgfpathlineto{\pgfqpoint{4.670400in}{2.695000in}}%
\pgfpathlineto{\pgfqpoint{4.671640in}{2.240000in}}%
\pgfpathlineto{\pgfqpoint{4.674120in}{2.170000in}}%
\pgfpathlineto{\pgfqpoint{4.676600in}{2.450000in}}%
\pgfpathlineto{\pgfqpoint{4.677840in}{2.065000in}}%
\pgfpathlineto{\pgfqpoint{4.680320in}{2.485000in}}%
\pgfpathlineto{\pgfqpoint{4.681560in}{2.625000in}}%
\pgfpathlineto{\pgfqpoint{4.682800in}{2.625000in}}%
\pgfpathlineto{\pgfqpoint{4.684040in}{2.275000in}}%
\pgfpathlineto{\pgfqpoint{4.685280in}{2.555000in}}%
\pgfpathlineto{\pgfqpoint{4.686520in}{2.450000in}}%
\pgfpathlineto{\pgfqpoint{4.687760in}{1.960000in}}%
\pgfpathlineto{\pgfqpoint{4.689000in}{2.065000in}}%
\pgfpathlineto{\pgfqpoint{4.690240in}{2.450000in}}%
\pgfpathlineto{\pgfqpoint{4.691480in}{2.310000in}}%
\pgfpathlineto{\pgfqpoint{4.695200in}{2.625000in}}%
\pgfpathlineto{\pgfqpoint{4.696440in}{2.275000in}}%
\pgfpathlineto{\pgfqpoint{4.697680in}{2.625000in}}%
\pgfpathlineto{\pgfqpoint{4.698920in}{2.240000in}}%
\pgfpathlineto{\pgfqpoint{4.701400in}{2.730000in}}%
\pgfpathlineto{\pgfqpoint{4.703880in}{2.135000in}}%
\pgfpathlineto{\pgfqpoint{4.705120in}{2.555000in}}%
\pgfpathlineto{\pgfqpoint{4.707600in}{2.380000in}}%
\pgfpathlineto{\pgfqpoint{4.708840in}{2.485000in}}%
\pgfpathlineto{\pgfqpoint{4.710080in}{2.310000in}}%
\pgfpathlineto{\pgfqpoint{4.711320in}{2.310000in}}%
\pgfpathlineto{\pgfqpoint{4.712560in}{2.590000in}}%
\pgfpathlineto{\pgfqpoint{4.713800in}{2.240000in}}%
\pgfpathlineto{\pgfqpoint{4.716280in}{2.590000in}}%
\pgfpathlineto{\pgfqpoint{4.717520in}{2.275000in}}%
\pgfpathlineto{\pgfqpoint{4.720000in}{2.275000in}}%
\pgfpathlineto{\pgfqpoint{4.721240in}{2.695000in}}%
\pgfpathlineto{\pgfqpoint{4.722480in}{2.240000in}}%
\pgfpathlineto{\pgfqpoint{4.723720in}{2.205000in}}%
\pgfpathlineto{\pgfqpoint{4.724960in}{2.485000in}}%
\pgfpathlineto{\pgfqpoint{4.726200in}{2.485000in}}%
\pgfpathlineto{\pgfqpoint{4.727440in}{2.065000in}}%
\pgfpathlineto{\pgfqpoint{4.728680in}{2.730000in}}%
\pgfpathlineto{\pgfqpoint{4.731160in}{2.275000in}}%
\pgfpathlineto{\pgfqpoint{4.732400in}{2.275000in}}%
\pgfpathlineto{\pgfqpoint{4.733640in}{2.415000in}}%
\pgfpathlineto{\pgfqpoint{4.734880in}{2.170000in}}%
\pgfpathlineto{\pgfqpoint{4.736120in}{2.240000in}}%
\pgfpathlineto{\pgfqpoint{4.737360in}{2.100000in}}%
\pgfpathlineto{\pgfqpoint{4.739840in}{2.730000in}}%
\pgfpathlineto{\pgfqpoint{4.741080in}{2.100000in}}%
\pgfpathlineto{\pgfqpoint{4.742320in}{2.730000in}}%
\pgfpathlineto{\pgfqpoint{4.743560in}{2.625000in}}%
\pgfpathlineto{\pgfqpoint{4.746040in}{2.100000in}}%
\pgfpathlineto{\pgfqpoint{4.747280in}{2.415000in}}%
\pgfpathlineto{\pgfqpoint{4.748520in}{2.275000in}}%
\pgfpathlineto{\pgfqpoint{4.751000in}{2.660000in}}%
\pgfpathlineto{\pgfqpoint{4.754720in}{2.030000in}}%
\pgfpathlineto{\pgfqpoint{4.755960in}{2.240000in}}%
\pgfpathlineto{\pgfqpoint{4.757200in}{2.695000in}}%
\pgfpathlineto{\pgfqpoint{4.760920in}{2.100000in}}%
\pgfpathlineto{\pgfqpoint{4.763400in}{2.450000in}}%
\pgfpathlineto{\pgfqpoint{4.764640in}{2.415000in}}%
\pgfpathlineto{\pgfqpoint{4.765880in}{2.520000in}}%
\pgfpathlineto{\pgfqpoint{4.767120in}{2.100000in}}%
\pgfpathlineto{\pgfqpoint{4.769600in}{2.485000in}}%
\pgfpathlineto{\pgfqpoint{4.770840in}{2.135000in}}%
\pgfpathlineto{\pgfqpoint{4.775800in}{2.730000in}}%
\pgfpathlineto{\pgfqpoint{4.777040in}{2.205000in}}%
\pgfpathlineto{\pgfqpoint{4.778280in}{2.625000in}}%
\pgfpathlineto{\pgfqpoint{4.782000in}{2.415000in}}%
\pgfpathlineto{\pgfqpoint{4.783240in}{2.450000in}}%
\pgfpathlineto{\pgfqpoint{4.784480in}{2.730000in}}%
\pgfpathlineto{\pgfqpoint{4.785720in}{2.695000in}}%
\pgfpathlineto{\pgfqpoint{4.788200in}{2.485000in}}%
\pgfpathlineto{\pgfqpoint{4.789440in}{2.520000in}}%
\pgfpathlineto{\pgfqpoint{4.790680in}{2.485000in}}%
\pgfpathlineto{\pgfqpoint{4.791920in}{2.485000in}}%
\pgfpathlineto{\pgfqpoint{4.794400in}{2.205000in}}%
\pgfpathlineto{\pgfqpoint{4.796880in}{2.800000in}}%
\pgfpathlineto{\pgfqpoint{4.798120in}{2.415000in}}%
\pgfpathlineto{\pgfqpoint{4.799360in}{2.555000in}}%
\pgfpathlineto{\pgfqpoint{4.800600in}{2.275000in}}%
\pgfpathlineto{\pgfqpoint{4.803080in}{2.555000in}}%
\pgfpathlineto{\pgfqpoint{4.805560in}{1.995000in}}%
\pgfpathlineto{\pgfqpoint{4.806800in}{2.205000in}}%
\pgfpathlineto{\pgfqpoint{4.808040in}{2.100000in}}%
\pgfpathlineto{\pgfqpoint{4.809280in}{2.695000in}}%
\pgfpathlineto{\pgfqpoint{4.811760in}{2.415000in}}%
\pgfpathlineto{\pgfqpoint{4.814240in}{2.660000in}}%
\pgfpathlineto{\pgfqpoint{4.815480in}{2.275000in}}%
\pgfpathlineto{\pgfqpoint{4.816720in}{2.800000in}}%
\pgfpathlineto{\pgfqpoint{4.819200in}{2.380000in}}%
\pgfpathlineto{\pgfqpoint{4.820440in}{2.765000in}}%
\pgfpathlineto{\pgfqpoint{4.822920in}{2.415000in}}%
\pgfpathlineto{\pgfqpoint{4.824160in}{2.415000in}}%
\pgfpathlineto{\pgfqpoint{4.825400in}{1.925000in}}%
\pgfpathlineto{\pgfqpoint{4.827880in}{2.555000in}}%
\pgfpathlineto{\pgfqpoint{4.829120in}{2.345000in}}%
\pgfpathlineto{\pgfqpoint{4.830360in}{2.380000in}}%
\pgfpathlineto{\pgfqpoint{4.831600in}{2.450000in}}%
\pgfpathlineto{\pgfqpoint{4.832840in}{2.450000in}}%
\pgfpathlineto{\pgfqpoint{4.834080in}{2.625000in}}%
\pgfpathlineto{\pgfqpoint{4.836560in}{2.170000in}}%
\pgfpathlineto{\pgfqpoint{4.837800in}{2.380000in}}%
\pgfpathlineto{\pgfqpoint{4.839040in}{2.345000in}}%
\pgfpathlineto{\pgfqpoint{4.840280in}{2.380000in}}%
\pgfpathlineto{\pgfqpoint{4.841520in}{2.450000in}}%
\pgfpathlineto{\pgfqpoint{4.844000in}{2.905000in}}%
\pgfpathlineto{\pgfqpoint{4.846480in}{2.450000in}}%
\pgfpathlineto{\pgfqpoint{4.847720in}{2.415000in}}%
\pgfpathlineto{\pgfqpoint{4.848960in}{2.450000in}}%
\pgfpathlineto{\pgfqpoint{4.850200in}{2.765000in}}%
\pgfpathlineto{\pgfqpoint{4.852680in}{2.520000in}}%
\pgfpathlineto{\pgfqpoint{4.853920in}{2.800000in}}%
\pgfpathlineto{\pgfqpoint{4.855160in}{2.800000in}}%
\pgfpathlineto{\pgfqpoint{4.857640in}{2.590000in}}%
\pgfpathlineto{\pgfqpoint{4.858880in}{2.695000in}}%
\pgfpathlineto{\pgfqpoint{4.860120in}{2.660000in}}%
\pgfpathlineto{\pgfqpoint{4.861360in}{2.310000in}}%
\pgfpathlineto{\pgfqpoint{4.862600in}{2.870000in}}%
\pgfpathlineto{\pgfqpoint{4.865080in}{2.415000in}}%
\pgfpathlineto{\pgfqpoint{4.866320in}{2.555000in}}%
\pgfpathlineto{\pgfqpoint{4.867560in}{2.870000in}}%
\pgfpathlineto{\pgfqpoint{4.868800in}{2.170000in}}%
\pgfpathlineto{\pgfqpoint{4.870040in}{2.100000in}}%
\pgfpathlineto{\pgfqpoint{4.872520in}{2.765000in}}%
\pgfpathlineto{\pgfqpoint{4.873760in}{2.310000in}}%
\pgfpathlineto{\pgfqpoint{4.875000in}{2.555000in}}%
\pgfpathlineto{\pgfqpoint{4.876240in}{2.485000in}}%
\pgfpathlineto{\pgfqpoint{4.878720in}{2.590000in}}%
\pgfpathlineto{\pgfqpoint{4.879960in}{2.590000in}}%
\pgfpathlineto{\pgfqpoint{4.881200in}{2.625000in}}%
\pgfpathlineto{\pgfqpoint{4.882440in}{2.730000in}}%
\pgfpathlineto{\pgfqpoint{4.883680in}{2.310000in}}%
\pgfpathlineto{\pgfqpoint{4.884920in}{2.415000in}}%
\pgfpathlineto{\pgfqpoint{4.886160in}{2.100000in}}%
\pgfpathlineto{\pgfqpoint{4.887400in}{2.380000in}}%
\pgfpathlineto{\pgfqpoint{4.888640in}{2.345000in}}%
\pgfpathlineto{\pgfqpoint{4.891120in}{2.205000in}}%
\pgfpathlineto{\pgfqpoint{4.892360in}{2.030000in}}%
\pgfpathlineto{\pgfqpoint{4.894840in}{2.380000in}}%
\pgfpathlineto{\pgfqpoint{4.897320in}{2.695000in}}%
\pgfpathlineto{\pgfqpoint{4.899800in}{2.345000in}}%
\pgfpathlineto{\pgfqpoint{4.901040in}{2.555000in}}%
\pgfpathlineto{\pgfqpoint{4.902280in}{2.275000in}}%
\pgfpathlineto{\pgfqpoint{4.903520in}{2.555000in}}%
\pgfpathlineto{\pgfqpoint{4.904760in}{2.520000in}}%
\pgfpathlineto{\pgfqpoint{4.907240in}{2.065000in}}%
\pgfpathlineto{\pgfqpoint{4.910960in}{2.415000in}}%
\pgfpathlineto{\pgfqpoint{4.912200in}{2.345000in}}%
\pgfpathlineto{\pgfqpoint{4.913440in}{2.345000in}}%
\pgfpathlineto{\pgfqpoint{4.914680in}{2.590000in}}%
\pgfpathlineto{\pgfqpoint{4.915920in}{2.100000in}}%
\pgfpathlineto{\pgfqpoint{4.917160in}{2.170000in}}%
\pgfpathlineto{\pgfqpoint{4.918400in}{2.100000in}}%
\pgfpathlineto{\pgfqpoint{4.919640in}{2.135000in}}%
\pgfpathlineto{\pgfqpoint{4.920880in}{2.240000in}}%
\pgfpathlineto{\pgfqpoint{4.922120in}{2.065000in}}%
\pgfpathlineto{\pgfqpoint{4.925840in}{2.555000in}}%
\pgfpathlineto{\pgfqpoint{4.927080in}{2.170000in}}%
\pgfpathlineto{\pgfqpoint{4.928320in}{2.240000in}}%
\pgfpathlineto{\pgfqpoint{4.929560in}{2.240000in}}%
\pgfpathlineto{\pgfqpoint{4.930800in}{2.625000in}}%
\pgfpathlineto{\pgfqpoint{4.932040in}{2.450000in}}%
\pgfpathlineto{\pgfqpoint{4.933280in}{2.450000in}}%
\pgfpathlineto{\pgfqpoint{4.934520in}{2.520000in}}%
\pgfpathlineto{\pgfqpoint{4.935760in}{2.415000in}}%
\pgfpathlineto{\pgfqpoint{4.937000in}{2.975000in}}%
\pgfpathlineto{\pgfqpoint{4.938240in}{2.625000in}}%
\pgfpathlineto{\pgfqpoint{4.939480in}{2.835000in}}%
\pgfpathlineto{\pgfqpoint{4.941960in}{2.555000in}}%
\pgfpathlineto{\pgfqpoint{4.943200in}{2.590000in}}%
\pgfpathlineto{\pgfqpoint{4.944440in}{2.660000in}}%
\pgfpathlineto{\pgfqpoint{4.946920in}{2.275000in}}%
\pgfpathlineto{\pgfqpoint{4.948160in}{2.520000in}}%
\pgfpathlineto{\pgfqpoint{4.949400in}{2.485000in}}%
\pgfpathlineto{\pgfqpoint{4.950640in}{2.100000in}}%
\pgfpathlineto{\pgfqpoint{4.951880in}{2.590000in}}%
\pgfpathlineto{\pgfqpoint{4.953120in}{2.240000in}}%
\pgfpathlineto{\pgfqpoint{4.954360in}{2.730000in}}%
\pgfpathlineto{\pgfqpoint{4.955600in}{2.380000in}}%
\pgfpathlineto{\pgfqpoint{4.956840in}{2.905000in}}%
\pgfpathlineto{\pgfqpoint{4.958080in}{2.660000in}}%
\pgfpathlineto{\pgfqpoint{4.959320in}{2.765000in}}%
\pgfpathlineto{\pgfqpoint{4.960560in}{2.275000in}}%
\pgfpathlineto{\pgfqpoint{4.961800in}{2.310000in}}%
\pgfpathlineto{\pgfqpoint{4.963040in}{2.170000in}}%
\pgfpathlineto{\pgfqpoint{4.965520in}{2.590000in}}%
\pgfpathlineto{\pgfqpoint{4.968000in}{2.205000in}}%
\pgfpathlineto{\pgfqpoint{4.969240in}{2.205000in}}%
\pgfpathlineto{\pgfqpoint{4.971720in}{2.450000in}}%
\pgfpathlineto{\pgfqpoint{4.972960in}{2.415000in}}%
\pgfpathlineto{\pgfqpoint{4.975440in}{2.555000in}}%
\pgfpathlineto{\pgfqpoint{4.976680in}{2.590000in}}%
\pgfpathlineto{\pgfqpoint{4.980400in}{1.995000in}}%
\pgfpathlineto{\pgfqpoint{4.982880in}{2.555000in}}%
\pgfpathlineto{\pgfqpoint{4.984120in}{2.100000in}}%
\pgfpathlineto{\pgfqpoint{4.989080in}{2.625000in}}%
\pgfpathlineto{\pgfqpoint{4.991560in}{2.135000in}}%
\pgfpathlineto{\pgfqpoint{4.994040in}{2.555000in}}%
\pgfpathlineto{\pgfqpoint{4.995280in}{2.415000in}}%
\pgfpathlineto{\pgfqpoint{4.996520in}{2.485000in}}%
\pgfpathlineto{\pgfqpoint{4.997760in}{2.450000in}}%
\pgfpathlineto{\pgfqpoint{4.999000in}{2.730000in}}%
\pgfpathlineto{\pgfqpoint{5.000240in}{2.625000in}}%
\pgfpathlineto{\pgfqpoint{5.002720in}{2.695000in}}%
\pgfpathlineto{\pgfqpoint{5.003960in}{2.380000in}}%
\pgfpathlineto{\pgfqpoint{5.005200in}{2.485000in}}%
\pgfpathlineto{\pgfqpoint{5.006440in}{2.345000in}}%
\pgfpathlineto{\pgfqpoint{5.007680in}{2.485000in}}%
\pgfpathlineto{\pgfqpoint{5.008920in}{2.800000in}}%
\pgfpathlineto{\pgfqpoint{5.010160in}{2.485000in}}%
\pgfpathlineto{\pgfqpoint{5.011400in}{2.485000in}}%
\pgfpathlineto{\pgfqpoint{5.012640in}{2.450000in}}%
\pgfpathlineto{\pgfqpoint{5.013880in}{2.310000in}}%
\pgfpathlineto{\pgfqpoint{5.016360in}{2.555000in}}%
\pgfpathlineto{\pgfqpoint{5.017600in}{2.310000in}}%
\pgfpathlineto{\pgfqpoint{5.018840in}{2.485000in}}%
\pgfpathlineto{\pgfqpoint{5.020080in}{2.940000in}}%
\pgfpathlineto{\pgfqpoint{5.021320in}{2.940000in}}%
\pgfpathlineto{\pgfqpoint{5.022560in}{3.010000in}}%
\pgfpathlineto{\pgfqpoint{5.023800in}{2.555000in}}%
\pgfpathlineto{\pgfqpoint{5.025040in}{2.765000in}}%
\pgfpathlineto{\pgfqpoint{5.026280in}{2.345000in}}%
\pgfpathlineto{\pgfqpoint{5.027520in}{2.450000in}}%
\pgfpathlineto{\pgfqpoint{5.028760in}{2.415000in}}%
\pgfpathlineto{\pgfqpoint{5.030000in}{2.485000in}}%
\pgfpathlineto{\pgfqpoint{5.031240in}{2.765000in}}%
\pgfpathlineto{\pgfqpoint{5.032480in}{2.275000in}}%
\pgfpathlineto{\pgfqpoint{5.033720in}{2.800000in}}%
\pgfpathlineto{\pgfqpoint{5.034960in}{2.730000in}}%
\pgfpathlineto{\pgfqpoint{5.036200in}{2.415000in}}%
\pgfpathlineto{\pgfqpoint{5.038680in}{2.520000in}}%
\pgfpathlineto{\pgfqpoint{5.039920in}{2.380000in}}%
\pgfpathlineto{\pgfqpoint{5.041160in}{2.625000in}}%
\pgfpathlineto{\pgfqpoint{5.042400in}{2.555000in}}%
\pgfpathlineto{\pgfqpoint{5.043640in}{2.310000in}}%
\pgfpathlineto{\pgfqpoint{5.044880in}{2.485000in}}%
\pgfpathlineto{\pgfqpoint{5.047360in}{2.485000in}}%
\pgfpathlineto{\pgfqpoint{5.048600in}{2.275000in}}%
\pgfpathlineto{\pgfqpoint{5.049840in}{2.345000in}}%
\pgfpathlineto{\pgfqpoint{5.051080in}{2.730000in}}%
\pgfpathlineto{\pgfqpoint{5.054800in}{2.135000in}}%
\pgfpathlineto{\pgfqpoint{5.056040in}{2.310000in}}%
\pgfpathlineto{\pgfqpoint{5.058520in}{1.715000in}}%
\pgfpathlineto{\pgfqpoint{5.061000in}{2.555000in}}%
\pgfpathlineto{\pgfqpoint{5.062240in}{2.380000in}}%
\pgfpathlineto{\pgfqpoint{5.064720in}{1.855000in}}%
\pgfpathlineto{\pgfqpoint{5.065960in}{2.380000in}}%
\pgfpathlineto{\pgfqpoint{5.067200in}{1.680000in}}%
\pgfpathlineto{\pgfqpoint{5.068440in}{2.485000in}}%
\pgfpathlineto{\pgfqpoint{5.069680in}{2.170000in}}%
\pgfpathlineto{\pgfqpoint{5.070920in}{2.555000in}}%
\pgfpathlineto{\pgfqpoint{5.074640in}{2.135000in}}%
\pgfpathlineto{\pgfqpoint{5.075880in}{2.205000in}}%
\pgfpathlineto{\pgfqpoint{5.077120in}{2.520000in}}%
\pgfpathlineto{\pgfqpoint{5.078360in}{2.275000in}}%
\pgfpathlineto{\pgfqpoint{5.080840in}{2.485000in}}%
\pgfpathlineto{\pgfqpoint{5.083320in}{2.135000in}}%
\pgfpathlineto{\pgfqpoint{5.084560in}{2.450000in}}%
\pgfpathlineto{\pgfqpoint{5.085800in}{2.135000in}}%
\pgfpathlineto{\pgfqpoint{5.087040in}{2.660000in}}%
\pgfpathlineto{\pgfqpoint{5.088280in}{2.170000in}}%
\pgfpathlineto{\pgfqpoint{5.089520in}{2.240000in}}%
\pgfpathlineto{\pgfqpoint{5.092000in}{2.170000in}}%
\pgfpathlineto{\pgfqpoint{5.093240in}{2.345000in}}%
\pgfpathlineto{\pgfqpoint{5.094480in}{2.240000in}}%
\pgfpathlineto{\pgfqpoint{5.096960in}{2.590000in}}%
\pgfpathlineto{\pgfqpoint{5.098200in}{2.100000in}}%
\pgfpathlineto{\pgfqpoint{5.099440in}{2.135000in}}%
\pgfpathlineto{\pgfqpoint{5.100680in}{2.415000in}}%
\pgfpathlineto{\pgfqpoint{5.101920in}{2.345000in}}%
\pgfpathlineto{\pgfqpoint{5.103160in}{2.485000in}}%
\pgfpathlineto{\pgfqpoint{5.105640in}{2.380000in}}%
\pgfpathlineto{\pgfqpoint{5.108120in}{2.310000in}}%
\pgfpathlineto{\pgfqpoint{5.110600in}{2.730000in}}%
\pgfpathlineto{\pgfqpoint{5.111840in}{2.310000in}}%
\pgfpathlineto{\pgfqpoint{5.113080in}{2.345000in}}%
\pgfpathlineto{\pgfqpoint{5.114320in}{2.275000in}}%
\pgfpathlineto{\pgfqpoint{5.115560in}{1.995000in}}%
\pgfpathlineto{\pgfqpoint{5.116800in}{2.625000in}}%
\pgfpathlineto{\pgfqpoint{5.118040in}{2.310000in}}%
\pgfpathlineto{\pgfqpoint{5.119280in}{2.660000in}}%
\pgfpathlineto{\pgfqpoint{5.120520in}{2.240000in}}%
\pgfpathlineto{\pgfqpoint{5.121760in}{2.450000in}}%
\pgfpathlineto{\pgfqpoint{5.123000in}{2.205000in}}%
\pgfpathlineto{\pgfqpoint{5.124240in}{2.695000in}}%
\pgfpathlineto{\pgfqpoint{5.125480in}{2.520000in}}%
\pgfpathlineto{\pgfqpoint{5.126720in}{2.590000in}}%
\pgfpathlineto{\pgfqpoint{5.127960in}{2.765000in}}%
\pgfpathlineto{\pgfqpoint{5.129200in}{2.555000in}}%
\pgfpathlineto{\pgfqpoint{5.130440in}{2.625000in}}%
\pgfpathlineto{\pgfqpoint{5.132920in}{2.135000in}}%
\pgfpathlineto{\pgfqpoint{5.134160in}{2.450000in}}%
\pgfpathlineto{\pgfqpoint{5.135400in}{2.135000in}}%
\pgfpathlineto{\pgfqpoint{5.136640in}{2.135000in}}%
\pgfpathlineto{\pgfqpoint{5.137880in}{2.065000in}}%
\pgfpathlineto{\pgfqpoint{5.139120in}{2.555000in}}%
\pgfpathlineto{\pgfqpoint{5.140360in}{2.380000in}}%
\pgfpathlineto{\pgfqpoint{5.141600in}{2.730000in}}%
\pgfpathlineto{\pgfqpoint{5.144080in}{2.415000in}}%
\pgfpathlineto{\pgfqpoint{5.145320in}{2.415000in}}%
\pgfpathlineto{\pgfqpoint{5.147800in}{2.730000in}}%
\pgfpathlineto{\pgfqpoint{5.149040in}{2.555000in}}%
\pgfpathlineto{\pgfqpoint{5.150280in}{2.170000in}}%
\pgfpathlineto{\pgfqpoint{5.151520in}{2.485000in}}%
\pgfpathlineto{\pgfqpoint{5.152760in}{2.485000in}}%
\pgfpathlineto{\pgfqpoint{5.154000in}{2.590000in}}%
\pgfpathlineto{\pgfqpoint{5.155240in}{2.590000in}}%
\pgfpathlineto{\pgfqpoint{5.156480in}{2.520000in}}%
\pgfpathlineto{\pgfqpoint{5.157720in}{2.205000in}}%
\pgfpathlineto{\pgfqpoint{5.158960in}{2.450000in}}%
\pgfpathlineto{\pgfqpoint{5.160200in}{2.415000in}}%
\pgfpathlineto{\pgfqpoint{5.161440in}{2.730000in}}%
\pgfpathlineto{\pgfqpoint{5.162680in}{2.415000in}}%
\pgfpathlineto{\pgfqpoint{5.163920in}{2.450000in}}%
\pgfpathlineto{\pgfqpoint{5.165160in}{2.345000in}}%
\pgfpathlineto{\pgfqpoint{5.166400in}{2.345000in}}%
\pgfpathlineto{\pgfqpoint{5.167640in}{2.380000in}}%
\pgfpathlineto{\pgfqpoint{5.168880in}{2.520000in}}%
\pgfpathlineto{\pgfqpoint{5.170120in}{2.380000in}}%
\pgfpathlineto{\pgfqpoint{5.171360in}{2.030000in}}%
\pgfpathlineto{\pgfqpoint{5.172600in}{2.485000in}}%
\pgfpathlineto{\pgfqpoint{5.173840in}{2.065000in}}%
\pgfpathlineto{\pgfqpoint{5.175080in}{2.275000in}}%
\pgfpathlineto{\pgfqpoint{5.176320in}{2.695000in}}%
\pgfpathlineto{\pgfqpoint{5.178800in}{2.345000in}}%
\pgfpathlineto{\pgfqpoint{5.180040in}{2.100000in}}%
\pgfpathlineto{\pgfqpoint{5.181280in}{2.590000in}}%
\pgfpathlineto{\pgfqpoint{5.182520in}{2.205000in}}%
\pgfpathlineto{\pgfqpoint{5.183760in}{2.380000in}}%
\pgfpathlineto{\pgfqpoint{5.185000in}{2.345000in}}%
\pgfpathlineto{\pgfqpoint{5.186240in}{2.275000in}}%
\pgfpathlineto{\pgfqpoint{5.187480in}{2.905000in}}%
\pgfpathlineto{\pgfqpoint{5.188720in}{2.380000in}}%
\pgfpathlineto{\pgfqpoint{5.189960in}{2.590000in}}%
\pgfpathlineto{\pgfqpoint{5.191200in}{2.485000in}}%
\pgfpathlineto{\pgfqpoint{5.192440in}{2.660000in}}%
\pgfpathlineto{\pgfqpoint{5.193680in}{2.450000in}}%
\pgfpathlineto{\pgfqpoint{5.194920in}{2.450000in}}%
\pgfpathlineto{\pgfqpoint{5.196160in}{2.310000in}}%
\pgfpathlineto{\pgfqpoint{5.197400in}{2.485000in}}%
\pgfpathlineto{\pgfqpoint{5.198640in}{2.100000in}}%
\pgfpathlineto{\pgfqpoint{5.199880in}{2.380000in}}%
\pgfpathlineto{\pgfqpoint{5.201120in}{1.995000in}}%
\pgfpathlineto{\pgfqpoint{5.202360in}{2.030000in}}%
\pgfpathlineto{\pgfqpoint{5.203600in}{2.310000in}}%
\pgfpathlineto{\pgfqpoint{5.206080in}{2.100000in}}%
\pgfpathlineto{\pgfqpoint{5.208560in}{2.450000in}}%
\pgfpathlineto{\pgfqpoint{5.209800in}{1.995000in}}%
\pgfpathlineto{\pgfqpoint{5.212280in}{2.415000in}}%
\pgfpathlineto{\pgfqpoint{5.213520in}{2.170000in}}%
\pgfpathlineto{\pgfqpoint{5.216000in}{2.380000in}}%
\pgfpathlineto{\pgfqpoint{5.217240in}{2.590000in}}%
\pgfpathlineto{\pgfqpoint{5.218480in}{2.380000in}}%
\pgfpathlineto{\pgfqpoint{5.219720in}{2.555000in}}%
\pgfpathlineto{\pgfqpoint{5.220960in}{2.100000in}}%
\pgfpathlineto{\pgfqpoint{5.223440in}{2.520000in}}%
\pgfpathlineto{\pgfqpoint{5.224680in}{2.485000in}}%
\pgfpathlineto{\pgfqpoint{5.225920in}{2.380000in}}%
\pgfpathlineto{\pgfqpoint{5.227160in}{2.380000in}}%
\pgfpathlineto{\pgfqpoint{5.228400in}{2.485000in}}%
\pgfpathlineto{\pgfqpoint{5.230880in}{2.030000in}}%
\pgfpathlineto{\pgfqpoint{5.233360in}{2.450000in}}%
\pgfpathlineto{\pgfqpoint{5.234600in}{2.345000in}}%
\pgfpathlineto{\pgfqpoint{5.235840in}{2.520000in}}%
\pgfpathlineto{\pgfqpoint{5.237080in}{2.485000in}}%
\pgfpathlineto{\pgfqpoint{5.238320in}{2.590000in}}%
\pgfpathlineto{\pgfqpoint{5.239560in}{2.240000in}}%
\pgfpathlineto{\pgfqpoint{5.242040in}{2.730000in}}%
\pgfpathlineto{\pgfqpoint{5.243280in}{2.240000in}}%
\pgfpathlineto{\pgfqpoint{5.245760in}{2.660000in}}%
\pgfpathlineto{\pgfqpoint{5.248240in}{2.555000in}}%
\pgfpathlineto{\pgfqpoint{5.249480in}{2.205000in}}%
\pgfpathlineto{\pgfqpoint{5.250720in}{2.275000in}}%
\pgfpathlineto{\pgfqpoint{5.251960in}{2.240000in}}%
\pgfpathlineto{\pgfqpoint{5.253200in}{1.995000in}}%
\pgfpathlineto{\pgfqpoint{5.254440in}{2.275000in}}%
\pgfpathlineto{\pgfqpoint{5.255680in}{2.205000in}}%
\pgfpathlineto{\pgfqpoint{5.258160in}{2.520000in}}%
\pgfpathlineto{\pgfqpoint{5.260640in}{1.925000in}}%
\pgfpathlineto{\pgfqpoint{5.263120in}{2.345000in}}%
\pgfpathlineto{\pgfqpoint{5.264360in}{2.345000in}}%
\pgfpathlineto{\pgfqpoint{5.265600in}{2.310000in}}%
\pgfpathlineto{\pgfqpoint{5.268080in}{1.750000in}}%
\pgfpathlineto{\pgfqpoint{5.269320in}{2.065000in}}%
\pgfpathlineto{\pgfqpoint{5.270560in}{1.995000in}}%
\pgfpathlineto{\pgfqpoint{5.273040in}{2.520000in}}%
\pgfpathlineto{\pgfqpoint{5.275520in}{2.100000in}}%
\pgfpathlineto{\pgfqpoint{5.276760in}{2.205000in}}%
\pgfpathlineto{\pgfqpoint{5.279240in}{2.450000in}}%
\pgfpathlineto{\pgfqpoint{5.280480in}{2.310000in}}%
\pgfpathlineto{\pgfqpoint{5.281720in}{2.310000in}}%
\pgfpathlineto{\pgfqpoint{5.284200in}{2.135000in}}%
\pgfpathlineto{\pgfqpoint{5.285440in}{2.205000in}}%
\pgfpathlineto{\pgfqpoint{5.286680in}{2.380000in}}%
\pgfpathlineto{\pgfqpoint{5.287920in}{2.240000in}}%
\pgfpathlineto{\pgfqpoint{5.289160in}{2.310000in}}%
\pgfpathlineto{\pgfqpoint{5.290400in}{2.275000in}}%
\pgfpathlineto{\pgfqpoint{5.291640in}{2.275000in}}%
\pgfpathlineto{\pgfqpoint{5.292880in}{2.135000in}}%
\pgfpathlineto{\pgfqpoint{5.295360in}{2.590000in}}%
\pgfpathlineto{\pgfqpoint{5.296600in}{2.380000in}}%
\pgfpathlineto{\pgfqpoint{5.297840in}{2.485000in}}%
\pgfpathlineto{\pgfqpoint{5.299080in}{2.345000in}}%
\pgfpathlineto{\pgfqpoint{5.300320in}{2.345000in}}%
\pgfpathlineto{\pgfqpoint{5.301560in}{2.275000in}}%
\pgfpathlineto{\pgfqpoint{5.302800in}{2.345000in}}%
\pgfpathlineto{\pgfqpoint{5.304040in}{2.345000in}}%
\pgfpathlineto{\pgfqpoint{5.305280in}{2.625000in}}%
\pgfpathlineto{\pgfqpoint{5.307760in}{2.240000in}}%
\pgfpathlineto{\pgfqpoint{5.310240in}{2.625000in}}%
\pgfpathlineto{\pgfqpoint{5.311480in}{2.415000in}}%
\pgfpathlineto{\pgfqpoint{5.312720in}{2.450000in}}%
\pgfpathlineto{\pgfqpoint{5.315200in}{2.205000in}}%
\pgfpathlineto{\pgfqpoint{5.316440in}{2.380000in}}%
\pgfpathlineto{\pgfqpoint{5.317680in}{2.205000in}}%
\pgfpathlineto{\pgfqpoint{5.318920in}{2.485000in}}%
\pgfpathlineto{\pgfqpoint{5.322640in}{2.100000in}}%
\pgfpathlineto{\pgfqpoint{5.323880in}{2.450000in}}%
\pgfpathlineto{\pgfqpoint{5.325120in}{2.170000in}}%
\pgfpathlineto{\pgfqpoint{5.326360in}{2.345000in}}%
\pgfpathlineto{\pgfqpoint{5.328840in}{2.135000in}}%
\pgfpathlineto{\pgfqpoint{5.330080in}{2.380000in}}%
\pgfpathlineto{\pgfqpoint{5.331320in}{2.345000in}}%
\pgfpathlineto{\pgfqpoint{5.332560in}{2.450000in}}%
\pgfpathlineto{\pgfqpoint{5.333800in}{2.275000in}}%
\pgfpathlineto{\pgfqpoint{5.336280in}{2.415000in}}%
\pgfpathlineto{\pgfqpoint{5.337520in}{2.135000in}}%
\pgfpathlineto{\pgfqpoint{5.338760in}{2.485000in}}%
\pgfpathlineto{\pgfqpoint{5.340000in}{2.520000in}}%
\pgfpathlineto{\pgfqpoint{5.342480in}{2.275000in}}%
\pgfpathlineto{\pgfqpoint{5.343720in}{2.485000in}}%
\pgfpathlineto{\pgfqpoint{5.344960in}{2.100000in}}%
\pgfpathlineto{\pgfqpoint{5.346200in}{2.240000in}}%
\pgfpathlineto{\pgfqpoint{5.347440in}{2.555000in}}%
\pgfpathlineto{\pgfqpoint{5.348680in}{2.520000in}}%
\pgfpathlineto{\pgfqpoint{5.349920in}{2.380000in}}%
\pgfpathlineto{\pgfqpoint{5.351160in}{2.625000in}}%
\pgfpathlineto{\pgfqpoint{5.352400in}{2.415000in}}%
\pgfpathlineto{\pgfqpoint{5.353640in}{1.820000in}}%
\pgfpathlineto{\pgfqpoint{5.356120in}{2.520000in}}%
\pgfpathlineto{\pgfqpoint{5.357360in}{2.730000in}}%
\pgfpathlineto{\pgfqpoint{5.358600in}{2.660000in}}%
\pgfpathlineto{\pgfqpoint{5.362320in}{2.030000in}}%
\pgfpathlineto{\pgfqpoint{5.363560in}{2.520000in}}%
\pgfpathlineto{\pgfqpoint{5.366040in}{2.205000in}}%
\pgfpathlineto{\pgfqpoint{5.367280in}{2.240000in}}%
\pgfpathlineto{\pgfqpoint{5.368520in}{2.135000in}}%
\pgfpathlineto{\pgfqpoint{5.371000in}{2.520000in}}%
\pgfpathlineto{\pgfqpoint{5.372240in}{1.750000in}}%
\pgfpathlineto{\pgfqpoint{5.373480in}{1.890000in}}%
\pgfpathlineto{\pgfqpoint{5.374720in}{2.590000in}}%
\pgfpathlineto{\pgfqpoint{5.375960in}{2.555000in}}%
\pgfpathlineto{\pgfqpoint{5.377200in}{2.380000in}}%
\pgfpathlineto{\pgfqpoint{5.379680in}{2.590000in}}%
\pgfpathlineto{\pgfqpoint{5.382160in}{2.345000in}}%
\pgfpathlineto{\pgfqpoint{5.383400in}{2.415000in}}%
\pgfpathlineto{\pgfqpoint{5.384640in}{2.555000in}}%
\pgfpathlineto{\pgfqpoint{5.385880in}{2.555000in}}%
\pgfpathlineto{\pgfqpoint{5.387120in}{2.030000in}}%
\pgfpathlineto{\pgfqpoint{5.389600in}{2.730000in}}%
\pgfpathlineto{\pgfqpoint{5.390840in}{2.660000in}}%
\pgfpathlineto{\pgfqpoint{5.392080in}{2.660000in}}%
\pgfpathlineto{\pgfqpoint{5.393320in}{2.415000in}}%
\pgfpathlineto{\pgfqpoint{5.394560in}{2.415000in}}%
\pgfpathlineto{\pgfqpoint{5.395800in}{2.135000in}}%
\pgfpathlineto{\pgfqpoint{5.397040in}{2.345000in}}%
\pgfpathlineto{\pgfqpoint{5.398280in}{2.870000in}}%
\pgfpathlineto{\pgfqpoint{5.400760in}{1.890000in}}%
\pgfpathlineto{\pgfqpoint{5.402000in}{2.100000in}}%
\pgfpathlineto{\pgfqpoint{5.403240in}{2.065000in}}%
\pgfpathlineto{\pgfqpoint{5.404480in}{1.995000in}}%
\pgfpathlineto{\pgfqpoint{5.405720in}{2.205000in}}%
\pgfpathlineto{\pgfqpoint{5.406960in}{2.625000in}}%
\pgfpathlineto{\pgfqpoint{5.408200in}{2.310000in}}%
\pgfpathlineto{\pgfqpoint{5.409440in}{2.450000in}}%
\pgfpathlineto{\pgfqpoint{5.410680in}{2.065000in}}%
\pgfpathlineto{\pgfqpoint{5.411920in}{2.590000in}}%
\pgfpathlineto{\pgfqpoint{5.413160in}{2.275000in}}%
\pgfpathlineto{\pgfqpoint{5.414400in}{2.485000in}}%
\pgfpathlineto{\pgfqpoint{5.415640in}{2.345000in}}%
\pgfpathlineto{\pgfqpoint{5.416880in}{2.555000in}}%
\pgfpathlineto{\pgfqpoint{5.418120in}{2.520000in}}%
\pgfpathlineto{\pgfqpoint{5.420600in}{1.995000in}}%
\pgfpathlineto{\pgfqpoint{5.421840in}{2.380000in}}%
\pgfpathlineto{\pgfqpoint{5.423080in}{2.170000in}}%
\pgfpathlineto{\pgfqpoint{5.424320in}{2.275000in}}%
\pgfpathlineto{\pgfqpoint{5.426800in}{2.030000in}}%
\pgfpathlineto{\pgfqpoint{5.429280in}{2.310000in}}%
\pgfpathlineto{\pgfqpoint{5.430520in}{1.890000in}}%
\pgfpathlineto{\pgfqpoint{5.431760in}{2.275000in}}%
\pgfpathlineto{\pgfqpoint{5.433000in}{2.135000in}}%
\pgfpathlineto{\pgfqpoint{5.435480in}{2.765000in}}%
\pgfpathlineto{\pgfqpoint{5.436720in}{2.310000in}}%
\pgfpathlineto{\pgfqpoint{5.437960in}{2.555000in}}%
\pgfpathlineto{\pgfqpoint{5.439200in}{2.240000in}}%
\pgfpathlineto{\pgfqpoint{5.440440in}{2.730000in}}%
\pgfpathlineto{\pgfqpoint{5.441680in}{2.660000in}}%
\pgfpathlineto{\pgfqpoint{5.442920in}{2.485000in}}%
\pgfpathlineto{\pgfqpoint{5.444160in}{2.765000in}}%
\pgfpathlineto{\pgfqpoint{5.446640in}{2.310000in}}%
\pgfpathlineto{\pgfqpoint{5.447880in}{2.275000in}}%
\pgfpathlineto{\pgfqpoint{5.450360in}{2.520000in}}%
\pgfpathlineto{\pgfqpoint{5.451600in}{2.275000in}}%
\pgfpathlineto{\pgfqpoint{5.452840in}{2.275000in}}%
\pgfpathlineto{\pgfqpoint{5.455320in}{2.485000in}}%
\pgfpathlineto{\pgfqpoint{5.456560in}{2.135000in}}%
\pgfpathlineto{\pgfqpoint{5.459040in}{2.765000in}}%
\pgfpathlineto{\pgfqpoint{5.460280in}{2.520000in}}%
\pgfpathlineto{\pgfqpoint{5.461520in}{2.555000in}}%
\pgfpathlineto{\pgfqpoint{5.462760in}{2.310000in}}%
\pgfpathlineto{\pgfqpoint{5.464000in}{2.345000in}}%
\pgfpathlineto{\pgfqpoint{5.465240in}{2.310000in}}%
\pgfpathlineto{\pgfqpoint{5.466480in}{2.415000in}}%
\pgfpathlineto{\pgfqpoint{5.467720in}{2.800000in}}%
\pgfpathlineto{\pgfqpoint{5.470200in}{2.345000in}}%
\pgfpathlineto{\pgfqpoint{5.471440in}{2.625000in}}%
\pgfpathlineto{\pgfqpoint{5.473920in}{2.240000in}}%
\pgfpathlineto{\pgfqpoint{5.475160in}{2.590000in}}%
\pgfpathlineto{\pgfqpoint{5.476400in}{2.485000in}}%
\pgfpathlineto{\pgfqpoint{5.477640in}{2.660000in}}%
\pgfpathlineto{\pgfqpoint{5.478880in}{2.380000in}}%
\pgfpathlineto{\pgfqpoint{5.480120in}{2.730000in}}%
\pgfpathlineto{\pgfqpoint{5.481360in}{2.660000in}}%
\pgfpathlineto{\pgfqpoint{5.483840in}{2.100000in}}%
\pgfpathlineto{\pgfqpoint{5.485080in}{2.520000in}}%
\pgfpathlineto{\pgfqpoint{5.486320in}{2.520000in}}%
\pgfpathlineto{\pgfqpoint{5.490040in}{2.135000in}}%
\pgfpathlineto{\pgfqpoint{5.491280in}{2.240000in}}%
\pgfpathlineto{\pgfqpoint{5.493760in}{2.065000in}}%
\pgfpathlineto{\pgfqpoint{5.495000in}{2.205000in}}%
\pgfpathlineto{\pgfqpoint{5.496240in}{2.135000in}}%
\pgfpathlineto{\pgfqpoint{5.497480in}{2.380000in}}%
\pgfpathlineto{\pgfqpoint{5.498720in}{1.995000in}}%
\pgfpathlineto{\pgfqpoint{5.502440in}{2.555000in}}%
\pgfpathlineto{\pgfqpoint{5.503680in}{2.205000in}}%
\pgfpathlineto{\pgfqpoint{5.506160in}{2.555000in}}%
\pgfpathlineto{\pgfqpoint{5.507400in}{2.275000in}}%
\pgfpathlineto{\pgfqpoint{5.508640in}{2.380000in}}%
\pgfpathlineto{\pgfqpoint{5.509880in}{2.100000in}}%
\pgfpathlineto{\pgfqpoint{5.511120in}{2.205000in}}%
\pgfpathlineto{\pgfqpoint{5.512360in}{2.520000in}}%
\pgfpathlineto{\pgfqpoint{5.514840in}{2.100000in}}%
\pgfpathlineto{\pgfqpoint{5.516080in}{2.485000in}}%
\pgfpathlineto{\pgfqpoint{5.518560in}{2.135000in}}%
\pgfpathlineto{\pgfqpoint{5.519800in}{2.205000in}}%
\pgfpathlineto{\pgfqpoint{5.521040in}{2.030000in}}%
\pgfpathlineto{\pgfqpoint{5.522280in}{2.485000in}}%
\pgfpathlineto{\pgfqpoint{5.523520in}{2.135000in}}%
\pgfpathlineto{\pgfqpoint{5.524760in}{2.240000in}}%
\pgfpathlineto{\pgfqpoint{5.526000in}{2.205000in}}%
\pgfpathlineto{\pgfqpoint{5.527240in}{2.310000in}}%
\pgfpathlineto{\pgfqpoint{5.528480in}{2.275000in}}%
\pgfpathlineto{\pgfqpoint{5.529720in}{2.345000in}}%
\pgfpathlineto{\pgfqpoint{5.530960in}{2.590000in}}%
\pgfpathlineto{\pgfqpoint{5.532200in}{2.555000in}}%
\pgfpathlineto{\pgfqpoint{5.533440in}{2.170000in}}%
\pgfpathlineto{\pgfqpoint{5.534680in}{2.275000in}}%
\pgfpathlineto{\pgfqpoint{5.535920in}{2.100000in}}%
\pgfpathlineto{\pgfqpoint{5.538400in}{2.450000in}}%
\pgfpathlineto{\pgfqpoint{5.540880in}{2.625000in}}%
\pgfpathlineto{\pgfqpoint{5.542120in}{2.450000in}}%
\pgfpathlineto{\pgfqpoint{5.543360in}{2.625000in}}%
\pgfpathlineto{\pgfqpoint{5.544600in}{2.415000in}}%
\pgfpathlineto{\pgfqpoint{5.548320in}{2.765000in}}%
\pgfpathlineto{\pgfqpoint{5.549560in}{2.275000in}}%
\pgfpathlineto{\pgfqpoint{5.552040in}{2.835000in}}%
\pgfpathlineto{\pgfqpoint{5.553280in}{2.450000in}}%
\pgfpathlineto{\pgfqpoint{5.554520in}{2.730000in}}%
\pgfpathlineto{\pgfqpoint{5.555760in}{2.065000in}}%
\pgfpathlineto{\pgfqpoint{5.557000in}{2.275000in}}%
\pgfpathlineto{\pgfqpoint{5.558240in}{2.100000in}}%
\pgfpathlineto{\pgfqpoint{5.560720in}{2.450000in}}%
\pgfpathlineto{\pgfqpoint{5.561960in}{2.170000in}}%
\pgfpathlineto{\pgfqpoint{5.564440in}{2.310000in}}%
\pgfpathlineto{\pgfqpoint{5.565680in}{2.520000in}}%
\pgfpathlineto{\pgfqpoint{5.566920in}{2.240000in}}%
\pgfpathlineto{\pgfqpoint{5.568160in}{2.625000in}}%
\pgfpathlineto{\pgfqpoint{5.569400in}{2.485000in}}%
\pgfpathlineto{\pgfqpoint{5.571880in}{2.765000in}}%
\pgfpathlineto{\pgfqpoint{5.573120in}{2.450000in}}%
\pgfpathlineto{\pgfqpoint{5.575600in}{2.800000in}}%
\pgfpathlineto{\pgfqpoint{5.576840in}{2.240000in}}%
\pgfpathlineto{\pgfqpoint{5.579320in}{2.555000in}}%
\pgfpathlineto{\pgfqpoint{5.580560in}{2.345000in}}%
\pgfpathlineto{\pgfqpoint{5.583040in}{2.450000in}}%
\pgfpathlineto{\pgfqpoint{5.584280in}{2.380000in}}%
\pgfpathlineto{\pgfqpoint{5.585520in}{2.240000in}}%
\pgfpathlineto{\pgfqpoint{5.586760in}{2.240000in}}%
\pgfpathlineto{\pgfqpoint{5.589240in}{2.695000in}}%
\pgfpathlineto{\pgfqpoint{5.591720in}{2.135000in}}%
\pgfpathlineto{\pgfqpoint{5.592960in}{2.380000in}}%
\pgfpathlineto{\pgfqpoint{5.594200in}{2.345000in}}%
\pgfpathlineto{\pgfqpoint{5.595440in}{2.730000in}}%
\pgfpathlineto{\pgfqpoint{5.599160in}{2.205000in}}%
\pgfpathlineto{\pgfqpoint{5.601640in}{2.555000in}}%
\pgfpathlineto{\pgfqpoint{5.602880in}{2.205000in}}%
\pgfpathlineto{\pgfqpoint{5.604120in}{2.205000in}}%
\pgfpathlineto{\pgfqpoint{5.605360in}{2.625000in}}%
\pgfpathlineto{\pgfqpoint{5.606600in}{2.240000in}}%
\pgfpathlineto{\pgfqpoint{5.607840in}{2.240000in}}%
\pgfpathlineto{\pgfqpoint{5.610320in}{2.485000in}}%
\pgfpathlineto{\pgfqpoint{5.612800in}{2.660000in}}%
\pgfpathlineto{\pgfqpoint{5.615280in}{2.380000in}}%
\pgfpathlineto{\pgfqpoint{5.616520in}{2.380000in}}%
\pgfpathlineto{\pgfqpoint{5.617760in}{2.345000in}}%
\pgfpathlineto{\pgfqpoint{5.619000in}{2.205000in}}%
\pgfpathlineto{\pgfqpoint{5.620240in}{2.345000in}}%
\pgfpathlineto{\pgfqpoint{5.621480in}{2.345000in}}%
\pgfpathlineto{\pgfqpoint{5.623960in}{2.100000in}}%
\pgfpathlineto{\pgfqpoint{5.627680in}{2.485000in}}%
\pgfpathlineto{\pgfqpoint{5.630160in}{2.345000in}}%
\pgfpathlineto{\pgfqpoint{5.631400in}{2.345000in}}%
\pgfpathlineto{\pgfqpoint{5.632640in}{2.275000in}}%
\pgfpathlineto{\pgfqpoint{5.633880in}{2.310000in}}%
\pgfpathlineto{\pgfqpoint{5.635120in}{2.240000in}}%
\pgfpathlineto{\pgfqpoint{5.636360in}{2.275000in}}%
\pgfpathlineto{\pgfqpoint{5.637600in}{2.555000in}}%
\pgfpathlineto{\pgfqpoint{5.638840in}{2.205000in}}%
\pgfpathlineto{\pgfqpoint{5.640080in}{2.415000in}}%
\pgfpathlineto{\pgfqpoint{5.641320in}{2.345000in}}%
\pgfpathlineto{\pgfqpoint{5.642560in}{2.555000in}}%
\pgfpathlineto{\pgfqpoint{5.643800in}{2.135000in}}%
\pgfpathlineto{\pgfqpoint{5.646280in}{2.800000in}}%
\pgfpathlineto{\pgfqpoint{5.647520in}{2.240000in}}%
\pgfpathlineto{\pgfqpoint{5.648760in}{2.240000in}}%
\pgfpathlineto{\pgfqpoint{5.650000in}{1.995000in}}%
\pgfpathlineto{\pgfqpoint{5.651240in}{2.520000in}}%
\pgfpathlineto{\pgfqpoint{5.653720in}{2.170000in}}%
\pgfpathlineto{\pgfqpoint{5.654960in}{2.555000in}}%
\pgfpathlineto{\pgfqpoint{5.656200in}{2.240000in}}%
\pgfpathlineto{\pgfqpoint{5.657440in}{2.415000in}}%
\pgfpathlineto{\pgfqpoint{5.658680in}{2.415000in}}%
\pgfpathlineto{\pgfqpoint{5.661160in}{2.275000in}}%
\pgfpathlineto{\pgfqpoint{5.662400in}{2.520000in}}%
\pgfpathlineto{\pgfqpoint{5.664880in}{2.345000in}}%
\pgfpathlineto{\pgfqpoint{5.666120in}{2.590000in}}%
\pgfpathlineto{\pgfqpoint{5.668600in}{2.590000in}}%
\pgfpathlineto{\pgfqpoint{5.669840in}{2.485000in}}%
\pgfpathlineto{\pgfqpoint{5.671080in}{2.905000in}}%
\pgfpathlineto{\pgfqpoint{5.673560in}{2.205000in}}%
\pgfpathlineto{\pgfqpoint{5.674800in}{2.450000in}}%
\pgfpathlineto{\pgfqpoint{5.676040in}{2.415000in}}%
\pgfpathlineto{\pgfqpoint{5.677280in}{2.415000in}}%
\pgfpathlineto{\pgfqpoint{5.678520in}{2.380000in}}%
\pgfpathlineto{\pgfqpoint{5.679760in}{2.205000in}}%
\pgfpathlineto{\pgfqpoint{5.681000in}{2.380000in}}%
\pgfpathlineto{\pgfqpoint{5.683480in}{2.275000in}}%
\pgfpathlineto{\pgfqpoint{5.684720in}{2.590000in}}%
\pgfpathlineto{\pgfqpoint{5.685960in}{2.205000in}}%
\pgfpathlineto{\pgfqpoint{5.687200in}{2.310000in}}%
\pgfpathlineto{\pgfqpoint{5.688440in}{2.065000in}}%
\pgfpathlineto{\pgfqpoint{5.689680in}{2.100000in}}%
\pgfpathlineto{\pgfqpoint{5.692160in}{2.485000in}}%
\pgfpathlineto{\pgfqpoint{5.693400in}{2.555000in}}%
\pgfpathlineto{\pgfqpoint{5.694640in}{2.555000in}}%
\pgfpathlineto{\pgfqpoint{5.695880in}{2.590000in}}%
\pgfpathlineto{\pgfqpoint{5.698360in}{2.520000in}}%
\pgfpathlineto{\pgfqpoint{5.699600in}{2.625000in}}%
\pgfpathlineto{\pgfqpoint{5.702080in}{2.170000in}}%
\pgfpathlineto{\pgfqpoint{5.703320in}{2.310000in}}%
\pgfpathlineto{\pgfqpoint{5.704560in}{2.065000in}}%
\pgfpathlineto{\pgfqpoint{5.707040in}{2.590000in}}%
\pgfpathlineto{\pgfqpoint{5.708280in}{2.660000in}}%
\pgfpathlineto{\pgfqpoint{5.709520in}{2.310000in}}%
\pgfpathlineto{\pgfqpoint{5.710760in}{2.345000in}}%
\pgfpathlineto{\pgfqpoint{5.712000in}{2.590000in}}%
\pgfpathlineto{\pgfqpoint{5.713240in}{2.380000in}}%
\pgfpathlineto{\pgfqpoint{5.714480in}{2.590000in}}%
\pgfpathlineto{\pgfqpoint{5.716960in}{2.170000in}}%
\pgfpathlineto{\pgfqpoint{5.718200in}{2.345000in}}%
\pgfpathlineto{\pgfqpoint{5.719440in}{2.205000in}}%
\pgfpathlineto{\pgfqpoint{5.720680in}{2.310000in}}%
\pgfpathlineto{\pgfqpoint{5.721920in}{2.625000in}}%
\pgfpathlineto{\pgfqpoint{5.723160in}{2.450000in}}%
\pgfpathlineto{\pgfqpoint{5.724400in}{2.870000in}}%
\pgfpathlineto{\pgfqpoint{5.725640in}{2.730000in}}%
\pgfpathlineto{\pgfqpoint{5.726880in}{2.205000in}}%
\pgfpathlineto{\pgfqpoint{5.728120in}{2.345000in}}%
\pgfpathlineto{\pgfqpoint{5.730600in}{2.240000in}}%
\pgfpathlineto{\pgfqpoint{5.731840in}{2.450000in}}%
\pgfpathlineto{\pgfqpoint{5.734320in}{2.275000in}}%
\pgfpathlineto{\pgfqpoint{5.736800in}{2.555000in}}%
\pgfpathlineto{\pgfqpoint{5.738040in}{2.275000in}}%
\pgfpathlineto{\pgfqpoint{5.739280in}{2.310000in}}%
\pgfpathlineto{\pgfqpoint{5.740520in}{1.925000in}}%
\pgfpathlineto{\pgfqpoint{5.743000in}{2.240000in}}%
\pgfpathlineto{\pgfqpoint{5.744240in}{2.205000in}}%
\pgfpathlineto{\pgfqpoint{5.745480in}{2.380000in}}%
\pgfpathlineto{\pgfqpoint{5.746720in}{2.345000in}}%
\pgfpathlineto{\pgfqpoint{5.747960in}{2.275000in}}%
\pgfpathlineto{\pgfqpoint{5.749200in}{2.380000in}}%
\pgfpathlineto{\pgfqpoint{5.750440in}{2.835000in}}%
\pgfpathlineto{\pgfqpoint{5.752920in}{2.380000in}}%
\pgfpathlineto{\pgfqpoint{5.754160in}{2.590000in}}%
\pgfpathlineto{\pgfqpoint{5.755400in}{2.590000in}}%
\pgfpathlineto{\pgfqpoint{5.756640in}{2.625000in}}%
\pgfpathlineto{\pgfqpoint{5.757880in}{2.520000in}}%
\pgfpathlineto{\pgfqpoint{5.759120in}{2.240000in}}%
\pgfpathlineto{\pgfqpoint{5.760360in}{2.450000in}}%
\pgfpathlineto{\pgfqpoint{5.761600in}{2.205000in}}%
\pgfpathlineto{\pgfqpoint{5.762840in}{2.625000in}}%
\pgfpathlineto{\pgfqpoint{5.764080in}{2.625000in}}%
\pgfpathlineto{\pgfqpoint{5.765320in}{2.170000in}}%
\pgfpathlineto{\pgfqpoint{5.766560in}{2.275000in}}%
\pgfpathlineto{\pgfqpoint{5.767800in}{1.960000in}}%
\pgfpathlineto{\pgfqpoint{5.769040in}{2.415000in}}%
\pgfpathlineto{\pgfqpoint{5.770280in}{2.415000in}}%
\pgfpathlineto{\pgfqpoint{5.771520in}{2.625000in}}%
\pgfpathlineto{\pgfqpoint{5.774000in}{2.345000in}}%
\pgfpathlineto{\pgfqpoint{5.775240in}{2.555000in}}%
\pgfpathlineto{\pgfqpoint{5.776480in}{2.345000in}}%
\pgfpathlineto{\pgfqpoint{5.780200in}{2.660000in}}%
\pgfpathlineto{\pgfqpoint{5.782680in}{2.275000in}}%
\pgfpathlineto{\pgfqpoint{5.785160in}{2.520000in}}%
\pgfpathlineto{\pgfqpoint{5.786400in}{2.485000in}}%
\pgfpathlineto{\pgfqpoint{5.787640in}{2.135000in}}%
\pgfpathlineto{\pgfqpoint{5.788880in}{2.485000in}}%
\pgfpathlineto{\pgfqpoint{5.790120in}{2.520000in}}%
\pgfpathlineto{\pgfqpoint{5.791360in}{2.660000in}}%
\pgfpathlineto{\pgfqpoint{5.795080in}{2.205000in}}%
\pgfpathlineto{\pgfqpoint{5.796320in}{2.135000in}}%
\pgfpathlineto{\pgfqpoint{5.797560in}{2.520000in}}%
\pgfpathlineto{\pgfqpoint{5.798800in}{2.485000in}}%
\pgfpathlineto{\pgfqpoint{5.800040in}{2.625000in}}%
\pgfpathlineto{\pgfqpoint{5.801280in}{2.345000in}}%
\pgfpathlineto{\pgfqpoint{5.802520in}{2.520000in}}%
\pgfpathlineto{\pgfqpoint{5.803760in}{2.415000in}}%
\pgfpathlineto{\pgfqpoint{5.805000in}{2.485000in}}%
\pgfpathlineto{\pgfqpoint{5.806240in}{2.205000in}}%
\pgfpathlineto{\pgfqpoint{5.807480in}{2.240000in}}%
\pgfpathlineto{\pgfqpoint{5.808720in}{2.135000in}}%
\pgfpathlineto{\pgfqpoint{5.809960in}{2.170000in}}%
\pgfpathlineto{\pgfqpoint{5.811200in}{1.960000in}}%
\pgfpathlineto{\pgfqpoint{5.812440in}{2.240000in}}%
\pgfpathlineto{\pgfqpoint{5.813680in}{2.065000in}}%
\pgfpathlineto{\pgfqpoint{5.814920in}{2.555000in}}%
\pgfpathlineto{\pgfqpoint{5.816160in}{2.415000in}}%
\pgfpathlineto{\pgfqpoint{5.817400in}{2.485000in}}%
\pgfpathlineto{\pgfqpoint{5.819880in}{2.170000in}}%
\pgfpathlineto{\pgfqpoint{5.821120in}{2.240000in}}%
\pgfpathlineto{\pgfqpoint{5.823600in}{2.625000in}}%
\pgfpathlineto{\pgfqpoint{5.824840in}{2.555000in}}%
\pgfpathlineto{\pgfqpoint{5.826080in}{2.065000in}}%
\pgfpathlineto{\pgfqpoint{5.828560in}{2.520000in}}%
\pgfpathlineto{\pgfqpoint{5.829800in}{2.765000in}}%
\pgfpathlineto{\pgfqpoint{5.832280in}{2.065000in}}%
\pgfpathlineto{\pgfqpoint{5.833520in}{2.275000in}}%
\pgfpathlineto{\pgfqpoint{5.834760in}{2.240000in}}%
\pgfpathlineto{\pgfqpoint{5.836000in}{2.275000in}}%
\pgfpathlineto{\pgfqpoint{5.837240in}{2.450000in}}%
\pgfpathlineto{\pgfqpoint{5.838480in}{2.170000in}}%
\pgfpathlineto{\pgfqpoint{5.840960in}{2.695000in}}%
\pgfpathlineto{\pgfqpoint{5.842200in}{2.800000in}}%
\pgfpathlineto{\pgfqpoint{5.843440in}{2.800000in}}%
\pgfpathlineto{\pgfqpoint{5.844680in}{2.730000in}}%
\pgfpathlineto{\pgfqpoint{5.845920in}{2.275000in}}%
\pgfpathlineto{\pgfqpoint{5.847160in}{2.450000in}}%
\pgfpathlineto{\pgfqpoint{5.848400in}{2.240000in}}%
\pgfpathlineto{\pgfqpoint{5.849640in}{2.625000in}}%
\pgfpathlineto{\pgfqpoint{5.850880in}{2.030000in}}%
\pgfpathlineto{\pgfqpoint{5.853360in}{2.590000in}}%
\pgfpathlineto{\pgfqpoint{5.854600in}{2.660000in}}%
\pgfpathlineto{\pgfqpoint{5.855840in}{2.975000in}}%
\pgfpathlineto{\pgfqpoint{5.857080in}{2.520000in}}%
\pgfpathlineto{\pgfqpoint{5.858320in}{2.695000in}}%
\pgfpathlineto{\pgfqpoint{5.859560in}{2.065000in}}%
\pgfpathlineto{\pgfqpoint{5.860800in}{2.205000in}}%
\pgfpathlineto{\pgfqpoint{5.862040in}{2.625000in}}%
\pgfpathlineto{\pgfqpoint{5.863280in}{2.345000in}}%
\pgfpathlineto{\pgfqpoint{5.864520in}{2.380000in}}%
\pgfpathlineto{\pgfqpoint{5.865760in}{2.275000in}}%
\pgfpathlineto{\pgfqpoint{5.867000in}{2.450000in}}%
\pgfpathlineto{\pgfqpoint{5.868240in}{2.450000in}}%
\pgfpathlineto{\pgfqpoint{5.869480in}{2.240000in}}%
\pgfpathlineto{\pgfqpoint{5.870720in}{2.625000in}}%
\pgfpathlineto{\pgfqpoint{5.871960in}{2.240000in}}%
\pgfpathlineto{\pgfqpoint{5.875680in}{2.730000in}}%
\pgfpathlineto{\pgfqpoint{5.876920in}{2.625000in}}%
\pgfpathlineto{\pgfqpoint{5.878160in}{2.660000in}}%
\pgfpathlineto{\pgfqpoint{5.879400in}{2.590000in}}%
\pgfpathlineto{\pgfqpoint{5.880640in}{2.450000in}}%
\pgfpathlineto{\pgfqpoint{5.881880in}{2.590000in}}%
\pgfpathlineto{\pgfqpoint{5.883120in}{2.555000in}}%
\pgfpathlineto{\pgfqpoint{5.884360in}{2.485000in}}%
\pgfpathlineto{\pgfqpoint{5.886840in}{2.205000in}}%
\pgfpathlineto{\pgfqpoint{5.889320in}{2.415000in}}%
\pgfpathlineto{\pgfqpoint{5.890560in}{2.065000in}}%
\pgfpathlineto{\pgfqpoint{5.891800in}{2.275000in}}%
\pgfpathlineto{\pgfqpoint{5.893040in}{2.275000in}}%
\pgfpathlineto{\pgfqpoint{5.894280in}{2.135000in}}%
\pgfpathlineto{\pgfqpoint{5.895520in}{2.485000in}}%
\pgfpathlineto{\pgfqpoint{5.896760in}{2.345000in}}%
\pgfpathlineto{\pgfqpoint{5.898000in}{2.555000in}}%
\pgfpathlineto{\pgfqpoint{5.899240in}{2.100000in}}%
\pgfpathlineto{\pgfqpoint{5.901720in}{2.485000in}}%
\pgfpathlineto{\pgfqpoint{5.902960in}{2.170000in}}%
\pgfpathlineto{\pgfqpoint{5.904200in}{2.240000in}}%
\pgfpathlineto{\pgfqpoint{5.905440in}{2.100000in}}%
\pgfpathlineto{\pgfqpoint{5.906680in}{2.520000in}}%
\pgfpathlineto{\pgfqpoint{5.907920in}{2.030000in}}%
\pgfpathlineto{\pgfqpoint{5.909160in}{1.995000in}}%
\pgfpathlineto{\pgfqpoint{5.910400in}{2.695000in}}%
\pgfpathlineto{\pgfqpoint{5.912880in}{2.345000in}}%
\pgfpathlineto{\pgfqpoint{5.914120in}{2.450000in}}%
\pgfpathlineto{\pgfqpoint{5.916600in}{2.870000in}}%
\pgfpathlineto{\pgfqpoint{5.917840in}{2.590000in}}%
\pgfpathlineto{\pgfqpoint{5.919080in}{2.800000in}}%
\pgfpathlineto{\pgfqpoint{5.920320in}{2.415000in}}%
\pgfpathlineto{\pgfqpoint{5.921560in}{2.555000in}}%
\pgfpathlineto{\pgfqpoint{5.922800in}{2.870000in}}%
\pgfpathlineto{\pgfqpoint{5.924040in}{2.625000in}}%
\pgfpathlineto{\pgfqpoint{5.925280in}{2.730000in}}%
\pgfpathlineto{\pgfqpoint{5.926520in}{2.660000in}}%
\pgfpathlineto{\pgfqpoint{5.927760in}{2.380000in}}%
\pgfpathlineto{\pgfqpoint{5.929000in}{3.045000in}}%
\pgfpathlineto{\pgfqpoint{5.931480in}{2.450000in}}%
\pgfpathlineto{\pgfqpoint{5.933960in}{2.100000in}}%
\pgfpathlineto{\pgfqpoint{5.935200in}{2.240000in}}%
\pgfpathlineto{\pgfqpoint{5.936440in}{1.995000in}}%
\pgfpathlineto{\pgfqpoint{5.938920in}{2.800000in}}%
\pgfpathlineto{\pgfqpoint{5.940160in}{2.730000in}}%
\pgfpathlineto{\pgfqpoint{5.941400in}{2.555000in}}%
\pgfpathlineto{\pgfqpoint{5.942640in}{2.205000in}}%
\pgfpathlineto{\pgfqpoint{5.943880in}{2.310000in}}%
\pgfpathlineto{\pgfqpoint{5.945120in}{2.205000in}}%
\pgfpathlineto{\pgfqpoint{5.946360in}{2.555000in}}%
\pgfpathlineto{\pgfqpoint{5.948840in}{2.205000in}}%
\pgfpathlineto{\pgfqpoint{5.950080in}{2.555000in}}%
\pgfpathlineto{\pgfqpoint{5.951320in}{2.345000in}}%
\pgfpathlineto{\pgfqpoint{5.953800in}{2.625000in}}%
\pgfpathlineto{\pgfqpoint{5.956280in}{2.520000in}}%
\pgfpathlineto{\pgfqpoint{5.957520in}{2.345000in}}%
\pgfpathlineto{\pgfqpoint{5.958760in}{2.660000in}}%
\pgfpathlineto{\pgfqpoint{5.960000in}{1.995000in}}%
\pgfpathlineto{\pgfqpoint{5.961240in}{2.415000in}}%
\pgfpathlineto{\pgfqpoint{5.962480in}{2.345000in}}%
\pgfpathlineto{\pgfqpoint{5.963720in}{2.590000in}}%
\pgfpathlineto{\pgfqpoint{5.964960in}{2.345000in}}%
\pgfpathlineto{\pgfqpoint{5.966200in}{2.660000in}}%
\pgfpathlineto{\pgfqpoint{5.967440in}{2.590000in}}%
\pgfpathlineto{\pgfqpoint{5.968680in}{2.345000in}}%
\pgfpathlineto{\pgfqpoint{5.969920in}{2.520000in}}%
\pgfpathlineto{\pgfqpoint{5.972400in}{2.240000in}}%
\pgfpathlineto{\pgfqpoint{5.973640in}{2.100000in}}%
\pgfpathlineto{\pgfqpoint{5.974880in}{2.380000in}}%
\pgfpathlineto{\pgfqpoint{5.976120in}{2.345000in}}%
\pgfpathlineto{\pgfqpoint{5.977360in}{2.240000in}}%
\pgfpathlineto{\pgfqpoint{5.978600in}{2.625000in}}%
\pgfpathlineto{\pgfqpoint{5.979840in}{2.275000in}}%
\pgfpathlineto{\pgfqpoint{5.981080in}{2.625000in}}%
\pgfpathlineto{\pgfqpoint{5.983560in}{2.345000in}}%
\pgfpathlineto{\pgfqpoint{5.984800in}{2.380000in}}%
\pgfpathlineto{\pgfqpoint{5.986040in}{2.485000in}}%
\pgfpathlineto{\pgfqpoint{5.987280in}{2.415000in}}%
\pgfpathlineto{\pgfqpoint{5.988520in}{2.030000in}}%
\pgfpathlineto{\pgfqpoint{5.989760in}{2.415000in}}%
\pgfpathlineto{\pgfqpoint{5.991000in}{2.240000in}}%
\pgfpathlineto{\pgfqpoint{5.992240in}{2.835000in}}%
\pgfpathlineto{\pgfqpoint{5.994720in}{1.855000in}}%
\pgfpathlineto{\pgfqpoint{5.995960in}{2.100000in}}%
\pgfpathlineto{\pgfqpoint{5.997200in}{1.960000in}}%
\pgfpathlineto{\pgfqpoint{5.998440in}{2.100000in}}%
\pgfpathlineto{\pgfqpoint{5.999680in}{2.485000in}}%
\pgfpathlineto{\pgfqpoint{6.000920in}{2.310000in}}%
\pgfpathlineto{\pgfqpoint{6.002160in}{2.555000in}}%
\pgfpathlineto{\pgfqpoint{6.003400in}{2.450000in}}%
\pgfpathlineto{\pgfqpoint{6.004640in}{2.520000in}}%
\pgfpathlineto{\pgfqpoint{6.007120in}{1.890000in}}%
\pgfpathlineto{\pgfqpoint{6.008360in}{1.995000in}}%
\pgfpathlineto{\pgfqpoint{6.010840in}{2.275000in}}%
\pgfpathlineto{\pgfqpoint{6.012080in}{2.240000in}}%
\pgfpathlineto{\pgfqpoint{6.014560in}{1.855000in}}%
\pgfpathlineto{\pgfqpoint{6.017040in}{2.765000in}}%
\pgfpathlineto{\pgfqpoint{6.018280in}{1.995000in}}%
\pgfpathlineto{\pgfqpoint{6.020760in}{2.485000in}}%
\pgfpathlineto{\pgfqpoint{6.022000in}{2.310000in}}%
\pgfpathlineto{\pgfqpoint{6.023240in}{2.345000in}}%
\pgfpathlineto{\pgfqpoint{6.024480in}{1.995000in}}%
\pgfpathlineto{\pgfqpoint{6.026960in}{2.485000in}}%
\pgfpathlineto{\pgfqpoint{6.029440in}{2.310000in}}%
\pgfpathlineto{\pgfqpoint{6.030680in}{2.520000in}}%
\pgfpathlineto{\pgfqpoint{6.033160in}{2.100000in}}%
\pgfpathlineto{\pgfqpoint{6.036880in}{2.730000in}}%
\pgfpathlineto{\pgfqpoint{6.039360in}{2.135000in}}%
\pgfpathlineto{\pgfqpoint{6.041840in}{2.940000in}}%
\pgfpathlineto{\pgfqpoint{6.043080in}{2.100000in}}%
\pgfpathlineto{\pgfqpoint{6.045560in}{2.555000in}}%
\pgfpathlineto{\pgfqpoint{6.048040in}{2.240000in}}%
\pgfpathlineto{\pgfqpoint{6.049280in}{2.310000in}}%
\pgfpathlineto{\pgfqpoint{6.050520in}{2.310000in}}%
\pgfpathlineto{\pgfqpoint{6.053000in}{2.065000in}}%
\pgfpathlineto{\pgfqpoint{6.055480in}{2.345000in}}%
\pgfpathlineto{\pgfqpoint{6.056720in}{2.625000in}}%
\pgfpathlineto{\pgfqpoint{6.059200in}{2.275000in}}%
\pgfpathlineto{\pgfqpoint{6.060440in}{2.170000in}}%
\pgfpathlineto{\pgfqpoint{6.061680in}{2.485000in}}%
\pgfpathlineto{\pgfqpoint{6.062920in}{2.275000in}}%
\pgfpathlineto{\pgfqpoint{6.064160in}{2.730000in}}%
\pgfpathlineto{\pgfqpoint{6.065400in}{2.100000in}}%
\pgfpathlineto{\pgfqpoint{6.066640in}{2.135000in}}%
\pgfpathlineto{\pgfqpoint{6.067880in}{2.555000in}}%
\pgfpathlineto{\pgfqpoint{6.069120in}{2.205000in}}%
\pgfpathlineto{\pgfqpoint{6.071600in}{2.205000in}}%
\pgfpathlineto{\pgfqpoint{6.072840in}{2.485000in}}%
\pgfpathlineto{\pgfqpoint{6.076560in}{2.135000in}}%
\pgfpathlineto{\pgfqpoint{6.077800in}{2.275000in}}%
\pgfpathlineto{\pgfqpoint{6.079040in}{2.275000in}}%
\pgfpathlineto{\pgfqpoint{6.080280in}{2.135000in}}%
\pgfpathlineto{\pgfqpoint{6.081520in}{2.625000in}}%
\pgfpathlineto{\pgfqpoint{6.082760in}{2.555000in}}%
\pgfpathlineto{\pgfqpoint{6.084000in}{2.660000in}}%
\pgfpathlineto{\pgfqpoint{6.086480in}{2.170000in}}%
\pgfpathlineto{\pgfqpoint{6.087720in}{2.520000in}}%
\pgfpathlineto{\pgfqpoint{6.088960in}{2.310000in}}%
\pgfpathlineto{\pgfqpoint{6.090200in}{2.310000in}}%
\pgfpathlineto{\pgfqpoint{6.091440in}{2.240000in}}%
\pgfpathlineto{\pgfqpoint{6.092680in}{2.625000in}}%
\pgfpathlineto{\pgfqpoint{6.093920in}{2.660000in}}%
\pgfpathlineto{\pgfqpoint{6.095160in}{2.240000in}}%
\pgfpathlineto{\pgfqpoint{6.096400in}{2.730000in}}%
\pgfpathlineto{\pgfqpoint{6.097640in}{2.415000in}}%
\pgfpathlineto{\pgfqpoint{6.098880in}{2.940000in}}%
\pgfpathlineto{\pgfqpoint{6.101360in}{2.415000in}}%
\pgfpathlineto{\pgfqpoint{6.102600in}{2.695000in}}%
\pgfpathlineto{\pgfqpoint{6.105080in}{2.450000in}}%
\pgfpathlineto{\pgfqpoint{6.106320in}{2.625000in}}%
\pgfpathlineto{\pgfqpoint{6.110040in}{2.240000in}}%
\pgfpathlineto{\pgfqpoint{6.111280in}{2.590000in}}%
\pgfpathlineto{\pgfqpoint{6.112520in}{2.345000in}}%
\pgfpathlineto{\pgfqpoint{6.113760in}{2.520000in}}%
\pgfpathlineto{\pgfqpoint{6.115000in}{2.240000in}}%
\pgfpathlineto{\pgfqpoint{6.116240in}{2.485000in}}%
\pgfpathlineto{\pgfqpoint{6.117480in}{2.170000in}}%
\pgfpathlineto{\pgfqpoint{6.118720in}{2.415000in}}%
\pgfpathlineto{\pgfqpoint{6.121200in}{2.240000in}}%
\pgfpathlineto{\pgfqpoint{6.123680in}{2.625000in}}%
\pgfpathlineto{\pgfqpoint{6.124920in}{2.485000in}}%
\pgfpathlineto{\pgfqpoint{6.126160in}{2.625000in}}%
\pgfpathlineto{\pgfqpoint{6.127400in}{2.905000in}}%
\pgfpathlineto{\pgfqpoint{6.129880in}{2.380000in}}%
\pgfpathlineto{\pgfqpoint{6.131120in}{2.380000in}}%
\pgfpathlineto{\pgfqpoint{6.132360in}{2.275000in}}%
\pgfpathlineto{\pgfqpoint{6.133600in}{2.590000in}}%
\pgfpathlineto{\pgfqpoint{6.134840in}{2.520000in}}%
\pgfpathlineto{\pgfqpoint{6.136080in}{2.730000in}}%
\pgfpathlineto{\pgfqpoint{6.138560in}{2.240000in}}%
\pgfpathlineto{\pgfqpoint{6.141040in}{2.590000in}}%
\pgfpathlineto{\pgfqpoint{6.142280in}{2.275000in}}%
\pgfpathlineto{\pgfqpoint{6.143520in}{2.380000in}}%
\pgfpathlineto{\pgfqpoint{6.146000in}{2.240000in}}%
\pgfpathlineto{\pgfqpoint{6.147240in}{2.485000in}}%
\pgfpathlineto{\pgfqpoint{6.149720in}{2.205000in}}%
\pgfpathlineto{\pgfqpoint{6.150960in}{2.205000in}}%
\pgfpathlineto{\pgfqpoint{6.153440in}{2.415000in}}%
\pgfpathlineto{\pgfqpoint{6.154680in}{2.415000in}}%
\pgfpathlineto{\pgfqpoint{6.155920in}{2.450000in}}%
\pgfpathlineto{\pgfqpoint{6.157160in}{2.555000in}}%
\pgfpathlineto{\pgfqpoint{6.158400in}{2.765000in}}%
\pgfpathlineto{\pgfqpoint{6.159640in}{2.730000in}}%
\pgfpathlineto{\pgfqpoint{6.160880in}{2.660000in}}%
\pgfpathlineto{\pgfqpoint{6.162120in}{2.240000in}}%
\pgfpathlineto{\pgfqpoint{6.164600in}{2.555000in}}%
\pgfpathlineto{\pgfqpoint{6.165840in}{2.555000in}}%
\pgfpathlineto{\pgfqpoint{6.167080in}{2.450000in}}%
\pgfpathlineto{\pgfqpoint{6.168320in}{2.625000in}}%
\pgfpathlineto{\pgfqpoint{6.169560in}{2.555000in}}%
\pgfpathlineto{\pgfqpoint{6.170800in}{2.590000in}}%
\pgfpathlineto{\pgfqpoint{6.172040in}{2.380000in}}%
\pgfpathlineto{\pgfqpoint{6.173280in}{2.485000in}}%
\pgfpathlineto{\pgfqpoint{6.174520in}{2.450000in}}%
\pgfpathlineto{\pgfqpoint{6.177000in}{2.520000in}}%
\pgfpathlineto{\pgfqpoint{6.178240in}{2.240000in}}%
\pgfpathlineto{\pgfqpoint{6.179480in}{2.625000in}}%
\pgfpathlineto{\pgfqpoint{6.180720in}{2.485000in}}%
\pgfpathlineto{\pgfqpoint{6.181960in}{2.135000in}}%
\pgfpathlineto{\pgfqpoint{6.183200in}{2.695000in}}%
\pgfpathlineto{\pgfqpoint{6.184440in}{2.590000in}}%
\pgfpathlineto{\pgfqpoint{6.185680in}{2.380000in}}%
\pgfpathlineto{\pgfqpoint{6.186920in}{2.695000in}}%
\pgfpathlineto{\pgfqpoint{6.189400in}{2.415000in}}%
\pgfpathlineto{\pgfqpoint{6.190640in}{2.450000in}}%
\pgfpathlineto{\pgfqpoint{6.191880in}{2.450000in}}%
\pgfpathlineto{\pgfqpoint{6.194360in}{2.240000in}}%
\pgfpathlineto{\pgfqpoint{6.195600in}{2.555000in}}%
\pgfpathlineto{\pgfqpoint{6.196840in}{2.415000in}}%
\pgfpathlineto{\pgfqpoint{6.199320in}{2.625000in}}%
\pgfpathlineto{\pgfqpoint{6.200560in}{2.310000in}}%
\pgfpathlineto{\pgfqpoint{6.201800in}{2.310000in}}%
\pgfpathlineto{\pgfqpoint{6.203040in}{2.205000in}}%
\pgfpathlineto{\pgfqpoint{6.205520in}{2.905000in}}%
\pgfpathlineto{\pgfqpoint{6.206760in}{2.555000in}}%
\pgfpathlineto{\pgfqpoint{6.208000in}{2.590000in}}%
\pgfpathlineto{\pgfqpoint{6.209240in}{2.240000in}}%
\pgfpathlineto{\pgfqpoint{6.210480in}{2.450000in}}%
\pgfpathlineto{\pgfqpoint{6.211720in}{2.135000in}}%
\pgfpathlineto{\pgfqpoint{6.212960in}{2.380000in}}%
\pgfpathlineto{\pgfqpoint{6.214200in}{2.135000in}}%
\pgfpathlineto{\pgfqpoint{6.216680in}{2.555000in}}%
\pgfpathlineto{\pgfqpoint{6.217920in}{2.415000in}}%
\pgfpathlineto{\pgfqpoint{6.219160in}{2.555000in}}%
\pgfpathlineto{\pgfqpoint{6.220400in}{2.205000in}}%
\pgfpathlineto{\pgfqpoint{6.221640in}{2.520000in}}%
\pgfpathlineto{\pgfqpoint{6.224120in}{2.240000in}}%
\pgfpathlineto{\pgfqpoint{6.226600in}{2.450000in}}%
\pgfpathlineto{\pgfqpoint{6.227840in}{2.205000in}}%
\pgfpathlineto{\pgfqpoint{6.229080in}{2.310000in}}%
\pgfpathlineto{\pgfqpoint{6.231560in}{2.625000in}}%
\pgfpathlineto{\pgfqpoint{6.232800in}{2.205000in}}%
\pgfpathlineto{\pgfqpoint{6.235280in}{2.310000in}}%
\pgfpathlineto{\pgfqpoint{6.236520in}{2.240000in}}%
\pgfpathlineto{\pgfqpoint{6.237760in}{2.415000in}}%
\pgfpathlineto{\pgfqpoint{6.239000in}{2.345000in}}%
\pgfpathlineto{\pgfqpoint{6.240240in}{2.450000in}}%
\pgfpathlineto{\pgfqpoint{6.241480in}{2.450000in}}%
\pgfpathlineto{\pgfqpoint{6.242720in}{2.170000in}}%
\pgfpathlineto{\pgfqpoint{6.245200in}{2.660000in}}%
\pgfpathlineto{\pgfqpoint{6.246440in}{2.520000in}}%
\pgfpathlineto{\pgfqpoint{6.247680in}{2.205000in}}%
\pgfpathlineto{\pgfqpoint{6.248920in}{2.345000in}}%
\pgfpathlineto{\pgfqpoint{6.250160in}{2.135000in}}%
\pgfpathlineto{\pgfqpoint{6.251400in}{2.310000in}}%
\pgfpathlineto{\pgfqpoint{6.252640in}{2.660000in}}%
\pgfpathlineto{\pgfqpoint{6.255120in}{2.135000in}}%
\pgfpathlineto{\pgfqpoint{6.256360in}{2.450000in}}%
\pgfpathlineto{\pgfqpoint{6.258840in}{2.170000in}}%
\pgfpathlineto{\pgfqpoint{6.261320in}{2.695000in}}%
\pgfpathlineto{\pgfqpoint{6.265040in}{2.135000in}}%
\pgfpathlineto{\pgfqpoint{6.267520in}{2.485000in}}%
\pgfpathlineto{\pgfqpoint{6.270000in}{2.415000in}}%
\pgfpathlineto{\pgfqpoint{6.271240in}{2.345000in}}%
\pgfpathlineto{\pgfqpoint{6.272480in}{2.625000in}}%
\pgfpathlineto{\pgfqpoint{6.273720in}{2.205000in}}%
\pgfpathlineto{\pgfqpoint{6.274960in}{2.695000in}}%
\pgfpathlineto{\pgfqpoint{6.276200in}{2.485000in}}%
\pgfpathlineto{\pgfqpoint{6.277440in}{2.765000in}}%
\pgfpathlineto{\pgfqpoint{6.278680in}{2.310000in}}%
\pgfpathlineto{\pgfqpoint{6.282400in}{2.695000in}}%
\pgfpathlineto{\pgfqpoint{6.283640in}{2.765000in}}%
\pgfpathlineto{\pgfqpoint{6.284880in}{2.240000in}}%
\pgfpathlineto{\pgfqpoint{6.286120in}{2.205000in}}%
\pgfpathlineto{\pgfqpoint{6.288600in}{2.380000in}}%
\pgfpathlineto{\pgfqpoint{6.289840in}{1.960000in}}%
\pgfpathlineto{\pgfqpoint{6.291080in}{2.415000in}}%
\pgfpathlineto{\pgfqpoint{6.292320in}{2.275000in}}%
\pgfpathlineto{\pgfqpoint{6.293560in}{2.275000in}}%
\pgfpathlineto{\pgfqpoint{6.294800in}{2.450000in}}%
\pgfpathlineto{\pgfqpoint{6.297280in}{2.275000in}}%
\pgfpathlineto{\pgfqpoint{6.298520in}{2.450000in}}%
\pgfpathlineto{\pgfqpoint{6.299760in}{1.785000in}}%
\pgfpathlineto{\pgfqpoint{6.301000in}{2.485000in}}%
\pgfpathlineto{\pgfqpoint{6.302240in}{2.170000in}}%
\pgfpathlineto{\pgfqpoint{6.303480in}{2.310000in}}%
\pgfpathlineto{\pgfqpoint{6.304720in}{2.135000in}}%
\pgfpathlineto{\pgfqpoint{6.305960in}{2.555000in}}%
\pgfpathlineto{\pgfqpoint{6.307200in}{2.520000in}}%
\pgfpathlineto{\pgfqpoint{6.308440in}{2.625000in}}%
\pgfpathlineto{\pgfqpoint{6.309680in}{2.555000in}}%
\pgfpathlineto{\pgfqpoint{6.310920in}{2.415000in}}%
\pgfpathlineto{\pgfqpoint{6.312160in}{2.625000in}}%
\pgfpathlineto{\pgfqpoint{6.313400in}{2.625000in}}%
\pgfpathlineto{\pgfqpoint{6.315880in}{2.240000in}}%
\pgfpathlineto{\pgfqpoint{6.320840in}{2.730000in}}%
\pgfpathlineto{\pgfqpoint{6.322080in}{2.695000in}}%
\pgfpathlineto{\pgfqpoint{6.323320in}{2.345000in}}%
\pgfpathlineto{\pgfqpoint{6.324560in}{2.765000in}}%
\pgfpathlineto{\pgfqpoint{6.325800in}{2.800000in}}%
\pgfpathlineto{\pgfqpoint{6.328280in}{2.345000in}}%
\pgfpathlineto{\pgfqpoint{6.330760in}{2.170000in}}%
\pgfpathlineto{\pgfqpoint{6.332000in}{2.485000in}}%
\pgfpathlineto{\pgfqpoint{6.333240in}{2.485000in}}%
\pgfpathlineto{\pgfqpoint{6.334480in}{2.205000in}}%
\pgfpathlineto{\pgfqpoint{6.335720in}{2.380000in}}%
\pgfpathlineto{\pgfqpoint{6.336960in}{1.960000in}}%
\pgfpathlineto{\pgfqpoint{6.339440in}{2.590000in}}%
\pgfpathlineto{\pgfqpoint{6.340680in}{2.590000in}}%
\pgfpathlineto{\pgfqpoint{6.343160in}{2.170000in}}%
\pgfpathlineto{\pgfqpoint{6.344400in}{2.240000in}}%
\pgfpathlineto{\pgfqpoint{6.345640in}{2.450000in}}%
\pgfpathlineto{\pgfqpoint{6.346880in}{2.275000in}}%
\pgfpathlineto{\pgfqpoint{6.348120in}{2.380000in}}%
\pgfpathlineto{\pgfqpoint{6.349360in}{2.590000in}}%
\pgfpathlineto{\pgfqpoint{6.351840in}{2.170000in}}%
\pgfpathlineto{\pgfqpoint{6.353080in}{2.555000in}}%
\pgfpathlineto{\pgfqpoint{6.358040in}{2.135000in}}%
\pgfpathlineto{\pgfqpoint{6.360520in}{2.520000in}}%
\pgfpathlineto{\pgfqpoint{6.361760in}{2.345000in}}%
\pgfpathlineto{\pgfqpoint{6.363000in}{2.485000in}}%
\pgfpathlineto{\pgfqpoint{6.364240in}{2.800000in}}%
\pgfpathlineto{\pgfqpoint{6.367960in}{2.100000in}}%
\pgfpathlineto{\pgfqpoint{6.369200in}{2.415000in}}%
\pgfpathlineto{\pgfqpoint{6.370440in}{2.380000in}}%
\pgfpathlineto{\pgfqpoint{6.371680in}{2.135000in}}%
\pgfpathlineto{\pgfqpoint{6.372920in}{2.345000in}}%
\pgfpathlineto{\pgfqpoint{6.374160in}{2.205000in}}%
\pgfpathlineto{\pgfqpoint{6.375400in}{2.660000in}}%
\pgfpathlineto{\pgfqpoint{6.376640in}{2.310000in}}%
\pgfpathlineto{\pgfqpoint{6.377880in}{2.450000in}}%
\pgfpathlineto{\pgfqpoint{6.379120in}{2.030000in}}%
\pgfpathlineto{\pgfqpoint{6.380360in}{2.695000in}}%
\pgfpathlineto{\pgfqpoint{6.381600in}{2.380000in}}%
\pgfpathlineto{\pgfqpoint{6.385320in}{2.730000in}}%
\pgfpathlineto{\pgfqpoint{6.387800in}{2.380000in}}%
\pgfpathlineto{\pgfqpoint{6.389040in}{2.555000in}}%
\pgfpathlineto{\pgfqpoint{6.390280in}{2.275000in}}%
\pgfpathlineto{\pgfqpoint{6.394000in}{2.660000in}}%
\pgfpathlineto{\pgfqpoint{6.395240in}{2.520000in}}%
\pgfpathlineto{\pgfqpoint{6.396480in}{2.555000in}}%
\pgfpathlineto{\pgfqpoint{6.397720in}{2.555000in}}%
\pgfpathlineto{\pgfqpoint{6.398960in}{2.905000in}}%
\pgfpathlineto{\pgfqpoint{6.402680in}{2.380000in}}%
\pgfpathlineto{\pgfqpoint{6.403920in}{2.695000in}}%
\pgfpathlineto{\pgfqpoint{6.406400in}{2.345000in}}%
\pgfpathlineto{\pgfqpoint{6.407640in}{2.415000in}}%
\pgfpathlineto{\pgfqpoint{6.408880in}{2.310000in}}%
\pgfpathlineto{\pgfqpoint{6.410120in}{2.380000in}}%
\pgfpathlineto{\pgfqpoint{6.411360in}{2.730000in}}%
\pgfpathlineto{\pgfqpoint{6.412600in}{2.590000in}}%
\pgfpathlineto{\pgfqpoint{6.413840in}{2.870000in}}%
\pgfpathlineto{\pgfqpoint{6.416320in}{2.450000in}}%
\pgfpathlineto{\pgfqpoint{6.417560in}{2.275000in}}%
\pgfpathlineto{\pgfqpoint{6.418800in}{2.275000in}}%
\pgfpathlineto{\pgfqpoint{6.420040in}{2.415000in}}%
\pgfpathlineto{\pgfqpoint{6.421280in}{2.205000in}}%
\pgfpathlineto{\pgfqpoint{6.422520in}{2.660000in}}%
\pgfpathlineto{\pgfqpoint{6.423760in}{2.660000in}}%
\pgfpathlineto{\pgfqpoint{6.425000in}{2.590000in}}%
\pgfpathlineto{\pgfqpoint{6.426240in}{2.240000in}}%
\pgfpathlineto{\pgfqpoint{6.427480in}{2.625000in}}%
\pgfpathlineto{\pgfqpoint{6.428720in}{2.485000in}}%
\pgfpathlineto{\pgfqpoint{6.429960in}{2.695000in}}%
\pgfpathlineto{\pgfqpoint{6.431200in}{2.380000in}}%
\pgfpathlineto{\pgfqpoint{6.432440in}{2.380000in}}%
\pgfpathlineto{\pgfqpoint{6.433680in}{2.450000in}}%
\pgfpathlineto{\pgfqpoint{6.434920in}{2.240000in}}%
\pgfpathlineto{\pgfqpoint{6.436160in}{2.450000in}}%
\pgfpathlineto{\pgfqpoint{6.437400in}{2.205000in}}%
\pgfpathlineto{\pgfqpoint{6.438640in}{2.590000in}}%
\pgfpathlineto{\pgfqpoint{6.439880in}{2.555000in}}%
\pgfpathlineto{\pgfqpoint{6.441120in}{2.275000in}}%
\pgfpathlineto{\pgfqpoint{6.442360in}{2.415000in}}%
\pgfpathlineto{\pgfqpoint{6.443600in}{2.135000in}}%
\pgfpathlineto{\pgfqpoint{6.444840in}{2.485000in}}%
\pgfpathlineto{\pgfqpoint{6.446080in}{2.310000in}}%
\pgfpathlineto{\pgfqpoint{6.447320in}{2.660000in}}%
\pgfpathlineto{\pgfqpoint{6.448560in}{2.625000in}}%
\pgfpathlineto{\pgfqpoint{6.449800in}{2.450000in}}%
\pgfpathlineto{\pgfqpoint{6.451040in}{2.695000in}}%
\pgfpathlineto{\pgfqpoint{6.453520in}{2.380000in}}%
\pgfpathlineto{\pgfqpoint{6.454760in}{2.345000in}}%
\pgfpathlineto{\pgfqpoint{6.456000in}{2.520000in}}%
\pgfpathlineto{\pgfqpoint{6.457240in}{2.065000in}}%
\pgfpathlineto{\pgfqpoint{6.458480in}{2.485000in}}%
\pgfpathlineto{\pgfqpoint{6.460960in}{2.100000in}}%
\pgfpathlineto{\pgfqpoint{6.462200in}{2.170000in}}%
\pgfpathlineto{\pgfqpoint{6.463440in}{2.380000in}}%
\pgfpathlineto{\pgfqpoint{6.464680in}{2.240000in}}%
\pgfpathlineto{\pgfqpoint{6.465920in}{2.520000in}}%
\pgfpathlineto{\pgfqpoint{6.467160in}{2.450000in}}%
\pgfpathlineto{\pgfqpoint{6.468400in}{2.660000in}}%
\pgfpathlineto{\pgfqpoint{6.469640in}{2.310000in}}%
\pgfpathlineto{\pgfqpoint{6.470880in}{2.450000in}}%
\pgfpathlineto{\pgfqpoint{6.472120in}{2.345000in}}%
\pgfpathlineto{\pgfqpoint{6.474600in}{2.485000in}}%
\pgfpathlineto{\pgfqpoint{6.475840in}{2.485000in}}%
\pgfpathlineto{\pgfqpoint{6.477080in}{2.730000in}}%
\pgfpathlineto{\pgfqpoint{6.478320in}{2.695000in}}%
\pgfpathlineto{\pgfqpoint{6.479560in}{2.240000in}}%
\pgfpathlineto{\pgfqpoint{6.480800in}{2.415000in}}%
\pgfpathlineto{\pgfqpoint{6.482040in}{2.415000in}}%
\pgfpathlineto{\pgfqpoint{6.483280in}{2.240000in}}%
\pgfpathlineto{\pgfqpoint{6.484520in}{2.590000in}}%
\pgfpathlineto{\pgfqpoint{6.485760in}{2.520000in}}%
\pgfpathlineto{\pgfqpoint{6.487000in}{2.555000in}}%
\pgfpathlineto{\pgfqpoint{6.488240in}{2.240000in}}%
\pgfpathlineto{\pgfqpoint{6.489480in}{2.450000in}}%
\pgfpathlineto{\pgfqpoint{6.490720in}{2.450000in}}%
\pgfpathlineto{\pgfqpoint{6.491960in}{2.520000in}}%
\pgfpathlineto{\pgfqpoint{6.493200in}{2.170000in}}%
\pgfpathlineto{\pgfqpoint{6.494440in}{2.275000in}}%
\pgfpathlineto{\pgfqpoint{6.495680in}{1.995000in}}%
\pgfpathlineto{\pgfqpoint{6.496920in}{2.345000in}}%
\pgfpathlineto{\pgfqpoint{6.498160in}{2.310000in}}%
\pgfpathlineto{\pgfqpoint{6.499400in}{2.450000in}}%
\pgfpathlineto{\pgfqpoint{6.500640in}{2.205000in}}%
\pgfpathlineto{\pgfqpoint{6.501880in}{2.240000in}}%
\pgfpathlineto{\pgfqpoint{6.503120in}{2.030000in}}%
\pgfpathlineto{\pgfqpoint{6.505600in}{2.205000in}}%
\pgfpathlineto{\pgfqpoint{6.506840in}{2.205000in}}%
\pgfpathlineto{\pgfqpoint{6.508080in}{2.415000in}}%
\pgfpathlineto{\pgfqpoint{6.510560in}{2.170000in}}%
\pgfpathlineto{\pgfqpoint{6.513040in}{2.520000in}}%
\pgfpathlineto{\pgfqpoint{6.514280in}{2.135000in}}%
\pgfpathlineto{\pgfqpoint{6.515520in}{2.380000in}}%
\pgfpathlineto{\pgfqpoint{6.516760in}{1.925000in}}%
\pgfpathlineto{\pgfqpoint{6.518000in}{2.310000in}}%
\pgfpathlineto{\pgfqpoint{6.519240in}{2.240000in}}%
\pgfpathlineto{\pgfqpoint{6.520480in}{2.275000in}}%
\pgfpathlineto{\pgfqpoint{6.521720in}{2.485000in}}%
\pgfpathlineto{\pgfqpoint{6.522960in}{2.275000in}}%
\pgfpathlineto{\pgfqpoint{6.525440in}{2.555000in}}%
\pgfpathlineto{\pgfqpoint{6.526680in}{2.135000in}}%
\pgfpathlineto{\pgfqpoint{6.527920in}{2.275000in}}%
\pgfpathlineto{\pgfqpoint{6.529160in}{2.205000in}}%
\pgfpathlineto{\pgfqpoint{6.530400in}{2.555000in}}%
\pgfpathlineto{\pgfqpoint{6.531640in}{2.380000in}}%
\pgfpathlineto{\pgfqpoint{6.532880in}{2.555000in}}%
\pgfpathlineto{\pgfqpoint{6.534120in}{2.310000in}}%
\pgfpathlineto{\pgfqpoint{6.535360in}{2.380000in}}%
\pgfpathlineto{\pgfqpoint{6.537840in}{2.100000in}}%
\pgfpathlineto{\pgfqpoint{6.539080in}{2.275000in}}%
\pgfpathlineto{\pgfqpoint{6.540320in}{2.275000in}}%
\pgfpathlineto{\pgfqpoint{6.541560in}{2.135000in}}%
\pgfpathlineto{\pgfqpoint{6.544040in}{2.415000in}}%
\pgfpathlineto{\pgfqpoint{6.545280in}{2.415000in}}%
\pgfpathlineto{\pgfqpoint{6.546520in}{2.275000in}}%
\pgfpathlineto{\pgfqpoint{6.547760in}{2.345000in}}%
\pgfpathlineto{\pgfqpoint{6.549000in}{2.275000in}}%
\pgfpathlineto{\pgfqpoint{6.550240in}{2.660000in}}%
\pgfpathlineto{\pgfqpoint{6.551480in}{2.170000in}}%
\pgfpathlineto{\pgfqpoint{6.552720in}{2.240000in}}%
\pgfpathlineto{\pgfqpoint{6.553960in}{2.170000in}}%
\pgfpathlineto{\pgfqpoint{6.556440in}{2.555000in}}%
\pgfpathlineto{\pgfqpoint{6.557680in}{1.890000in}}%
\pgfpathlineto{\pgfqpoint{6.558920in}{2.380000in}}%
\pgfpathlineto{\pgfqpoint{6.560160in}{2.310000in}}%
\pgfpathlineto{\pgfqpoint{6.561400in}{2.380000in}}%
\pgfpathlineto{\pgfqpoint{6.562640in}{2.555000in}}%
\pgfpathlineto{\pgfqpoint{6.563880in}{2.415000in}}%
\pgfpathlineto{\pgfqpoint{6.565120in}{2.415000in}}%
\pgfpathlineto{\pgfqpoint{6.566360in}{2.380000in}}%
\pgfpathlineto{\pgfqpoint{6.567600in}{2.310000in}}%
\pgfpathlineto{\pgfqpoint{6.568840in}{2.485000in}}%
\pgfpathlineto{\pgfqpoint{6.570080in}{2.835000in}}%
\pgfpathlineto{\pgfqpoint{6.571320in}{2.450000in}}%
\pgfpathlineto{\pgfqpoint{6.572560in}{2.520000in}}%
\pgfpathlineto{\pgfqpoint{6.573800in}{2.170000in}}%
\pgfpathlineto{\pgfqpoint{6.576280in}{2.345000in}}%
\pgfpathlineto{\pgfqpoint{6.577520in}{2.135000in}}%
\pgfpathlineto{\pgfqpoint{6.578760in}{2.240000in}}%
\pgfpathlineto{\pgfqpoint{6.580000in}{2.170000in}}%
\pgfpathlineto{\pgfqpoint{6.581240in}{1.995000in}}%
\pgfpathlineto{\pgfqpoint{6.582480in}{2.310000in}}%
\pgfpathlineto{\pgfqpoint{6.583720in}{2.205000in}}%
\pgfpathlineto{\pgfqpoint{6.584960in}{2.380000in}}%
\pgfpathlineto{\pgfqpoint{6.586200in}{2.310000in}}%
\pgfpathlineto{\pgfqpoint{6.588680in}{2.730000in}}%
\pgfpathlineto{\pgfqpoint{6.591160in}{2.380000in}}%
\pgfpathlineto{\pgfqpoint{6.592400in}{2.590000in}}%
\pgfpathlineto{\pgfqpoint{6.594880in}{2.240000in}}%
\pgfpathlineto{\pgfqpoint{6.596120in}{2.555000in}}%
\pgfpathlineto{\pgfqpoint{6.597360in}{2.520000in}}%
\pgfpathlineto{\pgfqpoint{6.598600in}{2.275000in}}%
\pgfpathlineto{\pgfqpoint{6.599840in}{2.310000in}}%
\pgfpathlineto{\pgfqpoint{6.601080in}{2.660000in}}%
\pgfpathlineto{\pgfqpoint{6.602320in}{2.555000in}}%
\pgfpathlineto{\pgfqpoint{6.603560in}{2.555000in}}%
\pgfpathlineto{\pgfqpoint{6.604800in}{2.380000in}}%
\pgfpathlineto{\pgfqpoint{6.606040in}{2.485000in}}%
\pgfpathlineto{\pgfqpoint{6.608520in}{2.135000in}}%
\pgfpathlineto{\pgfqpoint{6.609760in}{2.135000in}}%
\pgfpathlineto{\pgfqpoint{6.611000in}{2.240000in}}%
\pgfpathlineto{\pgfqpoint{6.612240in}{2.625000in}}%
\pgfpathlineto{\pgfqpoint{6.614720in}{2.205000in}}%
\pgfpathlineto{\pgfqpoint{6.615960in}{2.520000in}}%
\pgfpathlineto{\pgfqpoint{6.617200in}{1.995000in}}%
\pgfpathlineto{\pgfqpoint{6.619680in}{2.310000in}}%
\pgfpathlineto{\pgfqpoint{6.620920in}{2.310000in}}%
\pgfpathlineto{\pgfqpoint{6.622160in}{2.485000in}}%
\pgfpathlineto{\pgfqpoint{6.623400in}{2.450000in}}%
\pgfpathlineto{\pgfqpoint{6.624640in}{2.660000in}}%
\pgfpathlineto{\pgfqpoint{6.625880in}{2.310000in}}%
\pgfpathlineto{\pgfqpoint{6.627120in}{2.380000in}}%
\pgfpathlineto{\pgfqpoint{6.628360in}{2.625000in}}%
\pgfpathlineto{\pgfqpoint{6.629600in}{2.205000in}}%
\pgfpathlineto{\pgfqpoint{6.630840in}{2.345000in}}%
\pgfpathlineto{\pgfqpoint{6.632080in}{2.310000in}}%
\pgfpathlineto{\pgfqpoint{6.633320in}{2.520000in}}%
\pgfpathlineto{\pgfqpoint{6.634560in}{2.170000in}}%
\pgfpathlineto{\pgfqpoint{6.635800in}{2.170000in}}%
\pgfpathlineto{\pgfqpoint{6.637040in}{2.520000in}}%
\pgfpathlineto{\pgfqpoint{6.639520in}{2.240000in}}%
\pgfpathlineto{\pgfqpoint{6.640760in}{2.485000in}}%
\pgfpathlineto{\pgfqpoint{6.642000in}{2.275000in}}%
\pgfpathlineto{\pgfqpoint{6.643240in}{2.380000in}}%
\pgfpathlineto{\pgfqpoint{6.644480in}{2.310000in}}%
\pgfpathlineto{\pgfqpoint{6.645720in}{2.660000in}}%
\pgfpathlineto{\pgfqpoint{6.646960in}{2.695000in}}%
\pgfpathlineto{\pgfqpoint{6.649440in}{2.310000in}}%
\pgfpathlineto{\pgfqpoint{6.650680in}{2.205000in}}%
\pgfpathlineto{\pgfqpoint{6.653160in}{2.380000in}}%
\pgfpathlineto{\pgfqpoint{6.655640in}{2.100000in}}%
\pgfpathlineto{\pgfqpoint{6.658120in}{2.555000in}}%
\pgfpathlineto{\pgfqpoint{6.659360in}{2.555000in}}%
\pgfpathlineto{\pgfqpoint{6.660600in}{2.240000in}}%
\pgfpathlineto{\pgfqpoint{6.661840in}{2.660000in}}%
\pgfpathlineto{\pgfqpoint{6.663080in}{2.380000in}}%
\pgfpathlineto{\pgfqpoint{6.664320in}{2.415000in}}%
\pgfpathlineto{\pgfqpoint{6.665560in}{2.660000in}}%
\pgfpathlineto{\pgfqpoint{6.666800in}{2.310000in}}%
\pgfpathlineto{\pgfqpoint{6.669280in}{2.765000in}}%
\pgfpathlineto{\pgfqpoint{6.671760in}{2.170000in}}%
\pgfpathlineto{\pgfqpoint{6.673000in}{2.520000in}}%
\pgfpathlineto{\pgfqpoint{6.674240in}{2.520000in}}%
\pgfpathlineto{\pgfqpoint{6.676720in}{2.135000in}}%
\pgfpathlineto{\pgfqpoint{6.677960in}{2.345000in}}%
\pgfpathlineto{\pgfqpoint{6.680440in}{2.170000in}}%
\pgfpathlineto{\pgfqpoint{6.681680in}{2.660000in}}%
\pgfpathlineto{\pgfqpoint{6.682920in}{2.380000in}}%
\pgfpathlineto{\pgfqpoint{6.684160in}{2.590000in}}%
\pgfpathlineto{\pgfqpoint{6.685400in}{2.590000in}}%
\pgfpathlineto{\pgfqpoint{6.687880in}{2.415000in}}%
\pgfpathlineto{\pgfqpoint{6.689120in}{2.380000in}}%
\pgfpathlineto{\pgfqpoint{6.690360in}{2.450000in}}%
\pgfpathlineto{\pgfqpoint{6.691600in}{2.065000in}}%
\pgfpathlineto{\pgfqpoint{6.694080in}{2.625000in}}%
\pgfpathlineto{\pgfqpoint{6.696560in}{1.960000in}}%
\pgfpathlineto{\pgfqpoint{6.697800in}{2.380000in}}%
\pgfpathlineto{\pgfqpoint{6.699040in}{2.275000in}}%
\pgfpathlineto{\pgfqpoint{6.700280in}{2.310000in}}%
\pgfpathlineto{\pgfqpoint{6.701520in}{2.450000in}}%
\pgfpathlineto{\pgfqpoint{6.702760in}{2.380000in}}%
\pgfpathlineto{\pgfqpoint{6.704000in}{2.135000in}}%
\pgfpathlineto{\pgfqpoint{6.705240in}{2.205000in}}%
\pgfpathlineto{\pgfqpoint{6.707720in}{2.660000in}}%
\pgfpathlineto{\pgfqpoint{6.708960in}{2.450000in}}%
\pgfpathlineto{\pgfqpoint{6.710200in}{2.730000in}}%
\pgfpathlineto{\pgfqpoint{6.711440in}{2.415000in}}%
\pgfpathlineto{\pgfqpoint{6.712680in}{2.555000in}}%
\pgfpathlineto{\pgfqpoint{6.713920in}{2.380000in}}%
\pgfpathlineto{\pgfqpoint{6.715160in}{2.590000in}}%
\pgfpathlineto{\pgfqpoint{6.718880in}{2.345000in}}%
\pgfpathlineto{\pgfqpoint{6.720120in}{2.485000in}}%
\pgfpathlineto{\pgfqpoint{6.721360in}{2.485000in}}%
\pgfpathlineto{\pgfqpoint{6.722600in}{2.555000in}}%
\pgfpathlineto{\pgfqpoint{6.723840in}{2.765000in}}%
\pgfpathlineto{\pgfqpoint{6.725080in}{2.765000in}}%
\pgfpathlineto{\pgfqpoint{6.727560in}{2.345000in}}%
\pgfpathlineto{\pgfqpoint{6.728800in}{2.555000in}}%
\pgfpathlineto{\pgfqpoint{6.731280in}{1.960000in}}%
\pgfpathlineto{\pgfqpoint{6.733760in}{1.610000in}}%
\pgfpathlineto{\pgfqpoint{6.735000in}{2.275000in}}%
\pgfpathlineto{\pgfqpoint{6.736240in}{2.275000in}}%
\pgfpathlineto{\pgfqpoint{6.737480in}{2.065000in}}%
\pgfpathlineto{\pgfqpoint{6.738720in}{2.065000in}}%
\pgfpathlineto{\pgfqpoint{6.739960in}{2.135000in}}%
\pgfpathlineto{\pgfqpoint{6.742440in}{2.555000in}}%
\pgfpathlineto{\pgfqpoint{6.746160in}{2.240000in}}%
\pgfpathlineto{\pgfqpoint{6.747400in}{2.590000in}}%
\pgfpathlineto{\pgfqpoint{6.749880in}{2.660000in}}%
\pgfpathlineto{\pgfqpoint{6.751120in}{2.800000in}}%
\pgfpathlineto{\pgfqpoint{6.753600in}{2.345000in}}%
\pgfpathlineto{\pgfqpoint{6.754840in}{2.905000in}}%
\pgfpathlineto{\pgfqpoint{6.756080in}{2.450000in}}%
\pgfpathlineto{\pgfqpoint{6.757320in}{2.730000in}}%
\pgfpathlineto{\pgfqpoint{6.758560in}{2.660000in}}%
\pgfpathlineto{\pgfqpoint{6.759800in}{2.660000in}}%
\pgfpathlineto{\pgfqpoint{6.761040in}{2.625000in}}%
\pgfpathlineto{\pgfqpoint{6.762280in}{2.625000in}}%
\pgfpathlineto{\pgfqpoint{6.763520in}{2.905000in}}%
\pgfpathlineto{\pgfqpoint{6.764760in}{2.345000in}}%
\pgfpathlineto{\pgfqpoint{6.766000in}{2.555000in}}%
\pgfpathlineto{\pgfqpoint{6.767240in}{2.065000in}}%
\pgfpathlineto{\pgfqpoint{6.769720in}{2.660000in}}%
\pgfpathlineto{\pgfqpoint{6.770960in}{2.800000in}}%
\pgfpathlineto{\pgfqpoint{6.772200in}{2.415000in}}%
\pgfpathlineto{\pgfqpoint{6.773440in}{2.625000in}}%
\pgfpathlineto{\pgfqpoint{6.774680in}{2.240000in}}%
\pgfpathlineto{\pgfqpoint{6.775920in}{2.485000in}}%
\pgfpathlineto{\pgfqpoint{6.778400in}{2.170000in}}%
\pgfpathlineto{\pgfqpoint{6.779640in}{2.695000in}}%
\pgfpathlineto{\pgfqpoint{6.780880in}{2.730000in}}%
\pgfpathlineto{\pgfqpoint{6.782120in}{2.345000in}}%
\pgfpathlineto{\pgfqpoint{6.783360in}{2.590000in}}%
\pgfpathlineto{\pgfqpoint{6.784600in}{2.310000in}}%
\pgfpathlineto{\pgfqpoint{6.785840in}{2.415000in}}%
\pgfpathlineto{\pgfqpoint{6.787080in}{2.415000in}}%
\pgfpathlineto{\pgfqpoint{6.788320in}{2.905000in}}%
\pgfpathlineto{\pgfqpoint{6.792040in}{2.275000in}}%
\pgfpathlineto{\pgfqpoint{6.793280in}{2.870000in}}%
\pgfpathlineto{\pgfqpoint{6.794520in}{2.660000in}}%
\pgfpathlineto{\pgfqpoint{6.795760in}{2.800000in}}%
\pgfpathlineto{\pgfqpoint{6.797000in}{2.695000in}}%
\pgfpathlineto{\pgfqpoint{6.798240in}{2.310000in}}%
\pgfpathlineto{\pgfqpoint{6.799480in}{2.380000in}}%
\pgfpathlineto{\pgfqpoint{6.800720in}{2.765000in}}%
\pgfpathlineto{\pgfqpoint{6.801960in}{2.380000in}}%
\pgfpathlineto{\pgfqpoint{6.803200in}{2.415000in}}%
\pgfpathlineto{\pgfqpoint{6.804440in}{2.730000in}}%
\pgfpathlineto{\pgfqpoint{6.805680in}{2.695000in}}%
\pgfpathlineto{\pgfqpoint{6.806920in}{2.345000in}}%
\pgfpathlineto{\pgfqpoint{6.809400in}{2.940000in}}%
\pgfpathlineto{\pgfqpoint{6.810640in}{2.800000in}}%
\pgfpathlineto{\pgfqpoint{6.813120in}{2.170000in}}%
\pgfpathlineto{\pgfqpoint{6.814360in}{2.485000in}}%
\pgfpathlineto{\pgfqpoint{6.815600in}{2.450000in}}%
\pgfpathlineto{\pgfqpoint{6.816840in}{2.310000in}}%
\pgfpathlineto{\pgfqpoint{6.818080in}{2.450000in}}%
\pgfpathlineto{\pgfqpoint{6.819320in}{2.380000in}}%
\pgfpathlineto{\pgfqpoint{6.820560in}{2.415000in}}%
\pgfpathlineto{\pgfqpoint{6.821800in}{2.485000in}}%
\pgfpathlineto{\pgfqpoint{6.823040in}{2.170000in}}%
\pgfpathlineto{\pgfqpoint{6.824280in}{2.485000in}}%
\pgfpathlineto{\pgfqpoint{6.825520in}{2.100000in}}%
\pgfpathlineto{\pgfqpoint{6.828000in}{2.730000in}}%
\pgfpathlineto{\pgfqpoint{6.829240in}{2.275000in}}%
\pgfpathlineto{\pgfqpoint{6.830480in}{2.450000in}}%
\pgfpathlineto{\pgfqpoint{6.831720in}{2.240000in}}%
\pgfpathlineto{\pgfqpoint{6.832960in}{2.695000in}}%
\pgfpathlineto{\pgfqpoint{6.834200in}{1.925000in}}%
\pgfpathlineto{\pgfqpoint{6.835440in}{2.485000in}}%
\pgfpathlineto{\pgfqpoint{6.836680in}{2.135000in}}%
\pgfpathlineto{\pgfqpoint{6.837920in}{2.485000in}}%
\pgfpathlineto{\pgfqpoint{6.839160in}{2.310000in}}%
\pgfpathlineto{\pgfqpoint{6.840400in}{2.345000in}}%
\pgfpathlineto{\pgfqpoint{6.842880in}{2.835000in}}%
\pgfpathlineto{\pgfqpoint{6.846600in}{2.030000in}}%
\pgfpathlineto{\pgfqpoint{6.850320in}{2.800000in}}%
\pgfpathlineto{\pgfqpoint{6.851560in}{2.275000in}}%
\pgfpathlineto{\pgfqpoint{6.852800in}{2.310000in}}%
\pgfpathlineto{\pgfqpoint{6.854040in}{2.800000in}}%
\pgfpathlineto{\pgfqpoint{6.855280in}{2.310000in}}%
\pgfpathlineto{\pgfqpoint{6.857760in}{2.590000in}}%
\pgfpathlineto{\pgfqpoint{6.859000in}{2.380000in}}%
\pgfpathlineto{\pgfqpoint{6.860240in}{2.555000in}}%
\pgfpathlineto{\pgfqpoint{6.861480in}{2.415000in}}%
\pgfpathlineto{\pgfqpoint{6.862720in}{2.485000in}}%
\pgfpathlineto{\pgfqpoint{6.863960in}{2.205000in}}%
\pgfpathlineto{\pgfqpoint{6.866440in}{2.730000in}}%
\pgfpathlineto{\pgfqpoint{6.867680in}{2.485000in}}%
\pgfpathlineto{\pgfqpoint{6.868920in}{2.520000in}}%
\pgfpathlineto{\pgfqpoint{6.870160in}{2.205000in}}%
\pgfpathlineto{\pgfqpoint{6.871400in}{2.205000in}}%
\pgfpathlineto{\pgfqpoint{6.872640in}{2.415000in}}%
\pgfpathlineto{\pgfqpoint{6.873880in}{2.415000in}}%
\pgfpathlineto{\pgfqpoint{6.875120in}{2.275000in}}%
\pgfpathlineto{\pgfqpoint{6.877600in}{2.275000in}}%
\pgfpathlineto{\pgfqpoint{6.881320in}{2.870000in}}%
\pgfpathlineto{\pgfqpoint{6.885040in}{2.135000in}}%
\pgfpathlineto{\pgfqpoint{6.887520in}{2.590000in}}%
\pgfpathlineto{\pgfqpoint{6.891240in}{2.485000in}}%
\pgfpathlineto{\pgfqpoint{6.892480in}{2.765000in}}%
\pgfpathlineto{\pgfqpoint{6.893720in}{2.520000in}}%
\pgfpathlineto{\pgfqpoint{6.894960in}{1.995000in}}%
\pgfpathlineto{\pgfqpoint{6.897440in}{2.380000in}}%
\pgfpathlineto{\pgfqpoint{6.898680in}{2.135000in}}%
\pgfpathlineto{\pgfqpoint{6.899920in}{2.555000in}}%
\pgfpathlineto{\pgfqpoint{6.901160in}{2.450000in}}%
\pgfpathlineto{\pgfqpoint{6.902400in}{2.730000in}}%
\pgfpathlineto{\pgfqpoint{6.903640in}{2.555000in}}%
\pgfpathlineto{\pgfqpoint{6.904880in}{2.660000in}}%
\pgfpathlineto{\pgfqpoint{6.906120in}{2.625000in}}%
\pgfpathlineto{\pgfqpoint{6.908600in}{2.240000in}}%
\pgfpathlineto{\pgfqpoint{6.909840in}{2.625000in}}%
\pgfpathlineto{\pgfqpoint{6.912320in}{2.520000in}}%
\pgfpathlineto{\pgfqpoint{6.913560in}{2.800000in}}%
\pgfpathlineto{\pgfqpoint{6.914800in}{2.485000in}}%
\pgfpathlineto{\pgfqpoint{6.916040in}{2.625000in}}%
\pgfpathlineto{\pgfqpoint{6.917280in}{2.275000in}}%
\pgfpathlineto{\pgfqpoint{6.918520in}{2.905000in}}%
\pgfpathlineto{\pgfqpoint{6.919760in}{2.205000in}}%
\pgfpathlineto{\pgfqpoint{6.922240in}{2.660000in}}%
\pgfpathlineto{\pgfqpoint{6.923480in}{2.555000in}}%
\pgfpathlineto{\pgfqpoint{6.924720in}{2.940000in}}%
\pgfpathlineto{\pgfqpoint{6.925960in}{2.625000in}}%
\pgfpathlineto{\pgfqpoint{6.928440in}{3.080000in}}%
\pgfpathlineto{\pgfqpoint{6.929680in}{2.520000in}}%
\pgfpathlineto{\pgfqpoint{6.930920in}{2.520000in}}%
\pgfpathlineto{\pgfqpoint{6.932160in}{2.310000in}}%
\pgfpathlineto{\pgfqpoint{6.933400in}{2.590000in}}%
\pgfpathlineto{\pgfqpoint{6.934640in}{2.275000in}}%
\pgfpathlineto{\pgfqpoint{6.935880in}{2.310000in}}%
\pgfpathlineto{\pgfqpoint{6.937120in}{2.205000in}}%
\pgfpathlineto{\pgfqpoint{6.940840in}{2.765000in}}%
\pgfpathlineto{\pgfqpoint{6.943320in}{2.520000in}}%
\pgfpathlineto{\pgfqpoint{6.944560in}{2.520000in}}%
\pgfpathlineto{\pgfqpoint{6.945800in}{2.485000in}}%
\pgfpathlineto{\pgfqpoint{6.947040in}{2.835000in}}%
\pgfpathlineto{\pgfqpoint{6.948280in}{2.415000in}}%
\pgfpathlineto{\pgfqpoint{6.949520in}{2.380000in}}%
\pgfpathlineto{\pgfqpoint{6.950760in}{2.695000in}}%
\pgfpathlineto{\pgfqpoint{6.952000in}{2.590000in}}%
\pgfpathlineto{\pgfqpoint{6.953240in}{2.205000in}}%
\pgfpathlineto{\pgfqpoint{6.954480in}{2.625000in}}%
\pgfpathlineto{\pgfqpoint{6.956960in}{2.415000in}}%
\pgfpathlineto{\pgfqpoint{6.959440in}{2.660000in}}%
\pgfpathlineto{\pgfqpoint{6.961920in}{2.135000in}}%
\pgfpathlineto{\pgfqpoint{6.963160in}{2.415000in}}%
\pgfpathlineto{\pgfqpoint{6.964400in}{2.415000in}}%
\pgfpathlineto{\pgfqpoint{6.965640in}{2.625000in}}%
\pgfpathlineto{\pgfqpoint{6.966880in}{2.415000in}}%
\pgfpathlineto{\pgfqpoint{6.968120in}{2.520000in}}%
\pgfpathlineto{\pgfqpoint{6.969360in}{2.345000in}}%
\pgfpathlineto{\pgfqpoint{6.970600in}{2.660000in}}%
\pgfpathlineto{\pgfqpoint{6.971840in}{2.415000in}}%
\pgfpathlineto{\pgfqpoint{6.973080in}{2.975000in}}%
\pgfpathlineto{\pgfqpoint{6.974320in}{2.555000in}}%
\pgfpathlineto{\pgfqpoint{6.975560in}{2.625000in}}%
\pgfpathlineto{\pgfqpoint{6.976800in}{2.870000in}}%
\pgfpathlineto{\pgfqpoint{6.978040in}{2.660000in}}%
\pgfpathlineto{\pgfqpoint{6.979280in}{2.800000in}}%
\pgfpathlineto{\pgfqpoint{6.981760in}{2.485000in}}%
\pgfpathlineto{\pgfqpoint{6.983000in}{2.485000in}}%
\pgfpathlineto{\pgfqpoint{6.984240in}{2.205000in}}%
\pgfpathlineto{\pgfqpoint{6.986720in}{2.730000in}}%
\pgfpathlineto{\pgfqpoint{6.987960in}{2.345000in}}%
\pgfpathlineto{\pgfqpoint{6.990440in}{2.695000in}}%
\pgfpathlineto{\pgfqpoint{6.992920in}{2.415000in}}%
\pgfpathlineto{\pgfqpoint{6.994160in}{2.380000in}}%
\pgfpathlineto{\pgfqpoint{6.995400in}{2.205000in}}%
\pgfpathlineto{\pgfqpoint{6.999120in}{2.625000in}}%
\pgfpathlineto{\pgfqpoint{7.000360in}{2.625000in}}%
\pgfpathlineto{\pgfqpoint{7.002840in}{2.345000in}}%
\pgfpathlineto{\pgfqpoint{7.004080in}{2.625000in}}%
\pgfpathlineto{\pgfqpoint{7.005320in}{2.240000in}}%
\pgfpathlineto{\pgfqpoint{7.006560in}{2.520000in}}%
\pgfpathlineto{\pgfqpoint{7.007800in}{2.415000in}}%
\pgfpathlineto{\pgfqpoint{7.009040in}{2.625000in}}%
\pgfpathlineto{\pgfqpoint{7.010280in}{2.345000in}}%
\pgfpathlineto{\pgfqpoint{7.011520in}{2.380000in}}%
\pgfpathlineto{\pgfqpoint{7.014000in}{2.555000in}}%
\pgfpathlineto{\pgfqpoint{7.015240in}{2.590000in}}%
\pgfpathlineto{\pgfqpoint{7.016480in}{2.205000in}}%
\pgfpathlineto{\pgfqpoint{7.017720in}{2.520000in}}%
\pgfpathlineto{\pgfqpoint{7.018960in}{2.450000in}}%
\pgfpathlineto{\pgfqpoint{7.020200in}{2.240000in}}%
\pgfpathlineto{\pgfqpoint{7.022680in}{2.450000in}}%
\pgfpathlineto{\pgfqpoint{7.023920in}{2.555000in}}%
\pgfpathlineto{\pgfqpoint{7.025160in}{2.415000in}}%
\pgfpathlineto{\pgfqpoint{7.027640in}{2.835000in}}%
\pgfpathlineto{\pgfqpoint{7.030120in}{2.695000in}}%
\pgfpathlineto{\pgfqpoint{7.031360in}{2.695000in}}%
\pgfpathlineto{\pgfqpoint{7.032600in}{2.520000in}}%
\pgfpathlineto{\pgfqpoint{7.033840in}{2.765000in}}%
\pgfpathlineto{\pgfqpoint{7.035080in}{2.695000in}}%
\pgfpathlineto{\pgfqpoint{7.036320in}{2.940000in}}%
\pgfpathlineto{\pgfqpoint{7.038800in}{2.310000in}}%
\pgfpathlineto{\pgfqpoint{7.041280in}{2.625000in}}%
\pgfpathlineto{\pgfqpoint{7.042520in}{2.590000in}}%
\pgfpathlineto{\pgfqpoint{7.043760in}{2.275000in}}%
\pgfpathlineto{\pgfqpoint{7.045000in}{2.660000in}}%
\pgfpathlineto{\pgfqpoint{7.046240in}{2.520000in}}%
\pgfpathlineto{\pgfqpoint{7.047480in}{2.625000in}}%
\pgfpathlineto{\pgfqpoint{7.049960in}{2.275000in}}%
\pgfpathlineto{\pgfqpoint{7.051200in}{2.520000in}}%
\pgfpathlineto{\pgfqpoint{7.052440in}{2.485000in}}%
\pgfpathlineto{\pgfqpoint{7.053680in}{2.415000in}}%
\pgfpathlineto{\pgfqpoint{7.054920in}{2.590000in}}%
\pgfpathlineto{\pgfqpoint{7.056160in}{2.380000in}}%
\pgfpathlineto{\pgfqpoint{7.057400in}{2.520000in}}%
\pgfpathlineto{\pgfqpoint{7.058640in}{2.450000in}}%
\pgfpathlineto{\pgfqpoint{7.059880in}{2.310000in}}%
\pgfpathlineto{\pgfqpoint{7.061120in}{2.310000in}}%
\pgfpathlineto{\pgfqpoint{7.062360in}{2.205000in}}%
\pgfpathlineto{\pgfqpoint{7.063600in}{2.380000in}}%
\pgfpathlineto{\pgfqpoint{7.064840in}{2.240000in}}%
\pgfpathlineto{\pgfqpoint{7.066080in}{2.485000in}}%
\pgfpathlineto{\pgfqpoint{7.067320in}{1.995000in}}%
\pgfpathlineto{\pgfqpoint{7.069800in}{2.345000in}}%
\pgfpathlineto{\pgfqpoint{7.071040in}{2.450000in}}%
\pgfpathlineto{\pgfqpoint{7.073520in}{2.030000in}}%
\pgfpathlineto{\pgfqpoint{7.074760in}{1.995000in}}%
\pgfpathlineto{\pgfqpoint{7.076000in}{2.520000in}}%
\pgfpathlineto{\pgfqpoint{7.077240in}{2.520000in}}%
\pgfpathlineto{\pgfqpoint{7.078480in}{2.835000in}}%
\pgfpathlineto{\pgfqpoint{7.079720in}{2.380000in}}%
\pgfpathlineto{\pgfqpoint{7.080960in}{2.345000in}}%
\pgfpathlineto{\pgfqpoint{7.082200in}{2.765000in}}%
\pgfpathlineto{\pgfqpoint{7.083440in}{2.450000in}}%
\pgfpathlineto{\pgfqpoint{7.084680in}{2.660000in}}%
\pgfpathlineto{\pgfqpoint{7.085920in}{1.925000in}}%
\pgfpathlineto{\pgfqpoint{7.087160in}{2.695000in}}%
\pgfpathlineto{\pgfqpoint{7.088400in}{2.275000in}}%
\pgfpathlineto{\pgfqpoint{7.089640in}{2.485000in}}%
\pgfpathlineto{\pgfqpoint{7.090880in}{2.240000in}}%
\pgfpathlineto{\pgfqpoint{7.092120in}{2.590000in}}%
\pgfpathlineto{\pgfqpoint{7.093360in}{2.555000in}}%
\pgfpathlineto{\pgfqpoint{7.094600in}{2.240000in}}%
\pgfpathlineto{\pgfqpoint{7.097080in}{2.415000in}}%
\pgfpathlineto{\pgfqpoint{7.099560in}{2.170000in}}%
\pgfpathlineto{\pgfqpoint{7.100800in}{2.240000in}}%
\pgfpathlineto{\pgfqpoint{7.102040in}{2.030000in}}%
\pgfpathlineto{\pgfqpoint{7.103280in}{2.380000in}}%
\pgfpathlineto{\pgfqpoint{7.104520in}{2.345000in}}%
\pgfpathlineto{\pgfqpoint{7.107000in}{2.065000in}}%
\pgfpathlineto{\pgfqpoint{7.109480in}{2.590000in}}%
\pgfpathlineto{\pgfqpoint{7.110720in}{2.310000in}}%
\pgfpathlineto{\pgfqpoint{7.111960in}{2.380000in}}%
\pgfpathlineto{\pgfqpoint{7.113200in}{2.100000in}}%
\pgfpathlineto{\pgfqpoint{7.114440in}{2.800000in}}%
\pgfpathlineto{\pgfqpoint{7.116920in}{2.380000in}}%
\pgfpathlineto{\pgfqpoint{7.118160in}{2.730000in}}%
\pgfpathlineto{\pgfqpoint{7.119400in}{2.240000in}}%
\pgfpathlineto{\pgfqpoint{7.120640in}{2.450000in}}%
\pgfpathlineto{\pgfqpoint{7.121880in}{2.310000in}}%
\pgfpathlineto{\pgfqpoint{7.124360in}{2.835000in}}%
\pgfpathlineto{\pgfqpoint{7.125600in}{2.380000in}}%
\pgfpathlineto{\pgfqpoint{7.126840in}{2.765000in}}%
\pgfpathlineto{\pgfqpoint{7.128080in}{2.660000in}}%
\pgfpathlineto{\pgfqpoint{7.129320in}{2.380000in}}%
\pgfpathlineto{\pgfqpoint{7.131800in}{2.450000in}}%
\pgfpathlineto{\pgfqpoint{7.134280in}{2.275000in}}%
\pgfpathlineto{\pgfqpoint{7.135520in}{2.485000in}}%
\pgfpathlineto{\pgfqpoint{7.136760in}{2.275000in}}%
\pgfpathlineto{\pgfqpoint{7.138000in}{2.380000in}}%
\pgfpathlineto{\pgfqpoint{7.139240in}{2.625000in}}%
\pgfpathlineto{\pgfqpoint{7.140480in}{2.380000in}}%
\pgfpathlineto{\pgfqpoint{7.141720in}{2.415000in}}%
\pgfpathlineto{\pgfqpoint{7.142960in}{1.995000in}}%
\pgfpathlineto{\pgfqpoint{7.146680in}{2.765000in}}%
\pgfpathlineto{\pgfqpoint{7.149160in}{2.555000in}}%
\pgfpathlineto{\pgfqpoint{7.150400in}{2.660000in}}%
\pgfpathlineto{\pgfqpoint{7.152880in}{2.170000in}}%
\pgfpathlineto{\pgfqpoint{7.155360in}{2.590000in}}%
\pgfpathlineto{\pgfqpoint{7.156600in}{2.275000in}}%
\pgfpathlineto{\pgfqpoint{7.160320in}{2.660000in}}%
\pgfpathlineto{\pgfqpoint{7.161560in}{2.485000in}}%
\pgfpathlineto{\pgfqpoint{7.164040in}{2.730000in}}%
\pgfpathlineto{\pgfqpoint{7.167760in}{2.135000in}}%
\pgfpathlineto{\pgfqpoint{7.170240in}{2.450000in}}%
\pgfpathlineto{\pgfqpoint{7.171480in}{2.275000in}}%
\pgfpathlineto{\pgfqpoint{7.172720in}{2.380000in}}%
\pgfpathlineto{\pgfqpoint{7.173960in}{1.890000in}}%
\pgfpathlineto{\pgfqpoint{7.176440in}{2.485000in}}%
\pgfpathlineto{\pgfqpoint{7.177680in}{2.450000in}}%
\pgfpathlineto{\pgfqpoint{7.178920in}{2.170000in}}%
\pgfpathlineto{\pgfqpoint{7.181400in}{2.380000in}}%
\pgfpathlineto{\pgfqpoint{7.182640in}{2.695000in}}%
\pgfpathlineto{\pgfqpoint{7.183880in}{2.520000in}}%
\pgfpathlineto{\pgfqpoint{7.185120in}{2.625000in}}%
\pgfpathlineto{\pgfqpoint{7.186360in}{2.590000in}}%
\pgfpathlineto{\pgfqpoint{7.187600in}{2.135000in}}%
\pgfpathlineto{\pgfqpoint{7.190080in}{2.485000in}}%
\pgfpathlineto{\pgfqpoint{7.191320in}{2.625000in}}%
\pgfpathlineto{\pgfqpoint{7.192560in}{2.240000in}}%
\pgfpathlineto{\pgfqpoint{7.193800in}{2.275000in}}%
\pgfpathlineto{\pgfqpoint{7.195040in}{1.960000in}}%
\pgfpathlineto{\pgfqpoint{7.196280in}{2.415000in}}%
\pgfpathlineto{\pgfqpoint{7.198760in}{2.415000in}}%
\pgfpathlineto{\pgfqpoint{7.200000in}{2.240000in}}%
\pgfpathlineto{\pgfqpoint{7.200000in}{2.240000in}}%
\pgfusepath{stroke}%
\end{pgfscope}%
\begin{pgfscope}%
\pgfpathrectangle{\pgfqpoint{1.000000in}{0.350000in}}{\pgfqpoint{6.200000in}{2.800000in}} %
\pgfusepath{clip}%
\pgfsetrectcap%
\pgfsetroundjoin%
\pgfsetlinewidth{1.003750pt}%
\definecolor{currentstroke}{rgb}{1.000000,0.000000,0.000000}%
\pgfsetstrokecolor{currentstroke}%
\pgfsetdash{}{0pt}%
\pgfpathmoveto{\pgfqpoint{1.000000in}{2.555000in}}%
\pgfpathlineto{\pgfqpoint{1.001240in}{2.065000in}}%
\pgfpathlineto{\pgfqpoint{1.003720in}{2.135000in}}%
\pgfpathlineto{\pgfqpoint{1.006200in}{1.645000in}}%
\pgfpathlineto{\pgfqpoint{1.007440in}{1.750000in}}%
\pgfpathlineto{\pgfqpoint{1.009920in}{1.400000in}}%
\pgfpathlineto{\pgfqpoint{1.013640in}{1.715000in}}%
\pgfpathlineto{\pgfqpoint{1.014880in}{1.645000in}}%
\pgfpathlineto{\pgfqpoint{1.016120in}{1.855000in}}%
\pgfpathlineto{\pgfqpoint{1.019840in}{1.120000in}}%
\pgfpathlineto{\pgfqpoint{1.021080in}{1.540000in}}%
\pgfpathlineto{\pgfqpoint{1.022320in}{1.400000in}}%
\pgfpathlineto{\pgfqpoint{1.023560in}{1.400000in}}%
\pgfpathlineto{\pgfqpoint{1.024800in}{1.190000in}}%
\pgfpathlineto{\pgfqpoint{1.026040in}{1.295000in}}%
\pgfpathlineto{\pgfqpoint{1.027280in}{1.260000in}}%
\pgfpathlineto{\pgfqpoint{1.028520in}{1.365000in}}%
\pgfpathlineto{\pgfqpoint{1.029760in}{1.295000in}}%
\pgfpathlineto{\pgfqpoint{1.031000in}{1.155000in}}%
\pgfpathlineto{\pgfqpoint{1.032240in}{1.715000in}}%
\pgfpathlineto{\pgfqpoint{1.033480in}{1.715000in}}%
\pgfpathlineto{\pgfqpoint{1.034720in}{1.330000in}}%
\pgfpathlineto{\pgfqpoint{1.037200in}{1.540000in}}%
\pgfpathlineto{\pgfqpoint{1.038440in}{1.505000in}}%
\pgfpathlineto{\pgfqpoint{1.039680in}{1.680000in}}%
\pgfpathlineto{\pgfqpoint{1.040920in}{1.435000in}}%
\pgfpathlineto{\pgfqpoint{1.042160in}{1.645000in}}%
\pgfpathlineto{\pgfqpoint{1.043400in}{1.645000in}}%
\pgfpathlineto{\pgfqpoint{1.044640in}{1.505000in}}%
\pgfpathlineto{\pgfqpoint{1.045880in}{1.715000in}}%
\pgfpathlineto{\pgfqpoint{1.047120in}{1.400000in}}%
\pgfpathlineto{\pgfqpoint{1.048360in}{1.680000in}}%
\pgfpathlineto{\pgfqpoint{1.049600in}{1.645000in}}%
\pgfpathlineto{\pgfqpoint{1.052080in}{1.050000in}}%
\pgfpathlineto{\pgfqpoint{1.054560in}{1.610000in}}%
\pgfpathlineto{\pgfqpoint{1.057040in}{0.945000in}}%
\pgfpathlineto{\pgfqpoint{1.058280in}{1.540000in}}%
\pgfpathlineto{\pgfqpoint{1.059520in}{1.400000in}}%
\pgfpathlineto{\pgfqpoint{1.060760in}{2.065000in}}%
\pgfpathlineto{\pgfqpoint{1.062000in}{1.365000in}}%
\pgfpathlineto{\pgfqpoint{1.063240in}{1.575000in}}%
\pgfpathlineto{\pgfqpoint{1.064480in}{1.190000in}}%
\pgfpathlineto{\pgfqpoint{1.065720in}{1.575000in}}%
\pgfpathlineto{\pgfqpoint{1.066960in}{1.085000in}}%
\pgfpathlineto{\pgfqpoint{1.070680in}{1.750000in}}%
\pgfpathlineto{\pgfqpoint{1.071920in}{1.540000in}}%
\pgfpathlineto{\pgfqpoint{1.073160in}{1.610000in}}%
\pgfpathlineto{\pgfqpoint{1.074400in}{1.190000in}}%
\pgfpathlineto{\pgfqpoint{1.075640in}{1.890000in}}%
\pgfpathlineto{\pgfqpoint{1.076880in}{1.295000in}}%
\pgfpathlineto{\pgfqpoint{1.078120in}{1.365000in}}%
\pgfpathlineto{\pgfqpoint{1.079360in}{1.155000in}}%
\pgfpathlineto{\pgfqpoint{1.080600in}{1.295000in}}%
\pgfpathlineto{\pgfqpoint{1.081840in}{1.890000in}}%
\pgfpathlineto{\pgfqpoint{1.083080in}{1.435000in}}%
\pgfpathlineto{\pgfqpoint{1.084320in}{1.400000in}}%
\pgfpathlineto{\pgfqpoint{1.085560in}{1.400000in}}%
\pgfpathlineto{\pgfqpoint{1.086800in}{1.435000in}}%
\pgfpathlineto{\pgfqpoint{1.088040in}{1.750000in}}%
\pgfpathlineto{\pgfqpoint{1.089280in}{1.400000in}}%
\pgfpathlineto{\pgfqpoint{1.091760in}{1.715000in}}%
\pgfpathlineto{\pgfqpoint{1.094240in}{0.805000in}}%
\pgfpathlineto{\pgfqpoint{1.096720in}{1.785000in}}%
\pgfpathlineto{\pgfqpoint{1.097960in}{0.980000in}}%
\pgfpathlineto{\pgfqpoint{1.100440in}{1.400000in}}%
\pgfpathlineto{\pgfqpoint{1.101680in}{0.560000in}}%
\pgfpathlineto{\pgfqpoint{1.102920in}{1.470000in}}%
\pgfpathlineto{\pgfqpoint{1.105400in}{1.260000in}}%
\pgfpathlineto{\pgfqpoint{1.106640in}{1.435000in}}%
\pgfpathlineto{\pgfqpoint{1.107880in}{1.050000in}}%
\pgfpathlineto{\pgfqpoint{1.109120in}{1.365000in}}%
\pgfpathlineto{\pgfqpoint{1.110360in}{1.365000in}}%
\pgfpathlineto{\pgfqpoint{1.111600in}{1.050000in}}%
\pgfpathlineto{\pgfqpoint{1.112840in}{1.540000in}}%
\pgfpathlineto{\pgfqpoint{1.114080in}{1.505000in}}%
\pgfpathlineto{\pgfqpoint{1.115320in}{1.820000in}}%
\pgfpathlineto{\pgfqpoint{1.119040in}{1.050000in}}%
\pgfpathlineto{\pgfqpoint{1.120280in}{1.260000in}}%
\pgfpathlineto{\pgfqpoint{1.121520in}{1.120000in}}%
\pgfpathlineto{\pgfqpoint{1.122760in}{1.225000in}}%
\pgfpathlineto{\pgfqpoint{1.125240in}{0.840000in}}%
\pgfpathlineto{\pgfqpoint{1.127720in}{1.295000in}}%
\pgfpathlineto{\pgfqpoint{1.128960in}{1.575000in}}%
\pgfpathlineto{\pgfqpoint{1.131440in}{0.945000in}}%
\pgfpathlineto{\pgfqpoint{1.132680in}{1.540000in}}%
\pgfpathlineto{\pgfqpoint{1.133920in}{1.295000in}}%
\pgfpathlineto{\pgfqpoint{1.135160in}{1.645000in}}%
\pgfpathlineto{\pgfqpoint{1.137640in}{1.435000in}}%
\pgfpathlineto{\pgfqpoint{1.138880in}{1.400000in}}%
\pgfpathlineto{\pgfqpoint{1.140120in}{1.610000in}}%
\pgfpathlineto{\pgfqpoint{1.141360in}{1.470000in}}%
\pgfpathlineto{\pgfqpoint{1.142600in}{1.680000in}}%
\pgfpathlineto{\pgfqpoint{1.143840in}{1.190000in}}%
\pgfpathlineto{\pgfqpoint{1.146320in}{1.365000in}}%
\pgfpathlineto{\pgfqpoint{1.147560in}{1.155000in}}%
\pgfpathlineto{\pgfqpoint{1.148800in}{1.785000in}}%
\pgfpathlineto{\pgfqpoint{1.150040in}{1.225000in}}%
\pgfpathlineto{\pgfqpoint{1.151280in}{1.470000in}}%
\pgfpathlineto{\pgfqpoint{1.153760in}{1.050000in}}%
\pgfpathlineto{\pgfqpoint{1.155000in}{1.400000in}}%
\pgfpathlineto{\pgfqpoint{1.156240in}{1.400000in}}%
\pgfpathlineto{\pgfqpoint{1.157480in}{1.365000in}}%
\pgfpathlineto{\pgfqpoint{1.158720in}{1.050000in}}%
\pgfpathlineto{\pgfqpoint{1.159960in}{1.365000in}}%
\pgfpathlineto{\pgfqpoint{1.162440in}{1.365000in}}%
\pgfpathlineto{\pgfqpoint{1.163680in}{1.820000in}}%
\pgfpathlineto{\pgfqpoint{1.164920in}{1.540000in}}%
\pgfpathlineto{\pgfqpoint{1.166160in}{2.170000in}}%
\pgfpathlineto{\pgfqpoint{1.167400in}{0.945000in}}%
\pgfpathlineto{\pgfqpoint{1.168640in}{1.750000in}}%
\pgfpathlineto{\pgfqpoint{1.169880in}{1.400000in}}%
\pgfpathlineto{\pgfqpoint{1.171120in}{1.470000in}}%
\pgfpathlineto{\pgfqpoint{1.173600in}{1.785000in}}%
\pgfpathlineto{\pgfqpoint{1.176080in}{1.330000in}}%
\pgfpathlineto{\pgfqpoint{1.177320in}{1.505000in}}%
\pgfpathlineto{\pgfqpoint{1.178560in}{0.805000in}}%
\pgfpathlineto{\pgfqpoint{1.179800in}{1.120000in}}%
\pgfpathlineto{\pgfqpoint{1.181040in}{1.050000in}}%
\pgfpathlineto{\pgfqpoint{1.182280in}{1.470000in}}%
\pgfpathlineto{\pgfqpoint{1.183520in}{1.470000in}}%
\pgfpathlineto{\pgfqpoint{1.184760in}{1.890000in}}%
\pgfpathlineto{\pgfqpoint{1.187240in}{1.295000in}}%
\pgfpathlineto{\pgfqpoint{1.188480in}{1.435000in}}%
\pgfpathlineto{\pgfqpoint{1.189720in}{1.260000in}}%
\pgfpathlineto{\pgfqpoint{1.192200in}{1.435000in}}%
\pgfpathlineto{\pgfqpoint{1.193440in}{1.225000in}}%
\pgfpathlineto{\pgfqpoint{1.194680in}{2.030000in}}%
\pgfpathlineto{\pgfqpoint{1.197160in}{1.470000in}}%
\pgfpathlineto{\pgfqpoint{1.198400in}{1.365000in}}%
\pgfpathlineto{\pgfqpoint{1.199640in}{1.155000in}}%
\pgfpathlineto{\pgfqpoint{1.200880in}{1.225000in}}%
\pgfpathlineto{\pgfqpoint{1.202120in}{1.610000in}}%
\pgfpathlineto{\pgfqpoint{1.204600in}{1.155000in}}%
\pgfpathlineto{\pgfqpoint{1.207080in}{1.505000in}}%
\pgfpathlineto{\pgfqpoint{1.209560in}{1.785000in}}%
\pgfpathlineto{\pgfqpoint{1.210800in}{1.435000in}}%
\pgfpathlineto{\pgfqpoint{1.213280in}{1.925000in}}%
\pgfpathlineto{\pgfqpoint{1.214520in}{1.400000in}}%
\pgfpathlineto{\pgfqpoint{1.215760in}{1.890000in}}%
\pgfpathlineto{\pgfqpoint{1.217000in}{1.295000in}}%
\pgfpathlineto{\pgfqpoint{1.219480in}{2.030000in}}%
\pgfpathlineto{\pgfqpoint{1.220720in}{1.365000in}}%
\pgfpathlineto{\pgfqpoint{1.221960in}{1.715000in}}%
\pgfpathlineto{\pgfqpoint{1.224440in}{1.505000in}}%
\pgfpathlineto{\pgfqpoint{1.225680in}{1.645000in}}%
\pgfpathlineto{\pgfqpoint{1.226920in}{1.400000in}}%
\pgfpathlineto{\pgfqpoint{1.228160in}{1.750000in}}%
\pgfpathlineto{\pgfqpoint{1.229400in}{1.610000in}}%
\pgfpathlineto{\pgfqpoint{1.230640in}{1.785000in}}%
\pgfpathlineto{\pgfqpoint{1.231880in}{1.750000in}}%
\pgfpathlineto{\pgfqpoint{1.233120in}{1.680000in}}%
\pgfpathlineto{\pgfqpoint{1.234360in}{1.365000in}}%
\pgfpathlineto{\pgfqpoint{1.235600in}{1.750000in}}%
\pgfpathlineto{\pgfqpoint{1.236840in}{1.155000in}}%
\pgfpathlineto{\pgfqpoint{1.239320in}{1.435000in}}%
\pgfpathlineto{\pgfqpoint{1.240560in}{1.225000in}}%
\pgfpathlineto{\pgfqpoint{1.241800in}{1.225000in}}%
\pgfpathlineto{\pgfqpoint{1.243040in}{1.155000in}}%
\pgfpathlineto{\pgfqpoint{1.244280in}{1.260000in}}%
\pgfpathlineto{\pgfqpoint{1.245520in}{1.505000in}}%
\pgfpathlineto{\pgfqpoint{1.246760in}{1.470000in}}%
\pgfpathlineto{\pgfqpoint{1.248000in}{1.750000in}}%
\pgfpathlineto{\pgfqpoint{1.249240in}{1.260000in}}%
\pgfpathlineto{\pgfqpoint{1.250480in}{1.505000in}}%
\pgfpathlineto{\pgfqpoint{1.251720in}{1.400000in}}%
\pgfpathlineto{\pgfqpoint{1.252960in}{1.400000in}}%
\pgfpathlineto{\pgfqpoint{1.254200in}{1.680000in}}%
\pgfpathlineto{\pgfqpoint{1.256680in}{1.435000in}}%
\pgfpathlineto{\pgfqpoint{1.257920in}{1.575000in}}%
\pgfpathlineto{\pgfqpoint{1.259160in}{1.400000in}}%
\pgfpathlineto{\pgfqpoint{1.260400in}{1.540000in}}%
\pgfpathlineto{\pgfqpoint{1.261640in}{2.065000in}}%
\pgfpathlineto{\pgfqpoint{1.262880in}{1.365000in}}%
\pgfpathlineto{\pgfqpoint{1.264120in}{1.330000in}}%
\pgfpathlineto{\pgfqpoint{1.265360in}{1.680000in}}%
\pgfpathlineto{\pgfqpoint{1.266600in}{1.435000in}}%
\pgfpathlineto{\pgfqpoint{1.267840in}{1.540000in}}%
\pgfpathlineto{\pgfqpoint{1.269080in}{1.260000in}}%
\pgfpathlineto{\pgfqpoint{1.270320in}{1.435000in}}%
\pgfpathlineto{\pgfqpoint{1.272800in}{0.875000in}}%
\pgfpathlineto{\pgfqpoint{1.275280in}{1.470000in}}%
\pgfpathlineto{\pgfqpoint{1.276520in}{0.840000in}}%
\pgfpathlineto{\pgfqpoint{1.277760in}{1.540000in}}%
\pgfpathlineto{\pgfqpoint{1.279000in}{1.260000in}}%
\pgfpathlineto{\pgfqpoint{1.281480in}{1.540000in}}%
\pgfpathlineto{\pgfqpoint{1.283960in}{1.225000in}}%
\pgfpathlineto{\pgfqpoint{1.285200in}{1.575000in}}%
\pgfpathlineto{\pgfqpoint{1.286440in}{1.470000in}}%
\pgfpathlineto{\pgfqpoint{1.287680in}{1.680000in}}%
\pgfpathlineto{\pgfqpoint{1.288920in}{1.470000in}}%
\pgfpathlineto{\pgfqpoint{1.290160in}{1.785000in}}%
\pgfpathlineto{\pgfqpoint{1.291400in}{1.295000in}}%
\pgfpathlineto{\pgfqpoint{1.292640in}{1.680000in}}%
\pgfpathlineto{\pgfqpoint{1.297600in}{1.050000in}}%
\pgfpathlineto{\pgfqpoint{1.298840in}{1.330000in}}%
\pgfpathlineto{\pgfqpoint{1.300080in}{1.015000in}}%
\pgfpathlineto{\pgfqpoint{1.302560in}{1.260000in}}%
\pgfpathlineto{\pgfqpoint{1.305040in}{1.680000in}}%
\pgfpathlineto{\pgfqpoint{1.306280in}{1.400000in}}%
\pgfpathlineto{\pgfqpoint{1.307520in}{1.680000in}}%
\pgfpathlineto{\pgfqpoint{1.308760in}{1.085000in}}%
\pgfpathlineto{\pgfqpoint{1.311240in}{1.575000in}}%
\pgfpathlineto{\pgfqpoint{1.312480in}{1.295000in}}%
\pgfpathlineto{\pgfqpoint{1.313720in}{1.400000in}}%
\pgfpathlineto{\pgfqpoint{1.314960in}{1.400000in}}%
\pgfpathlineto{\pgfqpoint{1.316200in}{1.435000in}}%
\pgfpathlineto{\pgfqpoint{1.317440in}{1.050000in}}%
\pgfpathlineto{\pgfqpoint{1.318680in}{1.295000in}}%
\pgfpathlineto{\pgfqpoint{1.319920in}{0.980000in}}%
\pgfpathlineto{\pgfqpoint{1.321160in}{1.190000in}}%
\pgfpathlineto{\pgfqpoint{1.322400in}{1.015000in}}%
\pgfpathlineto{\pgfqpoint{1.323640in}{1.190000in}}%
\pgfpathlineto{\pgfqpoint{1.324880in}{1.645000in}}%
\pgfpathlineto{\pgfqpoint{1.327360in}{1.190000in}}%
\pgfpathlineto{\pgfqpoint{1.329840in}{1.995000in}}%
\pgfpathlineto{\pgfqpoint{1.331080in}{1.750000in}}%
\pgfpathlineto{\pgfqpoint{1.332320in}{1.190000in}}%
\pgfpathlineto{\pgfqpoint{1.333560in}{1.225000in}}%
\pgfpathlineto{\pgfqpoint{1.334800in}{1.855000in}}%
\pgfpathlineto{\pgfqpoint{1.336040in}{1.645000in}}%
\pgfpathlineto{\pgfqpoint{1.337280in}{1.715000in}}%
\pgfpathlineto{\pgfqpoint{1.339760in}{1.400000in}}%
\pgfpathlineto{\pgfqpoint{1.341000in}{1.715000in}}%
\pgfpathlineto{\pgfqpoint{1.343480in}{1.400000in}}%
\pgfpathlineto{\pgfqpoint{1.344720in}{1.400000in}}%
\pgfpathlineto{\pgfqpoint{1.345960in}{1.435000in}}%
\pgfpathlineto{\pgfqpoint{1.347200in}{1.575000in}}%
\pgfpathlineto{\pgfqpoint{1.349680in}{1.225000in}}%
\pgfpathlineto{\pgfqpoint{1.350920in}{1.610000in}}%
\pgfpathlineto{\pgfqpoint{1.352160in}{1.540000in}}%
\pgfpathlineto{\pgfqpoint{1.354640in}{2.100000in}}%
\pgfpathlineto{\pgfqpoint{1.357120in}{1.365000in}}%
\pgfpathlineto{\pgfqpoint{1.358360in}{1.960000in}}%
\pgfpathlineto{\pgfqpoint{1.359600in}{1.645000in}}%
\pgfpathlineto{\pgfqpoint{1.360840in}{1.925000in}}%
\pgfpathlineto{\pgfqpoint{1.362080in}{1.890000in}}%
\pgfpathlineto{\pgfqpoint{1.363320in}{0.910000in}}%
\pgfpathlineto{\pgfqpoint{1.364560in}{1.680000in}}%
\pgfpathlineto{\pgfqpoint{1.368280in}{1.260000in}}%
\pgfpathlineto{\pgfqpoint{1.369520in}{1.260000in}}%
\pgfpathlineto{\pgfqpoint{1.372000in}{1.680000in}}%
\pgfpathlineto{\pgfqpoint{1.373240in}{1.330000in}}%
\pgfpathlineto{\pgfqpoint{1.376960in}{1.890000in}}%
\pgfpathlineto{\pgfqpoint{1.379440in}{1.295000in}}%
\pgfpathlineto{\pgfqpoint{1.380680in}{1.190000in}}%
\pgfpathlineto{\pgfqpoint{1.381920in}{1.610000in}}%
\pgfpathlineto{\pgfqpoint{1.383160in}{0.805000in}}%
\pgfpathlineto{\pgfqpoint{1.385640in}{1.645000in}}%
\pgfpathlineto{\pgfqpoint{1.386880in}{1.540000in}}%
\pgfpathlineto{\pgfqpoint{1.389360in}{1.050000in}}%
\pgfpathlineto{\pgfqpoint{1.390600in}{1.330000in}}%
\pgfpathlineto{\pgfqpoint{1.391840in}{1.330000in}}%
\pgfpathlineto{\pgfqpoint{1.394320in}{1.435000in}}%
\pgfpathlineto{\pgfqpoint{1.395560in}{1.260000in}}%
\pgfpathlineto{\pgfqpoint{1.396800in}{1.785000in}}%
\pgfpathlineto{\pgfqpoint{1.399280in}{1.295000in}}%
\pgfpathlineto{\pgfqpoint{1.400520in}{1.050000in}}%
\pgfpathlineto{\pgfqpoint{1.403000in}{1.505000in}}%
\pgfpathlineto{\pgfqpoint{1.404240in}{1.050000in}}%
\pgfpathlineto{\pgfqpoint{1.406720in}{1.505000in}}%
\pgfpathlineto{\pgfqpoint{1.407960in}{0.980000in}}%
\pgfpathlineto{\pgfqpoint{1.410440in}{1.505000in}}%
\pgfpathlineto{\pgfqpoint{1.412920in}{1.365000in}}%
\pgfpathlineto{\pgfqpoint{1.415400in}{1.995000in}}%
\pgfpathlineto{\pgfqpoint{1.416640in}{1.470000in}}%
\pgfpathlineto{\pgfqpoint{1.417880in}{1.610000in}}%
\pgfpathlineto{\pgfqpoint{1.419120in}{1.190000in}}%
\pgfpathlineto{\pgfqpoint{1.420360in}{1.575000in}}%
\pgfpathlineto{\pgfqpoint{1.421600in}{1.610000in}}%
\pgfpathlineto{\pgfqpoint{1.422840in}{1.295000in}}%
\pgfpathlineto{\pgfqpoint{1.424080in}{1.645000in}}%
\pgfpathlineto{\pgfqpoint{1.425320in}{1.365000in}}%
\pgfpathlineto{\pgfqpoint{1.426560in}{1.400000in}}%
\pgfpathlineto{\pgfqpoint{1.427800in}{1.470000in}}%
\pgfpathlineto{\pgfqpoint{1.429040in}{1.435000in}}%
\pgfpathlineto{\pgfqpoint{1.430280in}{1.540000in}}%
\pgfpathlineto{\pgfqpoint{1.431520in}{1.505000in}}%
\pgfpathlineto{\pgfqpoint{1.432760in}{1.435000in}}%
\pgfpathlineto{\pgfqpoint{1.435240in}{1.750000in}}%
\pgfpathlineto{\pgfqpoint{1.437720in}{1.365000in}}%
\pgfpathlineto{\pgfqpoint{1.438960in}{1.785000in}}%
\pgfpathlineto{\pgfqpoint{1.440200in}{1.015000in}}%
\pgfpathlineto{\pgfqpoint{1.441440in}{1.540000in}}%
\pgfpathlineto{\pgfqpoint{1.442680in}{1.435000in}}%
\pgfpathlineto{\pgfqpoint{1.445160in}{1.015000in}}%
\pgfpathlineto{\pgfqpoint{1.447640in}{1.715000in}}%
\pgfpathlineto{\pgfqpoint{1.448880in}{1.190000in}}%
\pgfpathlineto{\pgfqpoint{1.450120in}{1.330000in}}%
\pgfpathlineto{\pgfqpoint{1.451360in}{1.330000in}}%
\pgfpathlineto{\pgfqpoint{1.455080in}{1.050000in}}%
\pgfpathlineto{\pgfqpoint{1.456320in}{1.540000in}}%
\pgfpathlineto{\pgfqpoint{1.457560in}{1.225000in}}%
\pgfpathlineto{\pgfqpoint{1.458800in}{1.750000in}}%
\pgfpathlineto{\pgfqpoint{1.461280in}{1.365000in}}%
\pgfpathlineto{\pgfqpoint{1.462520in}{1.295000in}}%
\pgfpathlineto{\pgfqpoint{1.463760in}{1.330000in}}%
\pgfpathlineto{\pgfqpoint{1.465000in}{1.120000in}}%
\pgfpathlineto{\pgfqpoint{1.466240in}{1.260000in}}%
\pgfpathlineto{\pgfqpoint{1.467480in}{1.260000in}}%
\pgfpathlineto{\pgfqpoint{1.468720in}{1.645000in}}%
\pgfpathlineto{\pgfqpoint{1.471200in}{0.875000in}}%
\pgfpathlineto{\pgfqpoint{1.474920in}{1.645000in}}%
\pgfpathlineto{\pgfqpoint{1.476160in}{1.295000in}}%
\pgfpathlineto{\pgfqpoint{1.477400in}{1.470000in}}%
\pgfpathlineto{\pgfqpoint{1.478640in}{1.855000in}}%
\pgfpathlineto{\pgfqpoint{1.479880in}{1.400000in}}%
\pgfpathlineto{\pgfqpoint{1.481120in}{1.890000in}}%
\pgfpathlineto{\pgfqpoint{1.482360in}{1.295000in}}%
\pgfpathlineto{\pgfqpoint{1.483600in}{1.610000in}}%
\pgfpathlineto{\pgfqpoint{1.484840in}{1.540000in}}%
\pgfpathlineto{\pgfqpoint{1.486080in}{1.540000in}}%
\pgfpathlineto{\pgfqpoint{1.487320in}{1.645000in}}%
\pgfpathlineto{\pgfqpoint{1.488560in}{1.610000in}}%
\pgfpathlineto{\pgfqpoint{1.489800in}{1.470000in}}%
\pgfpathlineto{\pgfqpoint{1.491040in}{1.470000in}}%
\pgfpathlineto{\pgfqpoint{1.493520in}{1.715000in}}%
\pgfpathlineto{\pgfqpoint{1.494760in}{1.715000in}}%
\pgfpathlineto{\pgfqpoint{1.496000in}{1.470000in}}%
\pgfpathlineto{\pgfqpoint{1.498480in}{1.610000in}}%
\pgfpathlineto{\pgfqpoint{1.499720in}{1.925000in}}%
\pgfpathlineto{\pgfqpoint{1.500960in}{1.435000in}}%
\pgfpathlineto{\pgfqpoint{1.502200in}{1.400000in}}%
\pgfpathlineto{\pgfqpoint{1.503440in}{1.400000in}}%
\pgfpathlineto{\pgfqpoint{1.504680in}{1.575000in}}%
\pgfpathlineto{\pgfqpoint{1.505920in}{2.415000in}}%
\pgfpathlineto{\pgfqpoint{1.507160in}{1.365000in}}%
\pgfpathlineto{\pgfqpoint{1.509640in}{1.645000in}}%
\pgfpathlineto{\pgfqpoint{1.510880in}{1.680000in}}%
\pgfpathlineto{\pgfqpoint{1.512120in}{1.365000in}}%
\pgfpathlineto{\pgfqpoint{1.514600in}{1.505000in}}%
\pgfpathlineto{\pgfqpoint{1.515840in}{1.295000in}}%
\pgfpathlineto{\pgfqpoint{1.517080in}{1.680000in}}%
\pgfpathlineto{\pgfqpoint{1.518320in}{1.470000in}}%
\pgfpathlineto{\pgfqpoint{1.519560in}{1.855000in}}%
\pgfpathlineto{\pgfqpoint{1.522040in}{1.505000in}}%
\pgfpathlineto{\pgfqpoint{1.523280in}{1.575000in}}%
\pgfpathlineto{\pgfqpoint{1.524520in}{1.575000in}}%
\pgfpathlineto{\pgfqpoint{1.527000in}{1.155000in}}%
\pgfpathlineto{\pgfqpoint{1.528240in}{1.540000in}}%
\pgfpathlineto{\pgfqpoint{1.529480in}{1.260000in}}%
\pgfpathlineto{\pgfqpoint{1.530720in}{1.295000in}}%
\pgfpathlineto{\pgfqpoint{1.531960in}{1.470000in}}%
\pgfpathlineto{\pgfqpoint{1.534440in}{1.155000in}}%
\pgfpathlineto{\pgfqpoint{1.535680in}{1.225000in}}%
\pgfpathlineto{\pgfqpoint{1.536920in}{1.400000in}}%
\pgfpathlineto{\pgfqpoint{1.538160in}{1.295000in}}%
\pgfpathlineto{\pgfqpoint{1.539400in}{1.400000in}}%
\pgfpathlineto{\pgfqpoint{1.540640in}{1.365000in}}%
\pgfpathlineto{\pgfqpoint{1.541880in}{1.295000in}}%
\pgfpathlineto{\pgfqpoint{1.543120in}{1.085000in}}%
\pgfpathlineto{\pgfqpoint{1.544360in}{1.645000in}}%
\pgfpathlineto{\pgfqpoint{1.546840in}{1.120000in}}%
\pgfpathlineto{\pgfqpoint{1.548080in}{1.295000in}}%
\pgfpathlineto{\pgfqpoint{1.549320in}{2.100000in}}%
\pgfpathlineto{\pgfqpoint{1.550560in}{1.610000in}}%
\pgfpathlineto{\pgfqpoint{1.551800in}{1.645000in}}%
\pgfpathlineto{\pgfqpoint{1.553040in}{1.575000in}}%
\pgfpathlineto{\pgfqpoint{1.554280in}{1.365000in}}%
\pgfpathlineto{\pgfqpoint{1.556760in}{1.785000in}}%
\pgfpathlineto{\pgfqpoint{1.558000in}{1.050000in}}%
\pgfpathlineto{\pgfqpoint{1.559240in}{1.260000in}}%
\pgfpathlineto{\pgfqpoint{1.560480in}{0.805000in}}%
\pgfpathlineto{\pgfqpoint{1.562960in}{1.190000in}}%
\pgfpathlineto{\pgfqpoint{1.564200in}{1.190000in}}%
\pgfpathlineto{\pgfqpoint{1.565440in}{1.715000in}}%
\pgfpathlineto{\pgfqpoint{1.566680in}{1.155000in}}%
\pgfpathlineto{\pgfqpoint{1.570400in}{1.680000in}}%
\pgfpathlineto{\pgfqpoint{1.571640in}{1.610000in}}%
\pgfpathlineto{\pgfqpoint{1.572880in}{1.750000in}}%
\pgfpathlineto{\pgfqpoint{1.574120in}{1.155000in}}%
\pgfpathlineto{\pgfqpoint{1.575360in}{1.750000in}}%
\pgfpathlineto{\pgfqpoint{1.576600in}{1.505000in}}%
\pgfpathlineto{\pgfqpoint{1.577840in}{1.680000in}}%
\pgfpathlineto{\pgfqpoint{1.579080in}{1.645000in}}%
\pgfpathlineto{\pgfqpoint{1.580320in}{1.540000in}}%
\pgfpathlineto{\pgfqpoint{1.581560in}{1.575000in}}%
\pgfpathlineto{\pgfqpoint{1.582800in}{1.365000in}}%
\pgfpathlineto{\pgfqpoint{1.584040in}{1.680000in}}%
\pgfpathlineto{\pgfqpoint{1.585280in}{1.470000in}}%
\pgfpathlineto{\pgfqpoint{1.586520in}{1.785000in}}%
\pgfpathlineto{\pgfqpoint{1.587760in}{1.190000in}}%
\pgfpathlineto{\pgfqpoint{1.589000in}{1.435000in}}%
\pgfpathlineto{\pgfqpoint{1.591480in}{1.225000in}}%
\pgfpathlineto{\pgfqpoint{1.593960in}{1.890000in}}%
\pgfpathlineto{\pgfqpoint{1.595200in}{1.680000in}}%
\pgfpathlineto{\pgfqpoint{1.596440in}{1.680000in}}%
\pgfpathlineto{\pgfqpoint{1.597680in}{0.980000in}}%
\pgfpathlineto{\pgfqpoint{1.598920in}{1.400000in}}%
\pgfpathlineto{\pgfqpoint{1.600160in}{1.260000in}}%
\pgfpathlineto{\pgfqpoint{1.602640in}{1.820000in}}%
\pgfpathlineto{\pgfqpoint{1.603880in}{1.330000in}}%
\pgfpathlineto{\pgfqpoint{1.605120in}{1.295000in}}%
\pgfpathlineto{\pgfqpoint{1.606360in}{1.190000in}}%
\pgfpathlineto{\pgfqpoint{1.607600in}{1.260000in}}%
\pgfpathlineto{\pgfqpoint{1.608840in}{1.575000in}}%
\pgfpathlineto{\pgfqpoint{1.610080in}{1.225000in}}%
\pgfpathlineto{\pgfqpoint{1.611320in}{1.190000in}}%
\pgfpathlineto{\pgfqpoint{1.612560in}{1.295000in}}%
\pgfpathlineto{\pgfqpoint{1.613800in}{0.875000in}}%
\pgfpathlineto{\pgfqpoint{1.615040in}{1.925000in}}%
\pgfpathlineto{\pgfqpoint{1.616280in}{1.120000in}}%
\pgfpathlineto{\pgfqpoint{1.617520in}{1.470000in}}%
\pgfpathlineto{\pgfqpoint{1.618760in}{1.085000in}}%
\pgfpathlineto{\pgfqpoint{1.620000in}{1.645000in}}%
\pgfpathlineto{\pgfqpoint{1.621240in}{1.155000in}}%
\pgfpathlineto{\pgfqpoint{1.623720in}{1.505000in}}%
\pgfpathlineto{\pgfqpoint{1.624960in}{1.225000in}}%
\pgfpathlineto{\pgfqpoint{1.626200in}{1.470000in}}%
\pgfpathlineto{\pgfqpoint{1.627440in}{1.365000in}}%
\pgfpathlineto{\pgfqpoint{1.628680in}{1.505000in}}%
\pgfpathlineto{\pgfqpoint{1.629920in}{1.190000in}}%
\pgfpathlineto{\pgfqpoint{1.632400in}{1.330000in}}%
\pgfpathlineto{\pgfqpoint{1.633640in}{0.945000in}}%
\pgfpathlineto{\pgfqpoint{1.636120in}{1.820000in}}%
\pgfpathlineto{\pgfqpoint{1.637360in}{1.225000in}}%
\pgfpathlineto{\pgfqpoint{1.638600in}{1.505000in}}%
\pgfpathlineto{\pgfqpoint{1.639840in}{1.400000in}}%
\pgfpathlineto{\pgfqpoint{1.641080in}{1.610000in}}%
\pgfpathlineto{\pgfqpoint{1.642320in}{1.085000in}}%
\pgfpathlineto{\pgfqpoint{1.643560in}{1.365000in}}%
\pgfpathlineto{\pgfqpoint{1.644800in}{1.225000in}}%
\pgfpathlineto{\pgfqpoint{1.646040in}{1.540000in}}%
\pgfpathlineto{\pgfqpoint{1.647280in}{1.435000in}}%
\pgfpathlineto{\pgfqpoint{1.648520in}{1.435000in}}%
\pgfpathlineto{\pgfqpoint{1.649760in}{1.260000in}}%
\pgfpathlineto{\pgfqpoint{1.651000in}{1.540000in}}%
\pgfpathlineto{\pgfqpoint{1.652240in}{0.910000in}}%
\pgfpathlineto{\pgfqpoint{1.654720in}{1.785000in}}%
\pgfpathlineto{\pgfqpoint{1.658440in}{1.120000in}}%
\pgfpathlineto{\pgfqpoint{1.660920in}{1.890000in}}%
\pgfpathlineto{\pgfqpoint{1.662160in}{1.610000in}}%
\pgfpathlineto{\pgfqpoint{1.663400in}{1.715000in}}%
\pgfpathlineto{\pgfqpoint{1.664640in}{1.190000in}}%
\pgfpathlineto{\pgfqpoint{1.665880in}{1.505000in}}%
\pgfpathlineto{\pgfqpoint{1.667120in}{1.400000in}}%
\pgfpathlineto{\pgfqpoint{1.668360in}{1.435000in}}%
\pgfpathlineto{\pgfqpoint{1.669600in}{1.505000in}}%
\pgfpathlineto{\pgfqpoint{1.670840in}{1.400000in}}%
\pgfpathlineto{\pgfqpoint{1.672080in}{1.540000in}}%
\pgfpathlineto{\pgfqpoint{1.673320in}{1.085000in}}%
\pgfpathlineto{\pgfqpoint{1.674560in}{1.260000in}}%
\pgfpathlineto{\pgfqpoint{1.675800in}{1.085000in}}%
\pgfpathlineto{\pgfqpoint{1.677040in}{1.155000in}}%
\pgfpathlineto{\pgfqpoint{1.678280in}{2.030000in}}%
\pgfpathlineto{\pgfqpoint{1.680760in}{1.575000in}}%
\pgfpathlineto{\pgfqpoint{1.682000in}{1.925000in}}%
\pgfpathlineto{\pgfqpoint{1.683240in}{1.785000in}}%
\pgfpathlineto{\pgfqpoint{1.684480in}{1.890000in}}%
\pgfpathlineto{\pgfqpoint{1.686960in}{1.330000in}}%
\pgfpathlineto{\pgfqpoint{1.688200in}{1.890000in}}%
\pgfpathlineto{\pgfqpoint{1.690680in}{1.400000in}}%
\pgfpathlineto{\pgfqpoint{1.691920in}{1.470000in}}%
\pgfpathlineto{\pgfqpoint{1.693160in}{1.435000in}}%
\pgfpathlineto{\pgfqpoint{1.695640in}{1.050000in}}%
\pgfpathlineto{\pgfqpoint{1.696880in}{1.050000in}}%
\pgfpathlineto{\pgfqpoint{1.699360in}{1.225000in}}%
\pgfpathlineto{\pgfqpoint{1.700600in}{1.715000in}}%
\pgfpathlineto{\pgfqpoint{1.703080in}{1.225000in}}%
\pgfpathlineto{\pgfqpoint{1.704320in}{1.155000in}}%
\pgfpathlineto{\pgfqpoint{1.705560in}{1.645000in}}%
\pgfpathlineto{\pgfqpoint{1.706800in}{1.260000in}}%
\pgfpathlineto{\pgfqpoint{1.708040in}{1.855000in}}%
\pgfpathlineto{\pgfqpoint{1.710520in}{1.400000in}}%
\pgfpathlineto{\pgfqpoint{1.711760in}{1.820000in}}%
\pgfpathlineto{\pgfqpoint{1.714240in}{0.840000in}}%
\pgfpathlineto{\pgfqpoint{1.715480in}{1.330000in}}%
\pgfpathlineto{\pgfqpoint{1.716720in}{1.365000in}}%
\pgfpathlineto{\pgfqpoint{1.717960in}{1.155000in}}%
\pgfpathlineto{\pgfqpoint{1.720440in}{1.470000in}}%
\pgfpathlineto{\pgfqpoint{1.721680in}{1.225000in}}%
\pgfpathlineto{\pgfqpoint{1.722920in}{1.750000in}}%
\pgfpathlineto{\pgfqpoint{1.724160in}{1.225000in}}%
\pgfpathlineto{\pgfqpoint{1.725400in}{1.260000in}}%
\pgfpathlineto{\pgfqpoint{1.727880in}{0.875000in}}%
\pgfpathlineto{\pgfqpoint{1.729120in}{1.190000in}}%
\pgfpathlineto{\pgfqpoint{1.730360in}{1.120000in}}%
\pgfpathlineto{\pgfqpoint{1.732840in}{1.890000in}}%
\pgfpathlineto{\pgfqpoint{1.734080in}{1.365000in}}%
\pgfpathlineto{\pgfqpoint{1.735320in}{1.330000in}}%
\pgfpathlineto{\pgfqpoint{1.736560in}{1.610000in}}%
\pgfpathlineto{\pgfqpoint{1.739040in}{1.260000in}}%
\pgfpathlineto{\pgfqpoint{1.742760in}{1.505000in}}%
\pgfpathlineto{\pgfqpoint{1.744000in}{1.470000in}}%
\pgfpathlineto{\pgfqpoint{1.745240in}{1.680000in}}%
\pgfpathlineto{\pgfqpoint{1.746480in}{1.575000in}}%
\pgfpathlineto{\pgfqpoint{1.747720in}{1.365000in}}%
\pgfpathlineto{\pgfqpoint{1.748960in}{1.400000in}}%
\pgfpathlineto{\pgfqpoint{1.750200in}{1.190000in}}%
\pgfpathlineto{\pgfqpoint{1.751440in}{1.715000in}}%
\pgfpathlineto{\pgfqpoint{1.753920in}{1.505000in}}%
\pgfpathlineto{\pgfqpoint{1.755160in}{1.680000in}}%
\pgfpathlineto{\pgfqpoint{1.757640in}{1.400000in}}%
\pgfpathlineto{\pgfqpoint{1.758880in}{1.575000in}}%
\pgfpathlineto{\pgfqpoint{1.760120in}{1.505000in}}%
\pgfpathlineto{\pgfqpoint{1.761360in}{1.050000in}}%
\pgfpathlineto{\pgfqpoint{1.762600in}{1.925000in}}%
\pgfpathlineto{\pgfqpoint{1.763840in}{1.260000in}}%
\pgfpathlineto{\pgfqpoint{1.765080in}{1.365000in}}%
\pgfpathlineto{\pgfqpoint{1.766320in}{1.295000in}}%
\pgfpathlineto{\pgfqpoint{1.767560in}{2.240000in}}%
\pgfpathlineto{\pgfqpoint{1.770040in}{1.505000in}}%
\pgfpathlineto{\pgfqpoint{1.771280in}{1.995000in}}%
\pgfpathlineto{\pgfqpoint{1.772520in}{1.925000in}}%
\pgfpathlineto{\pgfqpoint{1.773760in}{1.435000in}}%
\pgfpathlineto{\pgfqpoint{1.776240in}{1.715000in}}%
\pgfpathlineto{\pgfqpoint{1.777480in}{0.980000in}}%
\pgfpathlineto{\pgfqpoint{1.778720in}{0.980000in}}%
\pgfpathlineto{\pgfqpoint{1.781200in}{1.575000in}}%
\pgfpathlineto{\pgfqpoint{1.782440in}{1.295000in}}%
\pgfpathlineto{\pgfqpoint{1.784920in}{1.505000in}}%
\pgfpathlineto{\pgfqpoint{1.786160in}{1.505000in}}%
\pgfpathlineto{\pgfqpoint{1.788640in}{1.750000in}}%
\pgfpathlineto{\pgfqpoint{1.789880in}{1.855000in}}%
\pgfpathlineto{\pgfqpoint{1.791120in}{1.260000in}}%
\pgfpathlineto{\pgfqpoint{1.792360in}{1.260000in}}%
\pgfpathlineto{\pgfqpoint{1.793600in}{1.890000in}}%
\pgfpathlineto{\pgfqpoint{1.794840in}{1.715000in}}%
\pgfpathlineto{\pgfqpoint{1.796080in}{1.715000in}}%
\pgfpathlineto{\pgfqpoint{1.798560in}{1.190000in}}%
\pgfpathlineto{\pgfqpoint{1.799800in}{1.225000in}}%
\pgfpathlineto{\pgfqpoint{1.801040in}{1.015000in}}%
\pgfpathlineto{\pgfqpoint{1.803520in}{1.750000in}}%
\pgfpathlineto{\pgfqpoint{1.804760in}{1.435000in}}%
\pgfpathlineto{\pgfqpoint{1.806000in}{1.960000in}}%
\pgfpathlineto{\pgfqpoint{1.808480in}{1.330000in}}%
\pgfpathlineto{\pgfqpoint{1.809720in}{1.155000in}}%
\pgfpathlineto{\pgfqpoint{1.810960in}{1.365000in}}%
\pgfpathlineto{\pgfqpoint{1.812200in}{1.260000in}}%
\pgfpathlineto{\pgfqpoint{1.815920in}{1.785000in}}%
\pgfpathlineto{\pgfqpoint{1.817160in}{1.750000in}}%
\pgfpathlineto{\pgfqpoint{1.818400in}{1.715000in}}%
\pgfpathlineto{\pgfqpoint{1.819640in}{1.295000in}}%
\pgfpathlineto{\pgfqpoint{1.820880in}{1.645000in}}%
\pgfpathlineto{\pgfqpoint{1.822120in}{1.610000in}}%
\pgfpathlineto{\pgfqpoint{1.823360in}{1.610000in}}%
\pgfpathlineto{\pgfqpoint{1.824600in}{1.680000in}}%
\pgfpathlineto{\pgfqpoint{1.825840in}{1.435000in}}%
\pgfpathlineto{\pgfqpoint{1.827080in}{1.435000in}}%
\pgfpathlineto{\pgfqpoint{1.828320in}{1.855000in}}%
\pgfpathlineto{\pgfqpoint{1.829560in}{1.365000in}}%
\pgfpathlineto{\pgfqpoint{1.832040in}{1.785000in}}%
\pgfpathlineto{\pgfqpoint{1.835760in}{1.295000in}}%
\pgfpathlineto{\pgfqpoint{1.837000in}{1.680000in}}%
\pgfpathlineto{\pgfqpoint{1.838240in}{1.260000in}}%
\pgfpathlineto{\pgfqpoint{1.839480in}{1.540000in}}%
\pgfpathlineto{\pgfqpoint{1.840720in}{1.400000in}}%
\pgfpathlineto{\pgfqpoint{1.841960in}{1.120000in}}%
\pgfpathlineto{\pgfqpoint{1.843200in}{1.365000in}}%
\pgfpathlineto{\pgfqpoint{1.844440in}{1.330000in}}%
\pgfpathlineto{\pgfqpoint{1.846920in}{1.330000in}}%
\pgfpathlineto{\pgfqpoint{1.848160in}{1.400000in}}%
\pgfpathlineto{\pgfqpoint{1.849400in}{1.680000in}}%
\pgfpathlineto{\pgfqpoint{1.850640in}{1.435000in}}%
\pgfpathlineto{\pgfqpoint{1.851880in}{1.540000in}}%
\pgfpathlineto{\pgfqpoint{1.853120in}{1.785000in}}%
\pgfpathlineto{\pgfqpoint{1.854360in}{1.295000in}}%
\pgfpathlineto{\pgfqpoint{1.855600in}{1.260000in}}%
\pgfpathlineto{\pgfqpoint{1.856840in}{1.120000in}}%
\pgfpathlineto{\pgfqpoint{1.858080in}{1.260000in}}%
\pgfpathlineto{\pgfqpoint{1.859320in}{1.225000in}}%
\pgfpathlineto{\pgfqpoint{1.860560in}{1.155000in}}%
\pgfpathlineto{\pgfqpoint{1.861800in}{0.840000in}}%
\pgfpathlineto{\pgfqpoint{1.863040in}{1.225000in}}%
\pgfpathlineto{\pgfqpoint{1.864280in}{0.840000in}}%
\pgfpathlineto{\pgfqpoint{1.866760in}{1.610000in}}%
\pgfpathlineto{\pgfqpoint{1.868000in}{1.190000in}}%
\pgfpathlineto{\pgfqpoint{1.869240in}{1.190000in}}%
\pgfpathlineto{\pgfqpoint{1.870480in}{1.645000in}}%
\pgfpathlineto{\pgfqpoint{1.871720in}{1.365000in}}%
\pgfpathlineto{\pgfqpoint{1.872960in}{1.505000in}}%
\pgfpathlineto{\pgfqpoint{1.874200in}{0.945000in}}%
\pgfpathlineto{\pgfqpoint{1.875440in}{1.400000in}}%
\pgfpathlineto{\pgfqpoint{1.876680in}{1.330000in}}%
\pgfpathlineto{\pgfqpoint{1.877920in}{1.575000in}}%
\pgfpathlineto{\pgfqpoint{1.879160in}{1.540000in}}%
\pgfpathlineto{\pgfqpoint{1.881640in}{1.155000in}}%
\pgfpathlineto{\pgfqpoint{1.882880in}{1.225000in}}%
\pgfpathlineto{\pgfqpoint{1.884120in}{1.715000in}}%
\pgfpathlineto{\pgfqpoint{1.889080in}{1.120000in}}%
\pgfpathlineto{\pgfqpoint{1.890320in}{1.470000in}}%
\pgfpathlineto{\pgfqpoint{1.891560in}{1.435000in}}%
\pgfpathlineto{\pgfqpoint{1.892800in}{1.820000in}}%
\pgfpathlineto{\pgfqpoint{1.894040in}{1.260000in}}%
\pgfpathlineto{\pgfqpoint{1.895280in}{1.820000in}}%
\pgfpathlineto{\pgfqpoint{1.896520in}{1.260000in}}%
\pgfpathlineto{\pgfqpoint{1.897760in}{1.680000in}}%
\pgfpathlineto{\pgfqpoint{1.900240in}{1.050000in}}%
\pgfpathlineto{\pgfqpoint{1.901480in}{1.505000in}}%
\pgfpathlineto{\pgfqpoint{1.903960in}{1.575000in}}%
\pgfpathlineto{\pgfqpoint{1.906440in}{1.120000in}}%
\pgfpathlineto{\pgfqpoint{1.908920in}{1.505000in}}%
\pgfpathlineto{\pgfqpoint{1.910160in}{1.470000in}}%
\pgfpathlineto{\pgfqpoint{1.911400in}{1.645000in}}%
\pgfpathlineto{\pgfqpoint{1.912640in}{1.295000in}}%
\pgfpathlineto{\pgfqpoint{1.913880in}{1.435000in}}%
\pgfpathlineto{\pgfqpoint{1.915120in}{1.400000in}}%
\pgfpathlineto{\pgfqpoint{1.916360in}{1.750000in}}%
\pgfpathlineto{\pgfqpoint{1.917600in}{1.260000in}}%
\pgfpathlineto{\pgfqpoint{1.918840in}{1.505000in}}%
\pgfpathlineto{\pgfqpoint{1.920080in}{1.435000in}}%
\pgfpathlineto{\pgfqpoint{1.921320in}{1.435000in}}%
\pgfpathlineto{\pgfqpoint{1.922560in}{1.505000in}}%
\pgfpathlineto{\pgfqpoint{1.923800in}{1.330000in}}%
\pgfpathlineto{\pgfqpoint{1.925040in}{1.680000in}}%
\pgfpathlineto{\pgfqpoint{1.926280in}{1.190000in}}%
\pgfpathlineto{\pgfqpoint{1.927520in}{1.470000in}}%
\pgfpathlineto{\pgfqpoint{1.930000in}{1.085000in}}%
\pgfpathlineto{\pgfqpoint{1.931240in}{1.435000in}}%
\pgfpathlineto{\pgfqpoint{1.932480in}{1.470000in}}%
\pgfpathlineto{\pgfqpoint{1.934960in}{1.190000in}}%
\pgfpathlineto{\pgfqpoint{1.936200in}{1.575000in}}%
\pgfpathlineto{\pgfqpoint{1.937440in}{1.295000in}}%
\pgfpathlineto{\pgfqpoint{1.938680in}{1.680000in}}%
\pgfpathlineto{\pgfqpoint{1.941160in}{1.260000in}}%
\pgfpathlineto{\pgfqpoint{1.942400in}{1.295000in}}%
\pgfpathlineto{\pgfqpoint{1.943640in}{0.875000in}}%
\pgfpathlineto{\pgfqpoint{1.944880in}{1.925000in}}%
\pgfpathlineto{\pgfqpoint{1.946120in}{1.015000in}}%
\pgfpathlineto{\pgfqpoint{1.947360in}{1.470000in}}%
\pgfpathlineto{\pgfqpoint{1.948600in}{1.400000in}}%
\pgfpathlineto{\pgfqpoint{1.949840in}{1.925000in}}%
\pgfpathlineto{\pgfqpoint{1.952320in}{1.400000in}}%
\pgfpathlineto{\pgfqpoint{1.953560in}{0.945000in}}%
\pgfpathlineto{\pgfqpoint{1.954800in}{1.015000in}}%
\pgfpathlineto{\pgfqpoint{1.956040in}{1.750000in}}%
\pgfpathlineto{\pgfqpoint{1.957280in}{1.085000in}}%
\pgfpathlineto{\pgfqpoint{1.958520in}{1.540000in}}%
\pgfpathlineto{\pgfqpoint{1.959760in}{1.435000in}}%
\pgfpathlineto{\pgfqpoint{1.961000in}{1.680000in}}%
\pgfpathlineto{\pgfqpoint{1.962240in}{1.540000in}}%
\pgfpathlineto{\pgfqpoint{1.963480in}{1.610000in}}%
\pgfpathlineto{\pgfqpoint{1.964720in}{1.435000in}}%
\pgfpathlineto{\pgfqpoint{1.965960in}{1.680000in}}%
\pgfpathlineto{\pgfqpoint{1.967200in}{1.610000in}}%
\pgfpathlineto{\pgfqpoint{1.969680in}{1.015000in}}%
\pgfpathlineto{\pgfqpoint{1.970920in}{1.610000in}}%
\pgfpathlineto{\pgfqpoint{1.972160in}{1.295000in}}%
\pgfpathlineto{\pgfqpoint{1.973400in}{1.400000in}}%
\pgfpathlineto{\pgfqpoint{1.974640in}{2.100000in}}%
\pgfpathlineto{\pgfqpoint{1.977120in}{1.435000in}}%
\pgfpathlineto{\pgfqpoint{1.978360in}{1.645000in}}%
\pgfpathlineto{\pgfqpoint{1.979600in}{1.260000in}}%
\pgfpathlineto{\pgfqpoint{1.982080in}{1.890000in}}%
\pgfpathlineto{\pgfqpoint{1.983320in}{1.680000in}}%
\pgfpathlineto{\pgfqpoint{1.984560in}{1.680000in}}%
\pgfpathlineto{\pgfqpoint{1.987040in}{1.925000in}}%
\pgfpathlineto{\pgfqpoint{1.988280in}{1.540000in}}%
\pgfpathlineto{\pgfqpoint{1.989520in}{1.890000in}}%
\pgfpathlineto{\pgfqpoint{1.990760in}{1.260000in}}%
\pgfpathlineto{\pgfqpoint{1.992000in}{1.785000in}}%
\pgfpathlineto{\pgfqpoint{1.993240in}{1.050000in}}%
\pgfpathlineto{\pgfqpoint{1.995720in}{1.365000in}}%
\pgfpathlineto{\pgfqpoint{1.996960in}{1.260000in}}%
\pgfpathlineto{\pgfqpoint{1.998200in}{1.505000in}}%
\pgfpathlineto{\pgfqpoint{1.999440in}{0.805000in}}%
\pgfpathlineto{\pgfqpoint{2.000680in}{1.575000in}}%
\pgfpathlineto{\pgfqpoint{2.003160in}{1.435000in}}%
\pgfpathlineto{\pgfqpoint{2.004400in}{1.785000in}}%
\pgfpathlineto{\pgfqpoint{2.005640in}{1.435000in}}%
\pgfpathlineto{\pgfqpoint{2.006880in}{1.820000in}}%
\pgfpathlineto{\pgfqpoint{2.008120in}{1.400000in}}%
\pgfpathlineto{\pgfqpoint{2.009360in}{1.505000in}}%
\pgfpathlineto{\pgfqpoint{2.010600in}{1.505000in}}%
\pgfpathlineto{\pgfqpoint{2.011840in}{1.575000in}}%
\pgfpathlineto{\pgfqpoint{2.014320in}{1.260000in}}%
\pgfpathlineto{\pgfqpoint{2.015560in}{1.435000in}}%
\pgfpathlineto{\pgfqpoint{2.016800in}{1.225000in}}%
\pgfpathlineto{\pgfqpoint{2.019280in}{1.750000in}}%
\pgfpathlineto{\pgfqpoint{2.021760in}{1.225000in}}%
\pgfpathlineto{\pgfqpoint{2.023000in}{1.785000in}}%
\pgfpathlineto{\pgfqpoint{2.024240in}{1.680000in}}%
\pgfpathlineto{\pgfqpoint{2.025480in}{1.680000in}}%
\pgfpathlineto{\pgfqpoint{2.027960in}{1.470000in}}%
\pgfpathlineto{\pgfqpoint{2.029200in}{1.505000in}}%
\pgfpathlineto{\pgfqpoint{2.030440in}{1.680000in}}%
\pgfpathlineto{\pgfqpoint{2.031680in}{1.295000in}}%
\pgfpathlineto{\pgfqpoint{2.032920in}{1.645000in}}%
\pgfpathlineto{\pgfqpoint{2.034160in}{1.575000in}}%
\pgfpathlineto{\pgfqpoint{2.036640in}{1.120000in}}%
\pgfpathlineto{\pgfqpoint{2.037880in}{1.295000in}}%
\pgfpathlineto{\pgfqpoint{2.039120in}{1.225000in}}%
\pgfpathlineto{\pgfqpoint{2.040360in}{1.540000in}}%
\pgfpathlineto{\pgfqpoint{2.041600in}{0.840000in}}%
\pgfpathlineto{\pgfqpoint{2.044080in}{1.330000in}}%
\pgfpathlineto{\pgfqpoint{2.047800in}{1.435000in}}%
\pgfpathlineto{\pgfqpoint{2.050280in}{1.575000in}}%
\pgfpathlineto{\pgfqpoint{2.051520in}{1.645000in}}%
\pgfpathlineto{\pgfqpoint{2.052760in}{1.155000in}}%
\pgfpathlineto{\pgfqpoint{2.055240in}{1.680000in}}%
\pgfpathlineto{\pgfqpoint{2.056480in}{1.680000in}}%
\pgfpathlineto{\pgfqpoint{2.057720in}{1.960000in}}%
\pgfpathlineto{\pgfqpoint{2.058960in}{1.890000in}}%
\pgfpathlineto{\pgfqpoint{2.060200in}{2.205000in}}%
\pgfpathlineto{\pgfqpoint{2.061440in}{1.610000in}}%
\pgfpathlineto{\pgfqpoint{2.062680in}{1.785000in}}%
\pgfpathlineto{\pgfqpoint{2.063920in}{1.715000in}}%
\pgfpathlineto{\pgfqpoint{2.065160in}{1.470000in}}%
\pgfpathlineto{\pgfqpoint{2.066400in}{1.470000in}}%
\pgfpathlineto{\pgfqpoint{2.067640in}{1.295000in}}%
\pgfpathlineto{\pgfqpoint{2.068880in}{1.435000in}}%
\pgfpathlineto{\pgfqpoint{2.070120in}{1.365000in}}%
\pgfpathlineto{\pgfqpoint{2.071360in}{1.365000in}}%
\pgfpathlineto{\pgfqpoint{2.073840in}{1.750000in}}%
\pgfpathlineto{\pgfqpoint{2.075080in}{1.260000in}}%
\pgfpathlineto{\pgfqpoint{2.077560in}{1.400000in}}%
\pgfpathlineto{\pgfqpoint{2.078800in}{1.785000in}}%
\pgfpathlineto{\pgfqpoint{2.081280in}{1.505000in}}%
\pgfpathlineto{\pgfqpoint{2.082520in}{1.435000in}}%
\pgfpathlineto{\pgfqpoint{2.083760in}{1.680000in}}%
\pgfpathlineto{\pgfqpoint{2.086240in}{1.680000in}}%
\pgfpathlineto{\pgfqpoint{2.089960in}{1.190000in}}%
\pgfpathlineto{\pgfqpoint{2.091200in}{1.400000in}}%
\pgfpathlineto{\pgfqpoint{2.092440in}{1.225000in}}%
\pgfpathlineto{\pgfqpoint{2.093680in}{1.400000in}}%
\pgfpathlineto{\pgfqpoint{2.096160in}{1.155000in}}%
\pgfpathlineto{\pgfqpoint{2.097400in}{1.330000in}}%
\pgfpathlineto{\pgfqpoint{2.098640in}{1.260000in}}%
\pgfpathlineto{\pgfqpoint{2.099880in}{1.435000in}}%
\pgfpathlineto{\pgfqpoint{2.102360in}{1.330000in}}%
\pgfpathlineto{\pgfqpoint{2.103600in}{1.400000in}}%
\pgfpathlineto{\pgfqpoint{2.106080in}{1.820000in}}%
\pgfpathlineto{\pgfqpoint{2.107320in}{1.295000in}}%
\pgfpathlineto{\pgfqpoint{2.108560in}{1.645000in}}%
\pgfpathlineto{\pgfqpoint{2.109800in}{1.085000in}}%
\pgfpathlineto{\pgfqpoint{2.113520in}{1.750000in}}%
\pgfpathlineto{\pgfqpoint{2.116000in}{1.890000in}}%
\pgfpathlineto{\pgfqpoint{2.117240in}{1.330000in}}%
\pgfpathlineto{\pgfqpoint{2.118480in}{1.645000in}}%
\pgfpathlineto{\pgfqpoint{2.119720in}{1.085000in}}%
\pgfpathlineto{\pgfqpoint{2.122200in}{1.365000in}}%
\pgfpathlineto{\pgfqpoint{2.123440in}{1.995000in}}%
\pgfpathlineto{\pgfqpoint{2.124680in}{1.925000in}}%
\pgfpathlineto{\pgfqpoint{2.125920in}{1.925000in}}%
\pgfpathlineto{\pgfqpoint{2.128400in}{1.120000in}}%
\pgfpathlineto{\pgfqpoint{2.129640in}{1.365000in}}%
\pgfpathlineto{\pgfqpoint{2.130880in}{1.050000in}}%
\pgfpathlineto{\pgfqpoint{2.132120in}{1.715000in}}%
\pgfpathlineto{\pgfqpoint{2.133360in}{1.750000in}}%
\pgfpathlineto{\pgfqpoint{2.134600in}{1.505000in}}%
\pgfpathlineto{\pgfqpoint{2.135840in}{1.505000in}}%
\pgfpathlineto{\pgfqpoint{2.137080in}{1.575000in}}%
\pgfpathlineto{\pgfqpoint{2.138320in}{1.435000in}}%
\pgfpathlineto{\pgfqpoint{2.139560in}{1.120000in}}%
\pgfpathlineto{\pgfqpoint{2.140800in}{1.400000in}}%
\pgfpathlineto{\pgfqpoint{2.142040in}{1.330000in}}%
\pgfpathlineto{\pgfqpoint{2.144520in}{1.855000in}}%
\pgfpathlineto{\pgfqpoint{2.145760in}{1.505000in}}%
\pgfpathlineto{\pgfqpoint{2.147000in}{1.505000in}}%
\pgfpathlineto{\pgfqpoint{2.148240in}{1.540000in}}%
\pgfpathlineto{\pgfqpoint{2.149480in}{1.680000in}}%
\pgfpathlineto{\pgfqpoint{2.150720in}{1.330000in}}%
\pgfpathlineto{\pgfqpoint{2.151960in}{1.435000in}}%
\pgfpathlineto{\pgfqpoint{2.153200in}{1.820000in}}%
\pgfpathlineto{\pgfqpoint{2.154440in}{1.225000in}}%
\pgfpathlineto{\pgfqpoint{2.155680in}{1.295000in}}%
\pgfpathlineto{\pgfqpoint{2.156920in}{1.260000in}}%
\pgfpathlineto{\pgfqpoint{2.158160in}{1.470000in}}%
\pgfpathlineto{\pgfqpoint{2.159400in}{1.435000in}}%
\pgfpathlineto{\pgfqpoint{2.161880in}{1.225000in}}%
\pgfpathlineto{\pgfqpoint{2.164360in}{1.680000in}}%
\pgfpathlineto{\pgfqpoint{2.165600in}{1.645000in}}%
\pgfpathlineto{\pgfqpoint{2.166840in}{1.680000in}}%
\pgfpathlineto{\pgfqpoint{2.168080in}{1.470000in}}%
\pgfpathlineto{\pgfqpoint{2.169320in}{1.680000in}}%
\pgfpathlineto{\pgfqpoint{2.170560in}{1.505000in}}%
\pgfpathlineto{\pgfqpoint{2.173040in}{1.715000in}}%
\pgfpathlineto{\pgfqpoint{2.174280in}{1.260000in}}%
\pgfpathlineto{\pgfqpoint{2.175520in}{1.260000in}}%
\pgfpathlineto{\pgfqpoint{2.176760in}{1.785000in}}%
\pgfpathlineto{\pgfqpoint{2.178000in}{1.225000in}}%
\pgfpathlineto{\pgfqpoint{2.179240in}{1.820000in}}%
\pgfpathlineto{\pgfqpoint{2.181720in}{1.120000in}}%
\pgfpathlineto{\pgfqpoint{2.182960in}{1.505000in}}%
\pgfpathlineto{\pgfqpoint{2.184200in}{1.225000in}}%
\pgfpathlineto{\pgfqpoint{2.185440in}{1.610000in}}%
\pgfpathlineto{\pgfqpoint{2.186680in}{1.400000in}}%
\pgfpathlineto{\pgfqpoint{2.189160in}{1.820000in}}%
\pgfpathlineto{\pgfqpoint{2.191640in}{1.225000in}}%
\pgfpathlineto{\pgfqpoint{2.192880in}{1.365000in}}%
\pgfpathlineto{\pgfqpoint{2.194120in}{1.190000in}}%
\pgfpathlineto{\pgfqpoint{2.195360in}{1.365000in}}%
\pgfpathlineto{\pgfqpoint{2.197840in}{0.910000in}}%
\pgfpathlineto{\pgfqpoint{2.199080in}{0.980000in}}%
\pgfpathlineto{\pgfqpoint{2.200320in}{1.295000in}}%
\pgfpathlineto{\pgfqpoint{2.201560in}{1.015000in}}%
\pgfpathlineto{\pgfqpoint{2.204040in}{1.365000in}}%
\pgfpathlineto{\pgfqpoint{2.205280in}{1.365000in}}%
\pgfpathlineto{\pgfqpoint{2.206520in}{1.015000in}}%
\pgfpathlineto{\pgfqpoint{2.207760in}{1.400000in}}%
\pgfpathlineto{\pgfqpoint{2.209000in}{1.085000in}}%
\pgfpathlineto{\pgfqpoint{2.210240in}{1.785000in}}%
\pgfpathlineto{\pgfqpoint{2.212720in}{1.225000in}}%
\pgfpathlineto{\pgfqpoint{2.213960in}{1.925000in}}%
\pgfpathlineto{\pgfqpoint{2.216440in}{0.700000in}}%
\pgfpathlineto{\pgfqpoint{2.217680in}{1.610000in}}%
\pgfpathlineto{\pgfqpoint{2.218920in}{1.330000in}}%
\pgfpathlineto{\pgfqpoint{2.220160in}{1.505000in}}%
\pgfpathlineto{\pgfqpoint{2.221400in}{1.330000in}}%
\pgfpathlineto{\pgfqpoint{2.223880in}{1.575000in}}%
\pgfpathlineto{\pgfqpoint{2.226360in}{1.085000in}}%
\pgfpathlineto{\pgfqpoint{2.228840in}{1.960000in}}%
\pgfpathlineto{\pgfqpoint{2.230080in}{1.715000in}}%
\pgfpathlineto{\pgfqpoint{2.231320in}{1.925000in}}%
\pgfpathlineto{\pgfqpoint{2.232560in}{1.750000in}}%
\pgfpathlineto{\pgfqpoint{2.233800in}{1.365000in}}%
\pgfpathlineto{\pgfqpoint{2.235040in}{1.645000in}}%
\pgfpathlineto{\pgfqpoint{2.236280in}{1.400000in}}%
\pgfpathlineto{\pgfqpoint{2.237520in}{1.750000in}}%
\pgfpathlineto{\pgfqpoint{2.238760in}{1.680000in}}%
\pgfpathlineto{\pgfqpoint{2.240000in}{1.855000in}}%
\pgfpathlineto{\pgfqpoint{2.241240in}{1.575000in}}%
\pgfpathlineto{\pgfqpoint{2.242480in}{1.575000in}}%
\pgfpathlineto{\pgfqpoint{2.243720in}{1.680000in}}%
\pgfpathlineto{\pgfqpoint{2.244960in}{1.330000in}}%
\pgfpathlineto{\pgfqpoint{2.246200in}{1.295000in}}%
\pgfpathlineto{\pgfqpoint{2.247440in}{1.295000in}}%
\pgfpathlineto{\pgfqpoint{2.248680in}{1.400000in}}%
\pgfpathlineto{\pgfqpoint{2.249920in}{1.190000in}}%
\pgfpathlineto{\pgfqpoint{2.252400in}{1.365000in}}%
\pgfpathlineto{\pgfqpoint{2.253640in}{1.610000in}}%
\pgfpathlineto{\pgfqpoint{2.254880in}{1.365000in}}%
\pgfpathlineto{\pgfqpoint{2.256120in}{1.715000in}}%
\pgfpathlineto{\pgfqpoint{2.257360in}{1.190000in}}%
\pgfpathlineto{\pgfqpoint{2.258600in}{1.610000in}}%
\pgfpathlineto{\pgfqpoint{2.259840in}{1.190000in}}%
\pgfpathlineto{\pgfqpoint{2.261080in}{1.155000in}}%
\pgfpathlineto{\pgfqpoint{2.262320in}{1.645000in}}%
\pgfpathlineto{\pgfqpoint{2.263560in}{1.225000in}}%
\pgfpathlineto{\pgfqpoint{2.264800in}{1.400000in}}%
\pgfpathlineto{\pgfqpoint{2.266040in}{1.120000in}}%
\pgfpathlineto{\pgfqpoint{2.267280in}{1.155000in}}%
\pgfpathlineto{\pgfqpoint{2.268520in}{1.855000in}}%
\pgfpathlineto{\pgfqpoint{2.269760in}{1.785000in}}%
\pgfpathlineto{\pgfqpoint{2.271000in}{1.330000in}}%
\pgfpathlineto{\pgfqpoint{2.272240in}{1.715000in}}%
\pgfpathlineto{\pgfqpoint{2.273480in}{1.575000in}}%
\pgfpathlineto{\pgfqpoint{2.274720in}{1.575000in}}%
\pgfpathlineto{\pgfqpoint{2.275960in}{2.065000in}}%
\pgfpathlineto{\pgfqpoint{2.278440in}{0.980000in}}%
\pgfpathlineto{\pgfqpoint{2.279680in}{1.785000in}}%
\pgfpathlineto{\pgfqpoint{2.280920in}{1.540000in}}%
\pgfpathlineto{\pgfqpoint{2.282160in}{1.015000in}}%
\pgfpathlineto{\pgfqpoint{2.283400in}{1.190000in}}%
\pgfpathlineto{\pgfqpoint{2.284640in}{1.820000in}}%
\pgfpathlineto{\pgfqpoint{2.285880in}{1.365000in}}%
\pgfpathlineto{\pgfqpoint{2.287120in}{1.715000in}}%
\pgfpathlineto{\pgfqpoint{2.288360in}{1.400000in}}%
\pgfpathlineto{\pgfqpoint{2.289600in}{1.680000in}}%
\pgfpathlineto{\pgfqpoint{2.292080in}{1.680000in}}%
\pgfpathlineto{\pgfqpoint{2.293320in}{1.645000in}}%
\pgfpathlineto{\pgfqpoint{2.294560in}{1.820000in}}%
\pgfpathlineto{\pgfqpoint{2.297040in}{1.435000in}}%
\pgfpathlineto{\pgfqpoint{2.299520in}{1.960000in}}%
\pgfpathlineto{\pgfqpoint{2.300760in}{1.820000in}}%
\pgfpathlineto{\pgfqpoint{2.302000in}{1.995000in}}%
\pgfpathlineto{\pgfqpoint{2.303240in}{1.330000in}}%
\pgfpathlineto{\pgfqpoint{2.304480in}{1.400000in}}%
\pgfpathlineto{\pgfqpoint{2.306960in}{1.750000in}}%
\pgfpathlineto{\pgfqpoint{2.310680in}{1.120000in}}%
\pgfpathlineto{\pgfqpoint{2.311920in}{1.715000in}}%
\pgfpathlineto{\pgfqpoint{2.313160in}{1.575000in}}%
\pgfpathlineto{\pgfqpoint{2.314400in}{1.575000in}}%
\pgfpathlineto{\pgfqpoint{2.315640in}{0.980000in}}%
\pgfpathlineto{\pgfqpoint{2.318120in}{1.715000in}}%
\pgfpathlineto{\pgfqpoint{2.319360in}{1.330000in}}%
\pgfpathlineto{\pgfqpoint{2.321840in}{1.820000in}}%
\pgfpathlineto{\pgfqpoint{2.325560in}{1.085000in}}%
\pgfpathlineto{\pgfqpoint{2.326800in}{1.715000in}}%
\pgfpathlineto{\pgfqpoint{2.328040in}{1.260000in}}%
\pgfpathlineto{\pgfqpoint{2.329280in}{1.505000in}}%
\pgfpathlineto{\pgfqpoint{2.330520in}{1.470000in}}%
\pgfpathlineto{\pgfqpoint{2.331760in}{1.680000in}}%
\pgfpathlineto{\pgfqpoint{2.333000in}{1.295000in}}%
\pgfpathlineto{\pgfqpoint{2.337960in}{1.925000in}}%
\pgfpathlineto{\pgfqpoint{2.339200in}{1.435000in}}%
\pgfpathlineto{\pgfqpoint{2.340440in}{2.275000in}}%
\pgfpathlineto{\pgfqpoint{2.341680in}{1.470000in}}%
\pgfpathlineto{\pgfqpoint{2.342920in}{1.575000in}}%
\pgfpathlineto{\pgfqpoint{2.344160in}{1.295000in}}%
\pgfpathlineto{\pgfqpoint{2.346640in}{1.715000in}}%
\pgfpathlineto{\pgfqpoint{2.347880in}{1.645000in}}%
\pgfpathlineto{\pgfqpoint{2.349120in}{1.435000in}}%
\pgfpathlineto{\pgfqpoint{2.350360in}{1.470000in}}%
\pgfpathlineto{\pgfqpoint{2.351600in}{1.645000in}}%
\pgfpathlineto{\pgfqpoint{2.352840in}{1.645000in}}%
\pgfpathlineto{\pgfqpoint{2.354080in}{1.435000in}}%
\pgfpathlineto{\pgfqpoint{2.355320in}{1.610000in}}%
\pgfpathlineto{\pgfqpoint{2.357800in}{1.155000in}}%
\pgfpathlineto{\pgfqpoint{2.359040in}{1.120000in}}%
\pgfpathlineto{\pgfqpoint{2.361520in}{1.645000in}}%
\pgfpathlineto{\pgfqpoint{2.362760in}{1.575000in}}%
\pgfpathlineto{\pgfqpoint{2.364000in}{1.365000in}}%
\pgfpathlineto{\pgfqpoint{2.365240in}{1.715000in}}%
\pgfpathlineto{\pgfqpoint{2.366480in}{1.295000in}}%
\pgfpathlineto{\pgfqpoint{2.367720in}{1.540000in}}%
\pgfpathlineto{\pgfqpoint{2.368960in}{1.470000in}}%
\pgfpathlineto{\pgfqpoint{2.370200in}{1.820000in}}%
\pgfpathlineto{\pgfqpoint{2.371440in}{1.470000in}}%
\pgfpathlineto{\pgfqpoint{2.372680in}{1.435000in}}%
\pgfpathlineto{\pgfqpoint{2.373920in}{1.855000in}}%
\pgfpathlineto{\pgfqpoint{2.376400in}{1.505000in}}%
\pgfpathlineto{\pgfqpoint{2.377640in}{1.610000in}}%
\pgfpathlineto{\pgfqpoint{2.378880in}{2.065000in}}%
\pgfpathlineto{\pgfqpoint{2.380120in}{0.945000in}}%
\pgfpathlineto{\pgfqpoint{2.381360in}{1.260000in}}%
\pgfpathlineto{\pgfqpoint{2.382600in}{1.120000in}}%
\pgfpathlineto{\pgfqpoint{2.383840in}{1.435000in}}%
\pgfpathlineto{\pgfqpoint{2.385080in}{1.435000in}}%
\pgfpathlineto{\pgfqpoint{2.386320in}{1.400000in}}%
\pgfpathlineto{\pgfqpoint{2.387560in}{1.400000in}}%
\pgfpathlineto{\pgfqpoint{2.388800in}{1.260000in}}%
\pgfpathlineto{\pgfqpoint{2.390040in}{1.400000in}}%
\pgfpathlineto{\pgfqpoint{2.391280in}{1.820000in}}%
\pgfpathlineto{\pgfqpoint{2.392520in}{1.365000in}}%
\pgfpathlineto{\pgfqpoint{2.393760in}{1.540000in}}%
\pgfpathlineto{\pgfqpoint{2.395000in}{1.470000in}}%
\pgfpathlineto{\pgfqpoint{2.396240in}{1.505000in}}%
\pgfpathlineto{\pgfqpoint{2.398720in}{1.225000in}}%
\pgfpathlineto{\pgfqpoint{2.401200in}{1.680000in}}%
\pgfpathlineto{\pgfqpoint{2.402440in}{0.910000in}}%
\pgfpathlineto{\pgfqpoint{2.403680in}{1.435000in}}%
\pgfpathlineto{\pgfqpoint{2.404920in}{1.400000in}}%
\pgfpathlineto{\pgfqpoint{2.406160in}{1.225000in}}%
\pgfpathlineto{\pgfqpoint{2.407400in}{1.575000in}}%
\pgfpathlineto{\pgfqpoint{2.408640in}{1.225000in}}%
\pgfpathlineto{\pgfqpoint{2.411120in}{1.715000in}}%
\pgfpathlineto{\pgfqpoint{2.412360in}{1.715000in}}%
\pgfpathlineto{\pgfqpoint{2.413600in}{1.785000in}}%
\pgfpathlineto{\pgfqpoint{2.414840in}{1.680000in}}%
\pgfpathlineto{\pgfqpoint{2.417320in}{1.015000in}}%
\pgfpathlineto{\pgfqpoint{2.418560in}{1.575000in}}%
\pgfpathlineto{\pgfqpoint{2.419800in}{1.575000in}}%
\pgfpathlineto{\pgfqpoint{2.421040in}{1.085000in}}%
\pgfpathlineto{\pgfqpoint{2.422280in}{1.470000in}}%
\pgfpathlineto{\pgfqpoint{2.423520in}{1.365000in}}%
\pgfpathlineto{\pgfqpoint{2.424760in}{1.015000in}}%
\pgfpathlineto{\pgfqpoint{2.426000in}{1.680000in}}%
\pgfpathlineto{\pgfqpoint{2.427240in}{1.330000in}}%
\pgfpathlineto{\pgfqpoint{2.428480in}{1.575000in}}%
\pgfpathlineto{\pgfqpoint{2.429720in}{1.330000in}}%
\pgfpathlineto{\pgfqpoint{2.432200in}{1.575000in}}%
\pgfpathlineto{\pgfqpoint{2.434680in}{1.295000in}}%
\pgfpathlineto{\pgfqpoint{2.435920in}{1.400000in}}%
\pgfpathlineto{\pgfqpoint{2.437160in}{1.890000in}}%
\pgfpathlineto{\pgfqpoint{2.438400in}{1.505000in}}%
\pgfpathlineto{\pgfqpoint{2.440880in}{1.435000in}}%
\pgfpathlineto{\pgfqpoint{2.442120in}{1.540000in}}%
\pgfpathlineto{\pgfqpoint{2.443360in}{1.120000in}}%
\pgfpathlineto{\pgfqpoint{2.444600in}{1.260000in}}%
\pgfpathlineto{\pgfqpoint{2.445840in}{1.260000in}}%
\pgfpathlineto{\pgfqpoint{2.447080in}{1.785000in}}%
\pgfpathlineto{\pgfqpoint{2.449560in}{1.435000in}}%
\pgfpathlineto{\pgfqpoint{2.450800in}{1.190000in}}%
\pgfpathlineto{\pgfqpoint{2.452040in}{1.330000in}}%
\pgfpathlineto{\pgfqpoint{2.453280in}{1.155000in}}%
\pgfpathlineto{\pgfqpoint{2.454520in}{1.820000in}}%
\pgfpathlineto{\pgfqpoint{2.455760in}{1.645000in}}%
\pgfpathlineto{\pgfqpoint{2.457000in}{1.785000in}}%
\pgfpathlineto{\pgfqpoint{2.458240in}{1.575000in}}%
\pgfpathlineto{\pgfqpoint{2.459480in}{1.120000in}}%
\pgfpathlineto{\pgfqpoint{2.460720in}{1.750000in}}%
\pgfpathlineto{\pgfqpoint{2.463200in}{1.575000in}}%
\pgfpathlineto{\pgfqpoint{2.464440in}{1.610000in}}%
\pgfpathlineto{\pgfqpoint{2.465680in}{1.295000in}}%
\pgfpathlineto{\pgfqpoint{2.466920in}{1.330000in}}%
\pgfpathlineto{\pgfqpoint{2.469400in}{1.470000in}}%
\pgfpathlineto{\pgfqpoint{2.470640in}{1.120000in}}%
\pgfpathlineto{\pgfqpoint{2.471880in}{1.575000in}}%
\pgfpathlineto{\pgfqpoint{2.475600in}{1.050000in}}%
\pgfpathlineto{\pgfqpoint{2.478080in}{1.540000in}}%
\pgfpathlineto{\pgfqpoint{2.479320in}{1.400000in}}%
\pgfpathlineto{\pgfqpoint{2.480560in}{1.855000in}}%
\pgfpathlineto{\pgfqpoint{2.483040in}{1.050000in}}%
\pgfpathlineto{\pgfqpoint{2.484280in}{1.330000in}}%
\pgfpathlineto{\pgfqpoint{2.485520in}{0.980000in}}%
\pgfpathlineto{\pgfqpoint{2.486760in}{1.540000in}}%
\pgfpathlineto{\pgfqpoint{2.488000in}{1.190000in}}%
\pgfpathlineto{\pgfqpoint{2.489240in}{1.330000in}}%
\pgfpathlineto{\pgfqpoint{2.490480in}{1.820000in}}%
\pgfpathlineto{\pgfqpoint{2.491720in}{1.785000in}}%
\pgfpathlineto{\pgfqpoint{2.492960in}{1.365000in}}%
\pgfpathlineto{\pgfqpoint{2.494200in}{1.505000in}}%
\pgfpathlineto{\pgfqpoint{2.495440in}{1.260000in}}%
\pgfpathlineto{\pgfqpoint{2.496680in}{1.645000in}}%
\pgfpathlineto{\pgfqpoint{2.497920in}{1.050000in}}%
\pgfpathlineto{\pgfqpoint{2.499160in}{1.680000in}}%
\pgfpathlineto{\pgfqpoint{2.501640in}{1.190000in}}%
\pgfpathlineto{\pgfqpoint{2.502880in}{1.610000in}}%
\pgfpathlineto{\pgfqpoint{2.504120in}{1.435000in}}%
\pgfpathlineto{\pgfqpoint{2.505360in}{1.645000in}}%
\pgfpathlineto{\pgfqpoint{2.506600in}{1.295000in}}%
\pgfpathlineto{\pgfqpoint{2.507840in}{1.575000in}}%
\pgfpathlineto{\pgfqpoint{2.509080in}{1.400000in}}%
\pgfpathlineto{\pgfqpoint{2.510320in}{1.750000in}}%
\pgfpathlineto{\pgfqpoint{2.512800in}{1.365000in}}%
\pgfpathlineto{\pgfqpoint{2.514040in}{1.400000in}}%
\pgfpathlineto{\pgfqpoint{2.515280in}{2.135000in}}%
\pgfpathlineto{\pgfqpoint{2.520240in}{1.050000in}}%
\pgfpathlineto{\pgfqpoint{2.521480in}{1.295000in}}%
\pgfpathlineto{\pgfqpoint{2.522720in}{1.260000in}}%
\pgfpathlineto{\pgfqpoint{2.523960in}{1.295000in}}%
\pgfpathlineto{\pgfqpoint{2.525200in}{1.505000in}}%
\pgfpathlineto{\pgfqpoint{2.526440in}{1.400000in}}%
\pgfpathlineto{\pgfqpoint{2.527680in}{1.505000in}}%
\pgfpathlineto{\pgfqpoint{2.528920in}{1.505000in}}%
\pgfpathlineto{\pgfqpoint{2.530160in}{0.910000in}}%
\pgfpathlineto{\pgfqpoint{2.531400in}{1.085000in}}%
\pgfpathlineto{\pgfqpoint{2.532640in}{1.540000in}}%
\pgfpathlineto{\pgfqpoint{2.533880in}{1.330000in}}%
\pgfpathlineto{\pgfqpoint{2.535120in}{1.365000in}}%
\pgfpathlineto{\pgfqpoint{2.536360in}{1.295000in}}%
\pgfpathlineto{\pgfqpoint{2.537600in}{1.365000in}}%
\pgfpathlineto{\pgfqpoint{2.538840in}{0.805000in}}%
\pgfpathlineto{\pgfqpoint{2.540080in}{1.435000in}}%
\pgfpathlineto{\pgfqpoint{2.541320in}{1.330000in}}%
\pgfpathlineto{\pgfqpoint{2.542560in}{1.505000in}}%
\pgfpathlineto{\pgfqpoint{2.543800in}{0.945000in}}%
\pgfpathlineto{\pgfqpoint{2.545040in}{1.435000in}}%
\pgfpathlineto{\pgfqpoint{2.546280in}{1.155000in}}%
\pgfpathlineto{\pgfqpoint{2.547520in}{1.785000in}}%
\pgfpathlineto{\pgfqpoint{2.548760in}{1.295000in}}%
\pgfpathlineto{\pgfqpoint{2.550000in}{1.680000in}}%
\pgfpathlineto{\pgfqpoint{2.551240in}{1.365000in}}%
\pgfpathlineto{\pgfqpoint{2.552480in}{1.435000in}}%
\pgfpathlineto{\pgfqpoint{2.553720in}{1.330000in}}%
\pgfpathlineto{\pgfqpoint{2.554960in}{1.400000in}}%
\pgfpathlineto{\pgfqpoint{2.556200in}{0.980000in}}%
\pgfpathlineto{\pgfqpoint{2.557440in}{1.610000in}}%
\pgfpathlineto{\pgfqpoint{2.558680in}{1.470000in}}%
\pgfpathlineto{\pgfqpoint{2.559920in}{1.680000in}}%
\pgfpathlineto{\pgfqpoint{2.562400in}{1.085000in}}%
\pgfpathlineto{\pgfqpoint{2.564880in}{1.540000in}}%
\pgfpathlineto{\pgfqpoint{2.566120in}{1.260000in}}%
\pgfpathlineto{\pgfqpoint{2.567360in}{1.610000in}}%
\pgfpathlineto{\pgfqpoint{2.568600in}{1.260000in}}%
\pgfpathlineto{\pgfqpoint{2.569840in}{1.260000in}}%
\pgfpathlineto{\pgfqpoint{2.571080in}{1.505000in}}%
\pgfpathlineto{\pgfqpoint{2.572320in}{1.260000in}}%
\pgfpathlineto{\pgfqpoint{2.574800in}{1.610000in}}%
\pgfpathlineto{\pgfqpoint{2.576040in}{0.770000in}}%
\pgfpathlineto{\pgfqpoint{2.577280in}{0.875000in}}%
\pgfpathlineto{\pgfqpoint{2.578520in}{1.470000in}}%
\pgfpathlineto{\pgfqpoint{2.579760in}{1.435000in}}%
\pgfpathlineto{\pgfqpoint{2.582240in}{1.855000in}}%
\pgfpathlineto{\pgfqpoint{2.584720in}{1.540000in}}%
\pgfpathlineto{\pgfqpoint{2.585960in}{1.785000in}}%
\pgfpathlineto{\pgfqpoint{2.587200in}{1.715000in}}%
\pgfpathlineto{\pgfqpoint{2.588440in}{1.260000in}}%
\pgfpathlineto{\pgfqpoint{2.589680in}{1.645000in}}%
\pgfpathlineto{\pgfqpoint{2.590920in}{1.295000in}}%
\pgfpathlineto{\pgfqpoint{2.592160in}{1.330000in}}%
\pgfpathlineto{\pgfqpoint{2.593400in}{1.295000in}}%
\pgfpathlineto{\pgfqpoint{2.594640in}{1.785000in}}%
\pgfpathlineto{\pgfqpoint{2.597120in}{1.610000in}}%
\pgfpathlineto{\pgfqpoint{2.598360in}{1.785000in}}%
\pgfpathlineto{\pgfqpoint{2.599600in}{1.330000in}}%
\pgfpathlineto{\pgfqpoint{2.600840in}{1.400000in}}%
\pgfpathlineto{\pgfqpoint{2.602080in}{1.365000in}}%
\pgfpathlineto{\pgfqpoint{2.603320in}{1.120000in}}%
\pgfpathlineto{\pgfqpoint{2.604560in}{1.400000in}}%
\pgfpathlineto{\pgfqpoint{2.605800in}{0.735000in}}%
\pgfpathlineto{\pgfqpoint{2.607040in}{1.470000in}}%
\pgfpathlineto{\pgfqpoint{2.608280in}{1.505000in}}%
\pgfpathlineto{\pgfqpoint{2.609520in}{1.190000in}}%
\pgfpathlineto{\pgfqpoint{2.612000in}{1.400000in}}%
\pgfpathlineto{\pgfqpoint{2.613240in}{1.365000in}}%
\pgfpathlineto{\pgfqpoint{2.614480in}{1.540000in}}%
\pgfpathlineto{\pgfqpoint{2.615720in}{1.190000in}}%
\pgfpathlineto{\pgfqpoint{2.616960in}{1.610000in}}%
\pgfpathlineto{\pgfqpoint{2.619440in}{1.120000in}}%
\pgfpathlineto{\pgfqpoint{2.620680in}{1.505000in}}%
\pgfpathlineto{\pgfqpoint{2.621920in}{1.505000in}}%
\pgfpathlineto{\pgfqpoint{2.623160in}{1.225000in}}%
\pgfpathlineto{\pgfqpoint{2.624400in}{1.435000in}}%
\pgfpathlineto{\pgfqpoint{2.625640in}{1.120000in}}%
\pgfpathlineto{\pgfqpoint{2.628120in}{1.365000in}}%
\pgfpathlineto{\pgfqpoint{2.629360in}{1.505000in}}%
\pgfpathlineto{\pgfqpoint{2.630600in}{1.365000in}}%
\pgfpathlineto{\pgfqpoint{2.631840in}{1.680000in}}%
\pgfpathlineto{\pgfqpoint{2.633080in}{1.225000in}}%
\pgfpathlineto{\pgfqpoint{2.635560in}{1.715000in}}%
\pgfpathlineto{\pgfqpoint{2.636800in}{1.750000in}}%
\pgfpathlineto{\pgfqpoint{2.638040in}{1.645000in}}%
\pgfpathlineto{\pgfqpoint{2.639280in}{1.820000in}}%
\pgfpathlineto{\pgfqpoint{2.640520in}{1.750000in}}%
\pgfpathlineto{\pgfqpoint{2.641760in}{1.365000in}}%
\pgfpathlineto{\pgfqpoint{2.643000in}{1.610000in}}%
\pgfpathlineto{\pgfqpoint{2.645480in}{1.365000in}}%
\pgfpathlineto{\pgfqpoint{2.646720in}{1.505000in}}%
\pgfpathlineto{\pgfqpoint{2.647960in}{1.470000in}}%
\pgfpathlineto{\pgfqpoint{2.649200in}{2.240000in}}%
\pgfpathlineto{\pgfqpoint{2.651680in}{1.610000in}}%
\pgfpathlineto{\pgfqpoint{2.652920in}{1.610000in}}%
\pgfpathlineto{\pgfqpoint{2.654160in}{1.295000in}}%
\pgfpathlineto{\pgfqpoint{2.655400in}{1.435000in}}%
\pgfpathlineto{\pgfqpoint{2.656640in}{1.400000in}}%
\pgfpathlineto{\pgfqpoint{2.659120in}{1.610000in}}%
\pgfpathlineto{\pgfqpoint{2.660360in}{1.295000in}}%
\pgfpathlineto{\pgfqpoint{2.662840in}{1.890000in}}%
\pgfpathlineto{\pgfqpoint{2.664080in}{1.785000in}}%
\pgfpathlineto{\pgfqpoint{2.665320in}{1.855000in}}%
\pgfpathlineto{\pgfqpoint{2.667800in}{1.295000in}}%
\pgfpathlineto{\pgfqpoint{2.670280in}{1.680000in}}%
\pgfpathlineto{\pgfqpoint{2.671520in}{1.050000in}}%
\pgfpathlineto{\pgfqpoint{2.674000in}{1.330000in}}%
\pgfpathlineto{\pgfqpoint{2.675240in}{1.400000in}}%
\pgfpathlineto{\pgfqpoint{2.676480in}{1.890000in}}%
\pgfpathlineto{\pgfqpoint{2.677720in}{1.330000in}}%
\pgfpathlineto{\pgfqpoint{2.678960in}{1.330000in}}%
\pgfpathlineto{\pgfqpoint{2.681440in}{0.945000in}}%
\pgfpathlineto{\pgfqpoint{2.682680in}{1.645000in}}%
\pgfpathlineto{\pgfqpoint{2.683920in}{1.575000in}}%
\pgfpathlineto{\pgfqpoint{2.685160in}{1.925000in}}%
\pgfpathlineto{\pgfqpoint{2.687640in}{1.750000in}}%
\pgfpathlineto{\pgfqpoint{2.688880in}{1.225000in}}%
\pgfpathlineto{\pgfqpoint{2.691360in}{1.645000in}}%
\pgfpathlineto{\pgfqpoint{2.692600in}{1.855000in}}%
\pgfpathlineto{\pgfqpoint{2.693840in}{1.470000in}}%
\pgfpathlineto{\pgfqpoint{2.695080in}{1.715000in}}%
\pgfpathlineto{\pgfqpoint{2.696320in}{1.680000in}}%
\pgfpathlineto{\pgfqpoint{2.697560in}{1.855000in}}%
\pgfpathlineto{\pgfqpoint{2.700040in}{1.575000in}}%
\pgfpathlineto{\pgfqpoint{2.701280in}{1.505000in}}%
\pgfpathlineto{\pgfqpoint{2.702520in}{2.100000in}}%
\pgfpathlineto{\pgfqpoint{2.703760in}{1.575000in}}%
\pgfpathlineto{\pgfqpoint{2.705000in}{1.575000in}}%
\pgfpathlineto{\pgfqpoint{2.706240in}{1.435000in}}%
\pgfpathlineto{\pgfqpoint{2.707480in}{1.715000in}}%
\pgfpathlineto{\pgfqpoint{2.708720in}{1.575000in}}%
\pgfpathlineto{\pgfqpoint{2.709960in}{1.680000in}}%
\pgfpathlineto{\pgfqpoint{2.712440in}{1.505000in}}%
\pgfpathlineto{\pgfqpoint{2.713680in}{1.610000in}}%
\pgfpathlineto{\pgfqpoint{2.714920in}{1.610000in}}%
\pgfpathlineto{\pgfqpoint{2.716160in}{1.925000in}}%
\pgfpathlineto{\pgfqpoint{2.718640in}{1.120000in}}%
\pgfpathlineto{\pgfqpoint{2.719880in}{1.890000in}}%
\pgfpathlineto{\pgfqpoint{2.721120in}{1.540000in}}%
\pgfpathlineto{\pgfqpoint{2.722360in}{1.715000in}}%
\pgfpathlineto{\pgfqpoint{2.723600in}{1.715000in}}%
\pgfpathlineto{\pgfqpoint{2.724840in}{1.330000in}}%
\pgfpathlineto{\pgfqpoint{2.726080in}{1.820000in}}%
\pgfpathlineto{\pgfqpoint{2.728560in}{1.295000in}}%
\pgfpathlineto{\pgfqpoint{2.729800in}{1.435000in}}%
\pgfpathlineto{\pgfqpoint{2.731040in}{1.260000in}}%
\pgfpathlineto{\pgfqpoint{2.734760in}{1.645000in}}%
\pgfpathlineto{\pgfqpoint{2.738480in}{0.980000in}}%
\pgfpathlineto{\pgfqpoint{2.739720in}{1.680000in}}%
\pgfpathlineto{\pgfqpoint{2.740960in}{1.750000in}}%
\pgfpathlineto{\pgfqpoint{2.743440in}{1.365000in}}%
\pgfpathlineto{\pgfqpoint{2.744680in}{1.085000in}}%
\pgfpathlineto{\pgfqpoint{2.745920in}{1.435000in}}%
\pgfpathlineto{\pgfqpoint{2.748400in}{1.155000in}}%
\pgfpathlineto{\pgfqpoint{2.750880in}{1.645000in}}%
\pgfpathlineto{\pgfqpoint{2.752120in}{1.225000in}}%
\pgfpathlineto{\pgfqpoint{2.754600in}{1.505000in}}%
\pgfpathlineto{\pgfqpoint{2.755840in}{1.400000in}}%
\pgfpathlineto{\pgfqpoint{2.757080in}{1.435000in}}%
\pgfpathlineto{\pgfqpoint{2.758320in}{1.540000in}}%
\pgfpathlineto{\pgfqpoint{2.759560in}{1.540000in}}%
\pgfpathlineto{\pgfqpoint{2.760800in}{1.680000in}}%
\pgfpathlineto{\pgfqpoint{2.763280in}{1.295000in}}%
\pgfpathlineto{\pgfqpoint{2.764520in}{1.260000in}}%
\pgfpathlineto{\pgfqpoint{2.765760in}{1.540000in}}%
\pgfpathlineto{\pgfqpoint{2.767000in}{1.050000in}}%
\pgfpathlineto{\pgfqpoint{2.768240in}{1.155000in}}%
\pgfpathlineto{\pgfqpoint{2.769480in}{1.505000in}}%
\pgfpathlineto{\pgfqpoint{2.770720in}{1.225000in}}%
\pgfpathlineto{\pgfqpoint{2.771960in}{1.750000in}}%
\pgfpathlineto{\pgfqpoint{2.773200in}{1.225000in}}%
\pgfpathlineto{\pgfqpoint{2.774440in}{1.260000in}}%
\pgfpathlineto{\pgfqpoint{2.775680in}{1.470000in}}%
\pgfpathlineto{\pgfqpoint{2.776920in}{1.330000in}}%
\pgfpathlineto{\pgfqpoint{2.778160in}{1.715000in}}%
\pgfpathlineto{\pgfqpoint{2.779400in}{1.470000in}}%
\pgfpathlineto{\pgfqpoint{2.781880in}{1.890000in}}%
\pgfpathlineto{\pgfqpoint{2.784360in}{1.260000in}}%
\pgfpathlineto{\pgfqpoint{2.785600in}{1.190000in}}%
\pgfpathlineto{\pgfqpoint{2.786840in}{1.505000in}}%
\pgfpathlineto{\pgfqpoint{2.790560in}{1.050000in}}%
\pgfpathlineto{\pgfqpoint{2.791800in}{1.505000in}}%
\pgfpathlineto{\pgfqpoint{2.793040in}{0.875000in}}%
\pgfpathlineto{\pgfqpoint{2.794280in}{1.330000in}}%
\pgfpathlineto{\pgfqpoint{2.795520in}{0.980000in}}%
\pgfpathlineto{\pgfqpoint{2.798000in}{1.715000in}}%
\pgfpathlineto{\pgfqpoint{2.800480in}{0.910000in}}%
\pgfpathlineto{\pgfqpoint{2.801720in}{1.330000in}}%
\pgfpathlineto{\pgfqpoint{2.802960in}{1.120000in}}%
\pgfpathlineto{\pgfqpoint{2.804200in}{1.645000in}}%
\pgfpathlineto{\pgfqpoint{2.806680in}{1.190000in}}%
\pgfpathlineto{\pgfqpoint{2.807920in}{1.470000in}}%
\pgfpathlineto{\pgfqpoint{2.809160in}{1.330000in}}%
\pgfpathlineto{\pgfqpoint{2.810400in}{1.750000in}}%
\pgfpathlineto{\pgfqpoint{2.811640in}{1.120000in}}%
\pgfpathlineto{\pgfqpoint{2.812880in}{1.540000in}}%
\pgfpathlineto{\pgfqpoint{2.814120in}{1.120000in}}%
\pgfpathlineto{\pgfqpoint{2.815360in}{1.785000in}}%
\pgfpathlineto{\pgfqpoint{2.816600in}{1.610000in}}%
\pgfpathlineto{\pgfqpoint{2.817840in}{1.785000in}}%
\pgfpathlineto{\pgfqpoint{2.819080in}{1.330000in}}%
\pgfpathlineto{\pgfqpoint{2.820320in}{1.575000in}}%
\pgfpathlineto{\pgfqpoint{2.822800in}{1.050000in}}%
\pgfpathlineto{\pgfqpoint{2.824040in}{1.715000in}}%
\pgfpathlineto{\pgfqpoint{2.825280in}{0.805000in}}%
\pgfpathlineto{\pgfqpoint{2.826520in}{1.260000in}}%
\pgfpathlineto{\pgfqpoint{2.827760in}{1.050000in}}%
\pgfpathlineto{\pgfqpoint{2.830240in}{1.575000in}}%
\pgfpathlineto{\pgfqpoint{2.831480in}{1.575000in}}%
\pgfpathlineto{\pgfqpoint{2.832720in}{1.225000in}}%
\pgfpathlineto{\pgfqpoint{2.833960in}{1.855000in}}%
\pgfpathlineto{\pgfqpoint{2.835200in}{1.050000in}}%
\pgfpathlineto{\pgfqpoint{2.837680in}{1.400000in}}%
\pgfpathlineto{\pgfqpoint{2.838920in}{0.980000in}}%
\pgfpathlineto{\pgfqpoint{2.840160in}{1.435000in}}%
\pgfpathlineto{\pgfqpoint{2.841400in}{1.400000in}}%
\pgfpathlineto{\pgfqpoint{2.843880in}{1.610000in}}%
\pgfpathlineto{\pgfqpoint{2.845120in}{1.400000in}}%
\pgfpathlineto{\pgfqpoint{2.846360in}{0.840000in}}%
\pgfpathlineto{\pgfqpoint{2.848840in}{1.750000in}}%
\pgfpathlineto{\pgfqpoint{2.850080in}{0.980000in}}%
\pgfpathlineto{\pgfqpoint{2.851320in}{1.715000in}}%
\pgfpathlineto{\pgfqpoint{2.852560in}{1.190000in}}%
\pgfpathlineto{\pgfqpoint{2.855040in}{1.330000in}}%
\pgfpathlineto{\pgfqpoint{2.856280in}{1.085000in}}%
\pgfpathlineto{\pgfqpoint{2.857520in}{1.540000in}}%
\pgfpathlineto{\pgfqpoint{2.858760in}{1.225000in}}%
\pgfpathlineto{\pgfqpoint{2.860000in}{1.330000in}}%
\pgfpathlineto{\pgfqpoint{2.861240in}{1.540000in}}%
\pgfpathlineto{\pgfqpoint{2.863720in}{1.260000in}}%
\pgfpathlineto{\pgfqpoint{2.866200in}{1.085000in}}%
\pgfpathlineto{\pgfqpoint{2.867440in}{1.575000in}}%
\pgfpathlineto{\pgfqpoint{2.868680in}{1.190000in}}%
\pgfpathlineto{\pgfqpoint{2.869920in}{1.155000in}}%
\pgfpathlineto{\pgfqpoint{2.871160in}{1.470000in}}%
\pgfpathlineto{\pgfqpoint{2.872400in}{1.365000in}}%
\pgfpathlineto{\pgfqpoint{2.874880in}{1.960000in}}%
\pgfpathlineto{\pgfqpoint{2.876120in}{1.820000in}}%
\pgfpathlineto{\pgfqpoint{2.877360in}{1.120000in}}%
\pgfpathlineto{\pgfqpoint{2.879840in}{1.785000in}}%
\pgfpathlineto{\pgfqpoint{2.881080in}{1.820000in}}%
\pgfpathlineto{\pgfqpoint{2.882320in}{1.715000in}}%
\pgfpathlineto{\pgfqpoint{2.884800in}{1.400000in}}%
\pgfpathlineto{\pgfqpoint{2.886040in}{1.645000in}}%
\pgfpathlineto{\pgfqpoint{2.887280in}{1.365000in}}%
\pgfpathlineto{\pgfqpoint{2.888520in}{1.575000in}}%
\pgfpathlineto{\pgfqpoint{2.889760in}{1.400000in}}%
\pgfpathlineto{\pgfqpoint{2.891000in}{1.400000in}}%
\pgfpathlineto{\pgfqpoint{2.892240in}{1.085000in}}%
\pgfpathlineto{\pgfqpoint{2.893480in}{1.260000in}}%
\pgfpathlineto{\pgfqpoint{2.894720in}{1.155000in}}%
\pgfpathlineto{\pgfqpoint{2.898440in}{1.890000in}}%
\pgfpathlineto{\pgfqpoint{2.899680in}{1.295000in}}%
\pgfpathlineto{\pgfqpoint{2.900920in}{1.365000in}}%
\pgfpathlineto{\pgfqpoint{2.903400in}{1.085000in}}%
\pgfpathlineto{\pgfqpoint{2.904640in}{1.680000in}}%
\pgfpathlineto{\pgfqpoint{2.905880in}{1.295000in}}%
\pgfpathlineto{\pgfqpoint{2.907120in}{1.645000in}}%
\pgfpathlineto{\pgfqpoint{2.908360in}{1.190000in}}%
\pgfpathlineto{\pgfqpoint{2.909600in}{1.680000in}}%
\pgfpathlineto{\pgfqpoint{2.910840in}{1.575000in}}%
\pgfpathlineto{\pgfqpoint{2.912080in}{1.680000in}}%
\pgfpathlineto{\pgfqpoint{2.913320in}{1.260000in}}%
\pgfpathlineto{\pgfqpoint{2.914560in}{1.225000in}}%
\pgfpathlineto{\pgfqpoint{2.917040in}{1.540000in}}%
\pgfpathlineto{\pgfqpoint{2.918280in}{1.015000in}}%
\pgfpathlineto{\pgfqpoint{2.920760in}{1.540000in}}%
\pgfpathlineto{\pgfqpoint{2.922000in}{1.820000in}}%
\pgfpathlineto{\pgfqpoint{2.923240in}{1.610000in}}%
\pgfpathlineto{\pgfqpoint{2.924480in}{1.855000in}}%
\pgfpathlineto{\pgfqpoint{2.925720in}{1.260000in}}%
\pgfpathlineto{\pgfqpoint{2.928200in}{1.645000in}}%
\pgfpathlineto{\pgfqpoint{2.929440in}{1.575000in}}%
\pgfpathlineto{\pgfqpoint{2.930680in}{1.295000in}}%
\pgfpathlineto{\pgfqpoint{2.931920in}{1.680000in}}%
\pgfpathlineto{\pgfqpoint{2.933160in}{1.365000in}}%
\pgfpathlineto{\pgfqpoint{2.934400in}{1.925000in}}%
\pgfpathlineto{\pgfqpoint{2.935640in}{1.610000in}}%
\pgfpathlineto{\pgfqpoint{2.936880in}{1.820000in}}%
\pgfpathlineto{\pgfqpoint{2.939360in}{0.910000in}}%
\pgfpathlineto{\pgfqpoint{2.941840in}{1.505000in}}%
\pgfpathlineto{\pgfqpoint{2.943080in}{1.645000in}}%
\pgfpathlineto{\pgfqpoint{2.944320in}{1.295000in}}%
\pgfpathlineto{\pgfqpoint{2.946800in}{1.680000in}}%
\pgfpathlineto{\pgfqpoint{2.949280in}{1.015000in}}%
\pgfpathlineto{\pgfqpoint{2.950520in}{1.750000in}}%
\pgfpathlineto{\pgfqpoint{2.951760in}{1.750000in}}%
\pgfpathlineto{\pgfqpoint{2.954240in}{1.400000in}}%
\pgfpathlineto{\pgfqpoint{2.955480in}{1.190000in}}%
\pgfpathlineto{\pgfqpoint{2.957960in}{1.680000in}}%
\pgfpathlineto{\pgfqpoint{2.960440in}{1.190000in}}%
\pgfpathlineto{\pgfqpoint{2.964160in}{1.820000in}}%
\pgfpathlineto{\pgfqpoint{2.965400in}{1.120000in}}%
\pgfpathlineto{\pgfqpoint{2.966640in}{1.575000in}}%
\pgfpathlineto{\pgfqpoint{2.969120in}{1.225000in}}%
\pgfpathlineto{\pgfqpoint{2.971600in}{1.365000in}}%
\pgfpathlineto{\pgfqpoint{2.972840in}{1.610000in}}%
\pgfpathlineto{\pgfqpoint{2.974080in}{1.505000in}}%
\pgfpathlineto{\pgfqpoint{2.975320in}{1.505000in}}%
\pgfpathlineto{\pgfqpoint{2.976560in}{1.120000in}}%
\pgfpathlineto{\pgfqpoint{2.977800in}{1.470000in}}%
\pgfpathlineto{\pgfqpoint{2.979040in}{1.155000in}}%
\pgfpathlineto{\pgfqpoint{2.980280in}{1.260000in}}%
\pgfpathlineto{\pgfqpoint{2.982760in}{1.540000in}}%
\pgfpathlineto{\pgfqpoint{2.984000in}{1.155000in}}%
\pgfpathlineto{\pgfqpoint{2.985240in}{1.435000in}}%
\pgfpathlineto{\pgfqpoint{2.986480in}{1.225000in}}%
\pgfpathlineto{\pgfqpoint{2.987720in}{1.260000in}}%
\pgfpathlineto{\pgfqpoint{2.988960in}{0.875000in}}%
\pgfpathlineto{\pgfqpoint{2.990200in}{1.400000in}}%
\pgfpathlineto{\pgfqpoint{2.991440in}{1.400000in}}%
\pgfpathlineto{\pgfqpoint{2.992680in}{1.715000in}}%
\pgfpathlineto{\pgfqpoint{2.993920in}{1.330000in}}%
\pgfpathlineto{\pgfqpoint{2.995160in}{1.505000in}}%
\pgfpathlineto{\pgfqpoint{2.996400in}{1.400000in}}%
\pgfpathlineto{\pgfqpoint{2.997640in}{1.435000in}}%
\pgfpathlineto{\pgfqpoint{2.998880in}{1.610000in}}%
\pgfpathlineto{\pgfqpoint{3.000120in}{1.400000in}}%
\pgfpathlineto{\pgfqpoint{3.001360in}{1.470000in}}%
\pgfpathlineto{\pgfqpoint{3.002600in}{1.435000in}}%
\pgfpathlineto{\pgfqpoint{3.003840in}{1.890000in}}%
\pgfpathlineto{\pgfqpoint{3.006320in}{1.610000in}}%
\pgfpathlineto{\pgfqpoint{3.007560in}{1.925000in}}%
\pgfpathlineto{\pgfqpoint{3.010040in}{1.225000in}}%
\pgfpathlineto{\pgfqpoint{3.011280in}{1.295000in}}%
\pgfpathlineto{\pgfqpoint{3.012520in}{1.295000in}}%
\pgfpathlineto{\pgfqpoint{3.013760in}{1.260000in}}%
\pgfpathlineto{\pgfqpoint{3.015000in}{1.400000in}}%
\pgfpathlineto{\pgfqpoint{3.016240in}{1.680000in}}%
\pgfpathlineto{\pgfqpoint{3.017480in}{1.330000in}}%
\pgfpathlineto{\pgfqpoint{3.018720in}{1.330000in}}%
\pgfpathlineto{\pgfqpoint{3.019960in}{1.505000in}}%
\pgfpathlineto{\pgfqpoint{3.021200in}{1.295000in}}%
\pgfpathlineto{\pgfqpoint{3.022440in}{1.505000in}}%
\pgfpathlineto{\pgfqpoint{3.023680in}{1.330000in}}%
\pgfpathlineto{\pgfqpoint{3.024920in}{1.365000in}}%
\pgfpathlineto{\pgfqpoint{3.026160in}{1.470000in}}%
\pgfpathlineto{\pgfqpoint{3.027400in}{1.085000in}}%
\pgfpathlineto{\pgfqpoint{3.029880in}{1.680000in}}%
\pgfpathlineto{\pgfqpoint{3.031120in}{1.575000in}}%
\pgfpathlineto{\pgfqpoint{3.032360in}{1.085000in}}%
\pgfpathlineto{\pgfqpoint{3.033600in}{1.505000in}}%
\pgfpathlineto{\pgfqpoint{3.034840in}{1.225000in}}%
\pgfpathlineto{\pgfqpoint{3.036080in}{1.470000in}}%
\pgfpathlineto{\pgfqpoint{3.037320in}{1.120000in}}%
\pgfpathlineto{\pgfqpoint{3.038560in}{1.365000in}}%
\pgfpathlineto{\pgfqpoint{3.039800in}{1.330000in}}%
\pgfpathlineto{\pgfqpoint{3.041040in}{1.015000in}}%
\pgfpathlineto{\pgfqpoint{3.042280in}{1.295000in}}%
\pgfpathlineto{\pgfqpoint{3.043520in}{1.260000in}}%
\pgfpathlineto{\pgfqpoint{3.044760in}{1.610000in}}%
\pgfpathlineto{\pgfqpoint{3.047240in}{1.225000in}}%
\pgfpathlineto{\pgfqpoint{3.048480in}{1.925000in}}%
\pgfpathlineto{\pgfqpoint{3.049720in}{1.155000in}}%
\pgfpathlineto{\pgfqpoint{3.052200in}{1.470000in}}%
\pgfpathlineto{\pgfqpoint{3.053440in}{1.540000in}}%
\pgfpathlineto{\pgfqpoint{3.054680in}{1.540000in}}%
\pgfpathlineto{\pgfqpoint{3.055920in}{1.330000in}}%
\pgfpathlineto{\pgfqpoint{3.057160in}{1.575000in}}%
\pgfpathlineto{\pgfqpoint{3.058400in}{1.225000in}}%
\pgfpathlineto{\pgfqpoint{3.059640in}{1.190000in}}%
\pgfpathlineto{\pgfqpoint{3.060880in}{1.610000in}}%
\pgfpathlineto{\pgfqpoint{3.062120in}{1.260000in}}%
\pgfpathlineto{\pgfqpoint{3.063360in}{1.365000in}}%
\pgfpathlineto{\pgfqpoint{3.064600in}{1.260000in}}%
\pgfpathlineto{\pgfqpoint{3.065840in}{1.610000in}}%
\pgfpathlineto{\pgfqpoint{3.067080in}{1.330000in}}%
\pgfpathlineto{\pgfqpoint{3.068320in}{1.400000in}}%
\pgfpathlineto{\pgfqpoint{3.069560in}{1.295000in}}%
\pgfpathlineto{\pgfqpoint{3.070800in}{1.400000in}}%
\pgfpathlineto{\pgfqpoint{3.072040in}{1.715000in}}%
\pgfpathlineto{\pgfqpoint{3.073280in}{1.190000in}}%
\pgfpathlineto{\pgfqpoint{3.074520in}{1.820000in}}%
\pgfpathlineto{\pgfqpoint{3.075760in}{1.330000in}}%
\pgfpathlineto{\pgfqpoint{3.077000in}{1.505000in}}%
\pgfpathlineto{\pgfqpoint{3.078240in}{1.330000in}}%
\pgfpathlineto{\pgfqpoint{3.079480in}{1.610000in}}%
\pgfpathlineto{\pgfqpoint{3.080720in}{1.120000in}}%
\pgfpathlineto{\pgfqpoint{3.083200in}{1.470000in}}%
\pgfpathlineto{\pgfqpoint{3.084440in}{1.085000in}}%
\pgfpathlineto{\pgfqpoint{3.085680in}{1.400000in}}%
\pgfpathlineto{\pgfqpoint{3.086920in}{1.365000in}}%
\pgfpathlineto{\pgfqpoint{3.088160in}{1.190000in}}%
\pgfpathlineto{\pgfqpoint{3.089400in}{1.225000in}}%
\pgfpathlineto{\pgfqpoint{3.090640in}{1.050000in}}%
\pgfpathlineto{\pgfqpoint{3.091880in}{0.700000in}}%
\pgfpathlineto{\pgfqpoint{3.095600in}{1.610000in}}%
\pgfpathlineto{\pgfqpoint{3.096840in}{1.470000in}}%
\pgfpathlineto{\pgfqpoint{3.099320in}{1.855000in}}%
\pgfpathlineto{\pgfqpoint{3.100560in}{1.715000in}}%
\pgfpathlineto{\pgfqpoint{3.101800in}{1.295000in}}%
\pgfpathlineto{\pgfqpoint{3.103040in}{1.855000in}}%
\pgfpathlineto{\pgfqpoint{3.104280in}{1.085000in}}%
\pgfpathlineto{\pgfqpoint{3.105520in}{1.715000in}}%
\pgfpathlineto{\pgfqpoint{3.106760in}{1.645000in}}%
\pgfpathlineto{\pgfqpoint{3.108000in}{1.295000in}}%
\pgfpathlineto{\pgfqpoint{3.110480in}{1.540000in}}%
\pgfpathlineto{\pgfqpoint{3.111720in}{1.470000in}}%
\pgfpathlineto{\pgfqpoint{3.112960in}{1.260000in}}%
\pgfpathlineto{\pgfqpoint{3.114200in}{1.540000in}}%
\pgfpathlineto{\pgfqpoint{3.115440in}{1.015000in}}%
\pgfpathlineto{\pgfqpoint{3.116680in}{1.365000in}}%
\pgfpathlineto{\pgfqpoint{3.117920in}{1.260000in}}%
\pgfpathlineto{\pgfqpoint{3.119160in}{1.610000in}}%
\pgfpathlineto{\pgfqpoint{3.122880in}{1.085000in}}%
\pgfpathlineto{\pgfqpoint{3.124120in}{1.190000in}}%
\pgfpathlineto{\pgfqpoint{3.126600in}{1.715000in}}%
\pgfpathlineto{\pgfqpoint{3.129080in}{1.995000in}}%
\pgfpathlineto{\pgfqpoint{3.130320in}{1.750000in}}%
\pgfpathlineto{\pgfqpoint{3.131560in}{2.100000in}}%
\pgfpathlineto{\pgfqpoint{3.132800in}{1.680000in}}%
\pgfpathlineto{\pgfqpoint{3.134040in}{1.750000in}}%
\pgfpathlineto{\pgfqpoint{3.135280in}{1.645000in}}%
\pgfpathlineto{\pgfqpoint{3.136520in}{1.295000in}}%
\pgfpathlineto{\pgfqpoint{3.137760in}{1.435000in}}%
\pgfpathlineto{\pgfqpoint{3.139000in}{1.015000in}}%
\pgfpathlineto{\pgfqpoint{3.141480in}{1.820000in}}%
\pgfpathlineto{\pgfqpoint{3.143960in}{1.155000in}}%
\pgfpathlineto{\pgfqpoint{3.145200in}{1.470000in}}%
\pgfpathlineto{\pgfqpoint{3.146440in}{1.155000in}}%
\pgfpathlineto{\pgfqpoint{3.147680in}{1.785000in}}%
\pgfpathlineto{\pgfqpoint{3.148920in}{1.190000in}}%
\pgfpathlineto{\pgfqpoint{3.151400in}{1.470000in}}%
\pgfpathlineto{\pgfqpoint{3.152640in}{1.260000in}}%
\pgfpathlineto{\pgfqpoint{3.153880in}{1.505000in}}%
\pgfpathlineto{\pgfqpoint{3.156360in}{1.400000in}}%
\pgfpathlineto{\pgfqpoint{3.158840in}{0.980000in}}%
\pgfpathlineto{\pgfqpoint{3.160080in}{1.750000in}}%
\pgfpathlineto{\pgfqpoint{3.161320in}{1.365000in}}%
\pgfpathlineto{\pgfqpoint{3.162560in}{1.610000in}}%
\pgfpathlineto{\pgfqpoint{3.163800in}{1.260000in}}%
\pgfpathlineto{\pgfqpoint{3.165040in}{1.295000in}}%
\pgfpathlineto{\pgfqpoint{3.166280in}{1.645000in}}%
\pgfpathlineto{\pgfqpoint{3.167520in}{1.505000in}}%
\pgfpathlineto{\pgfqpoint{3.168760in}{1.750000in}}%
\pgfpathlineto{\pgfqpoint{3.171240in}{1.225000in}}%
\pgfpathlineto{\pgfqpoint{3.173720in}{1.645000in}}%
\pgfpathlineto{\pgfqpoint{3.174960in}{1.295000in}}%
\pgfpathlineto{\pgfqpoint{3.176200in}{1.295000in}}%
\pgfpathlineto{\pgfqpoint{3.177440in}{1.540000in}}%
\pgfpathlineto{\pgfqpoint{3.178680in}{1.085000in}}%
\pgfpathlineto{\pgfqpoint{3.179920in}{1.435000in}}%
\pgfpathlineto{\pgfqpoint{3.181160in}{1.400000in}}%
\pgfpathlineto{\pgfqpoint{3.182400in}{1.330000in}}%
\pgfpathlineto{\pgfqpoint{3.183640in}{1.155000in}}%
\pgfpathlineto{\pgfqpoint{3.184880in}{1.855000in}}%
\pgfpathlineto{\pgfqpoint{3.187360in}{1.260000in}}%
\pgfpathlineto{\pgfqpoint{3.188600in}{1.785000in}}%
\pgfpathlineto{\pgfqpoint{3.189840in}{1.715000in}}%
\pgfpathlineto{\pgfqpoint{3.191080in}{1.330000in}}%
\pgfpathlineto{\pgfqpoint{3.193560in}{1.715000in}}%
\pgfpathlineto{\pgfqpoint{3.194800in}{1.365000in}}%
\pgfpathlineto{\pgfqpoint{3.196040in}{1.505000in}}%
\pgfpathlineto{\pgfqpoint{3.197280in}{1.435000in}}%
\pgfpathlineto{\pgfqpoint{3.198520in}{1.435000in}}%
\pgfpathlineto{\pgfqpoint{3.199760in}{1.540000in}}%
\pgfpathlineto{\pgfqpoint{3.201000in}{1.120000in}}%
\pgfpathlineto{\pgfqpoint{3.203480in}{1.995000in}}%
\pgfpathlineto{\pgfqpoint{3.204720in}{1.155000in}}%
\pgfpathlineto{\pgfqpoint{3.205960in}{1.785000in}}%
\pgfpathlineto{\pgfqpoint{3.207200in}{1.365000in}}%
\pgfpathlineto{\pgfqpoint{3.208440in}{1.505000in}}%
\pgfpathlineto{\pgfqpoint{3.209680in}{1.155000in}}%
\pgfpathlineto{\pgfqpoint{3.210920in}{1.750000in}}%
\pgfpathlineto{\pgfqpoint{3.212160in}{1.295000in}}%
\pgfpathlineto{\pgfqpoint{3.214640in}{1.995000in}}%
\pgfpathlineto{\pgfqpoint{3.215880in}{1.400000in}}%
\pgfpathlineto{\pgfqpoint{3.217120in}{1.820000in}}%
\pgfpathlineto{\pgfqpoint{3.218360in}{1.225000in}}%
\pgfpathlineto{\pgfqpoint{3.219600in}{1.330000in}}%
\pgfpathlineto{\pgfqpoint{3.220840in}{1.540000in}}%
\pgfpathlineto{\pgfqpoint{3.222080in}{1.540000in}}%
\pgfpathlineto{\pgfqpoint{3.224560in}{1.400000in}}%
\pgfpathlineto{\pgfqpoint{3.225800in}{1.645000in}}%
\pgfpathlineto{\pgfqpoint{3.228280in}{1.085000in}}%
\pgfpathlineto{\pgfqpoint{3.229520in}{1.575000in}}%
\pgfpathlineto{\pgfqpoint{3.230760in}{1.540000in}}%
\pgfpathlineto{\pgfqpoint{3.233240in}{1.085000in}}%
\pgfpathlineto{\pgfqpoint{3.235720in}{1.540000in}}%
\pgfpathlineto{\pgfqpoint{3.236960in}{1.260000in}}%
\pgfpathlineto{\pgfqpoint{3.239440in}{1.610000in}}%
\pgfpathlineto{\pgfqpoint{3.240680in}{1.225000in}}%
\pgfpathlineto{\pgfqpoint{3.241920in}{1.225000in}}%
\pgfpathlineto{\pgfqpoint{3.243160in}{1.330000in}}%
\pgfpathlineto{\pgfqpoint{3.245640in}{1.225000in}}%
\pgfpathlineto{\pgfqpoint{3.246880in}{1.295000in}}%
\pgfpathlineto{\pgfqpoint{3.248120in}{1.785000in}}%
\pgfpathlineto{\pgfqpoint{3.250600in}{1.120000in}}%
\pgfpathlineto{\pgfqpoint{3.251840in}{1.890000in}}%
\pgfpathlineto{\pgfqpoint{3.254320in}{1.540000in}}%
\pgfpathlineto{\pgfqpoint{3.259280in}{0.945000in}}%
\pgfpathlineto{\pgfqpoint{3.263000in}{1.540000in}}%
\pgfpathlineto{\pgfqpoint{3.264240in}{1.610000in}}%
\pgfpathlineto{\pgfqpoint{3.265480in}{1.540000in}}%
\pgfpathlineto{\pgfqpoint{3.266720in}{1.365000in}}%
\pgfpathlineto{\pgfqpoint{3.267960in}{1.610000in}}%
\pgfpathlineto{\pgfqpoint{3.270440in}{1.365000in}}%
\pgfpathlineto{\pgfqpoint{3.272920in}{1.505000in}}%
\pgfpathlineto{\pgfqpoint{3.274160in}{1.645000in}}%
\pgfpathlineto{\pgfqpoint{3.275400in}{1.225000in}}%
\pgfpathlineto{\pgfqpoint{3.276640in}{1.295000in}}%
\pgfpathlineto{\pgfqpoint{3.277880in}{1.540000in}}%
\pgfpathlineto{\pgfqpoint{3.279120in}{1.155000in}}%
\pgfpathlineto{\pgfqpoint{3.281600in}{1.540000in}}%
\pgfpathlineto{\pgfqpoint{3.282840in}{1.750000in}}%
\pgfpathlineto{\pgfqpoint{3.284080in}{1.750000in}}%
\pgfpathlineto{\pgfqpoint{3.285320in}{1.330000in}}%
\pgfpathlineto{\pgfqpoint{3.286560in}{1.400000in}}%
\pgfpathlineto{\pgfqpoint{3.289040in}{1.750000in}}%
\pgfpathlineto{\pgfqpoint{3.290280in}{1.400000in}}%
\pgfpathlineto{\pgfqpoint{3.291520in}{1.680000in}}%
\pgfpathlineto{\pgfqpoint{3.292760in}{1.470000in}}%
\pgfpathlineto{\pgfqpoint{3.295240in}{1.960000in}}%
\pgfpathlineto{\pgfqpoint{3.297720in}{1.400000in}}%
\pgfpathlineto{\pgfqpoint{3.300200in}{1.820000in}}%
\pgfpathlineto{\pgfqpoint{3.302680in}{1.085000in}}%
\pgfpathlineto{\pgfqpoint{3.303920in}{1.295000in}}%
\pgfpathlineto{\pgfqpoint{3.305160in}{1.295000in}}%
\pgfpathlineto{\pgfqpoint{3.306400in}{1.575000in}}%
\pgfpathlineto{\pgfqpoint{3.308880in}{1.120000in}}%
\pgfpathlineto{\pgfqpoint{3.310120in}{1.470000in}}%
\pgfpathlineto{\pgfqpoint{3.311360in}{1.155000in}}%
\pgfpathlineto{\pgfqpoint{3.312600in}{1.715000in}}%
\pgfpathlineto{\pgfqpoint{3.313840in}{1.680000in}}%
\pgfpathlineto{\pgfqpoint{3.315080in}{1.540000in}}%
\pgfpathlineto{\pgfqpoint{3.317560in}{1.225000in}}%
\pgfpathlineto{\pgfqpoint{3.318800in}{1.155000in}}%
\pgfpathlineto{\pgfqpoint{3.320040in}{1.295000in}}%
\pgfpathlineto{\pgfqpoint{3.321280in}{1.190000in}}%
\pgfpathlineto{\pgfqpoint{3.322520in}{1.225000in}}%
\pgfpathlineto{\pgfqpoint{3.325000in}{1.890000in}}%
\pgfpathlineto{\pgfqpoint{3.327480in}{1.295000in}}%
\pgfpathlineto{\pgfqpoint{3.328720in}{1.330000in}}%
\pgfpathlineto{\pgfqpoint{3.329960in}{1.400000in}}%
\pgfpathlineto{\pgfqpoint{3.331200in}{1.750000in}}%
\pgfpathlineto{\pgfqpoint{3.332440in}{1.330000in}}%
\pgfpathlineto{\pgfqpoint{3.333680in}{1.575000in}}%
\pgfpathlineto{\pgfqpoint{3.334920in}{1.260000in}}%
\pgfpathlineto{\pgfqpoint{3.337400in}{1.610000in}}%
\pgfpathlineto{\pgfqpoint{3.338640in}{1.330000in}}%
\pgfpathlineto{\pgfqpoint{3.339880in}{1.855000in}}%
\pgfpathlineto{\pgfqpoint{3.341120in}{1.365000in}}%
\pgfpathlineto{\pgfqpoint{3.342360in}{1.890000in}}%
\pgfpathlineto{\pgfqpoint{3.343600in}{1.540000in}}%
\pgfpathlineto{\pgfqpoint{3.344840in}{1.575000in}}%
\pgfpathlineto{\pgfqpoint{3.346080in}{1.785000in}}%
\pgfpathlineto{\pgfqpoint{3.348560in}{1.540000in}}%
\pgfpathlineto{\pgfqpoint{3.351040in}{1.820000in}}%
\pgfpathlineto{\pgfqpoint{3.353520in}{2.100000in}}%
\pgfpathlineto{\pgfqpoint{3.354760in}{1.470000in}}%
\pgfpathlineto{\pgfqpoint{3.356000in}{1.785000in}}%
\pgfpathlineto{\pgfqpoint{3.358480in}{1.365000in}}%
\pgfpathlineto{\pgfqpoint{3.359720in}{1.610000in}}%
\pgfpathlineto{\pgfqpoint{3.360960in}{1.610000in}}%
\pgfpathlineto{\pgfqpoint{3.362200in}{1.505000in}}%
\pgfpathlineto{\pgfqpoint{3.364680in}{1.155000in}}%
\pgfpathlineto{\pgfqpoint{3.368400in}{1.540000in}}%
\pgfpathlineto{\pgfqpoint{3.369640in}{1.120000in}}%
\pgfpathlineto{\pgfqpoint{3.370880in}{1.330000in}}%
\pgfpathlineto{\pgfqpoint{3.372120in}{1.785000in}}%
\pgfpathlineto{\pgfqpoint{3.375840in}{0.945000in}}%
\pgfpathlineto{\pgfqpoint{3.377080in}{1.505000in}}%
\pgfpathlineto{\pgfqpoint{3.378320in}{1.365000in}}%
\pgfpathlineto{\pgfqpoint{3.379560in}{1.680000in}}%
\pgfpathlineto{\pgfqpoint{3.380800in}{1.260000in}}%
\pgfpathlineto{\pgfqpoint{3.383280in}{1.505000in}}%
\pgfpathlineto{\pgfqpoint{3.384520in}{2.135000in}}%
\pgfpathlineto{\pgfqpoint{3.385760in}{1.995000in}}%
\pgfpathlineto{\pgfqpoint{3.387000in}{1.645000in}}%
\pgfpathlineto{\pgfqpoint{3.388240in}{1.715000in}}%
\pgfpathlineto{\pgfqpoint{3.389480in}{1.890000in}}%
\pgfpathlineto{\pgfqpoint{3.390720in}{1.680000in}}%
\pgfpathlineto{\pgfqpoint{3.393200in}{1.050000in}}%
\pgfpathlineto{\pgfqpoint{3.394440in}{1.120000in}}%
\pgfpathlineto{\pgfqpoint{3.395680in}{1.085000in}}%
\pgfpathlineto{\pgfqpoint{3.396920in}{1.610000in}}%
\pgfpathlineto{\pgfqpoint{3.399400in}{1.295000in}}%
\pgfpathlineto{\pgfqpoint{3.400640in}{1.680000in}}%
\pgfpathlineto{\pgfqpoint{3.401880in}{1.575000in}}%
\pgfpathlineto{\pgfqpoint{3.403120in}{1.295000in}}%
\pgfpathlineto{\pgfqpoint{3.404360in}{1.645000in}}%
\pgfpathlineto{\pgfqpoint{3.405600in}{1.295000in}}%
\pgfpathlineto{\pgfqpoint{3.406840in}{1.400000in}}%
\pgfpathlineto{\pgfqpoint{3.408080in}{1.190000in}}%
\pgfpathlineto{\pgfqpoint{3.409320in}{1.435000in}}%
\pgfpathlineto{\pgfqpoint{3.410560in}{1.260000in}}%
\pgfpathlineto{\pgfqpoint{3.411800in}{1.540000in}}%
\pgfpathlineto{\pgfqpoint{3.413040in}{1.435000in}}%
\pgfpathlineto{\pgfqpoint{3.414280in}{1.155000in}}%
\pgfpathlineto{\pgfqpoint{3.415520in}{1.330000in}}%
\pgfpathlineto{\pgfqpoint{3.416760in}{1.260000in}}%
\pgfpathlineto{\pgfqpoint{3.418000in}{1.330000in}}%
\pgfpathlineto{\pgfqpoint{3.419240in}{1.715000in}}%
\pgfpathlineto{\pgfqpoint{3.420480in}{1.155000in}}%
\pgfpathlineto{\pgfqpoint{3.422960in}{1.435000in}}%
\pgfpathlineto{\pgfqpoint{3.424200in}{2.240000in}}%
\pgfpathlineto{\pgfqpoint{3.425440in}{1.155000in}}%
\pgfpathlineto{\pgfqpoint{3.426680in}{1.575000in}}%
\pgfpathlineto{\pgfqpoint{3.427920in}{1.540000in}}%
\pgfpathlineto{\pgfqpoint{3.430400in}{1.610000in}}%
\pgfpathlineto{\pgfqpoint{3.431640in}{1.365000in}}%
\pgfpathlineto{\pgfqpoint{3.432880in}{1.365000in}}%
\pgfpathlineto{\pgfqpoint{3.434120in}{1.680000in}}%
\pgfpathlineto{\pgfqpoint{3.435360in}{1.435000in}}%
\pgfpathlineto{\pgfqpoint{3.436600in}{1.435000in}}%
\pgfpathlineto{\pgfqpoint{3.437840in}{1.015000in}}%
\pgfpathlineto{\pgfqpoint{3.439080in}{1.715000in}}%
\pgfpathlineto{\pgfqpoint{3.440320in}{1.225000in}}%
\pgfpathlineto{\pgfqpoint{3.441560in}{1.505000in}}%
\pgfpathlineto{\pgfqpoint{3.442800in}{1.470000in}}%
\pgfpathlineto{\pgfqpoint{3.444040in}{1.750000in}}%
\pgfpathlineto{\pgfqpoint{3.445280in}{1.225000in}}%
\pgfpathlineto{\pgfqpoint{3.446520in}{1.295000in}}%
\pgfpathlineto{\pgfqpoint{3.447760in}{1.155000in}}%
\pgfpathlineto{\pgfqpoint{3.449000in}{1.610000in}}%
\pgfpathlineto{\pgfqpoint{3.450240in}{1.260000in}}%
\pgfpathlineto{\pgfqpoint{3.451480in}{1.540000in}}%
\pgfpathlineto{\pgfqpoint{3.452720in}{1.540000in}}%
\pgfpathlineto{\pgfqpoint{3.453960in}{1.610000in}}%
\pgfpathlineto{\pgfqpoint{3.455200in}{1.750000in}}%
\pgfpathlineto{\pgfqpoint{3.456440in}{1.750000in}}%
\pgfpathlineto{\pgfqpoint{3.457680in}{1.470000in}}%
\pgfpathlineto{\pgfqpoint{3.458920in}{1.470000in}}%
\pgfpathlineto{\pgfqpoint{3.460160in}{1.645000in}}%
\pgfpathlineto{\pgfqpoint{3.462640in}{1.575000in}}%
\pgfpathlineto{\pgfqpoint{3.463880in}{1.575000in}}%
\pgfpathlineto{\pgfqpoint{3.465120in}{1.855000in}}%
\pgfpathlineto{\pgfqpoint{3.466360in}{1.435000in}}%
\pgfpathlineto{\pgfqpoint{3.467600in}{1.400000in}}%
\pgfpathlineto{\pgfqpoint{3.468840in}{1.225000in}}%
\pgfpathlineto{\pgfqpoint{3.470080in}{1.260000in}}%
\pgfpathlineto{\pgfqpoint{3.472560in}{1.050000in}}%
\pgfpathlineto{\pgfqpoint{3.473800in}{1.400000in}}%
\pgfpathlineto{\pgfqpoint{3.475040in}{0.980000in}}%
\pgfpathlineto{\pgfqpoint{3.477520in}{1.435000in}}%
\pgfpathlineto{\pgfqpoint{3.478760in}{1.365000in}}%
\pgfpathlineto{\pgfqpoint{3.480000in}{1.190000in}}%
\pgfpathlineto{\pgfqpoint{3.481240in}{1.260000in}}%
\pgfpathlineto{\pgfqpoint{3.482480in}{1.225000in}}%
\pgfpathlineto{\pgfqpoint{3.483720in}{1.610000in}}%
\pgfpathlineto{\pgfqpoint{3.486200in}{1.260000in}}%
\pgfpathlineto{\pgfqpoint{3.489920in}{1.750000in}}%
\pgfpathlineto{\pgfqpoint{3.491160in}{1.610000in}}%
\pgfpathlineto{\pgfqpoint{3.492400in}{1.015000in}}%
\pgfpathlineto{\pgfqpoint{3.493640in}{1.400000in}}%
\pgfpathlineto{\pgfqpoint{3.494880in}{1.190000in}}%
\pgfpathlineto{\pgfqpoint{3.496120in}{1.610000in}}%
\pgfpathlineto{\pgfqpoint{3.497360in}{1.575000in}}%
\pgfpathlineto{\pgfqpoint{3.499840in}{1.085000in}}%
\pgfpathlineto{\pgfqpoint{3.501080in}{1.330000in}}%
\pgfpathlineto{\pgfqpoint{3.502320in}{1.330000in}}%
\pgfpathlineto{\pgfqpoint{3.503560in}{1.435000in}}%
\pgfpathlineto{\pgfqpoint{3.506040in}{1.785000in}}%
\pgfpathlineto{\pgfqpoint{3.507280in}{1.645000in}}%
\pgfpathlineto{\pgfqpoint{3.508520in}{1.960000in}}%
\pgfpathlineto{\pgfqpoint{3.509760in}{1.820000in}}%
\pgfpathlineto{\pgfqpoint{3.511000in}{1.925000in}}%
\pgfpathlineto{\pgfqpoint{3.512240in}{1.120000in}}%
\pgfpathlineto{\pgfqpoint{3.514720in}{1.680000in}}%
\pgfpathlineto{\pgfqpoint{3.515960in}{1.540000in}}%
\pgfpathlineto{\pgfqpoint{3.517200in}{1.575000in}}%
\pgfpathlineto{\pgfqpoint{3.518440in}{1.085000in}}%
\pgfpathlineto{\pgfqpoint{3.519680in}{1.645000in}}%
\pgfpathlineto{\pgfqpoint{3.520920in}{1.540000in}}%
\pgfpathlineto{\pgfqpoint{3.522160in}{1.190000in}}%
\pgfpathlineto{\pgfqpoint{3.523400in}{1.330000in}}%
\pgfpathlineto{\pgfqpoint{3.524640in}{1.750000in}}%
\pgfpathlineto{\pgfqpoint{3.525880in}{1.260000in}}%
\pgfpathlineto{\pgfqpoint{3.527120in}{1.645000in}}%
\pgfpathlineto{\pgfqpoint{3.528360in}{1.575000in}}%
\pgfpathlineto{\pgfqpoint{3.529600in}{1.015000in}}%
\pgfpathlineto{\pgfqpoint{3.530840in}{1.085000in}}%
\pgfpathlineto{\pgfqpoint{3.532080in}{1.505000in}}%
\pgfpathlineto{\pgfqpoint{3.533320in}{1.540000in}}%
\pgfpathlineto{\pgfqpoint{3.534560in}{1.435000in}}%
\pgfpathlineto{\pgfqpoint{3.535800in}{1.610000in}}%
\pgfpathlineto{\pgfqpoint{3.538280in}{1.015000in}}%
\pgfpathlineto{\pgfqpoint{3.540760in}{1.750000in}}%
\pgfpathlineto{\pgfqpoint{3.543240in}{1.330000in}}%
\pgfpathlineto{\pgfqpoint{3.544480in}{1.295000in}}%
\pgfpathlineto{\pgfqpoint{3.545720in}{1.680000in}}%
\pgfpathlineto{\pgfqpoint{3.546960in}{1.365000in}}%
\pgfpathlineto{\pgfqpoint{3.548200in}{1.925000in}}%
\pgfpathlineto{\pgfqpoint{3.550680in}{1.505000in}}%
\pgfpathlineto{\pgfqpoint{3.553160in}{1.295000in}}%
\pgfpathlineto{\pgfqpoint{3.554400in}{1.365000in}}%
\pgfpathlineto{\pgfqpoint{3.555640in}{1.820000in}}%
\pgfpathlineto{\pgfqpoint{3.558120in}{1.190000in}}%
\pgfpathlineto{\pgfqpoint{3.559360in}{1.400000in}}%
\pgfpathlineto{\pgfqpoint{3.560600in}{1.120000in}}%
\pgfpathlineto{\pgfqpoint{3.561840in}{1.645000in}}%
\pgfpathlineto{\pgfqpoint{3.563080in}{1.470000in}}%
\pgfpathlineto{\pgfqpoint{3.564320in}{1.120000in}}%
\pgfpathlineto{\pgfqpoint{3.566800in}{1.715000in}}%
\pgfpathlineto{\pgfqpoint{3.568040in}{1.715000in}}%
\pgfpathlineto{\pgfqpoint{3.569280in}{1.960000in}}%
\pgfpathlineto{\pgfqpoint{3.570520in}{1.225000in}}%
\pgfpathlineto{\pgfqpoint{3.571760in}{1.190000in}}%
\pgfpathlineto{\pgfqpoint{3.573000in}{1.365000in}}%
\pgfpathlineto{\pgfqpoint{3.574240in}{1.120000in}}%
\pgfpathlineto{\pgfqpoint{3.576720in}{1.400000in}}%
\pgfpathlineto{\pgfqpoint{3.577960in}{1.260000in}}%
\pgfpathlineto{\pgfqpoint{3.579200in}{1.295000in}}%
\pgfpathlineto{\pgfqpoint{3.580440in}{1.260000in}}%
\pgfpathlineto{\pgfqpoint{3.581680in}{1.190000in}}%
\pgfpathlineto{\pgfqpoint{3.584160in}{1.750000in}}%
\pgfpathlineto{\pgfqpoint{3.585400in}{1.365000in}}%
\pgfpathlineto{\pgfqpoint{3.587880in}{2.065000in}}%
\pgfpathlineto{\pgfqpoint{3.589120in}{1.400000in}}%
\pgfpathlineto{\pgfqpoint{3.590360in}{1.435000in}}%
\pgfpathlineto{\pgfqpoint{3.592840in}{1.610000in}}%
\pgfpathlineto{\pgfqpoint{3.595320in}{1.330000in}}%
\pgfpathlineto{\pgfqpoint{3.596560in}{1.505000in}}%
\pgfpathlineto{\pgfqpoint{3.597800in}{1.505000in}}%
\pgfpathlineto{\pgfqpoint{3.599040in}{1.715000in}}%
\pgfpathlineto{\pgfqpoint{3.600280in}{1.505000in}}%
\pgfpathlineto{\pgfqpoint{3.601520in}{1.785000in}}%
\pgfpathlineto{\pgfqpoint{3.602760in}{1.190000in}}%
\pgfpathlineto{\pgfqpoint{3.605240in}{1.610000in}}%
\pgfpathlineto{\pgfqpoint{3.606480in}{1.050000in}}%
\pgfpathlineto{\pgfqpoint{3.607720in}{1.890000in}}%
\pgfpathlineto{\pgfqpoint{3.610200in}{1.295000in}}%
\pgfpathlineto{\pgfqpoint{3.611440in}{1.645000in}}%
\pgfpathlineto{\pgfqpoint{3.612680in}{1.540000in}}%
\pgfpathlineto{\pgfqpoint{3.613920in}{1.295000in}}%
\pgfpathlineto{\pgfqpoint{3.616400in}{1.435000in}}%
\pgfpathlineto{\pgfqpoint{3.618880in}{1.750000in}}%
\pgfpathlineto{\pgfqpoint{3.620120in}{1.540000in}}%
\pgfpathlineto{\pgfqpoint{3.621360in}{1.540000in}}%
\pgfpathlineto{\pgfqpoint{3.623840in}{1.820000in}}%
\pgfpathlineto{\pgfqpoint{3.625080in}{1.715000in}}%
\pgfpathlineto{\pgfqpoint{3.626320in}{1.260000in}}%
\pgfpathlineto{\pgfqpoint{3.627560in}{1.295000in}}%
\pgfpathlineto{\pgfqpoint{3.628800in}{1.505000in}}%
\pgfpathlineto{\pgfqpoint{3.630040in}{1.435000in}}%
\pgfpathlineto{\pgfqpoint{3.631280in}{1.505000in}}%
\pgfpathlineto{\pgfqpoint{3.632520in}{1.155000in}}%
\pgfpathlineto{\pgfqpoint{3.633760in}{1.715000in}}%
\pgfpathlineto{\pgfqpoint{3.635000in}{1.750000in}}%
\pgfpathlineto{\pgfqpoint{3.637480in}{1.365000in}}%
\pgfpathlineto{\pgfqpoint{3.638720in}{1.435000in}}%
\pgfpathlineto{\pgfqpoint{3.641200in}{1.155000in}}%
\pgfpathlineto{\pgfqpoint{3.642440in}{1.785000in}}%
\pgfpathlineto{\pgfqpoint{3.643680in}{1.470000in}}%
\pgfpathlineto{\pgfqpoint{3.644920in}{1.470000in}}%
\pgfpathlineto{\pgfqpoint{3.646160in}{1.330000in}}%
\pgfpathlineto{\pgfqpoint{3.647400in}{1.470000in}}%
\pgfpathlineto{\pgfqpoint{3.648640in}{1.470000in}}%
\pgfpathlineto{\pgfqpoint{3.649880in}{0.735000in}}%
\pgfpathlineto{\pgfqpoint{3.651120in}{1.855000in}}%
\pgfpathlineto{\pgfqpoint{3.653600in}{1.225000in}}%
\pgfpathlineto{\pgfqpoint{3.654840in}{1.540000in}}%
\pgfpathlineto{\pgfqpoint{3.657320in}{1.400000in}}%
\pgfpathlineto{\pgfqpoint{3.658560in}{1.785000in}}%
\pgfpathlineto{\pgfqpoint{3.659800in}{1.785000in}}%
\pgfpathlineto{\pgfqpoint{3.661040in}{1.330000in}}%
\pgfpathlineto{\pgfqpoint{3.662280in}{1.295000in}}%
\pgfpathlineto{\pgfqpoint{3.663520in}{1.295000in}}%
\pgfpathlineto{\pgfqpoint{3.664760in}{1.365000in}}%
\pgfpathlineto{\pgfqpoint{3.666000in}{1.225000in}}%
\pgfpathlineto{\pgfqpoint{3.667240in}{1.610000in}}%
\pgfpathlineto{\pgfqpoint{3.668480in}{0.980000in}}%
\pgfpathlineto{\pgfqpoint{3.669720in}{1.330000in}}%
\pgfpathlineto{\pgfqpoint{3.670960in}{0.840000in}}%
\pgfpathlineto{\pgfqpoint{3.672200in}{1.225000in}}%
\pgfpathlineto{\pgfqpoint{3.673440in}{1.190000in}}%
\pgfpathlineto{\pgfqpoint{3.674680in}{1.470000in}}%
\pgfpathlineto{\pgfqpoint{3.675920in}{1.330000in}}%
\pgfpathlineto{\pgfqpoint{3.677160in}{1.540000in}}%
\pgfpathlineto{\pgfqpoint{3.679640in}{1.190000in}}%
\pgfpathlineto{\pgfqpoint{3.680880in}{1.190000in}}%
\pgfpathlineto{\pgfqpoint{3.682120in}{1.575000in}}%
\pgfpathlineto{\pgfqpoint{3.683360in}{1.330000in}}%
\pgfpathlineto{\pgfqpoint{3.684600in}{1.575000in}}%
\pgfpathlineto{\pgfqpoint{3.687080in}{1.050000in}}%
\pgfpathlineto{\pgfqpoint{3.689560in}{1.575000in}}%
\pgfpathlineto{\pgfqpoint{3.690800in}{1.505000in}}%
\pgfpathlineto{\pgfqpoint{3.692040in}{1.540000in}}%
\pgfpathlineto{\pgfqpoint{3.693280in}{0.875000in}}%
\pgfpathlineto{\pgfqpoint{3.697000in}{1.715000in}}%
\pgfpathlineto{\pgfqpoint{3.698240in}{1.540000in}}%
\pgfpathlineto{\pgfqpoint{3.699480in}{1.050000in}}%
\pgfpathlineto{\pgfqpoint{3.700720in}{1.645000in}}%
\pgfpathlineto{\pgfqpoint{3.701960in}{1.435000in}}%
\pgfpathlineto{\pgfqpoint{3.703200in}{1.505000in}}%
\pgfpathlineto{\pgfqpoint{3.705680in}{1.260000in}}%
\pgfpathlineto{\pgfqpoint{3.706920in}{1.085000in}}%
\pgfpathlineto{\pgfqpoint{3.708160in}{1.680000in}}%
\pgfpathlineto{\pgfqpoint{3.709400in}{1.680000in}}%
\pgfpathlineto{\pgfqpoint{3.710640in}{1.400000in}}%
\pgfpathlineto{\pgfqpoint{3.711880in}{1.505000in}}%
\pgfpathlineto{\pgfqpoint{3.713120in}{1.225000in}}%
\pgfpathlineto{\pgfqpoint{3.715600in}{1.750000in}}%
\pgfpathlineto{\pgfqpoint{3.718080in}{1.120000in}}%
\pgfpathlineto{\pgfqpoint{3.719320in}{1.085000in}}%
\pgfpathlineto{\pgfqpoint{3.720560in}{1.190000in}}%
\pgfpathlineto{\pgfqpoint{3.721800in}{1.435000in}}%
\pgfpathlineto{\pgfqpoint{3.723040in}{1.190000in}}%
\pgfpathlineto{\pgfqpoint{3.724280in}{1.610000in}}%
\pgfpathlineto{\pgfqpoint{3.725520in}{1.575000in}}%
\pgfpathlineto{\pgfqpoint{3.726760in}{1.365000in}}%
\pgfpathlineto{\pgfqpoint{3.728000in}{1.680000in}}%
\pgfpathlineto{\pgfqpoint{3.729240in}{1.470000in}}%
\pgfpathlineto{\pgfqpoint{3.730480in}{1.820000in}}%
\pgfpathlineto{\pgfqpoint{3.732960in}{1.120000in}}%
\pgfpathlineto{\pgfqpoint{3.734200in}{1.540000in}}%
\pgfpathlineto{\pgfqpoint{3.735440in}{1.470000in}}%
\pgfpathlineto{\pgfqpoint{3.736680in}{1.750000in}}%
\pgfpathlineto{\pgfqpoint{3.737920in}{1.680000in}}%
\pgfpathlineto{\pgfqpoint{3.739160in}{1.925000in}}%
\pgfpathlineto{\pgfqpoint{3.741640in}{1.260000in}}%
\pgfpathlineto{\pgfqpoint{3.742880in}{1.330000in}}%
\pgfpathlineto{\pgfqpoint{3.744120in}{1.155000in}}%
\pgfpathlineto{\pgfqpoint{3.745360in}{1.575000in}}%
\pgfpathlineto{\pgfqpoint{3.746600in}{1.575000in}}%
\pgfpathlineto{\pgfqpoint{3.747840in}{2.030000in}}%
\pgfpathlineto{\pgfqpoint{3.749080in}{1.365000in}}%
\pgfpathlineto{\pgfqpoint{3.750320in}{1.645000in}}%
\pgfpathlineto{\pgfqpoint{3.752800in}{1.470000in}}%
\pgfpathlineto{\pgfqpoint{3.754040in}{1.715000in}}%
\pgfpathlineto{\pgfqpoint{3.756520in}{1.260000in}}%
\pgfpathlineto{\pgfqpoint{3.757760in}{1.750000in}}%
\pgfpathlineto{\pgfqpoint{3.759000in}{1.540000in}}%
\pgfpathlineto{\pgfqpoint{3.760240in}{1.540000in}}%
\pgfpathlineto{\pgfqpoint{3.761480in}{1.645000in}}%
\pgfpathlineto{\pgfqpoint{3.762720in}{1.610000in}}%
\pgfpathlineto{\pgfqpoint{3.763960in}{1.540000in}}%
\pgfpathlineto{\pgfqpoint{3.765200in}{1.400000in}}%
\pgfpathlineto{\pgfqpoint{3.766440in}{1.085000in}}%
\pgfpathlineto{\pgfqpoint{3.767680in}{1.715000in}}%
\pgfpathlineto{\pgfqpoint{3.768920in}{1.365000in}}%
\pgfpathlineto{\pgfqpoint{3.770160in}{1.785000in}}%
\pgfpathlineto{\pgfqpoint{3.771400in}{1.155000in}}%
\pgfpathlineto{\pgfqpoint{3.772640in}{1.365000in}}%
\pgfpathlineto{\pgfqpoint{3.773880in}{1.295000in}}%
\pgfpathlineto{\pgfqpoint{3.776360in}{1.575000in}}%
\pgfpathlineto{\pgfqpoint{3.777600in}{1.645000in}}%
\pgfpathlineto{\pgfqpoint{3.778840in}{1.190000in}}%
\pgfpathlineto{\pgfqpoint{3.780080in}{1.540000in}}%
\pgfpathlineto{\pgfqpoint{3.782560in}{1.190000in}}%
\pgfpathlineto{\pgfqpoint{3.786280in}{1.715000in}}%
\pgfpathlineto{\pgfqpoint{3.787520in}{1.890000in}}%
\pgfpathlineto{\pgfqpoint{3.791240in}{1.085000in}}%
\pgfpathlineto{\pgfqpoint{3.792480in}{1.750000in}}%
\pgfpathlineto{\pgfqpoint{3.793720in}{1.365000in}}%
\pgfpathlineto{\pgfqpoint{3.794960in}{1.330000in}}%
\pgfpathlineto{\pgfqpoint{3.796200in}{0.910000in}}%
\pgfpathlineto{\pgfqpoint{3.797440in}{1.820000in}}%
\pgfpathlineto{\pgfqpoint{3.801160in}{1.260000in}}%
\pgfpathlineto{\pgfqpoint{3.802400in}{1.785000in}}%
\pgfpathlineto{\pgfqpoint{3.803640in}{1.680000in}}%
\pgfpathlineto{\pgfqpoint{3.804880in}{1.085000in}}%
\pgfpathlineto{\pgfqpoint{3.806120in}{1.260000in}}%
\pgfpathlineto{\pgfqpoint{3.808600in}{2.030000in}}%
\pgfpathlineto{\pgfqpoint{3.809840in}{1.470000in}}%
\pgfpathlineto{\pgfqpoint{3.811080in}{1.505000in}}%
\pgfpathlineto{\pgfqpoint{3.812320in}{1.785000in}}%
\pgfpathlineto{\pgfqpoint{3.813560in}{1.225000in}}%
\pgfpathlineto{\pgfqpoint{3.814800in}{1.260000in}}%
\pgfpathlineto{\pgfqpoint{3.816040in}{1.505000in}}%
\pgfpathlineto{\pgfqpoint{3.817280in}{0.980000in}}%
\pgfpathlineto{\pgfqpoint{3.819760in}{1.610000in}}%
\pgfpathlineto{\pgfqpoint{3.821000in}{1.435000in}}%
\pgfpathlineto{\pgfqpoint{3.822240in}{0.665000in}}%
\pgfpathlineto{\pgfqpoint{3.824720in}{1.225000in}}%
\pgfpathlineto{\pgfqpoint{3.827200in}{1.540000in}}%
\pgfpathlineto{\pgfqpoint{3.828440in}{2.065000in}}%
\pgfpathlineto{\pgfqpoint{3.829680in}{1.365000in}}%
\pgfpathlineto{\pgfqpoint{3.832160in}{1.575000in}}%
\pgfpathlineto{\pgfqpoint{3.833400in}{1.295000in}}%
\pgfpathlineto{\pgfqpoint{3.834640in}{1.365000in}}%
\pgfpathlineto{\pgfqpoint{3.835880in}{1.225000in}}%
\pgfpathlineto{\pgfqpoint{3.837120in}{1.435000in}}%
\pgfpathlineto{\pgfqpoint{3.838360in}{1.400000in}}%
\pgfpathlineto{\pgfqpoint{3.839600in}{1.680000in}}%
\pgfpathlineto{\pgfqpoint{3.840840in}{1.330000in}}%
\pgfpathlineto{\pgfqpoint{3.842080in}{1.540000in}}%
\pgfpathlineto{\pgfqpoint{3.843320in}{1.190000in}}%
\pgfpathlineto{\pgfqpoint{3.844560in}{1.505000in}}%
\pgfpathlineto{\pgfqpoint{3.845800in}{1.435000in}}%
\pgfpathlineto{\pgfqpoint{3.847040in}{1.680000in}}%
\pgfpathlineto{\pgfqpoint{3.848280in}{1.155000in}}%
\pgfpathlineto{\pgfqpoint{3.850760in}{1.645000in}}%
\pgfpathlineto{\pgfqpoint{3.852000in}{1.470000in}}%
\pgfpathlineto{\pgfqpoint{3.853240in}{1.085000in}}%
\pgfpathlineto{\pgfqpoint{3.854480in}{1.750000in}}%
\pgfpathlineto{\pgfqpoint{3.855720in}{1.400000in}}%
\pgfpathlineto{\pgfqpoint{3.858200in}{1.645000in}}%
\pgfpathlineto{\pgfqpoint{3.859440in}{1.155000in}}%
\pgfpathlineto{\pgfqpoint{3.860680in}{1.715000in}}%
\pgfpathlineto{\pgfqpoint{3.861920in}{1.295000in}}%
\pgfpathlineto{\pgfqpoint{3.863160in}{1.435000in}}%
\pgfpathlineto{\pgfqpoint{3.864400in}{1.365000in}}%
\pgfpathlineto{\pgfqpoint{3.865640in}{1.190000in}}%
\pgfpathlineto{\pgfqpoint{3.866880in}{1.505000in}}%
\pgfpathlineto{\pgfqpoint{3.868120in}{1.435000in}}%
\pgfpathlineto{\pgfqpoint{3.869360in}{0.805000in}}%
\pgfpathlineto{\pgfqpoint{3.870600in}{1.365000in}}%
\pgfpathlineto{\pgfqpoint{3.871840in}{0.945000in}}%
\pgfpathlineto{\pgfqpoint{3.873080in}{1.155000in}}%
\pgfpathlineto{\pgfqpoint{3.874320in}{1.575000in}}%
\pgfpathlineto{\pgfqpoint{3.875560in}{1.505000in}}%
\pgfpathlineto{\pgfqpoint{3.876800in}{1.365000in}}%
\pgfpathlineto{\pgfqpoint{3.878040in}{0.875000in}}%
\pgfpathlineto{\pgfqpoint{3.879280in}{1.120000in}}%
\pgfpathlineto{\pgfqpoint{3.880520in}{1.085000in}}%
\pgfpathlineto{\pgfqpoint{3.881760in}{0.770000in}}%
\pgfpathlineto{\pgfqpoint{3.883000in}{1.295000in}}%
\pgfpathlineto{\pgfqpoint{3.884240in}{1.085000in}}%
\pgfpathlineto{\pgfqpoint{3.885480in}{1.120000in}}%
\pgfpathlineto{\pgfqpoint{3.887960in}{1.470000in}}%
\pgfpathlineto{\pgfqpoint{3.889200in}{0.980000in}}%
\pgfpathlineto{\pgfqpoint{3.891680in}{1.540000in}}%
\pgfpathlineto{\pgfqpoint{3.892920in}{1.470000in}}%
\pgfpathlineto{\pgfqpoint{3.894160in}{1.470000in}}%
\pgfpathlineto{\pgfqpoint{3.895400in}{1.645000in}}%
\pgfpathlineto{\pgfqpoint{3.897880in}{1.050000in}}%
\pgfpathlineto{\pgfqpoint{3.899120in}{1.435000in}}%
\pgfpathlineto{\pgfqpoint{3.900360in}{1.435000in}}%
\pgfpathlineto{\pgfqpoint{3.901600in}{1.645000in}}%
\pgfpathlineto{\pgfqpoint{3.902840in}{1.400000in}}%
\pgfpathlineto{\pgfqpoint{3.905320in}{1.820000in}}%
\pgfpathlineto{\pgfqpoint{3.906560in}{1.785000in}}%
\pgfpathlineto{\pgfqpoint{3.907800in}{1.120000in}}%
\pgfpathlineto{\pgfqpoint{3.910280in}{1.470000in}}%
\pgfpathlineto{\pgfqpoint{3.912760in}{1.295000in}}%
\pgfpathlineto{\pgfqpoint{3.915240in}{1.680000in}}%
\pgfpathlineto{\pgfqpoint{3.916480in}{1.715000in}}%
\pgfpathlineto{\pgfqpoint{3.917720in}{1.820000in}}%
\pgfpathlineto{\pgfqpoint{3.918960in}{1.435000in}}%
\pgfpathlineto{\pgfqpoint{3.920200in}{1.715000in}}%
\pgfpathlineto{\pgfqpoint{3.923920in}{1.365000in}}%
\pgfpathlineto{\pgfqpoint{3.925160in}{1.400000in}}%
\pgfpathlineto{\pgfqpoint{3.926400in}{1.260000in}}%
\pgfpathlineto{\pgfqpoint{3.927640in}{1.470000in}}%
\pgfpathlineto{\pgfqpoint{3.928880in}{1.120000in}}%
\pgfpathlineto{\pgfqpoint{3.930120in}{1.190000in}}%
\pgfpathlineto{\pgfqpoint{3.931360in}{1.610000in}}%
\pgfpathlineto{\pgfqpoint{3.932600in}{1.645000in}}%
\pgfpathlineto{\pgfqpoint{3.933840in}{1.365000in}}%
\pgfpathlineto{\pgfqpoint{3.936320in}{1.820000in}}%
\pgfpathlineto{\pgfqpoint{3.937560in}{1.785000in}}%
\pgfpathlineto{\pgfqpoint{3.940040in}{1.295000in}}%
\pgfpathlineto{\pgfqpoint{3.941280in}{1.330000in}}%
\pgfpathlineto{\pgfqpoint{3.942520in}{1.120000in}}%
\pgfpathlineto{\pgfqpoint{3.943760in}{1.575000in}}%
\pgfpathlineto{\pgfqpoint{3.945000in}{1.260000in}}%
\pgfpathlineto{\pgfqpoint{3.947480in}{1.715000in}}%
\pgfpathlineto{\pgfqpoint{3.951200in}{1.190000in}}%
\pgfpathlineto{\pgfqpoint{3.952440in}{1.505000in}}%
\pgfpathlineto{\pgfqpoint{3.953680in}{1.400000in}}%
\pgfpathlineto{\pgfqpoint{3.954920in}{1.435000in}}%
\pgfpathlineto{\pgfqpoint{3.957400in}{1.050000in}}%
\pgfpathlineto{\pgfqpoint{3.958640in}{1.155000in}}%
\pgfpathlineto{\pgfqpoint{3.959880in}{1.575000in}}%
\pgfpathlineto{\pgfqpoint{3.961120in}{1.575000in}}%
\pgfpathlineto{\pgfqpoint{3.962360in}{1.540000in}}%
\pgfpathlineto{\pgfqpoint{3.963600in}{1.540000in}}%
\pgfpathlineto{\pgfqpoint{3.964840in}{1.470000in}}%
\pgfpathlineto{\pgfqpoint{3.966080in}{1.610000in}}%
\pgfpathlineto{\pgfqpoint{3.967320in}{1.470000in}}%
\pgfpathlineto{\pgfqpoint{3.971040in}{1.995000in}}%
\pgfpathlineto{\pgfqpoint{3.972280in}{1.470000in}}%
\pgfpathlineto{\pgfqpoint{3.974760in}{1.820000in}}%
\pgfpathlineto{\pgfqpoint{3.977240in}{1.225000in}}%
\pgfpathlineto{\pgfqpoint{3.979720in}{1.645000in}}%
\pgfpathlineto{\pgfqpoint{3.980960in}{1.365000in}}%
\pgfpathlineto{\pgfqpoint{3.982200in}{1.645000in}}%
\pgfpathlineto{\pgfqpoint{3.983440in}{1.540000in}}%
\pgfpathlineto{\pgfqpoint{3.984680in}{1.645000in}}%
\pgfpathlineto{\pgfqpoint{3.985920in}{0.910000in}}%
\pgfpathlineto{\pgfqpoint{3.987160in}{1.470000in}}%
\pgfpathlineto{\pgfqpoint{3.989640in}{1.225000in}}%
\pgfpathlineto{\pgfqpoint{3.992120in}{1.435000in}}%
\pgfpathlineto{\pgfqpoint{3.993360in}{1.680000in}}%
\pgfpathlineto{\pgfqpoint{3.995840in}{0.980000in}}%
\pgfpathlineto{\pgfqpoint{3.997080in}{1.225000in}}%
\pgfpathlineto{\pgfqpoint{3.998320in}{1.225000in}}%
\pgfpathlineto{\pgfqpoint{3.999560in}{1.960000in}}%
\pgfpathlineto{\pgfqpoint{4.002040in}{1.295000in}}%
\pgfpathlineto{\pgfqpoint{4.003280in}{1.400000in}}%
\pgfpathlineto{\pgfqpoint{4.004520in}{1.365000in}}%
\pgfpathlineto{\pgfqpoint{4.005760in}{1.645000in}}%
\pgfpathlineto{\pgfqpoint{4.007000in}{1.120000in}}%
\pgfpathlineto{\pgfqpoint{4.009480in}{1.820000in}}%
\pgfpathlineto{\pgfqpoint{4.010720in}{1.190000in}}%
\pgfpathlineto{\pgfqpoint{4.011960in}{1.785000in}}%
\pgfpathlineto{\pgfqpoint{4.013200in}{1.330000in}}%
\pgfpathlineto{\pgfqpoint{4.014440in}{2.030000in}}%
\pgfpathlineto{\pgfqpoint{4.016920in}{1.400000in}}%
\pgfpathlineto{\pgfqpoint{4.018160in}{1.015000in}}%
\pgfpathlineto{\pgfqpoint{4.019400in}{1.155000in}}%
\pgfpathlineto{\pgfqpoint{4.020640in}{1.575000in}}%
\pgfpathlineto{\pgfqpoint{4.021880in}{1.225000in}}%
\pgfpathlineto{\pgfqpoint{4.023120in}{1.715000in}}%
\pgfpathlineto{\pgfqpoint{4.024360in}{1.400000in}}%
\pgfpathlineto{\pgfqpoint{4.025600in}{1.750000in}}%
\pgfpathlineto{\pgfqpoint{4.026840in}{1.575000in}}%
\pgfpathlineto{\pgfqpoint{4.028080in}{1.715000in}}%
\pgfpathlineto{\pgfqpoint{4.029320in}{1.505000in}}%
\pgfpathlineto{\pgfqpoint{4.030560in}{1.750000in}}%
\pgfpathlineto{\pgfqpoint{4.033040in}{1.260000in}}%
\pgfpathlineto{\pgfqpoint{4.034280in}{1.505000in}}%
\pgfpathlineto{\pgfqpoint{4.035520in}{0.770000in}}%
\pgfpathlineto{\pgfqpoint{4.038000in}{1.400000in}}%
\pgfpathlineto{\pgfqpoint{4.039240in}{1.540000in}}%
\pgfpathlineto{\pgfqpoint{4.040480in}{1.295000in}}%
\pgfpathlineto{\pgfqpoint{4.041720in}{1.505000in}}%
\pgfpathlineto{\pgfqpoint{4.042960in}{1.435000in}}%
\pgfpathlineto{\pgfqpoint{4.044200in}{1.260000in}}%
\pgfpathlineto{\pgfqpoint{4.045440in}{1.260000in}}%
\pgfpathlineto{\pgfqpoint{4.046680in}{1.540000in}}%
\pgfpathlineto{\pgfqpoint{4.047920in}{1.330000in}}%
\pgfpathlineto{\pgfqpoint{4.049160in}{1.470000in}}%
\pgfpathlineto{\pgfqpoint{4.050400in}{1.120000in}}%
\pgfpathlineto{\pgfqpoint{4.052880in}{1.995000in}}%
\pgfpathlineto{\pgfqpoint{4.054120in}{1.540000in}}%
\pgfpathlineto{\pgfqpoint{4.055360in}{1.680000in}}%
\pgfpathlineto{\pgfqpoint{4.056600in}{1.260000in}}%
\pgfpathlineto{\pgfqpoint{4.057840in}{1.575000in}}%
\pgfpathlineto{\pgfqpoint{4.059080in}{1.155000in}}%
\pgfpathlineto{\pgfqpoint{4.060320in}{1.120000in}}%
\pgfpathlineto{\pgfqpoint{4.061560in}{1.225000in}}%
\pgfpathlineto{\pgfqpoint{4.062800in}{1.505000in}}%
\pgfpathlineto{\pgfqpoint{4.064040in}{1.050000in}}%
\pgfpathlineto{\pgfqpoint{4.065280in}{1.260000in}}%
\pgfpathlineto{\pgfqpoint{4.066520in}{1.120000in}}%
\pgfpathlineto{\pgfqpoint{4.067760in}{1.610000in}}%
\pgfpathlineto{\pgfqpoint{4.070240in}{1.365000in}}%
\pgfpathlineto{\pgfqpoint{4.072720in}{1.645000in}}%
\pgfpathlineto{\pgfqpoint{4.073960in}{1.540000in}}%
\pgfpathlineto{\pgfqpoint{4.075200in}{1.295000in}}%
\pgfpathlineto{\pgfqpoint{4.077680in}{1.785000in}}%
\pgfpathlineto{\pgfqpoint{4.080160in}{1.050000in}}%
\pgfpathlineto{\pgfqpoint{4.081400in}{1.155000in}}%
\pgfpathlineto{\pgfqpoint{4.083880in}{1.575000in}}%
\pgfpathlineto{\pgfqpoint{4.085120in}{1.330000in}}%
\pgfpathlineto{\pgfqpoint{4.086360in}{1.680000in}}%
\pgfpathlineto{\pgfqpoint{4.087600in}{1.610000in}}%
\pgfpathlineto{\pgfqpoint{4.088840in}{1.680000in}}%
\pgfpathlineto{\pgfqpoint{4.090080in}{1.260000in}}%
\pgfpathlineto{\pgfqpoint{4.091320in}{1.750000in}}%
\pgfpathlineto{\pgfqpoint{4.092560in}{1.365000in}}%
\pgfpathlineto{\pgfqpoint{4.095040in}{1.890000in}}%
\pgfpathlineto{\pgfqpoint{4.096280in}{0.980000in}}%
\pgfpathlineto{\pgfqpoint{4.098760in}{1.680000in}}%
\pgfpathlineto{\pgfqpoint{4.100000in}{1.050000in}}%
\pgfpathlineto{\pgfqpoint{4.101240in}{1.155000in}}%
\pgfpathlineto{\pgfqpoint{4.102480in}{1.050000in}}%
\pgfpathlineto{\pgfqpoint{4.104960in}{1.715000in}}%
\pgfpathlineto{\pgfqpoint{4.106200in}{1.505000in}}%
\pgfpathlineto{\pgfqpoint{4.107440in}{1.610000in}}%
\pgfpathlineto{\pgfqpoint{4.108680in}{1.855000in}}%
\pgfpathlineto{\pgfqpoint{4.109920in}{1.540000in}}%
\pgfpathlineto{\pgfqpoint{4.111160in}{1.575000in}}%
\pgfpathlineto{\pgfqpoint{4.112400in}{1.540000in}}%
\pgfpathlineto{\pgfqpoint{4.116120in}{1.365000in}}%
\pgfpathlineto{\pgfqpoint{4.117360in}{1.050000in}}%
\pgfpathlineto{\pgfqpoint{4.121080in}{1.820000in}}%
\pgfpathlineto{\pgfqpoint{4.122320in}{1.015000in}}%
\pgfpathlineto{\pgfqpoint{4.123560in}{1.505000in}}%
\pgfpathlineto{\pgfqpoint{4.124800in}{1.435000in}}%
\pgfpathlineto{\pgfqpoint{4.127280in}{1.820000in}}%
\pgfpathlineto{\pgfqpoint{4.129760in}{1.260000in}}%
\pgfpathlineto{\pgfqpoint{4.132240in}{1.400000in}}%
\pgfpathlineto{\pgfqpoint{4.133480in}{1.050000in}}%
\pgfpathlineto{\pgfqpoint{4.135960in}{1.295000in}}%
\pgfpathlineto{\pgfqpoint{4.137200in}{1.225000in}}%
\pgfpathlineto{\pgfqpoint{4.138440in}{1.925000in}}%
\pgfpathlineto{\pgfqpoint{4.139680in}{1.715000in}}%
\pgfpathlineto{\pgfqpoint{4.142160in}{0.735000in}}%
\pgfpathlineto{\pgfqpoint{4.143400in}{1.400000in}}%
\pgfpathlineto{\pgfqpoint{4.144640in}{1.050000in}}%
\pgfpathlineto{\pgfqpoint{4.147120in}{1.470000in}}%
\pgfpathlineto{\pgfqpoint{4.148360in}{1.155000in}}%
\pgfpathlineto{\pgfqpoint{4.149600in}{1.505000in}}%
\pgfpathlineto{\pgfqpoint{4.150840in}{1.015000in}}%
\pgfpathlineto{\pgfqpoint{4.152080in}{1.715000in}}%
\pgfpathlineto{\pgfqpoint{4.155800in}{0.945000in}}%
\pgfpathlineto{\pgfqpoint{4.157040in}{1.575000in}}%
\pgfpathlineto{\pgfqpoint{4.159520in}{1.190000in}}%
\pgfpathlineto{\pgfqpoint{4.160760in}{1.330000in}}%
\pgfpathlineto{\pgfqpoint{4.162000in}{1.330000in}}%
\pgfpathlineto{\pgfqpoint{4.163240in}{2.100000in}}%
\pgfpathlineto{\pgfqpoint{4.166960in}{1.190000in}}%
\pgfpathlineto{\pgfqpoint{4.169440in}{1.680000in}}%
\pgfpathlineto{\pgfqpoint{4.170680in}{1.295000in}}%
\pgfpathlineto{\pgfqpoint{4.171920in}{1.785000in}}%
\pgfpathlineto{\pgfqpoint{4.173160in}{1.330000in}}%
\pgfpathlineto{\pgfqpoint{4.174400in}{1.435000in}}%
\pgfpathlineto{\pgfqpoint{4.175640in}{1.400000in}}%
\pgfpathlineto{\pgfqpoint{4.178120in}{2.135000in}}%
\pgfpathlineto{\pgfqpoint{4.179360in}{1.155000in}}%
\pgfpathlineto{\pgfqpoint{4.181840in}{1.680000in}}%
\pgfpathlineto{\pgfqpoint{4.183080in}{1.680000in}}%
\pgfpathlineto{\pgfqpoint{4.184320in}{1.365000in}}%
\pgfpathlineto{\pgfqpoint{4.185560in}{1.680000in}}%
\pgfpathlineto{\pgfqpoint{4.186800in}{1.120000in}}%
\pgfpathlineto{\pgfqpoint{4.188040in}{1.120000in}}%
\pgfpathlineto{\pgfqpoint{4.189280in}{1.890000in}}%
\pgfpathlineto{\pgfqpoint{4.191760in}{1.085000in}}%
\pgfpathlineto{\pgfqpoint{4.193000in}{1.575000in}}%
\pgfpathlineto{\pgfqpoint{4.194240in}{1.120000in}}%
\pgfpathlineto{\pgfqpoint{4.195480in}{1.470000in}}%
\pgfpathlineto{\pgfqpoint{4.197960in}{1.540000in}}%
\pgfpathlineto{\pgfqpoint{4.199200in}{1.610000in}}%
\pgfpathlineto{\pgfqpoint{4.201680in}{0.910000in}}%
\pgfpathlineto{\pgfqpoint{4.204160in}{1.540000in}}%
\pgfpathlineto{\pgfqpoint{4.205400in}{1.820000in}}%
\pgfpathlineto{\pgfqpoint{4.207880in}{1.295000in}}%
\pgfpathlineto{\pgfqpoint{4.209120in}{1.365000in}}%
\pgfpathlineto{\pgfqpoint{4.210360in}{1.365000in}}%
\pgfpathlineto{\pgfqpoint{4.211600in}{1.155000in}}%
\pgfpathlineto{\pgfqpoint{4.212840in}{1.155000in}}%
\pgfpathlineto{\pgfqpoint{4.214080in}{1.120000in}}%
\pgfpathlineto{\pgfqpoint{4.216560in}{1.435000in}}%
\pgfpathlineto{\pgfqpoint{4.217800in}{1.435000in}}%
\pgfpathlineto{\pgfqpoint{4.219040in}{1.085000in}}%
\pgfpathlineto{\pgfqpoint{4.220280in}{1.295000in}}%
\pgfpathlineto{\pgfqpoint{4.221520in}{1.015000in}}%
\pgfpathlineto{\pgfqpoint{4.222760in}{1.610000in}}%
\pgfpathlineto{\pgfqpoint{4.224000in}{1.645000in}}%
\pgfpathlineto{\pgfqpoint{4.225240in}{1.610000in}}%
\pgfpathlineto{\pgfqpoint{4.226480in}{1.610000in}}%
\pgfpathlineto{\pgfqpoint{4.228960in}{1.015000in}}%
\pgfpathlineto{\pgfqpoint{4.231440in}{1.575000in}}%
\pgfpathlineto{\pgfqpoint{4.233920in}{1.330000in}}%
\pgfpathlineto{\pgfqpoint{4.235160in}{1.610000in}}%
\pgfpathlineto{\pgfqpoint{4.237640in}{1.295000in}}%
\pgfpathlineto{\pgfqpoint{4.238880in}{1.540000in}}%
\pgfpathlineto{\pgfqpoint{4.240120in}{1.540000in}}%
\pgfpathlineto{\pgfqpoint{4.241360in}{0.700000in}}%
\pgfpathlineto{\pgfqpoint{4.242600in}{1.330000in}}%
\pgfpathlineto{\pgfqpoint{4.243840in}{1.295000in}}%
\pgfpathlineto{\pgfqpoint{4.245080in}{1.400000in}}%
\pgfpathlineto{\pgfqpoint{4.246320in}{1.190000in}}%
\pgfpathlineto{\pgfqpoint{4.247560in}{1.505000in}}%
\pgfpathlineto{\pgfqpoint{4.251280in}{1.120000in}}%
\pgfpathlineto{\pgfqpoint{4.252520in}{1.260000in}}%
\pgfpathlineto{\pgfqpoint{4.253760in}{1.750000in}}%
\pgfpathlineto{\pgfqpoint{4.256240in}{0.910000in}}%
\pgfpathlineto{\pgfqpoint{4.259960in}{1.575000in}}%
\pgfpathlineto{\pgfqpoint{4.262440in}{1.190000in}}%
\pgfpathlineto{\pgfqpoint{4.264920in}{1.715000in}}%
\pgfpathlineto{\pgfqpoint{4.267400in}{1.295000in}}%
\pgfpathlineto{\pgfqpoint{4.268640in}{1.505000in}}%
\pgfpathlineto{\pgfqpoint{4.269880in}{1.225000in}}%
\pgfpathlineto{\pgfqpoint{4.272360in}{1.505000in}}%
\pgfpathlineto{\pgfqpoint{4.273600in}{1.155000in}}%
\pgfpathlineto{\pgfqpoint{4.277320in}{1.715000in}}%
\pgfpathlineto{\pgfqpoint{4.278560in}{1.645000in}}%
\pgfpathlineto{\pgfqpoint{4.279800in}{1.925000in}}%
\pgfpathlineto{\pgfqpoint{4.282280in}{1.050000in}}%
\pgfpathlineto{\pgfqpoint{4.283520in}{1.400000in}}%
\pgfpathlineto{\pgfqpoint{4.284760in}{1.330000in}}%
\pgfpathlineto{\pgfqpoint{4.286000in}{1.855000in}}%
\pgfpathlineto{\pgfqpoint{4.288480in}{1.400000in}}%
\pgfpathlineto{\pgfqpoint{4.292200in}{1.925000in}}%
\pgfpathlineto{\pgfqpoint{4.294680in}{1.225000in}}%
\pgfpathlineto{\pgfqpoint{4.295920in}{1.820000in}}%
\pgfpathlineto{\pgfqpoint{4.297160in}{1.470000in}}%
\pgfpathlineto{\pgfqpoint{4.298400in}{1.820000in}}%
\pgfpathlineto{\pgfqpoint{4.300880in}{1.190000in}}%
\pgfpathlineto{\pgfqpoint{4.302120in}{1.295000in}}%
\pgfpathlineto{\pgfqpoint{4.303360in}{1.225000in}}%
\pgfpathlineto{\pgfqpoint{4.304600in}{1.085000in}}%
\pgfpathlineto{\pgfqpoint{4.305840in}{1.645000in}}%
\pgfpathlineto{\pgfqpoint{4.307080in}{1.330000in}}%
\pgfpathlineto{\pgfqpoint{4.308320in}{1.715000in}}%
\pgfpathlineto{\pgfqpoint{4.310800in}{1.365000in}}%
\pgfpathlineto{\pgfqpoint{4.312040in}{1.365000in}}%
\pgfpathlineto{\pgfqpoint{4.313280in}{1.890000in}}%
\pgfpathlineto{\pgfqpoint{4.314520in}{1.155000in}}%
\pgfpathlineto{\pgfqpoint{4.315760in}{1.680000in}}%
\pgfpathlineto{\pgfqpoint{4.317000in}{1.575000in}}%
\pgfpathlineto{\pgfqpoint{4.319480in}{1.785000in}}%
\pgfpathlineto{\pgfqpoint{4.320720in}{1.645000in}}%
\pgfpathlineto{\pgfqpoint{4.321960in}{1.715000in}}%
\pgfpathlineto{\pgfqpoint{4.323200in}{1.365000in}}%
\pgfpathlineto{\pgfqpoint{4.325680in}{1.610000in}}%
\pgfpathlineto{\pgfqpoint{4.326920in}{1.645000in}}%
\pgfpathlineto{\pgfqpoint{4.328160in}{1.120000in}}%
\pgfpathlineto{\pgfqpoint{4.329400in}{1.435000in}}%
\pgfpathlineto{\pgfqpoint{4.330640in}{1.260000in}}%
\pgfpathlineto{\pgfqpoint{4.333120in}{1.400000in}}%
\pgfpathlineto{\pgfqpoint{4.334360in}{0.945000in}}%
\pgfpathlineto{\pgfqpoint{4.335600in}{1.295000in}}%
\pgfpathlineto{\pgfqpoint{4.336840in}{1.225000in}}%
\pgfpathlineto{\pgfqpoint{4.338080in}{1.680000in}}%
\pgfpathlineto{\pgfqpoint{4.339320in}{1.330000in}}%
\pgfpathlineto{\pgfqpoint{4.340560in}{1.435000in}}%
\pgfpathlineto{\pgfqpoint{4.341800in}{1.120000in}}%
\pgfpathlineto{\pgfqpoint{4.343040in}{1.190000in}}%
\pgfpathlineto{\pgfqpoint{4.344280in}{0.665000in}}%
\pgfpathlineto{\pgfqpoint{4.345520in}{1.260000in}}%
\pgfpathlineto{\pgfqpoint{4.346760in}{1.050000in}}%
\pgfpathlineto{\pgfqpoint{4.350480in}{1.575000in}}%
\pgfpathlineto{\pgfqpoint{4.352960in}{1.225000in}}%
\pgfpathlineto{\pgfqpoint{4.354200in}{1.890000in}}%
\pgfpathlineto{\pgfqpoint{4.357920in}{1.400000in}}%
\pgfpathlineto{\pgfqpoint{4.359160in}{1.890000in}}%
\pgfpathlineto{\pgfqpoint{4.360400in}{1.505000in}}%
\pgfpathlineto{\pgfqpoint{4.361640in}{1.575000in}}%
\pgfpathlineto{\pgfqpoint{4.362880in}{1.085000in}}%
\pgfpathlineto{\pgfqpoint{4.364120in}{1.155000in}}%
\pgfpathlineto{\pgfqpoint{4.365360in}{1.330000in}}%
\pgfpathlineto{\pgfqpoint{4.366600in}{0.980000in}}%
\pgfpathlineto{\pgfqpoint{4.370320in}{1.575000in}}%
\pgfpathlineto{\pgfqpoint{4.371560in}{1.505000in}}%
\pgfpathlineto{\pgfqpoint{4.372800in}{1.365000in}}%
\pgfpathlineto{\pgfqpoint{4.374040in}{1.085000in}}%
\pgfpathlineto{\pgfqpoint{4.375280in}{1.260000in}}%
\pgfpathlineto{\pgfqpoint{4.376520in}{1.610000in}}%
\pgfpathlineto{\pgfqpoint{4.377760in}{1.610000in}}%
\pgfpathlineto{\pgfqpoint{4.380240in}{1.330000in}}%
\pgfpathlineto{\pgfqpoint{4.381480in}{1.435000in}}%
\pgfpathlineto{\pgfqpoint{4.382720in}{1.400000in}}%
\pgfpathlineto{\pgfqpoint{4.383960in}{1.400000in}}%
\pgfpathlineto{\pgfqpoint{4.385200in}{1.610000in}}%
\pgfpathlineto{\pgfqpoint{4.386440in}{1.260000in}}%
\pgfpathlineto{\pgfqpoint{4.387680in}{1.715000in}}%
\pgfpathlineto{\pgfqpoint{4.391400in}{1.225000in}}%
\pgfpathlineto{\pgfqpoint{4.392640in}{1.470000in}}%
\pgfpathlineto{\pgfqpoint{4.393880in}{1.365000in}}%
\pgfpathlineto{\pgfqpoint{4.395120in}{1.820000in}}%
\pgfpathlineto{\pgfqpoint{4.397600in}{1.365000in}}%
\pgfpathlineto{\pgfqpoint{4.398840in}{1.295000in}}%
\pgfpathlineto{\pgfqpoint{4.400080in}{1.575000in}}%
\pgfpathlineto{\pgfqpoint{4.402560in}{1.330000in}}%
\pgfpathlineto{\pgfqpoint{4.403800in}{1.925000in}}%
\pgfpathlineto{\pgfqpoint{4.406280in}{1.225000in}}%
\pgfpathlineto{\pgfqpoint{4.407520in}{1.435000in}}%
\pgfpathlineto{\pgfqpoint{4.408760in}{1.400000in}}%
\pgfpathlineto{\pgfqpoint{4.410000in}{1.680000in}}%
\pgfpathlineto{\pgfqpoint{4.411240in}{1.470000in}}%
\pgfpathlineto{\pgfqpoint{4.413720in}{2.205000in}}%
\pgfpathlineto{\pgfqpoint{4.414960in}{1.645000in}}%
\pgfpathlineto{\pgfqpoint{4.416200in}{1.680000in}}%
\pgfpathlineto{\pgfqpoint{4.418680in}{1.190000in}}%
\pgfpathlineto{\pgfqpoint{4.421160in}{2.170000in}}%
\pgfpathlineto{\pgfqpoint{4.423640in}{1.820000in}}%
\pgfpathlineto{\pgfqpoint{4.424880in}{1.715000in}}%
\pgfpathlineto{\pgfqpoint{4.426120in}{1.155000in}}%
\pgfpathlineto{\pgfqpoint{4.428600in}{1.890000in}}%
\pgfpathlineto{\pgfqpoint{4.429840in}{1.470000in}}%
\pgfpathlineto{\pgfqpoint{4.432320in}{1.995000in}}%
\pgfpathlineto{\pgfqpoint{4.434800in}{1.890000in}}%
\pgfpathlineto{\pgfqpoint{4.436040in}{1.925000in}}%
\pgfpathlineto{\pgfqpoint{4.437280in}{1.820000in}}%
\pgfpathlineto{\pgfqpoint{4.439760in}{1.190000in}}%
\pgfpathlineto{\pgfqpoint{4.442240in}{1.680000in}}%
\pgfpathlineto{\pgfqpoint{4.443480in}{1.540000in}}%
\pgfpathlineto{\pgfqpoint{4.444720in}{1.190000in}}%
\pgfpathlineto{\pgfqpoint{4.445960in}{1.330000in}}%
\pgfpathlineto{\pgfqpoint{4.447200in}{1.085000in}}%
\pgfpathlineto{\pgfqpoint{4.448440in}{1.260000in}}%
\pgfpathlineto{\pgfqpoint{4.449680in}{1.610000in}}%
\pgfpathlineto{\pgfqpoint{4.452160in}{1.015000in}}%
\pgfpathlineto{\pgfqpoint{4.453400in}{2.030000in}}%
\pgfpathlineto{\pgfqpoint{4.454640in}{1.225000in}}%
\pgfpathlineto{\pgfqpoint{4.455880in}{1.505000in}}%
\pgfpathlineto{\pgfqpoint{4.457120in}{1.015000in}}%
\pgfpathlineto{\pgfqpoint{4.459600in}{1.610000in}}%
\pgfpathlineto{\pgfqpoint{4.460840in}{1.050000in}}%
\pgfpathlineto{\pgfqpoint{4.462080in}{1.225000in}}%
\pgfpathlineto{\pgfqpoint{4.464560in}{1.750000in}}%
\pgfpathlineto{\pgfqpoint{4.465800in}{1.435000in}}%
\pgfpathlineto{\pgfqpoint{4.467040in}{1.645000in}}%
\pgfpathlineto{\pgfqpoint{4.468280in}{1.295000in}}%
\pgfpathlineto{\pgfqpoint{4.469520in}{1.435000in}}%
\pgfpathlineto{\pgfqpoint{4.470760in}{1.400000in}}%
\pgfpathlineto{\pgfqpoint{4.472000in}{1.680000in}}%
\pgfpathlineto{\pgfqpoint{4.474480in}{0.945000in}}%
\pgfpathlineto{\pgfqpoint{4.475720in}{1.050000in}}%
\pgfpathlineto{\pgfqpoint{4.476960in}{1.400000in}}%
\pgfpathlineto{\pgfqpoint{4.478200in}{1.365000in}}%
\pgfpathlineto{\pgfqpoint{4.479440in}{1.365000in}}%
\pgfpathlineto{\pgfqpoint{4.480680in}{1.190000in}}%
\pgfpathlineto{\pgfqpoint{4.481920in}{1.785000in}}%
\pgfpathlineto{\pgfqpoint{4.484400in}{1.295000in}}%
\pgfpathlineto{\pgfqpoint{4.485640in}{1.365000in}}%
\pgfpathlineto{\pgfqpoint{4.488120in}{0.980000in}}%
\pgfpathlineto{\pgfqpoint{4.490600in}{1.330000in}}%
\pgfpathlineto{\pgfqpoint{4.491840in}{1.400000in}}%
\pgfpathlineto{\pgfqpoint{4.494320in}{1.050000in}}%
\pgfpathlineto{\pgfqpoint{4.496800in}{1.610000in}}%
\pgfpathlineto{\pgfqpoint{4.499280in}{0.980000in}}%
\pgfpathlineto{\pgfqpoint{4.500520in}{1.190000in}}%
\pgfpathlineto{\pgfqpoint{4.501760in}{1.085000in}}%
\pgfpathlineto{\pgfqpoint{4.505480in}{1.715000in}}%
\pgfpathlineto{\pgfqpoint{4.509200in}{1.190000in}}%
\pgfpathlineto{\pgfqpoint{4.510440in}{1.575000in}}%
\pgfpathlineto{\pgfqpoint{4.512920in}{1.015000in}}%
\pgfpathlineto{\pgfqpoint{4.514160in}{0.980000in}}%
\pgfpathlineto{\pgfqpoint{4.516640in}{1.540000in}}%
\pgfpathlineto{\pgfqpoint{4.517880in}{1.155000in}}%
\pgfpathlineto{\pgfqpoint{4.520360in}{1.995000in}}%
\pgfpathlineto{\pgfqpoint{4.521600in}{1.575000in}}%
\pgfpathlineto{\pgfqpoint{4.522840in}{1.575000in}}%
\pgfpathlineto{\pgfqpoint{4.524080in}{1.645000in}}%
\pgfpathlineto{\pgfqpoint{4.525320in}{0.875000in}}%
\pgfpathlineto{\pgfqpoint{4.526560in}{1.785000in}}%
\pgfpathlineto{\pgfqpoint{4.527800in}{1.470000in}}%
\pgfpathlineto{\pgfqpoint{4.529040in}{1.610000in}}%
\pgfpathlineto{\pgfqpoint{4.531520in}{1.085000in}}%
\pgfpathlineto{\pgfqpoint{4.532760in}{1.365000in}}%
\pgfpathlineto{\pgfqpoint{4.534000in}{1.155000in}}%
\pgfpathlineto{\pgfqpoint{4.535240in}{1.750000in}}%
\pgfpathlineto{\pgfqpoint{4.536480in}{1.295000in}}%
\pgfpathlineto{\pgfqpoint{4.537720in}{1.400000in}}%
\pgfpathlineto{\pgfqpoint{4.538960in}{1.365000in}}%
\pgfpathlineto{\pgfqpoint{4.540200in}{1.365000in}}%
\pgfpathlineto{\pgfqpoint{4.542680in}{1.085000in}}%
\pgfpathlineto{\pgfqpoint{4.545160in}{1.470000in}}%
\pgfpathlineto{\pgfqpoint{4.546400in}{1.400000in}}%
\pgfpathlineto{\pgfqpoint{4.547640in}{0.980000in}}%
\pgfpathlineto{\pgfqpoint{4.548880in}{1.330000in}}%
\pgfpathlineto{\pgfqpoint{4.550120in}{1.330000in}}%
\pgfpathlineto{\pgfqpoint{4.551360in}{1.295000in}}%
\pgfpathlineto{\pgfqpoint{4.552600in}{1.015000in}}%
\pgfpathlineto{\pgfqpoint{4.555080in}{1.540000in}}%
\pgfpathlineto{\pgfqpoint{4.557560in}{1.435000in}}%
\pgfpathlineto{\pgfqpoint{4.558800in}{1.575000in}}%
\pgfpathlineto{\pgfqpoint{4.560040in}{1.330000in}}%
\pgfpathlineto{\pgfqpoint{4.561280in}{1.330000in}}%
\pgfpathlineto{\pgfqpoint{4.562520in}{1.715000in}}%
\pgfpathlineto{\pgfqpoint{4.563760in}{1.610000in}}%
\pgfpathlineto{\pgfqpoint{4.565000in}{1.645000in}}%
\pgfpathlineto{\pgfqpoint{4.566240in}{1.050000in}}%
\pgfpathlineto{\pgfqpoint{4.567480in}{1.610000in}}%
\pgfpathlineto{\pgfqpoint{4.568720in}{1.575000in}}%
\pgfpathlineto{\pgfqpoint{4.569960in}{1.645000in}}%
\pgfpathlineto{\pgfqpoint{4.571200in}{1.785000in}}%
\pgfpathlineto{\pgfqpoint{4.573680in}{1.295000in}}%
\pgfpathlineto{\pgfqpoint{4.576160in}{1.610000in}}%
\pgfpathlineto{\pgfqpoint{4.577400in}{1.505000in}}%
\pgfpathlineto{\pgfqpoint{4.578640in}{1.085000in}}%
\pgfpathlineto{\pgfqpoint{4.579880in}{1.540000in}}%
\pgfpathlineto{\pgfqpoint{4.582360in}{1.050000in}}%
\pgfpathlineto{\pgfqpoint{4.583600in}{1.120000in}}%
\pgfpathlineto{\pgfqpoint{4.586080in}{1.750000in}}%
\pgfpathlineto{\pgfqpoint{4.587320in}{1.330000in}}%
\pgfpathlineto{\pgfqpoint{4.589800in}{1.785000in}}%
\pgfpathlineto{\pgfqpoint{4.591040in}{1.750000in}}%
\pgfpathlineto{\pgfqpoint{4.592280in}{1.540000in}}%
\pgfpathlineto{\pgfqpoint{4.593520in}{1.540000in}}%
\pgfpathlineto{\pgfqpoint{4.594760in}{1.435000in}}%
\pgfpathlineto{\pgfqpoint{4.596000in}{1.645000in}}%
\pgfpathlineto{\pgfqpoint{4.597240in}{1.295000in}}%
\pgfpathlineto{\pgfqpoint{4.598480in}{1.365000in}}%
\pgfpathlineto{\pgfqpoint{4.599720in}{1.015000in}}%
\pgfpathlineto{\pgfqpoint{4.602200in}{1.470000in}}%
\pgfpathlineto{\pgfqpoint{4.603440in}{1.365000in}}%
\pgfpathlineto{\pgfqpoint{4.604680in}{1.400000in}}%
\pgfpathlineto{\pgfqpoint{4.607160in}{1.190000in}}%
\pgfpathlineto{\pgfqpoint{4.608400in}{1.190000in}}%
\pgfpathlineto{\pgfqpoint{4.610880in}{1.645000in}}%
\pgfpathlineto{\pgfqpoint{4.612120in}{1.085000in}}%
\pgfpathlineto{\pgfqpoint{4.613360in}{1.085000in}}%
\pgfpathlineto{\pgfqpoint{4.614600in}{1.505000in}}%
\pgfpathlineto{\pgfqpoint{4.615840in}{1.260000in}}%
\pgfpathlineto{\pgfqpoint{4.618320in}{1.575000in}}%
\pgfpathlineto{\pgfqpoint{4.619560in}{1.400000in}}%
\pgfpathlineto{\pgfqpoint{4.620800in}{1.715000in}}%
\pgfpathlineto{\pgfqpoint{4.622040in}{1.400000in}}%
\pgfpathlineto{\pgfqpoint{4.623280in}{1.750000in}}%
\pgfpathlineto{\pgfqpoint{4.625760in}{1.435000in}}%
\pgfpathlineto{\pgfqpoint{4.627000in}{1.260000in}}%
\pgfpathlineto{\pgfqpoint{4.628240in}{1.540000in}}%
\pgfpathlineto{\pgfqpoint{4.629480in}{1.295000in}}%
\pgfpathlineto{\pgfqpoint{4.630720in}{0.805000in}}%
\pgfpathlineto{\pgfqpoint{4.631960in}{1.645000in}}%
\pgfpathlineto{\pgfqpoint{4.633200in}{1.610000in}}%
\pgfpathlineto{\pgfqpoint{4.634440in}{1.330000in}}%
\pgfpathlineto{\pgfqpoint{4.635680in}{1.365000in}}%
\pgfpathlineto{\pgfqpoint{4.636920in}{0.980000in}}%
\pgfpathlineto{\pgfqpoint{4.638160in}{1.120000in}}%
\pgfpathlineto{\pgfqpoint{4.639400in}{1.505000in}}%
\pgfpathlineto{\pgfqpoint{4.641880in}{0.980000in}}%
\pgfpathlineto{\pgfqpoint{4.643120in}{1.120000in}}%
\pgfpathlineto{\pgfqpoint{4.644360in}{0.665000in}}%
\pgfpathlineto{\pgfqpoint{4.645600in}{1.400000in}}%
\pgfpathlineto{\pgfqpoint{4.646840in}{1.400000in}}%
\pgfpathlineto{\pgfqpoint{4.649320in}{1.890000in}}%
\pgfpathlineto{\pgfqpoint{4.650560in}{1.120000in}}%
\pgfpathlineto{\pgfqpoint{4.651800in}{1.365000in}}%
\pgfpathlineto{\pgfqpoint{4.653040in}{1.050000in}}%
\pgfpathlineto{\pgfqpoint{4.654280in}{1.715000in}}%
\pgfpathlineto{\pgfqpoint{4.655520in}{1.750000in}}%
\pgfpathlineto{\pgfqpoint{4.656760in}{1.715000in}}%
\pgfpathlineto{\pgfqpoint{4.658000in}{1.820000in}}%
\pgfpathlineto{\pgfqpoint{4.659240in}{1.435000in}}%
\pgfpathlineto{\pgfqpoint{4.660480in}{1.505000in}}%
\pgfpathlineto{\pgfqpoint{4.661720in}{1.400000in}}%
\pgfpathlineto{\pgfqpoint{4.662960in}{1.855000in}}%
\pgfpathlineto{\pgfqpoint{4.664200in}{1.470000in}}%
\pgfpathlineto{\pgfqpoint{4.665440in}{1.645000in}}%
\pgfpathlineto{\pgfqpoint{4.666680in}{1.610000in}}%
\pgfpathlineto{\pgfqpoint{4.667920in}{1.295000in}}%
\pgfpathlineto{\pgfqpoint{4.669160in}{1.645000in}}%
\pgfpathlineto{\pgfqpoint{4.670400in}{1.505000in}}%
\pgfpathlineto{\pgfqpoint{4.671640in}{0.945000in}}%
\pgfpathlineto{\pgfqpoint{4.672880in}{1.680000in}}%
\pgfpathlineto{\pgfqpoint{4.674120in}{1.260000in}}%
\pgfpathlineto{\pgfqpoint{4.676600in}{1.855000in}}%
\pgfpathlineto{\pgfqpoint{4.679080in}{1.575000in}}%
\pgfpathlineto{\pgfqpoint{4.680320in}{1.050000in}}%
\pgfpathlineto{\pgfqpoint{4.682800in}{1.890000in}}%
\pgfpathlineto{\pgfqpoint{4.684040in}{1.925000in}}%
\pgfpathlineto{\pgfqpoint{4.685280in}{1.295000in}}%
\pgfpathlineto{\pgfqpoint{4.686520in}{1.295000in}}%
\pgfpathlineto{\pgfqpoint{4.689000in}{1.575000in}}%
\pgfpathlineto{\pgfqpoint{4.690240in}{1.190000in}}%
\pgfpathlineto{\pgfqpoint{4.691480in}{1.435000in}}%
\pgfpathlineto{\pgfqpoint{4.692720in}{1.120000in}}%
\pgfpathlineto{\pgfqpoint{4.695200in}{1.855000in}}%
\pgfpathlineto{\pgfqpoint{4.696440in}{1.820000in}}%
\pgfpathlineto{\pgfqpoint{4.697680in}{1.750000in}}%
\pgfpathlineto{\pgfqpoint{4.698920in}{1.470000in}}%
\pgfpathlineto{\pgfqpoint{4.700160in}{1.610000in}}%
\pgfpathlineto{\pgfqpoint{4.701400in}{1.365000in}}%
\pgfpathlineto{\pgfqpoint{4.702640in}{1.540000in}}%
\pgfpathlineto{\pgfqpoint{4.703880in}{1.225000in}}%
\pgfpathlineto{\pgfqpoint{4.705120in}{1.505000in}}%
\pgfpathlineto{\pgfqpoint{4.706360in}{0.980000in}}%
\pgfpathlineto{\pgfqpoint{4.707600in}{1.680000in}}%
\pgfpathlineto{\pgfqpoint{4.708840in}{1.365000in}}%
\pgfpathlineto{\pgfqpoint{4.710080in}{1.435000in}}%
\pgfpathlineto{\pgfqpoint{4.711320in}{1.925000in}}%
\pgfpathlineto{\pgfqpoint{4.712560in}{1.435000in}}%
\pgfpathlineto{\pgfqpoint{4.713800in}{1.540000in}}%
\pgfpathlineto{\pgfqpoint{4.715040in}{1.155000in}}%
\pgfpathlineto{\pgfqpoint{4.716280in}{1.995000in}}%
\pgfpathlineto{\pgfqpoint{4.718760in}{1.400000in}}%
\pgfpathlineto{\pgfqpoint{4.720000in}{1.155000in}}%
\pgfpathlineto{\pgfqpoint{4.721240in}{1.750000in}}%
\pgfpathlineto{\pgfqpoint{4.723720in}{1.330000in}}%
\pgfpathlineto{\pgfqpoint{4.724960in}{1.365000in}}%
\pgfpathlineto{\pgfqpoint{4.726200in}{1.190000in}}%
\pgfpathlineto{\pgfqpoint{4.727440in}{1.260000in}}%
\pgfpathlineto{\pgfqpoint{4.728680in}{1.890000in}}%
\pgfpathlineto{\pgfqpoint{4.729920in}{1.365000in}}%
\pgfpathlineto{\pgfqpoint{4.731160in}{1.960000in}}%
\pgfpathlineto{\pgfqpoint{4.733640in}{1.610000in}}%
\pgfpathlineto{\pgfqpoint{4.736120in}{1.120000in}}%
\pgfpathlineto{\pgfqpoint{4.737360in}{1.050000in}}%
\pgfpathlineto{\pgfqpoint{4.738600in}{1.365000in}}%
\pgfpathlineto{\pgfqpoint{4.739840in}{1.330000in}}%
\pgfpathlineto{\pgfqpoint{4.741080in}{1.365000in}}%
\pgfpathlineto{\pgfqpoint{4.742320in}{1.330000in}}%
\pgfpathlineto{\pgfqpoint{4.743560in}{1.120000in}}%
\pgfpathlineto{\pgfqpoint{4.746040in}{1.785000in}}%
\pgfpathlineto{\pgfqpoint{4.747280in}{1.855000in}}%
\pgfpathlineto{\pgfqpoint{4.748520in}{1.435000in}}%
\pgfpathlineto{\pgfqpoint{4.749760in}{1.820000in}}%
\pgfpathlineto{\pgfqpoint{4.751000in}{1.610000in}}%
\pgfpathlineto{\pgfqpoint{4.752240in}{1.785000in}}%
\pgfpathlineto{\pgfqpoint{4.755960in}{1.435000in}}%
\pgfpathlineto{\pgfqpoint{4.757200in}{1.155000in}}%
\pgfpathlineto{\pgfqpoint{4.758440in}{1.645000in}}%
\pgfpathlineto{\pgfqpoint{4.760920in}{1.085000in}}%
\pgfpathlineto{\pgfqpoint{4.763400in}{1.645000in}}%
\pgfpathlineto{\pgfqpoint{4.764640in}{1.645000in}}%
\pgfpathlineto{\pgfqpoint{4.767120in}{1.085000in}}%
\pgfpathlineto{\pgfqpoint{4.769600in}{1.610000in}}%
\pgfpathlineto{\pgfqpoint{4.770840in}{1.330000in}}%
\pgfpathlineto{\pgfqpoint{4.773320in}{1.610000in}}%
\pgfpathlineto{\pgfqpoint{4.775800in}{1.295000in}}%
\pgfpathlineto{\pgfqpoint{4.779520in}{1.540000in}}%
\pgfpathlineto{\pgfqpoint{4.780760in}{1.365000in}}%
\pgfpathlineto{\pgfqpoint{4.782000in}{1.645000in}}%
\pgfpathlineto{\pgfqpoint{4.783240in}{1.015000in}}%
\pgfpathlineto{\pgfqpoint{4.784480in}{1.540000in}}%
\pgfpathlineto{\pgfqpoint{4.785720in}{1.295000in}}%
\pgfpathlineto{\pgfqpoint{4.786960in}{1.610000in}}%
\pgfpathlineto{\pgfqpoint{4.788200in}{1.365000in}}%
\pgfpathlineto{\pgfqpoint{4.790680in}{0.665000in}}%
\pgfpathlineto{\pgfqpoint{4.791920in}{1.400000in}}%
\pgfpathlineto{\pgfqpoint{4.794400in}{1.575000in}}%
\pgfpathlineto{\pgfqpoint{4.795640in}{1.890000in}}%
\pgfpathlineto{\pgfqpoint{4.796880in}{1.715000in}}%
\pgfpathlineto{\pgfqpoint{4.798120in}{1.365000in}}%
\pgfpathlineto{\pgfqpoint{4.799360in}{1.540000in}}%
\pgfpathlineto{\pgfqpoint{4.800600in}{0.980000in}}%
\pgfpathlineto{\pgfqpoint{4.801840in}{1.575000in}}%
\pgfpathlineto{\pgfqpoint{4.803080in}{1.435000in}}%
\pgfpathlineto{\pgfqpoint{4.804320in}{1.610000in}}%
\pgfpathlineto{\pgfqpoint{4.805560in}{1.015000in}}%
\pgfpathlineto{\pgfqpoint{4.806800in}{1.400000in}}%
\pgfpathlineto{\pgfqpoint{4.808040in}{1.295000in}}%
\pgfpathlineto{\pgfqpoint{4.810520in}{0.875000in}}%
\pgfpathlineto{\pgfqpoint{4.811760in}{1.680000in}}%
\pgfpathlineto{\pgfqpoint{4.813000in}{1.680000in}}%
\pgfpathlineto{\pgfqpoint{4.815480in}{1.330000in}}%
\pgfpathlineto{\pgfqpoint{4.817960in}{0.910000in}}%
\pgfpathlineto{\pgfqpoint{4.820440in}{1.750000in}}%
\pgfpathlineto{\pgfqpoint{4.822920in}{1.155000in}}%
\pgfpathlineto{\pgfqpoint{4.824160in}{0.875000in}}%
\pgfpathlineto{\pgfqpoint{4.826640in}{1.540000in}}%
\pgfpathlineto{\pgfqpoint{4.827880in}{1.190000in}}%
\pgfpathlineto{\pgfqpoint{4.830360in}{1.785000in}}%
\pgfpathlineto{\pgfqpoint{4.831600in}{1.575000in}}%
\pgfpathlineto{\pgfqpoint{4.832840in}{1.890000in}}%
\pgfpathlineto{\pgfqpoint{4.834080in}{1.470000in}}%
\pgfpathlineto{\pgfqpoint{4.835320in}{1.610000in}}%
\pgfpathlineto{\pgfqpoint{4.836560in}{1.225000in}}%
\pgfpathlineto{\pgfqpoint{4.837800in}{1.610000in}}%
\pgfpathlineto{\pgfqpoint{4.839040in}{1.365000in}}%
\pgfpathlineto{\pgfqpoint{4.841520in}{1.575000in}}%
\pgfpathlineto{\pgfqpoint{4.842760in}{1.540000in}}%
\pgfpathlineto{\pgfqpoint{4.844000in}{1.155000in}}%
\pgfpathlineto{\pgfqpoint{4.846480in}{1.785000in}}%
\pgfpathlineto{\pgfqpoint{4.847720in}{1.400000in}}%
\pgfpathlineto{\pgfqpoint{4.851440in}{1.960000in}}%
\pgfpathlineto{\pgfqpoint{4.852680in}{1.890000in}}%
\pgfpathlineto{\pgfqpoint{4.855160in}{1.015000in}}%
\pgfpathlineto{\pgfqpoint{4.856400in}{1.470000in}}%
\pgfpathlineto{\pgfqpoint{4.857640in}{1.050000in}}%
\pgfpathlineto{\pgfqpoint{4.858880in}{1.085000in}}%
\pgfpathlineto{\pgfqpoint{4.861360in}{1.505000in}}%
\pgfpathlineto{\pgfqpoint{4.862600in}{1.470000in}}%
\pgfpathlineto{\pgfqpoint{4.863840in}{1.890000in}}%
\pgfpathlineto{\pgfqpoint{4.865080in}{1.330000in}}%
\pgfpathlineto{\pgfqpoint{4.866320in}{1.785000in}}%
\pgfpathlineto{\pgfqpoint{4.867560in}{1.820000in}}%
\pgfpathlineto{\pgfqpoint{4.868800in}{1.400000in}}%
\pgfpathlineto{\pgfqpoint{4.870040in}{1.610000in}}%
\pgfpathlineto{\pgfqpoint{4.871280in}{1.610000in}}%
\pgfpathlineto{\pgfqpoint{4.872520in}{1.890000in}}%
\pgfpathlineto{\pgfqpoint{4.873760in}{1.890000in}}%
\pgfpathlineto{\pgfqpoint{4.875000in}{1.610000in}}%
\pgfpathlineto{\pgfqpoint{4.876240in}{1.785000in}}%
\pgfpathlineto{\pgfqpoint{4.878720in}{1.225000in}}%
\pgfpathlineto{\pgfqpoint{4.879960in}{1.435000in}}%
\pgfpathlineto{\pgfqpoint{4.881200in}{1.960000in}}%
\pgfpathlineto{\pgfqpoint{4.882440in}{1.365000in}}%
\pgfpathlineto{\pgfqpoint{4.884920in}{1.855000in}}%
\pgfpathlineto{\pgfqpoint{4.886160in}{1.680000in}}%
\pgfpathlineto{\pgfqpoint{4.887400in}{1.960000in}}%
\pgfpathlineto{\pgfqpoint{4.888640in}{1.890000in}}%
\pgfpathlineto{\pgfqpoint{4.889880in}{2.030000in}}%
\pgfpathlineto{\pgfqpoint{4.891120in}{1.470000in}}%
\pgfpathlineto{\pgfqpoint{4.892360in}{1.680000in}}%
\pgfpathlineto{\pgfqpoint{4.893600in}{1.505000in}}%
\pgfpathlineto{\pgfqpoint{4.894840in}{1.715000in}}%
\pgfpathlineto{\pgfqpoint{4.896080in}{1.470000in}}%
\pgfpathlineto{\pgfqpoint{4.897320in}{1.645000in}}%
\pgfpathlineto{\pgfqpoint{4.898560in}{1.610000in}}%
\pgfpathlineto{\pgfqpoint{4.899800in}{1.610000in}}%
\pgfpathlineto{\pgfqpoint{4.902280in}{1.400000in}}%
\pgfpathlineto{\pgfqpoint{4.903520in}{1.435000in}}%
\pgfpathlineto{\pgfqpoint{4.904760in}{1.540000in}}%
\pgfpathlineto{\pgfqpoint{4.906000in}{1.260000in}}%
\pgfpathlineto{\pgfqpoint{4.908480in}{1.610000in}}%
\pgfpathlineto{\pgfqpoint{4.912200in}{1.085000in}}%
\pgfpathlineto{\pgfqpoint{4.914680in}{1.505000in}}%
\pgfpathlineto{\pgfqpoint{4.915920in}{1.330000in}}%
\pgfpathlineto{\pgfqpoint{4.917160in}{1.575000in}}%
\pgfpathlineto{\pgfqpoint{4.918400in}{1.575000in}}%
\pgfpathlineto{\pgfqpoint{4.919640in}{1.330000in}}%
\pgfpathlineto{\pgfqpoint{4.920880in}{1.505000in}}%
\pgfpathlineto{\pgfqpoint{4.922120in}{1.855000in}}%
\pgfpathlineto{\pgfqpoint{4.923360in}{1.785000in}}%
\pgfpathlineto{\pgfqpoint{4.924600in}{1.995000in}}%
\pgfpathlineto{\pgfqpoint{4.925840in}{1.330000in}}%
\pgfpathlineto{\pgfqpoint{4.927080in}{1.400000in}}%
\pgfpathlineto{\pgfqpoint{4.928320in}{1.330000in}}%
\pgfpathlineto{\pgfqpoint{4.929560in}{1.015000in}}%
\pgfpathlineto{\pgfqpoint{4.930800in}{1.470000in}}%
\pgfpathlineto{\pgfqpoint{4.932040in}{0.875000in}}%
\pgfpathlineto{\pgfqpoint{4.933280in}{1.505000in}}%
\pgfpathlineto{\pgfqpoint{4.934520in}{1.050000in}}%
\pgfpathlineto{\pgfqpoint{4.937000in}{2.100000in}}%
\pgfpathlineto{\pgfqpoint{4.938240in}{1.225000in}}%
\pgfpathlineto{\pgfqpoint{4.939480in}{1.610000in}}%
\pgfpathlineto{\pgfqpoint{4.941960in}{1.225000in}}%
\pgfpathlineto{\pgfqpoint{4.944440in}{1.575000in}}%
\pgfpathlineto{\pgfqpoint{4.945680in}{2.065000in}}%
\pgfpathlineto{\pgfqpoint{4.946920in}{1.400000in}}%
\pgfpathlineto{\pgfqpoint{4.948160in}{1.400000in}}%
\pgfpathlineto{\pgfqpoint{4.949400in}{1.715000in}}%
\pgfpathlineto{\pgfqpoint{4.950640in}{1.575000in}}%
\pgfpathlineto{\pgfqpoint{4.951880in}{1.995000in}}%
\pgfpathlineto{\pgfqpoint{4.955600in}{1.435000in}}%
\pgfpathlineto{\pgfqpoint{4.958080in}{1.610000in}}%
\pgfpathlineto{\pgfqpoint{4.959320in}{1.015000in}}%
\pgfpathlineto{\pgfqpoint{4.961800in}{1.785000in}}%
\pgfpathlineto{\pgfqpoint{4.963040in}{1.435000in}}%
\pgfpathlineto{\pgfqpoint{4.964280in}{1.680000in}}%
\pgfpathlineto{\pgfqpoint{4.965520in}{1.645000in}}%
\pgfpathlineto{\pgfqpoint{4.968000in}{1.820000in}}%
\pgfpathlineto{\pgfqpoint{4.969240in}{1.890000in}}%
\pgfpathlineto{\pgfqpoint{4.971720in}{1.400000in}}%
\pgfpathlineto{\pgfqpoint{4.974200in}{1.645000in}}%
\pgfpathlineto{\pgfqpoint{4.976680in}{1.330000in}}%
\pgfpathlineto{\pgfqpoint{4.977920in}{1.890000in}}%
\pgfpathlineto{\pgfqpoint{4.980400in}{1.295000in}}%
\pgfpathlineto{\pgfqpoint{4.981640in}{1.785000in}}%
\pgfpathlineto{\pgfqpoint{4.984120in}{1.085000in}}%
\pgfpathlineto{\pgfqpoint{4.985360in}{1.295000in}}%
\pgfpathlineto{\pgfqpoint{4.986600in}{1.260000in}}%
\pgfpathlineto{\pgfqpoint{4.987840in}{1.085000in}}%
\pgfpathlineto{\pgfqpoint{4.989080in}{1.995000in}}%
\pgfpathlineto{\pgfqpoint{4.990320in}{0.980000in}}%
\pgfpathlineto{\pgfqpoint{4.994040in}{1.680000in}}%
\pgfpathlineto{\pgfqpoint{4.995280in}{1.225000in}}%
\pgfpathlineto{\pgfqpoint{4.996520in}{1.260000in}}%
\pgfpathlineto{\pgfqpoint{4.997760in}{1.715000in}}%
\pgfpathlineto{\pgfqpoint{4.999000in}{1.540000in}}%
\pgfpathlineto{\pgfqpoint{5.000240in}{1.540000in}}%
\pgfpathlineto{\pgfqpoint{5.001480in}{1.190000in}}%
\pgfpathlineto{\pgfqpoint{5.002720in}{1.435000in}}%
\pgfpathlineto{\pgfqpoint{5.006440in}{1.015000in}}%
\pgfpathlineto{\pgfqpoint{5.007680in}{1.645000in}}%
\pgfpathlineto{\pgfqpoint{5.008920in}{1.365000in}}%
\pgfpathlineto{\pgfqpoint{5.010160in}{1.575000in}}%
\pgfpathlineto{\pgfqpoint{5.011400in}{1.365000in}}%
\pgfpathlineto{\pgfqpoint{5.012640in}{1.715000in}}%
\pgfpathlineto{\pgfqpoint{5.015120in}{1.050000in}}%
\pgfpathlineto{\pgfqpoint{5.016360in}{1.190000in}}%
\pgfpathlineto{\pgfqpoint{5.017600in}{0.910000in}}%
\pgfpathlineto{\pgfqpoint{5.018840in}{1.330000in}}%
\pgfpathlineto{\pgfqpoint{5.020080in}{1.120000in}}%
\pgfpathlineto{\pgfqpoint{5.021320in}{1.120000in}}%
\pgfpathlineto{\pgfqpoint{5.023800in}{1.575000in}}%
\pgfpathlineto{\pgfqpoint{5.025040in}{1.120000in}}%
\pgfpathlineto{\pgfqpoint{5.026280in}{1.295000in}}%
\pgfpathlineto{\pgfqpoint{5.027520in}{1.155000in}}%
\pgfpathlineto{\pgfqpoint{5.028760in}{0.735000in}}%
\pgfpathlineto{\pgfqpoint{5.031240in}{1.680000in}}%
\pgfpathlineto{\pgfqpoint{5.032480in}{1.295000in}}%
\pgfpathlineto{\pgfqpoint{5.033720in}{1.435000in}}%
\pgfpathlineto{\pgfqpoint{5.036200in}{1.995000in}}%
\pgfpathlineto{\pgfqpoint{5.037440in}{1.890000in}}%
\pgfpathlineto{\pgfqpoint{5.038680in}{1.645000in}}%
\pgfpathlineto{\pgfqpoint{5.039920in}{1.715000in}}%
\pgfpathlineto{\pgfqpoint{5.041160in}{1.680000in}}%
\pgfpathlineto{\pgfqpoint{5.042400in}{1.820000in}}%
\pgfpathlineto{\pgfqpoint{5.044880in}{1.330000in}}%
\pgfpathlineto{\pgfqpoint{5.046120in}{1.400000in}}%
\pgfpathlineto{\pgfqpoint{5.047360in}{0.910000in}}%
\pgfpathlineto{\pgfqpoint{5.048600in}{1.855000in}}%
\pgfpathlineto{\pgfqpoint{5.049840in}{1.505000in}}%
\pgfpathlineto{\pgfqpoint{5.051080in}{1.470000in}}%
\pgfpathlineto{\pgfqpoint{5.052320in}{1.295000in}}%
\pgfpathlineto{\pgfqpoint{5.053560in}{1.820000in}}%
\pgfpathlineto{\pgfqpoint{5.054800in}{1.260000in}}%
\pgfpathlineto{\pgfqpoint{5.056040in}{1.470000in}}%
\pgfpathlineto{\pgfqpoint{5.057280in}{1.890000in}}%
\pgfpathlineto{\pgfqpoint{5.058520in}{1.260000in}}%
\pgfpathlineto{\pgfqpoint{5.059760in}{1.365000in}}%
\pgfpathlineto{\pgfqpoint{5.061000in}{1.155000in}}%
\pgfpathlineto{\pgfqpoint{5.062240in}{1.680000in}}%
\pgfpathlineto{\pgfqpoint{5.063480in}{1.225000in}}%
\pgfpathlineto{\pgfqpoint{5.065960in}{1.470000in}}%
\pgfpathlineto{\pgfqpoint{5.067200in}{1.400000in}}%
\pgfpathlineto{\pgfqpoint{5.068440in}{1.505000in}}%
\pgfpathlineto{\pgfqpoint{5.070920in}{1.050000in}}%
\pgfpathlineto{\pgfqpoint{5.072160in}{1.680000in}}%
\pgfpathlineto{\pgfqpoint{5.078360in}{1.050000in}}%
\pgfpathlineto{\pgfqpoint{5.080840in}{1.610000in}}%
\pgfpathlineto{\pgfqpoint{5.082080in}{1.610000in}}%
\pgfpathlineto{\pgfqpoint{5.083320in}{1.155000in}}%
\pgfpathlineto{\pgfqpoint{5.084560in}{1.785000in}}%
\pgfpathlineto{\pgfqpoint{5.088280in}{1.260000in}}%
\pgfpathlineto{\pgfqpoint{5.089520in}{1.680000in}}%
\pgfpathlineto{\pgfqpoint{5.092000in}{1.050000in}}%
\pgfpathlineto{\pgfqpoint{5.093240in}{1.715000in}}%
\pgfpathlineto{\pgfqpoint{5.094480in}{1.540000in}}%
\pgfpathlineto{\pgfqpoint{5.095720in}{0.945000in}}%
\pgfpathlineto{\pgfqpoint{5.096960in}{1.540000in}}%
\pgfpathlineto{\pgfqpoint{5.098200in}{1.575000in}}%
\pgfpathlineto{\pgfqpoint{5.099440in}{1.680000in}}%
\pgfpathlineto{\pgfqpoint{5.100680in}{1.435000in}}%
\pgfpathlineto{\pgfqpoint{5.101920in}{0.875000in}}%
\pgfpathlineto{\pgfqpoint{5.103160in}{0.875000in}}%
\pgfpathlineto{\pgfqpoint{5.104400in}{1.365000in}}%
\pgfpathlineto{\pgfqpoint{5.105640in}{1.260000in}}%
\pgfpathlineto{\pgfqpoint{5.106880in}{1.365000in}}%
\pgfpathlineto{\pgfqpoint{5.108120in}{1.295000in}}%
\pgfpathlineto{\pgfqpoint{5.110600in}{1.295000in}}%
\pgfpathlineto{\pgfqpoint{5.111840in}{1.750000in}}%
\pgfpathlineto{\pgfqpoint{5.113080in}{0.980000in}}%
\pgfpathlineto{\pgfqpoint{5.114320in}{1.540000in}}%
\pgfpathlineto{\pgfqpoint{5.116800in}{0.665000in}}%
\pgfpathlineto{\pgfqpoint{5.119280in}{1.470000in}}%
\pgfpathlineto{\pgfqpoint{5.120520in}{0.980000in}}%
\pgfpathlineto{\pgfqpoint{5.123000in}{1.645000in}}%
\pgfpathlineto{\pgfqpoint{5.124240in}{1.960000in}}%
\pgfpathlineto{\pgfqpoint{5.125480in}{1.435000in}}%
\pgfpathlineto{\pgfqpoint{5.126720in}{1.470000in}}%
\pgfpathlineto{\pgfqpoint{5.127960in}{1.715000in}}%
\pgfpathlineto{\pgfqpoint{5.130440in}{1.400000in}}%
\pgfpathlineto{\pgfqpoint{5.131680in}{1.645000in}}%
\pgfpathlineto{\pgfqpoint{5.132920in}{1.645000in}}%
\pgfpathlineto{\pgfqpoint{5.134160in}{1.610000in}}%
\pgfpathlineto{\pgfqpoint{5.135400in}{0.980000in}}%
\pgfpathlineto{\pgfqpoint{5.140360in}{1.855000in}}%
\pgfpathlineto{\pgfqpoint{5.141600in}{1.505000in}}%
\pgfpathlineto{\pgfqpoint{5.144080in}{2.065000in}}%
\pgfpathlineto{\pgfqpoint{5.145320in}{1.120000in}}%
\pgfpathlineto{\pgfqpoint{5.149040in}{1.785000in}}%
\pgfpathlineto{\pgfqpoint{5.150280in}{1.645000in}}%
\pgfpathlineto{\pgfqpoint{5.151520in}{1.260000in}}%
\pgfpathlineto{\pgfqpoint{5.152760in}{1.505000in}}%
\pgfpathlineto{\pgfqpoint{5.154000in}{1.260000in}}%
\pgfpathlineto{\pgfqpoint{5.156480in}{2.100000in}}%
\pgfpathlineto{\pgfqpoint{5.157720in}{1.365000in}}%
\pgfpathlineto{\pgfqpoint{5.158960in}{1.365000in}}%
\pgfpathlineto{\pgfqpoint{5.160200in}{1.715000in}}%
\pgfpathlineto{\pgfqpoint{5.162680in}{1.225000in}}%
\pgfpathlineto{\pgfqpoint{5.163920in}{1.715000in}}%
\pgfpathlineto{\pgfqpoint{5.165160in}{1.680000in}}%
\pgfpathlineto{\pgfqpoint{5.166400in}{1.435000in}}%
\pgfpathlineto{\pgfqpoint{5.167640in}{1.505000in}}%
\pgfpathlineto{\pgfqpoint{5.168880in}{1.750000in}}%
\pgfpathlineto{\pgfqpoint{5.171360in}{1.470000in}}%
\pgfpathlineto{\pgfqpoint{5.172600in}{1.365000in}}%
\pgfpathlineto{\pgfqpoint{5.173840in}{1.820000in}}%
\pgfpathlineto{\pgfqpoint{5.175080in}{1.155000in}}%
\pgfpathlineto{\pgfqpoint{5.176320in}{1.715000in}}%
\pgfpathlineto{\pgfqpoint{5.177560in}{1.645000in}}%
\pgfpathlineto{\pgfqpoint{5.178800in}{1.505000in}}%
\pgfpathlineto{\pgfqpoint{5.180040in}{1.645000in}}%
\pgfpathlineto{\pgfqpoint{5.181280in}{1.540000in}}%
\pgfpathlineto{\pgfqpoint{5.182520in}{1.575000in}}%
\pgfpathlineto{\pgfqpoint{5.183760in}{1.190000in}}%
\pgfpathlineto{\pgfqpoint{5.185000in}{1.680000in}}%
\pgfpathlineto{\pgfqpoint{5.186240in}{1.470000in}}%
\pgfpathlineto{\pgfqpoint{5.187480in}{1.050000in}}%
\pgfpathlineto{\pgfqpoint{5.188720in}{1.540000in}}%
\pgfpathlineto{\pgfqpoint{5.189960in}{1.540000in}}%
\pgfpathlineto{\pgfqpoint{5.191200in}{1.435000in}}%
\pgfpathlineto{\pgfqpoint{5.192440in}{1.225000in}}%
\pgfpathlineto{\pgfqpoint{5.193680in}{1.365000in}}%
\pgfpathlineto{\pgfqpoint{5.194920in}{1.820000in}}%
\pgfpathlineto{\pgfqpoint{5.196160in}{1.400000in}}%
\pgfpathlineto{\pgfqpoint{5.197400in}{1.610000in}}%
\pgfpathlineto{\pgfqpoint{5.198640in}{1.470000in}}%
\pgfpathlineto{\pgfqpoint{5.201120in}{1.715000in}}%
\pgfpathlineto{\pgfqpoint{5.203600in}{1.365000in}}%
\pgfpathlineto{\pgfqpoint{5.206080in}{1.715000in}}%
\pgfpathlineto{\pgfqpoint{5.207320in}{1.435000in}}%
\pgfpathlineto{\pgfqpoint{5.211040in}{1.890000in}}%
\pgfpathlineto{\pgfqpoint{5.212280in}{1.435000in}}%
\pgfpathlineto{\pgfqpoint{5.213520in}{1.505000in}}%
\pgfpathlineto{\pgfqpoint{5.216000in}{1.785000in}}%
\pgfpathlineto{\pgfqpoint{5.218480in}{1.120000in}}%
\pgfpathlineto{\pgfqpoint{5.219720in}{1.610000in}}%
\pgfpathlineto{\pgfqpoint{5.220960in}{1.015000in}}%
\pgfpathlineto{\pgfqpoint{5.222200in}{1.365000in}}%
\pgfpathlineto{\pgfqpoint{5.223440in}{1.085000in}}%
\pgfpathlineto{\pgfqpoint{5.224680in}{1.925000in}}%
\pgfpathlineto{\pgfqpoint{5.225920in}{1.785000in}}%
\pgfpathlineto{\pgfqpoint{5.227160in}{1.225000in}}%
\pgfpathlineto{\pgfqpoint{5.229640in}{1.715000in}}%
\pgfpathlineto{\pgfqpoint{5.230880in}{1.505000in}}%
\pgfpathlineto{\pgfqpoint{5.232120in}{1.890000in}}%
\pgfpathlineto{\pgfqpoint{5.233360in}{1.505000in}}%
\pgfpathlineto{\pgfqpoint{5.234600in}{1.785000in}}%
\pgfpathlineto{\pgfqpoint{5.235840in}{1.470000in}}%
\pgfpathlineto{\pgfqpoint{5.238320in}{1.680000in}}%
\pgfpathlineto{\pgfqpoint{5.239560in}{1.190000in}}%
\pgfpathlineto{\pgfqpoint{5.240800in}{1.575000in}}%
\pgfpathlineto{\pgfqpoint{5.242040in}{1.540000in}}%
\pgfpathlineto{\pgfqpoint{5.243280in}{1.260000in}}%
\pgfpathlineto{\pgfqpoint{5.245760in}{1.365000in}}%
\pgfpathlineto{\pgfqpoint{5.247000in}{1.085000in}}%
\pgfpathlineto{\pgfqpoint{5.248240in}{1.435000in}}%
\pgfpathlineto{\pgfqpoint{5.249480in}{1.295000in}}%
\pgfpathlineto{\pgfqpoint{5.250720in}{1.540000in}}%
\pgfpathlineto{\pgfqpoint{5.251960in}{1.050000in}}%
\pgfpathlineto{\pgfqpoint{5.253200in}{1.295000in}}%
\pgfpathlineto{\pgfqpoint{5.254440in}{1.785000in}}%
\pgfpathlineto{\pgfqpoint{5.255680in}{1.155000in}}%
\pgfpathlineto{\pgfqpoint{5.256920in}{1.155000in}}%
\pgfpathlineto{\pgfqpoint{5.260640in}{1.855000in}}%
\pgfpathlineto{\pgfqpoint{5.263120in}{1.155000in}}%
\pgfpathlineto{\pgfqpoint{5.264360in}{1.540000in}}%
\pgfpathlineto{\pgfqpoint{5.265600in}{1.225000in}}%
\pgfpathlineto{\pgfqpoint{5.268080in}{1.715000in}}%
\pgfpathlineto{\pgfqpoint{5.270560in}{1.435000in}}%
\pgfpathlineto{\pgfqpoint{5.271800in}{1.400000in}}%
\pgfpathlineto{\pgfqpoint{5.273040in}{1.085000in}}%
\pgfpathlineto{\pgfqpoint{5.275520in}{1.400000in}}%
\pgfpathlineto{\pgfqpoint{5.276760in}{1.330000in}}%
\pgfpathlineto{\pgfqpoint{5.279240in}{1.785000in}}%
\pgfpathlineto{\pgfqpoint{5.281720in}{1.435000in}}%
\pgfpathlineto{\pgfqpoint{5.282960in}{1.505000in}}%
\pgfpathlineto{\pgfqpoint{5.284200in}{1.645000in}}%
\pgfpathlineto{\pgfqpoint{5.286680in}{1.295000in}}%
\pgfpathlineto{\pgfqpoint{5.287920in}{1.610000in}}%
\pgfpathlineto{\pgfqpoint{5.289160in}{1.400000in}}%
\pgfpathlineto{\pgfqpoint{5.290400in}{1.400000in}}%
\pgfpathlineto{\pgfqpoint{5.291640in}{1.155000in}}%
\pgfpathlineto{\pgfqpoint{5.292880in}{1.365000in}}%
\pgfpathlineto{\pgfqpoint{5.294120in}{1.295000in}}%
\pgfpathlineto{\pgfqpoint{5.295360in}{1.470000in}}%
\pgfpathlineto{\pgfqpoint{5.296600in}{0.980000in}}%
\pgfpathlineto{\pgfqpoint{5.297840in}{1.820000in}}%
\pgfpathlineto{\pgfqpoint{5.299080in}{1.330000in}}%
\pgfpathlineto{\pgfqpoint{5.300320in}{1.505000in}}%
\pgfpathlineto{\pgfqpoint{5.302800in}{1.190000in}}%
\pgfpathlineto{\pgfqpoint{5.305280in}{1.575000in}}%
\pgfpathlineto{\pgfqpoint{5.306520in}{1.470000in}}%
\pgfpathlineto{\pgfqpoint{5.307760in}{1.680000in}}%
\pgfpathlineto{\pgfqpoint{5.310240in}{1.155000in}}%
\pgfpathlineto{\pgfqpoint{5.311480in}{1.610000in}}%
\pgfpathlineto{\pgfqpoint{5.313960in}{1.015000in}}%
\pgfpathlineto{\pgfqpoint{5.315200in}{1.050000in}}%
\pgfpathlineto{\pgfqpoint{5.316440in}{1.820000in}}%
\pgfpathlineto{\pgfqpoint{5.317680in}{1.435000in}}%
\pgfpathlineto{\pgfqpoint{5.318920in}{1.470000in}}%
\pgfpathlineto{\pgfqpoint{5.320160in}{1.435000in}}%
\pgfpathlineto{\pgfqpoint{5.322640in}{1.295000in}}%
\pgfpathlineto{\pgfqpoint{5.323880in}{1.470000in}}%
\pgfpathlineto{\pgfqpoint{5.325120in}{1.260000in}}%
\pgfpathlineto{\pgfqpoint{5.326360in}{1.750000in}}%
\pgfpathlineto{\pgfqpoint{5.328840in}{0.875000in}}%
\pgfpathlineto{\pgfqpoint{5.330080in}{0.980000in}}%
\pgfpathlineto{\pgfqpoint{5.331320in}{1.680000in}}%
\pgfpathlineto{\pgfqpoint{5.332560in}{1.715000in}}%
\pgfpathlineto{\pgfqpoint{5.333800in}{1.645000in}}%
\pgfpathlineto{\pgfqpoint{5.336280in}{1.260000in}}%
\pgfpathlineto{\pgfqpoint{5.337520in}{1.190000in}}%
\pgfpathlineto{\pgfqpoint{5.338760in}{1.225000in}}%
\pgfpathlineto{\pgfqpoint{5.340000in}{1.085000in}}%
\pgfpathlineto{\pgfqpoint{5.341240in}{1.680000in}}%
\pgfpathlineto{\pgfqpoint{5.342480in}{1.295000in}}%
\pgfpathlineto{\pgfqpoint{5.344960in}{1.540000in}}%
\pgfpathlineto{\pgfqpoint{5.346200in}{1.505000in}}%
\pgfpathlineto{\pgfqpoint{5.347440in}{2.030000in}}%
\pgfpathlineto{\pgfqpoint{5.348680in}{1.575000in}}%
\pgfpathlineto{\pgfqpoint{5.349920in}{1.820000in}}%
\pgfpathlineto{\pgfqpoint{5.351160in}{1.750000in}}%
\pgfpathlineto{\pgfqpoint{5.352400in}{1.365000in}}%
\pgfpathlineto{\pgfqpoint{5.353640in}{1.540000in}}%
\pgfpathlineto{\pgfqpoint{5.354880in}{1.260000in}}%
\pgfpathlineto{\pgfqpoint{5.356120in}{1.540000in}}%
\pgfpathlineto{\pgfqpoint{5.357360in}{1.015000in}}%
\pgfpathlineto{\pgfqpoint{5.358600in}{1.190000in}}%
\pgfpathlineto{\pgfqpoint{5.359840in}{1.680000in}}%
\pgfpathlineto{\pgfqpoint{5.362320in}{1.050000in}}%
\pgfpathlineto{\pgfqpoint{5.363560in}{1.085000in}}%
\pgfpathlineto{\pgfqpoint{5.364800in}{1.015000in}}%
\pgfpathlineto{\pgfqpoint{5.366040in}{1.540000in}}%
\pgfpathlineto{\pgfqpoint{5.367280in}{1.400000in}}%
\pgfpathlineto{\pgfqpoint{5.368520in}{1.085000in}}%
\pgfpathlineto{\pgfqpoint{5.372240in}{1.540000in}}%
\pgfpathlineto{\pgfqpoint{5.373480in}{1.400000in}}%
\pgfpathlineto{\pgfqpoint{5.374720in}{1.470000in}}%
\pgfpathlineto{\pgfqpoint{5.375960in}{1.225000in}}%
\pgfpathlineto{\pgfqpoint{5.378440in}{1.995000in}}%
\pgfpathlineto{\pgfqpoint{5.379680in}{1.400000in}}%
\pgfpathlineto{\pgfqpoint{5.380920in}{1.540000in}}%
\pgfpathlineto{\pgfqpoint{5.382160in}{1.295000in}}%
\pgfpathlineto{\pgfqpoint{5.383400in}{0.735000in}}%
\pgfpathlineto{\pgfqpoint{5.385880in}{1.435000in}}%
\pgfpathlineto{\pgfqpoint{5.387120in}{1.015000in}}%
\pgfpathlineto{\pgfqpoint{5.389600in}{1.575000in}}%
\pgfpathlineto{\pgfqpoint{5.390840in}{0.770000in}}%
\pgfpathlineto{\pgfqpoint{5.393320in}{1.785000in}}%
\pgfpathlineto{\pgfqpoint{5.394560in}{1.505000in}}%
\pgfpathlineto{\pgfqpoint{5.395800in}{0.945000in}}%
\pgfpathlineto{\pgfqpoint{5.398280in}{1.960000in}}%
\pgfpathlineto{\pgfqpoint{5.399520in}{1.400000in}}%
\pgfpathlineto{\pgfqpoint{5.400760in}{1.505000in}}%
\pgfpathlineto{\pgfqpoint{5.402000in}{1.365000in}}%
\pgfpathlineto{\pgfqpoint{5.404480in}{1.575000in}}%
\pgfpathlineto{\pgfqpoint{5.405720in}{1.610000in}}%
\pgfpathlineto{\pgfqpoint{5.408200in}{1.295000in}}%
\pgfpathlineto{\pgfqpoint{5.409440in}{1.225000in}}%
\pgfpathlineto{\pgfqpoint{5.410680in}{1.785000in}}%
\pgfpathlineto{\pgfqpoint{5.413160in}{1.680000in}}%
\pgfpathlineto{\pgfqpoint{5.414400in}{2.100000in}}%
\pgfpathlineto{\pgfqpoint{5.415640in}{1.575000in}}%
\pgfpathlineto{\pgfqpoint{5.416880in}{1.890000in}}%
\pgfpathlineto{\pgfqpoint{5.418120in}{1.295000in}}%
\pgfpathlineto{\pgfqpoint{5.419360in}{1.260000in}}%
\pgfpathlineto{\pgfqpoint{5.420600in}{1.470000in}}%
\pgfpathlineto{\pgfqpoint{5.421840in}{1.260000in}}%
\pgfpathlineto{\pgfqpoint{5.424320in}{1.890000in}}%
\pgfpathlineto{\pgfqpoint{5.425560in}{1.260000in}}%
\pgfpathlineto{\pgfqpoint{5.426800in}{1.750000in}}%
\pgfpathlineto{\pgfqpoint{5.429280in}{1.820000in}}%
\pgfpathlineto{\pgfqpoint{5.431760in}{1.645000in}}%
\pgfpathlineto{\pgfqpoint{5.433000in}{1.960000in}}%
\pgfpathlineto{\pgfqpoint{5.435480in}{1.050000in}}%
\pgfpathlineto{\pgfqpoint{5.436720in}{1.155000in}}%
\pgfpathlineto{\pgfqpoint{5.437960in}{1.750000in}}%
\pgfpathlineto{\pgfqpoint{5.440440in}{1.050000in}}%
\pgfpathlineto{\pgfqpoint{5.444160in}{1.575000in}}%
\pgfpathlineto{\pgfqpoint{5.447880in}{1.120000in}}%
\pgfpathlineto{\pgfqpoint{5.451600in}{1.540000in}}%
\pgfpathlineto{\pgfqpoint{5.454080in}{1.225000in}}%
\pgfpathlineto{\pgfqpoint{5.455320in}{1.505000in}}%
\pgfpathlineto{\pgfqpoint{5.456560in}{1.085000in}}%
\pgfpathlineto{\pgfqpoint{5.457800in}{1.610000in}}%
\pgfpathlineto{\pgfqpoint{5.459040in}{1.400000in}}%
\pgfpathlineto{\pgfqpoint{5.460280in}{1.540000in}}%
\pgfpathlineto{\pgfqpoint{5.461520in}{1.190000in}}%
\pgfpathlineto{\pgfqpoint{5.462760in}{1.645000in}}%
\pgfpathlineto{\pgfqpoint{5.464000in}{1.645000in}}%
\pgfpathlineto{\pgfqpoint{5.467720in}{1.120000in}}%
\pgfpathlineto{\pgfqpoint{5.468960in}{1.575000in}}%
\pgfpathlineto{\pgfqpoint{5.471440in}{1.330000in}}%
\pgfpathlineto{\pgfqpoint{5.472680in}{1.470000in}}%
\pgfpathlineto{\pgfqpoint{5.473920in}{1.785000in}}%
\pgfpathlineto{\pgfqpoint{5.477640in}{1.435000in}}%
\pgfpathlineto{\pgfqpoint{5.478880in}{1.925000in}}%
\pgfpathlineto{\pgfqpoint{5.481360in}{1.085000in}}%
\pgfpathlineto{\pgfqpoint{5.483840in}{1.295000in}}%
\pgfpathlineto{\pgfqpoint{5.485080in}{1.190000in}}%
\pgfpathlineto{\pgfqpoint{5.486320in}{1.400000in}}%
\pgfpathlineto{\pgfqpoint{5.487560in}{1.400000in}}%
\pgfpathlineto{\pgfqpoint{5.488800in}{1.155000in}}%
\pgfpathlineto{\pgfqpoint{5.492520in}{1.610000in}}%
\pgfpathlineto{\pgfqpoint{5.493760in}{1.715000in}}%
\pgfpathlineto{\pgfqpoint{5.495000in}{2.065000in}}%
\pgfpathlineto{\pgfqpoint{5.496240in}{1.330000in}}%
\pgfpathlineto{\pgfqpoint{5.497480in}{1.435000in}}%
\pgfpathlineto{\pgfqpoint{5.498720in}{1.680000in}}%
\pgfpathlineto{\pgfqpoint{5.499960in}{1.225000in}}%
\pgfpathlineto{\pgfqpoint{5.501200in}{1.610000in}}%
\pgfpathlineto{\pgfqpoint{5.502440in}{1.505000in}}%
\pgfpathlineto{\pgfqpoint{5.503680in}{0.770000in}}%
\pgfpathlineto{\pgfqpoint{5.504920in}{1.190000in}}%
\pgfpathlineto{\pgfqpoint{5.506160in}{1.155000in}}%
\pgfpathlineto{\pgfqpoint{5.508640in}{1.890000in}}%
\pgfpathlineto{\pgfqpoint{5.511120in}{1.435000in}}%
\pgfpathlineto{\pgfqpoint{5.513600in}{1.505000in}}%
\pgfpathlineto{\pgfqpoint{5.514840in}{1.400000in}}%
\pgfpathlineto{\pgfqpoint{5.516080in}{1.470000in}}%
\pgfpathlineto{\pgfqpoint{5.517320in}{1.470000in}}%
\pgfpathlineto{\pgfqpoint{5.518560in}{1.295000in}}%
\pgfpathlineto{\pgfqpoint{5.519800in}{1.435000in}}%
\pgfpathlineto{\pgfqpoint{5.521040in}{1.050000in}}%
\pgfpathlineto{\pgfqpoint{5.522280in}{1.715000in}}%
\pgfpathlineto{\pgfqpoint{5.523520in}{1.295000in}}%
\pgfpathlineto{\pgfqpoint{5.526000in}{1.610000in}}%
\pgfpathlineto{\pgfqpoint{5.527240in}{1.225000in}}%
\pgfpathlineto{\pgfqpoint{5.528480in}{1.260000in}}%
\pgfpathlineto{\pgfqpoint{5.530960in}{1.505000in}}%
\pgfpathlineto{\pgfqpoint{5.532200in}{1.435000in}}%
\pgfpathlineto{\pgfqpoint{5.534680in}{1.085000in}}%
\pgfpathlineto{\pgfqpoint{5.535920in}{1.715000in}}%
\pgfpathlineto{\pgfqpoint{5.537160in}{1.155000in}}%
\pgfpathlineto{\pgfqpoint{5.538400in}{1.295000in}}%
\pgfpathlineto{\pgfqpoint{5.539640in}{1.225000in}}%
\pgfpathlineto{\pgfqpoint{5.540880in}{1.365000in}}%
\pgfpathlineto{\pgfqpoint{5.542120in}{1.225000in}}%
\pgfpathlineto{\pgfqpoint{5.543360in}{1.680000in}}%
\pgfpathlineto{\pgfqpoint{5.544600in}{1.330000in}}%
\pgfpathlineto{\pgfqpoint{5.545840in}{1.295000in}}%
\pgfpathlineto{\pgfqpoint{5.547080in}{1.435000in}}%
\pgfpathlineto{\pgfqpoint{5.548320in}{1.435000in}}%
\pgfpathlineto{\pgfqpoint{5.549560in}{1.470000in}}%
\pgfpathlineto{\pgfqpoint{5.550800in}{1.470000in}}%
\pgfpathlineto{\pgfqpoint{5.552040in}{1.120000in}}%
\pgfpathlineto{\pgfqpoint{5.554520in}{1.715000in}}%
\pgfpathlineto{\pgfqpoint{5.555760in}{1.400000in}}%
\pgfpathlineto{\pgfqpoint{5.557000in}{1.540000in}}%
\pgfpathlineto{\pgfqpoint{5.558240in}{1.540000in}}%
\pgfpathlineto{\pgfqpoint{5.559480in}{1.505000in}}%
\pgfpathlineto{\pgfqpoint{5.560720in}{1.365000in}}%
\pgfpathlineto{\pgfqpoint{5.561960in}{1.855000in}}%
\pgfpathlineto{\pgfqpoint{5.563200in}{1.505000in}}%
\pgfpathlineto{\pgfqpoint{5.564440in}{1.645000in}}%
\pgfpathlineto{\pgfqpoint{5.565680in}{1.015000in}}%
\pgfpathlineto{\pgfqpoint{5.568160in}{1.435000in}}%
\pgfpathlineto{\pgfqpoint{5.569400in}{1.260000in}}%
\pgfpathlineto{\pgfqpoint{5.570640in}{1.540000in}}%
\pgfpathlineto{\pgfqpoint{5.571880in}{1.225000in}}%
\pgfpathlineto{\pgfqpoint{5.573120in}{1.680000in}}%
\pgfpathlineto{\pgfqpoint{5.574360in}{0.980000in}}%
\pgfpathlineto{\pgfqpoint{5.575600in}{1.610000in}}%
\pgfpathlineto{\pgfqpoint{5.576840in}{1.085000in}}%
\pgfpathlineto{\pgfqpoint{5.579320in}{1.575000in}}%
\pgfpathlineto{\pgfqpoint{5.580560in}{1.400000in}}%
\pgfpathlineto{\pgfqpoint{5.581800in}{1.785000in}}%
\pgfpathlineto{\pgfqpoint{5.583040in}{1.295000in}}%
\pgfpathlineto{\pgfqpoint{5.584280in}{1.820000in}}%
\pgfpathlineto{\pgfqpoint{5.585520in}{1.400000in}}%
\pgfpathlineto{\pgfqpoint{5.586760in}{1.505000in}}%
\pgfpathlineto{\pgfqpoint{5.588000in}{1.785000in}}%
\pgfpathlineto{\pgfqpoint{5.589240in}{1.365000in}}%
\pgfpathlineto{\pgfqpoint{5.590480in}{2.065000in}}%
\pgfpathlineto{\pgfqpoint{5.591720in}{2.030000in}}%
\pgfpathlineto{\pgfqpoint{5.592960in}{1.365000in}}%
\pgfpathlineto{\pgfqpoint{5.594200in}{1.435000in}}%
\pgfpathlineto{\pgfqpoint{5.595440in}{1.785000in}}%
\pgfpathlineto{\pgfqpoint{5.596680in}{1.295000in}}%
\pgfpathlineto{\pgfqpoint{5.597920in}{1.680000in}}%
\pgfpathlineto{\pgfqpoint{5.599160in}{1.540000in}}%
\pgfpathlineto{\pgfqpoint{5.600400in}{1.785000in}}%
\pgfpathlineto{\pgfqpoint{5.601640in}{1.330000in}}%
\pgfpathlineto{\pgfqpoint{5.604120in}{1.960000in}}%
\pgfpathlineto{\pgfqpoint{5.606600in}{1.190000in}}%
\pgfpathlineto{\pgfqpoint{5.607840in}{1.190000in}}%
\pgfpathlineto{\pgfqpoint{5.610320in}{1.680000in}}%
\pgfpathlineto{\pgfqpoint{5.611560in}{1.680000in}}%
\pgfpathlineto{\pgfqpoint{5.612800in}{0.910000in}}%
\pgfpathlineto{\pgfqpoint{5.614040in}{1.680000in}}%
\pgfpathlineto{\pgfqpoint{5.615280in}{1.610000in}}%
\pgfpathlineto{\pgfqpoint{5.617760in}{1.155000in}}%
\pgfpathlineto{\pgfqpoint{5.619000in}{1.260000in}}%
\pgfpathlineto{\pgfqpoint{5.621480in}{1.715000in}}%
\pgfpathlineto{\pgfqpoint{5.623960in}{1.295000in}}%
\pgfpathlineto{\pgfqpoint{5.625200in}{1.820000in}}%
\pgfpathlineto{\pgfqpoint{5.627680in}{1.295000in}}%
\pgfpathlineto{\pgfqpoint{5.630160in}{2.100000in}}%
\pgfpathlineto{\pgfqpoint{5.632640in}{1.330000in}}%
\pgfpathlineto{\pgfqpoint{5.633880in}{1.505000in}}%
\pgfpathlineto{\pgfqpoint{5.635120in}{1.505000in}}%
\pgfpathlineto{\pgfqpoint{5.636360in}{1.085000in}}%
\pgfpathlineto{\pgfqpoint{5.638840in}{1.610000in}}%
\pgfpathlineto{\pgfqpoint{5.641320in}{1.225000in}}%
\pgfpathlineto{\pgfqpoint{5.643800in}{1.575000in}}%
\pgfpathlineto{\pgfqpoint{5.645040in}{1.505000in}}%
\pgfpathlineto{\pgfqpoint{5.646280in}{1.505000in}}%
\pgfpathlineto{\pgfqpoint{5.647520in}{1.435000in}}%
\pgfpathlineto{\pgfqpoint{5.648760in}{1.785000in}}%
\pgfpathlineto{\pgfqpoint{5.650000in}{1.540000in}}%
\pgfpathlineto{\pgfqpoint{5.652480in}{1.890000in}}%
\pgfpathlineto{\pgfqpoint{5.653720in}{1.785000in}}%
\pgfpathlineto{\pgfqpoint{5.656200in}{1.330000in}}%
\pgfpathlineto{\pgfqpoint{5.657440in}{0.735000in}}%
\pgfpathlineto{\pgfqpoint{5.659920in}{1.960000in}}%
\pgfpathlineto{\pgfqpoint{5.662400in}{1.575000in}}%
\pgfpathlineto{\pgfqpoint{5.663640in}{1.855000in}}%
\pgfpathlineto{\pgfqpoint{5.664880in}{1.365000in}}%
\pgfpathlineto{\pgfqpoint{5.666120in}{1.400000in}}%
\pgfpathlineto{\pgfqpoint{5.667360in}{1.365000in}}%
\pgfpathlineto{\pgfqpoint{5.669840in}{1.540000in}}%
\pgfpathlineto{\pgfqpoint{5.671080in}{1.295000in}}%
\pgfpathlineto{\pgfqpoint{5.673560in}{1.575000in}}%
\pgfpathlineto{\pgfqpoint{5.674800in}{1.505000in}}%
\pgfpathlineto{\pgfqpoint{5.676040in}{1.680000in}}%
\pgfpathlineto{\pgfqpoint{5.677280in}{1.645000in}}%
\pgfpathlineto{\pgfqpoint{5.678520in}{1.680000in}}%
\pgfpathlineto{\pgfqpoint{5.681000in}{1.225000in}}%
\pgfpathlineto{\pgfqpoint{5.683480in}{1.960000in}}%
\pgfpathlineto{\pgfqpoint{5.684720in}{0.910000in}}%
\pgfpathlineto{\pgfqpoint{5.685960in}{1.400000in}}%
\pgfpathlineto{\pgfqpoint{5.687200in}{1.295000in}}%
\pgfpathlineto{\pgfqpoint{5.688440in}{1.575000in}}%
\pgfpathlineto{\pgfqpoint{5.689680in}{1.330000in}}%
\pgfpathlineto{\pgfqpoint{5.690920in}{1.540000in}}%
\pgfpathlineto{\pgfqpoint{5.692160in}{1.330000in}}%
\pgfpathlineto{\pgfqpoint{5.693400in}{1.785000in}}%
\pgfpathlineto{\pgfqpoint{5.694640in}{1.435000in}}%
\pgfpathlineto{\pgfqpoint{5.695880in}{1.470000in}}%
\pgfpathlineto{\pgfqpoint{5.697120in}{1.330000in}}%
\pgfpathlineto{\pgfqpoint{5.698360in}{1.435000in}}%
\pgfpathlineto{\pgfqpoint{5.699600in}{1.435000in}}%
\pgfpathlineto{\pgfqpoint{5.700840in}{1.925000in}}%
\pgfpathlineto{\pgfqpoint{5.702080in}{1.225000in}}%
\pgfpathlineto{\pgfqpoint{5.704560in}{1.715000in}}%
\pgfpathlineto{\pgfqpoint{5.705800in}{1.260000in}}%
\pgfpathlineto{\pgfqpoint{5.708280in}{1.505000in}}%
\pgfpathlineto{\pgfqpoint{5.709520in}{1.400000in}}%
\pgfpathlineto{\pgfqpoint{5.710760in}{1.820000in}}%
\pgfpathlineto{\pgfqpoint{5.712000in}{1.470000in}}%
\pgfpathlineto{\pgfqpoint{5.713240in}{1.645000in}}%
\pgfpathlineto{\pgfqpoint{5.714480in}{1.575000in}}%
\pgfpathlineto{\pgfqpoint{5.715720in}{1.890000in}}%
\pgfpathlineto{\pgfqpoint{5.716960in}{1.610000in}}%
\pgfpathlineto{\pgfqpoint{5.718200in}{1.610000in}}%
\pgfpathlineto{\pgfqpoint{5.719440in}{1.645000in}}%
\pgfpathlineto{\pgfqpoint{5.721920in}{1.435000in}}%
\pgfpathlineto{\pgfqpoint{5.723160in}{1.645000in}}%
\pgfpathlineto{\pgfqpoint{5.725640in}{1.330000in}}%
\pgfpathlineto{\pgfqpoint{5.726880in}{1.750000in}}%
\pgfpathlineto{\pgfqpoint{5.728120in}{1.435000in}}%
\pgfpathlineto{\pgfqpoint{5.729360in}{1.575000in}}%
\pgfpathlineto{\pgfqpoint{5.730600in}{1.330000in}}%
\pgfpathlineto{\pgfqpoint{5.731840in}{1.365000in}}%
\pgfpathlineto{\pgfqpoint{5.733080in}{1.330000in}}%
\pgfpathlineto{\pgfqpoint{5.734320in}{1.365000in}}%
\pgfpathlineto{\pgfqpoint{5.736800in}{1.295000in}}%
\pgfpathlineto{\pgfqpoint{5.738040in}{1.015000in}}%
\pgfpathlineto{\pgfqpoint{5.739280in}{1.400000in}}%
\pgfpathlineto{\pgfqpoint{5.741760in}{1.120000in}}%
\pgfpathlineto{\pgfqpoint{5.743000in}{1.610000in}}%
\pgfpathlineto{\pgfqpoint{5.745480in}{0.980000in}}%
\pgfpathlineto{\pgfqpoint{5.746720in}{1.645000in}}%
\pgfpathlineto{\pgfqpoint{5.747960in}{1.400000in}}%
\pgfpathlineto{\pgfqpoint{5.749200in}{1.645000in}}%
\pgfpathlineto{\pgfqpoint{5.750440in}{1.575000in}}%
\pgfpathlineto{\pgfqpoint{5.751680in}{1.890000in}}%
\pgfpathlineto{\pgfqpoint{5.752920in}{0.945000in}}%
\pgfpathlineto{\pgfqpoint{5.754160in}{1.645000in}}%
\pgfpathlineto{\pgfqpoint{5.755400in}{1.645000in}}%
\pgfpathlineto{\pgfqpoint{5.757880in}{0.875000in}}%
\pgfpathlineto{\pgfqpoint{5.760360in}{1.750000in}}%
\pgfpathlineto{\pgfqpoint{5.761600in}{1.155000in}}%
\pgfpathlineto{\pgfqpoint{5.762840in}{1.260000in}}%
\pgfpathlineto{\pgfqpoint{5.764080in}{0.980000in}}%
\pgfpathlineto{\pgfqpoint{5.765320in}{1.435000in}}%
\pgfpathlineto{\pgfqpoint{5.766560in}{1.435000in}}%
\pgfpathlineto{\pgfqpoint{5.769040in}{1.680000in}}%
\pgfpathlineto{\pgfqpoint{5.770280in}{0.980000in}}%
\pgfpathlineto{\pgfqpoint{5.772760in}{1.505000in}}%
\pgfpathlineto{\pgfqpoint{5.774000in}{0.980000in}}%
\pgfpathlineto{\pgfqpoint{5.776480in}{1.400000in}}%
\pgfpathlineto{\pgfqpoint{5.777720in}{1.435000in}}%
\pgfpathlineto{\pgfqpoint{5.780200in}{1.820000in}}%
\pgfpathlineto{\pgfqpoint{5.783920in}{1.120000in}}%
\pgfpathlineto{\pgfqpoint{5.785160in}{1.295000in}}%
\pgfpathlineto{\pgfqpoint{5.786400in}{1.190000in}}%
\pgfpathlineto{\pgfqpoint{5.787640in}{1.505000in}}%
\pgfpathlineto{\pgfqpoint{5.788880in}{1.365000in}}%
\pgfpathlineto{\pgfqpoint{5.790120in}{1.750000in}}%
\pgfpathlineto{\pgfqpoint{5.791360in}{1.365000in}}%
\pgfpathlineto{\pgfqpoint{5.792600in}{1.680000in}}%
\pgfpathlineto{\pgfqpoint{5.793840in}{1.295000in}}%
\pgfpathlineto{\pgfqpoint{5.795080in}{1.365000in}}%
\pgfpathlineto{\pgfqpoint{5.797560in}{1.680000in}}%
\pgfpathlineto{\pgfqpoint{5.800040in}{1.155000in}}%
\pgfpathlineto{\pgfqpoint{5.801280in}{1.610000in}}%
\pgfpathlineto{\pgfqpoint{5.802520in}{1.120000in}}%
\pgfpathlineto{\pgfqpoint{5.805000in}{1.680000in}}%
\pgfpathlineto{\pgfqpoint{5.806240in}{1.785000in}}%
\pgfpathlineto{\pgfqpoint{5.809960in}{1.225000in}}%
\pgfpathlineto{\pgfqpoint{5.811200in}{1.645000in}}%
\pgfpathlineto{\pgfqpoint{5.812440in}{1.435000in}}%
\pgfpathlineto{\pgfqpoint{5.813680in}{1.505000in}}%
\pgfpathlineto{\pgfqpoint{5.816160in}{1.820000in}}%
\pgfpathlineto{\pgfqpoint{5.817400in}{1.120000in}}%
\pgfpathlineto{\pgfqpoint{5.818640in}{1.575000in}}%
\pgfpathlineto{\pgfqpoint{5.819880in}{1.330000in}}%
\pgfpathlineto{\pgfqpoint{5.823600in}{2.030000in}}%
\pgfpathlineto{\pgfqpoint{5.824840in}{1.715000in}}%
\pgfpathlineto{\pgfqpoint{5.826080in}{0.875000in}}%
\pgfpathlineto{\pgfqpoint{5.827320in}{1.680000in}}%
\pgfpathlineto{\pgfqpoint{5.828560in}{1.750000in}}%
\pgfpathlineto{\pgfqpoint{5.829800in}{1.330000in}}%
\pgfpathlineto{\pgfqpoint{5.831040in}{1.330000in}}%
\pgfpathlineto{\pgfqpoint{5.832280in}{1.260000in}}%
\pgfpathlineto{\pgfqpoint{5.833520in}{1.610000in}}%
\pgfpathlineto{\pgfqpoint{5.834760in}{1.190000in}}%
\pgfpathlineto{\pgfqpoint{5.836000in}{1.225000in}}%
\pgfpathlineto{\pgfqpoint{5.839720in}{1.435000in}}%
\pgfpathlineto{\pgfqpoint{5.840960in}{1.225000in}}%
\pgfpathlineto{\pgfqpoint{5.842200in}{1.715000in}}%
\pgfpathlineto{\pgfqpoint{5.843440in}{1.120000in}}%
\pgfpathlineto{\pgfqpoint{5.845920in}{1.925000in}}%
\pgfpathlineto{\pgfqpoint{5.850880in}{1.470000in}}%
\pgfpathlineto{\pgfqpoint{5.852120in}{1.610000in}}%
\pgfpathlineto{\pgfqpoint{5.853360in}{1.505000in}}%
\pgfpathlineto{\pgfqpoint{5.854600in}{1.190000in}}%
\pgfpathlineto{\pgfqpoint{5.855840in}{1.295000in}}%
\pgfpathlineto{\pgfqpoint{5.857080in}{1.155000in}}%
\pgfpathlineto{\pgfqpoint{5.858320in}{1.190000in}}%
\pgfpathlineto{\pgfqpoint{5.859560in}{1.330000in}}%
\pgfpathlineto{\pgfqpoint{5.860800in}{1.225000in}}%
\pgfpathlineto{\pgfqpoint{5.862040in}{1.365000in}}%
\pgfpathlineto{\pgfqpoint{5.863280in}{1.855000in}}%
\pgfpathlineto{\pgfqpoint{5.864520in}{1.435000in}}%
\pgfpathlineto{\pgfqpoint{5.865760in}{1.750000in}}%
\pgfpathlineto{\pgfqpoint{5.867000in}{1.470000in}}%
\pgfpathlineto{\pgfqpoint{5.868240in}{1.470000in}}%
\pgfpathlineto{\pgfqpoint{5.869480in}{1.155000in}}%
\pgfpathlineto{\pgfqpoint{5.870720in}{1.260000in}}%
\pgfpathlineto{\pgfqpoint{5.871960in}{1.260000in}}%
\pgfpathlineto{\pgfqpoint{5.874440in}{1.435000in}}%
\pgfpathlineto{\pgfqpoint{5.875680in}{1.820000in}}%
\pgfpathlineto{\pgfqpoint{5.876920in}{1.365000in}}%
\pgfpathlineto{\pgfqpoint{5.878160in}{1.400000in}}%
\pgfpathlineto{\pgfqpoint{5.879400in}{1.330000in}}%
\pgfpathlineto{\pgfqpoint{5.880640in}{1.050000in}}%
\pgfpathlineto{\pgfqpoint{5.883120in}{1.610000in}}%
\pgfpathlineto{\pgfqpoint{5.884360in}{1.050000in}}%
\pgfpathlineto{\pgfqpoint{5.886840in}{1.505000in}}%
\pgfpathlineto{\pgfqpoint{5.888080in}{1.365000in}}%
\pgfpathlineto{\pgfqpoint{5.889320in}{1.400000in}}%
\pgfpathlineto{\pgfqpoint{5.890560in}{1.400000in}}%
\pgfpathlineto{\pgfqpoint{5.893040in}{1.750000in}}%
\pgfpathlineto{\pgfqpoint{5.894280in}{1.785000in}}%
\pgfpathlineto{\pgfqpoint{5.895520in}{1.260000in}}%
\pgfpathlineto{\pgfqpoint{5.896760in}{1.575000in}}%
\pgfpathlineto{\pgfqpoint{5.898000in}{1.435000in}}%
\pgfpathlineto{\pgfqpoint{5.899240in}{1.540000in}}%
\pgfpathlineto{\pgfqpoint{5.900480in}{1.435000in}}%
\pgfpathlineto{\pgfqpoint{5.901720in}{1.715000in}}%
\pgfpathlineto{\pgfqpoint{5.902960in}{1.435000in}}%
\pgfpathlineto{\pgfqpoint{5.904200in}{1.505000in}}%
\pgfpathlineto{\pgfqpoint{5.905440in}{1.365000in}}%
\pgfpathlineto{\pgfqpoint{5.906680in}{1.785000in}}%
\pgfpathlineto{\pgfqpoint{5.907920in}{1.785000in}}%
\pgfpathlineto{\pgfqpoint{5.910400in}{1.260000in}}%
\pgfpathlineto{\pgfqpoint{5.911640in}{1.260000in}}%
\pgfpathlineto{\pgfqpoint{5.912880in}{1.610000in}}%
\pgfpathlineto{\pgfqpoint{5.914120in}{1.610000in}}%
\pgfpathlineto{\pgfqpoint{5.915360in}{1.995000in}}%
\pgfpathlineto{\pgfqpoint{5.916600in}{1.540000in}}%
\pgfpathlineto{\pgfqpoint{5.917840in}{1.540000in}}%
\pgfpathlineto{\pgfqpoint{5.919080in}{1.330000in}}%
\pgfpathlineto{\pgfqpoint{5.920320in}{1.540000in}}%
\pgfpathlineto{\pgfqpoint{5.921560in}{0.980000in}}%
\pgfpathlineto{\pgfqpoint{5.922800in}{1.645000in}}%
\pgfpathlineto{\pgfqpoint{5.925280in}{1.120000in}}%
\pgfpathlineto{\pgfqpoint{5.926520in}{1.540000in}}%
\pgfpathlineto{\pgfqpoint{5.929000in}{1.295000in}}%
\pgfpathlineto{\pgfqpoint{5.930240in}{1.365000in}}%
\pgfpathlineto{\pgfqpoint{5.931480in}{1.645000in}}%
\pgfpathlineto{\pgfqpoint{5.932720in}{1.505000in}}%
\pgfpathlineto{\pgfqpoint{5.933960in}{1.855000in}}%
\pgfpathlineto{\pgfqpoint{5.936440in}{1.190000in}}%
\pgfpathlineto{\pgfqpoint{5.937680in}{1.330000in}}%
\pgfpathlineto{\pgfqpoint{5.938920in}{1.260000in}}%
\pgfpathlineto{\pgfqpoint{5.940160in}{1.400000in}}%
\pgfpathlineto{\pgfqpoint{5.941400in}{1.365000in}}%
\pgfpathlineto{\pgfqpoint{5.942640in}{1.890000in}}%
\pgfpathlineto{\pgfqpoint{5.945120in}{1.505000in}}%
\pgfpathlineto{\pgfqpoint{5.946360in}{1.680000in}}%
\pgfpathlineto{\pgfqpoint{5.947600in}{1.645000in}}%
\pgfpathlineto{\pgfqpoint{5.948840in}{1.120000in}}%
\pgfpathlineto{\pgfqpoint{5.950080in}{1.715000in}}%
\pgfpathlineto{\pgfqpoint{5.951320in}{1.365000in}}%
\pgfpathlineto{\pgfqpoint{5.953800in}{1.540000in}}%
\pgfpathlineto{\pgfqpoint{5.956280in}{1.120000in}}%
\pgfpathlineto{\pgfqpoint{5.957520in}{1.260000in}}%
\pgfpathlineto{\pgfqpoint{5.958760in}{1.085000in}}%
\pgfpathlineto{\pgfqpoint{5.962480in}{1.925000in}}%
\pgfpathlineto{\pgfqpoint{5.963720in}{1.575000in}}%
\pgfpathlineto{\pgfqpoint{5.964960in}{1.540000in}}%
\pgfpathlineto{\pgfqpoint{5.966200in}{1.435000in}}%
\pgfpathlineto{\pgfqpoint{5.967440in}{1.820000in}}%
\pgfpathlineto{\pgfqpoint{5.968680in}{0.980000in}}%
\pgfpathlineto{\pgfqpoint{5.969920in}{1.575000in}}%
\pgfpathlineto{\pgfqpoint{5.971160in}{1.540000in}}%
\pgfpathlineto{\pgfqpoint{5.972400in}{1.960000in}}%
\pgfpathlineto{\pgfqpoint{5.976120in}{1.260000in}}%
\pgfpathlineto{\pgfqpoint{5.977360in}{1.330000in}}%
\pgfpathlineto{\pgfqpoint{5.978600in}{1.295000in}}%
\pgfpathlineto{\pgfqpoint{5.979840in}{1.435000in}}%
\pgfpathlineto{\pgfqpoint{5.981080in}{1.260000in}}%
\pgfpathlineto{\pgfqpoint{5.982320in}{1.260000in}}%
\pgfpathlineto{\pgfqpoint{5.983560in}{1.470000in}}%
\pgfpathlineto{\pgfqpoint{5.986040in}{1.155000in}}%
\pgfpathlineto{\pgfqpoint{5.988520in}{1.610000in}}%
\pgfpathlineto{\pgfqpoint{5.991000in}{1.295000in}}%
\pgfpathlineto{\pgfqpoint{5.992240in}{1.260000in}}%
\pgfpathlineto{\pgfqpoint{5.993480in}{1.575000in}}%
\pgfpathlineto{\pgfqpoint{5.995960in}{1.155000in}}%
\pgfpathlineto{\pgfqpoint{5.997200in}{1.155000in}}%
\pgfpathlineto{\pgfqpoint{5.998440in}{1.470000in}}%
\pgfpathlineto{\pgfqpoint{5.999680in}{1.085000in}}%
\pgfpathlineto{\pgfqpoint{6.002160in}{1.505000in}}%
\pgfpathlineto{\pgfqpoint{6.003400in}{1.365000in}}%
\pgfpathlineto{\pgfqpoint{6.004640in}{1.505000in}}%
\pgfpathlineto{\pgfqpoint{6.005880in}{1.085000in}}%
\pgfpathlineto{\pgfqpoint{6.007120in}{1.610000in}}%
\pgfpathlineto{\pgfqpoint{6.008360in}{1.505000in}}%
\pgfpathlineto{\pgfqpoint{6.009600in}{1.540000in}}%
\pgfpathlineto{\pgfqpoint{6.010840in}{2.030000in}}%
\pgfpathlineto{\pgfqpoint{6.012080in}{1.435000in}}%
\pgfpathlineto{\pgfqpoint{6.013320in}{1.995000in}}%
\pgfpathlineto{\pgfqpoint{6.015800in}{1.680000in}}%
\pgfpathlineto{\pgfqpoint{6.017040in}{1.785000in}}%
\pgfpathlineto{\pgfqpoint{6.018280in}{1.575000in}}%
\pgfpathlineto{\pgfqpoint{6.019520in}{1.750000in}}%
\pgfpathlineto{\pgfqpoint{6.020760in}{1.715000in}}%
\pgfpathlineto{\pgfqpoint{6.022000in}{1.610000in}}%
\pgfpathlineto{\pgfqpoint{6.023240in}{1.260000in}}%
\pgfpathlineto{\pgfqpoint{6.024480in}{1.295000in}}%
\pgfpathlineto{\pgfqpoint{6.025720in}{1.610000in}}%
\pgfpathlineto{\pgfqpoint{6.026960in}{0.980000in}}%
\pgfpathlineto{\pgfqpoint{6.028200in}{1.190000in}}%
\pgfpathlineto{\pgfqpoint{6.029440in}{1.820000in}}%
\pgfpathlineto{\pgfqpoint{6.030680in}{1.750000in}}%
\pgfpathlineto{\pgfqpoint{6.031920in}{1.505000in}}%
\pgfpathlineto{\pgfqpoint{6.033160in}{1.785000in}}%
\pgfpathlineto{\pgfqpoint{6.034400in}{1.750000in}}%
\pgfpathlineto{\pgfqpoint{6.035640in}{1.540000in}}%
\pgfpathlineto{\pgfqpoint{6.036880in}{1.680000in}}%
\pgfpathlineto{\pgfqpoint{6.038120in}{1.960000in}}%
\pgfpathlineto{\pgfqpoint{6.039360in}{1.435000in}}%
\pgfpathlineto{\pgfqpoint{6.040600in}{1.505000in}}%
\pgfpathlineto{\pgfqpoint{6.041840in}{1.925000in}}%
\pgfpathlineto{\pgfqpoint{6.043080in}{1.330000in}}%
\pgfpathlineto{\pgfqpoint{6.045560in}{1.785000in}}%
\pgfpathlineto{\pgfqpoint{6.049280in}{0.980000in}}%
\pgfpathlineto{\pgfqpoint{6.050520in}{1.785000in}}%
\pgfpathlineto{\pgfqpoint{6.051760in}{1.400000in}}%
\pgfpathlineto{\pgfqpoint{6.054240in}{1.680000in}}%
\pgfpathlineto{\pgfqpoint{6.055480in}{1.435000in}}%
\pgfpathlineto{\pgfqpoint{6.056720in}{1.715000in}}%
\pgfpathlineto{\pgfqpoint{6.057960in}{1.680000in}}%
\pgfpathlineto{\pgfqpoint{6.060440in}{1.295000in}}%
\pgfpathlineto{\pgfqpoint{6.061680in}{1.575000in}}%
\pgfpathlineto{\pgfqpoint{6.062920in}{1.155000in}}%
\pgfpathlineto{\pgfqpoint{6.064160in}{1.575000in}}%
\pgfpathlineto{\pgfqpoint{6.065400in}{1.470000in}}%
\pgfpathlineto{\pgfqpoint{6.066640in}{1.645000in}}%
\pgfpathlineto{\pgfqpoint{6.067880in}{1.470000in}}%
\pgfpathlineto{\pgfqpoint{6.071600in}{1.715000in}}%
\pgfpathlineto{\pgfqpoint{6.072840in}{1.610000in}}%
\pgfpathlineto{\pgfqpoint{6.075320in}{1.855000in}}%
\pgfpathlineto{\pgfqpoint{6.076560in}{1.575000in}}%
\pgfpathlineto{\pgfqpoint{6.077800in}{1.575000in}}%
\pgfpathlineto{\pgfqpoint{6.079040in}{1.190000in}}%
\pgfpathlineto{\pgfqpoint{6.080280in}{1.365000in}}%
\pgfpathlineto{\pgfqpoint{6.081520in}{1.365000in}}%
\pgfpathlineto{\pgfqpoint{6.082760in}{1.505000in}}%
\pgfpathlineto{\pgfqpoint{6.084000in}{1.470000in}}%
\pgfpathlineto{\pgfqpoint{6.085240in}{1.260000in}}%
\pgfpathlineto{\pgfqpoint{6.088960in}{1.715000in}}%
\pgfpathlineto{\pgfqpoint{6.090200in}{1.505000in}}%
\pgfpathlineto{\pgfqpoint{6.091440in}{1.505000in}}%
\pgfpathlineto{\pgfqpoint{6.092680in}{1.400000in}}%
\pgfpathlineto{\pgfqpoint{6.093920in}{1.505000in}}%
\pgfpathlineto{\pgfqpoint{6.095160in}{1.365000in}}%
\pgfpathlineto{\pgfqpoint{6.096400in}{0.805000in}}%
\pgfpathlineto{\pgfqpoint{6.097640in}{1.960000in}}%
\pgfpathlineto{\pgfqpoint{6.098880in}{1.680000in}}%
\pgfpathlineto{\pgfqpoint{6.100120in}{1.855000in}}%
\pgfpathlineto{\pgfqpoint{6.101360in}{1.190000in}}%
\pgfpathlineto{\pgfqpoint{6.102600in}{1.435000in}}%
\pgfpathlineto{\pgfqpoint{6.103840in}{1.155000in}}%
\pgfpathlineto{\pgfqpoint{6.105080in}{1.400000in}}%
\pgfpathlineto{\pgfqpoint{6.106320in}{1.050000in}}%
\pgfpathlineto{\pgfqpoint{6.108800in}{1.680000in}}%
\pgfpathlineto{\pgfqpoint{6.111280in}{1.295000in}}%
\pgfpathlineto{\pgfqpoint{6.112520in}{1.610000in}}%
\pgfpathlineto{\pgfqpoint{6.113760in}{1.085000in}}%
\pgfpathlineto{\pgfqpoint{6.115000in}{1.470000in}}%
\pgfpathlineto{\pgfqpoint{6.116240in}{1.365000in}}%
\pgfpathlineto{\pgfqpoint{6.117480in}{1.540000in}}%
\pgfpathlineto{\pgfqpoint{6.118720in}{1.470000in}}%
\pgfpathlineto{\pgfqpoint{6.119960in}{1.295000in}}%
\pgfpathlineto{\pgfqpoint{6.121200in}{1.645000in}}%
\pgfpathlineto{\pgfqpoint{6.122440in}{1.225000in}}%
\pgfpathlineto{\pgfqpoint{6.124920in}{1.855000in}}%
\pgfpathlineto{\pgfqpoint{6.126160in}{1.435000in}}%
\pgfpathlineto{\pgfqpoint{6.127400in}{1.785000in}}%
\pgfpathlineto{\pgfqpoint{6.128640in}{1.295000in}}%
\pgfpathlineto{\pgfqpoint{6.129880in}{1.330000in}}%
\pgfpathlineto{\pgfqpoint{6.131120in}{1.400000in}}%
\pgfpathlineto{\pgfqpoint{6.132360in}{1.365000in}}%
\pgfpathlineto{\pgfqpoint{6.133600in}{1.470000in}}%
\pgfpathlineto{\pgfqpoint{6.134840in}{1.365000in}}%
\pgfpathlineto{\pgfqpoint{6.136080in}{1.120000in}}%
\pgfpathlineto{\pgfqpoint{6.137320in}{1.890000in}}%
\pgfpathlineto{\pgfqpoint{6.138560in}{1.925000in}}%
\pgfpathlineto{\pgfqpoint{6.139800in}{1.330000in}}%
\pgfpathlineto{\pgfqpoint{6.141040in}{1.715000in}}%
\pgfpathlineto{\pgfqpoint{6.142280in}{1.680000in}}%
\pgfpathlineto{\pgfqpoint{6.143520in}{1.960000in}}%
\pgfpathlineto{\pgfqpoint{6.144760in}{1.400000in}}%
\pgfpathlineto{\pgfqpoint{6.146000in}{1.540000in}}%
\pgfpathlineto{\pgfqpoint{6.147240in}{1.470000in}}%
\pgfpathlineto{\pgfqpoint{6.148480in}{1.540000in}}%
\pgfpathlineto{\pgfqpoint{6.149720in}{1.260000in}}%
\pgfpathlineto{\pgfqpoint{6.152200in}{1.330000in}}%
\pgfpathlineto{\pgfqpoint{6.153440in}{1.260000in}}%
\pgfpathlineto{\pgfqpoint{6.155920in}{1.540000in}}%
\pgfpathlineto{\pgfqpoint{6.157160in}{1.435000in}}%
\pgfpathlineto{\pgfqpoint{6.158400in}{0.875000in}}%
\pgfpathlineto{\pgfqpoint{6.159640in}{1.470000in}}%
\pgfpathlineto{\pgfqpoint{6.160880in}{1.505000in}}%
\pgfpathlineto{\pgfqpoint{6.162120in}{1.330000in}}%
\pgfpathlineto{\pgfqpoint{6.164600in}{1.330000in}}%
\pgfpathlineto{\pgfqpoint{6.165840in}{1.260000in}}%
\pgfpathlineto{\pgfqpoint{6.167080in}{0.980000in}}%
\pgfpathlineto{\pgfqpoint{6.168320in}{1.540000in}}%
\pgfpathlineto{\pgfqpoint{6.169560in}{1.540000in}}%
\pgfpathlineto{\pgfqpoint{6.172040in}{1.155000in}}%
\pgfpathlineto{\pgfqpoint{6.173280in}{1.330000in}}%
\pgfpathlineto{\pgfqpoint{6.174520in}{1.190000in}}%
\pgfpathlineto{\pgfqpoint{6.175760in}{1.400000in}}%
\pgfpathlineto{\pgfqpoint{6.177000in}{1.120000in}}%
\pgfpathlineto{\pgfqpoint{6.180720in}{1.925000in}}%
\pgfpathlineto{\pgfqpoint{6.181960in}{1.505000in}}%
\pgfpathlineto{\pgfqpoint{6.183200in}{1.610000in}}%
\pgfpathlineto{\pgfqpoint{6.185680in}{1.015000in}}%
\pgfpathlineto{\pgfqpoint{6.186920in}{1.470000in}}%
\pgfpathlineto{\pgfqpoint{6.189400in}{1.295000in}}%
\pgfpathlineto{\pgfqpoint{6.190640in}{0.980000in}}%
\pgfpathlineto{\pgfqpoint{6.191880in}{1.715000in}}%
\pgfpathlineto{\pgfqpoint{6.193120in}{1.260000in}}%
\pgfpathlineto{\pgfqpoint{6.194360in}{1.575000in}}%
\pgfpathlineto{\pgfqpoint{6.195600in}{1.470000in}}%
\pgfpathlineto{\pgfqpoint{6.196840in}{1.575000in}}%
\pgfpathlineto{\pgfqpoint{6.198080in}{1.260000in}}%
\pgfpathlineto{\pgfqpoint{6.199320in}{1.260000in}}%
\pgfpathlineto{\pgfqpoint{6.203040in}{1.645000in}}%
\pgfpathlineto{\pgfqpoint{6.205520in}{1.225000in}}%
\pgfpathlineto{\pgfqpoint{6.206760in}{1.190000in}}%
\pgfpathlineto{\pgfqpoint{6.208000in}{1.785000in}}%
\pgfpathlineto{\pgfqpoint{6.209240in}{1.120000in}}%
\pgfpathlineto{\pgfqpoint{6.212960in}{1.820000in}}%
\pgfpathlineto{\pgfqpoint{6.214200in}{1.855000in}}%
\pgfpathlineto{\pgfqpoint{6.215440in}{1.400000in}}%
\pgfpathlineto{\pgfqpoint{6.216680in}{1.750000in}}%
\pgfpathlineto{\pgfqpoint{6.219160in}{1.085000in}}%
\pgfpathlineto{\pgfqpoint{6.220400in}{1.645000in}}%
\pgfpathlineto{\pgfqpoint{6.221640in}{1.575000in}}%
\pgfpathlineto{\pgfqpoint{6.222880in}{1.365000in}}%
\pgfpathlineto{\pgfqpoint{6.224120in}{1.470000in}}%
\pgfpathlineto{\pgfqpoint{6.225360in}{1.680000in}}%
\pgfpathlineto{\pgfqpoint{6.226600in}{1.610000in}}%
\pgfpathlineto{\pgfqpoint{6.227840in}{1.680000in}}%
\pgfpathlineto{\pgfqpoint{6.230320in}{1.155000in}}%
\pgfpathlineto{\pgfqpoint{6.232800in}{1.645000in}}%
\pgfpathlineto{\pgfqpoint{6.234040in}{1.015000in}}%
\pgfpathlineto{\pgfqpoint{6.236520in}{1.610000in}}%
\pgfpathlineto{\pgfqpoint{6.237760in}{0.980000in}}%
\pgfpathlineto{\pgfqpoint{6.239000in}{1.995000in}}%
\pgfpathlineto{\pgfqpoint{6.240240in}{1.260000in}}%
\pgfpathlineto{\pgfqpoint{6.241480in}{1.855000in}}%
\pgfpathlineto{\pgfqpoint{6.242720in}{1.575000in}}%
\pgfpathlineto{\pgfqpoint{6.243960in}{1.680000in}}%
\pgfpathlineto{\pgfqpoint{6.245200in}{1.260000in}}%
\pgfpathlineto{\pgfqpoint{6.247680in}{1.190000in}}%
\pgfpathlineto{\pgfqpoint{6.248920in}{1.505000in}}%
\pgfpathlineto{\pgfqpoint{6.250160in}{1.260000in}}%
\pgfpathlineto{\pgfqpoint{6.251400in}{1.365000in}}%
\pgfpathlineto{\pgfqpoint{6.252640in}{1.085000in}}%
\pgfpathlineto{\pgfqpoint{6.253880in}{1.295000in}}%
\pgfpathlineto{\pgfqpoint{6.255120in}{1.050000in}}%
\pgfpathlineto{\pgfqpoint{6.257600in}{2.030000in}}%
\pgfpathlineto{\pgfqpoint{6.260080in}{1.540000in}}%
\pgfpathlineto{\pgfqpoint{6.261320in}{1.365000in}}%
\pgfpathlineto{\pgfqpoint{6.262560in}{1.505000in}}%
\pgfpathlineto{\pgfqpoint{6.263800in}{1.225000in}}%
\pgfpathlineto{\pgfqpoint{6.265040in}{1.715000in}}%
\pgfpathlineto{\pgfqpoint{6.266280in}{1.225000in}}%
\pgfpathlineto{\pgfqpoint{6.268760in}{1.470000in}}%
\pgfpathlineto{\pgfqpoint{6.270000in}{1.190000in}}%
\pgfpathlineto{\pgfqpoint{6.271240in}{1.225000in}}%
\pgfpathlineto{\pgfqpoint{6.272480in}{1.575000in}}%
\pgfpathlineto{\pgfqpoint{6.273720in}{1.575000in}}%
\pgfpathlineto{\pgfqpoint{6.274960in}{1.330000in}}%
\pgfpathlineto{\pgfqpoint{6.278680in}{1.750000in}}%
\pgfpathlineto{\pgfqpoint{6.279920in}{1.715000in}}%
\pgfpathlineto{\pgfqpoint{6.283640in}{1.295000in}}%
\pgfpathlineto{\pgfqpoint{6.284880in}{1.540000in}}%
\pgfpathlineto{\pgfqpoint{6.287360in}{1.190000in}}%
\pgfpathlineto{\pgfqpoint{6.288600in}{1.890000in}}%
\pgfpathlineto{\pgfqpoint{6.289840in}{1.155000in}}%
\pgfpathlineto{\pgfqpoint{6.291080in}{1.085000in}}%
\pgfpathlineto{\pgfqpoint{6.292320in}{1.575000in}}%
\pgfpathlineto{\pgfqpoint{6.293560in}{0.980000in}}%
\pgfpathlineto{\pgfqpoint{6.294800in}{1.645000in}}%
\pgfpathlineto{\pgfqpoint{6.296040in}{1.470000in}}%
\pgfpathlineto{\pgfqpoint{6.297280in}{1.645000in}}%
\pgfpathlineto{\pgfqpoint{6.298520in}{1.470000in}}%
\pgfpathlineto{\pgfqpoint{6.299760in}{1.085000in}}%
\pgfpathlineto{\pgfqpoint{6.301000in}{1.400000in}}%
\pgfpathlineto{\pgfqpoint{6.302240in}{1.295000in}}%
\pgfpathlineto{\pgfqpoint{6.304720in}{1.715000in}}%
\pgfpathlineto{\pgfqpoint{6.305960in}{1.610000in}}%
\pgfpathlineto{\pgfqpoint{6.307200in}{1.260000in}}%
\pgfpathlineto{\pgfqpoint{6.308440in}{1.400000in}}%
\pgfpathlineto{\pgfqpoint{6.309680in}{1.260000in}}%
\pgfpathlineto{\pgfqpoint{6.312160in}{1.540000in}}%
\pgfpathlineto{\pgfqpoint{6.314640in}{1.050000in}}%
\pgfpathlineto{\pgfqpoint{6.315880in}{1.610000in}}%
\pgfpathlineto{\pgfqpoint{6.317120in}{1.435000in}}%
\pgfpathlineto{\pgfqpoint{6.318360in}{1.540000in}}%
\pgfpathlineto{\pgfqpoint{6.319600in}{0.875000in}}%
\pgfpathlineto{\pgfqpoint{6.320840in}{1.225000in}}%
\pgfpathlineto{\pgfqpoint{6.322080in}{1.225000in}}%
\pgfpathlineto{\pgfqpoint{6.323320in}{1.540000in}}%
\pgfpathlineto{\pgfqpoint{6.327040in}{1.295000in}}%
\pgfpathlineto{\pgfqpoint{6.328280in}{0.735000in}}%
\pgfpathlineto{\pgfqpoint{6.329520in}{1.295000in}}%
\pgfpathlineto{\pgfqpoint{6.330760in}{1.155000in}}%
\pgfpathlineto{\pgfqpoint{6.332000in}{1.505000in}}%
\pgfpathlineto{\pgfqpoint{6.333240in}{1.190000in}}%
\pgfpathlineto{\pgfqpoint{6.334480in}{1.680000in}}%
\pgfpathlineto{\pgfqpoint{6.335720in}{1.540000in}}%
\pgfpathlineto{\pgfqpoint{6.336960in}{1.610000in}}%
\pgfpathlineto{\pgfqpoint{6.338200in}{1.015000in}}%
\pgfpathlineto{\pgfqpoint{6.339440in}{1.575000in}}%
\pgfpathlineto{\pgfqpoint{6.340680in}{1.400000in}}%
\pgfpathlineto{\pgfqpoint{6.341920in}{1.400000in}}%
\pgfpathlineto{\pgfqpoint{6.343160in}{1.120000in}}%
\pgfpathlineto{\pgfqpoint{6.345640in}{1.505000in}}%
\pgfpathlineto{\pgfqpoint{6.346880in}{1.400000in}}%
\pgfpathlineto{\pgfqpoint{6.349360in}{1.680000in}}%
\pgfpathlineto{\pgfqpoint{6.350600in}{1.155000in}}%
\pgfpathlineto{\pgfqpoint{6.351840in}{1.155000in}}%
\pgfpathlineto{\pgfqpoint{6.353080in}{1.575000in}}%
\pgfpathlineto{\pgfqpoint{6.354320in}{1.575000in}}%
\pgfpathlineto{\pgfqpoint{6.355560in}{1.085000in}}%
\pgfpathlineto{\pgfqpoint{6.356800in}{1.295000in}}%
\pgfpathlineto{\pgfqpoint{6.358040in}{1.260000in}}%
\pgfpathlineto{\pgfqpoint{6.359280in}{1.680000in}}%
\pgfpathlineto{\pgfqpoint{6.360520in}{1.400000in}}%
\pgfpathlineto{\pgfqpoint{6.361760in}{1.435000in}}%
\pgfpathlineto{\pgfqpoint{6.363000in}{1.540000in}}%
\pgfpathlineto{\pgfqpoint{6.364240in}{0.980000in}}%
\pgfpathlineto{\pgfqpoint{6.365480in}{1.575000in}}%
\pgfpathlineto{\pgfqpoint{6.366720in}{1.540000in}}%
\pgfpathlineto{\pgfqpoint{6.367960in}{1.855000in}}%
\pgfpathlineto{\pgfqpoint{6.369200in}{1.225000in}}%
\pgfpathlineto{\pgfqpoint{6.370440in}{1.330000in}}%
\pgfpathlineto{\pgfqpoint{6.372920in}{1.260000in}}%
\pgfpathlineto{\pgfqpoint{6.375400in}{1.575000in}}%
\pgfpathlineto{\pgfqpoint{6.376640in}{1.575000in}}%
\pgfpathlineto{\pgfqpoint{6.379120in}{1.715000in}}%
\pgfpathlineto{\pgfqpoint{6.380360in}{1.190000in}}%
\pgfpathlineto{\pgfqpoint{6.381600in}{1.645000in}}%
\pgfpathlineto{\pgfqpoint{6.382840in}{1.365000in}}%
\pgfpathlineto{\pgfqpoint{6.385320in}{1.575000in}}%
\pgfpathlineto{\pgfqpoint{6.386560in}{1.155000in}}%
\pgfpathlineto{\pgfqpoint{6.389040in}{1.505000in}}%
\pgfpathlineto{\pgfqpoint{6.390280in}{1.435000in}}%
\pgfpathlineto{\pgfqpoint{6.391520in}{1.470000in}}%
\pgfpathlineto{\pgfqpoint{6.392760in}{1.645000in}}%
\pgfpathlineto{\pgfqpoint{6.396480in}{1.260000in}}%
\pgfpathlineto{\pgfqpoint{6.397720in}{1.225000in}}%
\pgfpathlineto{\pgfqpoint{6.398960in}{1.260000in}}%
\pgfpathlineto{\pgfqpoint{6.400200in}{1.050000in}}%
\pgfpathlineto{\pgfqpoint{6.401440in}{1.575000in}}%
\pgfpathlineto{\pgfqpoint{6.403920in}{1.435000in}}%
\pgfpathlineto{\pgfqpoint{6.405160in}{1.050000in}}%
\pgfpathlineto{\pgfqpoint{6.406400in}{1.050000in}}%
\pgfpathlineto{\pgfqpoint{6.410120in}{1.540000in}}%
\pgfpathlineto{\pgfqpoint{6.411360in}{1.155000in}}%
\pgfpathlineto{\pgfqpoint{6.412600in}{1.155000in}}%
\pgfpathlineto{\pgfqpoint{6.413840in}{1.365000in}}%
\pgfpathlineto{\pgfqpoint{6.415080in}{1.120000in}}%
\pgfpathlineto{\pgfqpoint{6.416320in}{1.190000in}}%
\pgfpathlineto{\pgfqpoint{6.417560in}{1.365000in}}%
\pgfpathlineto{\pgfqpoint{6.418800in}{1.330000in}}%
\pgfpathlineto{\pgfqpoint{6.420040in}{1.085000in}}%
\pgfpathlineto{\pgfqpoint{6.421280in}{1.120000in}}%
\pgfpathlineto{\pgfqpoint{6.422520in}{1.575000in}}%
\pgfpathlineto{\pgfqpoint{6.426240in}{1.120000in}}%
\pgfpathlineto{\pgfqpoint{6.427480in}{1.505000in}}%
\pgfpathlineto{\pgfqpoint{6.428720in}{1.400000in}}%
\pgfpathlineto{\pgfqpoint{6.429960in}{1.435000in}}%
\pgfpathlineto{\pgfqpoint{6.431200in}{1.435000in}}%
\pgfpathlineto{\pgfqpoint{6.433680in}{1.190000in}}%
\pgfpathlineto{\pgfqpoint{6.434920in}{1.575000in}}%
\pgfpathlineto{\pgfqpoint{6.437400in}{1.330000in}}%
\pgfpathlineto{\pgfqpoint{6.438640in}{1.610000in}}%
\pgfpathlineto{\pgfqpoint{6.439880in}{1.575000in}}%
\pgfpathlineto{\pgfqpoint{6.441120in}{1.680000in}}%
\pgfpathlineto{\pgfqpoint{6.442360in}{1.400000in}}%
\pgfpathlineto{\pgfqpoint{6.443600in}{1.750000in}}%
\pgfpathlineto{\pgfqpoint{6.444840in}{1.260000in}}%
\pgfpathlineto{\pgfqpoint{6.446080in}{1.820000in}}%
\pgfpathlineto{\pgfqpoint{6.451040in}{1.400000in}}%
\pgfpathlineto{\pgfqpoint{6.452280in}{1.645000in}}%
\pgfpathlineto{\pgfqpoint{6.453520in}{1.400000in}}%
\pgfpathlineto{\pgfqpoint{6.454760in}{1.435000in}}%
\pgfpathlineto{\pgfqpoint{6.456000in}{1.435000in}}%
\pgfpathlineto{\pgfqpoint{6.458480in}{1.190000in}}%
\pgfpathlineto{\pgfqpoint{6.459720in}{1.435000in}}%
\pgfpathlineto{\pgfqpoint{6.460960in}{1.330000in}}%
\pgfpathlineto{\pgfqpoint{6.462200in}{1.470000in}}%
\pgfpathlineto{\pgfqpoint{6.463440in}{1.295000in}}%
\pgfpathlineto{\pgfqpoint{6.464680in}{1.505000in}}%
\pgfpathlineto{\pgfqpoint{6.467160in}{1.120000in}}%
\pgfpathlineto{\pgfqpoint{6.468400in}{1.330000in}}%
\pgfpathlineto{\pgfqpoint{6.469640in}{1.120000in}}%
\pgfpathlineto{\pgfqpoint{6.470880in}{1.750000in}}%
\pgfpathlineto{\pgfqpoint{6.472120in}{1.330000in}}%
\pgfpathlineto{\pgfqpoint{6.473360in}{1.575000in}}%
\pgfpathlineto{\pgfqpoint{6.474600in}{1.505000in}}%
\pgfpathlineto{\pgfqpoint{6.475840in}{1.855000in}}%
\pgfpathlineto{\pgfqpoint{6.477080in}{1.715000in}}%
\pgfpathlineto{\pgfqpoint{6.478320in}{1.820000in}}%
\pgfpathlineto{\pgfqpoint{6.482040in}{1.155000in}}%
\pgfpathlineto{\pgfqpoint{6.484520in}{1.365000in}}%
\pgfpathlineto{\pgfqpoint{6.485760in}{1.260000in}}%
\pgfpathlineto{\pgfqpoint{6.488240in}{1.260000in}}%
\pgfpathlineto{\pgfqpoint{6.489480in}{1.400000in}}%
\pgfpathlineto{\pgfqpoint{6.490720in}{1.155000in}}%
\pgfpathlineto{\pgfqpoint{6.493200in}{1.575000in}}%
\pgfpathlineto{\pgfqpoint{6.494440in}{1.120000in}}%
\pgfpathlineto{\pgfqpoint{6.495680in}{1.645000in}}%
\pgfpathlineto{\pgfqpoint{6.498160in}{1.155000in}}%
\pgfpathlineto{\pgfqpoint{6.500640in}{1.575000in}}%
\pgfpathlineto{\pgfqpoint{6.501880in}{1.610000in}}%
\pgfpathlineto{\pgfqpoint{6.503120in}{1.085000in}}%
\pgfpathlineto{\pgfqpoint{6.505600in}{1.750000in}}%
\pgfpathlineto{\pgfqpoint{6.506840in}{1.505000in}}%
\pgfpathlineto{\pgfqpoint{6.508080in}{1.610000in}}%
\pgfpathlineto{\pgfqpoint{6.509320in}{1.435000in}}%
\pgfpathlineto{\pgfqpoint{6.510560in}{1.785000in}}%
\pgfpathlineto{\pgfqpoint{6.511800in}{1.820000in}}%
\pgfpathlineto{\pgfqpoint{6.513040in}{1.295000in}}%
\pgfpathlineto{\pgfqpoint{6.515520in}{1.610000in}}%
\pgfpathlineto{\pgfqpoint{6.516760in}{0.840000in}}%
\pgfpathlineto{\pgfqpoint{6.518000in}{1.400000in}}%
\pgfpathlineto{\pgfqpoint{6.519240in}{1.295000in}}%
\pgfpathlineto{\pgfqpoint{6.520480in}{0.980000in}}%
\pgfpathlineto{\pgfqpoint{6.521720in}{1.330000in}}%
\pgfpathlineto{\pgfqpoint{6.522960in}{1.155000in}}%
\pgfpathlineto{\pgfqpoint{6.524200in}{1.470000in}}%
\pgfpathlineto{\pgfqpoint{6.525440in}{1.435000in}}%
\pgfpathlineto{\pgfqpoint{6.526680in}{1.330000in}}%
\pgfpathlineto{\pgfqpoint{6.527920in}{1.925000in}}%
\pgfpathlineto{\pgfqpoint{6.530400in}{1.330000in}}%
\pgfpathlineto{\pgfqpoint{6.531640in}{1.680000in}}%
\pgfpathlineto{\pgfqpoint{6.532880in}{1.435000in}}%
\pgfpathlineto{\pgfqpoint{6.534120in}{1.750000in}}%
\pgfpathlineto{\pgfqpoint{6.535360in}{1.225000in}}%
\pgfpathlineto{\pgfqpoint{6.537840in}{1.505000in}}%
\pgfpathlineto{\pgfqpoint{6.539080in}{1.120000in}}%
\pgfpathlineto{\pgfqpoint{6.540320in}{1.750000in}}%
\pgfpathlineto{\pgfqpoint{6.541560in}{0.980000in}}%
\pgfpathlineto{\pgfqpoint{6.542800in}{1.855000in}}%
\pgfpathlineto{\pgfqpoint{6.544040in}{1.085000in}}%
\pgfpathlineto{\pgfqpoint{6.545280in}{1.225000in}}%
\pgfpathlineto{\pgfqpoint{6.546520in}{1.785000in}}%
\pgfpathlineto{\pgfqpoint{6.547760in}{1.680000in}}%
\pgfpathlineto{\pgfqpoint{6.549000in}{1.330000in}}%
\pgfpathlineto{\pgfqpoint{6.550240in}{1.680000in}}%
\pgfpathlineto{\pgfqpoint{6.551480in}{1.680000in}}%
\pgfpathlineto{\pgfqpoint{6.552720in}{1.575000in}}%
\pgfpathlineto{\pgfqpoint{6.553960in}{1.295000in}}%
\pgfpathlineto{\pgfqpoint{6.555200in}{1.750000in}}%
\pgfpathlineto{\pgfqpoint{6.556440in}{1.610000in}}%
\pgfpathlineto{\pgfqpoint{6.557680in}{1.680000in}}%
\pgfpathlineto{\pgfqpoint{6.560160in}{1.085000in}}%
\pgfpathlineto{\pgfqpoint{6.561400in}{1.750000in}}%
\pgfpathlineto{\pgfqpoint{6.562640in}{1.085000in}}%
\pgfpathlineto{\pgfqpoint{6.565120in}{1.750000in}}%
\pgfpathlineto{\pgfqpoint{6.570080in}{1.400000in}}%
\pgfpathlineto{\pgfqpoint{6.571320in}{1.505000in}}%
\pgfpathlineto{\pgfqpoint{6.573800in}{1.505000in}}%
\pgfpathlineto{\pgfqpoint{6.575040in}{1.610000in}}%
\pgfpathlineto{\pgfqpoint{6.577520in}{1.540000in}}%
\pgfpathlineto{\pgfqpoint{6.578760in}{1.575000in}}%
\pgfpathlineto{\pgfqpoint{6.580000in}{1.435000in}}%
\pgfpathlineto{\pgfqpoint{6.582480in}{0.805000in}}%
\pgfpathlineto{\pgfqpoint{6.583720in}{1.645000in}}%
\pgfpathlineto{\pgfqpoint{6.584960in}{1.295000in}}%
\pgfpathlineto{\pgfqpoint{6.586200in}{1.610000in}}%
\pgfpathlineto{\pgfqpoint{6.587440in}{1.435000in}}%
\pgfpathlineto{\pgfqpoint{6.589920in}{0.945000in}}%
\pgfpathlineto{\pgfqpoint{6.592400in}{1.540000in}}%
\pgfpathlineto{\pgfqpoint{6.593640in}{0.980000in}}%
\pgfpathlineto{\pgfqpoint{6.594880in}{1.225000in}}%
\pgfpathlineto{\pgfqpoint{6.596120in}{1.225000in}}%
\pgfpathlineto{\pgfqpoint{6.597360in}{1.400000in}}%
\pgfpathlineto{\pgfqpoint{6.598600in}{0.945000in}}%
\pgfpathlineto{\pgfqpoint{6.601080in}{2.065000in}}%
\pgfpathlineto{\pgfqpoint{6.604800in}{1.225000in}}%
\pgfpathlineto{\pgfqpoint{6.606040in}{1.610000in}}%
\pgfpathlineto{\pgfqpoint{6.607280in}{1.645000in}}%
\pgfpathlineto{\pgfqpoint{6.608520in}{1.050000in}}%
\pgfpathlineto{\pgfqpoint{6.609760in}{1.820000in}}%
\pgfpathlineto{\pgfqpoint{6.611000in}{1.155000in}}%
\pgfpathlineto{\pgfqpoint{6.612240in}{1.820000in}}%
\pgfpathlineto{\pgfqpoint{6.613480in}{1.785000in}}%
\pgfpathlineto{\pgfqpoint{6.614720in}{1.365000in}}%
\pgfpathlineto{\pgfqpoint{6.615960in}{1.820000in}}%
\pgfpathlineto{\pgfqpoint{6.618440in}{1.225000in}}%
\pgfpathlineto{\pgfqpoint{6.622160in}{1.400000in}}%
\pgfpathlineto{\pgfqpoint{6.623400in}{1.365000in}}%
\pgfpathlineto{\pgfqpoint{6.624640in}{1.365000in}}%
\pgfpathlineto{\pgfqpoint{6.625880in}{1.575000in}}%
\pgfpathlineto{\pgfqpoint{6.627120in}{1.330000in}}%
\pgfpathlineto{\pgfqpoint{6.629600in}{1.960000in}}%
\pgfpathlineto{\pgfqpoint{6.630840in}{1.855000in}}%
\pgfpathlineto{\pgfqpoint{6.632080in}{1.610000in}}%
\pgfpathlineto{\pgfqpoint{6.633320in}{1.960000in}}%
\pgfpathlineto{\pgfqpoint{6.635800in}{1.750000in}}%
\pgfpathlineto{\pgfqpoint{6.637040in}{1.540000in}}%
\pgfpathlineto{\pgfqpoint{6.638280in}{1.120000in}}%
\pgfpathlineto{\pgfqpoint{6.639520in}{1.610000in}}%
\pgfpathlineto{\pgfqpoint{6.640760in}{1.400000in}}%
\pgfpathlineto{\pgfqpoint{6.642000in}{1.680000in}}%
\pgfpathlineto{\pgfqpoint{6.643240in}{1.575000in}}%
\pgfpathlineto{\pgfqpoint{6.645720in}{1.225000in}}%
\pgfpathlineto{\pgfqpoint{6.646960in}{1.260000in}}%
\pgfpathlineto{\pgfqpoint{6.648200in}{1.435000in}}%
\pgfpathlineto{\pgfqpoint{6.649440in}{1.855000in}}%
\pgfpathlineto{\pgfqpoint{6.653160in}{1.330000in}}%
\pgfpathlineto{\pgfqpoint{6.655640in}{1.505000in}}%
\pgfpathlineto{\pgfqpoint{6.656880in}{1.365000in}}%
\pgfpathlineto{\pgfqpoint{6.658120in}{1.575000in}}%
\pgfpathlineto{\pgfqpoint{6.660600in}{1.260000in}}%
\pgfpathlineto{\pgfqpoint{6.661840in}{1.540000in}}%
\pgfpathlineto{\pgfqpoint{6.663080in}{1.120000in}}%
\pgfpathlineto{\pgfqpoint{6.665560in}{1.610000in}}%
\pgfpathlineto{\pgfqpoint{6.666800in}{1.680000in}}%
\pgfpathlineto{\pgfqpoint{6.668040in}{1.435000in}}%
\pgfpathlineto{\pgfqpoint{6.669280in}{1.540000in}}%
\pgfpathlineto{\pgfqpoint{6.670520in}{1.330000in}}%
\pgfpathlineto{\pgfqpoint{6.673000in}{1.715000in}}%
\pgfpathlineto{\pgfqpoint{6.675480in}{1.435000in}}%
\pgfpathlineto{\pgfqpoint{6.676720in}{1.960000in}}%
\pgfpathlineto{\pgfqpoint{6.677960in}{1.575000in}}%
\pgfpathlineto{\pgfqpoint{6.679200in}{1.645000in}}%
\pgfpathlineto{\pgfqpoint{6.681680in}{1.365000in}}%
\pgfpathlineto{\pgfqpoint{6.682920in}{1.435000in}}%
\pgfpathlineto{\pgfqpoint{6.684160in}{1.295000in}}%
\pgfpathlineto{\pgfqpoint{6.686640in}{1.610000in}}%
\pgfpathlineto{\pgfqpoint{6.689120in}{1.225000in}}%
\pgfpathlineto{\pgfqpoint{6.690360in}{1.085000in}}%
\pgfpathlineto{\pgfqpoint{6.691600in}{1.120000in}}%
\pgfpathlineto{\pgfqpoint{6.692840in}{1.645000in}}%
\pgfpathlineto{\pgfqpoint{6.694080in}{1.470000in}}%
\pgfpathlineto{\pgfqpoint{6.695320in}{1.575000in}}%
\pgfpathlineto{\pgfqpoint{6.696560in}{1.505000in}}%
\pgfpathlineto{\pgfqpoint{6.697800in}{1.330000in}}%
\pgfpathlineto{\pgfqpoint{6.699040in}{1.365000in}}%
\pgfpathlineto{\pgfqpoint{6.700280in}{1.365000in}}%
\pgfpathlineto{\pgfqpoint{6.701520in}{1.470000in}}%
\pgfpathlineto{\pgfqpoint{6.702760in}{1.715000in}}%
\pgfpathlineto{\pgfqpoint{6.705240in}{1.330000in}}%
\pgfpathlineto{\pgfqpoint{6.706480in}{1.295000in}}%
\pgfpathlineto{\pgfqpoint{6.707720in}{1.750000in}}%
\pgfpathlineto{\pgfqpoint{6.708960in}{1.330000in}}%
\pgfpathlineto{\pgfqpoint{6.710200in}{1.295000in}}%
\pgfpathlineto{\pgfqpoint{6.711440in}{1.540000in}}%
\pgfpathlineto{\pgfqpoint{6.712680in}{1.225000in}}%
\pgfpathlineto{\pgfqpoint{6.715160in}{1.610000in}}%
\pgfpathlineto{\pgfqpoint{6.716400in}{1.610000in}}%
\pgfpathlineto{\pgfqpoint{6.717640in}{1.470000in}}%
\pgfpathlineto{\pgfqpoint{6.718880in}{1.470000in}}%
\pgfpathlineto{\pgfqpoint{6.720120in}{1.785000in}}%
\pgfpathlineto{\pgfqpoint{6.722600in}{1.260000in}}%
\pgfpathlineto{\pgfqpoint{6.723840in}{1.155000in}}%
\pgfpathlineto{\pgfqpoint{6.725080in}{1.540000in}}%
\pgfpathlineto{\pgfqpoint{6.726320in}{1.435000in}}%
\pgfpathlineto{\pgfqpoint{6.727560in}{1.435000in}}%
\pgfpathlineto{\pgfqpoint{6.728800in}{1.855000in}}%
\pgfpathlineto{\pgfqpoint{6.730040in}{1.540000in}}%
\pgfpathlineto{\pgfqpoint{6.731280in}{1.715000in}}%
\pgfpathlineto{\pgfqpoint{6.732520in}{1.680000in}}%
\pgfpathlineto{\pgfqpoint{6.733760in}{1.155000in}}%
\pgfpathlineto{\pgfqpoint{6.735000in}{1.330000in}}%
\pgfpathlineto{\pgfqpoint{6.736240in}{1.260000in}}%
\pgfpathlineto{\pgfqpoint{6.737480in}{1.680000in}}%
\pgfpathlineto{\pgfqpoint{6.738720in}{1.295000in}}%
\pgfpathlineto{\pgfqpoint{6.739960in}{1.330000in}}%
\pgfpathlineto{\pgfqpoint{6.741200in}{1.225000in}}%
\pgfpathlineto{\pgfqpoint{6.742440in}{1.645000in}}%
\pgfpathlineto{\pgfqpoint{6.743680in}{1.365000in}}%
\pgfpathlineto{\pgfqpoint{6.744920in}{1.435000in}}%
\pgfpathlineto{\pgfqpoint{6.746160in}{1.925000in}}%
\pgfpathlineto{\pgfqpoint{6.747400in}{1.680000in}}%
\pgfpathlineto{\pgfqpoint{6.748640in}{1.680000in}}%
\pgfpathlineto{\pgfqpoint{6.749880in}{1.470000in}}%
\pgfpathlineto{\pgfqpoint{6.751120in}{1.995000in}}%
\pgfpathlineto{\pgfqpoint{6.752360in}{1.540000in}}%
\pgfpathlineto{\pgfqpoint{6.753600in}{1.540000in}}%
\pgfpathlineto{\pgfqpoint{6.754840in}{1.470000in}}%
\pgfpathlineto{\pgfqpoint{6.756080in}{1.505000in}}%
\pgfpathlineto{\pgfqpoint{6.757320in}{1.155000in}}%
\pgfpathlineto{\pgfqpoint{6.758560in}{1.365000in}}%
\pgfpathlineto{\pgfqpoint{6.759800in}{2.030000in}}%
\pgfpathlineto{\pgfqpoint{6.762280in}{1.505000in}}%
\pgfpathlineto{\pgfqpoint{6.763520in}{1.470000in}}%
\pgfpathlineto{\pgfqpoint{6.766000in}{1.855000in}}%
\pgfpathlineto{\pgfqpoint{6.769720in}{1.435000in}}%
\pgfpathlineto{\pgfqpoint{6.770960in}{1.960000in}}%
\pgfpathlineto{\pgfqpoint{6.772200in}{1.750000in}}%
\pgfpathlineto{\pgfqpoint{6.773440in}{1.260000in}}%
\pgfpathlineto{\pgfqpoint{6.774680in}{1.540000in}}%
\pgfpathlineto{\pgfqpoint{6.775920in}{1.260000in}}%
\pgfpathlineto{\pgfqpoint{6.777160in}{1.575000in}}%
\pgfpathlineto{\pgfqpoint{6.778400in}{1.330000in}}%
\pgfpathlineto{\pgfqpoint{6.779640in}{1.330000in}}%
\pgfpathlineto{\pgfqpoint{6.780880in}{1.750000in}}%
\pgfpathlineto{\pgfqpoint{6.782120in}{1.540000in}}%
\pgfpathlineto{\pgfqpoint{6.783360in}{1.995000in}}%
\pgfpathlineto{\pgfqpoint{6.784600in}{1.820000in}}%
\pgfpathlineto{\pgfqpoint{6.785840in}{1.295000in}}%
\pgfpathlineto{\pgfqpoint{6.788320in}{1.890000in}}%
\pgfpathlineto{\pgfqpoint{6.789560in}{1.890000in}}%
\pgfpathlineto{\pgfqpoint{6.790800in}{1.295000in}}%
\pgfpathlineto{\pgfqpoint{6.793280in}{1.540000in}}%
\pgfpathlineto{\pgfqpoint{6.795760in}{1.295000in}}%
\pgfpathlineto{\pgfqpoint{6.798240in}{1.855000in}}%
\pgfpathlineto{\pgfqpoint{6.799480in}{1.155000in}}%
\pgfpathlineto{\pgfqpoint{6.800720in}{1.505000in}}%
\pgfpathlineto{\pgfqpoint{6.801960in}{1.470000in}}%
\pgfpathlineto{\pgfqpoint{6.803200in}{1.260000in}}%
\pgfpathlineto{\pgfqpoint{6.804440in}{1.540000in}}%
\pgfpathlineto{\pgfqpoint{6.805680in}{0.980000in}}%
\pgfpathlineto{\pgfqpoint{6.808160in}{1.470000in}}%
\pgfpathlineto{\pgfqpoint{6.809400in}{1.365000in}}%
\pgfpathlineto{\pgfqpoint{6.810640in}{1.120000in}}%
\pgfpathlineto{\pgfqpoint{6.811880in}{1.855000in}}%
\pgfpathlineto{\pgfqpoint{6.813120in}{1.470000in}}%
\pgfpathlineto{\pgfqpoint{6.815600in}{1.575000in}}%
\pgfpathlineto{\pgfqpoint{6.816840in}{1.260000in}}%
\pgfpathlineto{\pgfqpoint{6.818080in}{1.610000in}}%
\pgfpathlineto{\pgfqpoint{6.819320in}{1.610000in}}%
\pgfpathlineto{\pgfqpoint{6.820560in}{1.470000in}}%
\pgfpathlineto{\pgfqpoint{6.821800in}{1.505000in}}%
\pgfpathlineto{\pgfqpoint{6.823040in}{1.225000in}}%
\pgfpathlineto{\pgfqpoint{6.824280in}{1.225000in}}%
\pgfpathlineto{\pgfqpoint{6.825520in}{1.365000in}}%
\pgfpathlineto{\pgfqpoint{6.826760in}{1.225000in}}%
\pgfpathlineto{\pgfqpoint{6.828000in}{1.505000in}}%
\pgfpathlineto{\pgfqpoint{6.829240in}{0.875000in}}%
\pgfpathlineto{\pgfqpoint{6.831720in}{1.610000in}}%
\pgfpathlineto{\pgfqpoint{6.834200in}{1.400000in}}%
\pgfpathlineto{\pgfqpoint{6.835440in}{1.715000in}}%
\pgfpathlineto{\pgfqpoint{6.836680in}{1.715000in}}%
\pgfpathlineto{\pgfqpoint{6.837920in}{1.890000in}}%
\pgfpathlineto{\pgfqpoint{6.839160in}{1.505000in}}%
\pgfpathlineto{\pgfqpoint{6.840400in}{1.610000in}}%
\pgfpathlineto{\pgfqpoint{6.841640in}{1.505000in}}%
\pgfpathlineto{\pgfqpoint{6.842880in}{1.505000in}}%
\pgfpathlineto{\pgfqpoint{6.844120in}{1.995000in}}%
\pgfpathlineto{\pgfqpoint{6.845360in}{1.575000in}}%
\pgfpathlineto{\pgfqpoint{6.846600in}{1.855000in}}%
\pgfpathlineto{\pgfqpoint{6.847840in}{1.750000in}}%
\pgfpathlineto{\pgfqpoint{6.849080in}{1.400000in}}%
\pgfpathlineto{\pgfqpoint{6.851560in}{1.785000in}}%
\pgfpathlineto{\pgfqpoint{6.852800in}{1.750000in}}%
\pgfpathlineto{\pgfqpoint{6.854040in}{2.100000in}}%
\pgfpathlineto{\pgfqpoint{6.856520in}{1.680000in}}%
\pgfpathlineto{\pgfqpoint{6.857760in}{1.960000in}}%
\pgfpathlineto{\pgfqpoint{6.859000in}{1.645000in}}%
\pgfpathlineto{\pgfqpoint{6.860240in}{1.680000in}}%
\pgfpathlineto{\pgfqpoint{6.861480in}{1.645000in}}%
\pgfpathlineto{\pgfqpoint{6.862720in}{1.505000in}}%
\pgfpathlineto{\pgfqpoint{6.863960in}{1.715000in}}%
\pgfpathlineto{\pgfqpoint{6.865200in}{1.645000in}}%
\pgfpathlineto{\pgfqpoint{6.866440in}{1.750000in}}%
\pgfpathlineto{\pgfqpoint{6.867680in}{1.680000in}}%
\pgfpathlineto{\pgfqpoint{6.868920in}{1.750000in}}%
\pgfpathlineto{\pgfqpoint{6.870160in}{1.400000in}}%
\pgfpathlineto{\pgfqpoint{6.872640in}{1.785000in}}%
\pgfpathlineto{\pgfqpoint{6.873880in}{1.260000in}}%
\pgfpathlineto{\pgfqpoint{6.875120in}{1.505000in}}%
\pgfpathlineto{\pgfqpoint{6.876360in}{1.470000in}}%
\pgfpathlineto{\pgfqpoint{6.878840in}{1.470000in}}%
\pgfpathlineto{\pgfqpoint{6.880080in}{1.435000in}}%
\pgfpathlineto{\pgfqpoint{6.882560in}{1.120000in}}%
\pgfpathlineto{\pgfqpoint{6.883800in}{1.575000in}}%
\pgfpathlineto{\pgfqpoint{6.885040in}{1.225000in}}%
\pgfpathlineto{\pgfqpoint{6.886280in}{1.435000in}}%
\pgfpathlineto{\pgfqpoint{6.887520in}{1.400000in}}%
\pgfpathlineto{\pgfqpoint{6.888760in}{1.715000in}}%
\pgfpathlineto{\pgfqpoint{6.890000in}{1.295000in}}%
\pgfpathlineto{\pgfqpoint{6.891240in}{1.470000in}}%
\pgfpathlineto{\pgfqpoint{6.894960in}{0.770000in}}%
\pgfpathlineto{\pgfqpoint{6.896200in}{1.540000in}}%
\pgfpathlineto{\pgfqpoint{6.898680in}{1.435000in}}%
\pgfpathlineto{\pgfqpoint{6.899920in}{0.980000in}}%
\pgfpathlineto{\pgfqpoint{6.903640in}{1.575000in}}%
\pgfpathlineto{\pgfqpoint{6.904880in}{1.190000in}}%
\pgfpathlineto{\pgfqpoint{6.907360in}{1.995000in}}%
\pgfpathlineto{\pgfqpoint{6.908600in}{1.435000in}}%
\pgfpathlineto{\pgfqpoint{6.909840in}{1.540000in}}%
\pgfpathlineto{\pgfqpoint{6.911080in}{1.155000in}}%
\pgfpathlineto{\pgfqpoint{6.912320in}{1.680000in}}%
\pgfpathlineto{\pgfqpoint{6.913560in}{1.540000in}}%
\pgfpathlineto{\pgfqpoint{6.914800in}{1.890000in}}%
\pgfpathlineto{\pgfqpoint{6.917280in}{1.505000in}}%
\pgfpathlineto{\pgfqpoint{6.918520in}{1.470000in}}%
\pgfpathlineto{\pgfqpoint{6.919760in}{1.610000in}}%
\pgfpathlineto{\pgfqpoint{6.923480in}{1.225000in}}%
\pgfpathlineto{\pgfqpoint{6.924720in}{1.365000in}}%
\pgfpathlineto{\pgfqpoint{6.925960in}{1.365000in}}%
\pgfpathlineto{\pgfqpoint{6.928440in}{1.610000in}}%
\pgfpathlineto{\pgfqpoint{6.929680in}{1.155000in}}%
\pgfpathlineto{\pgfqpoint{6.930920in}{1.680000in}}%
\pgfpathlineto{\pgfqpoint{6.932160in}{1.645000in}}%
\pgfpathlineto{\pgfqpoint{6.933400in}{1.435000in}}%
\pgfpathlineto{\pgfqpoint{6.934640in}{1.470000in}}%
\pgfpathlineto{\pgfqpoint{6.935880in}{1.435000in}}%
\pgfpathlineto{\pgfqpoint{6.937120in}{1.470000in}}%
\pgfpathlineto{\pgfqpoint{6.939600in}{1.890000in}}%
\pgfpathlineto{\pgfqpoint{6.940840in}{1.365000in}}%
\pgfpathlineto{\pgfqpoint{6.942080in}{1.400000in}}%
\pgfpathlineto{\pgfqpoint{6.943320in}{1.330000in}}%
\pgfpathlineto{\pgfqpoint{6.944560in}{1.155000in}}%
\pgfpathlineto{\pgfqpoint{6.945800in}{1.155000in}}%
\pgfpathlineto{\pgfqpoint{6.948280in}{1.925000in}}%
\pgfpathlineto{\pgfqpoint{6.949520in}{1.575000in}}%
\pgfpathlineto{\pgfqpoint{6.952000in}{1.750000in}}%
\pgfpathlineto{\pgfqpoint{6.953240in}{1.645000in}}%
\pgfpathlineto{\pgfqpoint{6.954480in}{1.855000in}}%
\pgfpathlineto{\pgfqpoint{6.955720in}{1.715000in}}%
\pgfpathlineto{\pgfqpoint{6.956960in}{1.435000in}}%
\pgfpathlineto{\pgfqpoint{6.958200in}{1.540000in}}%
\pgfpathlineto{\pgfqpoint{6.959440in}{1.400000in}}%
\pgfpathlineto{\pgfqpoint{6.960680in}{1.470000in}}%
\pgfpathlineto{\pgfqpoint{6.961920in}{1.225000in}}%
\pgfpathlineto{\pgfqpoint{6.963160in}{1.575000in}}%
\pgfpathlineto{\pgfqpoint{6.964400in}{1.470000in}}%
\pgfpathlineto{\pgfqpoint{6.965640in}{1.855000in}}%
\pgfpathlineto{\pgfqpoint{6.969360in}{0.805000in}}%
\pgfpathlineto{\pgfqpoint{6.970600in}{0.980000in}}%
\pgfpathlineto{\pgfqpoint{6.971840in}{1.505000in}}%
\pgfpathlineto{\pgfqpoint{6.974320in}{1.365000in}}%
\pgfpathlineto{\pgfqpoint{6.975560in}{1.540000in}}%
\pgfpathlineto{\pgfqpoint{6.976800in}{1.190000in}}%
\pgfpathlineto{\pgfqpoint{6.978040in}{1.155000in}}%
\pgfpathlineto{\pgfqpoint{6.979280in}{1.295000in}}%
\pgfpathlineto{\pgfqpoint{6.980520in}{2.065000in}}%
\pgfpathlineto{\pgfqpoint{6.983000in}{1.400000in}}%
\pgfpathlineto{\pgfqpoint{6.984240in}{1.715000in}}%
\pgfpathlineto{\pgfqpoint{6.986720in}{1.190000in}}%
\pgfpathlineto{\pgfqpoint{6.987960in}{1.435000in}}%
\pgfpathlineto{\pgfqpoint{6.989200in}{1.225000in}}%
\pgfpathlineto{\pgfqpoint{6.990440in}{1.750000in}}%
\pgfpathlineto{\pgfqpoint{6.994160in}{1.540000in}}%
\pgfpathlineto{\pgfqpoint{6.995400in}{1.365000in}}%
\pgfpathlineto{\pgfqpoint{6.996640in}{1.890000in}}%
\pgfpathlineto{\pgfqpoint{6.997880in}{1.540000in}}%
\pgfpathlineto{\pgfqpoint{6.999120in}{1.855000in}}%
\pgfpathlineto{\pgfqpoint{7.001600in}{1.505000in}}%
\pgfpathlineto{\pgfqpoint{7.002840in}{1.610000in}}%
\pgfpathlineto{\pgfqpoint{7.004080in}{1.925000in}}%
\pgfpathlineto{\pgfqpoint{7.005320in}{1.575000in}}%
\pgfpathlineto{\pgfqpoint{7.006560in}{2.030000in}}%
\pgfpathlineto{\pgfqpoint{7.009040in}{1.575000in}}%
\pgfpathlineto{\pgfqpoint{7.010280in}{1.820000in}}%
\pgfpathlineto{\pgfqpoint{7.011520in}{1.435000in}}%
\pgfpathlineto{\pgfqpoint{7.012760in}{1.680000in}}%
\pgfpathlineto{\pgfqpoint{7.016480in}{1.225000in}}%
\pgfpathlineto{\pgfqpoint{7.017720in}{1.330000in}}%
\pgfpathlineto{\pgfqpoint{7.018960in}{1.120000in}}%
\pgfpathlineto{\pgfqpoint{7.020200in}{1.365000in}}%
\pgfpathlineto{\pgfqpoint{7.021440in}{1.225000in}}%
\pgfpathlineto{\pgfqpoint{7.022680in}{1.295000in}}%
\pgfpathlineto{\pgfqpoint{7.025160in}{1.505000in}}%
\pgfpathlineto{\pgfqpoint{7.026400in}{1.610000in}}%
\pgfpathlineto{\pgfqpoint{7.028880in}{1.295000in}}%
\pgfpathlineto{\pgfqpoint{7.030120in}{1.435000in}}%
\pgfpathlineto{\pgfqpoint{7.031360in}{1.330000in}}%
\pgfpathlineto{\pgfqpoint{7.032600in}{1.400000in}}%
\pgfpathlineto{\pgfqpoint{7.033840in}{1.260000in}}%
\pgfpathlineto{\pgfqpoint{7.035080in}{1.260000in}}%
\pgfpathlineto{\pgfqpoint{7.036320in}{1.925000in}}%
\pgfpathlineto{\pgfqpoint{7.037560in}{1.155000in}}%
\pgfpathlineto{\pgfqpoint{7.038800in}{1.820000in}}%
\pgfpathlineto{\pgfqpoint{7.040040in}{1.575000in}}%
\pgfpathlineto{\pgfqpoint{7.041280in}{1.890000in}}%
\pgfpathlineto{\pgfqpoint{7.045000in}{1.295000in}}%
\pgfpathlineto{\pgfqpoint{7.048720in}{1.680000in}}%
\pgfpathlineto{\pgfqpoint{7.049960in}{1.610000in}}%
\pgfpathlineto{\pgfqpoint{7.051200in}{1.435000in}}%
\pgfpathlineto{\pgfqpoint{7.052440in}{1.610000in}}%
\pgfpathlineto{\pgfqpoint{7.053680in}{1.470000in}}%
\pgfpathlineto{\pgfqpoint{7.054920in}{1.785000in}}%
\pgfpathlineto{\pgfqpoint{7.056160in}{1.295000in}}%
\pgfpathlineto{\pgfqpoint{7.058640in}{1.715000in}}%
\pgfpathlineto{\pgfqpoint{7.059880in}{1.925000in}}%
\pgfpathlineto{\pgfqpoint{7.062360in}{1.365000in}}%
\pgfpathlineto{\pgfqpoint{7.064840in}{1.680000in}}%
\pgfpathlineto{\pgfqpoint{7.066080in}{1.575000in}}%
\pgfpathlineto{\pgfqpoint{7.067320in}{1.330000in}}%
\pgfpathlineto{\pgfqpoint{7.068560in}{1.470000in}}%
\pgfpathlineto{\pgfqpoint{7.069800in}{1.225000in}}%
\pgfpathlineto{\pgfqpoint{7.072280in}{1.505000in}}%
\pgfpathlineto{\pgfqpoint{7.073520in}{2.065000in}}%
\pgfpathlineto{\pgfqpoint{7.076000in}{1.505000in}}%
\pgfpathlineto{\pgfqpoint{7.077240in}{1.505000in}}%
\pgfpathlineto{\pgfqpoint{7.078480in}{1.470000in}}%
\pgfpathlineto{\pgfqpoint{7.079720in}{1.680000in}}%
\pgfpathlineto{\pgfqpoint{7.080960in}{1.260000in}}%
\pgfpathlineto{\pgfqpoint{7.083440in}{1.715000in}}%
\pgfpathlineto{\pgfqpoint{7.085920in}{1.610000in}}%
\pgfpathlineto{\pgfqpoint{7.087160in}{1.260000in}}%
\pgfpathlineto{\pgfqpoint{7.088400in}{1.715000in}}%
\pgfpathlineto{\pgfqpoint{7.090880in}{1.260000in}}%
\pgfpathlineto{\pgfqpoint{7.092120in}{1.610000in}}%
\pgfpathlineto{\pgfqpoint{7.093360in}{1.295000in}}%
\pgfpathlineto{\pgfqpoint{7.094600in}{1.400000in}}%
\pgfpathlineto{\pgfqpoint{7.095840in}{1.680000in}}%
\pgfpathlineto{\pgfqpoint{7.097080in}{1.645000in}}%
\pgfpathlineto{\pgfqpoint{7.098320in}{1.260000in}}%
\pgfpathlineto{\pgfqpoint{7.099560in}{1.470000in}}%
\pgfpathlineto{\pgfqpoint{7.100800in}{1.365000in}}%
\pgfpathlineto{\pgfqpoint{7.102040in}{1.540000in}}%
\pgfpathlineto{\pgfqpoint{7.104520in}{0.980000in}}%
\pgfpathlineto{\pgfqpoint{7.107000in}{1.820000in}}%
\pgfpathlineto{\pgfqpoint{7.108240in}{1.120000in}}%
\pgfpathlineto{\pgfqpoint{7.109480in}{1.435000in}}%
\pgfpathlineto{\pgfqpoint{7.110720in}{1.050000in}}%
\pgfpathlineto{\pgfqpoint{7.111960in}{1.540000in}}%
\pgfpathlineto{\pgfqpoint{7.113200in}{1.330000in}}%
\pgfpathlineto{\pgfqpoint{7.114440in}{1.505000in}}%
\pgfpathlineto{\pgfqpoint{7.115680in}{1.470000in}}%
\pgfpathlineto{\pgfqpoint{7.116920in}{1.610000in}}%
\pgfpathlineto{\pgfqpoint{7.118160in}{1.575000in}}%
\pgfpathlineto{\pgfqpoint{7.119400in}{1.855000in}}%
\pgfpathlineto{\pgfqpoint{7.120640in}{1.715000in}}%
\pgfpathlineto{\pgfqpoint{7.123120in}{1.155000in}}%
\pgfpathlineto{\pgfqpoint{7.124360in}{2.030000in}}%
\pgfpathlineto{\pgfqpoint{7.126840in}{1.295000in}}%
\pgfpathlineto{\pgfqpoint{7.128080in}{1.890000in}}%
\pgfpathlineto{\pgfqpoint{7.130560in}{0.805000in}}%
\pgfpathlineto{\pgfqpoint{7.133040in}{1.330000in}}%
\pgfpathlineto{\pgfqpoint{7.134280in}{1.330000in}}%
\pgfpathlineto{\pgfqpoint{7.135520in}{1.295000in}}%
\pgfpathlineto{\pgfqpoint{7.136760in}{1.610000in}}%
\pgfpathlineto{\pgfqpoint{7.138000in}{1.610000in}}%
\pgfpathlineto{\pgfqpoint{7.139240in}{1.365000in}}%
\pgfpathlineto{\pgfqpoint{7.140480in}{1.750000in}}%
\pgfpathlineto{\pgfqpoint{7.141720in}{1.085000in}}%
\pgfpathlineto{\pgfqpoint{7.142960in}{1.225000in}}%
\pgfpathlineto{\pgfqpoint{7.145440in}{1.015000in}}%
\pgfpathlineto{\pgfqpoint{7.147920in}{1.715000in}}%
\pgfpathlineto{\pgfqpoint{7.149160in}{1.190000in}}%
\pgfpathlineto{\pgfqpoint{7.150400in}{1.225000in}}%
\pgfpathlineto{\pgfqpoint{7.151640in}{1.610000in}}%
\pgfpathlineto{\pgfqpoint{7.152880in}{1.365000in}}%
\pgfpathlineto{\pgfqpoint{7.154120in}{1.505000in}}%
\pgfpathlineto{\pgfqpoint{7.156600in}{1.505000in}}%
\pgfpathlineto{\pgfqpoint{7.157840in}{1.330000in}}%
\pgfpathlineto{\pgfqpoint{7.159080in}{1.400000in}}%
\pgfpathlineto{\pgfqpoint{7.161560in}{1.610000in}}%
\pgfpathlineto{\pgfqpoint{7.162800in}{1.330000in}}%
\pgfpathlineto{\pgfqpoint{7.164040in}{1.470000in}}%
\pgfpathlineto{\pgfqpoint{7.165280in}{0.945000in}}%
\pgfpathlineto{\pgfqpoint{7.166520in}{1.015000in}}%
\pgfpathlineto{\pgfqpoint{7.167760in}{0.805000in}}%
\pgfpathlineto{\pgfqpoint{7.169000in}{1.295000in}}%
\pgfpathlineto{\pgfqpoint{7.170240in}{1.225000in}}%
\pgfpathlineto{\pgfqpoint{7.171480in}{1.085000in}}%
\pgfpathlineto{\pgfqpoint{7.173960in}{1.295000in}}%
\pgfpathlineto{\pgfqpoint{7.175200in}{1.085000in}}%
\pgfpathlineto{\pgfqpoint{7.176440in}{1.155000in}}%
\pgfpathlineto{\pgfqpoint{7.178920in}{1.610000in}}%
\pgfpathlineto{\pgfqpoint{7.180160in}{1.645000in}}%
\pgfpathlineto{\pgfqpoint{7.182640in}{1.400000in}}%
\pgfpathlineto{\pgfqpoint{7.186360in}{1.645000in}}%
\pgfpathlineto{\pgfqpoint{7.187600in}{1.435000in}}%
\pgfpathlineto{\pgfqpoint{7.188840in}{1.470000in}}%
\pgfpathlineto{\pgfqpoint{7.191320in}{1.470000in}}%
\pgfpathlineto{\pgfqpoint{7.193800in}{1.645000in}}%
\pgfpathlineto{\pgfqpoint{7.198760in}{0.945000in}}%
\pgfpathlineto{\pgfqpoint{7.200000in}{1.120000in}}%
\pgfpathlineto{\pgfqpoint{7.200000in}{1.120000in}}%
\pgfusepath{stroke}%
\end{pgfscope}%
\begin{pgfscope}%
\pgfpathrectangle{\pgfqpoint{1.000000in}{0.350000in}}{\pgfqpoint{6.200000in}{2.800000in}} %
\pgfusepath{clip}%
\pgfsetbuttcap%
\pgfsetroundjoin%
\pgfsetlinewidth{0.501875pt}%
\definecolor{currentstroke}{rgb}{0.000000,0.000000,0.000000}%
\pgfsetstrokecolor{currentstroke}%
\pgfsetdash{{1.000000pt}{3.000000pt}}{0.000000pt}%
\pgfpathmoveto{\pgfqpoint{1.000000in}{0.350000in}}%
\pgfpathlineto{\pgfqpoint{1.000000in}{3.150000in}}%
\pgfusepath{stroke}%
\end{pgfscope}%
\begin{pgfscope}%
\pgfsetbuttcap%
\pgfsetroundjoin%
\definecolor{currentfill}{rgb}{0.000000,0.000000,0.000000}%
\pgfsetfillcolor{currentfill}%
\pgfsetlinewidth{0.501875pt}%
\definecolor{currentstroke}{rgb}{0.000000,0.000000,0.000000}%
\pgfsetstrokecolor{currentstroke}%
\pgfsetdash{}{0pt}%
\pgfsys@defobject{currentmarker}{\pgfqpoint{0.000000in}{0.000000in}}{\pgfqpoint{0.000000in}{0.055556in}}{%
\pgfpathmoveto{\pgfqpoint{0.000000in}{0.000000in}}%
\pgfpathlineto{\pgfqpoint{0.000000in}{0.055556in}}%
\pgfusepath{stroke,fill}%
}%
\begin{pgfscope}%
\pgfsys@transformshift{1.000000in}{0.350000in}%
\pgfsys@useobject{currentmarker}{}%
\end{pgfscope}%
\end{pgfscope}%
\begin{pgfscope}%
\pgfsetbuttcap%
\pgfsetroundjoin%
\definecolor{currentfill}{rgb}{0.000000,0.000000,0.000000}%
\pgfsetfillcolor{currentfill}%
\pgfsetlinewidth{0.501875pt}%
\definecolor{currentstroke}{rgb}{0.000000,0.000000,0.000000}%
\pgfsetstrokecolor{currentstroke}%
\pgfsetdash{}{0pt}%
\pgfsys@defobject{currentmarker}{\pgfqpoint{0.000000in}{-0.055556in}}{\pgfqpoint{0.000000in}{0.000000in}}{%
\pgfpathmoveto{\pgfqpoint{0.000000in}{0.000000in}}%
\pgfpathlineto{\pgfqpoint{0.000000in}{-0.055556in}}%
\pgfusepath{stroke,fill}%
}%
\begin{pgfscope}%
\pgfsys@transformshift{1.000000in}{3.150000in}%
\pgfsys@useobject{currentmarker}{}%
\end{pgfscope}%
\end{pgfscope}%
\begin{pgfscope}%
\pgftext[left,bottom,x=0.946981in,y=0.168387in,rotate=0.000000]{{\sffamily\fontsize{12.000000}{14.400000}\selectfont 0}}
%
\end{pgfscope}%
\begin{pgfscope}%
\pgfpathrectangle{\pgfqpoint{1.000000in}{0.350000in}}{\pgfqpoint{6.200000in}{2.800000in}} %
\pgfusepath{clip}%
\pgfsetbuttcap%
\pgfsetroundjoin%
\pgfsetlinewidth{0.501875pt}%
\definecolor{currentstroke}{rgb}{0.000000,0.000000,0.000000}%
\pgfsetstrokecolor{currentstroke}%
\pgfsetdash{{1.000000pt}{3.000000pt}}{0.000000pt}%
\pgfpathmoveto{\pgfqpoint{2.240000in}{0.350000in}}%
\pgfpathlineto{\pgfqpoint{2.240000in}{3.150000in}}%
\pgfusepath{stroke}%
\end{pgfscope}%
\begin{pgfscope}%
\pgfsetbuttcap%
\pgfsetroundjoin%
\definecolor{currentfill}{rgb}{0.000000,0.000000,0.000000}%
\pgfsetfillcolor{currentfill}%
\pgfsetlinewidth{0.501875pt}%
\definecolor{currentstroke}{rgb}{0.000000,0.000000,0.000000}%
\pgfsetstrokecolor{currentstroke}%
\pgfsetdash{}{0pt}%
\pgfsys@defobject{currentmarker}{\pgfqpoint{0.000000in}{0.000000in}}{\pgfqpoint{0.000000in}{0.055556in}}{%
\pgfpathmoveto{\pgfqpoint{0.000000in}{0.000000in}}%
\pgfpathlineto{\pgfqpoint{0.000000in}{0.055556in}}%
\pgfusepath{stroke,fill}%
}%
\begin{pgfscope}%
\pgfsys@transformshift{2.240000in}{0.350000in}%
\pgfsys@useobject{currentmarker}{}%
\end{pgfscope}%
\end{pgfscope}%
\begin{pgfscope}%
\pgfsetbuttcap%
\pgfsetroundjoin%
\definecolor{currentfill}{rgb}{0.000000,0.000000,0.000000}%
\pgfsetfillcolor{currentfill}%
\pgfsetlinewidth{0.501875pt}%
\definecolor{currentstroke}{rgb}{0.000000,0.000000,0.000000}%
\pgfsetstrokecolor{currentstroke}%
\pgfsetdash{}{0pt}%
\pgfsys@defobject{currentmarker}{\pgfqpoint{0.000000in}{-0.055556in}}{\pgfqpoint{0.000000in}{0.000000in}}{%
\pgfpathmoveto{\pgfqpoint{0.000000in}{0.000000in}}%
\pgfpathlineto{\pgfqpoint{0.000000in}{-0.055556in}}%
\pgfusepath{stroke,fill}%
}%
\begin{pgfscope}%
\pgfsys@transformshift{2.240000in}{3.150000in}%
\pgfsys@useobject{currentmarker}{}%
\end{pgfscope}%
\end{pgfscope}%
\begin{pgfscope}%
\pgftext[left,bottom,x=2.080942in,y=0.168387in,rotate=0.000000]{{\sffamily\fontsize{12.000000}{14.400000}\selectfont 100}}
%
\end{pgfscope}%
\begin{pgfscope}%
\pgfpathrectangle{\pgfqpoint{1.000000in}{0.350000in}}{\pgfqpoint{6.200000in}{2.800000in}} %
\pgfusepath{clip}%
\pgfsetbuttcap%
\pgfsetroundjoin%
\pgfsetlinewidth{0.501875pt}%
\definecolor{currentstroke}{rgb}{0.000000,0.000000,0.000000}%
\pgfsetstrokecolor{currentstroke}%
\pgfsetdash{{1.000000pt}{3.000000pt}}{0.000000pt}%
\pgfpathmoveto{\pgfqpoint{3.480000in}{0.350000in}}%
\pgfpathlineto{\pgfqpoint{3.480000in}{3.150000in}}%
\pgfusepath{stroke}%
\end{pgfscope}%
\begin{pgfscope}%
\pgfsetbuttcap%
\pgfsetroundjoin%
\definecolor{currentfill}{rgb}{0.000000,0.000000,0.000000}%
\pgfsetfillcolor{currentfill}%
\pgfsetlinewidth{0.501875pt}%
\definecolor{currentstroke}{rgb}{0.000000,0.000000,0.000000}%
\pgfsetstrokecolor{currentstroke}%
\pgfsetdash{}{0pt}%
\pgfsys@defobject{currentmarker}{\pgfqpoint{0.000000in}{0.000000in}}{\pgfqpoint{0.000000in}{0.055556in}}{%
\pgfpathmoveto{\pgfqpoint{0.000000in}{0.000000in}}%
\pgfpathlineto{\pgfqpoint{0.000000in}{0.055556in}}%
\pgfusepath{stroke,fill}%
}%
\begin{pgfscope}%
\pgfsys@transformshift{3.480000in}{0.350000in}%
\pgfsys@useobject{currentmarker}{}%
\end{pgfscope}%
\end{pgfscope}%
\begin{pgfscope}%
\pgfsetbuttcap%
\pgfsetroundjoin%
\definecolor{currentfill}{rgb}{0.000000,0.000000,0.000000}%
\pgfsetfillcolor{currentfill}%
\pgfsetlinewidth{0.501875pt}%
\definecolor{currentstroke}{rgb}{0.000000,0.000000,0.000000}%
\pgfsetstrokecolor{currentstroke}%
\pgfsetdash{}{0pt}%
\pgfsys@defobject{currentmarker}{\pgfqpoint{0.000000in}{-0.055556in}}{\pgfqpoint{0.000000in}{0.000000in}}{%
\pgfpathmoveto{\pgfqpoint{0.000000in}{0.000000in}}%
\pgfpathlineto{\pgfqpoint{0.000000in}{-0.055556in}}%
\pgfusepath{stroke,fill}%
}%
\begin{pgfscope}%
\pgfsys@transformshift{3.480000in}{3.150000in}%
\pgfsys@useobject{currentmarker}{}%
\end{pgfscope}%
\end{pgfscope}%
\begin{pgfscope}%
\pgftext[left,bottom,x=3.320942in,y=0.168387in,rotate=0.000000]{{\sffamily\fontsize{12.000000}{14.400000}\selectfont 200}}
%
\end{pgfscope}%
\begin{pgfscope}%
\pgfpathrectangle{\pgfqpoint{1.000000in}{0.350000in}}{\pgfqpoint{6.200000in}{2.800000in}} %
\pgfusepath{clip}%
\pgfsetbuttcap%
\pgfsetroundjoin%
\pgfsetlinewidth{0.501875pt}%
\definecolor{currentstroke}{rgb}{0.000000,0.000000,0.000000}%
\pgfsetstrokecolor{currentstroke}%
\pgfsetdash{{1.000000pt}{3.000000pt}}{0.000000pt}%
\pgfpathmoveto{\pgfqpoint{4.720000in}{0.350000in}}%
\pgfpathlineto{\pgfqpoint{4.720000in}{3.150000in}}%
\pgfusepath{stroke}%
\end{pgfscope}%
\begin{pgfscope}%
\pgfsetbuttcap%
\pgfsetroundjoin%
\definecolor{currentfill}{rgb}{0.000000,0.000000,0.000000}%
\pgfsetfillcolor{currentfill}%
\pgfsetlinewidth{0.501875pt}%
\definecolor{currentstroke}{rgb}{0.000000,0.000000,0.000000}%
\pgfsetstrokecolor{currentstroke}%
\pgfsetdash{}{0pt}%
\pgfsys@defobject{currentmarker}{\pgfqpoint{0.000000in}{0.000000in}}{\pgfqpoint{0.000000in}{0.055556in}}{%
\pgfpathmoveto{\pgfqpoint{0.000000in}{0.000000in}}%
\pgfpathlineto{\pgfqpoint{0.000000in}{0.055556in}}%
\pgfusepath{stroke,fill}%
}%
\begin{pgfscope}%
\pgfsys@transformshift{4.720000in}{0.350000in}%
\pgfsys@useobject{currentmarker}{}%
\end{pgfscope}%
\end{pgfscope}%
\begin{pgfscope}%
\pgfsetbuttcap%
\pgfsetroundjoin%
\definecolor{currentfill}{rgb}{0.000000,0.000000,0.000000}%
\pgfsetfillcolor{currentfill}%
\pgfsetlinewidth{0.501875pt}%
\definecolor{currentstroke}{rgb}{0.000000,0.000000,0.000000}%
\pgfsetstrokecolor{currentstroke}%
\pgfsetdash{}{0pt}%
\pgfsys@defobject{currentmarker}{\pgfqpoint{0.000000in}{-0.055556in}}{\pgfqpoint{0.000000in}{0.000000in}}{%
\pgfpathmoveto{\pgfqpoint{0.000000in}{0.000000in}}%
\pgfpathlineto{\pgfqpoint{0.000000in}{-0.055556in}}%
\pgfusepath{stroke,fill}%
}%
\begin{pgfscope}%
\pgfsys@transformshift{4.720000in}{3.150000in}%
\pgfsys@useobject{currentmarker}{}%
\end{pgfscope}%
\end{pgfscope}%
\begin{pgfscope}%
\pgftext[left,bottom,x=4.560942in,y=0.168387in,rotate=0.000000]{{\sffamily\fontsize{12.000000}{14.400000}\selectfont 300}}
%
\end{pgfscope}%
\begin{pgfscope}%
\pgfpathrectangle{\pgfqpoint{1.000000in}{0.350000in}}{\pgfqpoint{6.200000in}{2.800000in}} %
\pgfusepath{clip}%
\pgfsetbuttcap%
\pgfsetroundjoin%
\pgfsetlinewidth{0.501875pt}%
\definecolor{currentstroke}{rgb}{0.000000,0.000000,0.000000}%
\pgfsetstrokecolor{currentstroke}%
\pgfsetdash{{1.000000pt}{3.000000pt}}{0.000000pt}%
\pgfpathmoveto{\pgfqpoint{5.960000in}{0.350000in}}%
\pgfpathlineto{\pgfqpoint{5.960000in}{3.150000in}}%
\pgfusepath{stroke}%
\end{pgfscope}%
\begin{pgfscope}%
\pgfsetbuttcap%
\pgfsetroundjoin%
\definecolor{currentfill}{rgb}{0.000000,0.000000,0.000000}%
\pgfsetfillcolor{currentfill}%
\pgfsetlinewidth{0.501875pt}%
\definecolor{currentstroke}{rgb}{0.000000,0.000000,0.000000}%
\pgfsetstrokecolor{currentstroke}%
\pgfsetdash{}{0pt}%
\pgfsys@defobject{currentmarker}{\pgfqpoint{0.000000in}{0.000000in}}{\pgfqpoint{0.000000in}{0.055556in}}{%
\pgfpathmoveto{\pgfqpoint{0.000000in}{0.000000in}}%
\pgfpathlineto{\pgfqpoint{0.000000in}{0.055556in}}%
\pgfusepath{stroke,fill}%
}%
\begin{pgfscope}%
\pgfsys@transformshift{5.960000in}{0.350000in}%
\pgfsys@useobject{currentmarker}{}%
\end{pgfscope}%
\end{pgfscope}%
\begin{pgfscope}%
\pgfsetbuttcap%
\pgfsetroundjoin%
\definecolor{currentfill}{rgb}{0.000000,0.000000,0.000000}%
\pgfsetfillcolor{currentfill}%
\pgfsetlinewidth{0.501875pt}%
\definecolor{currentstroke}{rgb}{0.000000,0.000000,0.000000}%
\pgfsetstrokecolor{currentstroke}%
\pgfsetdash{}{0pt}%
\pgfsys@defobject{currentmarker}{\pgfqpoint{0.000000in}{-0.055556in}}{\pgfqpoint{0.000000in}{0.000000in}}{%
\pgfpathmoveto{\pgfqpoint{0.000000in}{0.000000in}}%
\pgfpathlineto{\pgfqpoint{0.000000in}{-0.055556in}}%
\pgfusepath{stroke,fill}%
}%
\begin{pgfscope}%
\pgfsys@transformshift{5.960000in}{3.150000in}%
\pgfsys@useobject{currentmarker}{}%
\end{pgfscope}%
\end{pgfscope}%
\begin{pgfscope}%
\pgftext[left,bottom,x=5.800942in,y=0.168387in,rotate=0.000000]{{\sffamily\fontsize{12.000000}{14.400000}\selectfont 400}}
%
\end{pgfscope}%
\begin{pgfscope}%
\pgfpathrectangle{\pgfqpoint{1.000000in}{0.350000in}}{\pgfqpoint{6.200000in}{2.800000in}} %
\pgfusepath{clip}%
\pgfsetbuttcap%
\pgfsetroundjoin%
\pgfsetlinewidth{0.501875pt}%
\definecolor{currentstroke}{rgb}{0.000000,0.000000,0.000000}%
\pgfsetstrokecolor{currentstroke}%
\pgfsetdash{{1.000000pt}{3.000000pt}}{0.000000pt}%
\pgfpathmoveto{\pgfqpoint{7.200000in}{0.350000in}}%
\pgfpathlineto{\pgfqpoint{7.200000in}{3.150000in}}%
\pgfusepath{stroke}%
\end{pgfscope}%
\begin{pgfscope}%
\pgfsetbuttcap%
\pgfsetroundjoin%
\definecolor{currentfill}{rgb}{0.000000,0.000000,0.000000}%
\pgfsetfillcolor{currentfill}%
\pgfsetlinewidth{0.501875pt}%
\definecolor{currentstroke}{rgb}{0.000000,0.000000,0.000000}%
\pgfsetstrokecolor{currentstroke}%
\pgfsetdash{}{0pt}%
\pgfsys@defobject{currentmarker}{\pgfqpoint{0.000000in}{0.000000in}}{\pgfqpoint{0.000000in}{0.055556in}}{%
\pgfpathmoveto{\pgfqpoint{0.000000in}{0.000000in}}%
\pgfpathlineto{\pgfqpoint{0.000000in}{0.055556in}}%
\pgfusepath{stroke,fill}%
}%
\begin{pgfscope}%
\pgfsys@transformshift{7.200000in}{0.350000in}%
\pgfsys@useobject{currentmarker}{}%
\end{pgfscope}%
\end{pgfscope}%
\begin{pgfscope}%
\pgfsetbuttcap%
\pgfsetroundjoin%
\definecolor{currentfill}{rgb}{0.000000,0.000000,0.000000}%
\pgfsetfillcolor{currentfill}%
\pgfsetlinewidth{0.501875pt}%
\definecolor{currentstroke}{rgb}{0.000000,0.000000,0.000000}%
\pgfsetstrokecolor{currentstroke}%
\pgfsetdash{}{0pt}%
\pgfsys@defobject{currentmarker}{\pgfqpoint{0.000000in}{-0.055556in}}{\pgfqpoint{0.000000in}{0.000000in}}{%
\pgfpathmoveto{\pgfqpoint{0.000000in}{0.000000in}}%
\pgfpathlineto{\pgfqpoint{0.000000in}{-0.055556in}}%
\pgfusepath{stroke,fill}%
}%
\begin{pgfscope}%
\pgfsys@transformshift{7.200000in}{3.150000in}%
\pgfsys@useobject{currentmarker}{}%
\end{pgfscope}%
\end{pgfscope}%
\begin{pgfscope}%
\pgftext[left,bottom,x=7.040942in,y=0.168387in,rotate=0.000000]{{\sffamily\fontsize{12.000000}{14.400000}\selectfont 500}}
%
\end{pgfscope}%
\begin{pgfscope}%
\pgftext[left,bottom,x=3.911727in,y=-0.030045in,rotate=0.000000]{{\sffamily\fontsize{12.000000}{14.400000}\selectfont time}}
%
\end{pgfscope}%
\begin{pgfscope}%
\pgfpathrectangle{\pgfqpoint{1.000000in}{0.350000in}}{\pgfqpoint{6.200000in}{2.800000in}} %
\pgfusepath{clip}%
\pgfsetbuttcap%
\pgfsetroundjoin%
\pgfsetlinewidth{0.501875pt}%
\definecolor{currentstroke}{rgb}{0.000000,0.000000,0.000000}%
\pgfsetstrokecolor{currentstroke}%
\pgfsetdash{{1.000000pt}{3.000000pt}}{0.000000pt}%
\pgfpathmoveto{\pgfqpoint{1.000000in}{0.350000in}}%
\pgfpathlineto{\pgfqpoint{7.200000in}{0.350000in}}%
\pgfusepath{stroke}%
\end{pgfscope}%
\begin{pgfscope}%
\pgfsetbuttcap%
\pgfsetroundjoin%
\definecolor{currentfill}{rgb}{0.000000,0.000000,0.000000}%
\pgfsetfillcolor{currentfill}%
\pgfsetlinewidth{0.501875pt}%
\definecolor{currentstroke}{rgb}{0.000000,0.000000,0.000000}%
\pgfsetstrokecolor{currentstroke}%
\pgfsetdash{}{0pt}%
\pgfsys@defobject{currentmarker}{\pgfqpoint{0.000000in}{0.000000in}}{\pgfqpoint{0.055556in}{0.000000in}}{%
\pgfpathmoveto{\pgfqpoint{0.000000in}{0.000000in}}%
\pgfpathlineto{\pgfqpoint{0.055556in}{0.000000in}}%
\pgfusepath{stroke,fill}%
}%
\begin{pgfscope}%
\pgfsys@transformshift{1.000000in}{0.350000in}%
\pgfsys@useobject{currentmarker}{}%
\end{pgfscope}%
\end{pgfscope}%
\begin{pgfscope}%
\pgfsetbuttcap%
\pgfsetroundjoin%
\definecolor{currentfill}{rgb}{0.000000,0.000000,0.000000}%
\pgfsetfillcolor{currentfill}%
\pgfsetlinewidth{0.501875pt}%
\definecolor{currentstroke}{rgb}{0.000000,0.000000,0.000000}%
\pgfsetstrokecolor{currentstroke}%
\pgfsetdash{}{0pt}%
\pgfsys@defobject{currentmarker}{\pgfqpoint{-0.055556in}{0.000000in}}{\pgfqpoint{0.000000in}{0.000000in}}{%
\pgfpathmoveto{\pgfqpoint{0.000000in}{0.000000in}}%
\pgfpathlineto{\pgfqpoint{-0.055556in}{0.000000in}}%
\pgfusepath{stroke,fill}%
}%
\begin{pgfscope}%
\pgfsys@transformshift{7.200000in}{0.350000in}%
\pgfsys@useobject{currentmarker}{}%
\end{pgfscope}%
\end{pgfscope}%
\begin{pgfscope}%
\pgftext[left,bottom,x=0.626329in,y=0.286971in,rotate=0.000000]{{\sffamily\fontsize{12.000000}{14.400000}\selectfont 330}}
%
\end{pgfscope}%
\begin{pgfscope}%
\pgfpathrectangle{\pgfqpoint{1.000000in}{0.350000in}}{\pgfqpoint{6.200000in}{2.800000in}} %
\pgfusepath{clip}%
\pgfsetbuttcap%
\pgfsetroundjoin%
\pgfsetlinewidth{0.501875pt}%
\definecolor{currentstroke}{rgb}{0.000000,0.000000,0.000000}%
\pgfsetstrokecolor{currentstroke}%
\pgfsetdash{{1.000000pt}{3.000000pt}}{0.000000pt}%
\pgfpathmoveto{\pgfqpoint{1.000000in}{0.700000in}}%
\pgfpathlineto{\pgfqpoint{7.200000in}{0.700000in}}%
\pgfusepath{stroke}%
\end{pgfscope}%
\begin{pgfscope}%
\pgfsetbuttcap%
\pgfsetroundjoin%
\definecolor{currentfill}{rgb}{0.000000,0.000000,0.000000}%
\pgfsetfillcolor{currentfill}%
\pgfsetlinewidth{0.501875pt}%
\definecolor{currentstroke}{rgb}{0.000000,0.000000,0.000000}%
\pgfsetstrokecolor{currentstroke}%
\pgfsetdash{}{0pt}%
\pgfsys@defobject{currentmarker}{\pgfqpoint{0.000000in}{0.000000in}}{\pgfqpoint{0.055556in}{0.000000in}}{%
\pgfpathmoveto{\pgfqpoint{0.000000in}{0.000000in}}%
\pgfpathlineto{\pgfqpoint{0.055556in}{0.000000in}}%
\pgfusepath{stroke,fill}%
}%
\begin{pgfscope}%
\pgfsys@transformshift{1.000000in}{0.700000in}%
\pgfsys@useobject{currentmarker}{}%
\end{pgfscope}%
\end{pgfscope}%
\begin{pgfscope}%
\pgfsetbuttcap%
\pgfsetroundjoin%
\definecolor{currentfill}{rgb}{0.000000,0.000000,0.000000}%
\pgfsetfillcolor{currentfill}%
\pgfsetlinewidth{0.501875pt}%
\definecolor{currentstroke}{rgb}{0.000000,0.000000,0.000000}%
\pgfsetstrokecolor{currentstroke}%
\pgfsetdash{}{0pt}%
\pgfsys@defobject{currentmarker}{\pgfqpoint{-0.055556in}{0.000000in}}{\pgfqpoint{0.000000in}{0.000000in}}{%
\pgfpathmoveto{\pgfqpoint{0.000000in}{0.000000in}}%
\pgfpathlineto{\pgfqpoint{-0.055556in}{0.000000in}}%
\pgfusepath{stroke,fill}%
}%
\begin{pgfscope}%
\pgfsys@transformshift{7.200000in}{0.700000in}%
\pgfsys@useobject{currentmarker}{}%
\end{pgfscope}%
\end{pgfscope}%
\begin{pgfscope}%
\pgftext[left,bottom,x=0.626329in,y=0.636971in,rotate=0.000000]{{\sffamily\fontsize{12.000000}{14.400000}\selectfont 340}}
%
\end{pgfscope}%
\begin{pgfscope}%
\pgfpathrectangle{\pgfqpoint{1.000000in}{0.350000in}}{\pgfqpoint{6.200000in}{2.800000in}} %
\pgfusepath{clip}%
\pgfsetbuttcap%
\pgfsetroundjoin%
\pgfsetlinewidth{0.501875pt}%
\definecolor{currentstroke}{rgb}{0.000000,0.000000,0.000000}%
\pgfsetstrokecolor{currentstroke}%
\pgfsetdash{{1.000000pt}{3.000000pt}}{0.000000pt}%
\pgfpathmoveto{\pgfqpoint{1.000000in}{1.050000in}}%
\pgfpathlineto{\pgfqpoint{7.200000in}{1.050000in}}%
\pgfusepath{stroke}%
\end{pgfscope}%
\begin{pgfscope}%
\pgfsetbuttcap%
\pgfsetroundjoin%
\definecolor{currentfill}{rgb}{0.000000,0.000000,0.000000}%
\pgfsetfillcolor{currentfill}%
\pgfsetlinewidth{0.501875pt}%
\definecolor{currentstroke}{rgb}{0.000000,0.000000,0.000000}%
\pgfsetstrokecolor{currentstroke}%
\pgfsetdash{}{0pt}%
\pgfsys@defobject{currentmarker}{\pgfqpoint{0.000000in}{0.000000in}}{\pgfqpoint{0.055556in}{0.000000in}}{%
\pgfpathmoveto{\pgfqpoint{0.000000in}{0.000000in}}%
\pgfpathlineto{\pgfqpoint{0.055556in}{0.000000in}}%
\pgfusepath{stroke,fill}%
}%
\begin{pgfscope}%
\pgfsys@transformshift{1.000000in}{1.050000in}%
\pgfsys@useobject{currentmarker}{}%
\end{pgfscope}%
\end{pgfscope}%
\begin{pgfscope}%
\pgfsetbuttcap%
\pgfsetroundjoin%
\definecolor{currentfill}{rgb}{0.000000,0.000000,0.000000}%
\pgfsetfillcolor{currentfill}%
\pgfsetlinewidth{0.501875pt}%
\definecolor{currentstroke}{rgb}{0.000000,0.000000,0.000000}%
\pgfsetstrokecolor{currentstroke}%
\pgfsetdash{}{0pt}%
\pgfsys@defobject{currentmarker}{\pgfqpoint{-0.055556in}{0.000000in}}{\pgfqpoint{0.000000in}{0.000000in}}{%
\pgfpathmoveto{\pgfqpoint{0.000000in}{0.000000in}}%
\pgfpathlineto{\pgfqpoint{-0.055556in}{0.000000in}}%
\pgfusepath{stroke,fill}%
}%
\begin{pgfscope}%
\pgfsys@transformshift{7.200000in}{1.050000in}%
\pgfsys@useobject{currentmarker}{}%
\end{pgfscope}%
\end{pgfscope}%
\begin{pgfscope}%
\pgftext[left,bottom,x=0.626329in,y=0.986971in,rotate=0.000000]{{\sffamily\fontsize{12.000000}{14.400000}\selectfont 350}}
%
\end{pgfscope}%
\begin{pgfscope}%
\pgfpathrectangle{\pgfqpoint{1.000000in}{0.350000in}}{\pgfqpoint{6.200000in}{2.800000in}} %
\pgfusepath{clip}%
\pgfsetbuttcap%
\pgfsetroundjoin%
\pgfsetlinewidth{0.501875pt}%
\definecolor{currentstroke}{rgb}{0.000000,0.000000,0.000000}%
\pgfsetstrokecolor{currentstroke}%
\pgfsetdash{{1.000000pt}{3.000000pt}}{0.000000pt}%
\pgfpathmoveto{\pgfqpoint{1.000000in}{1.400000in}}%
\pgfpathlineto{\pgfqpoint{7.200000in}{1.400000in}}%
\pgfusepath{stroke}%
\end{pgfscope}%
\begin{pgfscope}%
\pgfsetbuttcap%
\pgfsetroundjoin%
\definecolor{currentfill}{rgb}{0.000000,0.000000,0.000000}%
\pgfsetfillcolor{currentfill}%
\pgfsetlinewidth{0.501875pt}%
\definecolor{currentstroke}{rgb}{0.000000,0.000000,0.000000}%
\pgfsetstrokecolor{currentstroke}%
\pgfsetdash{}{0pt}%
\pgfsys@defobject{currentmarker}{\pgfqpoint{0.000000in}{0.000000in}}{\pgfqpoint{0.055556in}{0.000000in}}{%
\pgfpathmoveto{\pgfqpoint{0.000000in}{0.000000in}}%
\pgfpathlineto{\pgfqpoint{0.055556in}{0.000000in}}%
\pgfusepath{stroke,fill}%
}%
\begin{pgfscope}%
\pgfsys@transformshift{1.000000in}{1.400000in}%
\pgfsys@useobject{currentmarker}{}%
\end{pgfscope}%
\end{pgfscope}%
\begin{pgfscope}%
\pgfsetbuttcap%
\pgfsetroundjoin%
\definecolor{currentfill}{rgb}{0.000000,0.000000,0.000000}%
\pgfsetfillcolor{currentfill}%
\pgfsetlinewidth{0.501875pt}%
\definecolor{currentstroke}{rgb}{0.000000,0.000000,0.000000}%
\pgfsetstrokecolor{currentstroke}%
\pgfsetdash{}{0pt}%
\pgfsys@defobject{currentmarker}{\pgfqpoint{-0.055556in}{0.000000in}}{\pgfqpoint{0.000000in}{0.000000in}}{%
\pgfpathmoveto{\pgfqpoint{0.000000in}{0.000000in}}%
\pgfpathlineto{\pgfqpoint{-0.055556in}{0.000000in}}%
\pgfusepath{stroke,fill}%
}%
\begin{pgfscope}%
\pgfsys@transformshift{7.200000in}{1.400000in}%
\pgfsys@useobject{currentmarker}{}%
\end{pgfscope}%
\end{pgfscope}%
\begin{pgfscope}%
\pgftext[left,bottom,x=0.626329in,y=1.336971in,rotate=0.000000]{{\sffamily\fontsize{12.000000}{14.400000}\selectfont 360}}
%
\end{pgfscope}%
\begin{pgfscope}%
\pgfpathrectangle{\pgfqpoint{1.000000in}{0.350000in}}{\pgfqpoint{6.200000in}{2.800000in}} %
\pgfusepath{clip}%
\pgfsetbuttcap%
\pgfsetroundjoin%
\pgfsetlinewidth{0.501875pt}%
\definecolor{currentstroke}{rgb}{0.000000,0.000000,0.000000}%
\pgfsetstrokecolor{currentstroke}%
\pgfsetdash{{1.000000pt}{3.000000pt}}{0.000000pt}%
\pgfpathmoveto{\pgfqpoint{1.000000in}{1.750000in}}%
\pgfpathlineto{\pgfqpoint{7.200000in}{1.750000in}}%
\pgfusepath{stroke}%
\end{pgfscope}%
\begin{pgfscope}%
\pgfsetbuttcap%
\pgfsetroundjoin%
\definecolor{currentfill}{rgb}{0.000000,0.000000,0.000000}%
\pgfsetfillcolor{currentfill}%
\pgfsetlinewidth{0.501875pt}%
\definecolor{currentstroke}{rgb}{0.000000,0.000000,0.000000}%
\pgfsetstrokecolor{currentstroke}%
\pgfsetdash{}{0pt}%
\pgfsys@defobject{currentmarker}{\pgfqpoint{0.000000in}{0.000000in}}{\pgfqpoint{0.055556in}{0.000000in}}{%
\pgfpathmoveto{\pgfqpoint{0.000000in}{0.000000in}}%
\pgfpathlineto{\pgfqpoint{0.055556in}{0.000000in}}%
\pgfusepath{stroke,fill}%
}%
\begin{pgfscope}%
\pgfsys@transformshift{1.000000in}{1.750000in}%
\pgfsys@useobject{currentmarker}{}%
\end{pgfscope}%
\end{pgfscope}%
\begin{pgfscope}%
\pgfsetbuttcap%
\pgfsetroundjoin%
\definecolor{currentfill}{rgb}{0.000000,0.000000,0.000000}%
\pgfsetfillcolor{currentfill}%
\pgfsetlinewidth{0.501875pt}%
\definecolor{currentstroke}{rgb}{0.000000,0.000000,0.000000}%
\pgfsetstrokecolor{currentstroke}%
\pgfsetdash{}{0pt}%
\pgfsys@defobject{currentmarker}{\pgfqpoint{-0.055556in}{0.000000in}}{\pgfqpoint{0.000000in}{0.000000in}}{%
\pgfpathmoveto{\pgfqpoint{0.000000in}{0.000000in}}%
\pgfpathlineto{\pgfqpoint{-0.055556in}{0.000000in}}%
\pgfusepath{stroke,fill}%
}%
\begin{pgfscope}%
\pgfsys@transformshift{7.200000in}{1.750000in}%
\pgfsys@useobject{currentmarker}{}%
\end{pgfscope}%
\end{pgfscope}%
\begin{pgfscope}%
\pgftext[left,bottom,x=0.626329in,y=1.686971in,rotate=0.000000]{{\sffamily\fontsize{12.000000}{14.400000}\selectfont 370}}
%
\end{pgfscope}%
\begin{pgfscope}%
\pgfpathrectangle{\pgfqpoint{1.000000in}{0.350000in}}{\pgfqpoint{6.200000in}{2.800000in}} %
\pgfusepath{clip}%
\pgfsetbuttcap%
\pgfsetroundjoin%
\pgfsetlinewidth{0.501875pt}%
\definecolor{currentstroke}{rgb}{0.000000,0.000000,0.000000}%
\pgfsetstrokecolor{currentstroke}%
\pgfsetdash{{1.000000pt}{3.000000pt}}{0.000000pt}%
\pgfpathmoveto{\pgfqpoint{1.000000in}{2.100000in}}%
\pgfpathlineto{\pgfqpoint{7.200000in}{2.100000in}}%
\pgfusepath{stroke}%
\end{pgfscope}%
\begin{pgfscope}%
\pgfsetbuttcap%
\pgfsetroundjoin%
\definecolor{currentfill}{rgb}{0.000000,0.000000,0.000000}%
\pgfsetfillcolor{currentfill}%
\pgfsetlinewidth{0.501875pt}%
\definecolor{currentstroke}{rgb}{0.000000,0.000000,0.000000}%
\pgfsetstrokecolor{currentstroke}%
\pgfsetdash{}{0pt}%
\pgfsys@defobject{currentmarker}{\pgfqpoint{0.000000in}{0.000000in}}{\pgfqpoint{0.055556in}{0.000000in}}{%
\pgfpathmoveto{\pgfqpoint{0.000000in}{0.000000in}}%
\pgfpathlineto{\pgfqpoint{0.055556in}{0.000000in}}%
\pgfusepath{stroke,fill}%
}%
\begin{pgfscope}%
\pgfsys@transformshift{1.000000in}{2.100000in}%
\pgfsys@useobject{currentmarker}{}%
\end{pgfscope}%
\end{pgfscope}%
\begin{pgfscope}%
\pgfsetbuttcap%
\pgfsetroundjoin%
\definecolor{currentfill}{rgb}{0.000000,0.000000,0.000000}%
\pgfsetfillcolor{currentfill}%
\pgfsetlinewidth{0.501875pt}%
\definecolor{currentstroke}{rgb}{0.000000,0.000000,0.000000}%
\pgfsetstrokecolor{currentstroke}%
\pgfsetdash{}{0pt}%
\pgfsys@defobject{currentmarker}{\pgfqpoint{-0.055556in}{0.000000in}}{\pgfqpoint{0.000000in}{0.000000in}}{%
\pgfpathmoveto{\pgfqpoint{0.000000in}{0.000000in}}%
\pgfpathlineto{\pgfqpoint{-0.055556in}{0.000000in}}%
\pgfusepath{stroke,fill}%
}%
\begin{pgfscope}%
\pgfsys@transformshift{7.200000in}{2.100000in}%
\pgfsys@useobject{currentmarker}{}%
\end{pgfscope}%
\end{pgfscope}%
\begin{pgfscope}%
\pgftext[left,bottom,x=0.626329in,y=2.036971in,rotate=0.000000]{{\sffamily\fontsize{12.000000}{14.400000}\selectfont 380}}
%
\end{pgfscope}%
\begin{pgfscope}%
\pgfpathrectangle{\pgfqpoint{1.000000in}{0.350000in}}{\pgfqpoint{6.200000in}{2.800000in}} %
\pgfusepath{clip}%
\pgfsetbuttcap%
\pgfsetroundjoin%
\pgfsetlinewidth{0.501875pt}%
\definecolor{currentstroke}{rgb}{0.000000,0.000000,0.000000}%
\pgfsetstrokecolor{currentstroke}%
\pgfsetdash{{1.000000pt}{3.000000pt}}{0.000000pt}%
\pgfpathmoveto{\pgfqpoint{1.000000in}{2.450000in}}%
\pgfpathlineto{\pgfqpoint{7.200000in}{2.450000in}}%
\pgfusepath{stroke}%
\end{pgfscope}%
\begin{pgfscope}%
\pgfsetbuttcap%
\pgfsetroundjoin%
\definecolor{currentfill}{rgb}{0.000000,0.000000,0.000000}%
\pgfsetfillcolor{currentfill}%
\pgfsetlinewidth{0.501875pt}%
\definecolor{currentstroke}{rgb}{0.000000,0.000000,0.000000}%
\pgfsetstrokecolor{currentstroke}%
\pgfsetdash{}{0pt}%
\pgfsys@defobject{currentmarker}{\pgfqpoint{0.000000in}{0.000000in}}{\pgfqpoint{0.055556in}{0.000000in}}{%
\pgfpathmoveto{\pgfqpoint{0.000000in}{0.000000in}}%
\pgfpathlineto{\pgfqpoint{0.055556in}{0.000000in}}%
\pgfusepath{stroke,fill}%
}%
\begin{pgfscope}%
\pgfsys@transformshift{1.000000in}{2.450000in}%
\pgfsys@useobject{currentmarker}{}%
\end{pgfscope}%
\end{pgfscope}%
\begin{pgfscope}%
\pgfsetbuttcap%
\pgfsetroundjoin%
\definecolor{currentfill}{rgb}{0.000000,0.000000,0.000000}%
\pgfsetfillcolor{currentfill}%
\pgfsetlinewidth{0.501875pt}%
\definecolor{currentstroke}{rgb}{0.000000,0.000000,0.000000}%
\pgfsetstrokecolor{currentstroke}%
\pgfsetdash{}{0pt}%
\pgfsys@defobject{currentmarker}{\pgfqpoint{-0.055556in}{0.000000in}}{\pgfqpoint{0.000000in}{0.000000in}}{%
\pgfpathmoveto{\pgfqpoint{0.000000in}{0.000000in}}%
\pgfpathlineto{\pgfqpoint{-0.055556in}{0.000000in}}%
\pgfusepath{stroke,fill}%
}%
\begin{pgfscope}%
\pgfsys@transformshift{7.200000in}{2.450000in}%
\pgfsys@useobject{currentmarker}{}%
\end{pgfscope}%
\end{pgfscope}%
\begin{pgfscope}%
\pgftext[left,bottom,x=0.626329in,y=2.386971in,rotate=0.000000]{{\sffamily\fontsize{12.000000}{14.400000}\selectfont 390}}
%
\end{pgfscope}%
\begin{pgfscope}%
\pgfpathrectangle{\pgfqpoint{1.000000in}{0.350000in}}{\pgfqpoint{6.200000in}{2.800000in}} %
\pgfusepath{clip}%
\pgfsetbuttcap%
\pgfsetroundjoin%
\pgfsetlinewidth{0.501875pt}%
\definecolor{currentstroke}{rgb}{0.000000,0.000000,0.000000}%
\pgfsetstrokecolor{currentstroke}%
\pgfsetdash{{1.000000pt}{3.000000pt}}{0.000000pt}%
\pgfpathmoveto{\pgfqpoint{1.000000in}{2.800000in}}%
\pgfpathlineto{\pgfqpoint{7.200000in}{2.800000in}}%
\pgfusepath{stroke}%
\end{pgfscope}%
\begin{pgfscope}%
\pgfsetbuttcap%
\pgfsetroundjoin%
\definecolor{currentfill}{rgb}{0.000000,0.000000,0.000000}%
\pgfsetfillcolor{currentfill}%
\pgfsetlinewidth{0.501875pt}%
\definecolor{currentstroke}{rgb}{0.000000,0.000000,0.000000}%
\pgfsetstrokecolor{currentstroke}%
\pgfsetdash{}{0pt}%
\pgfsys@defobject{currentmarker}{\pgfqpoint{0.000000in}{0.000000in}}{\pgfqpoint{0.055556in}{0.000000in}}{%
\pgfpathmoveto{\pgfqpoint{0.000000in}{0.000000in}}%
\pgfpathlineto{\pgfqpoint{0.055556in}{0.000000in}}%
\pgfusepath{stroke,fill}%
}%
\begin{pgfscope}%
\pgfsys@transformshift{1.000000in}{2.800000in}%
\pgfsys@useobject{currentmarker}{}%
\end{pgfscope}%
\end{pgfscope}%
\begin{pgfscope}%
\pgfsetbuttcap%
\pgfsetroundjoin%
\definecolor{currentfill}{rgb}{0.000000,0.000000,0.000000}%
\pgfsetfillcolor{currentfill}%
\pgfsetlinewidth{0.501875pt}%
\definecolor{currentstroke}{rgb}{0.000000,0.000000,0.000000}%
\pgfsetstrokecolor{currentstroke}%
\pgfsetdash{}{0pt}%
\pgfsys@defobject{currentmarker}{\pgfqpoint{-0.055556in}{0.000000in}}{\pgfqpoint{0.000000in}{0.000000in}}{%
\pgfpathmoveto{\pgfqpoint{0.000000in}{0.000000in}}%
\pgfpathlineto{\pgfqpoint{-0.055556in}{0.000000in}}%
\pgfusepath{stroke,fill}%
}%
\begin{pgfscope}%
\pgfsys@transformshift{7.200000in}{2.800000in}%
\pgfsys@useobject{currentmarker}{}%
\end{pgfscope}%
\end{pgfscope}%
\begin{pgfscope}%
\pgftext[left,bottom,x=0.626329in,y=2.736971in,rotate=0.000000]{{\sffamily\fontsize{12.000000}{14.400000}\selectfont 400}}
%
\end{pgfscope}%
\begin{pgfscope}%
\pgfpathrectangle{\pgfqpoint{1.000000in}{0.350000in}}{\pgfqpoint{6.200000in}{2.800000in}} %
\pgfusepath{clip}%
\pgfsetbuttcap%
\pgfsetroundjoin%
\pgfsetlinewidth{0.501875pt}%
\definecolor{currentstroke}{rgb}{0.000000,0.000000,0.000000}%
\pgfsetstrokecolor{currentstroke}%
\pgfsetdash{{1.000000pt}{3.000000pt}}{0.000000pt}%
\pgfpathmoveto{\pgfqpoint{1.000000in}{3.150000in}}%
\pgfpathlineto{\pgfqpoint{7.200000in}{3.150000in}}%
\pgfusepath{stroke}%
\end{pgfscope}%
\begin{pgfscope}%
\pgfsetbuttcap%
\pgfsetroundjoin%
\definecolor{currentfill}{rgb}{0.000000,0.000000,0.000000}%
\pgfsetfillcolor{currentfill}%
\pgfsetlinewidth{0.501875pt}%
\definecolor{currentstroke}{rgb}{0.000000,0.000000,0.000000}%
\pgfsetstrokecolor{currentstroke}%
\pgfsetdash{}{0pt}%
\pgfsys@defobject{currentmarker}{\pgfqpoint{0.000000in}{0.000000in}}{\pgfqpoint{0.055556in}{0.000000in}}{%
\pgfpathmoveto{\pgfqpoint{0.000000in}{0.000000in}}%
\pgfpathlineto{\pgfqpoint{0.055556in}{0.000000in}}%
\pgfusepath{stroke,fill}%
}%
\begin{pgfscope}%
\pgfsys@transformshift{1.000000in}{3.150000in}%
\pgfsys@useobject{currentmarker}{}%
\end{pgfscope}%
\end{pgfscope}%
\begin{pgfscope}%
\pgfsetbuttcap%
\pgfsetroundjoin%
\definecolor{currentfill}{rgb}{0.000000,0.000000,0.000000}%
\pgfsetfillcolor{currentfill}%
\pgfsetlinewidth{0.501875pt}%
\definecolor{currentstroke}{rgb}{0.000000,0.000000,0.000000}%
\pgfsetstrokecolor{currentstroke}%
\pgfsetdash{}{0pt}%
\pgfsys@defobject{currentmarker}{\pgfqpoint{-0.055556in}{0.000000in}}{\pgfqpoint{0.000000in}{0.000000in}}{%
\pgfpathmoveto{\pgfqpoint{0.000000in}{0.000000in}}%
\pgfpathlineto{\pgfqpoint{-0.055556in}{0.000000in}}%
\pgfusepath{stroke,fill}%
}%
\begin{pgfscope}%
\pgfsys@transformshift{7.200000in}{3.150000in}%
\pgfsys@useobject{currentmarker}{}%
\end{pgfscope}%
\end{pgfscope}%
\begin{pgfscope}%
\pgftext[left,bottom,x=0.626329in,y=3.086971in,rotate=0.000000]{{\sffamily\fontsize{12.000000}{14.400000}\selectfont 410}}
%
\end{pgfscope}%
\begin{pgfscope}%
\pgftext[left,bottom,x=0.556885in,y=0.616536in,rotate=90.000000]{{\sffamily\fontsize{12.000000}{14.400000}\selectfont number of hydrogen bonds}}
%
\end{pgfscope}%
\begin{pgfscope}%
\pgfsetrectcap%
\pgfsetroundjoin%
\pgfsetlinewidth{1.003750pt}%
\definecolor{currentstroke}{rgb}{0.000000,0.000000,0.000000}%
\pgfsetstrokecolor{currentstroke}%
\pgfsetdash{}{0pt}%
\pgfpathmoveto{\pgfqpoint{1.000000in}{3.150000in}}%
\pgfpathlineto{\pgfqpoint{7.200000in}{3.150000in}}%
\pgfusepath{stroke}%
\end{pgfscope}%
\begin{pgfscope}%
\pgfsetrectcap%
\pgfsetroundjoin%
\pgfsetlinewidth{1.003750pt}%
\definecolor{currentstroke}{rgb}{0.000000,0.000000,0.000000}%
\pgfsetstrokecolor{currentstroke}%
\pgfsetdash{}{0pt}%
\pgfpathmoveto{\pgfqpoint{7.200000in}{0.350000in}}%
\pgfpathlineto{\pgfqpoint{7.200000in}{3.150000in}}%
\pgfusepath{stroke}%
\end{pgfscope}%
\begin{pgfscope}%
\pgfsetrectcap%
\pgfsetroundjoin%
\pgfsetlinewidth{1.003750pt}%
\definecolor{currentstroke}{rgb}{0.000000,0.000000,0.000000}%
\pgfsetstrokecolor{currentstroke}%
\pgfsetdash{}{0pt}%
\pgfpathmoveto{\pgfqpoint{1.000000in}{0.350000in}}%
\pgfpathlineto{\pgfqpoint{7.200000in}{0.350000in}}%
\pgfusepath{stroke}%
\end{pgfscope}%
\begin{pgfscope}%
\pgfsetrectcap%
\pgfsetroundjoin%
\pgfsetlinewidth{1.003750pt}%
\definecolor{currentstroke}{rgb}{0.000000,0.000000,0.000000}%
\pgfsetstrokecolor{currentstroke}%
\pgfsetdash{}{0pt}%
\pgfpathmoveto{\pgfqpoint{1.000000in}{0.350000in}}%
\pgfpathlineto{\pgfqpoint{1.000000in}{3.150000in}}%
\pgfusepath{stroke}%
\end{pgfscope}%
\begin{pgfscope}%
\pgfsetrectcap%
\pgfsetroundjoin%
\definecolor{currentfill}{rgb}{1.000000,1.000000,1.000000}%
\pgfsetfillcolor{currentfill}%
\pgfsetlinewidth{1.003750pt}%
\definecolor{currentstroke}{rgb}{0.000000,0.000000,0.000000}%
\pgfsetstrokecolor{currentstroke}%
\pgfsetdash{}{0pt}%
\pgfpathmoveto{\pgfqpoint{1.069417in}{2.427606in}}%
\pgfpathlineto{\pgfqpoint{1.926808in}{2.427606in}}%
\pgfpathlineto{\pgfqpoint{1.926808in}{3.080583in}}%
\pgfpathlineto{\pgfqpoint{1.069417in}{3.080583in}}%
\pgfpathlineto{\pgfqpoint{1.069417in}{2.427606in}}%
\pgfpathclose%
\pgfusepath{stroke,fill}%
\end{pgfscope}%
\begin{pgfscope}%
\pgfsetrectcap%
\pgfsetroundjoin%
\pgfsetlinewidth{1.003750pt}%
\definecolor{currentstroke}{rgb}{0.000000,0.000000,1.000000}%
\pgfsetstrokecolor{currentstroke}%
\pgfsetdash{}{0pt}%
\pgfpathmoveto{\pgfqpoint{1.166600in}{2.968161in}}%
\pgfpathlineto{\pgfqpoint{1.360967in}{2.968161in}}%
\pgfusepath{stroke}%
\end{pgfscope}%
\begin{pgfscope}%
\pgftext[left,bottom,x=1.513683in,y=2.890691in,rotate=0.000000]{{\sffamily\fontsize{9.996000}{11.995200}\selectfont spc}}
%
\end{pgfscope}%
\begin{pgfscope}%
\pgfsetrectcap%
\pgfsetroundjoin%
\pgfsetlinewidth{1.003750pt}%
\definecolor{currentstroke}{rgb}{0.000000,0.500000,0.000000}%
\pgfsetstrokecolor{currentstroke}%
\pgfsetdash{}{0pt}%
\pgfpathmoveto{\pgfqpoint{1.166600in}{2.764385in}}%
\pgfpathlineto{\pgfqpoint{1.360967in}{2.764385in}}%
\pgfusepath{stroke}%
\end{pgfscope}%
\begin{pgfscope}%
\pgftext[left,bottom,x=1.513683in,y=2.686915in,rotate=0.000000]{{\sffamily\fontsize{9.996000}{11.995200}\selectfont spce}}
%
\end{pgfscope}%
\begin{pgfscope}%
\pgfsetrectcap%
\pgfsetroundjoin%
\pgfsetlinewidth{1.003750pt}%
\definecolor{currentstroke}{rgb}{1.000000,0.000000,0.000000}%
\pgfsetstrokecolor{currentstroke}%
\pgfsetdash{}{0pt}%
\pgfpathmoveto{\pgfqpoint{1.166600in}{2.560609in}}%
\pgfpathlineto{\pgfqpoint{1.360967in}{2.560609in}}%
\pgfusepath{stroke}%
\end{pgfscope}%
\begin{pgfscope}%
\pgftext[left,bottom,x=1.513683in,y=2.483139in,rotate=0.000000]{{\sffamily\fontsize{9.996000}{11.995200}\selectfont tip3p}}
%
\end{pgfscope}%
\end{pgfpicture}%
\makeatother%
\endgroup%
}
    \caption{Fluctuation of the number of hydrogen bonds.} \label{fig:hbnum}
\end{figure}

The water molecule works as a donor and an acceptor for hydrogen bridges due to its electronegative oxygen and electropositive hydrogen. The number of hydrogen bonds fluctuates. The average was calculated starting at $\unit[100]{ps}$ to ensure equilibrium. \textit{SPC/E} yields the highest and \textit{TIP3P} the lowest value. The difference between the values is up to $\unit[9]{\%}$.

\subsection{Mean-square displacement}
\begin{figure}[H]
	\resizebox{\linewidth}{!}{%% Creator: Matplotlib, PGF backend
%%
%% To include the figure in your LaTeX document, write
%%   \input{<filename>.pgf}
%%
%% Make sure the required packages are loaded in your preamble
%%   \usepackage{pgf}
%%
%% Figures using additional raster images can only be included by \input if
%% they are in the same directory as the main LaTeX file. For loading figures
%% from other directories you can use the `import` package
%%   \usepackage{import}
%% and then include the figures with
%%   \import{<path to file>}{<filename>.pgf}
%%
%% Matplotlib used the following preamble
%%   \usepackage{fontspec}
%%   \setmainfont{DejaVu Serif}
%%   \setsansfont{DejaVu Sans}
%%   \setmonofont{DejaVu Sans Mono}
%%
\begingroup%
\makeatletter%
\begin{pgfpicture}%
\pgfpathrectangle{\pgfpointorigin}{\pgfqpoint{8.000000in}{3.000000in}}%
\pgfusepath{use as bounding box}%
\begin{pgfscope}%
\pgfsetrectcap%
\pgfsetroundjoin%
\definecolor{currentfill}{rgb}{1.000000,1.000000,1.000000}%
\pgfsetfillcolor{currentfill}%
\pgfsetlinewidth{0.000000pt}%
\definecolor{currentstroke}{rgb}{1.000000,1.000000,1.000000}%
\pgfsetstrokecolor{currentstroke}%
\pgfsetdash{}{0pt}%
\pgfpathmoveto{\pgfqpoint{0.000000in}{0.000000in}}%
\pgfpathlineto{\pgfqpoint{8.000000in}{0.000000in}}%
\pgfpathlineto{\pgfqpoint{8.000000in}{3.000000in}}%
\pgfpathlineto{\pgfqpoint{0.000000in}{3.000000in}}%
\pgfpathclose%
\pgfusepath{fill}%
\end{pgfscope}%
\begin{pgfscope}%
\pgfsetrectcap%
\pgfsetroundjoin%
\definecolor{currentfill}{rgb}{1.000000,1.000000,1.000000}%
\pgfsetfillcolor{currentfill}%
\pgfsetlinewidth{0.000000pt}%
\definecolor{currentstroke}{rgb}{0.000000,0.000000,0.000000}%
\pgfsetstrokecolor{currentstroke}%
\pgfsetdash{}{0pt}%
\pgfpathmoveto{\pgfqpoint{1.000000in}{0.300000in}}%
\pgfpathlineto{\pgfqpoint{7.200000in}{0.300000in}}%
\pgfpathlineto{\pgfqpoint{7.200000in}{2.700000in}}%
\pgfpathlineto{\pgfqpoint{1.000000in}{2.700000in}}%
\pgfpathclose%
\pgfusepath{fill}%
\end{pgfscope}%
\begin{pgfscope}%
\pgfpathrectangle{\pgfqpoint{1.000000in}{0.300000in}}{\pgfqpoint{6.200000in}{2.400000in}} %
\pgfusepath{clip}%
\pgfsetrectcap%
\pgfsetroundjoin%
\pgfsetlinewidth{1.003750pt}%
\definecolor{currentstroke}{rgb}{0.000000,0.000000,1.000000}%
\pgfsetstrokecolor{currentstroke}%
\pgfsetdash{}{0pt}%
\pgfpathmoveto{\pgfqpoint{1.000000in}{0.490652in}}%
\pgfpathlineto{\pgfqpoint{1.466596in}{0.678595in}}%
\pgfpathlineto{\pgfqpoint{1.739538in}{0.776805in}}%
\pgfpathlineto{\pgfqpoint{1.933193in}{0.837675in}}%
\pgfpathlineto{\pgfqpoint{2.614159in}{1.018632in}}%
\pgfpathlineto{\pgfqpoint{2.822941in}{1.068905in}}%
\pgfpathlineto{\pgfqpoint{2.945672in}{1.103021in}}%
\pgfpathlineto{\pgfqpoint{3.311611in}{1.206745in}}%
\pgfpathlineto{\pgfqpoint{3.448665in}{1.243931in}}%
\pgfpathlineto{\pgfqpoint{3.562479in}{1.278696in}}%
\pgfpathlineto{\pgfqpoint{3.697562in}{1.318435in}}%
\pgfpathlineto{\pgfqpoint{3.810016in}{1.353491in}}%
\pgfpathlineto{\pgfqpoint{3.982633in}{1.403738in}}%
\pgfpathlineto{\pgfqpoint{4.006255in}{1.409618in}}%
\pgfpathlineto{\pgfqpoint{4.036514in}{1.418703in}}%
\pgfpathlineto{\pgfqpoint{4.188203in}{1.463418in}}%
\pgfpathlineto{\pgfqpoint{4.211417in}{1.470487in}}%
\pgfpathlineto{\pgfqpoint{4.244804in}{1.481321in}}%
\pgfpathlineto{\pgfqpoint{4.331290in}{1.507673in}}%
\pgfpathlineto{\pgfqpoint{4.345462in}{1.513228in}}%
\pgfpathlineto{\pgfqpoint{4.395014in}{1.527894in}}%
\pgfpathlineto{\pgfqpoint{4.453224in}{1.547018in}}%
\pgfpathlineto{\pgfqpoint{4.491922in}{1.557454in}}%
\pgfpathlineto{\pgfqpoint{4.950955in}{1.696155in}}%
\pgfpathlineto{\pgfqpoint{4.956636in}{1.697696in}}%
\pgfpathlineto{\pgfqpoint{4.969707in}{1.702707in}}%
\pgfpathlineto{\pgfqpoint{4.998665in}{1.710671in}}%
\pgfpathlineto{\pgfqpoint{5.017874in}{1.716260in}}%
\pgfpathlineto{\pgfqpoint{5.263288in}{1.791778in}}%
\pgfpathlineto{\pgfqpoint{5.270424in}{1.793495in}}%
\pgfpathlineto{\pgfqpoint{5.283313in}{1.797389in}}%
\pgfpathlineto{\pgfqpoint{5.300501in}{1.802135in}}%
\pgfpathlineto{\pgfqpoint{5.309492in}{1.805282in}}%
\pgfpathlineto{\pgfqpoint{5.347472in}{1.816202in}}%
\pgfpathlineto{\pgfqpoint{5.354818in}{1.818604in}}%
\pgfpathlineto{\pgfqpoint{5.380416in}{1.826285in}}%
\pgfpathlineto{\pgfqpoint{5.385421in}{1.827569in}}%
\pgfpathlineto{\pgfqpoint{5.394337in}{1.830562in}}%
\pgfpathlineto{\pgfqpoint{5.406044in}{1.833712in}}%
\pgfpathlineto{\pgfqpoint{5.432595in}{1.842333in}}%
\pgfpathlineto{\pgfqpoint{5.453647in}{1.847937in}}%
\pgfpathlineto{\pgfqpoint{5.461709in}{1.850274in}}%
\pgfpathlineto{\pgfqpoint{5.501470in}{1.862880in}}%
\pgfpathlineto{\pgfqpoint{5.505654in}{1.864186in}}%
\pgfpathlineto{\pgfqpoint{5.547689in}{1.876256in}}%
\pgfpathlineto{\pgfqpoint{5.553930in}{1.878220in}}%
\pgfpathlineto{\pgfqpoint{5.562416in}{1.880542in}}%
\pgfpathlineto{\pgfqpoint{5.574573in}{1.883765in}}%
\pgfpathlineto{\pgfqpoint{5.630190in}{1.900503in}}%
\pgfpathlineto{\pgfqpoint{5.633647in}{1.901690in}}%
\pgfpathlineto{\pgfqpoint{5.639142in}{1.903373in}}%
\pgfpathlineto{\pgfqpoint{5.642554in}{1.904314in}}%
\pgfpathlineto{\pgfqpoint{5.647977in}{1.905600in}}%
\pgfpathlineto{\pgfqpoint{5.651345in}{1.907215in}}%
\pgfpathlineto{\pgfqpoint{5.654696in}{1.908233in}}%
\pgfpathlineto{\pgfqpoint{5.680274in}{1.915580in}}%
\pgfpathlineto{\pgfqpoint{5.694915in}{1.920019in}}%
\pgfpathlineto{\pgfqpoint{5.698057in}{1.920621in}}%
\pgfpathlineto{\pgfqpoint{5.705536in}{1.922882in}}%
\pgfpathlineto{\pgfqpoint{5.714159in}{1.925454in}}%
\pgfpathlineto{\pgfqpoint{5.716602in}{1.926911in}}%
\pgfpathlineto{\pgfqpoint{5.727489in}{1.929980in}}%
\pgfpathlineto{\pgfqpoint{5.735244in}{1.932092in}}%
\pgfpathlineto{\pgfqpoint{5.741149in}{1.933727in}}%
\pgfpathlineto{\pgfqpoint{5.746419in}{1.935760in}}%
\pgfpathlineto{\pgfqpoint{5.749329in}{1.936639in}}%
\pgfpathlineto{\pgfqpoint{5.755112in}{1.938305in}}%
\pgfpathlineto{\pgfqpoint{5.758559in}{1.939157in}}%
\pgfpathlineto{\pgfqpoint{5.769920in}{1.942704in}}%
\pgfpathlineto{\pgfqpoint{5.784409in}{1.947343in}}%
\pgfpathlineto{\pgfqpoint{5.790447in}{1.949074in}}%
\pgfpathlineto{\pgfqpoint{5.796972in}{1.950721in}}%
\pgfpathlineto{\pgfqpoint{5.810366in}{1.954884in}}%
\pgfpathlineto{\pgfqpoint{5.813541in}{1.955664in}}%
\pgfpathlineto{\pgfqpoint{5.822978in}{1.958735in}}%
\pgfpathlineto{\pgfqpoint{5.829712in}{1.961101in}}%
\pgfpathlineto{\pgfqpoint{5.862408in}{1.970415in}}%
\pgfpathlineto{\pgfqpoint{5.866325in}{1.971851in}}%
\pgfpathlineto{\pgfqpoint{5.909800in}{1.984563in}}%
\pgfpathlineto{\pgfqpoint{5.914360in}{1.986203in}}%
\pgfpathlineto{\pgfqpoint{5.929640in}{1.991107in}}%
\pgfpathlineto{\pgfqpoint{5.932743in}{1.991880in}}%
\pgfpathlineto{\pgfqpoint{5.952786in}{1.997284in}}%
\pgfpathlineto{\pgfqpoint{5.957066in}{1.998419in}}%
\pgfpathlineto{\pgfqpoint{5.972250in}{2.003475in}}%
\pgfpathlineto{\pgfqpoint{5.975579in}{2.004147in}}%
\pgfpathlineto{\pgfqpoint{5.994807in}{2.009010in}}%
\pgfpathlineto{\pgfqpoint{5.999229in}{2.010413in}}%
\pgfpathlineto{\pgfqpoint{6.010356in}{2.013824in}}%
\pgfpathlineto{\pgfqpoint{6.025939in}{2.017674in}}%
\pgfpathlineto{\pgfqpoint{6.033217in}{2.019566in}}%
\pgfpathlineto{\pgfqpoint{6.036257in}{2.020037in}}%
\pgfpathlineto{\pgfqpoint{6.040039in}{2.021398in}}%
\pgfpathlineto{\pgfqpoint{6.041922in}{2.022084in}}%
\pgfpathlineto{\pgfqpoint{6.070274in}{2.029160in}}%
\pgfpathlineto{\pgfqpoint{6.072793in}{2.029274in}}%
\pgfpathlineto{\pgfqpoint{6.079228in}{2.031932in}}%
\pgfpathlineto{\pgfqpoint{6.103340in}{2.038358in}}%
\pgfpathlineto{\pgfqpoint{6.105397in}{2.038453in}}%
\pgfpathlineto{\pgfqpoint{6.108129in}{2.038930in}}%
\pgfpathlineto{\pgfqpoint{6.111868in}{2.040383in}}%
\pgfpathlineto{\pgfqpoint{6.113898in}{2.041364in}}%
\pgfpathlineto{\pgfqpoint{6.131245in}{2.045936in}}%
\pgfpathlineto{\pgfqpoint{6.133547in}{2.046757in}}%
\pgfpathlineto{\pgfqpoint{6.137474in}{2.048063in}}%
\pgfpathlineto{\pgfqpoint{6.142351in}{2.048772in}}%
\pgfpathlineto{\pgfqpoint{6.145260in}{2.049656in}}%
\pgfpathlineto{\pgfqpoint{6.159936in}{2.054396in}}%
\pgfpathlineto{\pgfqpoint{6.164340in}{2.055031in}}%
\pgfpathlineto{\pgfqpoint{6.177381in}{2.058923in}}%
\pgfpathlineto{\pgfqpoint{6.180143in}{2.060203in}}%
\pgfpathlineto{\pgfqpoint{6.187756in}{2.062394in}}%
\pgfpathlineto{\pgfqpoint{6.190174in}{2.063065in}}%
\pgfpathlineto{\pgfqpoint{6.192884in}{2.063385in}}%
\pgfpathlineto{\pgfqpoint{6.195284in}{2.063814in}}%
\pgfpathlineto{\pgfqpoint{6.202729in}{2.065390in}}%
\pgfpathlineto{\pgfqpoint{6.204504in}{2.066239in}}%
\pgfpathlineto{\pgfqpoint{6.218245in}{2.070677in}}%
\pgfpathlineto{\pgfqpoint{6.241008in}{2.077864in}}%
\pgfpathlineto{\pgfqpoint{6.244079in}{2.078182in}}%
\pgfpathlineto{\pgfqpoint{6.263568in}{2.082853in}}%
\pgfpathlineto{\pgfqpoint{6.266538in}{2.084775in}}%
\pgfpathlineto{\pgfqpoint{6.268152in}{2.085261in}}%
\pgfpathlineto{\pgfqpoint{6.271369in}{2.085694in}}%
\pgfpathlineto{\pgfqpoint{6.276962in}{2.087250in}}%
\pgfpathlineto{\pgfqpoint{6.278816in}{2.087385in}}%
\pgfpathlineto{\pgfqpoint{6.283560in}{2.088839in}}%
\pgfpathlineto{\pgfqpoint{6.286705in}{2.088934in}}%
\pgfpathlineto{\pgfqpoint{6.288532in}{2.089970in}}%
\pgfpathlineto{\pgfqpoint{6.290355in}{2.090515in}}%
\pgfpathlineto{\pgfqpoint{6.292950in}{2.091989in}}%
\pgfpathlineto{\pgfqpoint{6.294502in}{2.092319in}}%
\pgfpathlineto{\pgfqpoint{6.301954in}{2.094096in}}%
\pgfpathlineto{\pgfqpoint{6.302976in}{2.094429in}}%
\pgfpathlineto{\pgfqpoint{6.304505in}{2.094523in}}%
\pgfpathlineto{\pgfqpoint{6.308567in}{2.095866in}}%
\pgfpathlineto{\pgfqpoint{6.334871in}{2.102772in}}%
\pgfpathlineto{\pgfqpoint{6.337058in}{2.102965in}}%
\pgfpathlineto{\pgfqpoint{6.339237in}{2.104060in}}%
\pgfpathlineto{\pgfqpoint{6.342854in}{2.105757in}}%
\pgfpathlineto{\pgfqpoint{6.344536in}{2.106093in}}%
\pgfpathlineto{\pgfqpoint{6.347170in}{2.106529in}}%
\pgfpathlineto{\pgfqpoint{6.392002in}{2.119209in}}%
\pgfpathlineto{\pgfqpoint{6.393565in}{2.119220in}}%
\pgfpathlineto{\pgfqpoint{6.396902in}{2.120088in}}%
\pgfpathlineto{\pgfqpoint{6.399560in}{2.120643in}}%
\pgfpathlineto{\pgfqpoint{6.401327in}{2.121254in}}%
\pgfpathlineto{\pgfqpoint{6.410959in}{2.124137in}}%
\pgfpathlineto{\pgfqpoint{6.414860in}{2.125281in}}%
\pgfpathlineto{\pgfqpoint{6.416371in}{2.125686in}}%
\pgfpathlineto{\pgfqpoint{6.421954in}{2.126974in}}%
\pgfpathlineto{\pgfqpoint{6.426004in}{2.127602in}}%
\pgfpathlineto{\pgfqpoint{6.427491in}{2.127653in}}%
\pgfpathlineto{\pgfqpoint{6.430242in}{2.128678in}}%
\pgfpathlineto{\pgfqpoint{6.440929in}{2.132036in}}%
\pgfpathlineto{\pgfqpoint{6.444040in}{2.133084in}}%
\pgfpathlineto{\pgfqpoint{6.451244in}{2.134426in}}%
\pgfpathlineto{\pgfqpoint{6.453697in}{2.135367in}}%
\pgfpathlineto{\pgfqpoint{6.455733in}{2.136042in}}%
\pgfpathlineto{\pgfqpoint{6.458372in}{2.137109in}}%
\pgfpathlineto{\pgfqpoint{6.460395in}{2.137370in}}%
\pgfpathlineto{\pgfqpoint{6.465826in}{2.139498in}}%
\pgfpathlineto{\pgfqpoint{6.467227in}{2.139225in}}%
\pgfpathlineto{\pgfqpoint{6.468425in}{2.139067in}}%
\pgfpathlineto{\pgfqpoint{6.474782in}{2.141288in}}%
\pgfpathlineto{\pgfqpoint{6.476164in}{2.142283in}}%
\pgfpathlineto{\pgfqpoint{6.478724in}{2.142863in}}%
\pgfpathlineto{\pgfqpoint{6.480098in}{2.143210in}}%
\pgfpathlineto{\pgfqpoint{6.485179in}{2.145018in}}%
\pgfpathlineto{\pgfqpoint{6.486928in}{2.144984in}}%
\pgfpathlineto{\pgfqpoint{6.487705in}{2.144757in}}%
\pgfpathlineto{\pgfqpoint{6.488092in}{2.144965in}}%
\pgfpathlineto{\pgfqpoint{6.493305in}{2.147049in}}%
\pgfpathlineto{\pgfqpoint{6.494458in}{2.146772in}}%
\pgfpathlineto{\pgfqpoint{6.498288in}{2.148178in}}%
\pgfpathlineto{\pgfqpoint{6.500004in}{2.148699in}}%
\pgfpathlineto{\pgfqpoint{6.502664in}{2.149678in}}%
\pgfpathlineto{\pgfqpoint{6.503612in}{2.150240in}}%
\pgfpathlineto{\pgfqpoint{6.505881in}{2.150714in}}%
\pgfpathlineto{\pgfqpoint{6.507201in}{2.150395in}}%
\pgfpathlineto{\pgfqpoint{6.512269in}{2.152104in}}%
\pgfpathlineto{\pgfqpoint{6.513950in}{2.152363in}}%
\pgfpathlineto{\pgfqpoint{6.516184in}{2.153438in}}%
\pgfpathlineto{\pgfqpoint{6.539748in}{2.160542in}}%
\pgfpathlineto{\pgfqpoint{6.541362in}{2.161838in}}%
\pgfpathlineto{\pgfqpoint{6.542793in}{2.161362in}}%
\pgfpathlineto{\pgfqpoint{6.544043in}{2.161378in}}%
\pgfpathlineto{\pgfqpoint{6.549019in}{2.162800in}}%
\pgfpathlineto{\pgfqpoint{6.552551in}{2.164128in}}%
\pgfpathlineto{\pgfqpoint{6.555538in}{2.165406in}}%
\pgfpathlineto{\pgfqpoint{6.557114in}{2.165562in}}%
\pgfpathlineto{\pgfqpoint{6.560256in}{2.167340in}}%
\pgfpathlineto{\pgfqpoint{6.561126in}{2.166801in}}%
\pgfpathlineto{\pgfqpoint{6.561474in}{2.166970in}}%
\pgfpathlineto{\pgfqpoint{6.564595in}{2.167948in}}%
\pgfpathlineto{\pgfqpoint{6.565805in}{2.168142in}}%
\pgfpathlineto{\pgfqpoint{6.570966in}{2.169870in}}%
\pgfpathlineto{\pgfqpoint{6.572848in}{2.170470in}}%
\pgfpathlineto{\pgfqpoint{6.573702in}{2.170172in}}%
\pgfpathlineto{\pgfqpoint{6.574043in}{2.170764in}}%
\pgfpathlineto{\pgfqpoint{6.575066in}{2.171476in}}%
\pgfpathlineto{\pgfqpoint{6.576937in}{2.173129in}}%
\pgfpathlineto{\pgfqpoint{6.577107in}{2.172978in}}%
\pgfpathlineto{\pgfqpoint{6.578634in}{2.172310in}}%
\pgfpathlineto{\pgfqpoint{6.582856in}{2.173352in}}%
\pgfpathlineto{\pgfqpoint{6.583866in}{2.174314in}}%
\pgfpathlineto{\pgfqpoint{6.585713in}{2.175049in}}%
\pgfpathlineto{\pgfqpoint{6.586048in}{2.174278in}}%
\pgfpathlineto{\pgfqpoint{6.586885in}{2.174965in}}%
\pgfpathlineto{\pgfqpoint{6.588557in}{2.175851in}}%
\pgfpathlineto{\pgfqpoint{6.590058in}{2.175563in}}%
\pgfpathlineto{\pgfqpoint{6.596028in}{2.177899in}}%
\pgfpathlineto{\pgfqpoint{6.597183in}{2.178310in}}%
\pgfpathlineto{\pgfqpoint{6.598171in}{2.178648in}}%
\pgfpathlineto{\pgfqpoint{6.598336in}{2.178407in}}%
\pgfpathlineto{\pgfqpoint{6.599651in}{2.178726in}}%
\pgfpathlineto{\pgfqpoint{6.600471in}{2.180028in}}%
\pgfpathlineto{\pgfqpoint{6.602273in}{2.180906in}}%
\pgfpathlineto{\pgfqpoint{6.602437in}{2.180676in}}%
\pgfpathlineto{\pgfqpoint{6.602600in}{2.180206in}}%
\pgfpathlineto{\pgfqpoint{6.603417in}{2.180624in}}%
\pgfpathlineto{\pgfqpoint{6.605048in}{2.181798in}}%
\pgfpathlineto{\pgfqpoint{6.605211in}{2.181616in}}%
\pgfpathlineto{\pgfqpoint{6.607975in}{2.182094in}}%
\pgfpathlineto{\pgfqpoint{6.609595in}{2.183868in}}%
\pgfpathlineto{\pgfqpoint{6.612984in}{2.183317in}}%
\pgfpathlineto{\pgfqpoint{6.614432in}{2.184199in}}%
\pgfpathlineto{\pgfqpoint{6.615716in}{2.184056in}}%
\pgfpathlineto{\pgfqpoint{6.617797in}{2.185624in}}%
\pgfpathlineto{\pgfqpoint{6.618277in}{2.185843in}}%
\pgfpathlineto{\pgfqpoint{6.618596in}{2.185276in}}%
\pgfpathlineto{\pgfqpoint{6.619553in}{2.185081in}}%
\pgfpathlineto{\pgfqpoint{6.619713in}{2.185391in}}%
\pgfpathlineto{\pgfqpoint{6.620509in}{2.186252in}}%
\pgfpathlineto{\pgfqpoint{6.622258in}{2.187613in}}%
\pgfpathlineto{\pgfqpoint{6.622417in}{2.187389in}}%
\pgfpathlineto{\pgfqpoint{6.623052in}{2.186987in}}%
\pgfpathlineto{\pgfqpoint{6.623528in}{2.187721in}}%
\pgfpathlineto{\pgfqpoint{6.625111in}{2.188748in}}%
\pgfpathlineto{\pgfqpoint{6.626059in}{2.189038in}}%
\pgfpathlineto{\pgfqpoint{6.626217in}{2.188846in}}%
\pgfpathlineto{\pgfqpoint{6.627478in}{2.187994in}}%
\pgfpathlineto{\pgfqpoint{6.627636in}{2.188084in}}%
\pgfpathlineto{\pgfqpoint{6.629367in}{2.188924in}}%
\pgfpathlineto{\pgfqpoint{6.630779in}{2.189491in}}%
\pgfpathlineto{\pgfqpoint{6.632189in}{2.189305in}}%
\pgfpathlineto{\pgfqpoint{6.633908in}{2.190838in}}%
\pgfpathlineto{\pgfqpoint{6.634064in}{2.190760in}}%
\pgfpathlineto{\pgfqpoint{6.635311in}{2.189722in}}%
\pgfpathlineto{\pgfqpoint{6.635467in}{2.190004in}}%
\pgfpathlineto{\pgfqpoint{6.641667in}{2.193905in}}%
\pgfpathlineto{\pgfqpoint{6.644131in}{2.192785in}}%
\pgfpathlineto{\pgfqpoint{6.651622in}{2.195961in}}%
\pgfpathlineto{\pgfqpoint{6.653292in}{2.197653in}}%
\pgfpathlineto{\pgfqpoint{6.653444in}{2.197419in}}%
\pgfpathlineto{\pgfqpoint{6.654201in}{2.196954in}}%
\pgfpathlineto{\pgfqpoint{6.654655in}{2.197303in}}%
\pgfpathlineto{\pgfqpoint{6.657072in}{2.198880in}}%
\pgfpathlineto{\pgfqpoint{6.657223in}{2.198727in}}%
\pgfpathlineto{\pgfqpoint{6.659030in}{2.198228in}}%
\pgfpathlineto{\pgfqpoint{6.659481in}{2.198140in}}%
\pgfpathlineto{\pgfqpoint{6.659932in}{2.198542in}}%
\pgfpathlineto{\pgfqpoint{6.661581in}{2.198542in}}%
\pgfpathlineto{\pgfqpoint{6.662479in}{2.197958in}}%
\pgfpathlineto{\pgfqpoint{6.662779in}{2.198948in}}%
\pgfpathlineto{\pgfqpoint{6.663675in}{2.199790in}}%
\pgfpathlineto{\pgfqpoint{6.664123in}{2.199634in}}%
\pgfpathlineto{\pgfqpoint{6.664421in}{2.200219in}}%
\pgfpathlineto{\pgfqpoint{6.664869in}{2.199462in}}%
\pgfpathlineto{\pgfqpoint{6.664869in}{2.199462in}}%
\pgfpathlineto{\pgfqpoint{6.665018in}{2.199014in}}%
\pgfpathlineto{\pgfqpoint{6.665316in}{2.199485in}}%
\pgfpathlineto{\pgfqpoint{6.665316in}{2.199485in}}%
\pgfpathlineto{\pgfqpoint{6.667546in}{2.202355in}}%
\pgfpathlineto{\pgfqpoint{6.668881in}{2.203152in}}%
\pgfpathlineto{\pgfqpoint{6.669029in}{2.202673in}}%
\pgfpathlineto{\pgfqpoint{6.670213in}{2.203476in}}%
\pgfpathlineto{\pgfqpoint{6.671986in}{2.204608in}}%
\pgfpathlineto{\pgfqpoint{6.677275in}{2.202886in}}%
\pgfpathlineto{\pgfqpoint{6.678444in}{2.204860in}}%
\pgfpathlineto{\pgfqpoint{6.678590in}{2.204448in}}%
\pgfpathlineto{\pgfqpoint{6.679320in}{2.204816in}}%
\pgfpathlineto{\pgfqpoint{6.679612in}{2.204240in}}%
\pgfpathlineto{\pgfqpoint{6.679758in}{2.203927in}}%
\pgfpathlineto{\pgfqpoint{6.679904in}{2.204343in}}%
\pgfpathlineto{\pgfqpoint{6.679904in}{2.204343in}}%
\pgfpathlineto{\pgfqpoint{6.685998in}{2.209760in}}%
\pgfpathlineto{\pgfqpoint{6.686431in}{2.209102in}}%
\pgfpathlineto{\pgfqpoint{6.688881in}{2.207784in}}%
\pgfpathlineto{\pgfqpoint{6.689600in}{2.209137in}}%
\pgfpathlineto{\pgfqpoint{6.690461in}{2.208545in}}%
\pgfpathlineto{\pgfqpoint{6.691752in}{2.207458in}}%
\pgfpathlineto{\pgfqpoint{6.692038in}{2.208328in}}%
\pgfpathlineto{\pgfqpoint{6.693611in}{2.210186in}}%
\pgfpathlineto{\pgfqpoint{6.694895in}{2.209148in}}%
\pgfpathlineto{\pgfqpoint{6.695038in}{2.209693in}}%
\pgfpathlineto{\pgfqpoint{6.695465in}{2.210773in}}%
\pgfpathlineto{\pgfqpoint{6.700291in}{2.216004in}}%
\pgfpathlineto{\pgfqpoint{6.700432in}{2.215673in}}%
\pgfpathlineto{\pgfqpoint{6.701845in}{2.213630in}}%
\pgfpathlineto{\pgfqpoint{6.702268in}{2.214524in}}%
\pgfpathlineto{\pgfqpoint{6.702972in}{2.214140in}}%
\pgfpathlineto{\pgfqpoint{6.703113in}{2.213787in}}%
\pgfpathlineto{\pgfqpoint{6.703536in}{2.214169in}}%
\pgfpathlineto{\pgfqpoint{6.703536in}{2.214169in}}%
\pgfpathlineto{\pgfqpoint{6.703817in}{2.215381in}}%
\pgfpathlineto{\pgfqpoint{6.704661in}{2.214629in}}%
\pgfpathlineto{\pgfqpoint{6.706905in}{2.215050in}}%
\pgfpathlineto{\pgfqpoint{6.707045in}{2.215327in}}%
\pgfpathlineto{\pgfqpoint{6.707045in}{2.215327in}}%
\pgfpathlineto{\pgfqpoint{6.707045in}{2.215327in}}%
\pgfpathlineto{\pgfqpoint{6.707185in}{2.214886in}}%
\pgfpathlineto{\pgfqpoint{6.707605in}{2.215443in}}%
\pgfpathlineto{\pgfqpoint{6.707605in}{2.215443in}}%
\pgfpathlineto{\pgfqpoint{6.707745in}{2.215585in}}%
\pgfpathlineto{\pgfqpoint{6.707884in}{2.215429in}}%
\pgfpathlineto{\pgfqpoint{6.707884in}{2.215429in}}%
\pgfpathlineto{\pgfqpoint{6.708304in}{2.213859in}}%
\pgfpathlineto{\pgfqpoint{6.709002in}{2.214235in}}%
\pgfpathlineto{\pgfqpoint{6.711093in}{2.218737in}}%
\pgfpathlineto{\pgfqpoint{6.711927in}{2.218740in}}%
\pgfpathlineto{\pgfqpoint{6.712066in}{2.217961in}}%
\pgfpathlineto{\pgfqpoint{6.712205in}{2.218065in}}%
\pgfpathlineto{\pgfqpoint{6.712205in}{2.218065in}}%
\pgfpathlineto{\pgfqpoint{6.713455in}{2.219377in}}%
\pgfpathlineto{\pgfqpoint{6.714286in}{2.221357in}}%
\pgfpathlineto{\pgfqpoint{6.714840in}{2.219542in}}%
\pgfpathlineto{\pgfqpoint{6.717464in}{2.217770in}}%
\pgfpathlineto{\pgfqpoint{6.717602in}{2.218182in}}%
\pgfpathlineto{\pgfqpoint{6.718842in}{2.219703in}}%
\pgfpathlineto{\pgfqpoint{6.718979in}{2.219371in}}%
\pgfpathlineto{\pgfqpoint{6.719117in}{2.219549in}}%
\pgfpathlineto{\pgfqpoint{6.719117in}{2.219549in}}%
\pgfpathlineto{\pgfqpoint{6.719254in}{2.219912in}}%
\pgfpathlineto{\pgfqpoint{6.719254in}{2.219912in}}%
\pgfpathlineto{\pgfqpoint{6.719254in}{2.219912in}}%
\pgfpathlineto{\pgfqpoint{6.719529in}{2.219265in}}%
\pgfpathlineto{\pgfqpoint{6.719804in}{2.219508in}}%
\pgfpathlineto{\pgfqpoint{6.719804in}{2.219508in}}%
\pgfpathlineto{\pgfqpoint{6.720765in}{2.222614in}}%
\pgfpathlineto{\pgfqpoint{6.721039in}{2.221647in}}%
\pgfpathlineto{\pgfqpoint{6.722272in}{2.218901in}}%
\pgfpathlineto{\pgfqpoint{6.722819in}{2.219120in}}%
\pgfpathlineto{\pgfqpoint{6.725140in}{2.224938in}}%
\pgfpathlineto{\pgfqpoint{6.725277in}{2.223782in}}%
\pgfpathlineto{\pgfqpoint{6.725413in}{2.223461in}}%
\pgfpathlineto{\pgfqpoint{6.725413in}{2.223461in}}%
\pgfpathlineto{\pgfqpoint{6.725413in}{2.223461in}}%
\pgfpathlineto{\pgfqpoint{6.726774in}{2.225884in}}%
\pgfpathlineto{\pgfqpoint{6.727997in}{2.227680in}}%
\pgfpathlineto{\pgfqpoint{6.728132in}{2.226928in}}%
\pgfpathlineto{\pgfqpoint{6.729488in}{2.224938in}}%
\pgfpathlineto{\pgfqpoint{6.729623in}{2.224753in}}%
\pgfpathlineto{\pgfqpoint{6.729623in}{2.224753in}}%
\pgfpathlineto{\pgfqpoint{6.729623in}{2.224753in}}%
\pgfpathlineto{\pgfqpoint{6.729894in}{2.226158in}}%
\pgfpathlineto{\pgfqpoint{6.730570in}{2.225070in}}%
\pgfpathlineto{\pgfqpoint{6.730976in}{2.223766in}}%
\pgfpathlineto{\pgfqpoint{6.731246in}{2.224455in}}%
\pgfpathlineto{\pgfqpoint{6.731246in}{2.224455in}}%
\pgfpathlineto{\pgfqpoint{6.732460in}{2.226528in}}%
\pgfpathlineto{\pgfqpoint{6.732999in}{2.225536in}}%
\pgfpathlineto{\pgfqpoint{6.733404in}{2.226469in}}%
\pgfusepath{stroke}%
\end{pgfscope}%
\begin{pgfscope}%
\pgfpathrectangle{\pgfqpoint{1.000000in}{0.300000in}}{\pgfqpoint{6.200000in}{2.400000in}} %
\pgfusepath{clip}%
\pgfsetrectcap%
\pgfsetroundjoin%
\pgfsetlinewidth{1.003750pt}%
\definecolor{currentstroke}{rgb}{0.000000,0.500000,0.000000}%
\pgfsetstrokecolor{currentstroke}%
\pgfsetdash{}{0pt}%
\pgfpathmoveto{\pgfqpoint{1.000000in}{0.479596in}}%
\pgfpathlineto{\pgfqpoint{1.466596in}{0.659706in}}%
\pgfpathlineto{\pgfqpoint{1.739538in}{0.747768in}}%
\pgfpathlineto{\pgfqpoint{1.933193in}{0.797601in}}%
\pgfpathlineto{\pgfqpoint{2.550000in}{0.933805in}}%
\pgfpathlineto{\pgfqpoint{2.672731in}{0.960515in}}%
\pgfpathlineto{\pgfqpoint{2.907196in}{1.014980in}}%
\pgfpathlineto{\pgfqpoint{3.110678in}{1.066823in}}%
\pgfpathlineto{\pgfqpoint{3.193209in}{1.089701in}}%
\pgfpathlineto{\pgfqpoint{3.448665in}{1.163300in}}%
\pgfpathlineto{\pgfqpoint{3.516036in}{1.180984in}}%
\pgfpathlineto{\pgfqpoint{3.633404in}{1.213700in}}%
\pgfpathlineto{\pgfqpoint{3.685210in}{1.226580in}}%
\pgfpathlineto{\pgfqpoint{3.767261in}{1.251933in}}%
\pgfpathlineto{\pgfqpoint{3.820293in}{1.268300in}}%
\pgfpathlineto{\pgfqpoint{3.949789in}{1.302819in}}%
\pgfpathlineto{\pgfqpoint{4.199910in}{1.379119in}}%
\pgfpathlineto{\pgfqpoint{4.255574in}{1.393159in}}%
\pgfpathlineto{\pgfqpoint{4.271414in}{1.399346in}}%
\pgfpathlineto{\pgfqpoint{4.302017in}{1.408146in}}%
\pgfpathlineto{\pgfqpoint{4.331290in}{1.416301in}}%
\pgfpathlineto{\pgfqpoint{4.359342in}{1.425235in}}%
\pgfpathlineto{\pgfqpoint{4.372941in}{1.428946in}}%
\pgfpathlineto{\pgfqpoint{4.395014in}{1.434524in}}%
\pgfpathlineto{\pgfqpoint{4.453224in}{1.452799in}}%
\pgfpathlineto{\pgfqpoint{4.517745in}{1.471600in}}%
\pgfpathlineto{\pgfqpoint{4.528516in}{1.475431in}}%
\pgfpathlineto{\pgfqpoint{4.539117in}{1.479185in}}%
\pgfpathlineto{\pgfqpoint{4.573295in}{1.488764in}}%
\pgfpathlineto{\pgfqpoint{4.596227in}{1.496721in}}%
\pgfpathlineto{\pgfqpoint{4.686517in}{1.523112in}}%
\pgfpathlineto{\pgfqpoint{4.703207in}{1.528736in}}%
\pgfpathlineto{\pgfqpoint{4.722171in}{1.534127in}}%
\pgfpathlineto{\pgfqpoint{4.732772in}{1.536441in}}%
\pgfpathlineto{\pgfqpoint{4.761092in}{1.545700in}}%
\pgfpathlineto{\pgfqpoint{4.783409in}{1.552366in}}%
\pgfpathlineto{\pgfqpoint{4.835035in}{1.568195in}}%
\pgfpathlineto{\pgfqpoint{4.844011in}{1.570761in}}%
\pgfpathlineto{\pgfqpoint{4.885083in}{1.583742in}}%
\pgfpathlineto{\pgfqpoint{4.949051in}{1.603028in}}%
\pgfpathlineto{\pgfqpoint{4.956636in}{1.605310in}}%
\pgfpathlineto{\pgfqpoint{4.978891in}{1.612508in}}%
\pgfpathlineto{\pgfqpoint{4.986149in}{1.614374in}}%
\pgfpathlineto{\pgfqpoint{4.993329in}{1.617488in}}%
\pgfpathlineto{\pgfqpoint{5.000434in}{1.619183in}}%
\pgfpathlineto{\pgfqpoint{5.009210in}{1.622085in}}%
\pgfpathlineto{\pgfqpoint{5.043215in}{1.633270in}}%
\pgfpathlineto{\pgfqpoint{5.051455in}{1.635868in}}%
\pgfpathlineto{\pgfqpoint{5.062823in}{1.639198in}}%
\pgfpathlineto{\pgfqpoint{5.074003in}{1.641954in}}%
\pgfpathlineto{\pgfqpoint{5.088109in}{1.647035in}}%
\pgfpathlineto{\pgfqpoint{5.106469in}{1.652707in}}%
\pgfpathlineto{\pgfqpoint{5.147456in}{1.665077in}}%
\pgfpathlineto{\pgfqpoint{5.153113in}{1.667082in}}%
\pgfpathlineto{\pgfqpoint{5.161510in}{1.669145in}}%
\pgfpathlineto{\pgfqpoint{5.252440in}{1.697068in}}%
\pgfpathlineto{\pgfqpoint{5.259691in}{1.699665in}}%
\pgfpathlineto{\pgfqpoint{5.364146in}{1.731708in}}%
\pgfpathlineto{\pgfqpoint{5.369272in}{1.733604in}}%
\pgfpathlineto{\pgfqpoint{5.379410in}{1.736908in}}%
\pgfpathlineto{\pgfqpoint{5.382422in}{1.737538in}}%
\pgfpathlineto{\pgfqpoint{5.392366in}{1.741402in}}%
\pgfpathlineto{\pgfqpoint{5.398262in}{1.742668in}}%
\pgfpathlineto{\pgfqpoint{5.425115in}{1.751773in}}%
\pgfpathlineto{\pgfqpoint{5.430733in}{1.753310in}}%
\pgfpathlineto{\pgfqpoint{5.438150in}{1.755758in}}%
\pgfpathlineto{\pgfqpoint{5.459032in}{1.761687in}}%
\pgfpathlineto{\pgfqpoint{5.465261in}{1.763580in}}%
\pgfpathlineto{\pgfqpoint{5.485331in}{1.770267in}}%
\pgfpathlineto{\pgfqpoint{5.518870in}{1.780770in}}%
\pgfpathlineto{\pgfqpoint{5.522947in}{1.782075in}}%
\pgfpathlineto{\pgfqpoint{5.537424in}{1.786760in}}%
\pgfpathlineto{\pgfqpoint{5.542971in}{1.787668in}}%
\pgfpathlineto{\pgfqpoint{5.549255in}{1.789657in}}%
\pgfpathlineto{\pgfqpoint{5.553930in}{1.791037in}}%
\pgfpathlineto{\pgfqpoint{5.580570in}{1.799782in}}%
\pgfpathlineto{\pgfqpoint{5.587992in}{1.802328in}}%
\pgfpathlineto{\pgfqpoint{5.615472in}{1.810782in}}%
\pgfpathlineto{\pgfqpoint{5.617594in}{1.810897in}}%
\pgfpathlineto{\pgfqpoint{5.623221in}{1.812565in}}%
\pgfpathlineto{\pgfqpoint{5.627411in}{1.813759in}}%
\pgfpathlineto{\pgfqpoint{5.646626in}{1.819923in}}%
\pgfpathlineto{\pgfqpoint{5.652687in}{1.822527in}}%
\pgfpathlineto{\pgfqpoint{5.658695in}{1.824426in}}%
\pgfpathlineto{\pgfqpoint{5.665307in}{1.826128in}}%
\pgfpathlineto{\pgfqpoint{5.669244in}{1.827519in}}%
\pgfpathlineto{\pgfqpoint{5.678986in}{1.830939in}}%
\pgfpathlineto{\pgfqpoint{5.697430in}{1.835807in}}%
\pgfpathlineto{\pgfqpoint{5.701807in}{1.837529in}}%
\pgfpathlineto{\pgfqpoint{5.722068in}{1.844395in}}%
\pgfpathlineto{\pgfqpoint{5.728688in}{1.845831in}}%
\pgfpathlineto{\pgfqpoint{5.732272in}{1.847337in}}%
\pgfpathlineto{\pgfqpoint{5.738203in}{1.849341in}}%
\pgfpathlineto{\pgfqpoint{5.742910in}{1.850683in}}%
\pgfpathlineto{\pgfqpoint{5.745835in}{1.851165in}}%
\pgfpathlineto{\pgfqpoint{5.749329in}{1.852049in}}%
\pgfpathlineto{\pgfqpoint{5.776646in}{1.860705in}}%
\pgfpathlineto{\pgfqpoint{5.780539in}{1.862226in}}%
\pgfpathlineto{\pgfqpoint{5.792084in}{1.865512in}}%
\pgfpathlineto{\pgfqpoint{5.794804in}{1.866629in}}%
\pgfpathlineto{\pgfqpoint{5.806108in}{1.869476in}}%
\pgfpathlineto{\pgfqpoint{5.809835in}{1.870446in}}%
\pgfpathlineto{\pgfqpoint{5.813013in}{1.871819in}}%
\pgfpathlineto{\pgfqpoint{5.817226in}{1.873008in}}%
\pgfpathlineto{\pgfqpoint{5.820892in}{1.873906in}}%
\pgfpathlineto{\pgfqpoint{5.824018in}{1.874763in}}%
\pgfpathlineto{\pgfqpoint{5.829196in}{1.876692in}}%
\pgfpathlineto{\pgfqpoint{5.841970in}{1.880815in}}%
\pgfpathlineto{\pgfqpoint{5.845504in}{1.882184in}}%
\pgfpathlineto{\pgfqpoint{5.851019in}{1.883818in}}%
\pgfpathlineto{\pgfqpoint{5.853512in}{1.884259in}}%
\pgfpathlineto{\pgfqpoint{5.855995in}{1.884634in}}%
\pgfpathlineto{\pgfqpoint{5.870219in}{1.888887in}}%
\pgfpathlineto{\pgfqpoint{5.873124in}{1.889749in}}%
\pgfpathlineto{\pgfqpoint{5.881767in}{1.893095in}}%
\pgfpathlineto{\pgfqpoint{5.898261in}{1.898248in}}%
\pgfpathlineto{\pgfqpoint{5.900584in}{1.898506in}}%
\pgfpathlineto{\pgfqpoint{5.902438in}{1.898538in}}%
\pgfpathlineto{\pgfqpoint{5.932300in}{1.908398in}}%
\pgfpathlineto{\pgfqpoint{5.941967in}{1.911828in}}%
\pgfpathlineto{\pgfqpoint{5.945882in}{1.912883in}}%
\pgfpathlineto{\pgfqpoint{5.949774in}{1.913403in}}%
\pgfpathlineto{\pgfqpoint{5.953644in}{1.914457in}}%
\pgfpathlineto{\pgfqpoint{5.957492in}{1.916318in}}%
\pgfpathlineto{\pgfqpoint{5.963434in}{1.917881in}}%
\pgfpathlineto{\pgfqpoint{5.968067in}{1.919555in}}%
\pgfpathlineto{\pgfqpoint{5.981364in}{1.924081in}}%
\pgfpathlineto{\pgfqpoint{5.983828in}{1.924877in}}%
\pgfpathlineto{\pgfqpoint{5.987916in}{1.926036in}}%
\pgfpathlineto{\pgfqpoint{5.991573in}{1.926438in}}%
\pgfpathlineto{\pgfqpoint{5.997625in}{1.928497in}}%
\pgfpathlineto{\pgfqpoint{6.015069in}{1.934505in}}%
\pgfpathlineto{\pgfqpoint{6.018193in}{1.935223in}}%
\pgfpathlineto{\pgfqpoint{6.035878in}{1.940450in}}%
\pgfpathlineto{\pgfqpoint{6.038529in}{1.941391in}}%
\pgfpathlineto{\pgfqpoint{6.069913in}{1.950775in}}%
\pgfpathlineto{\pgfqpoint{6.072434in}{1.951633in}}%
\pgfpathlineto{\pgfqpoint{6.075660in}{1.953006in}}%
\pgfpathlineto{\pgfqpoint{6.078160in}{1.953508in}}%
\pgfpathlineto{\pgfqpoint{6.081004in}{1.954721in}}%
\pgfpathlineto{\pgfqpoint{6.084544in}{1.956512in}}%
\pgfpathlineto{\pgfqpoint{6.087713in}{1.957554in}}%
\pgfpathlineto{\pgfqpoint{6.089818in}{1.957861in}}%
\pgfpathlineto{\pgfqpoint{6.094356in}{1.959567in}}%
\pgfpathlineto{\pgfqpoint{6.097479in}{1.960809in}}%
\pgfpathlineto{\pgfqpoint{6.105397in}{1.961883in}}%
\pgfpathlineto{\pgfqpoint{6.112207in}{1.964551in}}%
\pgfpathlineto{\pgfqpoint{6.113560in}{1.965059in}}%
\pgfpathlineto{\pgfqpoint{6.119284in}{1.968015in}}%
\pgfpathlineto{\pgfqpoint{6.126950in}{1.970348in}}%
\pgfpathlineto{\pgfqpoint{6.134531in}{1.972318in}}%
\pgfpathlineto{\pgfqpoint{6.136821in}{1.972738in}}%
\pgfpathlineto{\pgfqpoint{6.169649in}{1.983884in}}%
\pgfpathlineto{\pgfqpoint{6.171202in}{1.984158in}}%
\pgfpathlineto{\pgfqpoint{6.173681in}{1.985355in}}%
\pgfpathlineto{\pgfqpoint{6.176766in}{1.986332in}}%
\pgfpathlineto{\pgfqpoint{6.177996in}{1.986749in}}%
\pgfpathlineto{\pgfqpoint{6.182589in}{1.989081in}}%
\pgfpathlineto{\pgfqpoint{6.186847in}{1.990530in}}%
\pgfpathlineto{\pgfqpoint{6.189873in}{1.991447in}}%
\pgfpathlineto{\pgfqpoint{6.194085in}{1.993247in}}%
\pgfpathlineto{\pgfqpoint{6.211263in}{1.998386in}}%
\pgfpathlineto{\pgfqpoint{6.215345in}{2.000194in}}%
\pgfpathlineto{\pgfqpoint{6.217376in}{2.000495in}}%
\pgfpathlineto{\pgfqpoint{6.219113in}{2.001067in}}%
\pgfpathlineto{\pgfqpoint{6.225155in}{2.003314in}}%
\pgfpathlineto{\pgfqpoint{6.227157in}{2.003341in}}%
\pgfpathlineto{\pgfqpoint{6.231144in}{2.003851in}}%
\pgfpathlineto{\pgfqpoint{6.233694in}{2.005412in}}%
\pgfpathlineto{\pgfqpoint{6.236798in}{2.006291in}}%
\pgfpathlineto{\pgfqpoint{6.240728in}{2.008142in}}%
\pgfpathlineto{\pgfqpoint{6.245748in}{2.009671in}}%
\pgfpathlineto{\pgfqpoint{6.247413in}{2.010169in}}%
\pgfpathlineto{\pgfqpoint{6.252383in}{2.012048in}}%
\pgfpathlineto{\pgfqpoint{6.260585in}{2.014068in}}%
\pgfpathlineto{\pgfqpoint{6.263027in}{2.014847in}}%
\pgfpathlineto{\pgfqpoint{6.265729in}{2.015551in}}%
\pgfpathlineto{\pgfqpoint{6.269226in}{2.017471in}}%
\pgfpathlineto{\pgfqpoint{6.272972in}{2.018469in}}%
\pgfpathlineto{\pgfqpoint{6.274571in}{2.018901in}}%
\pgfpathlineto{\pgfqpoint{6.280929in}{2.020776in}}%
\pgfpathlineto{\pgfqpoint{6.282772in}{2.020628in}}%
\pgfpathlineto{\pgfqpoint{6.307046in}{2.028435in}}%
\pgfpathlineto{\pgfqpoint{6.310588in}{2.028899in}}%
\pgfpathlineto{\pgfqpoint{6.314111in}{2.030325in}}%
\pgfpathlineto{\pgfqpoint{6.315615in}{2.030801in}}%
\pgfpathlineto{\pgfqpoint{6.318364in}{2.032047in}}%
\pgfpathlineto{\pgfqpoint{6.319610in}{2.032656in}}%
\pgfpathlineto{\pgfqpoint{6.322839in}{2.033888in}}%
\pgfpathlineto{\pgfqpoint{6.337785in}{2.037744in}}%
\pgfpathlineto{\pgfqpoint{6.351932in}{2.042716in}}%
\pgfpathlineto{\pgfqpoint{6.353117in}{2.043096in}}%
\pgfpathlineto{\pgfqpoint{6.355482in}{2.044065in}}%
\pgfpathlineto{\pgfqpoint{6.357132in}{2.043649in}}%
\pgfpathlineto{\pgfqpoint{6.368803in}{2.047886in}}%
\pgfpathlineto{\pgfqpoint{6.372035in}{2.049052in}}%
\pgfpathlineto{\pgfqpoint{6.373645in}{2.049603in}}%
\pgfpathlineto{\pgfqpoint{6.379136in}{2.050703in}}%
\pgfpathlineto{\pgfqpoint{6.381184in}{2.051064in}}%
\pgfpathlineto{\pgfqpoint{6.385035in}{2.052453in}}%
\pgfpathlineto{\pgfqpoint{6.388640in}{2.053733in}}%
\pgfpathlineto{\pgfqpoint{6.389987in}{2.053872in}}%
\pgfpathlineto{\pgfqpoint{6.394234in}{2.055730in}}%
\pgfpathlineto{\pgfqpoint{6.396236in}{2.056381in}}%
\pgfpathlineto{\pgfqpoint{6.404187in}{2.057541in}}%
\pgfpathlineto{\pgfqpoint{6.405722in}{2.058381in}}%
\pgfpathlineto{\pgfqpoint{6.408782in}{2.059130in}}%
\pgfpathlineto{\pgfqpoint{6.410741in}{2.060205in}}%
\pgfpathlineto{\pgfqpoint{6.412262in}{2.060617in}}%
\pgfpathlineto{\pgfqpoint{6.416802in}{2.062484in}}%
\pgfpathlineto{\pgfqpoint{6.418524in}{2.062961in}}%
\pgfpathlineto{\pgfqpoint{6.475572in}{2.079222in}}%
\pgfpathlineto{\pgfqpoint{6.477150in}{2.080135in}}%
\pgfpathlineto{\pgfqpoint{6.479510in}{2.080467in}}%
\pgfpathlineto{\pgfqpoint{6.514136in}{2.092148in}}%
\pgfpathlineto{\pgfqpoint{6.515440in}{2.092580in}}%
\pgfpathlineto{\pgfqpoint{6.517299in}{2.092479in}}%
\pgfpathlineto{\pgfqpoint{6.519152in}{2.092870in}}%
\pgfpathlineto{\pgfqpoint{6.520077in}{2.092963in}}%
\pgfpathlineto{\pgfqpoint{6.520262in}{2.093248in}}%
\pgfpathlineto{\pgfqpoint{6.521923in}{2.094245in}}%
\pgfpathlineto{\pgfqpoint{6.522291in}{2.093806in}}%
\pgfpathlineto{\pgfqpoint{6.525049in}{2.094114in}}%
\pgfpathlineto{\pgfqpoint{6.527064in}{2.095345in}}%
\pgfpathlineto{\pgfqpoint{6.530349in}{2.096927in}}%
\pgfpathlineto{\pgfqpoint{6.530713in}{2.096668in}}%
\pgfpathlineto{\pgfqpoint{6.531440in}{2.097211in}}%
\pgfpathlineto{\pgfqpoint{6.535788in}{2.098513in}}%
\pgfpathlineto{\pgfqpoint{6.537050in}{2.098731in}}%
\pgfpathlineto{\pgfqpoint{6.538671in}{2.099513in}}%
\pgfpathlineto{\pgfqpoint{6.539928in}{2.100138in}}%
\pgfpathlineto{\pgfqpoint{6.541362in}{2.100056in}}%
\pgfpathlineto{\pgfqpoint{6.550434in}{2.103519in}}%
\pgfpathlineto{\pgfqpoint{6.550964in}{2.104543in}}%
\pgfpathlineto{\pgfqpoint{6.551669in}{2.104323in}}%
\pgfpathlineto{\pgfqpoint{6.553431in}{2.104227in}}%
\pgfpathlineto{\pgfqpoint{6.558163in}{2.105831in}}%
\pgfpathlineto{\pgfqpoint{6.559559in}{2.105789in}}%
\pgfpathlineto{\pgfqpoint{6.561474in}{2.106198in}}%
\pgfpathlineto{\pgfqpoint{6.563036in}{2.106924in}}%
\pgfpathlineto{\pgfqpoint{6.564422in}{2.107370in}}%
\pgfpathlineto{\pgfqpoint{6.567874in}{2.109399in}}%
\pgfpathlineto{\pgfqpoint{6.568390in}{2.110566in}}%
\pgfpathlineto{\pgfqpoint{6.569078in}{2.110338in}}%
\pgfpathlineto{\pgfqpoint{6.570623in}{2.110155in}}%
\pgfpathlineto{\pgfqpoint{6.572506in}{2.110885in}}%
\pgfpathlineto{\pgfqpoint{6.573873in}{2.111213in}}%
\pgfpathlineto{\pgfqpoint{6.578634in}{2.111872in}}%
\pgfpathlineto{\pgfqpoint{6.580157in}{2.112620in}}%
\pgfpathlineto{\pgfqpoint{6.583361in}{2.113910in}}%
\pgfpathlineto{\pgfqpoint{6.587387in}{2.116166in}}%
\pgfpathlineto{\pgfqpoint{6.588557in}{2.116625in}}%
\pgfpathlineto{\pgfqpoint{6.590890in}{2.117380in}}%
\pgfpathlineto{\pgfqpoint{6.591056in}{2.117061in}}%
\pgfpathlineto{\pgfqpoint{6.591555in}{2.117607in}}%
\pgfpathlineto{\pgfqpoint{6.591555in}{2.117607in}}%
\pgfpathlineto{\pgfqpoint{6.592718in}{2.117558in}}%
\pgfpathlineto{\pgfqpoint{6.594375in}{2.117055in}}%
\pgfpathlineto{\pgfqpoint{6.596523in}{2.118472in}}%
\pgfpathlineto{\pgfqpoint{6.598994in}{2.119406in}}%
\pgfpathlineto{\pgfqpoint{6.601455in}{2.121031in}}%
\pgfpathlineto{\pgfqpoint{6.601946in}{2.122079in}}%
\pgfpathlineto{\pgfqpoint{6.602600in}{2.121474in}}%
\pgfpathlineto{\pgfqpoint{6.604233in}{2.121620in}}%
\pgfpathlineto{\pgfqpoint{6.607001in}{2.122978in}}%
\pgfpathlineto{\pgfqpoint{6.607650in}{2.122634in}}%
\pgfpathlineto{\pgfqpoint{6.608137in}{2.123397in}}%
\pgfpathlineto{\pgfqpoint{6.608299in}{2.123608in}}%
\pgfpathlineto{\pgfqpoint{6.608623in}{2.123156in}}%
\pgfpathlineto{\pgfqpoint{6.608623in}{2.123156in}}%
\pgfpathlineto{\pgfqpoint{6.609756in}{2.122990in}}%
\pgfpathlineto{\pgfqpoint{6.625111in}{2.128013in}}%
\pgfpathlineto{\pgfqpoint{6.627006in}{2.127666in}}%
\pgfpathlineto{\pgfqpoint{6.629681in}{2.129842in}}%
\pgfpathlineto{\pgfqpoint{6.629838in}{2.129599in}}%
\pgfpathlineto{\pgfqpoint{6.630779in}{2.129583in}}%
\pgfpathlineto{\pgfqpoint{6.630936in}{2.129777in}}%
\pgfpathlineto{\pgfqpoint{6.632658in}{2.131126in}}%
\pgfpathlineto{\pgfqpoint{6.633908in}{2.132723in}}%
\pgfpathlineto{\pgfqpoint{6.634376in}{2.131801in}}%
\pgfpathlineto{\pgfqpoint{6.640277in}{2.133117in}}%
\pgfpathlineto{\pgfqpoint{6.641050in}{2.132653in}}%
\pgfpathlineto{\pgfqpoint{6.641358in}{2.133099in}}%
\pgfpathlineto{\pgfqpoint{6.642900in}{2.132895in}}%
\pgfpathlineto{\pgfqpoint{6.643054in}{2.133103in}}%
\pgfpathlineto{\pgfqpoint{6.645206in}{2.134478in}}%
\pgfpathlineto{\pgfqpoint{6.645360in}{2.134181in}}%
\pgfpathlineto{\pgfqpoint{6.645667in}{2.133964in}}%
\pgfpathlineto{\pgfqpoint{6.646433in}{2.134449in}}%
\pgfpathlineto{\pgfqpoint{6.649185in}{2.136144in}}%
\pgfpathlineto{\pgfqpoint{6.650405in}{2.136674in}}%
\pgfpathlineto{\pgfqpoint{6.653292in}{2.136863in}}%
\pgfpathlineto{\pgfqpoint{6.654958in}{2.137517in}}%
\pgfpathlineto{\pgfqpoint{6.655260in}{2.138043in}}%
\pgfpathlineto{\pgfqpoint{6.655412in}{2.137449in}}%
\pgfpathlineto{\pgfqpoint{6.655412in}{2.137449in}}%
\pgfpathlineto{\pgfqpoint{6.656922in}{2.136413in}}%
\pgfpathlineto{\pgfqpoint{6.659781in}{2.138675in}}%
\pgfpathlineto{\pgfqpoint{6.664272in}{2.140814in}}%
\pgfpathlineto{\pgfqpoint{6.664421in}{2.141100in}}%
\pgfpathlineto{\pgfqpoint{6.664421in}{2.141100in}}%
\pgfpathlineto{\pgfqpoint{6.664421in}{2.141100in}}%
\pgfpathlineto{\pgfqpoint{6.664720in}{2.140339in}}%
\pgfpathlineto{\pgfqpoint{6.665465in}{2.140970in}}%
\pgfpathlineto{\pgfqpoint{6.670213in}{2.141699in}}%
\pgfpathlineto{\pgfqpoint{6.670361in}{2.140978in}}%
\pgfpathlineto{\pgfqpoint{6.671248in}{2.140211in}}%
\pgfpathlineto{\pgfqpoint{6.671543in}{2.140615in}}%
\pgfpathlineto{\pgfqpoint{6.673312in}{2.140832in}}%
\pgfpathlineto{\pgfqpoint{6.674782in}{2.142776in}}%
\pgfpathlineto{\pgfqpoint{6.677275in}{2.142176in}}%
\pgfpathlineto{\pgfqpoint{6.677421in}{2.142559in}}%
\pgfpathlineto{\pgfqpoint{6.682958in}{2.147154in}}%
\pgfpathlineto{\pgfqpoint{6.683248in}{2.146174in}}%
\pgfpathlineto{\pgfqpoint{6.686864in}{2.144579in}}%
\pgfpathlineto{\pgfqpoint{6.687009in}{2.144204in}}%
\pgfpathlineto{\pgfqpoint{6.687297in}{2.144712in}}%
\pgfpathlineto{\pgfqpoint{6.687297in}{2.144712in}}%
\pgfpathlineto{\pgfqpoint{6.689312in}{2.147332in}}%
\pgfpathlineto{\pgfqpoint{6.689744in}{2.146980in}}%
\pgfpathlineto{\pgfqpoint{6.690174in}{2.146246in}}%
\pgfpathlineto{\pgfqpoint{6.690892in}{2.146993in}}%
\pgfpathlineto{\pgfqpoint{6.692754in}{2.147797in}}%
\pgfpathlineto{\pgfqpoint{6.696035in}{2.151148in}}%
\pgfpathlineto{\pgfqpoint{6.696177in}{2.150874in}}%
\pgfpathlineto{\pgfqpoint{6.696319in}{2.150398in}}%
\pgfpathlineto{\pgfqpoint{6.696462in}{2.150930in}}%
\pgfpathlineto{\pgfqpoint{6.696462in}{2.150930in}}%
\pgfpathlineto{\pgfqpoint{6.697030in}{2.152528in}}%
\pgfpathlineto{\pgfqpoint{6.697598in}{2.150817in}}%
\pgfpathlineto{\pgfqpoint{6.702127in}{2.150412in}}%
\pgfpathlineto{\pgfqpoint{6.703395in}{2.151676in}}%
\pgfpathlineto{\pgfqpoint{6.703536in}{2.151358in}}%
\pgfpathlineto{\pgfqpoint{6.704942in}{2.150389in}}%
\pgfpathlineto{\pgfqpoint{6.707045in}{2.152276in}}%
\pgfpathlineto{\pgfqpoint{6.708444in}{2.153634in}}%
\pgfpathlineto{\pgfqpoint{6.708723in}{2.152846in}}%
\pgfpathlineto{\pgfqpoint{6.709002in}{2.153261in}}%
\pgfpathlineto{\pgfqpoint{6.709002in}{2.153261in}}%
\pgfpathlineto{\pgfqpoint{6.709281in}{2.154393in}}%
\pgfpathlineto{\pgfqpoint{6.709421in}{2.153753in}}%
\pgfpathlineto{\pgfqpoint{6.709421in}{2.153753in}}%
\pgfpathlineto{\pgfqpoint{6.709700in}{2.152917in}}%
\pgfpathlineto{\pgfqpoint{6.709979in}{2.153173in}}%
\pgfpathlineto{\pgfqpoint{6.709979in}{2.153173in}}%
\pgfpathlineto{\pgfqpoint{6.711371in}{2.155587in}}%
\pgfpathlineto{\pgfqpoint{6.712483in}{2.153423in}}%
\pgfpathlineto{\pgfqpoint{6.712761in}{2.153998in}}%
\pgfpathlineto{\pgfqpoint{6.713038in}{2.155324in}}%
\pgfpathlineto{\pgfqpoint{6.714009in}{2.154639in}}%
\pgfpathlineto{\pgfqpoint{6.714286in}{2.155186in}}%
\pgfpathlineto{\pgfqpoint{6.714563in}{2.154536in}}%
\pgfpathlineto{\pgfqpoint{6.714563in}{2.154536in}}%
\pgfpathlineto{\pgfqpoint{6.715532in}{2.154008in}}%
\pgfpathlineto{\pgfqpoint{6.715670in}{2.154439in}}%
\pgfpathlineto{\pgfqpoint{6.716775in}{2.157093in}}%
\pgfpathlineto{\pgfqpoint{6.717189in}{2.156769in}}%
\pgfpathlineto{\pgfqpoint{6.717327in}{2.157056in}}%
\pgfpathlineto{\pgfqpoint{6.717464in}{2.156639in}}%
\pgfpathlineto{\pgfqpoint{6.717464in}{2.156639in}}%
\pgfpathlineto{\pgfqpoint{6.718291in}{2.154857in}}%
\pgfpathlineto{\pgfqpoint{6.718979in}{2.155261in}}%
\pgfpathlineto{\pgfqpoint{6.719117in}{2.154968in}}%
\pgfpathlineto{\pgfqpoint{6.719254in}{2.155293in}}%
\pgfpathlineto{\pgfqpoint{6.719254in}{2.155293in}}%
\pgfpathlineto{\pgfqpoint{6.719941in}{2.157962in}}%
\pgfpathlineto{\pgfqpoint{6.720628in}{2.156795in}}%
\pgfpathlineto{\pgfqpoint{6.722546in}{2.160032in}}%
\pgfpathlineto{\pgfqpoint{6.722819in}{2.160107in}}%
\pgfpathlineto{\pgfqpoint{6.722956in}{2.161124in}}%
\pgfpathlineto{\pgfqpoint{6.723230in}{2.160185in}}%
\pgfpathlineto{\pgfqpoint{6.723230in}{2.160185in}}%
\pgfpathlineto{\pgfqpoint{6.723503in}{2.158690in}}%
\pgfpathlineto{\pgfqpoint{6.723776in}{2.159416in}}%
\pgfpathlineto{\pgfqpoint{6.723776in}{2.159416in}}%
\pgfpathlineto{\pgfqpoint{6.723913in}{2.160772in}}%
\pgfpathlineto{\pgfqpoint{6.724186in}{2.159644in}}%
\pgfpathlineto{\pgfqpoint{6.724186in}{2.159644in}}%
\pgfpathlineto{\pgfqpoint{6.724595in}{2.159205in}}%
\pgfpathlineto{\pgfqpoint{6.725004in}{2.159743in}}%
\pgfpathlineto{\pgfqpoint{6.725004in}{2.159743in}}%
\pgfpathlineto{\pgfqpoint{6.725140in}{2.160135in}}%
\pgfpathlineto{\pgfqpoint{6.725140in}{2.160135in}}%
\pgfpathlineto{\pgfqpoint{6.725140in}{2.160135in}}%
\pgfpathlineto{\pgfqpoint{6.725277in}{2.159035in}}%
\pgfpathlineto{\pgfqpoint{6.725413in}{2.159853in}}%
\pgfpathlineto{\pgfqpoint{6.725413in}{2.159853in}}%
\pgfpathlineto{\pgfqpoint{6.725822in}{2.160487in}}%
\pgfpathlineto{\pgfqpoint{6.726094in}{2.159883in}}%
\pgfpathlineto{\pgfqpoint{6.726094in}{2.159883in}}%
\pgfpathlineto{\pgfqpoint{6.726230in}{2.158965in}}%
\pgfpathlineto{\pgfqpoint{6.726366in}{2.159126in}}%
\pgfpathlineto{\pgfqpoint{6.726366in}{2.159126in}}%
\pgfpathlineto{\pgfqpoint{6.726774in}{2.162268in}}%
\pgfpathlineto{\pgfqpoint{6.727454in}{2.159415in}}%
\pgfpathlineto{\pgfqpoint{6.728404in}{2.157636in}}%
\pgfpathlineto{\pgfqpoint{6.728539in}{2.158889in}}%
\pgfpathlineto{\pgfqpoint{6.730705in}{2.164150in}}%
\pgfpathlineto{\pgfqpoint{6.731516in}{2.160899in}}%
\pgfpathlineto{\pgfqpoint{6.732056in}{2.162660in}}%
\pgfpathlineto{\pgfqpoint{6.732191in}{2.163222in}}%
\pgfpathlineto{\pgfqpoint{6.732595in}{2.162848in}}%
\pgfpathlineto{\pgfqpoint{6.732595in}{2.162848in}}%
\pgfpathlineto{\pgfqpoint{6.732730in}{2.162368in}}%
\pgfpathlineto{\pgfqpoint{6.732865in}{2.162859in}}%
\pgfpathlineto{\pgfqpoint{6.732865in}{2.162859in}}%
\pgfpathlineto{\pgfqpoint{6.733269in}{2.163804in}}%
\pgfpathlineto{\pgfqpoint{6.733404in}{2.163490in}}%
\pgfusepath{stroke}%
\end{pgfscope}%
\begin{pgfscope}%
\pgfpathrectangle{\pgfqpoint{1.000000in}{0.300000in}}{\pgfqpoint{6.200000in}{2.400000in}} %
\pgfusepath{clip}%
\pgfsetrectcap%
\pgfsetroundjoin%
\pgfsetlinewidth{1.003750pt}%
\definecolor{currentstroke}{rgb}{1.000000,0.000000,0.000000}%
\pgfsetstrokecolor{currentstroke}%
\pgfsetdash{}{0pt}%
\pgfpathmoveto{\pgfqpoint{1.000000in}{0.497039in}}%
\pgfpathlineto{\pgfqpoint{1.466596in}{0.692308in}}%
\pgfpathlineto{\pgfqpoint{1.739538in}{0.795276in}}%
\pgfpathlineto{\pgfqpoint{1.933193in}{0.858682in}}%
\pgfpathlineto{\pgfqpoint{2.309902in}{0.971959in}}%
\pgfpathlineto{\pgfqpoint{2.866386in}{1.132900in}}%
\pgfpathlineto{\pgfqpoint{2.945672in}{1.155466in}}%
\pgfpathlineto{\pgfqpoint{3.049440in}{1.187056in}}%
\pgfpathlineto{\pgfqpoint{3.218614in}{1.238008in}}%
\pgfpathlineto{\pgfqpoint{3.332982in}{1.269423in}}%
\pgfpathlineto{\pgfqpoint{3.393305in}{1.287605in}}%
\pgfpathlineto{\pgfqpoint{3.448665in}{1.304469in}}%
\pgfpathlineto{\pgfqpoint{3.799579in}{1.406377in}}%
\pgfpathlineto{\pgfqpoint{3.897309in}{1.435879in}}%
\pgfpathlineto{\pgfqpoint{3.941322in}{1.448895in}}%
\pgfpathlineto{\pgfqpoint{4.043871in}{1.479495in}}%
\pgfpathlineto{\pgfqpoint{4.086400in}{1.493359in}}%
\pgfpathlineto{\pgfqpoint{4.119898in}{1.502007in}}%
\pgfpathlineto{\pgfqpoint{4.524945in}{1.625851in}}%
\pgfpathlineto{\pgfqpoint{4.535602in}{1.628961in}}%
\pgfpathlineto{\pgfqpoint{4.552997in}{1.634833in}}%
\pgfpathlineto{\pgfqpoint{4.563222in}{1.637750in}}%
\pgfpathlineto{\pgfqpoint{4.579927in}{1.641933in}}%
\pgfpathlineto{\pgfqpoint{4.998665in}{1.769457in}}%
\pgfpathlineto{\pgfqpoint{5.007464in}{1.771740in}}%
\pgfpathlineto{\pgfqpoint{5.021308in}{1.775907in}}%
\pgfpathlineto{\pgfqpoint{5.049815in}{1.784149in}}%
\pgfpathlineto{\pgfqpoint{5.081876in}{1.794044in}}%
\pgfpathlineto{\pgfqpoint{5.091204in}{1.796726in}}%
\pgfpathlineto{\pgfqpoint{5.130195in}{1.809168in}}%
\pgfpathlineto{\pgfqpoint{5.241415in}{1.842110in}}%
\pgfpathlineto{\pgfqpoint{5.264483in}{1.849405in}}%
\pgfpathlineto{\pgfqpoint{5.301632in}{1.860818in}}%
\pgfpathlineto{\pgfqpoint{5.305012in}{1.861593in}}%
\pgfpathlineto{\pgfqpoint{5.312833in}{1.863141in}}%
\pgfpathlineto{\pgfqpoint{5.334690in}{1.869678in}}%
\pgfpathlineto{\pgfqpoint{5.358979in}{1.877564in}}%
\pgfpathlineto{\pgfqpoint{5.365174in}{1.879731in}}%
\pgfpathlineto{\pgfqpoint{5.373346in}{1.882373in}}%
\pgfpathlineto{\pgfqpoint{5.380416in}{1.884268in}}%
\pgfpathlineto{\pgfqpoint{5.391378in}{1.887810in}}%
\pgfpathlineto{\pgfqpoint{5.400216in}{1.890299in}}%
\pgfpathlineto{\pgfqpoint{5.407976in}{1.892807in}}%
\pgfpathlineto{\pgfqpoint{5.411823in}{1.893105in}}%
\pgfpathlineto{\pgfqpoint{5.456345in}{1.906656in}}%
\pgfpathlineto{\pgfqpoint{5.460818in}{1.908371in}}%
\pgfpathlineto{\pgfqpoint{5.467030in}{1.910269in}}%
\pgfpathlineto{\pgfqpoint{5.619005in}{1.957172in}}%
\pgfpathlineto{\pgfqpoint{5.624621in}{1.958803in}}%
\pgfpathlineto{\pgfqpoint{5.629496in}{1.960127in}}%
\pgfpathlineto{\pgfqpoint{5.649327in}{1.966088in}}%
\pgfpathlineto{\pgfqpoint{5.653357in}{1.966965in}}%
\pgfpathlineto{\pgfqpoint{5.673808in}{1.973736in}}%
\pgfpathlineto{\pgfqpoint{5.679630in}{1.976203in}}%
\pgfpathlineto{\pgfqpoint{5.712934in}{1.986445in}}%
\pgfpathlineto{\pgfqpoint{5.715992in}{1.986608in}}%
\pgfpathlineto{\pgfqpoint{5.735836in}{1.993214in}}%
\pgfpathlineto{\pgfqpoint{5.741149in}{1.995296in}}%
\pgfpathlineto{\pgfqpoint{5.769357in}{2.003805in}}%
\pgfpathlineto{\pgfqpoint{5.785511in}{2.008552in}}%
\pgfpathlineto{\pgfqpoint{5.793717in}{2.011597in}}%
\pgfpathlineto{\pgfqpoint{5.815123in}{2.018426in}}%
\pgfpathlineto{\pgfqpoint{5.818276in}{2.019400in}}%
\pgfpathlineto{\pgfqpoint{5.830227in}{2.022609in}}%
\pgfpathlineto{\pgfqpoint{5.841970in}{2.027038in}}%
\pgfpathlineto{\pgfqpoint{5.846007in}{2.028141in}}%
\pgfpathlineto{\pgfqpoint{5.876017in}{2.037078in}}%
\pgfpathlineto{\pgfqpoint{5.879377in}{2.037980in}}%
\pgfpathlineto{\pgfqpoint{5.888885in}{2.041443in}}%
\pgfpathlineto{\pgfqpoint{5.906129in}{2.047084in}}%
\pgfpathlineto{\pgfqpoint{5.911628in}{2.048694in}}%
\pgfpathlineto{\pgfqpoint{5.917082in}{2.050500in}}%
\pgfpathlineto{\pgfqpoint{5.924734in}{2.052293in}}%
\pgfpathlineto{\pgfqpoint{5.945448in}{2.059634in}}%
\pgfpathlineto{\pgfqpoint{6.043049in}{2.090130in}}%
\pgfpathlineto{\pgfqpoint{6.047540in}{2.091141in}}%
\pgfpathlineto{\pgfqpoint{6.049774in}{2.092381in}}%
\pgfpathlineto{\pgfqpoint{6.052371in}{2.093284in}}%
\pgfpathlineto{\pgfqpoint{6.056063in}{2.094414in}}%
\pgfpathlineto{\pgfqpoint{6.064116in}{2.096893in}}%
\pgfpathlineto{\pgfqpoint{6.066296in}{2.097719in}}%
\pgfpathlineto{\pgfqpoint{6.069191in}{2.098393in}}%
\pgfpathlineto{\pgfqpoint{6.072434in}{2.099375in}}%
\pgfpathlineto{\pgfqpoint{6.085602in}{2.103313in}}%
\pgfpathlineto{\pgfqpoint{6.089117in}{2.104423in}}%
\pgfpathlineto{\pgfqpoint{6.094008in}{2.106086in}}%
\pgfpathlineto{\pgfqpoint{6.097133in}{2.107807in}}%
\pgfpathlineto{\pgfqpoint{6.174299in}{2.132652in}}%
\pgfpathlineto{\pgfqpoint{6.177073in}{2.133243in}}%
\pgfpathlineto{\pgfqpoint{6.179530in}{2.134044in}}%
\pgfpathlineto{\pgfqpoint{6.181978in}{2.134561in}}%
\pgfpathlineto{\pgfqpoint{6.191681in}{2.138286in}}%
\pgfpathlineto{\pgfqpoint{6.194685in}{2.139681in}}%
\pgfpathlineto{\pgfqpoint{6.197377in}{2.139982in}}%
\pgfpathlineto{\pgfqpoint{6.218245in}{2.146838in}}%
\pgfpathlineto{\pgfqpoint{6.318364in}{2.180251in}}%
\pgfpathlineto{\pgfqpoint{6.320357in}{2.180704in}}%
\pgfpathlineto{\pgfqpoint{6.322839in}{2.181591in}}%
\pgfpathlineto{\pgfqpoint{6.365090in}{2.195834in}}%
\pgfpathlineto{\pgfqpoint{6.367645in}{2.196672in}}%
\pgfpathlineto{\pgfqpoint{6.372955in}{2.198341in}}%
\pgfpathlineto{\pgfqpoint{6.377082in}{2.200538in}}%
\pgfpathlineto{\pgfqpoint{6.383452in}{2.202164in}}%
\pgfpathlineto{\pgfqpoint{6.385938in}{2.203395in}}%
\pgfpathlineto{\pgfqpoint{6.388190in}{2.203301in}}%
\pgfpathlineto{\pgfqpoint{6.396680in}{2.206281in}}%
\pgfpathlineto{\pgfqpoint{6.398454in}{2.207027in}}%
\pgfpathlineto{\pgfqpoint{6.400444in}{2.207682in}}%
\pgfpathlineto{\pgfqpoint{6.401767in}{2.208038in}}%
\pgfpathlineto{\pgfqpoint{6.409871in}{2.210441in}}%
\pgfpathlineto{\pgfqpoint{6.413562in}{2.211562in}}%
\pgfpathlineto{\pgfqpoint{6.430242in}{2.217153in}}%
\pgfpathlineto{\pgfqpoint{6.434663in}{2.217938in}}%
\pgfpathlineto{\pgfqpoint{6.436968in}{2.219095in}}%
\pgfpathlineto{\pgfqpoint{6.438847in}{2.219358in}}%
\pgfpathlineto{\pgfqpoint{6.449399in}{2.223862in}}%
\pgfpathlineto{\pgfqpoint{6.451040in}{2.223801in}}%
\pgfpathlineto{\pgfqpoint{6.455123in}{2.224347in}}%
\pgfpathlineto{\pgfqpoint{6.458372in}{2.225655in}}%
\pgfpathlineto{\pgfqpoint{6.468824in}{2.229473in}}%
\pgfpathlineto{\pgfqpoint{6.472405in}{2.230411in}}%
\pgfpathlineto{\pgfqpoint{6.473792in}{2.230417in}}%
\pgfpathlineto{\pgfqpoint{6.475374in}{2.231303in}}%
\pgfpathlineto{\pgfqpoint{6.477740in}{2.232369in}}%
\pgfpathlineto{\pgfqpoint{6.481861in}{2.234155in}}%
\pgfpathlineto{\pgfqpoint{6.483229in}{2.234577in}}%
\pgfpathlineto{\pgfqpoint{6.484400in}{2.234315in}}%
\pgfpathlineto{\pgfqpoint{6.484594in}{2.234523in}}%
\pgfpathlineto{\pgfqpoint{6.487899in}{2.235700in}}%
\pgfpathlineto{\pgfqpoint{6.488867in}{2.236107in}}%
\pgfpathlineto{\pgfqpoint{6.490607in}{2.236870in}}%
\pgfpathlineto{\pgfqpoint{6.495801in}{2.238429in}}%
\pgfpathlineto{\pgfqpoint{6.497715in}{2.238562in}}%
\pgfpathlineto{\pgfqpoint{6.500194in}{2.240128in}}%
\pgfpathlineto{\pgfqpoint{6.502285in}{2.240955in}}%
\pgfpathlineto{\pgfqpoint{6.506259in}{2.241595in}}%
\pgfpathlineto{\pgfqpoint{6.514323in}{2.244063in}}%
\pgfpathlineto{\pgfqpoint{6.515626in}{2.244475in}}%
\pgfpathlineto{\pgfqpoint{6.538131in}{2.251191in}}%
\pgfpathlineto{\pgfqpoint{6.539389in}{2.251686in}}%
\pgfpathlineto{\pgfqpoint{6.540466in}{2.251502in}}%
\pgfpathlineto{\pgfqpoint{6.540645in}{2.251699in}}%
\pgfpathlineto{\pgfqpoint{6.543151in}{2.252497in}}%
\pgfpathlineto{\pgfqpoint{6.544578in}{2.252838in}}%
\pgfpathlineto{\pgfqpoint{6.545824in}{2.253592in}}%
\pgfpathlineto{\pgfqpoint{6.546180in}{2.253302in}}%
\pgfpathlineto{\pgfqpoint{6.550610in}{2.254579in}}%
\pgfpathlineto{\pgfqpoint{6.551846in}{2.254795in}}%
\pgfpathlineto{\pgfqpoint{6.554134in}{2.255789in}}%
\pgfpathlineto{\pgfqpoint{6.568046in}{2.259637in}}%
\pgfpathlineto{\pgfqpoint{6.569250in}{2.259864in}}%
\pgfpathlineto{\pgfqpoint{6.572164in}{2.260993in}}%
\pgfpathlineto{\pgfqpoint{6.574385in}{2.261659in}}%
\pgfpathlineto{\pgfqpoint{6.574555in}{2.261326in}}%
\pgfpathlineto{\pgfqpoint{6.575747in}{2.261692in}}%
\pgfpathlineto{\pgfqpoint{6.576598in}{2.262053in}}%
\pgfpathlineto{\pgfqpoint{6.576937in}{2.261506in}}%
\pgfpathlineto{\pgfqpoint{6.579480in}{2.262356in}}%
\pgfpathlineto{\pgfqpoint{6.580495in}{2.262987in}}%
\pgfpathlineto{\pgfqpoint{6.580833in}{2.262700in}}%
\pgfpathlineto{\pgfqpoint{6.587053in}{2.264223in}}%
\pgfpathlineto{\pgfqpoint{6.588056in}{2.264886in}}%
\pgfpathlineto{\pgfqpoint{6.589391in}{2.265758in}}%
\pgfpathlineto{\pgfqpoint{6.589724in}{2.265532in}}%
\pgfpathlineto{\pgfqpoint{6.591555in}{2.265608in}}%
\pgfpathlineto{\pgfqpoint{6.592552in}{2.266069in}}%
\pgfpathlineto{\pgfqpoint{6.593381in}{2.266255in}}%
\pgfpathlineto{\pgfqpoint{6.593547in}{2.265880in}}%
\pgfpathlineto{\pgfqpoint{6.594209in}{2.265656in}}%
\pgfpathlineto{\pgfqpoint{6.594540in}{2.265951in}}%
\pgfpathlineto{\pgfqpoint{6.597348in}{2.267362in}}%
\pgfpathlineto{\pgfqpoint{6.598994in}{2.267561in}}%
\pgfpathlineto{\pgfqpoint{6.600307in}{2.268166in}}%
\pgfpathlineto{\pgfqpoint{6.602437in}{2.268418in}}%
\pgfpathlineto{\pgfqpoint{6.609756in}{2.270727in}}%
\pgfpathlineto{\pgfqpoint{6.609918in}{2.270273in}}%
\pgfpathlineto{\pgfqpoint{6.610242in}{2.269786in}}%
\pgfpathlineto{\pgfqpoint{6.610888in}{2.270219in}}%
\pgfpathlineto{\pgfqpoint{6.612984in}{2.271715in}}%
\pgfpathlineto{\pgfqpoint{6.614592in}{2.271477in}}%
\pgfpathlineto{\pgfqpoint{6.616837in}{2.272204in}}%
\pgfpathlineto{\pgfqpoint{6.618756in}{2.272739in}}%
\pgfpathlineto{\pgfqpoint{6.621146in}{2.273749in}}%
\pgfpathlineto{\pgfqpoint{6.622576in}{2.273532in}}%
\pgfpathlineto{\pgfqpoint{6.623686in}{2.274094in}}%
\pgfpathlineto{\pgfqpoint{6.623844in}{2.273952in}}%
\pgfpathlineto{\pgfqpoint{6.624953in}{2.274563in}}%
\pgfpathlineto{\pgfqpoint{6.625743in}{2.275000in}}%
\pgfpathlineto{\pgfqpoint{6.625901in}{2.274494in}}%
\pgfpathlineto{\pgfqpoint{6.626217in}{2.273954in}}%
\pgfpathlineto{\pgfqpoint{6.626848in}{2.274423in}}%
\pgfpathlineto{\pgfqpoint{6.628108in}{2.274722in}}%
\pgfpathlineto{\pgfqpoint{6.630152in}{2.275772in}}%
\pgfpathlineto{\pgfqpoint{6.630936in}{2.275640in}}%
\pgfpathlineto{\pgfqpoint{6.631093in}{2.275884in}}%
\pgfpathlineto{\pgfqpoint{6.631563in}{2.276792in}}%
\pgfpathlineto{\pgfqpoint{6.631876in}{2.276250in}}%
\pgfpathlineto{\pgfqpoint{6.631876in}{2.276250in}}%
\pgfpathlineto{\pgfqpoint{6.632033in}{2.275627in}}%
\pgfpathlineto{\pgfqpoint{6.632971in}{2.276252in}}%
\pgfpathlineto{\pgfqpoint{6.634064in}{2.276792in}}%
\pgfpathlineto{\pgfqpoint{6.634220in}{2.276519in}}%
\pgfpathlineto{\pgfqpoint{6.636090in}{2.276819in}}%
\pgfpathlineto{\pgfqpoint{6.637644in}{2.277894in}}%
\pgfpathlineto{\pgfqpoint{6.638264in}{2.277680in}}%
\pgfpathlineto{\pgfqpoint{6.638729in}{2.278328in}}%
\pgfpathlineto{\pgfqpoint{6.641358in}{2.279088in}}%
\pgfpathlineto{\pgfqpoint{6.641821in}{2.277942in}}%
\pgfpathlineto{\pgfqpoint{6.642592in}{2.278738in}}%
\pgfpathlineto{\pgfqpoint{6.645206in}{2.279635in}}%
\pgfpathlineto{\pgfqpoint{6.651622in}{2.281431in}}%
\pgfpathlineto{\pgfqpoint{6.654655in}{2.282137in}}%
\pgfpathlineto{\pgfqpoint{6.655563in}{2.281478in}}%
\pgfpathlineto{\pgfqpoint{6.655865in}{2.282233in}}%
\pgfpathlineto{\pgfqpoint{6.656620in}{2.282307in}}%
\pgfpathlineto{\pgfqpoint{6.656771in}{2.281878in}}%
\pgfpathlineto{\pgfqpoint{6.656922in}{2.281424in}}%
\pgfpathlineto{\pgfqpoint{6.657826in}{2.282217in}}%
\pgfpathlineto{\pgfqpoint{6.663376in}{2.283856in}}%
\pgfpathlineto{\pgfqpoint{6.665762in}{2.284350in}}%
\pgfpathlineto{\pgfqpoint{6.669178in}{2.285518in}}%
\pgfpathlineto{\pgfqpoint{6.672870in}{2.286023in}}%
\pgfpathlineto{\pgfqpoint{6.673017in}{2.286341in}}%
\pgfpathlineto{\pgfqpoint{6.673165in}{2.285980in}}%
\pgfpathlineto{\pgfqpoint{6.673165in}{2.285980in}}%
\pgfpathlineto{\pgfqpoint{6.673606in}{2.285751in}}%
\pgfpathlineto{\pgfqpoint{6.674194in}{2.286356in}}%
\pgfpathlineto{\pgfqpoint{6.676249in}{2.287045in}}%
\pgfpathlineto{\pgfqpoint{6.678298in}{2.286788in}}%
\pgfpathlineto{\pgfqpoint{6.678590in}{2.287506in}}%
\pgfpathlineto{\pgfqpoint{6.679320in}{2.287033in}}%
\pgfpathlineto{\pgfqpoint{6.679612in}{2.286041in}}%
\pgfpathlineto{\pgfqpoint{6.680486in}{2.286905in}}%
\pgfpathlineto{\pgfqpoint{6.681505in}{2.287784in}}%
\pgfpathlineto{\pgfqpoint{6.682232in}{2.288656in}}%
\pgfpathlineto{\pgfqpoint{6.682667in}{2.288533in}}%
\pgfpathlineto{\pgfqpoint{6.685131in}{2.287809in}}%
\pgfpathlineto{\pgfqpoint{6.685709in}{2.289285in}}%
\pgfpathlineto{\pgfqpoint{6.686287in}{2.288631in}}%
\pgfpathlineto{\pgfqpoint{6.686864in}{2.288141in}}%
\pgfpathlineto{\pgfqpoint{6.687297in}{2.288927in}}%
\pgfpathlineto{\pgfqpoint{6.687585in}{2.289597in}}%
\pgfpathlineto{\pgfqpoint{6.687873in}{2.288742in}}%
\pgfpathlineto{\pgfqpoint{6.687873in}{2.288742in}}%
\pgfpathlineto{\pgfqpoint{6.688593in}{2.288180in}}%
\pgfpathlineto{\pgfqpoint{6.688881in}{2.288912in}}%
\pgfpathlineto{\pgfqpoint{6.691608in}{2.290338in}}%
\pgfpathlineto{\pgfqpoint{6.691752in}{2.290147in}}%
\pgfpathlineto{\pgfqpoint{6.691895in}{2.289788in}}%
\pgfpathlineto{\pgfqpoint{6.692038in}{2.290214in}}%
\pgfpathlineto{\pgfqpoint{6.692038in}{2.290214in}}%
\pgfpathlineto{\pgfqpoint{6.693039in}{2.291058in}}%
\pgfpathlineto{\pgfqpoint{6.693325in}{2.290763in}}%
\pgfpathlineto{\pgfqpoint{6.694039in}{2.289694in}}%
\pgfpathlineto{\pgfqpoint{6.694467in}{2.291079in}}%
\pgfpathlineto{\pgfqpoint{6.695750in}{2.291208in}}%
\pgfpathlineto{\pgfqpoint{6.697314in}{2.291631in}}%
\pgfpathlineto{\pgfqpoint{6.698592in}{2.291839in}}%
\pgfpathlineto{\pgfqpoint{6.698733in}{2.291732in}}%
\pgfpathlineto{\pgfqpoint{6.699017in}{2.291252in}}%
\pgfpathlineto{\pgfqpoint{6.699442in}{2.291606in}}%
\pgfpathlineto{\pgfqpoint{6.700008in}{2.293288in}}%
\pgfpathlineto{\pgfqpoint{6.700574in}{2.291912in}}%
\pgfpathlineto{\pgfqpoint{6.700715in}{2.291184in}}%
\pgfpathlineto{\pgfqpoint{6.701421in}{2.291878in}}%
\pgfpathlineto{\pgfqpoint{6.701421in}{2.291878in}}%
\pgfpathlineto{\pgfqpoint{6.701845in}{2.293825in}}%
\pgfpathlineto{\pgfqpoint{6.702550in}{2.292129in}}%
\pgfpathlineto{\pgfqpoint{6.703536in}{2.293723in}}%
\pgfpathlineto{\pgfqpoint{6.703676in}{2.294322in}}%
\pgfpathlineto{\pgfqpoint{6.704098in}{2.293846in}}%
\pgfpathlineto{\pgfqpoint{6.704098in}{2.293846in}}%
\pgfpathlineto{\pgfqpoint{6.705082in}{2.293502in}}%
\pgfpathlineto{\pgfqpoint{6.705222in}{2.293631in}}%
\pgfpathlineto{\pgfqpoint{6.707185in}{2.295791in}}%
\pgfpathlineto{\pgfqpoint{6.707605in}{2.295392in}}%
\pgfpathlineto{\pgfqpoint{6.707745in}{2.295855in}}%
\pgfpathlineto{\pgfqpoint{6.707745in}{2.295855in}}%
\pgfpathlineto{\pgfqpoint{6.707884in}{2.296096in}}%
\pgfpathlineto{\pgfqpoint{6.708024in}{2.295766in}}%
\pgfpathlineto{\pgfqpoint{6.708024in}{2.295766in}}%
\pgfpathlineto{\pgfqpoint{6.708164in}{2.295348in}}%
\pgfpathlineto{\pgfqpoint{6.708164in}{2.295348in}}%
\pgfpathlineto{\pgfqpoint{6.708164in}{2.295348in}}%
\pgfpathlineto{\pgfqpoint{6.708723in}{2.297120in}}%
\pgfpathlineto{\pgfqpoint{6.709421in}{2.296296in}}%
\pgfpathlineto{\pgfqpoint{6.709700in}{2.295692in}}%
\pgfpathlineto{\pgfqpoint{6.709979in}{2.296607in}}%
\pgfpathlineto{\pgfqpoint{6.709979in}{2.296607in}}%
\pgfpathlineto{\pgfqpoint{6.711093in}{2.297926in}}%
\pgfpathlineto{\pgfqpoint{6.711371in}{2.297576in}}%
\pgfpathlineto{\pgfqpoint{6.713038in}{2.296047in}}%
\pgfpathlineto{\pgfqpoint{6.713177in}{2.296442in}}%
\pgfpathlineto{\pgfqpoint{6.713871in}{2.298722in}}%
\pgfpathlineto{\pgfqpoint{6.714425in}{2.297264in}}%
\pgfpathlineto{\pgfqpoint{6.714563in}{2.296599in}}%
\pgfpathlineto{\pgfqpoint{6.714840in}{2.297346in}}%
\pgfpathlineto{\pgfqpoint{6.714840in}{2.297346in}}%
\pgfpathlineto{\pgfqpoint{6.714978in}{2.297531in}}%
\pgfpathlineto{\pgfqpoint{6.714978in}{2.297531in}}%
\pgfpathlineto{\pgfqpoint{6.714978in}{2.297531in}}%
\pgfpathlineto{\pgfqpoint{6.715117in}{2.297084in}}%
\pgfpathlineto{\pgfqpoint{6.715393in}{2.297530in}}%
\pgfpathlineto{\pgfqpoint{6.715393in}{2.297530in}}%
\pgfpathlineto{\pgfqpoint{6.715808in}{2.300027in}}%
\pgfpathlineto{\pgfqpoint{6.716223in}{2.297609in}}%
\pgfpathlineto{\pgfqpoint{6.716637in}{2.296383in}}%
\pgfpathlineto{\pgfqpoint{6.717051in}{2.297414in}}%
\pgfpathlineto{\pgfqpoint{6.717051in}{2.297414in}}%
\pgfpathlineto{\pgfqpoint{6.717602in}{2.298901in}}%
\pgfpathlineto{\pgfqpoint{6.718429in}{2.298031in}}%
\pgfpathlineto{\pgfqpoint{6.718566in}{2.297780in}}%
\pgfpathlineto{\pgfqpoint{6.718566in}{2.297780in}}%
\pgfpathlineto{\pgfqpoint{6.718566in}{2.297780in}}%
\pgfpathlineto{\pgfqpoint{6.720216in}{2.299884in}}%
\pgfpathlineto{\pgfqpoint{6.721176in}{2.298939in}}%
\pgfpathlineto{\pgfqpoint{6.721313in}{2.299009in}}%
\pgfpathlineto{\pgfqpoint{6.722683in}{2.304020in}}%
\pgfpathlineto{\pgfqpoint{6.723230in}{2.302970in}}%
\pgfpathlineto{\pgfqpoint{6.723503in}{2.302791in}}%
\pgfpathlineto{\pgfqpoint{6.723503in}{2.302791in}}%
\pgfpathlineto{\pgfqpoint{6.723503in}{2.302791in}}%
\pgfpathlineto{\pgfqpoint{6.724868in}{2.305950in}}%
\pgfpathlineto{\pgfqpoint{6.725140in}{2.305648in}}%
\pgfpathlineto{\pgfqpoint{6.726638in}{2.303047in}}%
\pgfpathlineto{\pgfqpoint{6.727725in}{2.306156in}}%
\pgfpathlineto{\pgfqpoint{6.727861in}{2.305689in}}%
\pgfpathlineto{\pgfqpoint{6.728132in}{2.303617in}}%
\pgfpathlineto{\pgfqpoint{6.728539in}{2.305674in}}%
\pgfpathlineto{\pgfqpoint{6.728539in}{2.305674in}}%
\pgfpathlineto{\pgfqpoint{6.728675in}{2.305895in}}%
\pgfpathlineto{\pgfqpoint{6.728675in}{2.305895in}}%
\pgfpathlineto{\pgfqpoint{6.728675in}{2.305895in}}%
\pgfpathlineto{\pgfqpoint{6.728810in}{2.305503in}}%
\pgfpathlineto{\pgfqpoint{6.728810in}{2.305503in}}%
\pgfpathlineto{\pgfqpoint{6.728810in}{2.305503in}}%
\pgfpathlineto{\pgfqpoint{6.729488in}{2.308633in}}%
\pgfpathlineto{\pgfqpoint{6.729894in}{2.305845in}}%
\pgfpathlineto{\pgfqpoint{6.730300in}{2.302788in}}%
\pgfpathlineto{\pgfqpoint{6.730976in}{2.304987in}}%
\pgfpathlineto{\pgfqpoint{6.731246in}{2.305758in}}%
\pgfpathlineto{\pgfqpoint{6.731381in}{2.305057in}}%
\pgfpathlineto{\pgfqpoint{6.731381in}{2.305057in}}%
\pgfpathlineto{\pgfqpoint{6.731786in}{2.303357in}}%
\pgfpathlineto{\pgfqpoint{6.732191in}{2.304877in}}%
\pgfpathlineto{\pgfqpoint{6.732191in}{2.304877in}}%
\pgfpathlineto{\pgfqpoint{6.732730in}{2.306896in}}%
\pgfpathlineto{\pgfqpoint{6.733269in}{2.305633in}}%
\pgfpathlineto{\pgfqpoint{6.733404in}{2.305000in}}%
\pgfpathlineto{\pgfqpoint{6.733404in}{2.305000in}}%
\pgfusepath{stroke}%
\end{pgfscope}%
\begin{pgfscope}%
\pgfpathrectangle{\pgfqpoint{1.000000in}{0.300000in}}{\pgfqpoint{6.200000in}{2.400000in}} %
\pgfusepath{clip}%
\pgfsetbuttcap%
\pgfsetroundjoin%
\pgfsetlinewidth{0.501875pt}%
\definecolor{currentstroke}{rgb}{0.000000,0.000000,0.000000}%
\pgfsetstrokecolor{currentstroke}%
\pgfsetdash{{1.000000pt}{3.000000pt}}{0.000000pt}%
\pgfpathmoveto{\pgfqpoint{1.000000in}{0.300000in}}%
\pgfpathlineto{\pgfqpoint{1.000000in}{2.700000in}}%
\pgfusepath{stroke}%
\end{pgfscope}%
\begin{pgfscope}%
\pgfsetbuttcap%
\pgfsetroundjoin%
\definecolor{currentfill}{rgb}{0.000000,0.000000,0.000000}%
\pgfsetfillcolor{currentfill}%
\pgfsetlinewidth{0.501875pt}%
\definecolor{currentstroke}{rgb}{0.000000,0.000000,0.000000}%
\pgfsetstrokecolor{currentstroke}%
\pgfsetdash{}{0pt}%
\pgfsys@defobject{currentmarker}{\pgfqpoint{0.000000in}{0.000000in}}{\pgfqpoint{0.000000in}{0.055556in}}{%
\pgfpathmoveto{\pgfqpoint{0.000000in}{0.000000in}}%
\pgfpathlineto{\pgfqpoint{0.000000in}{0.055556in}}%
\pgfusepath{stroke,fill}%
}%
\begin{pgfscope}%
\pgfsys@transformshift{1.000000in}{0.300000in}%
\pgfsys@useobject{currentmarker}{}%
\end{pgfscope}%
\end{pgfscope}%
\begin{pgfscope}%
\pgfsetbuttcap%
\pgfsetroundjoin%
\definecolor{currentfill}{rgb}{0.000000,0.000000,0.000000}%
\pgfsetfillcolor{currentfill}%
\pgfsetlinewidth{0.501875pt}%
\definecolor{currentstroke}{rgb}{0.000000,0.000000,0.000000}%
\pgfsetstrokecolor{currentstroke}%
\pgfsetdash{}{0pt}%
\pgfsys@defobject{currentmarker}{\pgfqpoint{0.000000in}{-0.055556in}}{\pgfqpoint{0.000000in}{0.000000in}}{%
\pgfpathmoveto{\pgfqpoint{0.000000in}{0.000000in}}%
\pgfpathlineto{\pgfqpoint{0.000000in}{-0.055556in}}%
\pgfusepath{stroke,fill}%
}%
\begin{pgfscope}%
\pgfsys@transformshift{1.000000in}{2.700000in}%
\pgfsys@useobject{currentmarker}{}%
\end{pgfscope}%
\end{pgfscope}%
\begin{pgfscope}%
\pgftext[left,bottom,x=0.839506in,y=0.104024in,rotate=0.000000]{{\sffamily\fontsize{12.000000}{14.400000}\selectfont \(\displaystyle {10^{-1}}\)}}
%
\end{pgfscope}%
\begin{pgfscope}%
\pgfpathrectangle{\pgfqpoint{1.000000in}{0.300000in}}{\pgfqpoint{6.200000in}{2.400000in}} %
\pgfusepath{clip}%
\pgfsetbuttcap%
\pgfsetroundjoin%
\pgfsetlinewidth{0.501875pt}%
\definecolor{currentstroke}{rgb}{0.000000,0.000000,0.000000}%
\pgfsetstrokecolor{currentstroke}%
\pgfsetdash{{1.000000pt}{3.000000pt}}{0.000000pt}%
\pgfpathmoveto{\pgfqpoint{2.550000in}{0.300000in}}%
\pgfpathlineto{\pgfqpoint{2.550000in}{2.700000in}}%
\pgfusepath{stroke}%
\end{pgfscope}%
\begin{pgfscope}%
\pgfsetbuttcap%
\pgfsetroundjoin%
\definecolor{currentfill}{rgb}{0.000000,0.000000,0.000000}%
\pgfsetfillcolor{currentfill}%
\pgfsetlinewidth{0.501875pt}%
\definecolor{currentstroke}{rgb}{0.000000,0.000000,0.000000}%
\pgfsetstrokecolor{currentstroke}%
\pgfsetdash{}{0pt}%
\pgfsys@defobject{currentmarker}{\pgfqpoint{0.000000in}{0.000000in}}{\pgfqpoint{0.000000in}{0.055556in}}{%
\pgfpathmoveto{\pgfqpoint{0.000000in}{0.000000in}}%
\pgfpathlineto{\pgfqpoint{0.000000in}{0.055556in}}%
\pgfusepath{stroke,fill}%
}%
\begin{pgfscope}%
\pgfsys@transformshift{2.550000in}{0.300000in}%
\pgfsys@useobject{currentmarker}{}%
\end{pgfscope}%
\end{pgfscope}%
\begin{pgfscope}%
\pgfsetbuttcap%
\pgfsetroundjoin%
\definecolor{currentfill}{rgb}{0.000000,0.000000,0.000000}%
\pgfsetfillcolor{currentfill}%
\pgfsetlinewidth{0.501875pt}%
\definecolor{currentstroke}{rgb}{0.000000,0.000000,0.000000}%
\pgfsetstrokecolor{currentstroke}%
\pgfsetdash{}{0pt}%
\pgfsys@defobject{currentmarker}{\pgfqpoint{0.000000in}{-0.055556in}}{\pgfqpoint{0.000000in}{0.000000in}}{%
\pgfpathmoveto{\pgfqpoint{0.000000in}{0.000000in}}%
\pgfpathlineto{\pgfqpoint{0.000000in}{-0.055556in}}%
\pgfusepath{stroke,fill}%
}%
\begin{pgfscope}%
\pgfsys@transformshift{2.550000in}{2.700000in}%
\pgfsys@useobject{currentmarker}{}%
\end{pgfscope}%
\end{pgfscope}%
\begin{pgfscope}%
\pgftext[left,bottom,x=2.435417in,y=0.104024in,rotate=0.000000]{{\sffamily\fontsize{12.000000}{14.400000}\selectfont \(\displaystyle {10^{0}}\)}}
%
\end{pgfscope}%
\begin{pgfscope}%
\pgfpathrectangle{\pgfqpoint{1.000000in}{0.300000in}}{\pgfqpoint{6.200000in}{2.400000in}} %
\pgfusepath{clip}%
\pgfsetbuttcap%
\pgfsetroundjoin%
\pgfsetlinewidth{0.501875pt}%
\definecolor{currentstroke}{rgb}{0.000000,0.000000,0.000000}%
\pgfsetstrokecolor{currentstroke}%
\pgfsetdash{{1.000000pt}{3.000000pt}}{0.000000pt}%
\pgfpathmoveto{\pgfqpoint{4.100000in}{0.300000in}}%
\pgfpathlineto{\pgfqpoint{4.100000in}{2.700000in}}%
\pgfusepath{stroke}%
\end{pgfscope}%
\begin{pgfscope}%
\pgfsetbuttcap%
\pgfsetroundjoin%
\definecolor{currentfill}{rgb}{0.000000,0.000000,0.000000}%
\pgfsetfillcolor{currentfill}%
\pgfsetlinewidth{0.501875pt}%
\definecolor{currentstroke}{rgb}{0.000000,0.000000,0.000000}%
\pgfsetstrokecolor{currentstroke}%
\pgfsetdash{}{0pt}%
\pgfsys@defobject{currentmarker}{\pgfqpoint{0.000000in}{0.000000in}}{\pgfqpoint{0.000000in}{0.055556in}}{%
\pgfpathmoveto{\pgfqpoint{0.000000in}{0.000000in}}%
\pgfpathlineto{\pgfqpoint{0.000000in}{0.055556in}}%
\pgfusepath{stroke,fill}%
}%
\begin{pgfscope}%
\pgfsys@transformshift{4.100000in}{0.300000in}%
\pgfsys@useobject{currentmarker}{}%
\end{pgfscope}%
\end{pgfscope}%
\begin{pgfscope}%
\pgfsetbuttcap%
\pgfsetroundjoin%
\definecolor{currentfill}{rgb}{0.000000,0.000000,0.000000}%
\pgfsetfillcolor{currentfill}%
\pgfsetlinewidth{0.501875pt}%
\definecolor{currentstroke}{rgb}{0.000000,0.000000,0.000000}%
\pgfsetstrokecolor{currentstroke}%
\pgfsetdash{}{0pt}%
\pgfsys@defobject{currentmarker}{\pgfqpoint{0.000000in}{-0.055556in}}{\pgfqpoint{0.000000in}{0.000000in}}{%
\pgfpathmoveto{\pgfqpoint{0.000000in}{0.000000in}}%
\pgfpathlineto{\pgfqpoint{0.000000in}{-0.055556in}}%
\pgfusepath{stroke,fill}%
}%
\begin{pgfscope}%
\pgfsys@transformshift{4.100000in}{2.700000in}%
\pgfsys@useobject{currentmarker}{}%
\end{pgfscope}%
\end{pgfscope}%
\begin{pgfscope}%
\pgftext[left,bottom,x=3.985417in,y=0.104024in,rotate=0.000000]{{\sffamily\fontsize{12.000000}{14.400000}\selectfont \(\displaystyle {10^{1}}\)}}
%
\end{pgfscope}%
\begin{pgfscope}%
\pgfpathrectangle{\pgfqpoint{1.000000in}{0.300000in}}{\pgfqpoint{6.200000in}{2.400000in}} %
\pgfusepath{clip}%
\pgfsetbuttcap%
\pgfsetroundjoin%
\pgfsetlinewidth{0.501875pt}%
\definecolor{currentstroke}{rgb}{0.000000,0.000000,0.000000}%
\pgfsetstrokecolor{currentstroke}%
\pgfsetdash{{1.000000pt}{3.000000pt}}{0.000000pt}%
\pgfpathmoveto{\pgfqpoint{5.650000in}{0.300000in}}%
\pgfpathlineto{\pgfqpoint{5.650000in}{2.700000in}}%
\pgfusepath{stroke}%
\end{pgfscope}%
\begin{pgfscope}%
\pgfsetbuttcap%
\pgfsetroundjoin%
\definecolor{currentfill}{rgb}{0.000000,0.000000,0.000000}%
\pgfsetfillcolor{currentfill}%
\pgfsetlinewidth{0.501875pt}%
\definecolor{currentstroke}{rgb}{0.000000,0.000000,0.000000}%
\pgfsetstrokecolor{currentstroke}%
\pgfsetdash{}{0pt}%
\pgfsys@defobject{currentmarker}{\pgfqpoint{0.000000in}{0.000000in}}{\pgfqpoint{0.000000in}{0.055556in}}{%
\pgfpathmoveto{\pgfqpoint{0.000000in}{0.000000in}}%
\pgfpathlineto{\pgfqpoint{0.000000in}{0.055556in}}%
\pgfusepath{stroke,fill}%
}%
\begin{pgfscope}%
\pgfsys@transformshift{5.650000in}{0.300000in}%
\pgfsys@useobject{currentmarker}{}%
\end{pgfscope}%
\end{pgfscope}%
\begin{pgfscope}%
\pgfsetbuttcap%
\pgfsetroundjoin%
\definecolor{currentfill}{rgb}{0.000000,0.000000,0.000000}%
\pgfsetfillcolor{currentfill}%
\pgfsetlinewidth{0.501875pt}%
\definecolor{currentstroke}{rgb}{0.000000,0.000000,0.000000}%
\pgfsetstrokecolor{currentstroke}%
\pgfsetdash{}{0pt}%
\pgfsys@defobject{currentmarker}{\pgfqpoint{0.000000in}{-0.055556in}}{\pgfqpoint{0.000000in}{0.000000in}}{%
\pgfpathmoveto{\pgfqpoint{0.000000in}{0.000000in}}%
\pgfpathlineto{\pgfqpoint{0.000000in}{-0.055556in}}%
\pgfusepath{stroke,fill}%
}%
\begin{pgfscope}%
\pgfsys@transformshift{5.650000in}{2.700000in}%
\pgfsys@useobject{currentmarker}{}%
\end{pgfscope}%
\end{pgfscope}%
\begin{pgfscope}%
\pgftext[left,bottom,x=5.535417in,y=0.104024in,rotate=0.000000]{{\sffamily\fontsize{12.000000}{14.400000}\selectfont \(\displaystyle {10^{2}}\)}}
%
\end{pgfscope}%
\begin{pgfscope}%
\pgfpathrectangle{\pgfqpoint{1.000000in}{0.300000in}}{\pgfqpoint{6.200000in}{2.400000in}} %
\pgfusepath{clip}%
\pgfsetbuttcap%
\pgfsetroundjoin%
\pgfsetlinewidth{0.501875pt}%
\definecolor{currentstroke}{rgb}{0.000000,0.000000,0.000000}%
\pgfsetstrokecolor{currentstroke}%
\pgfsetdash{{1.000000pt}{3.000000pt}}{0.000000pt}%
\pgfpathmoveto{\pgfqpoint{7.200000in}{0.300000in}}%
\pgfpathlineto{\pgfqpoint{7.200000in}{2.700000in}}%
\pgfusepath{stroke}%
\end{pgfscope}%
\begin{pgfscope}%
\pgfsetbuttcap%
\pgfsetroundjoin%
\definecolor{currentfill}{rgb}{0.000000,0.000000,0.000000}%
\pgfsetfillcolor{currentfill}%
\pgfsetlinewidth{0.501875pt}%
\definecolor{currentstroke}{rgb}{0.000000,0.000000,0.000000}%
\pgfsetstrokecolor{currentstroke}%
\pgfsetdash{}{0pt}%
\pgfsys@defobject{currentmarker}{\pgfqpoint{0.000000in}{0.000000in}}{\pgfqpoint{0.000000in}{0.055556in}}{%
\pgfpathmoveto{\pgfqpoint{0.000000in}{0.000000in}}%
\pgfpathlineto{\pgfqpoint{0.000000in}{0.055556in}}%
\pgfusepath{stroke,fill}%
}%
\begin{pgfscope}%
\pgfsys@transformshift{7.200000in}{0.300000in}%
\pgfsys@useobject{currentmarker}{}%
\end{pgfscope}%
\end{pgfscope}%
\begin{pgfscope}%
\pgfsetbuttcap%
\pgfsetroundjoin%
\definecolor{currentfill}{rgb}{0.000000,0.000000,0.000000}%
\pgfsetfillcolor{currentfill}%
\pgfsetlinewidth{0.501875pt}%
\definecolor{currentstroke}{rgb}{0.000000,0.000000,0.000000}%
\pgfsetstrokecolor{currentstroke}%
\pgfsetdash{}{0pt}%
\pgfsys@defobject{currentmarker}{\pgfqpoint{0.000000in}{-0.055556in}}{\pgfqpoint{0.000000in}{0.000000in}}{%
\pgfpathmoveto{\pgfqpoint{0.000000in}{0.000000in}}%
\pgfpathlineto{\pgfqpoint{0.000000in}{-0.055556in}}%
\pgfusepath{stroke,fill}%
}%
\begin{pgfscope}%
\pgfsys@transformshift{7.200000in}{2.700000in}%
\pgfsys@useobject{currentmarker}{}%
\end{pgfscope}%
\end{pgfscope}%
\begin{pgfscope}%
\pgftext[left,bottom,x=7.085417in,y=0.104024in,rotate=0.000000]{{\sffamily\fontsize{12.000000}{14.400000}\selectfont \(\displaystyle {10^{3}}\)}}
%
\end{pgfscope}%
\begin{pgfscope}%
\pgfsetbuttcap%
\pgfsetroundjoin%
\definecolor{currentfill}{rgb}{0.000000,0.000000,0.000000}%
\pgfsetfillcolor{currentfill}%
\pgfsetlinewidth{0.501875pt}%
\definecolor{currentstroke}{rgb}{0.000000,0.000000,0.000000}%
\pgfsetstrokecolor{currentstroke}%
\pgfsetdash{}{0pt}%
\pgfsys@defobject{currentmarker}{\pgfqpoint{0.000000in}{0.000000in}}{\pgfqpoint{0.000000in}{0.027778in}}{%
\pgfpathmoveto{\pgfqpoint{0.000000in}{0.000000in}}%
\pgfpathlineto{\pgfqpoint{0.000000in}{0.027778in}}%
\pgfusepath{stroke,fill}%
}%
\begin{pgfscope}%
\pgfsys@transformshift{1.466596in}{0.300000in}%
\pgfsys@useobject{currentmarker}{}%
\end{pgfscope}%
\end{pgfscope}%
\begin{pgfscope}%
\pgfsetbuttcap%
\pgfsetroundjoin%
\definecolor{currentfill}{rgb}{0.000000,0.000000,0.000000}%
\pgfsetfillcolor{currentfill}%
\pgfsetlinewidth{0.501875pt}%
\definecolor{currentstroke}{rgb}{0.000000,0.000000,0.000000}%
\pgfsetstrokecolor{currentstroke}%
\pgfsetdash{}{0pt}%
\pgfsys@defobject{currentmarker}{\pgfqpoint{0.000000in}{-0.027778in}}{\pgfqpoint{0.000000in}{0.000000in}}{%
\pgfpathmoveto{\pgfqpoint{0.000000in}{0.000000in}}%
\pgfpathlineto{\pgfqpoint{0.000000in}{-0.027778in}}%
\pgfusepath{stroke,fill}%
}%
\begin{pgfscope}%
\pgfsys@transformshift{1.466596in}{2.700000in}%
\pgfsys@useobject{currentmarker}{}%
\end{pgfscope}%
\end{pgfscope}%
\begin{pgfscope}%
\pgfsetbuttcap%
\pgfsetroundjoin%
\definecolor{currentfill}{rgb}{0.000000,0.000000,0.000000}%
\pgfsetfillcolor{currentfill}%
\pgfsetlinewidth{0.501875pt}%
\definecolor{currentstroke}{rgb}{0.000000,0.000000,0.000000}%
\pgfsetstrokecolor{currentstroke}%
\pgfsetdash{}{0pt}%
\pgfsys@defobject{currentmarker}{\pgfqpoint{0.000000in}{0.000000in}}{\pgfqpoint{0.000000in}{0.027778in}}{%
\pgfpathmoveto{\pgfqpoint{0.000000in}{0.000000in}}%
\pgfpathlineto{\pgfqpoint{0.000000in}{0.027778in}}%
\pgfusepath{stroke,fill}%
}%
\begin{pgfscope}%
\pgfsys@transformshift{1.739538in}{0.300000in}%
\pgfsys@useobject{currentmarker}{}%
\end{pgfscope}%
\end{pgfscope}%
\begin{pgfscope}%
\pgfsetbuttcap%
\pgfsetroundjoin%
\definecolor{currentfill}{rgb}{0.000000,0.000000,0.000000}%
\pgfsetfillcolor{currentfill}%
\pgfsetlinewidth{0.501875pt}%
\definecolor{currentstroke}{rgb}{0.000000,0.000000,0.000000}%
\pgfsetstrokecolor{currentstroke}%
\pgfsetdash{}{0pt}%
\pgfsys@defobject{currentmarker}{\pgfqpoint{0.000000in}{-0.027778in}}{\pgfqpoint{0.000000in}{0.000000in}}{%
\pgfpathmoveto{\pgfqpoint{0.000000in}{0.000000in}}%
\pgfpathlineto{\pgfqpoint{0.000000in}{-0.027778in}}%
\pgfusepath{stroke,fill}%
}%
\begin{pgfscope}%
\pgfsys@transformshift{1.739538in}{2.700000in}%
\pgfsys@useobject{currentmarker}{}%
\end{pgfscope}%
\end{pgfscope}%
\begin{pgfscope}%
\pgfsetbuttcap%
\pgfsetroundjoin%
\definecolor{currentfill}{rgb}{0.000000,0.000000,0.000000}%
\pgfsetfillcolor{currentfill}%
\pgfsetlinewidth{0.501875pt}%
\definecolor{currentstroke}{rgb}{0.000000,0.000000,0.000000}%
\pgfsetstrokecolor{currentstroke}%
\pgfsetdash{}{0pt}%
\pgfsys@defobject{currentmarker}{\pgfqpoint{0.000000in}{0.000000in}}{\pgfqpoint{0.000000in}{0.027778in}}{%
\pgfpathmoveto{\pgfqpoint{0.000000in}{0.000000in}}%
\pgfpathlineto{\pgfqpoint{0.000000in}{0.027778in}}%
\pgfusepath{stroke,fill}%
}%
\begin{pgfscope}%
\pgfsys@transformshift{1.933193in}{0.300000in}%
\pgfsys@useobject{currentmarker}{}%
\end{pgfscope}%
\end{pgfscope}%
\begin{pgfscope}%
\pgfsetbuttcap%
\pgfsetroundjoin%
\definecolor{currentfill}{rgb}{0.000000,0.000000,0.000000}%
\pgfsetfillcolor{currentfill}%
\pgfsetlinewidth{0.501875pt}%
\definecolor{currentstroke}{rgb}{0.000000,0.000000,0.000000}%
\pgfsetstrokecolor{currentstroke}%
\pgfsetdash{}{0pt}%
\pgfsys@defobject{currentmarker}{\pgfqpoint{0.000000in}{-0.027778in}}{\pgfqpoint{0.000000in}{0.000000in}}{%
\pgfpathmoveto{\pgfqpoint{0.000000in}{0.000000in}}%
\pgfpathlineto{\pgfqpoint{0.000000in}{-0.027778in}}%
\pgfusepath{stroke,fill}%
}%
\begin{pgfscope}%
\pgfsys@transformshift{1.933193in}{2.700000in}%
\pgfsys@useobject{currentmarker}{}%
\end{pgfscope}%
\end{pgfscope}%
\begin{pgfscope}%
\pgfsetbuttcap%
\pgfsetroundjoin%
\definecolor{currentfill}{rgb}{0.000000,0.000000,0.000000}%
\pgfsetfillcolor{currentfill}%
\pgfsetlinewidth{0.501875pt}%
\definecolor{currentstroke}{rgb}{0.000000,0.000000,0.000000}%
\pgfsetstrokecolor{currentstroke}%
\pgfsetdash{}{0pt}%
\pgfsys@defobject{currentmarker}{\pgfqpoint{0.000000in}{0.000000in}}{\pgfqpoint{0.000000in}{0.027778in}}{%
\pgfpathmoveto{\pgfqpoint{0.000000in}{0.000000in}}%
\pgfpathlineto{\pgfqpoint{0.000000in}{0.027778in}}%
\pgfusepath{stroke,fill}%
}%
\begin{pgfscope}%
\pgfsys@transformshift{2.083404in}{0.300000in}%
\pgfsys@useobject{currentmarker}{}%
\end{pgfscope}%
\end{pgfscope}%
\begin{pgfscope}%
\pgfsetbuttcap%
\pgfsetroundjoin%
\definecolor{currentfill}{rgb}{0.000000,0.000000,0.000000}%
\pgfsetfillcolor{currentfill}%
\pgfsetlinewidth{0.501875pt}%
\definecolor{currentstroke}{rgb}{0.000000,0.000000,0.000000}%
\pgfsetstrokecolor{currentstroke}%
\pgfsetdash{}{0pt}%
\pgfsys@defobject{currentmarker}{\pgfqpoint{0.000000in}{-0.027778in}}{\pgfqpoint{0.000000in}{0.000000in}}{%
\pgfpathmoveto{\pgfqpoint{0.000000in}{0.000000in}}%
\pgfpathlineto{\pgfqpoint{0.000000in}{-0.027778in}}%
\pgfusepath{stroke,fill}%
}%
\begin{pgfscope}%
\pgfsys@transformshift{2.083404in}{2.700000in}%
\pgfsys@useobject{currentmarker}{}%
\end{pgfscope}%
\end{pgfscope}%
\begin{pgfscope}%
\pgfsetbuttcap%
\pgfsetroundjoin%
\definecolor{currentfill}{rgb}{0.000000,0.000000,0.000000}%
\pgfsetfillcolor{currentfill}%
\pgfsetlinewidth{0.501875pt}%
\definecolor{currentstroke}{rgb}{0.000000,0.000000,0.000000}%
\pgfsetstrokecolor{currentstroke}%
\pgfsetdash{}{0pt}%
\pgfsys@defobject{currentmarker}{\pgfqpoint{0.000000in}{0.000000in}}{\pgfqpoint{0.000000in}{0.027778in}}{%
\pgfpathmoveto{\pgfqpoint{0.000000in}{0.000000in}}%
\pgfpathlineto{\pgfqpoint{0.000000in}{0.027778in}}%
\pgfusepath{stroke,fill}%
}%
\begin{pgfscope}%
\pgfsys@transformshift{2.206134in}{0.300000in}%
\pgfsys@useobject{currentmarker}{}%
\end{pgfscope}%
\end{pgfscope}%
\begin{pgfscope}%
\pgfsetbuttcap%
\pgfsetroundjoin%
\definecolor{currentfill}{rgb}{0.000000,0.000000,0.000000}%
\pgfsetfillcolor{currentfill}%
\pgfsetlinewidth{0.501875pt}%
\definecolor{currentstroke}{rgb}{0.000000,0.000000,0.000000}%
\pgfsetstrokecolor{currentstroke}%
\pgfsetdash{}{0pt}%
\pgfsys@defobject{currentmarker}{\pgfqpoint{0.000000in}{-0.027778in}}{\pgfqpoint{0.000000in}{0.000000in}}{%
\pgfpathmoveto{\pgfqpoint{0.000000in}{0.000000in}}%
\pgfpathlineto{\pgfqpoint{0.000000in}{-0.027778in}}%
\pgfusepath{stroke,fill}%
}%
\begin{pgfscope}%
\pgfsys@transformshift{2.206134in}{2.700000in}%
\pgfsys@useobject{currentmarker}{}%
\end{pgfscope}%
\end{pgfscope}%
\begin{pgfscope}%
\pgfsetbuttcap%
\pgfsetroundjoin%
\definecolor{currentfill}{rgb}{0.000000,0.000000,0.000000}%
\pgfsetfillcolor{currentfill}%
\pgfsetlinewidth{0.501875pt}%
\definecolor{currentstroke}{rgb}{0.000000,0.000000,0.000000}%
\pgfsetstrokecolor{currentstroke}%
\pgfsetdash{}{0pt}%
\pgfsys@defobject{currentmarker}{\pgfqpoint{0.000000in}{0.000000in}}{\pgfqpoint{0.000000in}{0.027778in}}{%
\pgfpathmoveto{\pgfqpoint{0.000000in}{0.000000in}}%
\pgfpathlineto{\pgfqpoint{0.000000in}{0.027778in}}%
\pgfusepath{stroke,fill}%
}%
\begin{pgfscope}%
\pgfsys@transformshift{2.309902in}{0.300000in}%
\pgfsys@useobject{currentmarker}{}%
\end{pgfscope}%
\end{pgfscope}%
\begin{pgfscope}%
\pgfsetbuttcap%
\pgfsetroundjoin%
\definecolor{currentfill}{rgb}{0.000000,0.000000,0.000000}%
\pgfsetfillcolor{currentfill}%
\pgfsetlinewidth{0.501875pt}%
\definecolor{currentstroke}{rgb}{0.000000,0.000000,0.000000}%
\pgfsetstrokecolor{currentstroke}%
\pgfsetdash{}{0pt}%
\pgfsys@defobject{currentmarker}{\pgfqpoint{0.000000in}{-0.027778in}}{\pgfqpoint{0.000000in}{0.000000in}}{%
\pgfpathmoveto{\pgfqpoint{0.000000in}{0.000000in}}%
\pgfpathlineto{\pgfqpoint{0.000000in}{-0.027778in}}%
\pgfusepath{stroke,fill}%
}%
\begin{pgfscope}%
\pgfsys@transformshift{2.309902in}{2.700000in}%
\pgfsys@useobject{currentmarker}{}%
\end{pgfscope}%
\end{pgfscope}%
\begin{pgfscope}%
\pgfsetbuttcap%
\pgfsetroundjoin%
\definecolor{currentfill}{rgb}{0.000000,0.000000,0.000000}%
\pgfsetfillcolor{currentfill}%
\pgfsetlinewidth{0.501875pt}%
\definecolor{currentstroke}{rgb}{0.000000,0.000000,0.000000}%
\pgfsetstrokecolor{currentstroke}%
\pgfsetdash{}{0pt}%
\pgfsys@defobject{currentmarker}{\pgfqpoint{0.000000in}{0.000000in}}{\pgfqpoint{0.000000in}{0.027778in}}{%
\pgfpathmoveto{\pgfqpoint{0.000000in}{0.000000in}}%
\pgfpathlineto{\pgfqpoint{0.000000in}{0.027778in}}%
\pgfusepath{stroke,fill}%
}%
\begin{pgfscope}%
\pgfsys@transformshift{2.399789in}{0.300000in}%
\pgfsys@useobject{currentmarker}{}%
\end{pgfscope}%
\end{pgfscope}%
\begin{pgfscope}%
\pgfsetbuttcap%
\pgfsetroundjoin%
\definecolor{currentfill}{rgb}{0.000000,0.000000,0.000000}%
\pgfsetfillcolor{currentfill}%
\pgfsetlinewidth{0.501875pt}%
\definecolor{currentstroke}{rgb}{0.000000,0.000000,0.000000}%
\pgfsetstrokecolor{currentstroke}%
\pgfsetdash{}{0pt}%
\pgfsys@defobject{currentmarker}{\pgfqpoint{0.000000in}{-0.027778in}}{\pgfqpoint{0.000000in}{0.000000in}}{%
\pgfpathmoveto{\pgfqpoint{0.000000in}{0.000000in}}%
\pgfpathlineto{\pgfqpoint{0.000000in}{-0.027778in}}%
\pgfusepath{stroke,fill}%
}%
\begin{pgfscope}%
\pgfsys@transformshift{2.399789in}{2.700000in}%
\pgfsys@useobject{currentmarker}{}%
\end{pgfscope}%
\end{pgfscope}%
\begin{pgfscope}%
\pgfsetbuttcap%
\pgfsetroundjoin%
\definecolor{currentfill}{rgb}{0.000000,0.000000,0.000000}%
\pgfsetfillcolor{currentfill}%
\pgfsetlinewidth{0.501875pt}%
\definecolor{currentstroke}{rgb}{0.000000,0.000000,0.000000}%
\pgfsetstrokecolor{currentstroke}%
\pgfsetdash{}{0pt}%
\pgfsys@defobject{currentmarker}{\pgfqpoint{0.000000in}{0.000000in}}{\pgfqpoint{0.000000in}{0.027778in}}{%
\pgfpathmoveto{\pgfqpoint{0.000000in}{0.000000in}}%
\pgfpathlineto{\pgfqpoint{0.000000in}{0.027778in}}%
\pgfusepath{stroke,fill}%
}%
\begin{pgfscope}%
\pgfsys@transformshift{2.479076in}{0.300000in}%
\pgfsys@useobject{currentmarker}{}%
\end{pgfscope}%
\end{pgfscope}%
\begin{pgfscope}%
\pgfsetbuttcap%
\pgfsetroundjoin%
\definecolor{currentfill}{rgb}{0.000000,0.000000,0.000000}%
\pgfsetfillcolor{currentfill}%
\pgfsetlinewidth{0.501875pt}%
\definecolor{currentstroke}{rgb}{0.000000,0.000000,0.000000}%
\pgfsetstrokecolor{currentstroke}%
\pgfsetdash{}{0pt}%
\pgfsys@defobject{currentmarker}{\pgfqpoint{0.000000in}{-0.027778in}}{\pgfqpoint{0.000000in}{0.000000in}}{%
\pgfpathmoveto{\pgfqpoint{0.000000in}{0.000000in}}%
\pgfpathlineto{\pgfqpoint{0.000000in}{-0.027778in}}%
\pgfusepath{stroke,fill}%
}%
\begin{pgfscope}%
\pgfsys@transformshift{2.479076in}{2.700000in}%
\pgfsys@useobject{currentmarker}{}%
\end{pgfscope}%
\end{pgfscope}%
\begin{pgfscope}%
\pgfsetbuttcap%
\pgfsetroundjoin%
\definecolor{currentfill}{rgb}{0.000000,0.000000,0.000000}%
\pgfsetfillcolor{currentfill}%
\pgfsetlinewidth{0.501875pt}%
\definecolor{currentstroke}{rgb}{0.000000,0.000000,0.000000}%
\pgfsetstrokecolor{currentstroke}%
\pgfsetdash{}{0pt}%
\pgfsys@defobject{currentmarker}{\pgfqpoint{0.000000in}{0.000000in}}{\pgfqpoint{0.000000in}{0.027778in}}{%
\pgfpathmoveto{\pgfqpoint{0.000000in}{0.000000in}}%
\pgfpathlineto{\pgfqpoint{0.000000in}{0.027778in}}%
\pgfusepath{stroke,fill}%
}%
\begin{pgfscope}%
\pgfsys@transformshift{3.016596in}{0.300000in}%
\pgfsys@useobject{currentmarker}{}%
\end{pgfscope}%
\end{pgfscope}%
\begin{pgfscope}%
\pgfsetbuttcap%
\pgfsetroundjoin%
\definecolor{currentfill}{rgb}{0.000000,0.000000,0.000000}%
\pgfsetfillcolor{currentfill}%
\pgfsetlinewidth{0.501875pt}%
\definecolor{currentstroke}{rgb}{0.000000,0.000000,0.000000}%
\pgfsetstrokecolor{currentstroke}%
\pgfsetdash{}{0pt}%
\pgfsys@defobject{currentmarker}{\pgfqpoint{0.000000in}{-0.027778in}}{\pgfqpoint{0.000000in}{0.000000in}}{%
\pgfpathmoveto{\pgfqpoint{0.000000in}{0.000000in}}%
\pgfpathlineto{\pgfqpoint{0.000000in}{-0.027778in}}%
\pgfusepath{stroke,fill}%
}%
\begin{pgfscope}%
\pgfsys@transformshift{3.016596in}{2.700000in}%
\pgfsys@useobject{currentmarker}{}%
\end{pgfscope}%
\end{pgfscope}%
\begin{pgfscope}%
\pgfsetbuttcap%
\pgfsetroundjoin%
\definecolor{currentfill}{rgb}{0.000000,0.000000,0.000000}%
\pgfsetfillcolor{currentfill}%
\pgfsetlinewidth{0.501875pt}%
\definecolor{currentstroke}{rgb}{0.000000,0.000000,0.000000}%
\pgfsetstrokecolor{currentstroke}%
\pgfsetdash{}{0pt}%
\pgfsys@defobject{currentmarker}{\pgfqpoint{0.000000in}{0.000000in}}{\pgfqpoint{0.000000in}{0.027778in}}{%
\pgfpathmoveto{\pgfqpoint{0.000000in}{0.000000in}}%
\pgfpathlineto{\pgfqpoint{0.000000in}{0.027778in}}%
\pgfusepath{stroke,fill}%
}%
\begin{pgfscope}%
\pgfsys@transformshift{3.289538in}{0.300000in}%
\pgfsys@useobject{currentmarker}{}%
\end{pgfscope}%
\end{pgfscope}%
\begin{pgfscope}%
\pgfsetbuttcap%
\pgfsetroundjoin%
\definecolor{currentfill}{rgb}{0.000000,0.000000,0.000000}%
\pgfsetfillcolor{currentfill}%
\pgfsetlinewidth{0.501875pt}%
\definecolor{currentstroke}{rgb}{0.000000,0.000000,0.000000}%
\pgfsetstrokecolor{currentstroke}%
\pgfsetdash{}{0pt}%
\pgfsys@defobject{currentmarker}{\pgfqpoint{0.000000in}{-0.027778in}}{\pgfqpoint{0.000000in}{0.000000in}}{%
\pgfpathmoveto{\pgfqpoint{0.000000in}{0.000000in}}%
\pgfpathlineto{\pgfqpoint{0.000000in}{-0.027778in}}%
\pgfusepath{stroke,fill}%
}%
\begin{pgfscope}%
\pgfsys@transformshift{3.289538in}{2.700000in}%
\pgfsys@useobject{currentmarker}{}%
\end{pgfscope}%
\end{pgfscope}%
\begin{pgfscope}%
\pgfsetbuttcap%
\pgfsetroundjoin%
\definecolor{currentfill}{rgb}{0.000000,0.000000,0.000000}%
\pgfsetfillcolor{currentfill}%
\pgfsetlinewidth{0.501875pt}%
\definecolor{currentstroke}{rgb}{0.000000,0.000000,0.000000}%
\pgfsetstrokecolor{currentstroke}%
\pgfsetdash{}{0pt}%
\pgfsys@defobject{currentmarker}{\pgfqpoint{0.000000in}{0.000000in}}{\pgfqpoint{0.000000in}{0.027778in}}{%
\pgfpathmoveto{\pgfqpoint{0.000000in}{0.000000in}}%
\pgfpathlineto{\pgfqpoint{0.000000in}{0.027778in}}%
\pgfusepath{stroke,fill}%
}%
\begin{pgfscope}%
\pgfsys@transformshift{3.483193in}{0.300000in}%
\pgfsys@useobject{currentmarker}{}%
\end{pgfscope}%
\end{pgfscope}%
\begin{pgfscope}%
\pgfsetbuttcap%
\pgfsetroundjoin%
\definecolor{currentfill}{rgb}{0.000000,0.000000,0.000000}%
\pgfsetfillcolor{currentfill}%
\pgfsetlinewidth{0.501875pt}%
\definecolor{currentstroke}{rgb}{0.000000,0.000000,0.000000}%
\pgfsetstrokecolor{currentstroke}%
\pgfsetdash{}{0pt}%
\pgfsys@defobject{currentmarker}{\pgfqpoint{0.000000in}{-0.027778in}}{\pgfqpoint{0.000000in}{0.000000in}}{%
\pgfpathmoveto{\pgfqpoint{0.000000in}{0.000000in}}%
\pgfpathlineto{\pgfqpoint{0.000000in}{-0.027778in}}%
\pgfusepath{stroke,fill}%
}%
\begin{pgfscope}%
\pgfsys@transformshift{3.483193in}{2.700000in}%
\pgfsys@useobject{currentmarker}{}%
\end{pgfscope}%
\end{pgfscope}%
\begin{pgfscope}%
\pgfsetbuttcap%
\pgfsetroundjoin%
\definecolor{currentfill}{rgb}{0.000000,0.000000,0.000000}%
\pgfsetfillcolor{currentfill}%
\pgfsetlinewidth{0.501875pt}%
\definecolor{currentstroke}{rgb}{0.000000,0.000000,0.000000}%
\pgfsetstrokecolor{currentstroke}%
\pgfsetdash{}{0pt}%
\pgfsys@defobject{currentmarker}{\pgfqpoint{0.000000in}{0.000000in}}{\pgfqpoint{0.000000in}{0.027778in}}{%
\pgfpathmoveto{\pgfqpoint{0.000000in}{0.000000in}}%
\pgfpathlineto{\pgfqpoint{0.000000in}{0.027778in}}%
\pgfusepath{stroke,fill}%
}%
\begin{pgfscope}%
\pgfsys@transformshift{3.633404in}{0.300000in}%
\pgfsys@useobject{currentmarker}{}%
\end{pgfscope}%
\end{pgfscope}%
\begin{pgfscope}%
\pgfsetbuttcap%
\pgfsetroundjoin%
\definecolor{currentfill}{rgb}{0.000000,0.000000,0.000000}%
\pgfsetfillcolor{currentfill}%
\pgfsetlinewidth{0.501875pt}%
\definecolor{currentstroke}{rgb}{0.000000,0.000000,0.000000}%
\pgfsetstrokecolor{currentstroke}%
\pgfsetdash{}{0pt}%
\pgfsys@defobject{currentmarker}{\pgfqpoint{0.000000in}{-0.027778in}}{\pgfqpoint{0.000000in}{0.000000in}}{%
\pgfpathmoveto{\pgfqpoint{0.000000in}{0.000000in}}%
\pgfpathlineto{\pgfqpoint{0.000000in}{-0.027778in}}%
\pgfusepath{stroke,fill}%
}%
\begin{pgfscope}%
\pgfsys@transformshift{3.633404in}{2.700000in}%
\pgfsys@useobject{currentmarker}{}%
\end{pgfscope}%
\end{pgfscope}%
\begin{pgfscope}%
\pgfsetbuttcap%
\pgfsetroundjoin%
\definecolor{currentfill}{rgb}{0.000000,0.000000,0.000000}%
\pgfsetfillcolor{currentfill}%
\pgfsetlinewidth{0.501875pt}%
\definecolor{currentstroke}{rgb}{0.000000,0.000000,0.000000}%
\pgfsetstrokecolor{currentstroke}%
\pgfsetdash{}{0pt}%
\pgfsys@defobject{currentmarker}{\pgfqpoint{0.000000in}{0.000000in}}{\pgfqpoint{0.000000in}{0.027778in}}{%
\pgfpathmoveto{\pgfqpoint{0.000000in}{0.000000in}}%
\pgfpathlineto{\pgfqpoint{0.000000in}{0.027778in}}%
\pgfusepath{stroke,fill}%
}%
\begin{pgfscope}%
\pgfsys@transformshift{3.756134in}{0.300000in}%
\pgfsys@useobject{currentmarker}{}%
\end{pgfscope}%
\end{pgfscope}%
\begin{pgfscope}%
\pgfsetbuttcap%
\pgfsetroundjoin%
\definecolor{currentfill}{rgb}{0.000000,0.000000,0.000000}%
\pgfsetfillcolor{currentfill}%
\pgfsetlinewidth{0.501875pt}%
\definecolor{currentstroke}{rgb}{0.000000,0.000000,0.000000}%
\pgfsetstrokecolor{currentstroke}%
\pgfsetdash{}{0pt}%
\pgfsys@defobject{currentmarker}{\pgfqpoint{0.000000in}{-0.027778in}}{\pgfqpoint{0.000000in}{0.000000in}}{%
\pgfpathmoveto{\pgfqpoint{0.000000in}{0.000000in}}%
\pgfpathlineto{\pgfqpoint{0.000000in}{-0.027778in}}%
\pgfusepath{stroke,fill}%
}%
\begin{pgfscope}%
\pgfsys@transformshift{3.756134in}{2.700000in}%
\pgfsys@useobject{currentmarker}{}%
\end{pgfscope}%
\end{pgfscope}%
\begin{pgfscope}%
\pgfsetbuttcap%
\pgfsetroundjoin%
\definecolor{currentfill}{rgb}{0.000000,0.000000,0.000000}%
\pgfsetfillcolor{currentfill}%
\pgfsetlinewidth{0.501875pt}%
\definecolor{currentstroke}{rgb}{0.000000,0.000000,0.000000}%
\pgfsetstrokecolor{currentstroke}%
\pgfsetdash{}{0pt}%
\pgfsys@defobject{currentmarker}{\pgfqpoint{0.000000in}{0.000000in}}{\pgfqpoint{0.000000in}{0.027778in}}{%
\pgfpathmoveto{\pgfqpoint{0.000000in}{0.000000in}}%
\pgfpathlineto{\pgfqpoint{0.000000in}{0.027778in}}%
\pgfusepath{stroke,fill}%
}%
\begin{pgfscope}%
\pgfsys@transformshift{3.859902in}{0.300000in}%
\pgfsys@useobject{currentmarker}{}%
\end{pgfscope}%
\end{pgfscope}%
\begin{pgfscope}%
\pgfsetbuttcap%
\pgfsetroundjoin%
\definecolor{currentfill}{rgb}{0.000000,0.000000,0.000000}%
\pgfsetfillcolor{currentfill}%
\pgfsetlinewidth{0.501875pt}%
\definecolor{currentstroke}{rgb}{0.000000,0.000000,0.000000}%
\pgfsetstrokecolor{currentstroke}%
\pgfsetdash{}{0pt}%
\pgfsys@defobject{currentmarker}{\pgfqpoint{0.000000in}{-0.027778in}}{\pgfqpoint{0.000000in}{0.000000in}}{%
\pgfpathmoveto{\pgfqpoint{0.000000in}{0.000000in}}%
\pgfpathlineto{\pgfqpoint{0.000000in}{-0.027778in}}%
\pgfusepath{stroke,fill}%
}%
\begin{pgfscope}%
\pgfsys@transformshift{3.859902in}{2.700000in}%
\pgfsys@useobject{currentmarker}{}%
\end{pgfscope}%
\end{pgfscope}%
\begin{pgfscope}%
\pgfsetbuttcap%
\pgfsetroundjoin%
\definecolor{currentfill}{rgb}{0.000000,0.000000,0.000000}%
\pgfsetfillcolor{currentfill}%
\pgfsetlinewidth{0.501875pt}%
\definecolor{currentstroke}{rgb}{0.000000,0.000000,0.000000}%
\pgfsetstrokecolor{currentstroke}%
\pgfsetdash{}{0pt}%
\pgfsys@defobject{currentmarker}{\pgfqpoint{0.000000in}{0.000000in}}{\pgfqpoint{0.000000in}{0.027778in}}{%
\pgfpathmoveto{\pgfqpoint{0.000000in}{0.000000in}}%
\pgfpathlineto{\pgfqpoint{0.000000in}{0.027778in}}%
\pgfusepath{stroke,fill}%
}%
\begin{pgfscope}%
\pgfsys@transformshift{3.949789in}{0.300000in}%
\pgfsys@useobject{currentmarker}{}%
\end{pgfscope}%
\end{pgfscope}%
\begin{pgfscope}%
\pgfsetbuttcap%
\pgfsetroundjoin%
\definecolor{currentfill}{rgb}{0.000000,0.000000,0.000000}%
\pgfsetfillcolor{currentfill}%
\pgfsetlinewidth{0.501875pt}%
\definecolor{currentstroke}{rgb}{0.000000,0.000000,0.000000}%
\pgfsetstrokecolor{currentstroke}%
\pgfsetdash{}{0pt}%
\pgfsys@defobject{currentmarker}{\pgfqpoint{0.000000in}{-0.027778in}}{\pgfqpoint{0.000000in}{0.000000in}}{%
\pgfpathmoveto{\pgfqpoint{0.000000in}{0.000000in}}%
\pgfpathlineto{\pgfqpoint{0.000000in}{-0.027778in}}%
\pgfusepath{stroke,fill}%
}%
\begin{pgfscope}%
\pgfsys@transformshift{3.949789in}{2.700000in}%
\pgfsys@useobject{currentmarker}{}%
\end{pgfscope}%
\end{pgfscope}%
\begin{pgfscope}%
\pgfsetbuttcap%
\pgfsetroundjoin%
\definecolor{currentfill}{rgb}{0.000000,0.000000,0.000000}%
\pgfsetfillcolor{currentfill}%
\pgfsetlinewidth{0.501875pt}%
\definecolor{currentstroke}{rgb}{0.000000,0.000000,0.000000}%
\pgfsetstrokecolor{currentstroke}%
\pgfsetdash{}{0pt}%
\pgfsys@defobject{currentmarker}{\pgfqpoint{0.000000in}{0.000000in}}{\pgfqpoint{0.000000in}{0.027778in}}{%
\pgfpathmoveto{\pgfqpoint{0.000000in}{0.000000in}}%
\pgfpathlineto{\pgfqpoint{0.000000in}{0.027778in}}%
\pgfusepath{stroke,fill}%
}%
\begin{pgfscope}%
\pgfsys@transformshift{4.029076in}{0.300000in}%
\pgfsys@useobject{currentmarker}{}%
\end{pgfscope}%
\end{pgfscope}%
\begin{pgfscope}%
\pgfsetbuttcap%
\pgfsetroundjoin%
\definecolor{currentfill}{rgb}{0.000000,0.000000,0.000000}%
\pgfsetfillcolor{currentfill}%
\pgfsetlinewidth{0.501875pt}%
\definecolor{currentstroke}{rgb}{0.000000,0.000000,0.000000}%
\pgfsetstrokecolor{currentstroke}%
\pgfsetdash{}{0pt}%
\pgfsys@defobject{currentmarker}{\pgfqpoint{0.000000in}{-0.027778in}}{\pgfqpoint{0.000000in}{0.000000in}}{%
\pgfpathmoveto{\pgfqpoint{0.000000in}{0.000000in}}%
\pgfpathlineto{\pgfqpoint{0.000000in}{-0.027778in}}%
\pgfusepath{stroke,fill}%
}%
\begin{pgfscope}%
\pgfsys@transformshift{4.029076in}{2.700000in}%
\pgfsys@useobject{currentmarker}{}%
\end{pgfscope}%
\end{pgfscope}%
\begin{pgfscope}%
\pgfsetbuttcap%
\pgfsetroundjoin%
\definecolor{currentfill}{rgb}{0.000000,0.000000,0.000000}%
\pgfsetfillcolor{currentfill}%
\pgfsetlinewidth{0.501875pt}%
\definecolor{currentstroke}{rgb}{0.000000,0.000000,0.000000}%
\pgfsetstrokecolor{currentstroke}%
\pgfsetdash{}{0pt}%
\pgfsys@defobject{currentmarker}{\pgfqpoint{0.000000in}{0.000000in}}{\pgfqpoint{0.000000in}{0.027778in}}{%
\pgfpathmoveto{\pgfqpoint{0.000000in}{0.000000in}}%
\pgfpathlineto{\pgfqpoint{0.000000in}{0.027778in}}%
\pgfusepath{stroke,fill}%
}%
\begin{pgfscope}%
\pgfsys@transformshift{4.566596in}{0.300000in}%
\pgfsys@useobject{currentmarker}{}%
\end{pgfscope}%
\end{pgfscope}%
\begin{pgfscope}%
\pgfsetbuttcap%
\pgfsetroundjoin%
\definecolor{currentfill}{rgb}{0.000000,0.000000,0.000000}%
\pgfsetfillcolor{currentfill}%
\pgfsetlinewidth{0.501875pt}%
\definecolor{currentstroke}{rgb}{0.000000,0.000000,0.000000}%
\pgfsetstrokecolor{currentstroke}%
\pgfsetdash{}{0pt}%
\pgfsys@defobject{currentmarker}{\pgfqpoint{0.000000in}{-0.027778in}}{\pgfqpoint{0.000000in}{0.000000in}}{%
\pgfpathmoveto{\pgfqpoint{0.000000in}{0.000000in}}%
\pgfpathlineto{\pgfqpoint{0.000000in}{-0.027778in}}%
\pgfusepath{stroke,fill}%
}%
\begin{pgfscope}%
\pgfsys@transformshift{4.566596in}{2.700000in}%
\pgfsys@useobject{currentmarker}{}%
\end{pgfscope}%
\end{pgfscope}%
\begin{pgfscope}%
\pgfsetbuttcap%
\pgfsetroundjoin%
\definecolor{currentfill}{rgb}{0.000000,0.000000,0.000000}%
\pgfsetfillcolor{currentfill}%
\pgfsetlinewidth{0.501875pt}%
\definecolor{currentstroke}{rgb}{0.000000,0.000000,0.000000}%
\pgfsetstrokecolor{currentstroke}%
\pgfsetdash{}{0pt}%
\pgfsys@defobject{currentmarker}{\pgfqpoint{0.000000in}{0.000000in}}{\pgfqpoint{0.000000in}{0.027778in}}{%
\pgfpathmoveto{\pgfqpoint{0.000000in}{0.000000in}}%
\pgfpathlineto{\pgfqpoint{0.000000in}{0.027778in}}%
\pgfusepath{stroke,fill}%
}%
\begin{pgfscope}%
\pgfsys@transformshift{4.839538in}{0.300000in}%
\pgfsys@useobject{currentmarker}{}%
\end{pgfscope}%
\end{pgfscope}%
\begin{pgfscope}%
\pgfsetbuttcap%
\pgfsetroundjoin%
\definecolor{currentfill}{rgb}{0.000000,0.000000,0.000000}%
\pgfsetfillcolor{currentfill}%
\pgfsetlinewidth{0.501875pt}%
\definecolor{currentstroke}{rgb}{0.000000,0.000000,0.000000}%
\pgfsetstrokecolor{currentstroke}%
\pgfsetdash{}{0pt}%
\pgfsys@defobject{currentmarker}{\pgfqpoint{0.000000in}{-0.027778in}}{\pgfqpoint{0.000000in}{0.000000in}}{%
\pgfpathmoveto{\pgfqpoint{0.000000in}{0.000000in}}%
\pgfpathlineto{\pgfqpoint{0.000000in}{-0.027778in}}%
\pgfusepath{stroke,fill}%
}%
\begin{pgfscope}%
\pgfsys@transformshift{4.839538in}{2.700000in}%
\pgfsys@useobject{currentmarker}{}%
\end{pgfscope}%
\end{pgfscope}%
\begin{pgfscope}%
\pgfsetbuttcap%
\pgfsetroundjoin%
\definecolor{currentfill}{rgb}{0.000000,0.000000,0.000000}%
\pgfsetfillcolor{currentfill}%
\pgfsetlinewidth{0.501875pt}%
\definecolor{currentstroke}{rgb}{0.000000,0.000000,0.000000}%
\pgfsetstrokecolor{currentstroke}%
\pgfsetdash{}{0pt}%
\pgfsys@defobject{currentmarker}{\pgfqpoint{0.000000in}{0.000000in}}{\pgfqpoint{0.000000in}{0.027778in}}{%
\pgfpathmoveto{\pgfqpoint{0.000000in}{0.000000in}}%
\pgfpathlineto{\pgfqpoint{0.000000in}{0.027778in}}%
\pgfusepath{stroke,fill}%
}%
\begin{pgfscope}%
\pgfsys@transformshift{5.033193in}{0.300000in}%
\pgfsys@useobject{currentmarker}{}%
\end{pgfscope}%
\end{pgfscope}%
\begin{pgfscope}%
\pgfsetbuttcap%
\pgfsetroundjoin%
\definecolor{currentfill}{rgb}{0.000000,0.000000,0.000000}%
\pgfsetfillcolor{currentfill}%
\pgfsetlinewidth{0.501875pt}%
\definecolor{currentstroke}{rgb}{0.000000,0.000000,0.000000}%
\pgfsetstrokecolor{currentstroke}%
\pgfsetdash{}{0pt}%
\pgfsys@defobject{currentmarker}{\pgfqpoint{0.000000in}{-0.027778in}}{\pgfqpoint{0.000000in}{0.000000in}}{%
\pgfpathmoveto{\pgfqpoint{0.000000in}{0.000000in}}%
\pgfpathlineto{\pgfqpoint{0.000000in}{-0.027778in}}%
\pgfusepath{stroke,fill}%
}%
\begin{pgfscope}%
\pgfsys@transformshift{5.033193in}{2.700000in}%
\pgfsys@useobject{currentmarker}{}%
\end{pgfscope}%
\end{pgfscope}%
\begin{pgfscope}%
\pgfsetbuttcap%
\pgfsetroundjoin%
\definecolor{currentfill}{rgb}{0.000000,0.000000,0.000000}%
\pgfsetfillcolor{currentfill}%
\pgfsetlinewidth{0.501875pt}%
\definecolor{currentstroke}{rgb}{0.000000,0.000000,0.000000}%
\pgfsetstrokecolor{currentstroke}%
\pgfsetdash{}{0pt}%
\pgfsys@defobject{currentmarker}{\pgfqpoint{0.000000in}{0.000000in}}{\pgfqpoint{0.000000in}{0.027778in}}{%
\pgfpathmoveto{\pgfqpoint{0.000000in}{0.000000in}}%
\pgfpathlineto{\pgfqpoint{0.000000in}{0.027778in}}%
\pgfusepath{stroke,fill}%
}%
\begin{pgfscope}%
\pgfsys@transformshift{5.183404in}{0.300000in}%
\pgfsys@useobject{currentmarker}{}%
\end{pgfscope}%
\end{pgfscope}%
\begin{pgfscope}%
\pgfsetbuttcap%
\pgfsetroundjoin%
\definecolor{currentfill}{rgb}{0.000000,0.000000,0.000000}%
\pgfsetfillcolor{currentfill}%
\pgfsetlinewidth{0.501875pt}%
\definecolor{currentstroke}{rgb}{0.000000,0.000000,0.000000}%
\pgfsetstrokecolor{currentstroke}%
\pgfsetdash{}{0pt}%
\pgfsys@defobject{currentmarker}{\pgfqpoint{0.000000in}{-0.027778in}}{\pgfqpoint{0.000000in}{0.000000in}}{%
\pgfpathmoveto{\pgfqpoint{0.000000in}{0.000000in}}%
\pgfpathlineto{\pgfqpoint{0.000000in}{-0.027778in}}%
\pgfusepath{stroke,fill}%
}%
\begin{pgfscope}%
\pgfsys@transformshift{5.183404in}{2.700000in}%
\pgfsys@useobject{currentmarker}{}%
\end{pgfscope}%
\end{pgfscope}%
\begin{pgfscope}%
\pgfsetbuttcap%
\pgfsetroundjoin%
\definecolor{currentfill}{rgb}{0.000000,0.000000,0.000000}%
\pgfsetfillcolor{currentfill}%
\pgfsetlinewidth{0.501875pt}%
\definecolor{currentstroke}{rgb}{0.000000,0.000000,0.000000}%
\pgfsetstrokecolor{currentstroke}%
\pgfsetdash{}{0pt}%
\pgfsys@defobject{currentmarker}{\pgfqpoint{0.000000in}{0.000000in}}{\pgfqpoint{0.000000in}{0.027778in}}{%
\pgfpathmoveto{\pgfqpoint{0.000000in}{0.000000in}}%
\pgfpathlineto{\pgfqpoint{0.000000in}{0.027778in}}%
\pgfusepath{stroke,fill}%
}%
\begin{pgfscope}%
\pgfsys@transformshift{5.306134in}{0.300000in}%
\pgfsys@useobject{currentmarker}{}%
\end{pgfscope}%
\end{pgfscope}%
\begin{pgfscope}%
\pgfsetbuttcap%
\pgfsetroundjoin%
\definecolor{currentfill}{rgb}{0.000000,0.000000,0.000000}%
\pgfsetfillcolor{currentfill}%
\pgfsetlinewidth{0.501875pt}%
\definecolor{currentstroke}{rgb}{0.000000,0.000000,0.000000}%
\pgfsetstrokecolor{currentstroke}%
\pgfsetdash{}{0pt}%
\pgfsys@defobject{currentmarker}{\pgfqpoint{0.000000in}{-0.027778in}}{\pgfqpoint{0.000000in}{0.000000in}}{%
\pgfpathmoveto{\pgfqpoint{0.000000in}{0.000000in}}%
\pgfpathlineto{\pgfqpoint{0.000000in}{-0.027778in}}%
\pgfusepath{stroke,fill}%
}%
\begin{pgfscope}%
\pgfsys@transformshift{5.306134in}{2.700000in}%
\pgfsys@useobject{currentmarker}{}%
\end{pgfscope}%
\end{pgfscope}%
\begin{pgfscope}%
\pgfsetbuttcap%
\pgfsetroundjoin%
\definecolor{currentfill}{rgb}{0.000000,0.000000,0.000000}%
\pgfsetfillcolor{currentfill}%
\pgfsetlinewidth{0.501875pt}%
\definecolor{currentstroke}{rgb}{0.000000,0.000000,0.000000}%
\pgfsetstrokecolor{currentstroke}%
\pgfsetdash{}{0pt}%
\pgfsys@defobject{currentmarker}{\pgfqpoint{0.000000in}{0.000000in}}{\pgfqpoint{0.000000in}{0.027778in}}{%
\pgfpathmoveto{\pgfqpoint{0.000000in}{0.000000in}}%
\pgfpathlineto{\pgfqpoint{0.000000in}{0.027778in}}%
\pgfusepath{stroke,fill}%
}%
\begin{pgfscope}%
\pgfsys@transformshift{5.409902in}{0.300000in}%
\pgfsys@useobject{currentmarker}{}%
\end{pgfscope}%
\end{pgfscope}%
\begin{pgfscope}%
\pgfsetbuttcap%
\pgfsetroundjoin%
\definecolor{currentfill}{rgb}{0.000000,0.000000,0.000000}%
\pgfsetfillcolor{currentfill}%
\pgfsetlinewidth{0.501875pt}%
\definecolor{currentstroke}{rgb}{0.000000,0.000000,0.000000}%
\pgfsetstrokecolor{currentstroke}%
\pgfsetdash{}{0pt}%
\pgfsys@defobject{currentmarker}{\pgfqpoint{0.000000in}{-0.027778in}}{\pgfqpoint{0.000000in}{0.000000in}}{%
\pgfpathmoveto{\pgfqpoint{0.000000in}{0.000000in}}%
\pgfpathlineto{\pgfqpoint{0.000000in}{-0.027778in}}%
\pgfusepath{stroke,fill}%
}%
\begin{pgfscope}%
\pgfsys@transformshift{5.409902in}{2.700000in}%
\pgfsys@useobject{currentmarker}{}%
\end{pgfscope}%
\end{pgfscope}%
\begin{pgfscope}%
\pgfsetbuttcap%
\pgfsetroundjoin%
\definecolor{currentfill}{rgb}{0.000000,0.000000,0.000000}%
\pgfsetfillcolor{currentfill}%
\pgfsetlinewidth{0.501875pt}%
\definecolor{currentstroke}{rgb}{0.000000,0.000000,0.000000}%
\pgfsetstrokecolor{currentstroke}%
\pgfsetdash{}{0pt}%
\pgfsys@defobject{currentmarker}{\pgfqpoint{0.000000in}{0.000000in}}{\pgfqpoint{0.000000in}{0.027778in}}{%
\pgfpathmoveto{\pgfqpoint{0.000000in}{0.000000in}}%
\pgfpathlineto{\pgfqpoint{0.000000in}{0.027778in}}%
\pgfusepath{stroke,fill}%
}%
\begin{pgfscope}%
\pgfsys@transformshift{5.499789in}{0.300000in}%
\pgfsys@useobject{currentmarker}{}%
\end{pgfscope}%
\end{pgfscope}%
\begin{pgfscope}%
\pgfsetbuttcap%
\pgfsetroundjoin%
\definecolor{currentfill}{rgb}{0.000000,0.000000,0.000000}%
\pgfsetfillcolor{currentfill}%
\pgfsetlinewidth{0.501875pt}%
\definecolor{currentstroke}{rgb}{0.000000,0.000000,0.000000}%
\pgfsetstrokecolor{currentstroke}%
\pgfsetdash{}{0pt}%
\pgfsys@defobject{currentmarker}{\pgfqpoint{0.000000in}{-0.027778in}}{\pgfqpoint{0.000000in}{0.000000in}}{%
\pgfpathmoveto{\pgfqpoint{0.000000in}{0.000000in}}%
\pgfpathlineto{\pgfqpoint{0.000000in}{-0.027778in}}%
\pgfusepath{stroke,fill}%
}%
\begin{pgfscope}%
\pgfsys@transformshift{5.499789in}{2.700000in}%
\pgfsys@useobject{currentmarker}{}%
\end{pgfscope}%
\end{pgfscope}%
\begin{pgfscope}%
\pgfsetbuttcap%
\pgfsetroundjoin%
\definecolor{currentfill}{rgb}{0.000000,0.000000,0.000000}%
\pgfsetfillcolor{currentfill}%
\pgfsetlinewidth{0.501875pt}%
\definecolor{currentstroke}{rgb}{0.000000,0.000000,0.000000}%
\pgfsetstrokecolor{currentstroke}%
\pgfsetdash{}{0pt}%
\pgfsys@defobject{currentmarker}{\pgfqpoint{0.000000in}{0.000000in}}{\pgfqpoint{0.000000in}{0.027778in}}{%
\pgfpathmoveto{\pgfqpoint{0.000000in}{0.000000in}}%
\pgfpathlineto{\pgfqpoint{0.000000in}{0.027778in}}%
\pgfusepath{stroke,fill}%
}%
\begin{pgfscope}%
\pgfsys@transformshift{5.579076in}{0.300000in}%
\pgfsys@useobject{currentmarker}{}%
\end{pgfscope}%
\end{pgfscope}%
\begin{pgfscope}%
\pgfsetbuttcap%
\pgfsetroundjoin%
\definecolor{currentfill}{rgb}{0.000000,0.000000,0.000000}%
\pgfsetfillcolor{currentfill}%
\pgfsetlinewidth{0.501875pt}%
\definecolor{currentstroke}{rgb}{0.000000,0.000000,0.000000}%
\pgfsetstrokecolor{currentstroke}%
\pgfsetdash{}{0pt}%
\pgfsys@defobject{currentmarker}{\pgfqpoint{0.000000in}{-0.027778in}}{\pgfqpoint{0.000000in}{0.000000in}}{%
\pgfpathmoveto{\pgfqpoint{0.000000in}{0.000000in}}%
\pgfpathlineto{\pgfqpoint{0.000000in}{-0.027778in}}%
\pgfusepath{stroke,fill}%
}%
\begin{pgfscope}%
\pgfsys@transformshift{5.579076in}{2.700000in}%
\pgfsys@useobject{currentmarker}{}%
\end{pgfscope}%
\end{pgfscope}%
\begin{pgfscope}%
\pgfsetbuttcap%
\pgfsetroundjoin%
\definecolor{currentfill}{rgb}{0.000000,0.000000,0.000000}%
\pgfsetfillcolor{currentfill}%
\pgfsetlinewidth{0.501875pt}%
\definecolor{currentstroke}{rgb}{0.000000,0.000000,0.000000}%
\pgfsetstrokecolor{currentstroke}%
\pgfsetdash{}{0pt}%
\pgfsys@defobject{currentmarker}{\pgfqpoint{0.000000in}{0.000000in}}{\pgfqpoint{0.000000in}{0.027778in}}{%
\pgfpathmoveto{\pgfqpoint{0.000000in}{0.000000in}}%
\pgfpathlineto{\pgfqpoint{0.000000in}{0.027778in}}%
\pgfusepath{stroke,fill}%
}%
\begin{pgfscope}%
\pgfsys@transformshift{6.116596in}{0.300000in}%
\pgfsys@useobject{currentmarker}{}%
\end{pgfscope}%
\end{pgfscope}%
\begin{pgfscope}%
\pgfsetbuttcap%
\pgfsetroundjoin%
\definecolor{currentfill}{rgb}{0.000000,0.000000,0.000000}%
\pgfsetfillcolor{currentfill}%
\pgfsetlinewidth{0.501875pt}%
\definecolor{currentstroke}{rgb}{0.000000,0.000000,0.000000}%
\pgfsetstrokecolor{currentstroke}%
\pgfsetdash{}{0pt}%
\pgfsys@defobject{currentmarker}{\pgfqpoint{0.000000in}{-0.027778in}}{\pgfqpoint{0.000000in}{0.000000in}}{%
\pgfpathmoveto{\pgfqpoint{0.000000in}{0.000000in}}%
\pgfpathlineto{\pgfqpoint{0.000000in}{-0.027778in}}%
\pgfusepath{stroke,fill}%
}%
\begin{pgfscope}%
\pgfsys@transformshift{6.116596in}{2.700000in}%
\pgfsys@useobject{currentmarker}{}%
\end{pgfscope}%
\end{pgfscope}%
\begin{pgfscope}%
\pgfsetbuttcap%
\pgfsetroundjoin%
\definecolor{currentfill}{rgb}{0.000000,0.000000,0.000000}%
\pgfsetfillcolor{currentfill}%
\pgfsetlinewidth{0.501875pt}%
\definecolor{currentstroke}{rgb}{0.000000,0.000000,0.000000}%
\pgfsetstrokecolor{currentstroke}%
\pgfsetdash{}{0pt}%
\pgfsys@defobject{currentmarker}{\pgfqpoint{0.000000in}{0.000000in}}{\pgfqpoint{0.000000in}{0.027778in}}{%
\pgfpathmoveto{\pgfqpoint{0.000000in}{0.000000in}}%
\pgfpathlineto{\pgfqpoint{0.000000in}{0.027778in}}%
\pgfusepath{stroke,fill}%
}%
\begin{pgfscope}%
\pgfsys@transformshift{6.389538in}{0.300000in}%
\pgfsys@useobject{currentmarker}{}%
\end{pgfscope}%
\end{pgfscope}%
\begin{pgfscope}%
\pgfsetbuttcap%
\pgfsetroundjoin%
\definecolor{currentfill}{rgb}{0.000000,0.000000,0.000000}%
\pgfsetfillcolor{currentfill}%
\pgfsetlinewidth{0.501875pt}%
\definecolor{currentstroke}{rgb}{0.000000,0.000000,0.000000}%
\pgfsetstrokecolor{currentstroke}%
\pgfsetdash{}{0pt}%
\pgfsys@defobject{currentmarker}{\pgfqpoint{0.000000in}{-0.027778in}}{\pgfqpoint{0.000000in}{0.000000in}}{%
\pgfpathmoveto{\pgfqpoint{0.000000in}{0.000000in}}%
\pgfpathlineto{\pgfqpoint{0.000000in}{-0.027778in}}%
\pgfusepath{stroke,fill}%
}%
\begin{pgfscope}%
\pgfsys@transformshift{6.389538in}{2.700000in}%
\pgfsys@useobject{currentmarker}{}%
\end{pgfscope}%
\end{pgfscope}%
\begin{pgfscope}%
\pgfsetbuttcap%
\pgfsetroundjoin%
\definecolor{currentfill}{rgb}{0.000000,0.000000,0.000000}%
\pgfsetfillcolor{currentfill}%
\pgfsetlinewidth{0.501875pt}%
\definecolor{currentstroke}{rgb}{0.000000,0.000000,0.000000}%
\pgfsetstrokecolor{currentstroke}%
\pgfsetdash{}{0pt}%
\pgfsys@defobject{currentmarker}{\pgfqpoint{0.000000in}{0.000000in}}{\pgfqpoint{0.000000in}{0.027778in}}{%
\pgfpathmoveto{\pgfqpoint{0.000000in}{0.000000in}}%
\pgfpathlineto{\pgfqpoint{0.000000in}{0.027778in}}%
\pgfusepath{stroke,fill}%
}%
\begin{pgfscope}%
\pgfsys@transformshift{6.583193in}{0.300000in}%
\pgfsys@useobject{currentmarker}{}%
\end{pgfscope}%
\end{pgfscope}%
\begin{pgfscope}%
\pgfsetbuttcap%
\pgfsetroundjoin%
\definecolor{currentfill}{rgb}{0.000000,0.000000,0.000000}%
\pgfsetfillcolor{currentfill}%
\pgfsetlinewidth{0.501875pt}%
\definecolor{currentstroke}{rgb}{0.000000,0.000000,0.000000}%
\pgfsetstrokecolor{currentstroke}%
\pgfsetdash{}{0pt}%
\pgfsys@defobject{currentmarker}{\pgfqpoint{0.000000in}{-0.027778in}}{\pgfqpoint{0.000000in}{0.000000in}}{%
\pgfpathmoveto{\pgfqpoint{0.000000in}{0.000000in}}%
\pgfpathlineto{\pgfqpoint{0.000000in}{-0.027778in}}%
\pgfusepath{stroke,fill}%
}%
\begin{pgfscope}%
\pgfsys@transformshift{6.583193in}{2.700000in}%
\pgfsys@useobject{currentmarker}{}%
\end{pgfscope}%
\end{pgfscope}%
\begin{pgfscope}%
\pgfsetbuttcap%
\pgfsetroundjoin%
\definecolor{currentfill}{rgb}{0.000000,0.000000,0.000000}%
\pgfsetfillcolor{currentfill}%
\pgfsetlinewidth{0.501875pt}%
\definecolor{currentstroke}{rgb}{0.000000,0.000000,0.000000}%
\pgfsetstrokecolor{currentstroke}%
\pgfsetdash{}{0pt}%
\pgfsys@defobject{currentmarker}{\pgfqpoint{0.000000in}{0.000000in}}{\pgfqpoint{0.000000in}{0.027778in}}{%
\pgfpathmoveto{\pgfqpoint{0.000000in}{0.000000in}}%
\pgfpathlineto{\pgfqpoint{0.000000in}{0.027778in}}%
\pgfusepath{stroke,fill}%
}%
\begin{pgfscope}%
\pgfsys@transformshift{6.733404in}{0.300000in}%
\pgfsys@useobject{currentmarker}{}%
\end{pgfscope}%
\end{pgfscope}%
\begin{pgfscope}%
\pgfsetbuttcap%
\pgfsetroundjoin%
\definecolor{currentfill}{rgb}{0.000000,0.000000,0.000000}%
\pgfsetfillcolor{currentfill}%
\pgfsetlinewidth{0.501875pt}%
\definecolor{currentstroke}{rgb}{0.000000,0.000000,0.000000}%
\pgfsetstrokecolor{currentstroke}%
\pgfsetdash{}{0pt}%
\pgfsys@defobject{currentmarker}{\pgfqpoint{0.000000in}{-0.027778in}}{\pgfqpoint{0.000000in}{0.000000in}}{%
\pgfpathmoveto{\pgfqpoint{0.000000in}{0.000000in}}%
\pgfpathlineto{\pgfqpoint{0.000000in}{-0.027778in}}%
\pgfusepath{stroke,fill}%
}%
\begin{pgfscope}%
\pgfsys@transformshift{6.733404in}{2.700000in}%
\pgfsys@useobject{currentmarker}{}%
\end{pgfscope}%
\end{pgfscope}%
\begin{pgfscope}%
\pgfsetbuttcap%
\pgfsetroundjoin%
\definecolor{currentfill}{rgb}{0.000000,0.000000,0.000000}%
\pgfsetfillcolor{currentfill}%
\pgfsetlinewidth{0.501875pt}%
\definecolor{currentstroke}{rgb}{0.000000,0.000000,0.000000}%
\pgfsetstrokecolor{currentstroke}%
\pgfsetdash{}{0pt}%
\pgfsys@defobject{currentmarker}{\pgfqpoint{0.000000in}{0.000000in}}{\pgfqpoint{0.000000in}{0.027778in}}{%
\pgfpathmoveto{\pgfqpoint{0.000000in}{0.000000in}}%
\pgfpathlineto{\pgfqpoint{0.000000in}{0.027778in}}%
\pgfusepath{stroke,fill}%
}%
\begin{pgfscope}%
\pgfsys@transformshift{6.856134in}{0.300000in}%
\pgfsys@useobject{currentmarker}{}%
\end{pgfscope}%
\end{pgfscope}%
\begin{pgfscope}%
\pgfsetbuttcap%
\pgfsetroundjoin%
\definecolor{currentfill}{rgb}{0.000000,0.000000,0.000000}%
\pgfsetfillcolor{currentfill}%
\pgfsetlinewidth{0.501875pt}%
\definecolor{currentstroke}{rgb}{0.000000,0.000000,0.000000}%
\pgfsetstrokecolor{currentstroke}%
\pgfsetdash{}{0pt}%
\pgfsys@defobject{currentmarker}{\pgfqpoint{0.000000in}{-0.027778in}}{\pgfqpoint{0.000000in}{0.000000in}}{%
\pgfpathmoveto{\pgfqpoint{0.000000in}{0.000000in}}%
\pgfpathlineto{\pgfqpoint{0.000000in}{-0.027778in}}%
\pgfusepath{stroke,fill}%
}%
\begin{pgfscope}%
\pgfsys@transformshift{6.856134in}{2.700000in}%
\pgfsys@useobject{currentmarker}{}%
\end{pgfscope}%
\end{pgfscope}%
\begin{pgfscope}%
\pgfsetbuttcap%
\pgfsetroundjoin%
\definecolor{currentfill}{rgb}{0.000000,0.000000,0.000000}%
\pgfsetfillcolor{currentfill}%
\pgfsetlinewidth{0.501875pt}%
\definecolor{currentstroke}{rgb}{0.000000,0.000000,0.000000}%
\pgfsetstrokecolor{currentstroke}%
\pgfsetdash{}{0pt}%
\pgfsys@defobject{currentmarker}{\pgfqpoint{0.000000in}{0.000000in}}{\pgfqpoint{0.000000in}{0.027778in}}{%
\pgfpathmoveto{\pgfqpoint{0.000000in}{0.000000in}}%
\pgfpathlineto{\pgfqpoint{0.000000in}{0.027778in}}%
\pgfusepath{stroke,fill}%
}%
\begin{pgfscope}%
\pgfsys@transformshift{6.959902in}{0.300000in}%
\pgfsys@useobject{currentmarker}{}%
\end{pgfscope}%
\end{pgfscope}%
\begin{pgfscope}%
\pgfsetbuttcap%
\pgfsetroundjoin%
\definecolor{currentfill}{rgb}{0.000000,0.000000,0.000000}%
\pgfsetfillcolor{currentfill}%
\pgfsetlinewidth{0.501875pt}%
\definecolor{currentstroke}{rgb}{0.000000,0.000000,0.000000}%
\pgfsetstrokecolor{currentstroke}%
\pgfsetdash{}{0pt}%
\pgfsys@defobject{currentmarker}{\pgfqpoint{0.000000in}{-0.027778in}}{\pgfqpoint{0.000000in}{0.000000in}}{%
\pgfpathmoveto{\pgfqpoint{0.000000in}{0.000000in}}%
\pgfpathlineto{\pgfqpoint{0.000000in}{-0.027778in}}%
\pgfusepath{stroke,fill}%
}%
\begin{pgfscope}%
\pgfsys@transformshift{6.959902in}{2.700000in}%
\pgfsys@useobject{currentmarker}{}%
\end{pgfscope}%
\end{pgfscope}%
\begin{pgfscope}%
\pgfsetbuttcap%
\pgfsetroundjoin%
\definecolor{currentfill}{rgb}{0.000000,0.000000,0.000000}%
\pgfsetfillcolor{currentfill}%
\pgfsetlinewidth{0.501875pt}%
\definecolor{currentstroke}{rgb}{0.000000,0.000000,0.000000}%
\pgfsetstrokecolor{currentstroke}%
\pgfsetdash{}{0pt}%
\pgfsys@defobject{currentmarker}{\pgfqpoint{0.000000in}{0.000000in}}{\pgfqpoint{0.000000in}{0.027778in}}{%
\pgfpathmoveto{\pgfqpoint{0.000000in}{0.000000in}}%
\pgfpathlineto{\pgfqpoint{0.000000in}{0.027778in}}%
\pgfusepath{stroke,fill}%
}%
\begin{pgfscope}%
\pgfsys@transformshift{7.049789in}{0.300000in}%
\pgfsys@useobject{currentmarker}{}%
\end{pgfscope}%
\end{pgfscope}%
\begin{pgfscope}%
\pgfsetbuttcap%
\pgfsetroundjoin%
\definecolor{currentfill}{rgb}{0.000000,0.000000,0.000000}%
\pgfsetfillcolor{currentfill}%
\pgfsetlinewidth{0.501875pt}%
\definecolor{currentstroke}{rgb}{0.000000,0.000000,0.000000}%
\pgfsetstrokecolor{currentstroke}%
\pgfsetdash{}{0pt}%
\pgfsys@defobject{currentmarker}{\pgfqpoint{0.000000in}{-0.027778in}}{\pgfqpoint{0.000000in}{0.000000in}}{%
\pgfpathmoveto{\pgfqpoint{0.000000in}{0.000000in}}%
\pgfpathlineto{\pgfqpoint{0.000000in}{-0.027778in}}%
\pgfusepath{stroke,fill}%
}%
\begin{pgfscope}%
\pgfsys@transformshift{7.049789in}{2.700000in}%
\pgfsys@useobject{currentmarker}{}%
\end{pgfscope}%
\end{pgfscope}%
\begin{pgfscope}%
\pgfsetbuttcap%
\pgfsetroundjoin%
\definecolor{currentfill}{rgb}{0.000000,0.000000,0.000000}%
\pgfsetfillcolor{currentfill}%
\pgfsetlinewidth{0.501875pt}%
\definecolor{currentstroke}{rgb}{0.000000,0.000000,0.000000}%
\pgfsetstrokecolor{currentstroke}%
\pgfsetdash{}{0pt}%
\pgfsys@defobject{currentmarker}{\pgfqpoint{0.000000in}{0.000000in}}{\pgfqpoint{0.000000in}{0.027778in}}{%
\pgfpathmoveto{\pgfqpoint{0.000000in}{0.000000in}}%
\pgfpathlineto{\pgfqpoint{0.000000in}{0.027778in}}%
\pgfusepath{stroke,fill}%
}%
\begin{pgfscope}%
\pgfsys@transformshift{7.129076in}{0.300000in}%
\pgfsys@useobject{currentmarker}{}%
\end{pgfscope}%
\end{pgfscope}%
\begin{pgfscope}%
\pgfsetbuttcap%
\pgfsetroundjoin%
\definecolor{currentfill}{rgb}{0.000000,0.000000,0.000000}%
\pgfsetfillcolor{currentfill}%
\pgfsetlinewidth{0.501875pt}%
\definecolor{currentstroke}{rgb}{0.000000,0.000000,0.000000}%
\pgfsetstrokecolor{currentstroke}%
\pgfsetdash{}{0pt}%
\pgfsys@defobject{currentmarker}{\pgfqpoint{0.000000in}{-0.027778in}}{\pgfqpoint{0.000000in}{0.000000in}}{%
\pgfpathmoveto{\pgfqpoint{0.000000in}{0.000000in}}%
\pgfpathlineto{\pgfqpoint{0.000000in}{-0.027778in}}%
\pgfusepath{stroke,fill}%
}%
\begin{pgfscope}%
\pgfsys@transformshift{7.129076in}{2.700000in}%
\pgfsys@useobject{currentmarker}{}%
\end{pgfscope}%
\end{pgfscope}%
\begin{pgfscope}%
\pgftext[left,bottom,x=3.723901in,y=-0.126716in,rotate=0.000000]{{\sffamily\fontsize{12.000000}{14.400000}\selectfont time [ps]}}
%
\end{pgfscope}%
\begin{pgfscope}%
\pgfpathrectangle{\pgfqpoint{1.000000in}{0.300000in}}{\pgfqpoint{6.200000in}{2.400000in}} %
\pgfusepath{clip}%
\pgfsetbuttcap%
\pgfsetroundjoin%
\pgfsetlinewidth{0.501875pt}%
\definecolor{currentstroke}{rgb}{0.000000,0.000000,0.000000}%
\pgfsetstrokecolor{currentstroke}%
\pgfsetdash{{1.000000pt}{3.000000pt}}{0.000000pt}%
\pgfpathmoveto{\pgfqpoint{1.000000in}{0.300000in}}%
\pgfpathlineto{\pgfqpoint{7.200000in}{0.300000in}}%
\pgfusepath{stroke}%
\end{pgfscope}%
\begin{pgfscope}%
\pgfsetbuttcap%
\pgfsetroundjoin%
\definecolor{currentfill}{rgb}{0.000000,0.000000,0.000000}%
\pgfsetfillcolor{currentfill}%
\pgfsetlinewidth{0.501875pt}%
\definecolor{currentstroke}{rgb}{0.000000,0.000000,0.000000}%
\pgfsetstrokecolor{currentstroke}%
\pgfsetdash{}{0pt}%
\pgfsys@defobject{currentmarker}{\pgfqpoint{0.000000in}{0.000000in}}{\pgfqpoint{0.055556in}{0.000000in}}{%
\pgfpathmoveto{\pgfqpoint{0.000000in}{0.000000in}}%
\pgfpathlineto{\pgfqpoint{0.055556in}{0.000000in}}%
\pgfusepath{stroke,fill}%
}%
\begin{pgfscope}%
\pgfsys@transformshift{1.000000in}{0.300000in}%
\pgfsys@useobject{currentmarker}{}%
\end{pgfscope}%
\end{pgfscope}%
\begin{pgfscope}%
\pgfsetbuttcap%
\pgfsetroundjoin%
\definecolor{currentfill}{rgb}{0.000000,0.000000,0.000000}%
\pgfsetfillcolor{currentfill}%
\pgfsetlinewidth{0.501875pt}%
\definecolor{currentstroke}{rgb}{0.000000,0.000000,0.000000}%
\pgfsetstrokecolor{currentstroke}%
\pgfsetdash{}{0pt}%
\pgfsys@defobject{currentmarker}{\pgfqpoint{-0.055556in}{0.000000in}}{\pgfqpoint{0.000000in}{0.000000in}}{%
\pgfpathmoveto{\pgfqpoint{0.000000in}{0.000000in}}%
\pgfpathlineto{\pgfqpoint{-0.055556in}{0.000000in}}%
\pgfusepath{stroke,fill}%
}%
\begin{pgfscope}%
\pgfsys@transformshift{7.200000in}{0.300000in}%
\pgfsys@useobject{currentmarker}{}%
\end{pgfscope}%
\end{pgfscope}%
\begin{pgfscope}%
\pgftext[left,bottom,x=0.623456in,y=0.229790in,rotate=0.000000]{{\sffamily\fontsize{12.000000}{14.400000}\selectfont \(\displaystyle {10^{-3}}\)}}
%
\end{pgfscope}%
\begin{pgfscope}%
\pgfpathrectangle{\pgfqpoint{1.000000in}{0.300000in}}{\pgfqpoint{6.200000in}{2.400000in}} %
\pgfusepath{clip}%
\pgfsetbuttcap%
\pgfsetroundjoin%
\pgfsetlinewidth{0.501875pt}%
\definecolor{currentstroke}{rgb}{0.000000,0.000000,0.000000}%
\pgfsetstrokecolor{currentstroke}%
\pgfsetdash{{1.000000pt}{3.000000pt}}{0.000000pt}%
\pgfpathmoveto{\pgfqpoint{1.000000in}{0.780000in}}%
\pgfpathlineto{\pgfqpoint{7.200000in}{0.780000in}}%
\pgfusepath{stroke}%
\end{pgfscope}%
\begin{pgfscope}%
\pgfsetbuttcap%
\pgfsetroundjoin%
\definecolor{currentfill}{rgb}{0.000000,0.000000,0.000000}%
\pgfsetfillcolor{currentfill}%
\pgfsetlinewidth{0.501875pt}%
\definecolor{currentstroke}{rgb}{0.000000,0.000000,0.000000}%
\pgfsetstrokecolor{currentstroke}%
\pgfsetdash{}{0pt}%
\pgfsys@defobject{currentmarker}{\pgfqpoint{0.000000in}{0.000000in}}{\pgfqpoint{0.055556in}{0.000000in}}{%
\pgfpathmoveto{\pgfqpoint{0.000000in}{0.000000in}}%
\pgfpathlineto{\pgfqpoint{0.055556in}{0.000000in}}%
\pgfusepath{stroke,fill}%
}%
\begin{pgfscope}%
\pgfsys@transformshift{1.000000in}{0.780000in}%
\pgfsys@useobject{currentmarker}{}%
\end{pgfscope}%
\end{pgfscope}%
\begin{pgfscope}%
\pgfsetbuttcap%
\pgfsetroundjoin%
\definecolor{currentfill}{rgb}{0.000000,0.000000,0.000000}%
\pgfsetfillcolor{currentfill}%
\pgfsetlinewidth{0.501875pt}%
\definecolor{currentstroke}{rgb}{0.000000,0.000000,0.000000}%
\pgfsetstrokecolor{currentstroke}%
\pgfsetdash{}{0pt}%
\pgfsys@defobject{currentmarker}{\pgfqpoint{-0.055556in}{0.000000in}}{\pgfqpoint{0.000000in}{0.000000in}}{%
\pgfpathmoveto{\pgfqpoint{0.000000in}{0.000000in}}%
\pgfpathlineto{\pgfqpoint{-0.055556in}{0.000000in}}%
\pgfusepath{stroke,fill}%
}%
\begin{pgfscope}%
\pgfsys@transformshift{7.200000in}{0.780000in}%
\pgfsys@useobject{currentmarker}{}%
\end{pgfscope}%
\end{pgfscope}%
\begin{pgfscope}%
\pgftext[left,bottom,x=0.623456in,y=0.709790in,rotate=0.000000]{{\sffamily\fontsize{12.000000}{14.400000}\selectfont \(\displaystyle {10^{-2}}\)}}
%
\end{pgfscope}%
\begin{pgfscope}%
\pgfpathrectangle{\pgfqpoint{1.000000in}{0.300000in}}{\pgfqpoint{6.200000in}{2.400000in}} %
\pgfusepath{clip}%
\pgfsetbuttcap%
\pgfsetroundjoin%
\pgfsetlinewidth{0.501875pt}%
\definecolor{currentstroke}{rgb}{0.000000,0.000000,0.000000}%
\pgfsetstrokecolor{currentstroke}%
\pgfsetdash{{1.000000pt}{3.000000pt}}{0.000000pt}%
\pgfpathmoveto{\pgfqpoint{1.000000in}{1.260000in}}%
\pgfpathlineto{\pgfqpoint{7.200000in}{1.260000in}}%
\pgfusepath{stroke}%
\end{pgfscope}%
\begin{pgfscope}%
\pgfsetbuttcap%
\pgfsetroundjoin%
\definecolor{currentfill}{rgb}{0.000000,0.000000,0.000000}%
\pgfsetfillcolor{currentfill}%
\pgfsetlinewidth{0.501875pt}%
\definecolor{currentstroke}{rgb}{0.000000,0.000000,0.000000}%
\pgfsetstrokecolor{currentstroke}%
\pgfsetdash{}{0pt}%
\pgfsys@defobject{currentmarker}{\pgfqpoint{0.000000in}{0.000000in}}{\pgfqpoint{0.055556in}{0.000000in}}{%
\pgfpathmoveto{\pgfqpoint{0.000000in}{0.000000in}}%
\pgfpathlineto{\pgfqpoint{0.055556in}{0.000000in}}%
\pgfusepath{stroke,fill}%
}%
\begin{pgfscope}%
\pgfsys@transformshift{1.000000in}{1.260000in}%
\pgfsys@useobject{currentmarker}{}%
\end{pgfscope}%
\end{pgfscope}%
\begin{pgfscope}%
\pgfsetbuttcap%
\pgfsetroundjoin%
\definecolor{currentfill}{rgb}{0.000000,0.000000,0.000000}%
\pgfsetfillcolor{currentfill}%
\pgfsetlinewidth{0.501875pt}%
\definecolor{currentstroke}{rgb}{0.000000,0.000000,0.000000}%
\pgfsetstrokecolor{currentstroke}%
\pgfsetdash{}{0pt}%
\pgfsys@defobject{currentmarker}{\pgfqpoint{-0.055556in}{0.000000in}}{\pgfqpoint{0.000000in}{0.000000in}}{%
\pgfpathmoveto{\pgfqpoint{0.000000in}{0.000000in}}%
\pgfpathlineto{\pgfqpoint{-0.055556in}{0.000000in}}%
\pgfusepath{stroke,fill}%
}%
\begin{pgfscope}%
\pgfsys@transformshift{7.200000in}{1.260000in}%
\pgfsys@useobject{currentmarker}{}%
\end{pgfscope}%
\end{pgfscope}%
\begin{pgfscope}%
\pgftext[left,bottom,x=0.623456in,y=1.189790in,rotate=0.000000]{{\sffamily\fontsize{12.000000}{14.400000}\selectfont \(\displaystyle {10^{-1}}\)}}
%
\end{pgfscope}%
\begin{pgfscope}%
\pgfpathrectangle{\pgfqpoint{1.000000in}{0.300000in}}{\pgfqpoint{6.200000in}{2.400000in}} %
\pgfusepath{clip}%
\pgfsetbuttcap%
\pgfsetroundjoin%
\pgfsetlinewidth{0.501875pt}%
\definecolor{currentstroke}{rgb}{0.000000,0.000000,0.000000}%
\pgfsetstrokecolor{currentstroke}%
\pgfsetdash{{1.000000pt}{3.000000pt}}{0.000000pt}%
\pgfpathmoveto{\pgfqpoint{1.000000in}{1.740000in}}%
\pgfpathlineto{\pgfqpoint{7.200000in}{1.740000in}}%
\pgfusepath{stroke}%
\end{pgfscope}%
\begin{pgfscope}%
\pgfsetbuttcap%
\pgfsetroundjoin%
\definecolor{currentfill}{rgb}{0.000000,0.000000,0.000000}%
\pgfsetfillcolor{currentfill}%
\pgfsetlinewidth{0.501875pt}%
\definecolor{currentstroke}{rgb}{0.000000,0.000000,0.000000}%
\pgfsetstrokecolor{currentstroke}%
\pgfsetdash{}{0pt}%
\pgfsys@defobject{currentmarker}{\pgfqpoint{0.000000in}{0.000000in}}{\pgfqpoint{0.055556in}{0.000000in}}{%
\pgfpathmoveto{\pgfqpoint{0.000000in}{0.000000in}}%
\pgfpathlineto{\pgfqpoint{0.055556in}{0.000000in}}%
\pgfusepath{stroke,fill}%
}%
\begin{pgfscope}%
\pgfsys@transformshift{1.000000in}{1.740000in}%
\pgfsys@useobject{currentmarker}{}%
\end{pgfscope}%
\end{pgfscope}%
\begin{pgfscope}%
\pgfsetbuttcap%
\pgfsetroundjoin%
\definecolor{currentfill}{rgb}{0.000000,0.000000,0.000000}%
\pgfsetfillcolor{currentfill}%
\pgfsetlinewidth{0.501875pt}%
\definecolor{currentstroke}{rgb}{0.000000,0.000000,0.000000}%
\pgfsetstrokecolor{currentstroke}%
\pgfsetdash{}{0pt}%
\pgfsys@defobject{currentmarker}{\pgfqpoint{-0.055556in}{0.000000in}}{\pgfqpoint{0.000000in}{0.000000in}}{%
\pgfpathmoveto{\pgfqpoint{0.000000in}{0.000000in}}%
\pgfpathlineto{\pgfqpoint{-0.055556in}{0.000000in}}%
\pgfusepath{stroke,fill}%
}%
\begin{pgfscope}%
\pgfsys@transformshift{7.200000in}{1.740000in}%
\pgfsys@useobject{currentmarker}{}%
\end{pgfscope}%
\end{pgfscope}%
\begin{pgfscope}%
\pgftext[left,bottom,x=0.715279in,y=1.669790in,rotate=0.000000]{{\sffamily\fontsize{12.000000}{14.400000}\selectfont \(\displaystyle {10^{0}}\)}}
%
\end{pgfscope}%
\begin{pgfscope}%
\pgfpathrectangle{\pgfqpoint{1.000000in}{0.300000in}}{\pgfqpoint{6.200000in}{2.400000in}} %
\pgfusepath{clip}%
\pgfsetbuttcap%
\pgfsetroundjoin%
\pgfsetlinewidth{0.501875pt}%
\definecolor{currentstroke}{rgb}{0.000000,0.000000,0.000000}%
\pgfsetstrokecolor{currentstroke}%
\pgfsetdash{{1.000000pt}{3.000000pt}}{0.000000pt}%
\pgfpathmoveto{\pgfqpoint{1.000000in}{2.220000in}}%
\pgfpathlineto{\pgfqpoint{7.200000in}{2.220000in}}%
\pgfusepath{stroke}%
\end{pgfscope}%
\begin{pgfscope}%
\pgfsetbuttcap%
\pgfsetroundjoin%
\definecolor{currentfill}{rgb}{0.000000,0.000000,0.000000}%
\pgfsetfillcolor{currentfill}%
\pgfsetlinewidth{0.501875pt}%
\definecolor{currentstroke}{rgb}{0.000000,0.000000,0.000000}%
\pgfsetstrokecolor{currentstroke}%
\pgfsetdash{}{0pt}%
\pgfsys@defobject{currentmarker}{\pgfqpoint{0.000000in}{0.000000in}}{\pgfqpoint{0.055556in}{0.000000in}}{%
\pgfpathmoveto{\pgfqpoint{0.000000in}{0.000000in}}%
\pgfpathlineto{\pgfqpoint{0.055556in}{0.000000in}}%
\pgfusepath{stroke,fill}%
}%
\begin{pgfscope}%
\pgfsys@transformshift{1.000000in}{2.220000in}%
\pgfsys@useobject{currentmarker}{}%
\end{pgfscope}%
\end{pgfscope}%
\begin{pgfscope}%
\pgfsetbuttcap%
\pgfsetroundjoin%
\definecolor{currentfill}{rgb}{0.000000,0.000000,0.000000}%
\pgfsetfillcolor{currentfill}%
\pgfsetlinewidth{0.501875pt}%
\definecolor{currentstroke}{rgb}{0.000000,0.000000,0.000000}%
\pgfsetstrokecolor{currentstroke}%
\pgfsetdash{}{0pt}%
\pgfsys@defobject{currentmarker}{\pgfqpoint{-0.055556in}{0.000000in}}{\pgfqpoint{0.000000in}{0.000000in}}{%
\pgfpathmoveto{\pgfqpoint{0.000000in}{0.000000in}}%
\pgfpathlineto{\pgfqpoint{-0.055556in}{0.000000in}}%
\pgfusepath{stroke,fill}%
}%
\begin{pgfscope}%
\pgfsys@transformshift{7.200000in}{2.220000in}%
\pgfsys@useobject{currentmarker}{}%
\end{pgfscope}%
\end{pgfscope}%
\begin{pgfscope}%
\pgftext[left,bottom,x=0.715279in,y=2.149790in,rotate=0.000000]{{\sffamily\fontsize{12.000000}{14.400000}\selectfont \(\displaystyle {10^{1}}\)}}
%
\end{pgfscope}%
\begin{pgfscope}%
\pgfpathrectangle{\pgfqpoint{1.000000in}{0.300000in}}{\pgfqpoint{6.200000in}{2.400000in}} %
\pgfusepath{clip}%
\pgfsetbuttcap%
\pgfsetroundjoin%
\pgfsetlinewidth{0.501875pt}%
\definecolor{currentstroke}{rgb}{0.000000,0.000000,0.000000}%
\pgfsetstrokecolor{currentstroke}%
\pgfsetdash{{1.000000pt}{3.000000pt}}{0.000000pt}%
\pgfpathmoveto{\pgfqpoint{1.000000in}{2.700000in}}%
\pgfpathlineto{\pgfqpoint{7.200000in}{2.700000in}}%
\pgfusepath{stroke}%
\end{pgfscope}%
\begin{pgfscope}%
\pgfsetbuttcap%
\pgfsetroundjoin%
\definecolor{currentfill}{rgb}{0.000000,0.000000,0.000000}%
\pgfsetfillcolor{currentfill}%
\pgfsetlinewidth{0.501875pt}%
\definecolor{currentstroke}{rgb}{0.000000,0.000000,0.000000}%
\pgfsetstrokecolor{currentstroke}%
\pgfsetdash{}{0pt}%
\pgfsys@defobject{currentmarker}{\pgfqpoint{0.000000in}{0.000000in}}{\pgfqpoint{0.055556in}{0.000000in}}{%
\pgfpathmoveto{\pgfqpoint{0.000000in}{0.000000in}}%
\pgfpathlineto{\pgfqpoint{0.055556in}{0.000000in}}%
\pgfusepath{stroke,fill}%
}%
\begin{pgfscope}%
\pgfsys@transformshift{1.000000in}{2.700000in}%
\pgfsys@useobject{currentmarker}{}%
\end{pgfscope}%
\end{pgfscope}%
\begin{pgfscope}%
\pgfsetbuttcap%
\pgfsetroundjoin%
\definecolor{currentfill}{rgb}{0.000000,0.000000,0.000000}%
\pgfsetfillcolor{currentfill}%
\pgfsetlinewidth{0.501875pt}%
\definecolor{currentstroke}{rgb}{0.000000,0.000000,0.000000}%
\pgfsetstrokecolor{currentstroke}%
\pgfsetdash{}{0pt}%
\pgfsys@defobject{currentmarker}{\pgfqpoint{-0.055556in}{0.000000in}}{\pgfqpoint{0.000000in}{0.000000in}}{%
\pgfpathmoveto{\pgfqpoint{0.000000in}{0.000000in}}%
\pgfpathlineto{\pgfqpoint{-0.055556in}{0.000000in}}%
\pgfusepath{stroke,fill}%
}%
\begin{pgfscope}%
\pgfsys@transformshift{7.200000in}{2.700000in}%
\pgfsys@useobject{currentmarker}{}%
\end{pgfscope}%
\end{pgfscope}%
\begin{pgfscope}%
\pgftext[left,bottom,x=0.715279in,y=2.629790in,rotate=0.000000]{{\sffamily\fontsize{12.000000}{14.400000}\selectfont \(\displaystyle {10^{2}}\)}}
%
\end{pgfscope}%
\begin{pgfscope}%
\pgfsetbuttcap%
\pgfsetroundjoin%
\definecolor{currentfill}{rgb}{0.000000,0.000000,0.000000}%
\pgfsetfillcolor{currentfill}%
\pgfsetlinewidth{0.501875pt}%
\definecolor{currentstroke}{rgb}{0.000000,0.000000,0.000000}%
\pgfsetstrokecolor{currentstroke}%
\pgfsetdash{}{0pt}%
\pgfsys@defobject{currentmarker}{\pgfqpoint{0.000000in}{0.000000in}}{\pgfqpoint{0.027778in}{0.000000in}}{%
\pgfpathmoveto{\pgfqpoint{0.000000in}{0.000000in}}%
\pgfpathlineto{\pgfqpoint{0.027778in}{0.000000in}}%
\pgfusepath{stroke,fill}%
}%
\begin{pgfscope}%
\pgfsys@transformshift{1.000000in}{0.444494in}%
\pgfsys@useobject{currentmarker}{}%
\end{pgfscope}%
\end{pgfscope}%
\begin{pgfscope}%
\pgfsetbuttcap%
\pgfsetroundjoin%
\definecolor{currentfill}{rgb}{0.000000,0.000000,0.000000}%
\pgfsetfillcolor{currentfill}%
\pgfsetlinewidth{0.501875pt}%
\definecolor{currentstroke}{rgb}{0.000000,0.000000,0.000000}%
\pgfsetstrokecolor{currentstroke}%
\pgfsetdash{}{0pt}%
\pgfsys@defobject{currentmarker}{\pgfqpoint{-0.027778in}{0.000000in}}{\pgfqpoint{0.000000in}{0.000000in}}{%
\pgfpathmoveto{\pgfqpoint{0.000000in}{0.000000in}}%
\pgfpathlineto{\pgfqpoint{-0.027778in}{0.000000in}}%
\pgfusepath{stroke,fill}%
}%
\begin{pgfscope}%
\pgfsys@transformshift{7.200000in}{0.444494in}%
\pgfsys@useobject{currentmarker}{}%
\end{pgfscope}%
\end{pgfscope}%
\begin{pgfscope}%
\pgfsetbuttcap%
\pgfsetroundjoin%
\definecolor{currentfill}{rgb}{0.000000,0.000000,0.000000}%
\pgfsetfillcolor{currentfill}%
\pgfsetlinewidth{0.501875pt}%
\definecolor{currentstroke}{rgb}{0.000000,0.000000,0.000000}%
\pgfsetstrokecolor{currentstroke}%
\pgfsetdash{}{0pt}%
\pgfsys@defobject{currentmarker}{\pgfqpoint{0.000000in}{0.000000in}}{\pgfqpoint{0.027778in}{0.000000in}}{%
\pgfpathmoveto{\pgfqpoint{0.000000in}{0.000000in}}%
\pgfpathlineto{\pgfqpoint{0.027778in}{0.000000in}}%
\pgfusepath{stroke,fill}%
}%
\begin{pgfscope}%
\pgfsys@transformshift{1.000000in}{0.529018in}%
\pgfsys@useobject{currentmarker}{}%
\end{pgfscope}%
\end{pgfscope}%
\begin{pgfscope}%
\pgfsetbuttcap%
\pgfsetroundjoin%
\definecolor{currentfill}{rgb}{0.000000,0.000000,0.000000}%
\pgfsetfillcolor{currentfill}%
\pgfsetlinewidth{0.501875pt}%
\definecolor{currentstroke}{rgb}{0.000000,0.000000,0.000000}%
\pgfsetstrokecolor{currentstroke}%
\pgfsetdash{}{0pt}%
\pgfsys@defobject{currentmarker}{\pgfqpoint{-0.027778in}{0.000000in}}{\pgfqpoint{0.000000in}{0.000000in}}{%
\pgfpathmoveto{\pgfqpoint{0.000000in}{0.000000in}}%
\pgfpathlineto{\pgfqpoint{-0.027778in}{0.000000in}}%
\pgfusepath{stroke,fill}%
}%
\begin{pgfscope}%
\pgfsys@transformshift{7.200000in}{0.529018in}%
\pgfsys@useobject{currentmarker}{}%
\end{pgfscope}%
\end{pgfscope}%
\begin{pgfscope}%
\pgfsetbuttcap%
\pgfsetroundjoin%
\definecolor{currentfill}{rgb}{0.000000,0.000000,0.000000}%
\pgfsetfillcolor{currentfill}%
\pgfsetlinewidth{0.501875pt}%
\definecolor{currentstroke}{rgb}{0.000000,0.000000,0.000000}%
\pgfsetstrokecolor{currentstroke}%
\pgfsetdash{}{0pt}%
\pgfsys@defobject{currentmarker}{\pgfqpoint{0.000000in}{0.000000in}}{\pgfqpoint{0.027778in}{0.000000in}}{%
\pgfpathmoveto{\pgfqpoint{0.000000in}{0.000000in}}%
\pgfpathlineto{\pgfqpoint{0.027778in}{0.000000in}}%
\pgfusepath{stroke,fill}%
}%
\begin{pgfscope}%
\pgfsys@transformshift{1.000000in}{0.588989in}%
\pgfsys@useobject{currentmarker}{}%
\end{pgfscope}%
\end{pgfscope}%
\begin{pgfscope}%
\pgfsetbuttcap%
\pgfsetroundjoin%
\definecolor{currentfill}{rgb}{0.000000,0.000000,0.000000}%
\pgfsetfillcolor{currentfill}%
\pgfsetlinewidth{0.501875pt}%
\definecolor{currentstroke}{rgb}{0.000000,0.000000,0.000000}%
\pgfsetstrokecolor{currentstroke}%
\pgfsetdash{}{0pt}%
\pgfsys@defobject{currentmarker}{\pgfqpoint{-0.027778in}{0.000000in}}{\pgfqpoint{0.000000in}{0.000000in}}{%
\pgfpathmoveto{\pgfqpoint{0.000000in}{0.000000in}}%
\pgfpathlineto{\pgfqpoint{-0.027778in}{0.000000in}}%
\pgfusepath{stroke,fill}%
}%
\begin{pgfscope}%
\pgfsys@transformshift{7.200000in}{0.588989in}%
\pgfsys@useobject{currentmarker}{}%
\end{pgfscope}%
\end{pgfscope}%
\begin{pgfscope}%
\pgfsetbuttcap%
\pgfsetroundjoin%
\definecolor{currentfill}{rgb}{0.000000,0.000000,0.000000}%
\pgfsetfillcolor{currentfill}%
\pgfsetlinewidth{0.501875pt}%
\definecolor{currentstroke}{rgb}{0.000000,0.000000,0.000000}%
\pgfsetstrokecolor{currentstroke}%
\pgfsetdash{}{0pt}%
\pgfsys@defobject{currentmarker}{\pgfqpoint{0.000000in}{0.000000in}}{\pgfqpoint{0.027778in}{0.000000in}}{%
\pgfpathmoveto{\pgfqpoint{0.000000in}{0.000000in}}%
\pgfpathlineto{\pgfqpoint{0.027778in}{0.000000in}}%
\pgfusepath{stroke,fill}%
}%
\begin{pgfscope}%
\pgfsys@transformshift{1.000000in}{0.635506in}%
\pgfsys@useobject{currentmarker}{}%
\end{pgfscope}%
\end{pgfscope}%
\begin{pgfscope}%
\pgfsetbuttcap%
\pgfsetroundjoin%
\definecolor{currentfill}{rgb}{0.000000,0.000000,0.000000}%
\pgfsetfillcolor{currentfill}%
\pgfsetlinewidth{0.501875pt}%
\definecolor{currentstroke}{rgb}{0.000000,0.000000,0.000000}%
\pgfsetstrokecolor{currentstroke}%
\pgfsetdash{}{0pt}%
\pgfsys@defobject{currentmarker}{\pgfqpoint{-0.027778in}{0.000000in}}{\pgfqpoint{0.000000in}{0.000000in}}{%
\pgfpathmoveto{\pgfqpoint{0.000000in}{0.000000in}}%
\pgfpathlineto{\pgfqpoint{-0.027778in}{0.000000in}}%
\pgfusepath{stroke,fill}%
}%
\begin{pgfscope}%
\pgfsys@transformshift{7.200000in}{0.635506in}%
\pgfsys@useobject{currentmarker}{}%
\end{pgfscope}%
\end{pgfscope}%
\begin{pgfscope}%
\pgfsetbuttcap%
\pgfsetroundjoin%
\definecolor{currentfill}{rgb}{0.000000,0.000000,0.000000}%
\pgfsetfillcolor{currentfill}%
\pgfsetlinewidth{0.501875pt}%
\definecolor{currentstroke}{rgb}{0.000000,0.000000,0.000000}%
\pgfsetstrokecolor{currentstroke}%
\pgfsetdash{}{0pt}%
\pgfsys@defobject{currentmarker}{\pgfqpoint{0.000000in}{0.000000in}}{\pgfqpoint{0.027778in}{0.000000in}}{%
\pgfpathmoveto{\pgfqpoint{0.000000in}{0.000000in}}%
\pgfpathlineto{\pgfqpoint{0.027778in}{0.000000in}}%
\pgfusepath{stroke,fill}%
}%
\begin{pgfscope}%
\pgfsys@transformshift{1.000000in}{0.673513in}%
\pgfsys@useobject{currentmarker}{}%
\end{pgfscope}%
\end{pgfscope}%
\begin{pgfscope}%
\pgfsetbuttcap%
\pgfsetroundjoin%
\definecolor{currentfill}{rgb}{0.000000,0.000000,0.000000}%
\pgfsetfillcolor{currentfill}%
\pgfsetlinewidth{0.501875pt}%
\definecolor{currentstroke}{rgb}{0.000000,0.000000,0.000000}%
\pgfsetstrokecolor{currentstroke}%
\pgfsetdash{}{0pt}%
\pgfsys@defobject{currentmarker}{\pgfqpoint{-0.027778in}{0.000000in}}{\pgfqpoint{0.000000in}{0.000000in}}{%
\pgfpathmoveto{\pgfqpoint{0.000000in}{0.000000in}}%
\pgfpathlineto{\pgfqpoint{-0.027778in}{0.000000in}}%
\pgfusepath{stroke,fill}%
}%
\begin{pgfscope}%
\pgfsys@transformshift{7.200000in}{0.673513in}%
\pgfsys@useobject{currentmarker}{}%
\end{pgfscope}%
\end{pgfscope}%
\begin{pgfscope}%
\pgfsetbuttcap%
\pgfsetroundjoin%
\definecolor{currentfill}{rgb}{0.000000,0.000000,0.000000}%
\pgfsetfillcolor{currentfill}%
\pgfsetlinewidth{0.501875pt}%
\definecolor{currentstroke}{rgb}{0.000000,0.000000,0.000000}%
\pgfsetstrokecolor{currentstroke}%
\pgfsetdash{}{0pt}%
\pgfsys@defobject{currentmarker}{\pgfqpoint{0.000000in}{0.000000in}}{\pgfqpoint{0.027778in}{0.000000in}}{%
\pgfpathmoveto{\pgfqpoint{0.000000in}{0.000000in}}%
\pgfpathlineto{\pgfqpoint{0.027778in}{0.000000in}}%
\pgfusepath{stroke,fill}%
}%
\begin{pgfscope}%
\pgfsys@transformshift{1.000000in}{0.705647in}%
\pgfsys@useobject{currentmarker}{}%
\end{pgfscope}%
\end{pgfscope}%
\begin{pgfscope}%
\pgfsetbuttcap%
\pgfsetroundjoin%
\definecolor{currentfill}{rgb}{0.000000,0.000000,0.000000}%
\pgfsetfillcolor{currentfill}%
\pgfsetlinewidth{0.501875pt}%
\definecolor{currentstroke}{rgb}{0.000000,0.000000,0.000000}%
\pgfsetstrokecolor{currentstroke}%
\pgfsetdash{}{0pt}%
\pgfsys@defobject{currentmarker}{\pgfqpoint{-0.027778in}{0.000000in}}{\pgfqpoint{0.000000in}{0.000000in}}{%
\pgfpathmoveto{\pgfqpoint{0.000000in}{0.000000in}}%
\pgfpathlineto{\pgfqpoint{-0.027778in}{0.000000in}}%
\pgfusepath{stroke,fill}%
}%
\begin{pgfscope}%
\pgfsys@transformshift{7.200000in}{0.705647in}%
\pgfsys@useobject{currentmarker}{}%
\end{pgfscope}%
\end{pgfscope}%
\begin{pgfscope}%
\pgfsetbuttcap%
\pgfsetroundjoin%
\definecolor{currentfill}{rgb}{0.000000,0.000000,0.000000}%
\pgfsetfillcolor{currentfill}%
\pgfsetlinewidth{0.501875pt}%
\definecolor{currentstroke}{rgb}{0.000000,0.000000,0.000000}%
\pgfsetstrokecolor{currentstroke}%
\pgfsetdash{}{0pt}%
\pgfsys@defobject{currentmarker}{\pgfqpoint{0.000000in}{0.000000in}}{\pgfqpoint{0.027778in}{0.000000in}}{%
\pgfpathmoveto{\pgfqpoint{0.000000in}{0.000000in}}%
\pgfpathlineto{\pgfqpoint{0.027778in}{0.000000in}}%
\pgfusepath{stroke,fill}%
}%
\begin{pgfscope}%
\pgfsys@transformshift{1.000000in}{0.733483in}%
\pgfsys@useobject{currentmarker}{}%
\end{pgfscope}%
\end{pgfscope}%
\begin{pgfscope}%
\pgfsetbuttcap%
\pgfsetroundjoin%
\definecolor{currentfill}{rgb}{0.000000,0.000000,0.000000}%
\pgfsetfillcolor{currentfill}%
\pgfsetlinewidth{0.501875pt}%
\definecolor{currentstroke}{rgb}{0.000000,0.000000,0.000000}%
\pgfsetstrokecolor{currentstroke}%
\pgfsetdash{}{0pt}%
\pgfsys@defobject{currentmarker}{\pgfqpoint{-0.027778in}{0.000000in}}{\pgfqpoint{0.000000in}{0.000000in}}{%
\pgfpathmoveto{\pgfqpoint{0.000000in}{0.000000in}}%
\pgfpathlineto{\pgfqpoint{-0.027778in}{0.000000in}}%
\pgfusepath{stroke,fill}%
}%
\begin{pgfscope}%
\pgfsys@transformshift{7.200000in}{0.733483in}%
\pgfsys@useobject{currentmarker}{}%
\end{pgfscope}%
\end{pgfscope}%
\begin{pgfscope}%
\pgfsetbuttcap%
\pgfsetroundjoin%
\definecolor{currentfill}{rgb}{0.000000,0.000000,0.000000}%
\pgfsetfillcolor{currentfill}%
\pgfsetlinewidth{0.501875pt}%
\definecolor{currentstroke}{rgb}{0.000000,0.000000,0.000000}%
\pgfsetstrokecolor{currentstroke}%
\pgfsetdash{}{0pt}%
\pgfsys@defobject{currentmarker}{\pgfqpoint{0.000000in}{0.000000in}}{\pgfqpoint{0.027778in}{0.000000in}}{%
\pgfpathmoveto{\pgfqpoint{0.000000in}{0.000000in}}%
\pgfpathlineto{\pgfqpoint{0.027778in}{0.000000in}}%
\pgfusepath{stroke,fill}%
}%
\begin{pgfscope}%
\pgfsys@transformshift{1.000000in}{0.758036in}%
\pgfsys@useobject{currentmarker}{}%
\end{pgfscope}%
\end{pgfscope}%
\begin{pgfscope}%
\pgfsetbuttcap%
\pgfsetroundjoin%
\definecolor{currentfill}{rgb}{0.000000,0.000000,0.000000}%
\pgfsetfillcolor{currentfill}%
\pgfsetlinewidth{0.501875pt}%
\definecolor{currentstroke}{rgb}{0.000000,0.000000,0.000000}%
\pgfsetstrokecolor{currentstroke}%
\pgfsetdash{}{0pt}%
\pgfsys@defobject{currentmarker}{\pgfqpoint{-0.027778in}{0.000000in}}{\pgfqpoint{0.000000in}{0.000000in}}{%
\pgfpathmoveto{\pgfqpoint{0.000000in}{0.000000in}}%
\pgfpathlineto{\pgfqpoint{-0.027778in}{0.000000in}}%
\pgfusepath{stroke,fill}%
}%
\begin{pgfscope}%
\pgfsys@transformshift{7.200000in}{0.758036in}%
\pgfsys@useobject{currentmarker}{}%
\end{pgfscope}%
\end{pgfscope}%
\begin{pgfscope}%
\pgfsetbuttcap%
\pgfsetroundjoin%
\definecolor{currentfill}{rgb}{0.000000,0.000000,0.000000}%
\pgfsetfillcolor{currentfill}%
\pgfsetlinewidth{0.501875pt}%
\definecolor{currentstroke}{rgb}{0.000000,0.000000,0.000000}%
\pgfsetstrokecolor{currentstroke}%
\pgfsetdash{}{0pt}%
\pgfsys@defobject{currentmarker}{\pgfqpoint{0.000000in}{0.000000in}}{\pgfqpoint{0.027778in}{0.000000in}}{%
\pgfpathmoveto{\pgfqpoint{0.000000in}{0.000000in}}%
\pgfpathlineto{\pgfqpoint{0.027778in}{0.000000in}}%
\pgfusepath{stroke,fill}%
}%
\begin{pgfscope}%
\pgfsys@transformshift{1.000000in}{0.924494in}%
\pgfsys@useobject{currentmarker}{}%
\end{pgfscope}%
\end{pgfscope}%
\begin{pgfscope}%
\pgfsetbuttcap%
\pgfsetroundjoin%
\definecolor{currentfill}{rgb}{0.000000,0.000000,0.000000}%
\pgfsetfillcolor{currentfill}%
\pgfsetlinewidth{0.501875pt}%
\definecolor{currentstroke}{rgb}{0.000000,0.000000,0.000000}%
\pgfsetstrokecolor{currentstroke}%
\pgfsetdash{}{0pt}%
\pgfsys@defobject{currentmarker}{\pgfqpoint{-0.027778in}{0.000000in}}{\pgfqpoint{0.000000in}{0.000000in}}{%
\pgfpathmoveto{\pgfqpoint{0.000000in}{0.000000in}}%
\pgfpathlineto{\pgfqpoint{-0.027778in}{0.000000in}}%
\pgfusepath{stroke,fill}%
}%
\begin{pgfscope}%
\pgfsys@transformshift{7.200000in}{0.924494in}%
\pgfsys@useobject{currentmarker}{}%
\end{pgfscope}%
\end{pgfscope}%
\begin{pgfscope}%
\pgfsetbuttcap%
\pgfsetroundjoin%
\definecolor{currentfill}{rgb}{0.000000,0.000000,0.000000}%
\pgfsetfillcolor{currentfill}%
\pgfsetlinewidth{0.501875pt}%
\definecolor{currentstroke}{rgb}{0.000000,0.000000,0.000000}%
\pgfsetstrokecolor{currentstroke}%
\pgfsetdash{}{0pt}%
\pgfsys@defobject{currentmarker}{\pgfqpoint{0.000000in}{0.000000in}}{\pgfqpoint{0.027778in}{0.000000in}}{%
\pgfpathmoveto{\pgfqpoint{0.000000in}{0.000000in}}%
\pgfpathlineto{\pgfqpoint{0.027778in}{0.000000in}}%
\pgfusepath{stroke,fill}%
}%
\begin{pgfscope}%
\pgfsys@transformshift{1.000000in}{1.009018in}%
\pgfsys@useobject{currentmarker}{}%
\end{pgfscope}%
\end{pgfscope}%
\begin{pgfscope}%
\pgfsetbuttcap%
\pgfsetroundjoin%
\definecolor{currentfill}{rgb}{0.000000,0.000000,0.000000}%
\pgfsetfillcolor{currentfill}%
\pgfsetlinewidth{0.501875pt}%
\definecolor{currentstroke}{rgb}{0.000000,0.000000,0.000000}%
\pgfsetstrokecolor{currentstroke}%
\pgfsetdash{}{0pt}%
\pgfsys@defobject{currentmarker}{\pgfqpoint{-0.027778in}{0.000000in}}{\pgfqpoint{0.000000in}{0.000000in}}{%
\pgfpathmoveto{\pgfqpoint{0.000000in}{0.000000in}}%
\pgfpathlineto{\pgfqpoint{-0.027778in}{0.000000in}}%
\pgfusepath{stroke,fill}%
}%
\begin{pgfscope}%
\pgfsys@transformshift{7.200000in}{1.009018in}%
\pgfsys@useobject{currentmarker}{}%
\end{pgfscope}%
\end{pgfscope}%
\begin{pgfscope}%
\pgfsetbuttcap%
\pgfsetroundjoin%
\definecolor{currentfill}{rgb}{0.000000,0.000000,0.000000}%
\pgfsetfillcolor{currentfill}%
\pgfsetlinewidth{0.501875pt}%
\definecolor{currentstroke}{rgb}{0.000000,0.000000,0.000000}%
\pgfsetstrokecolor{currentstroke}%
\pgfsetdash{}{0pt}%
\pgfsys@defobject{currentmarker}{\pgfqpoint{0.000000in}{0.000000in}}{\pgfqpoint{0.027778in}{0.000000in}}{%
\pgfpathmoveto{\pgfqpoint{0.000000in}{0.000000in}}%
\pgfpathlineto{\pgfqpoint{0.027778in}{0.000000in}}%
\pgfusepath{stroke,fill}%
}%
\begin{pgfscope}%
\pgfsys@transformshift{1.000000in}{1.068989in}%
\pgfsys@useobject{currentmarker}{}%
\end{pgfscope}%
\end{pgfscope}%
\begin{pgfscope}%
\pgfsetbuttcap%
\pgfsetroundjoin%
\definecolor{currentfill}{rgb}{0.000000,0.000000,0.000000}%
\pgfsetfillcolor{currentfill}%
\pgfsetlinewidth{0.501875pt}%
\definecolor{currentstroke}{rgb}{0.000000,0.000000,0.000000}%
\pgfsetstrokecolor{currentstroke}%
\pgfsetdash{}{0pt}%
\pgfsys@defobject{currentmarker}{\pgfqpoint{-0.027778in}{0.000000in}}{\pgfqpoint{0.000000in}{0.000000in}}{%
\pgfpathmoveto{\pgfqpoint{0.000000in}{0.000000in}}%
\pgfpathlineto{\pgfqpoint{-0.027778in}{0.000000in}}%
\pgfusepath{stroke,fill}%
}%
\begin{pgfscope}%
\pgfsys@transformshift{7.200000in}{1.068989in}%
\pgfsys@useobject{currentmarker}{}%
\end{pgfscope}%
\end{pgfscope}%
\begin{pgfscope}%
\pgfsetbuttcap%
\pgfsetroundjoin%
\definecolor{currentfill}{rgb}{0.000000,0.000000,0.000000}%
\pgfsetfillcolor{currentfill}%
\pgfsetlinewidth{0.501875pt}%
\definecolor{currentstroke}{rgb}{0.000000,0.000000,0.000000}%
\pgfsetstrokecolor{currentstroke}%
\pgfsetdash{}{0pt}%
\pgfsys@defobject{currentmarker}{\pgfqpoint{0.000000in}{0.000000in}}{\pgfqpoint{0.027778in}{0.000000in}}{%
\pgfpathmoveto{\pgfqpoint{0.000000in}{0.000000in}}%
\pgfpathlineto{\pgfqpoint{0.027778in}{0.000000in}}%
\pgfusepath{stroke,fill}%
}%
\begin{pgfscope}%
\pgfsys@transformshift{1.000000in}{1.115506in}%
\pgfsys@useobject{currentmarker}{}%
\end{pgfscope}%
\end{pgfscope}%
\begin{pgfscope}%
\pgfsetbuttcap%
\pgfsetroundjoin%
\definecolor{currentfill}{rgb}{0.000000,0.000000,0.000000}%
\pgfsetfillcolor{currentfill}%
\pgfsetlinewidth{0.501875pt}%
\definecolor{currentstroke}{rgb}{0.000000,0.000000,0.000000}%
\pgfsetstrokecolor{currentstroke}%
\pgfsetdash{}{0pt}%
\pgfsys@defobject{currentmarker}{\pgfqpoint{-0.027778in}{0.000000in}}{\pgfqpoint{0.000000in}{0.000000in}}{%
\pgfpathmoveto{\pgfqpoint{0.000000in}{0.000000in}}%
\pgfpathlineto{\pgfqpoint{-0.027778in}{0.000000in}}%
\pgfusepath{stroke,fill}%
}%
\begin{pgfscope}%
\pgfsys@transformshift{7.200000in}{1.115506in}%
\pgfsys@useobject{currentmarker}{}%
\end{pgfscope}%
\end{pgfscope}%
\begin{pgfscope}%
\pgfsetbuttcap%
\pgfsetroundjoin%
\definecolor{currentfill}{rgb}{0.000000,0.000000,0.000000}%
\pgfsetfillcolor{currentfill}%
\pgfsetlinewidth{0.501875pt}%
\definecolor{currentstroke}{rgb}{0.000000,0.000000,0.000000}%
\pgfsetstrokecolor{currentstroke}%
\pgfsetdash{}{0pt}%
\pgfsys@defobject{currentmarker}{\pgfqpoint{0.000000in}{0.000000in}}{\pgfqpoint{0.027778in}{0.000000in}}{%
\pgfpathmoveto{\pgfqpoint{0.000000in}{0.000000in}}%
\pgfpathlineto{\pgfqpoint{0.027778in}{0.000000in}}%
\pgfusepath{stroke,fill}%
}%
\begin{pgfscope}%
\pgfsys@transformshift{1.000000in}{1.153513in}%
\pgfsys@useobject{currentmarker}{}%
\end{pgfscope}%
\end{pgfscope}%
\begin{pgfscope}%
\pgfsetbuttcap%
\pgfsetroundjoin%
\definecolor{currentfill}{rgb}{0.000000,0.000000,0.000000}%
\pgfsetfillcolor{currentfill}%
\pgfsetlinewidth{0.501875pt}%
\definecolor{currentstroke}{rgb}{0.000000,0.000000,0.000000}%
\pgfsetstrokecolor{currentstroke}%
\pgfsetdash{}{0pt}%
\pgfsys@defobject{currentmarker}{\pgfqpoint{-0.027778in}{0.000000in}}{\pgfqpoint{0.000000in}{0.000000in}}{%
\pgfpathmoveto{\pgfqpoint{0.000000in}{0.000000in}}%
\pgfpathlineto{\pgfqpoint{-0.027778in}{0.000000in}}%
\pgfusepath{stroke,fill}%
}%
\begin{pgfscope}%
\pgfsys@transformshift{7.200000in}{1.153513in}%
\pgfsys@useobject{currentmarker}{}%
\end{pgfscope}%
\end{pgfscope}%
\begin{pgfscope}%
\pgfsetbuttcap%
\pgfsetroundjoin%
\definecolor{currentfill}{rgb}{0.000000,0.000000,0.000000}%
\pgfsetfillcolor{currentfill}%
\pgfsetlinewidth{0.501875pt}%
\definecolor{currentstroke}{rgb}{0.000000,0.000000,0.000000}%
\pgfsetstrokecolor{currentstroke}%
\pgfsetdash{}{0pt}%
\pgfsys@defobject{currentmarker}{\pgfqpoint{0.000000in}{0.000000in}}{\pgfqpoint{0.027778in}{0.000000in}}{%
\pgfpathmoveto{\pgfqpoint{0.000000in}{0.000000in}}%
\pgfpathlineto{\pgfqpoint{0.027778in}{0.000000in}}%
\pgfusepath{stroke,fill}%
}%
\begin{pgfscope}%
\pgfsys@transformshift{1.000000in}{1.185647in}%
\pgfsys@useobject{currentmarker}{}%
\end{pgfscope}%
\end{pgfscope}%
\begin{pgfscope}%
\pgfsetbuttcap%
\pgfsetroundjoin%
\definecolor{currentfill}{rgb}{0.000000,0.000000,0.000000}%
\pgfsetfillcolor{currentfill}%
\pgfsetlinewidth{0.501875pt}%
\definecolor{currentstroke}{rgb}{0.000000,0.000000,0.000000}%
\pgfsetstrokecolor{currentstroke}%
\pgfsetdash{}{0pt}%
\pgfsys@defobject{currentmarker}{\pgfqpoint{-0.027778in}{0.000000in}}{\pgfqpoint{0.000000in}{0.000000in}}{%
\pgfpathmoveto{\pgfqpoint{0.000000in}{0.000000in}}%
\pgfpathlineto{\pgfqpoint{-0.027778in}{0.000000in}}%
\pgfusepath{stroke,fill}%
}%
\begin{pgfscope}%
\pgfsys@transformshift{7.200000in}{1.185647in}%
\pgfsys@useobject{currentmarker}{}%
\end{pgfscope}%
\end{pgfscope}%
\begin{pgfscope}%
\pgfsetbuttcap%
\pgfsetroundjoin%
\definecolor{currentfill}{rgb}{0.000000,0.000000,0.000000}%
\pgfsetfillcolor{currentfill}%
\pgfsetlinewidth{0.501875pt}%
\definecolor{currentstroke}{rgb}{0.000000,0.000000,0.000000}%
\pgfsetstrokecolor{currentstroke}%
\pgfsetdash{}{0pt}%
\pgfsys@defobject{currentmarker}{\pgfqpoint{0.000000in}{0.000000in}}{\pgfqpoint{0.027778in}{0.000000in}}{%
\pgfpathmoveto{\pgfqpoint{0.000000in}{0.000000in}}%
\pgfpathlineto{\pgfqpoint{0.027778in}{0.000000in}}%
\pgfusepath{stroke,fill}%
}%
\begin{pgfscope}%
\pgfsys@transformshift{1.000000in}{1.213483in}%
\pgfsys@useobject{currentmarker}{}%
\end{pgfscope}%
\end{pgfscope}%
\begin{pgfscope}%
\pgfsetbuttcap%
\pgfsetroundjoin%
\definecolor{currentfill}{rgb}{0.000000,0.000000,0.000000}%
\pgfsetfillcolor{currentfill}%
\pgfsetlinewidth{0.501875pt}%
\definecolor{currentstroke}{rgb}{0.000000,0.000000,0.000000}%
\pgfsetstrokecolor{currentstroke}%
\pgfsetdash{}{0pt}%
\pgfsys@defobject{currentmarker}{\pgfqpoint{-0.027778in}{0.000000in}}{\pgfqpoint{0.000000in}{0.000000in}}{%
\pgfpathmoveto{\pgfqpoint{0.000000in}{0.000000in}}%
\pgfpathlineto{\pgfqpoint{-0.027778in}{0.000000in}}%
\pgfusepath{stroke,fill}%
}%
\begin{pgfscope}%
\pgfsys@transformshift{7.200000in}{1.213483in}%
\pgfsys@useobject{currentmarker}{}%
\end{pgfscope}%
\end{pgfscope}%
\begin{pgfscope}%
\pgfsetbuttcap%
\pgfsetroundjoin%
\definecolor{currentfill}{rgb}{0.000000,0.000000,0.000000}%
\pgfsetfillcolor{currentfill}%
\pgfsetlinewidth{0.501875pt}%
\definecolor{currentstroke}{rgb}{0.000000,0.000000,0.000000}%
\pgfsetstrokecolor{currentstroke}%
\pgfsetdash{}{0pt}%
\pgfsys@defobject{currentmarker}{\pgfqpoint{0.000000in}{0.000000in}}{\pgfqpoint{0.027778in}{0.000000in}}{%
\pgfpathmoveto{\pgfqpoint{0.000000in}{0.000000in}}%
\pgfpathlineto{\pgfqpoint{0.027778in}{0.000000in}}%
\pgfusepath{stroke,fill}%
}%
\begin{pgfscope}%
\pgfsys@transformshift{1.000000in}{1.238036in}%
\pgfsys@useobject{currentmarker}{}%
\end{pgfscope}%
\end{pgfscope}%
\begin{pgfscope}%
\pgfsetbuttcap%
\pgfsetroundjoin%
\definecolor{currentfill}{rgb}{0.000000,0.000000,0.000000}%
\pgfsetfillcolor{currentfill}%
\pgfsetlinewidth{0.501875pt}%
\definecolor{currentstroke}{rgb}{0.000000,0.000000,0.000000}%
\pgfsetstrokecolor{currentstroke}%
\pgfsetdash{}{0pt}%
\pgfsys@defobject{currentmarker}{\pgfqpoint{-0.027778in}{0.000000in}}{\pgfqpoint{0.000000in}{0.000000in}}{%
\pgfpathmoveto{\pgfqpoint{0.000000in}{0.000000in}}%
\pgfpathlineto{\pgfqpoint{-0.027778in}{0.000000in}}%
\pgfusepath{stroke,fill}%
}%
\begin{pgfscope}%
\pgfsys@transformshift{7.200000in}{1.238036in}%
\pgfsys@useobject{currentmarker}{}%
\end{pgfscope}%
\end{pgfscope}%
\begin{pgfscope}%
\pgfsetbuttcap%
\pgfsetroundjoin%
\definecolor{currentfill}{rgb}{0.000000,0.000000,0.000000}%
\pgfsetfillcolor{currentfill}%
\pgfsetlinewidth{0.501875pt}%
\definecolor{currentstroke}{rgb}{0.000000,0.000000,0.000000}%
\pgfsetstrokecolor{currentstroke}%
\pgfsetdash{}{0pt}%
\pgfsys@defobject{currentmarker}{\pgfqpoint{0.000000in}{0.000000in}}{\pgfqpoint{0.027778in}{0.000000in}}{%
\pgfpathmoveto{\pgfqpoint{0.000000in}{0.000000in}}%
\pgfpathlineto{\pgfqpoint{0.027778in}{0.000000in}}%
\pgfusepath{stroke,fill}%
}%
\begin{pgfscope}%
\pgfsys@transformshift{1.000000in}{1.404494in}%
\pgfsys@useobject{currentmarker}{}%
\end{pgfscope}%
\end{pgfscope}%
\begin{pgfscope}%
\pgfsetbuttcap%
\pgfsetroundjoin%
\definecolor{currentfill}{rgb}{0.000000,0.000000,0.000000}%
\pgfsetfillcolor{currentfill}%
\pgfsetlinewidth{0.501875pt}%
\definecolor{currentstroke}{rgb}{0.000000,0.000000,0.000000}%
\pgfsetstrokecolor{currentstroke}%
\pgfsetdash{}{0pt}%
\pgfsys@defobject{currentmarker}{\pgfqpoint{-0.027778in}{0.000000in}}{\pgfqpoint{0.000000in}{0.000000in}}{%
\pgfpathmoveto{\pgfqpoint{0.000000in}{0.000000in}}%
\pgfpathlineto{\pgfqpoint{-0.027778in}{0.000000in}}%
\pgfusepath{stroke,fill}%
}%
\begin{pgfscope}%
\pgfsys@transformshift{7.200000in}{1.404494in}%
\pgfsys@useobject{currentmarker}{}%
\end{pgfscope}%
\end{pgfscope}%
\begin{pgfscope}%
\pgfsetbuttcap%
\pgfsetroundjoin%
\definecolor{currentfill}{rgb}{0.000000,0.000000,0.000000}%
\pgfsetfillcolor{currentfill}%
\pgfsetlinewidth{0.501875pt}%
\definecolor{currentstroke}{rgb}{0.000000,0.000000,0.000000}%
\pgfsetstrokecolor{currentstroke}%
\pgfsetdash{}{0pt}%
\pgfsys@defobject{currentmarker}{\pgfqpoint{0.000000in}{0.000000in}}{\pgfqpoint{0.027778in}{0.000000in}}{%
\pgfpathmoveto{\pgfqpoint{0.000000in}{0.000000in}}%
\pgfpathlineto{\pgfqpoint{0.027778in}{0.000000in}}%
\pgfusepath{stroke,fill}%
}%
\begin{pgfscope}%
\pgfsys@transformshift{1.000000in}{1.489018in}%
\pgfsys@useobject{currentmarker}{}%
\end{pgfscope}%
\end{pgfscope}%
\begin{pgfscope}%
\pgfsetbuttcap%
\pgfsetroundjoin%
\definecolor{currentfill}{rgb}{0.000000,0.000000,0.000000}%
\pgfsetfillcolor{currentfill}%
\pgfsetlinewidth{0.501875pt}%
\definecolor{currentstroke}{rgb}{0.000000,0.000000,0.000000}%
\pgfsetstrokecolor{currentstroke}%
\pgfsetdash{}{0pt}%
\pgfsys@defobject{currentmarker}{\pgfqpoint{-0.027778in}{0.000000in}}{\pgfqpoint{0.000000in}{0.000000in}}{%
\pgfpathmoveto{\pgfqpoint{0.000000in}{0.000000in}}%
\pgfpathlineto{\pgfqpoint{-0.027778in}{0.000000in}}%
\pgfusepath{stroke,fill}%
}%
\begin{pgfscope}%
\pgfsys@transformshift{7.200000in}{1.489018in}%
\pgfsys@useobject{currentmarker}{}%
\end{pgfscope}%
\end{pgfscope}%
\begin{pgfscope}%
\pgfsetbuttcap%
\pgfsetroundjoin%
\definecolor{currentfill}{rgb}{0.000000,0.000000,0.000000}%
\pgfsetfillcolor{currentfill}%
\pgfsetlinewidth{0.501875pt}%
\definecolor{currentstroke}{rgb}{0.000000,0.000000,0.000000}%
\pgfsetstrokecolor{currentstroke}%
\pgfsetdash{}{0pt}%
\pgfsys@defobject{currentmarker}{\pgfqpoint{0.000000in}{0.000000in}}{\pgfqpoint{0.027778in}{0.000000in}}{%
\pgfpathmoveto{\pgfqpoint{0.000000in}{0.000000in}}%
\pgfpathlineto{\pgfqpoint{0.027778in}{0.000000in}}%
\pgfusepath{stroke,fill}%
}%
\begin{pgfscope}%
\pgfsys@transformshift{1.000000in}{1.548989in}%
\pgfsys@useobject{currentmarker}{}%
\end{pgfscope}%
\end{pgfscope}%
\begin{pgfscope}%
\pgfsetbuttcap%
\pgfsetroundjoin%
\definecolor{currentfill}{rgb}{0.000000,0.000000,0.000000}%
\pgfsetfillcolor{currentfill}%
\pgfsetlinewidth{0.501875pt}%
\definecolor{currentstroke}{rgb}{0.000000,0.000000,0.000000}%
\pgfsetstrokecolor{currentstroke}%
\pgfsetdash{}{0pt}%
\pgfsys@defobject{currentmarker}{\pgfqpoint{-0.027778in}{0.000000in}}{\pgfqpoint{0.000000in}{0.000000in}}{%
\pgfpathmoveto{\pgfqpoint{0.000000in}{0.000000in}}%
\pgfpathlineto{\pgfqpoint{-0.027778in}{0.000000in}}%
\pgfusepath{stroke,fill}%
}%
\begin{pgfscope}%
\pgfsys@transformshift{7.200000in}{1.548989in}%
\pgfsys@useobject{currentmarker}{}%
\end{pgfscope}%
\end{pgfscope}%
\begin{pgfscope}%
\pgfsetbuttcap%
\pgfsetroundjoin%
\definecolor{currentfill}{rgb}{0.000000,0.000000,0.000000}%
\pgfsetfillcolor{currentfill}%
\pgfsetlinewidth{0.501875pt}%
\definecolor{currentstroke}{rgb}{0.000000,0.000000,0.000000}%
\pgfsetstrokecolor{currentstroke}%
\pgfsetdash{}{0pt}%
\pgfsys@defobject{currentmarker}{\pgfqpoint{0.000000in}{0.000000in}}{\pgfqpoint{0.027778in}{0.000000in}}{%
\pgfpathmoveto{\pgfqpoint{0.000000in}{0.000000in}}%
\pgfpathlineto{\pgfqpoint{0.027778in}{0.000000in}}%
\pgfusepath{stroke,fill}%
}%
\begin{pgfscope}%
\pgfsys@transformshift{1.000000in}{1.595506in}%
\pgfsys@useobject{currentmarker}{}%
\end{pgfscope}%
\end{pgfscope}%
\begin{pgfscope}%
\pgfsetbuttcap%
\pgfsetroundjoin%
\definecolor{currentfill}{rgb}{0.000000,0.000000,0.000000}%
\pgfsetfillcolor{currentfill}%
\pgfsetlinewidth{0.501875pt}%
\definecolor{currentstroke}{rgb}{0.000000,0.000000,0.000000}%
\pgfsetstrokecolor{currentstroke}%
\pgfsetdash{}{0pt}%
\pgfsys@defobject{currentmarker}{\pgfqpoint{-0.027778in}{0.000000in}}{\pgfqpoint{0.000000in}{0.000000in}}{%
\pgfpathmoveto{\pgfqpoint{0.000000in}{0.000000in}}%
\pgfpathlineto{\pgfqpoint{-0.027778in}{0.000000in}}%
\pgfusepath{stroke,fill}%
}%
\begin{pgfscope}%
\pgfsys@transformshift{7.200000in}{1.595506in}%
\pgfsys@useobject{currentmarker}{}%
\end{pgfscope}%
\end{pgfscope}%
\begin{pgfscope}%
\pgfsetbuttcap%
\pgfsetroundjoin%
\definecolor{currentfill}{rgb}{0.000000,0.000000,0.000000}%
\pgfsetfillcolor{currentfill}%
\pgfsetlinewidth{0.501875pt}%
\definecolor{currentstroke}{rgb}{0.000000,0.000000,0.000000}%
\pgfsetstrokecolor{currentstroke}%
\pgfsetdash{}{0pt}%
\pgfsys@defobject{currentmarker}{\pgfqpoint{0.000000in}{0.000000in}}{\pgfqpoint{0.027778in}{0.000000in}}{%
\pgfpathmoveto{\pgfqpoint{0.000000in}{0.000000in}}%
\pgfpathlineto{\pgfqpoint{0.027778in}{0.000000in}}%
\pgfusepath{stroke,fill}%
}%
\begin{pgfscope}%
\pgfsys@transformshift{1.000000in}{1.633513in}%
\pgfsys@useobject{currentmarker}{}%
\end{pgfscope}%
\end{pgfscope}%
\begin{pgfscope}%
\pgfsetbuttcap%
\pgfsetroundjoin%
\definecolor{currentfill}{rgb}{0.000000,0.000000,0.000000}%
\pgfsetfillcolor{currentfill}%
\pgfsetlinewidth{0.501875pt}%
\definecolor{currentstroke}{rgb}{0.000000,0.000000,0.000000}%
\pgfsetstrokecolor{currentstroke}%
\pgfsetdash{}{0pt}%
\pgfsys@defobject{currentmarker}{\pgfqpoint{-0.027778in}{0.000000in}}{\pgfqpoint{0.000000in}{0.000000in}}{%
\pgfpathmoveto{\pgfqpoint{0.000000in}{0.000000in}}%
\pgfpathlineto{\pgfqpoint{-0.027778in}{0.000000in}}%
\pgfusepath{stroke,fill}%
}%
\begin{pgfscope}%
\pgfsys@transformshift{7.200000in}{1.633513in}%
\pgfsys@useobject{currentmarker}{}%
\end{pgfscope}%
\end{pgfscope}%
\begin{pgfscope}%
\pgfsetbuttcap%
\pgfsetroundjoin%
\definecolor{currentfill}{rgb}{0.000000,0.000000,0.000000}%
\pgfsetfillcolor{currentfill}%
\pgfsetlinewidth{0.501875pt}%
\definecolor{currentstroke}{rgb}{0.000000,0.000000,0.000000}%
\pgfsetstrokecolor{currentstroke}%
\pgfsetdash{}{0pt}%
\pgfsys@defobject{currentmarker}{\pgfqpoint{0.000000in}{0.000000in}}{\pgfqpoint{0.027778in}{0.000000in}}{%
\pgfpathmoveto{\pgfqpoint{0.000000in}{0.000000in}}%
\pgfpathlineto{\pgfqpoint{0.027778in}{0.000000in}}%
\pgfusepath{stroke,fill}%
}%
\begin{pgfscope}%
\pgfsys@transformshift{1.000000in}{1.665647in}%
\pgfsys@useobject{currentmarker}{}%
\end{pgfscope}%
\end{pgfscope}%
\begin{pgfscope}%
\pgfsetbuttcap%
\pgfsetroundjoin%
\definecolor{currentfill}{rgb}{0.000000,0.000000,0.000000}%
\pgfsetfillcolor{currentfill}%
\pgfsetlinewidth{0.501875pt}%
\definecolor{currentstroke}{rgb}{0.000000,0.000000,0.000000}%
\pgfsetstrokecolor{currentstroke}%
\pgfsetdash{}{0pt}%
\pgfsys@defobject{currentmarker}{\pgfqpoint{-0.027778in}{0.000000in}}{\pgfqpoint{0.000000in}{0.000000in}}{%
\pgfpathmoveto{\pgfqpoint{0.000000in}{0.000000in}}%
\pgfpathlineto{\pgfqpoint{-0.027778in}{0.000000in}}%
\pgfusepath{stroke,fill}%
}%
\begin{pgfscope}%
\pgfsys@transformshift{7.200000in}{1.665647in}%
\pgfsys@useobject{currentmarker}{}%
\end{pgfscope}%
\end{pgfscope}%
\begin{pgfscope}%
\pgfsetbuttcap%
\pgfsetroundjoin%
\definecolor{currentfill}{rgb}{0.000000,0.000000,0.000000}%
\pgfsetfillcolor{currentfill}%
\pgfsetlinewidth{0.501875pt}%
\definecolor{currentstroke}{rgb}{0.000000,0.000000,0.000000}%
\pgfsetstrokecolor{currentstroke}%
\pgfsetdash{}{0pt}%
\pgfsys@defobject{currentmarker}{\pgfqpoint{0.000000in}{0.000000in}}{\pgfqpoint{0.027778in}{0.000000in}}{%
\pgfpathmoveto{\pgfqpoint{0.000000in}{0.000000in}}%
\pgfpathlineto{\pgfqpoint{0.027778in}{0.000000in}}%
\pgfusepath{stroke,fill}%
}%
\begin{pgfscope}%
\pgfsys@transformshift{1.000000in}{1.693483in}%
\pgfsys@useobject{currentmarker}{}%
\end{pgfscope}%
\end{pgfscope}%
\begin{pgfscope}%
\pgfsetbuttcap%
\pgfsetroundjoin%
\definecolor{currentfill}{rgb}{0.000000,0.000000,0.000000}%
\pgfsetfillcolor{currentfill}%
\pgfsetlinewidth{0.501875pt}%
\definecolor{currentstroke}{rgb}{0.000000,0.000000,0.000000}%
\pgfsetstrokecolor{currentstroke}%
\pgfsetdash{}{0pt}%
\pgfsys@defobject{currentmarker}{\pgfqpoint{-0.027778in}{0.000000in}}{\pgfqpoint{0.000000in}{0.000000in}}{%
\pgfpathmoveto{\pgfqpoint{0.000000in}{0.000000in}}%
\pgfpathlineto{\pgfqpoint{-0.027778in}{0.000000in}}%
\pgfusepath{stroke,fill}%
}%
\begin{pgfscope}%
\pgfsys@transformshift{7.200000in}{1.693483in}%
\pgfsys@useobject{currentmarker}{}%
\end{pgfscope}%
\end{pgfscope}%
\begin{pgfscope}%
\pgfsetbuttcap%
\pgfsetroundjoin%
\definecolor{currentfill}{rgb}{0.000000,0.000000,0.000000}%
\pgfsetfillcolor{currentfill}%
\pgfsetlinewidth{0.501875pt}%
\definecolor{currentstroke}{rgb}{0.000000,0.000000,0.000000}%
\pgfsetstrokecolor{currentstroke}%
\pgfsetdash{}{0pt}%
\pgfsys@defobject{currentmarker}{\pgfqpoint{0.000000in}{0.000000in}}{\pgfqpoint{0.027778in}{0.000000in}}{%
\pgfpathmoveto{\pgfqpoint{0.000000in}{0.000000in}}%
\pgfpathlineto{\pgfqpoint{0.027778in}{0.000000in}}%
\pgfusepath{stroke,fill}%
}%
\begin{pgfscope}%
\pgfsys@transformshift{1.000000in}{1.718036in}%
\pgfsys@useobject{currentmarker}{}%
\end{pgfscope}%
\end{pgfscope}%
\begin{pgfscope}%
\pgfsetbuttcap%
\pgfsetroundjoin%
\definecolor{currentfill}{rgb}{0.000000,0.000000,0.000000}%
\pgfsetfillcolor{currentfill}%
\pgfsetlinewidth{0.501875pt}%
\definecolor{currentstroke}{rgb}{0.000000,0.000000,0.000000}%
\pgfsetstrokecolor{currentstroke}%
\pgfsetdash{}{0pt}%
\pgfsys@defobject{currentmarker}{\pgfqpoint{-0.027778in}{0.000000in}}{\pgfqpoint{0.000000in}{0.000000in}}{%
\pgfpathmoveto{\pgfqpoint{0.000000in}{0.000000in}}%
\pgfpathlineto{\pgfqpoint{-0.027778in}{0.000000in}}%
\pgfusepath{stroke,fill}%
}%
\begin{pgfscope}%
\pgfsys@transformshift{7.200000in}{1.718036in}%
\pgfsys@useobject{currentmarker}{}%
\end{pgfscope}%
\end{pgfscope}%
\begin{pgfscope}%
\pgfsetbuttcap%
\pgfsetroundjoin%
\definecolor{currentfill}{rgb}{0.000000,0.000000,0.000000}%
\pgfsetfillcolor{currentfill}%
\pgfsetlinewidth{0.501875pt}%
\definecolor{currentstroke}{rgb}{0.000000,0.000000,0.000000}%
\pgfsetstrokecolor{currentstroke}%
\pgfsetdash{}{0pt}%
\pgfsys@defobject{currentmarker}{\pgfqpoint{0.000000in}{0.000000in}}{\pgfqpoint{0.027778in}{0.000000in}}{%
\pgfpathmoveto{\pgfqpoint{0.000000in}{0.000000in}}%
\pgfpathlineto{\pgfqpoint{0.027778in}{0.000000in}}%
\pgfusepath{stroke,fill}%
}%
\begin{pgfscope}%
\pgfsys@transformshift{1.000000in}{1.884494in}%
\pgfsys@useobject{currentmarker}{}%
\end{pgfscope}%
\end{pgfscope}%
\begin{pgfscope}%
\pgfsetbuttcap%
\pgfsetroundjoin%
\definecolor{currentfill}{rgb}{0.000000,0.000000,0.000000}%
\pgfsetfillcolor{currentfill}%
\pgfsetlinewidth{0.501875pt}%
\definecolor{currentstroke}{rgb}{0.000000,0.000000,0.000000}%
\pgfsetstrokecolor{currentstroke}%
\pgfsetdash{}{0pt}%
\pgfsys@defobject{currentmarker}{\pgfqpoint{-0.027778in}{0.000000in}}{\pgfqpoint{0.000000in}{0.000000in}}{%
\pgfpathmoveto{\pgfqpoint{0.000000in}{0.000000in}}%
\pgfpathlineto{\pgfqpoint{-0.027778in}{0.000000in}}%
\pgfusepath{stroke,fill}%
}%
\begin{pgfscope}%
\pgfsys@transformshift{7.200000in}{1.884494in}%
\pgfsys@useobject{currentmarker}{}%
\end{pgfscope}%
\end{pgfscope}%
\begin{pgfscope}%
\pgfsetbuttcap%
\pgfsetroundjoin%
\definecolor{currentfill}{rgb}{0.000000,0.000000,0.000000}%
\pgfsetfillcolor{currentfill}%
\pgfsetlinewidth{0.501875pt}%
\definecolor{currentstroke}{rgb}{0.000000,0.000000,0.000000}%
\pgfsetstrokecolor{currentstroke}%
\pgfsetdash{}{0pt}%
\pgfsys@defobject{currentmarker}{\pgfqpoint{0.000000in}{0.000000in}}{\pgfqpoint{0.027778in}{0.000000in}}{%
\pgfpathmoveto{\pgfqpoint{0.000000in}{0.000000in}}%
\pgfpathlineto{\pgfqpoint{0.027778in}{0.000000in}}%
\pgfusepath{stroke,fill}%
}%
\begin{pgfscope}%
\pgfsys@transformshift{1.000000in}{1.969018in}%
\pgfsys@useobject{currentmarker}{}%
\end{pgfscope}%
\end{pgfscope}%
\begin{pgfscope}%
\pgfsetbuttcap%
\pgfsetroundjoin%
\definecolor{currentfill}{rgb}{0.000000,0.000000,0.000000}%
\pgfsetfillcolor{currentfill}%
\pgfsetlinewidth{0.501875pt}%
\definecolor{currentstroke}{rgb}{0.000000,0.000000,0.000000}%
\pgfsetstrokecolor{currentstroke}%
\pgfsetdash{}{0pt}%
\pgfsys@defobject{currentmarker}{\pgfqpoint{-0.027778in}{0.000000in}}{\pgfqpoint{0.000000in}{0.000000in}}{%
\pgfpathmoveto{\pgfqpoint{0.000000in}{0.000000in}}%
\pgfpathlineto{\pgfqpoint{-0.027778in}{0.000000in}}%
\pgfusepath{stroke,fill}%
}%
\begin{pgfscope}%
\pgfsys@transformshift{7.200000in}{1.969018in}%
\pgfsys@useobject{currentmarker}{}%
\end{pgfscope}%
\end{pgfscope}%
\begin{pgfscope}%
\pgfsetbuttcap%
\pgfsetroundjoin%
\definecolor{currentfill}{rgb}{0.000000,0.000000,0.000000}%
\pgfsetfillcolor{currentfill}%
\pgfsetlinewidth{0.501875pt}%
\definecolor{currentstroke}{rgb}{0.000000,0.000000,0.000000}%
\pgfsetstrokecolor{currentstroke}%
\pgfsetdash{}{0pt}%
\pgfsys@defobject{currentmarker}{\pgfqpoint{0.000000in}{0.000000in}}{\pgfqpoint{0.027778in}{0.000000in}}{%
\pgfpathmoveto{\pgfqpoint{0.000000in}{0.000000in}}%
\pgfpathlineto{\pgfqpoint{0.027778in}{0.000000in}}%
\pgfusepath{stroke,fill}%
}%
\begin{pgfscope}%
\pgfsys@transformshift{1.000000in}{2.028989in}%
\pgfsys@useobject{currentmarker}{}%
\end{pgfscope}%
\end{pgfscope}%
\begin{pgfscope}%
\pgfsetbuttcap%
\pgfsetroundjoin%
\definecolor{currentfill}{rgb}{0.000000,0.000000,0.000000}%
\pgfsetfillcolor{currentfill}%
\pgfsetlinewidth{0.501875pt}%
\definecolor{currentstroke}{rgb}{0.000000,0.000000,0.000000}%
\pgfsetstrokecolor{currentstroke}%
\pgfsetdash{}{0pt}%
\pgfsys@defobject{currentmarker}{\pgfqpoint{-0.027778in}{0.000000in}}{\pgfqpoint{0.000000in}{0.000000in}}{%
\pgfpathmoveto{\pgfqpoint{0.000000in}{0.000000in}}%
\pgfpathlineto{\pgfqpoint{-0.027778in}{0.000000in}}%
\pgfusepath{stroke,fill}%
}%
\begin{pgfscope}%
\pgfsys@transformshift{7.200000in}{2.028989in}%
\pgfsys@useobject{currentmarker}{}%
\end{pgfscope}%
\end{pgfscope}%
\begin{pgfscope}%
\pgfsetbuttcap%
\pgfsetroundjoin%
\definecolor{currentfill}{rgb}{0.000000,0.000000,0.000000}%
\pgfsetfillcolor{currentfill}%
\pgfsetlinewidth{0.501875pt}%
\definecolor{currentstroke}{rgb}{0.000000,0.000000,0.000000}%
\pgfsetstrokecolor{currentstroke}%
\pgfsetdash{}{0pt}%
\pgfsys@defobject{currentmarker}{\pgfqpoint{0.000000in}{0.000000in}}{\pgfqpoint{0.027778in}{0.000000in}}{%
\pgfpathmoveto{\pgfqpoint{0.000000in}{0.000000in}}%
\pgfpathlineto{\pgfqpoint{0.027778in}{0.000000in}}%
\pgfusepath{stroke,fill}%
}%
\begin{pgfscope}%
\pgfsys@transformshift{1.000000in}{2.075506in}%
\pgfsys@useobject{currentmarker}{}%
\end{pgfscope}%
\end{pgfscope}%
\begin{pgfscope}%
\pgfsetbuttcap%
\pgfsetroundjoin%
\definecolor{currentfill}{rgb}{0.000000,0.000000,0.000000}%
\pgfsetfillcolor{currentfill}%
\pgfsetlinewidth{0.501875pt}%
\definecolor{currentstroke}{rgb}{0.000000,0.000000,0.000000}%
\pgfsetstrokecolor{currentstroke}%
\pgfsetdash{}{0pt}%
\pgfsys@defobject{currentmarker}{\pgfqpoint{-0.027778in}{0.000000in}}{\pgfqpoint{0.000000in}{0.000000in}}{%
\pgfpathmoveto{\pgfqpoint{0.000000in}{0.000000in}}%
\pgfpathlineto{\pgfqpoint{-0.027778in}{0.000000in}}%
\pgfusepath{stroke,fill}%
}%
\begin{pgfscope}%
\pgfsys@transformshift{7.200000in}{2.075506in}%
\pgfsys@useobject{currentmarker}{}%
\end{pgfscope}%
\end{pgfscope}%
\begin{pgfscope}%
\pgfsetbuttcap%
\pgfsetroundjoin%
\definecolor{currentfill}{rgb}{0.000000,0.000000,0.000000}%
\pgfsetfillcolor{currentfill}%
\pgfsetlinewidth{0.501875pt}%
\definecolor{currentstroke}{rgb}{0.000000,0.000000,0.000000}%
\pgfsetstrokecolor{currentstroke}%
\pgfsetdash{}{0pt}%
\pgfsys@defobject{currentmarker}{\pgfqpoint{0.000000in}{0.000000in}}{\pgfqpoint{0.027778in}{0.000000in}}{%
\pgfpathmoveto{\pgfqpoint{0.000000in}{0.000000in}}%
\pgfpathlineto{\pgfqpoint{0.027778in}{0.000000in}}%
\pgfusepath{stroke,fill}%
}%
\begin{pgfscope}%
\pgfsys@transformshift{1.000000in}{2.113513in}%
\pgfsys@useobject{currentmarker}{}%
\end{pgfscope}%
\end{pgfscope}%
\begin{pgfscope}%
\pgfsetbuttcap%
\pgfsetroundjoin%
\definecolor{currentfill}{rgb}{0.000000,0.000000,0.000000}%
\pgfsetfillcolor{currentfill}%
\pgfsetlinewidth{0.501875pt}%
\definecolor{currentstroke}{rgb}{0.000000,0.000000,0.000000}%
\pgfsetstrokecolor{currentstroke}%
\pgfsetdash{}{0pt}%
\pgfsys@defobject{currentmarker}{\pgfqpoint{-0.027778in}{0.000000in}}{\pgfqpoint{0.000000in}{0.000000in}}{%
\pgfpathmoveto{\pgfqpoint{0.000000in}{0.000000in}}%
\pgfpathlineto{\pgfqpoint{-0.027778in}{0.000000in}}%
\pgfusepath{stroke,fill}%
}%
\begin{pgfscope}%
\pgfsys@transformshift{7.200000in}{2.113513in}%
\pgfsys@useobject{currentmarker}{}%
\end{pgfscope}%
\end{pgfscope}%
\begin{pgfscope}%
\pgfsetbuttcap%
\pgfsetroundjoin%
\definecolor{currentfill}{rgb}{0.000000,0.000000,0.000000}%
\pgfsetfillcolor{currentfill}%
\pgfsetlinewidth{0.501875pt}%
\definecolor{currentstroke}{rgb}{0.000000,0.000000,0.000000}%
\pgfsetstrokecolor{currentstroke}%
\pgfsetdash{}{0pt}%
\pgfsys@defobject{currentmarker}{\pgfqpoint{0.000000in}{0.000000in}}{\pgfqpoint{0.027778in}{0.000000in}}{%
\pgfpathmoveto{\pgfqpoint{0.000000in}{0.000000in}}%
\pgfpathlineto{\pgfqpoint{0.027778in}{0.000000in}}%
\pgfusepath{stroke,fill}%
}%
\begin{pgfscope}%
\pgfsys@transformshift{1.000000in}{2.145647in}%
\pgfsys@useobject{currentmarker}{}%
\end{pgfscope}%
\end{pgfscope}%
\begin{pgfscope}%
\pgfsetbuttcap%
\pgfsetroundjoin%
\definecolor{currentfill}{rgb}{0.000000,0.000000,0.000000}%
\pgfsetfillcolor{currentfill}%
\pgfsetlinewidth{0.501875pt}%
\definecolor{currentstroke}{rgb}{0.000000,0.000000,0.000000}%
\pgfsetstrokecolor{currentstroke}%
\pgfsetdash{}{0pt}%
\pgfsys@defobject{currentmarker}{\pgfqpoint{-0.027778in}{0.000000in}}{\pgfqpoint{0.000000in}{0.000000in}}{%
\pgfpathmoveto{\pgfqpoint{0.000000in}{0.000000in}}%
\pgfpathlineto{\pgfqpoint{-0.027778in}{0.000000in}}%
\pgfusepath{stroke,fill}%
}%
\begin{pgfscope}%
\pgfsys@transformshift{7.200000in}{2.145647in}%
\pgfsys@useobject{currentmarker}{}%
\end{pgfscope}%
\end{pgfscope}%
\begin{pgfscope}%
\pgfsetbuttcap%
\pgfsetroundjoin%
\definecolor{currentfill}{rgb}{0.000000,0.000000,0.000000}%
\pgfsetfillcolor{currentfill}%
\pgfsetlinewidth{0.501875pt}%
\definecolor{currentstroke}{rgb}{0.000000,0.000000,0.000000}%
\pgfsetstrokecolor{currentstroke}%
\pgfsetdash{}{0pt}%
\pgfsys@defobject{currentmarker}{\pgfqpoint{0.000000in}{0.000000in}}{\pgfqpoint{0.027778in}{0.000000in}}{%
\pgfpathmoveto{\pgfqpoint{0.000000in}{0.000000in}}%
\pgfpathlineto{\pgfqpoint{0.027778in}{0.000000in}}%
\pgfusepath{stroke,fill}%
}%
\begin{pgfscope}%
\pgfsys@transformshift{1.000000in}{2.173483in}%
\pgfsys@useobject{currentmarker}{}%
\end{pgfscope}%
\end{pgfscope}%
\begin{pgfscope}%
\pgfsetbuttcap%
\pgfsetroundjoin%
\definecolor{currentfill}{rgb}{0.000000,0.000000,0.000000}%
\pgfsetfillcolor{currentfill}%
\pgfsetlinewidth{0.501875pt}%
\definecolor{currentstroke}{rgb}{0.000000,0.000000,0.000000}%
\pgfsetstrokecolor{currentstroke}%
\pgfsetdash{}{0pt}%
\pgfsys@defobject{currentmarker}{\pgfqpoint{-0.027778in}{0.000000in}}{\pgfqpoint{0.000000in}{0.000000in}}{%
\pgfpathmoveto{\pgfqpoint{0.000000in}{0.000000in}}%
\pgfpathlineto{\pgfqpoint{-0.027778in}{0.000000in}}%
\pgfusepath{stroke,fill}%
}%
\begin{pgfscope}%
\pgfsys@transformshift{7.200000in}{2.173483in}%
\pgfsys@useobject{currentmarker}{}%
\end{pgfscope}%
\end{pgfscope}%
\begin{pgfscope}%
\pgfsetbuttcap%
\pgfsetroundjoin%
\definecolor{currentfill}{rgb}{0.000000,0.000000,0.000000}%
\pgfsetfillcolor{currentfill}%
\pgfsetlinewidth{0.501875pt}%
\definecolor{currentstroke}{rgb}{0.000000,0.000000,0.000000}%
\pgfsetstrokecolor{currentstroke}%
\pgfsetdash{}{0pt}%
\pgfsys@defobject{currentmarker}{\pgfqpoint{0.000000in}{0.000000in}}{\pgfqpoint{0.027778in}{0.000000in}}{%
\pgfpathmoveto{\pgfqpoint{0.000000in}{0.000000in}}%
\pgfpathlineto{\pgfqpoint{0.027778in}{0.000000in}}%
\pgfusepath{stroke,fill}%
}%
\begin{pgfscope}%
\pgfsys@transformshift{1.000000in}{2.198036in}%
\pgfsys@useobject{currentmarker}{}%
\end{pgfscope}%
\end{pgfscope}%
\begin{pgfscope}%
\pgfsetbuttcap%
\pgfsetroundjoin%
\definecolor{currentfill}{rgb}{0.000000,0.000000,0.000000}%
\pgfsetfillcolor{currentfill}%
\pgfsetlinewidth{0.501875pt}%
\definecolor{currentstroke}{rgb}{0.000000,0.000000,0.000000}%
\pgfsetstrokecolor{currentstroke}%
\pgfsetdash{}{0pt}%
\pgfsys@defobject{currentmarker}{\pgfqpoint{-0.027778in}{0.000000in}}{\pgfqpoint{0.000000in}{0.000000in}}{%
\pgfpathmoveto{\pgfqpoint{0.000000in}{0.000000in}}%
\pgfpathlineto{\pgfqpoint{-0.027778in}{0.000000in}}%
\pgfusepath{stroke,fill}%
}%
\begin{pgfscope}%
\pgfsys@transformshift{7.200000in}{2.198036in}%
\pgfsys@useobject{currentmarker}{}%
\end{pgfscope}%
\end{pgfscope}%
\begin{pgfscope}%
\pgfsetbuttcap%
\pgfsetroundjoin%
\definecolor{currentfill}{rgb}{0.000000,0.000000,0.000000}%
\pgfsetfillcolor{currentfill}%
\pgfsetlinewidth{0.501875pt}%
\definecolor{currentstroke}{rgb}{0.000000,0.000000,0.000000}%
\pgfsetstrokecolor{currentstroke}%
\pgfsetdash{}{0pt}%
\pgfsys@defobject{currentmarker}{\pgfqpoint{0.000000in}{0.000000in}}{\pgfqpoint{0.027778in}{0.000000in}}{%
\pgfpathmoveto{\pgfqpoint{0.000000in}{0.000000in}}%
\pgfpathlineto{\pgfqpoint{0.027778in}{0.000000in}}%
\pgfusepath{stroke,fill}%
}%
\begin{pgfscope}%
\pgfsys@transformshift{1.000000in}{2.364494in}%
\pgfsys@useobject{currentmarker}{}%
\end{pgfscope}%
\end{pgfscope}%
\begin{pgfscope}%
\pgfsetbuttcap%
\pgfsetroundjoin%
\definecolor{currentfill}{rgb}{0.000000,0.000000,0.000000}%
\pgfsetfillcolor{currentfill}%
\pgfsetlinewidth{0.501875pt}%
\definecolor{currentstroke}{rgb}{0.000000,0.000000,0.000000}%
\pgfsetstrokecolor{currentstroke}%
\pgfsetdash{}{0pt}%
\pgfsys@defobject{currentmarker}{\pgfqpoint{-0.027778in}{0.000000in}}{\pgfqpoint{0.000000in}{0.000000in}}{%
\pgfpathmoveto{\pgfqpoint{0.000000in}{0.000000in}}%
\pgfpathlineto{\pgfqpoint{-0.027778in}{0.000000in}}%
\pgfusepath{stroke,fill}%
}%
\begin{pgfscope}%
\pgfsys@transformshift{7.200000in}{2.364494in}%
\pgfsys@useobject{currentmarker}{}%
\end{pgfscope}%
\end{pgfscope}%
\begin{pgfscope}%
\pgfsetbuttcap%
\pgfsetroundjoin%
\definecolor{currentfill}{rgb}{0.000000,0.000000,0.000000}%
\pgfsetfillcolor{currentfill}%
\pgfsetlinewidth{0.501875pt}%
\definecolor{currentstroke}{rgb}{0.000000,0.000000,0.000000}%
\pgfsetstrokecolor{currentstroke}%
\pgfsetdash{}{0pt}%
\pgfsys@defobject{currentmarker}{\pgfqpoint{0.000000in}{0.000000in}}{\pgfqpoint{0.027778in}{0.000000in}}{%
\pgfpathmoveto{\pgfqpoint{0.000000in}{0.000000in}}%
\pgfpathlineto{\pgfqpoint{0.027778in}{0.000000in}}%
\pgfusepath{stroke,fill}%
}%
\begin{pgfscope}%
\pgfsys@transformshift{1.000000in}{2.449018in}%
\pgfsys@useobject{currentmarker}{}%
\end{pgfscope}%
\end{pgfscope}%
\begin{pgfscope}%
\pgfsetbuttcap%
\pgfsetroundjoin%
\definecolor{currentfill}{rgb}{0.000000,0.000000,0.000000}%
\pgfsetfillcolor{currentfill}%
\pgfsetlinewidth{0.501875pt}%
\definecolor{currentstroke}{rgb}{0.000000,0.000000,0.000000}%
\pgfsetstrokecolor{currentstroke}%
\pgfsetdash{}{0pt}%
\pgfsys@defobject{currentmarker}{\pgfqpoint{-0.027778in}{0.000000in}}{\pgfqpoint{0.000000in}{0.000000in}}{%
\pgfpathmoveto{\pgfqpoint{0.000000in}{0.000000in}}%
\pgfpathlineto{\pgfqpoint{-0.027778in}{0.000000in}}%
\pgfusepath{stroke,fill}%
}%
\begin{pgfscope}%
\pgfsys@transformshift{7.200000in}{2.449018in}%
\pgfsys@useobject{currentmarker}{}%
\end{pgfscope}%
\end{pgfscope}%
\begin{pgfscope}%
\pgfsetbuttcap%
\pgfsetroundjoin%
\definecolor{currentfill}{rgb}{0.000000,0.000000,0.000000}%
\pgfsetfillcolor{currentfill}%
\pgfsetlinewidth{0.501875pt}%
\definecolor{currentstroke}{rgb}{0.000000,0.000000,0.000000}%
\pgfsetstrokecolor{currentstroke}%
\pgfsetdash{}{0pt}%
\pgfsys@defobject{currentmarker}{\pgfqpoint{0.000000in}{0.000000in}}{\pgfqpoint{0.027778in}{0.000000in}}{%
\pgfpathmoveto{\pgfqpoint{0.000000in}{0.000000in}}%
\pgfpathlineto{\pgfqpoint{0.027778in}{0.000000in}}%
\pgfusepath{stroke,fill}%
}%
\begin{pgfscope}%
\pgfsys@transformshift{1.000000in}{2.508989in}%
\pgfsys@useobject{currentmarker}{}%
\end{pgfscope}%
\end{pgfscope}%
\begin{pgfscope}%
\pgfsetbuttcap%
\pgfsetroundjoin%
\definecolor{currentfill}{rgb}{0.000000,0.000000,0.000000}%
\pgfsetfillcolor{currentfill}%
\pgfsetlinewidth{0.501875pt}%
\definecolor{currentstroke}{rgb}{0.000000,0.000000,0.000000}%
\pgfsetstrokecolor{currentstroke}%
\pgfsetdash{}{0pt}%
\pgfsys@defobject{currentmarker}{\pgfqpoint{-0.027778in}{0.000000in}}{\pgfqpoint{0.000000in}{0.000000in}}{%
\pgfpathmoveto{\pgfqpoint{0.000000in}{0.000000in}}%
\pgfpathlineto{\pgfqpoint{-0.027778in}{0.000000in}}%
\pgfusepath{stroke,fill}%
}%
\begin{pgfscope}%
\pgfsys@transformshift{7.200000in}{2.508989in}%
\pgfsys@useobject{currentmarker}{}%
\end{pgfscope}%
\end{pgfscope}%
\begin{pgfscope}%
\pgfsetbuttcap%
\pgfsetroundjoin%
\definecolor{currentfill}{rgb}{0.000000,0.000000,0.000000}%
\pgfsetfillcolor{currentfill}%
\pgfsetlinewidth{0.501875pt}%
\definecolor{currentstroke}{rgb}{0.000000,0.000000,0.000000}%
\pgfsetstrokecolor{currentstroke}%
\pgfsetdash{}{0pt}%
\pgfsys@defobject{currentmarker}{\pgfqpoint{0.000000in}{0.000000in}}{\pgfqpoint{0.027778in}{0.000000in}}{%
\pgfpathmoveto{\pgfqpoint{0.000000in}{0.000000in}}%
\pgfpathlineto{\pgfqpoint{0.027778in}{0.000000in}}%
\pgfusepath{stroke,fill}%
}%
\begin{pgfscope}%
\pgfsys@transformshift{1.000000in}{2.555506in}%
\pgfsys@useobject{currentmarker}{}%
\end{pgfscope}%
\end{pgfscope}%
\begin{pgfscope}%
\pgfsetbuttcap%
\pgfsetroundjoin%
\definecolor{currentfill}{rgb}{0.000000,0.000000,0.000000}%
\pgfsetfillcolor{currentfill}%
\pgfsetlinewidth{0.501875pt}%
\definecolor{currentstroke}{rgb}{0.000000,0.000000,0.000000}%
\pgfsetstrokecolor{currentstroke}%
\pgfsetdash{}{0pt}%
\pgfsys@defobject{currentmarker}{\pgfqpoint{-0.027778in}{0.000000in}}{\pgfqpoint{0.000000in}{0.000000in}}{%
\pgfpathmoveto{\pgfqpoint{0.000000in}{0.000000in}}%
\pgfpathlineto{\pgfqpoint{-0.027778in}{0.000000in}}%
\pgfusepath{stroke,fill}%
}%
\begin{pgfscope}%
\pgfsys@transformshift{7.200000in}{2.555506in}%
\pgfsys@useobject{currentmarker}{}%
\end{pgfscope}%
\end{pgfscope}%
\begin{pgfscope}%
\pgfsetbuttcap%
\pgfsetroundjoin%
\definecolor{currentfill}{rgb}{0.000000,0.000000,0.000000}%
\pgfsetfillcolor{currentfill}%
\pgfsetlinewidth{0.501875pt}%
\definecolor{currentstroke}{rgb}{0.000000,0.000000,0.000000}%
\pgfsetstrokecolor{currentstroke}%
\pgfsetdash{}{0pt}%
\pgfsys@defobject{currentmarker}{\pgfqpoint{0.000000in}{0.000000in}}{\pgfqpoint{0.027778in}{0.000000in}}{%
\pgfpathmoveto{\pgfqpoint{0.000000in}{0.000000in}}%
\pgfpathlineto{\pgfqpoint{0.027778in}{0.000000in}}%
\pgfusepath{stroke,fill}%
}%
\begin{pgfscope}%
\pgfsys@transformshift{1.000000in}{2.593513in}%
\pgfsys@useobject{currentmarker}{}%
\end{pgfscope}%
\end{pgfscope}%
\begin{pgfscope}%
\pgfsetbuttcap%
\pgfsetroundjoin%
\definecolor{currentfill}{rgb}{0.000000,0.000000,0.000000}%
\pgfsetfillcolor{currentfill}%
\pgfsetlinewidth{0.501875pt}%
\definecolor{currentstroke}{rgb}{0.000000,0.000000,0.000000}%
\pgfsetstrokecolor{currentstroke}%
\pgfsetdash{}{0pt}%
\pgfsys@defobject{currentmarker}{\pgfqpoint{-0.027778in}{0.000000in}}{\pgfqpoint{0.000000in}{0.000000in}}{%
\pgfpathmoveto{\pgfqpoint{0.000000in}{0.000000in}}%
\pgfpathlineto{\pgfqpoint{-0.027778in}{0.000000in}}%
\pgfusepath{stroke,fill}%
}%
\begin{pgfscope}%
\pgfsys@transformshift{7.200000in}{2.593513in}%
\pgfsys@useobject{currentmarker}{}%
\end{pgfscope}%
\end{pgfscope}%
\begin{pgfscope}%
\pgfsetbuttcap%
\pgfsetroundjoin%
\definecolor{currentfill}{rgb}{0.000000,0.000000,0.000000}%
\pgfsetfillcolor{currentfill}%
\pgfsetlinewidth{0.501875pt}%
\definecolor{currentstroke}{rgb}{0.000000,0.000000,0.000000}%
\pgfsetstrokecolor{currentstroke}%
\pgfsetdash{}{0pt}%
\pgfsys@defobject{currentmarker}{\pgfqpoint{0.000000in}{0.000000in}}{\pgfqpoint{0.027778in}{0.000000in}}{%
\pgfpathmoveto{\pgfqpoint{0.000000in}{0.000000in}}%
\pgfpathlineto{\pgfqpoint{0.027778in}{0.000000in}}%
\pgfusepath{stroke,fill}%
}%
\begin{pgfscope}%
\pgfsys@transformshift{1.000000in}{2.625647in}%
\pgfsys@useobject{currentmarker}{}%
\end{pgfscope}%
\end{pgfscope}%
\begin{pgfscope}%
\pgfsetbuttcap%
\pgfsetroundjoin%
\definecolor{currentfill}{rgb}{0.000000,0.000000,0.000000}%
\pgfsetfillcolor{currentfill}%
\pgfsetlinewidth{0.501875pt}%
\definecolor{currentstroke}{rgb}{0.000000,0.000000,0.000000}%
\pgfsetstrokecolor{currentstroke}%
\pgfsetdash{}{0pt}%
\pgfsys@defobject{currentmarker}{\pgfqpoint{-0.027778in}{0.000000in}}{\pgfqpoint{0.000000in}{0.000000in}}{%
\pgfpathmoveto{\pgfqpoint{0.000000in}{0.000000in}}%
\pgfpathlineto{\pgfqpoint{-0.027778in}{0.000000in}}%
\pgfusepath{stroke,fill}%
}%
\begin{pgfscope}%
\pgfsys@transformshift{7.200000in}{2.625647in}%
\pgfsys@useobject{currentmarker}{}%
\end{pgfscope}%
\end{pgfscope}%
\begin{pgfscope}%
\pgfsetbuttcap%
\pgfsetroundjoin%
\definecolor{currentfill}{rgb}{0.000000,0.000000,0.000000}%
\pgfsetfillcolor{currentfill}%
\pgfsetlinewidth{0.501875pt}%
\definecolor{currentstroke}{rgb}{0.000000,0.000000,0.000000}%
\pgfsetstrokecolor{currentstroke}%
\pgfsetdash{}{0pt}%
\pgfsys@defobject{currentmarker}{\pgfqpoint{0.000000in}{0.000000in}}{\pgfqpoint{0.027778in}{0.000000in}}{%
\pgfpathmoveto{\pgfqpoint{0.000000in}{0.000000in}}%
\pgfpathlineto{\pgfqpoint{0.027778in}{0.000000in}}%
\pgfusepath{stroke,fill}%
}%
\begin{pgfscope}%
\pgfsys@transformshift{1.000000in}{2.653483in}%
\pgfsys@useobject{currentmarker}{}%
\end{pgfscope}%
\end{pgfscope}%
\begin{pgfscope}%
\pgfsetbuttcap%
\pgfsetroundjoin%
\definecolor{currentfill}{rgb}{0.000000,0.000000,0.000000}%
\pgfsetfillcolor{currentfill}%
\pgfsetlinewidth{0.501875pt}%
\definecolor{currentstroke}{rgb}{0.000000,0.000000,0.000000}%
\pgfsetstrokecolor{currentstroke}%
\pgfsetdash{}{0pt}%
\pgfsys@defobject{currentmarker}{\pgfqpoint{-0.027778in}{0.000000in}}{\pgfqpoint{0.000000in}{0.000000in}}{%
\pgfpathmoveto{\pgfqpoint{0.000000in}{0.000000in}}%
\pgfpathlineto{\pgfqpoint{-0.027778in}{0.000000in}}%
\pgfusepath{stroke,fill}%
}%
\begin{pgfscope}%
\pgfsys@transformshift{7.200000in}{2.653483in}%
\pgfsys@useobject{currentmarker}{}%
\end{pgfscope}%
\end{pgfscope}%
\begin{pgfscope}%
\pgfsetbuttcap%
\pgfsetroundjoin%
\definecolor{currentfill}{rgb}{0.000000,0.000000,0.000000}%
\pgfsetfillcolor{currentfill}%
\pgfsetlinewidth{0.501875pt}%
\definecolor{currentstroke}{rgb}{0.000000,0.000000,0.000000}%
\pgfsetstrokecolor{currentstroke}%
\pgfsetdash{}{0pt}%
\pgfsys@defobject{currentmarker}{\pgfqpoint{0.000000in}{0.000000in}}{\pgfqpoint{0.027778in}{0.000000in}}{%
\pgfpathmoveto{\pgfqpoint{0.000000in}{0.000000in}}%
\pgfpathlineto{\pgfqpoint{0.027778in}{0.000000in}}%
\pgfusepath{stroke,fill}%
}%
\begin{pgfscope}%
\pgfsys@transformshift{1.000000in}{2.678036in}%
\pgfsys@useobject{currentmarker}{}%
\end{pgfscope}%
\end{pgfscope}%
\begin{pgfscope}%
\pgfsetbuttcap%
\pgfsetroundjoin%
\definecolor{currentfill}{rgb}{0.000000,0.000000,0.000000}%
\pgfsetfillcolor{currentfill}%
\pgfsetlinewidth{0.501875pt}%
\definecolor{currentstroke}{rgb}{0.000000,0.000000,0.000000}%
\pgfsetstrokecolor{currentstroke}%
\pgfsetdash{}{0pt}%
\pgfsys@defobject{currentmarker}{\pgfqpoint{-0.027778in}{0.000000in}}{\pgfqpoint{0.000000in}{0.000000in}}{%
\pgfpathmoveto{\pgfqpoint{0.000000in}{0.000000in}}%
\pgfpathlineto{\pgfqpoint{-0.027778in}{0.000000in}}%
\pgfusepath{stroke,fill}%
}%
\begin{pgfscope}%
\pgfsys@transformshift{7.200000in}{2.678036in}%
\pgfsys@useobject{currentmarker}{}%
\end{pgfscope}%
\end{pgfscope}%
\begin{pgfscope}%
\pgftext[left,bottom,x=0.554012in,y=0.106703in,rotate=90.000000]{{\sffamily\fontsize{12.000000}{14.400000}\selectfont mean square displacement [nm\(\displaystyle ^2\)]}}
%
\end{pgfscope}%
\begin{pgfscope}%
\pgfsetrectcap%
\pgfsetroundjoin%
\pgfsetlinewidth{1.003750pt}%
\definecolor{currentstroke}{rgb}{0.000000,0.000000,0.000000}%
\pgfsetstrokecolor{currentstroke}%
\pgfsetdash{}{0pt}%
\pgfpathmoveto{\pgfqpoint{1.000000in}{2.700000in}}%
\pgfpathlineto{\pgfqpoint{7.200000in}{2.700000in}}%
\pgfusepath{stroke}%
\end{pgfscope}%
\begin{pgfscope}%
\pgfsetrectcap%
\pgfsetroundjoin%
\pgfsetlinewidth{1.003750pt}%
\definecolor{currentstroke}{rgb}{0.000000,0.000000,0.000000}%
\pgfsetstrokecolor{currentstroke}%
\pgfsetdash{}{0pt}%
\pgfpathmoveto{\pgfqpoint{7.200000in}{0.300000in}}%
\pgfpathlineto{\pgfqpoint{7.200000in}{2.700000in}}%
\pgfusepath{stroke}%
\end{pgfscope}%
\begin{pgfscope}%
\pgfsetrectcap%
\pgfsetroundjoin%
\pgfsetlinewidth{1.003750pt}%
\definecolor{currentstroke}{rgb}{0.000000,0.000000,0.000000}%
\pgfsetstrokecolor{currentstroke}%
\pgfsetdash{}{0pt}%
\pgfpathmoveto{\pgfqpoint{1.000000in}{0.300000in}}%
\pgfpathlineto{\pgfqpoint{7.200000in}{0.300000in}}%
\pgfusepath{stroke}%
\end{pgfscope}%
\begin{pgfscope}%
\pgfsetrectcap%
\pgfsetroundjoin%
\pgfsetlinewidth{1.003750pt}%
\definecolor{currentstroke}{rgb}{0.000000,0.000000,0.000000}%
\pgfsetstrokecolor{currentstroke}%
\pgfsetdash{}{0pt}%
\pgfpathmoveto{\pgfqpoint{1.000000in}{0.300000in}}%
\pgfpathlineto{\pgfqpoint{1.000000in}{2.700000in}}%
\pgfusepath{stroke}%
\end{pgfscope}%
\begin{pgfscope}%
\pgfsetrectcap%
\pgfsetroundjoin%
\definecolor{currentfill}{rgb}{1.000000,1.000000,1.000000}%
\pgfsetfillcolor{currentfill}%
\pgfsetlinewidth{1.003750pt}%
\definecolor{currentstroke}{rgb}{0.000000,0.000000,0.000000}%
\pgfsetstrokecolor{currentstroke}%
\pgfsetdash{}{0pt}%
\pgfpathmoveto{\pgfqpoint{1.069417in}{1.977606in}}%
\pgfpathlineto{\pgfqpoint{1.926808in}{1.977606in}}%
\pgfpathlineto{\pgfqpoint{1.926808in}{2.630583in}}%
\pgfpathlineto{\pgfqpoint{1.069417in}{2.630583in}}%
\pgfpathlineto{\pgfqpoint{1.069417in}{1.977606in}}%
\pgfpathclose%
\pgfusepath{stroke,fill}%
\end{pgfscope}%
\begin{pgfscope}%
\pgfsetrectcap%
\pgfsetroundjoin%
\pgfsetlinewidth{1.003750pt}%
\definecolor{currentstroke}{rgb}{0.000000,0.000000,1.000000}%
\pgfsetstrokecolor{currentstroke}%
\pgfsetdash{}{0pt}%
\pgfpathmoveto{\pgfqpoint{1.166600in}{2.518161in}}%
\pgfpathlineto{\pgfqpoint{1.360967in}{2.518161in}}%
\pgfusepath{stroke}%
\end{pgfscope}%
\begin{pgfscope}%
\pgftext[left,bottom,x=1.513683in,y=2.440691in,rotate=0.000000]{{\sffamily\fontsize{9.996000}{11.995200}\selectfont spc}}
%
\end{pgfscope}%
\begin{pgfscope}%
\pgfsetrectcap%
\pgfsetroundjoin%
\pgfsetlinewidth{1.003750pt}%
\definecolor{currentstroke}{rgb}{0.000000,0.500000,0.000000}%
\pgfsetstrokecolor{currentstroke}%
\pgfsetdash{}{0pt}%
\pgfpathmoveto{\pgfqpoint{1.166600in}{2.314385in}}%
\pgfpathlineto{\pgfqpoint{1.360967in}{2.314385in}}%
\pgfusepath{stroke}%
\end{pgfscope}%
\begin{pgfscope}%
\pgftext[left,bottom,x=1.513683in,y=2.236915in,rotate=0.000000]{{\sffamily\fontsize{9.996000}{11.995200}\selectfont spce}}
%
\end{pgfscope}%
\begin{pgfscope}%
\pgfsetrectcap%
\pgfsetroundjoin%
\pgfsetlinewidth{1.003750pt}%
\definecolor{currentstroke}{rgb}{1.000000,0.000000,0.000000}%
\pgfsetstrokecolor{currentstroke}%
\pgfsetdash{}{0pt}%
\pgfpathmoveto{\pgfqpoint{1.166600in}{2.110609in}}%
\pgfpathlineto{\pgfqpoint{1.360967in}{2.110609in}}%
\pgfusepath{stroke}%
\end{pgfscope}%
\begin{pgfscope}%
\pgftext[left,bottom,x=1.513683in,y=2.033139in,rotate=0.000000]{{\sffamily\fontsize{9.996000}{11.995200}\selectfont tip3p}}
%
\end{pgfscope}%
\end{pgfpicture}%
\makeatother%
\endgroup%
}
    \caption{Mean-square displacement.} \label{fig:msd}
\end{figure}

The full-logarithmic plot of the mean-square displacement $< \Delta r(t)^2 >$ shows the existence of two regimes. Until $\unit[0.3]{ps}$ the slope is higher than afterwards. In theory the slope at the beginning has to be twice as high because of the quadratic time-dependence of the means-square displacement in the ballistic regime. In the following linear regime the displacement is proportional to the time itself.

\begin{figure}[H]
	\resizebox{\linewidth}{!}{%% Creator: Matplotlib, PGF backend
%%
%% To include the figure in your LaTeX document, write
%%   \input{<filename>.pgf}
%%
%% Make sure the required packages are loaded in your preamble
%%   \usepackage{pgf}
%%
%% Figures using additional raster images can only be included by \input if
%% they are in the same directory as the main LaTeX file. For loading figures
%% from other directories you can use the `import` package
%%   \usepackage{import}
%% and then include the figures with
%%   \import{<path to file>}{<filename>.pgf}
%%
%% Matplotlib used the following preamble
%%   \usepackage{fontspec}
%%   \setmainfont{DejaVu Serif}
%%   \setsansfont{DejaVu Sans}
%%   \setmonofont{DejaVu Sans Mono}
%%
\begingroup%
\makeatletter%
\begin{pgfpicture}%
\pgfpathrectangle{\pgfpointorigin}{\pgfqpoint{8.000000in}{3.000000in}}%
\pgfusepath{use as bounding box}%
\begin{pgfscope}%
\pgfsetrectcap%
\pgfsetroundjoin%
\definecolor{currentfill}{rgb}{1.000000,1.000000,1.000000}%
\pgfsetfillcolor{currentfill}%
\pgfsetlinewidth{0.000000pt}%
\definecolor{currentstroke}{rgb}{1.000000,1.000000,1.000000}%
\pgfsetstrokecolor{currentstroke}%
\pgfsetdash{}{0pt}%
\pgfpathmoveto{\pgfqpoint{0.000000in}{0.000000in}}%
\pgfpathlineto{\pgfqpoint{8.000000in}{0.000000in}}%
\pgfpathlineto{\pgfqpoint{8.000000in}{3.000000in}}%
\pgfpathlineto{\pgfqpoint{0.000000in}{3.000000in}}%
\pgfpathclose%
\pgfusepath{fill}%
\end{pgfscope}%
\begin{pgfscope}%
\pgfsetrectcap%
\pgfsetroundjoin%
\definecolor{currentfill}{rgb}{1.000000,1.000000,1.000000}%
\pgfsetfillcolor{currentfill}%
\pgfsetlinewidth{0.000000pt}%
\definecolor{currentstroke}{rgb}{0.000000,0.000000,0.000000}%
\pgfsetstrokecolor{currentstroke}%
\pgfsetdash{}{0pt}%
\pgfpathmoveto{\pgfqpoint{1.000000in}{0.300000in}}%
\pgfpathlineto{\pgfqpoint{7.200000in}{0.300000in}}%
\pgfpathlineto{\pgfqpoint{7.200000in}{2.700000in}}%
\pgfpathlineto{\pgfqpoint{1.000000in}{2.700000in}}%
\pgfpathclose%
\pgfusepath{fill}%
\end{pgfscope}%
\begin{pgfscope}%
\pgfpathrectangle{\pgfqpoint{1.000000in}{0.300000in}}{\pgfqpoint{6.200000in}{2.400000in}} %
\pgfusepath{clip}%
\pgfsetrectcap%
\pgfsetroundjoin%
\pgfsetlinewidth{1.003750pt}%
\definecolor{currentstroke}{rgb}{0.000000,0.000000,1.000000}%
\pgfsetstrokecolor{currentstroke}%
\pgfsetdash{}{0pt}%
\pgfpathmoveto{\pgfqpoint{1.001240in}{1.451724in}}%
\pgfpathlineto{\pgfqpoint{1.002480in}{1.965809in}}%
\pgfpathlineto{\pgfqpoint{1.003720in}{2.151236in}}%
\pgfpathlineto{\pgfqpoint{1.004960in}{2.163844in}}%
\pgfpathlineto{\pgfqpoint{1.009920in}{1.875944in}}%
\pgfpathlineto{\pgfqpoint{1.019840in}{1.583228in}}%
\pgfpathlineto{\pgfqpoint{1.021080in}{1.577501in}}%
\pgfpathlineto{\pgfqpoint{1.023560in}{1.549291in}}%
\pgfpathlineto{\pgfqpoint{1.026040in}{1.526794in}}%
\pgfpathlineto{\pgfqpoint{1.027280in}{1.513883in}}%
\pgfpathlineto{\pgfqpoint{1.028520in}{1.513028in}}%
\pgfpathlineto{\pgfqpoint{1.029760in}{1.508574in}}%
\pgfpathlineto{\pgfqpoint{1.043400in}{1.412726in}}%
\pgfpathlineto{\pgfqpoint{1.044640in}{1.413151in}}%
\pgfpathlineto{\pgfqpoint{1.047120in}{1.398978in}}%
\pgfpathlineto{\pgfqpoint{1.049600in}{1.399489in}}%
\pgfpathlineto{\pgfqpoint{1.050840in}{1.391081in}}%
\pgfpathlineto{\pgfqpoint{1.054560in}{1.396424in}}%
\pgfpathlineto{\pgfqpoint{1.058280in}{1.386865in}}%
\pgfpathlineto{\pgfqpoint{1.059520in}{1.388074in}}%
\pgfpathlineto{\pgfqpoint{1.060760in}{1.381215in}}%
\pgfpathlineto{\pgfqpoint{1.063240in}{1.381987in}}%
\pgfpathlineto{\pgfqpoint{1.064480in}{1.382838in}}%
\pgfpathlineto{\pgfqpoint{1.068200in}{1.372404in}}%
\pgfpathlineto{\pgfqpoint{1.069440in}{1.373683in}}%
\pgfpathlineto{\pgfqpoint{1.070680in}{1.372967in}}%
\pgfpathlineto{\pgfqpoint{1.073160in}{1.379849in}}%
\pgfpathlineto{\pgfqpoint{1.075640in}{1.378965in}}%
\pgfpathlineto{\pgfqpoint{1.076880in}{1.379470in}}%
\pgfpathlineto{\pgfqpoint{1.081840in}{1.373293in}}%
\pgfpathlineto{\pgfqpoint{1.084320in}{1.370405in}}%
\pgfpathlineto{\pgfqpoint{1.085560in}{1.368715in}}%
\pgfpathlineto{\pgfqpoint{1.089280in}{1.358049in}}%
\pgfpathlineto{\pgfqpoint{1.091760in}{1.356565in}}%
\pgfpathlineto{\pgfqpoint{1.093000in}{1.357730in}}%
\pgfpathlineto{\pgfqpoint{1.096720in}{1.347590in}}%
\pgfpathlineto{\pgfqpoint{1.100440in}{1.348658in}}%
\pgfpathlineto{\pgfqpoint{1.101680in}{1.349225in}}%
\pgfpathlineto{\pgfqpoint{1.107880in}{1.327607in}}%
\pgfpathlineto{\pgfqpoint{1.110360in}{1.329243in}}%
\pgfpathlineto{\pgfqpoint{1.114080in}{1.324193in}}%
\pgfpathlineto{\pgfqpoint{1.116560in}{1.325551in}}%
\pgfpathlineto{\pgfqpoint{1.117800in}{1.325104in}}%
\pgfpathlineto{\pgfqpoint{1.120280in}{1.317979in}}%
\pgfpathlineto{\pgfqpoint{1.122760in}{1.318747in}}%
\pgfpathlineto{\pgfqpoint{1.125240in}{1.315314in}}%
\pgfpathlineto{\pgfqpoint{1.126480in}{1.315739in}}%
\pgfpathlineto{\pgfqpoint{1.128960in}{1.319538in}}%
\pgfpathlineto{\pgfqpoint{1.131440in}{1.316352in}}%
\pgfpathlineto{\pgfqpoint{1.133920in}{1.313259in}}%
\pgfpathlineto{\pgfqpoint{1.135160in}{1.311392in}}%
\pgfpathlineto{\pgfqpoint{1.137640in}{1.315183in}}%
\pgfpathlineto{\pgfqpoint{1.143840in}{1.299326in}}%
\pgfpathlineto{\pgfqpoint{1.145080in}{1.301550in}}%
\pgfpathlineto{\pgfqpoint{1.146320in}{1.301002in}}%
\pgfpathlineto{\pgfqpoint{1.148800in}{1.306652in}}%
\pgfpathlineto{\pgfqpoint{1.150040in}{1.305399in}}%
\pgfpathlineto{\pgfqpoint{1.152520in}{1.307379in}}%
\pgfpathlineto{\pgfqpoint{1.157480in}{1.301862in}}%
\pgfpathlineto{\pgfqpoint{1.158720in}{1.303194in}}%
\pgfpathlineto{\pgfqpoint{1.161200in}{1.302147in}}%
\pgfpathlineto{\pgfqpoint{1.163680in}{1.301401in}}%
\pgfpathlineto{\pgfqpoint{1.164920in}{1.297714in}}%
\pgfpathlineto{\pgfqpoint{1.168640in}{1.299379in}}%
\pgfpathlineto{\pgfqpoint{1.172360in}{1.303824in}}%
\pgfpathlineto{\pgfqpoint{1.174840in}{1.301633in}}%
\pgfpathlineto{\pgfqpoint{1.177320in}{1.312003in}}%
\pgfpathlineto{\pgfqpoint{1.178560in}{1.313241in}}%
\pgfpathlineto{\pgfqpoint{1.181040in}{1.310435in}}%
\pgfpathlineto{\pgfqpoint{1.183520in}{1.311718in}}%
\pgfpathlineto{\pgfqpoint{1.186000in}{1.308877in}}%
\pgfpathlineto{\pgfqpoint{1.188480in}{1.311474in}}%
\pgfpathlineto{\pgfqpoint{1.190960in}{1.309587in}}%
\pgfpathlineto{\pgfqpoint{1.192200in}{1.306469in}}%
\pgfpathlineto{\pgfqpoint{1.194680in}{1.308164in}}%
\pgfpathlineto{\pgfqpoint{1.197160in}{1.316760in}}%
\pgfpathlineto{\pgfqpoint{1.199640in}{1.321063in}}%
\pgfpathlineto{\pgfqpoint{1.200880in}{1.322573in}}%
\pgfpathlineto{\pgfqpoint{1.202120in}{1.320581in}}%
\pgfpathlineto{\pgfqpoint{1.204600in}{1.321075in}}%
\pgfpathlineto{\pgfqpoint{1.205840in}{1.317237in}}%
\pgfpathlineto{\pgfqpoint{1.207080in}{1.318613in}}%
\pgfpathlineto{\pgfqpoint{1.209560in}{1.317420in}}%
\pgfpathlineto{\pgfqpoint{1.214520in}{1.307491in}}%
\pgfpathlineto{\pgfqpoint{1.217000in}{1.304937in}}%
\pgfpathlineto{\pgfqpoint{1.219480in}{1.300723in}}%
\pgfpathlineto{\pgfqpoint{1.220720in}{1.300419in}}%
\pgfpathlineto{\pgfqpoint{1.223200in}{1.301793in}}%
\pgfpathlineto{\pgfqpoint{1.224440in}{1.302239in}}%
\pgfpathlineto{\pgfqpoint{1.225680in}{1.304591in}}%
\pgfpathlineto{\pgfqpoint{1.230640in}{1.301620in}}%
\pgfpathlineto{\pgfqpoint{1.231880in}{1.298631in}}%
\pgfpathlineto{\pgfqpoint{1.234360in}{1.299224in}}%
\pgfpathlineto{\pgfqpoint{1.235600in}{1.300449in}}%
\pgfpathlineto{\pgfqpoint{1.238080in}{1.299111in}}%
\pgfpathlineto{\pgfqpoint{1.239320in}{1.298457in}}%
\pgfpathlineto{\pgfqpoint{1.241800in}{1.294056in}}%
\pgfpathlineto{\pgfqpoint{1.244280in}{1.291544in}}%
\pgfpathlineto{\pgfqpoint{1.246760in}{1.294966in}}%
\pgfpathlineto{\pgfqpoint{1.248000in}{1.293742in}}%
\pgfpathlineto{\pgfqpoint{1.250480in}{1.298209in}}%
\pgfpathlineto{\pgfqpoint{1.251720in}{1.299794in}}%
\pgfpathlineto{\pgfqpoint{1.252960in}{1.299081in}}%
\pgfpathlineto{\pgfqpoint{1.255440in}{1.296686in}}%
\pgfpathlineto{\pgfqpoint{1.256680in}{1.297028in}}%
\pgfpathlineto{\pgfqpoint{1.259160in}{1.294097in}}%
\pgfpathlineto{\pgfqpoint{1.261640in}{1.296398in}}%
\pgfpathlineto{\pgfqpoint{1.265360in}{1.290280in}}%
\pgfpathlineto{\pgfqpoint{1.267840in}{1.289774in}}%
\pgfpathlineto{\pgfqpoint{1.272800in}{1.294166in}}%
\pgfpathlineto{\pgfqpoint{1.274040in}{1.293319in}}%
\pgfpathlineto{\pgfqpoint{1.277760in}{1.298119in}}%
\pgfpathlineto{\pgfqpoint{1.279000in}{1.297365in}}%
\pgfpathlineto{\pgfqpoint{1.281480in}{1.292533in}}%
\pgfpathlineto{\pgfqpoint{1.283960in}{1.297640in}}%
\pgfpathlineto{\pgfqpoint{1.285200in}{1.297909in}}%
\pgfpathlineto{\pgfqpoint{1.287680in}{1.294280in}}%
\pgfpathlineto{\pgfqpoint{1.290160in}{1.291028in}}%
\pgfpathlineto{\pgfqpoint{1.293880in}{1.288051in}}%
\pgfpathlineto{\pgfqpoint{1.295120in}{1.289244in}}%
\pgfpathlineto{\pgfqpoint{1.296360in}{1.292588in}}%
\pgfpathlineto{\pgfqpoint{1.298840in}{1.288013in}}%
\pgfpathlineto{\pgfqpoint{1.300080in}{1.291504in}}%
\pgfpathlineto{\pgfqpoint{1.302560in}{1.292022in}}%
\pgfpathlineto{\pgfqpoint{1.305040in}{1.289384in}}%
\pgfpathlineto{\pgfqpoint{1.306280in}{1.288963in}}%
\pgfpathlineto{\pgfqpoint{1.307520in}{1.290444in}}%
\pgfpathlineto{\pgfqpoint{1.310000in}{1.287264in}}%
\pgfpathlineto{\pgfqpoint{1.312480in}{1.289894in}}%
\pgfpathlineto{\pgfqpoint{1.317440in}{1.284427in}}%
\pgfpathlineto{\pgfqpoint{1.318680in}{1.284934in}}%
\pgfpathlineto{\pgfqpoint{1.323640in}{1.292821in}}%
\pgfpathlineto{\pgfqpoint{1.324880in}{1.293510in}}%
\pgfpathlineto{\pgfqpoint{1.326120in}{1.292443in}}%
\pgfpathlineto{\pgfqpoint{1.327360in}{1.293872in}}%
\pgfpathlineto{\pgfqpoint{1.328600in}{1.292711in}}%
\pgfpathlineto{\pgfqpoint{1.329840in}{1.288732in}}%
\pgfpathlineto{\pgfqpoint{1.331080in}{1.288950in}}%
\pgfpathlineto{\pgfqpoint{1.333560in}{1.285841in}}%
\pgfpathlineto{\pgfqpoint{1.337280in}{1.276837in}}%
\pgfpathlineto{\pgfqpoint{1.339760in}{1.278498in}}%
\pgfpathlineto{\pgfqpoint{1.341000in}{1.278314in}}%
\pgfpathlineto{\pgfqpoint{1.343480in}{1.276324in}}%
\pgfpathlineto{\pgfqpoint{1.350920in}{1.278701in}}%
\pgfpathlineto{\pgfqpoint{1.352160in}{1.278457in}}%
\pgfpathlineto{\pgfqpoint{1.355880in}{1.274194in}}%
\pgfpathlineto{\pgfqpoint{1.359600in}{1.276680in}}%
\pgfpathlineto{\pgfqpoint{1.363320in}{1.276965in}}%
\pgfpathlineto{\pgfqpoint{1.365800in}{1.275340in}}%
\pgfpathlineto{\pgfqpoint{1.368280in}{1.274938in}}%
\pgfpathlineto{\pgfqpoint{1.370760in}{1.278316in}}%
\pgfpathlineto{\pgfqpoint{1.372000in}{1.277938in}}%
\pgfpathlineto{\pgfqpoint{1.373240in}{1.281629in}}%
\pgfpathlineto{\pgfqpoint{1.374480in}{1.281475in}}%
\pgfpathlineto{\pgfqpoint{1.375720in}{1.282853in}}%
\pgfpathlineto{\pgfqpoint{1.379440in}{1.279407in}}%
\pgfpathlineto{\pgfqpoint{1.381920in}{1.280234in}}%
\pgfpathlineto{\pgfqpoint{1.383160in}{1.280548in}}%
\pgfpathlineto{\pgfqpoint{1.385640in}{1.282977in}}%
\pgfpathlineto{\pgfqpoint{1.386880in}{1.280781in}}%
\pgfpathlineto{\pgfqpoint{1.388120in}{1.280987in}}%
\pgfpathlineto{\pgfqpoint{1.393080in}{1.278540in}}%
\pgfpathlineto{\pgfqpoint{1.394320in}{1.278563in}}%
\pgfpathlineto{\pgfqpoint{1.396800in}{1.282653in}}%
\pgfpathlineto{\pgfqpoint{1.399280in}{1.283166in}}%
\pgfpathlineto{\pgfqpoint{1.400520in}{1.285208in}}%
\pgfpathlineto{\pgfqpoint{1.403000in}{1.283809in}}%
\pgfpathlineto{\pgfqpoint{1.405480in}{1.279382in}}%
\pgfpathlineto{\pgfqpoint{1.407960in}{1.282906in}}%
\pgfpathlineto{\pgfqpoint{1.409200in}{1.282069in}}%
\pgfpathlineto{\pgfqpoint{1.410440in}{1.283122in}}%
\pgfpathlineto{\pgfqpoint{1.419120in}{1.273370in}}%
\pgfpathlineto{\pgfqpoint{1.420360in}{1.275782in}}%
\pgfpathlineto{\pgfqpoint{1.422840in}{1.274081in}}%
\pgfpathlineto{\pgfqpoint{1.425320in}{1.277275in}}%
\pgfpathlineto{\pgfqpoint{1.427800in}{1.275080in}}%
\pgfpathlineto{\pgfqpoint{1.430280in}{1.273420in}}%
\pgfpathlineto{\pgfqpoint{1.431520in}{1.274715in}}%
\pgfpathlineto{\pgfqpoint{1.434000in}{1.269928in}}%
\pgfpathlineto{\pgfqpoint{1.436480in}{1.271452in}}%
\pgfpathlineto{\pgfqpoint{1.438960in}{1.268033in}}%
\pgfpathlineto{\pgfqpoint{1.440200in}{1.264547in}}%
\pgfpathlineto{\pgfqpoint{1.442680in}{1.265908in}}%
\pgfpathlineto{\pgfqpoint{1.445160in}{1.269525in}}%
\pgfpathlineto{\pgfqpoint{1.451360in}{1.275314in}}%
\pgfpathlineto{\pgfqpoint{1.452600in}{1.274037in}}%
\pgfpathlineto{\pgfqpoint{1.453840in}{1.270028in}}%
\pgfpathlineto{\pgfqpoint{1.456320in}{1.269563in}}%
\pgfpathlineto{\pgfqpoint{1.457560in}{1.269498in}}%
\pgfpathlineto{\pgfqpoint{1.461280in}{1.261508in}}%
\pgfpathlineto{\pgfqpoint{1.463760in}{1.261861in}}%
\pgfpathlineto{\pgfqpoint{1.466240in}{1.262021in}}%
\pgfpathlineto{\pgfqpoint{1.469960in}{1.264604in}}%
\pgfpathlineto{\pgfqpoint{1.473680in}{1.267374in}}%
\pgfpathlineto{\pgfqpoint{1.477400in}{1.262614in}}%
\pgfpathlineto{\pgfqpoint{1.479880in}{1.259862in}}%
\pgfpathlineto{\pgfqpoint{1.483600in}{1.262541in}}%
\pgfpathlineto{\pgfqpoint{1.491040in}{1.259912in}}%
\pgfpathlineto{\pgfqpoint{1.493520in}{1.261184in}}%
\pgfpathlineto{\pgfqpoint{1.494760in}{1.263021in}}%
\pgfpathlineto{\pgfqpoint{1.496000in}{1.262138in}}%
\pgfpathlineto{\pgfqpoint{1.497240in}{1.264540in}}%
\pgfpathlineto{\pgfqpoint{1.498480in}{1.263976in}}%
\pgfpathlineto{\pgfqpoint{1.499720in}{1.265669in}}%
\pgfpathlineto{\pgfqpoint{1.503440in}{1.263396in}}%
\pgfpathlineto{\pgfqpoint{1.504680in}{1.263560in}}%
\pgfpathlineto{\pgfqpoint{1.507160in}{1.261897in}}%
\pgfpathlineto{\pgfqpoint{1.509640in}{1.264540in}}%
\pgfpathlineto{\pgfqpoint{1.510880in}{1.262831in}}%
\pgfpathlineto{\pgfqpoint{1.513360in}{1.262849in}}%
\pgfpathlineto{\pgfqpoint{1.514600in}{1.261189in}}%
\pgfpathlineto{\pgfqpoint{1.519560in}{1.263029in}}%
\pgfpathlineto{\pgfqpoint{1.524520in}{1.267710in}}%
\pgfpathlineto{\pgfqpoint{1.527000in}{1.265986in}}%
\pgfpathlineto{\pgfqpoint{1.528240in}{1.262308in}}%
\pgfpathlineto{\pgfqpoint{1.530720in}{1.263202in}}%
\pgfpathlineto{\pgfqpoint{1.531960in}{1.263941in}}%
\pgfpathlineto{\pgfqpoint{1.539400in}{1.256816in}}%
\pgfpathlineto{\pgfqpoint{1.541880in}{1.254760in}}%
\pgfpathlineto{\pgfqpoint{1.545600in}{1.257879in}}%
\pgfpathlineto{\pgfqpoint{1.546840in}{1.257057in}}%
\pgfpathlineto{\pgfqpoint{1.549320in}{1.261324in}}%
\pgfpathlineto{\pgfqpoint{1.550560in}{1.260921in}}%
\pgfpathlineto{\pgfqpoint{1.554280in}{1.257117in}}%
\pgfpathlineto{\pgfqpoint{1.555520in}{1.259226in}}%
\pgfpathlineto{\pgfqpoint{1.558000in}{1.256109in}}%
\pgfpathlineto{\pgfqpoint{1.561720in}{1.259097in}}%
\pgfpathlineto{\pgfqpoint{1.565440in}{1.255848in}}%
\pgfpathlineto{\pgfqpoint{1.569160in}{1.260484in}}%
\pgfpathlineto{\pgfqpoint{1.572880in}{1.264560in}}%
\pgfpathlineto{\pgfqpoint{1.574120in}{1.264032in}}%
\pgfpathlineto{\pgfqpoint{1.575360in}{1.265153in}}%
\pgfpathlineto{\pgfqpoint{1.576600in}{1.264091in}}%
\pgfpathlineto{\pgfqpoint{1.579080in}{1.259962in}}%
\pgfpathlineto{\pgfqpoint{1.587760in}{1.253361in}}%
\pgfpathlineto{\pgfqpoint{1.593960in}{1.257031in}}%
\pgfpathlineto{\pgfqpoint{1.600160in}{1.256933in}}%
\pgfpathlineto{\pgfqpoint{1.603880in}{1.252991in}}%
\pgfpathlineto{\pgfqpoint{1.606360in}{1.254456in}}%
\pgfpathlineto{\pgfqpoint{1.607600in}{1.254721in}}%
\pgfpathlineto{\pgfqpoint{1.610080in}{1.253478in}}%
\pgfpathlineto{\pgfqpoint{1.611320in}{1.253527in}}%
\pgfpathlineto{\pgfqpoint{1.615040in}{1.250358in}}%
\pgfpathlineto{\pgfqpoint{1.617520in}{1.250647in}}%
\pgfpathlineto{\pgfqpoint{1.618760in}{1.253088in}}%
\pgfpathlineto{\pgfqpoint{1.620000in}{1.253013in}}%
\pgfpathlineto{\pgfqpoint{1.621240in}{1.254358in}}%
\pgfpathlineto{\pgfqpoint{1.622480in}{1.253891in}}%
\pgfpathlineto{\pgfqpoint{1.623720in}{1.255935in}}%
\pgfpathlineto{\pgfqpoint{1.626200in}{1.255776in}}%
\pgfpathlineto{\pgfqpoint{1.628680in}{1.255897in}}%
\pgfpathlineto{\pgfqpoint{1.631160in}{1.254709in}}%
\pgfpathlineto{\pgfqpoint{1.632400in}{1.256828in}}%
\pgfpathlineto{\pgfqpoint{1.634880in}{1.256076in}}%
\pgfpathlineto{\pgfqpoint{1.636120in}{1.256881in}}%
\pgfpathlineto{\pgfqpoint{1.637360in}{1.255625in}}%
\pgfpathlineto{\pgfqpoint{1.638600in}{1.252475in}}%
\pgfpathlineto{\pgfqpoint{1.642320in}{1.252879in}}%
\pgfpathlineto{\pgfqpoint{1.649760in}{1.257540in}}%
\pgfpathlineto{\pgfqpoint{1.651000in}{1.256698in}}%
\pgfpathlineto{\pgfqpoint{1.652240in}{1.253392in}}%
\pgfpathlineto{\pgfqpoint{1.654720in}{1.254226in}}%
\pgfpathlineto{\pgfqpoint{1.655960in}{1.254774in}}%
\pgfpathlineto{\pgfqpoint{1.665880in}{1.247916in}}%
\pgfpathlineto{\pgfqpoint{1.668360in}{1.251342in}}%
\pgfpathlineto{\pgfqpoint{1.670840in}{1.248351in}}%
\pgfpathlineto{\pgfqpoint{1.673320in}{1.252748in}}%
\pgfpathlineto{\pgfqpoint{1.677040in}{1.251665in}}%
\pgfpathlineto{\pgfqpoint{1.678280in}{1.250705in}}%
\pgfpathlineto{\pgfqpoint{1.679520in}{1.252052in}}%
\pgfpathlineto{\pgfqpoint{1.682000in}{1.251095in}}%
\pgfpathlineto{\pgfqpoint{1.684480in}{1.254702in}}%
\pgfpathlineto{\pgfqpoint{1.689440in}{1.251079in}}%
\pgfpathlineto{\pgfqpoint{1.699360in}{1.258440in}}%
\pgfpathlineto{\pgfqpoint{1.705560in}{1.252548in}}%
\pgfpathlineto{\pgfqpoint{1.708040in}{1.249103in}}%
\pgfpathlineto{\pgfqpoint{1.711760in}{1.246822in}}%
\pgfpathlineto{\pgfqpoint{1.716720in}{1.249573in}}%
\pgfpathlineto{\pgfqpoint{1.721680in}{1.252638in}}%
\pgfpathlineto{\pgfqpoint{1.724160in}{1.250776in}}%
\pgfpathlineto{\pgfqpoint{1.730360in}{1.247368in}}%
\pgfpathlineto{\pgfqpoint{1.735320in}{1.247828in}}%
\pgfpathlineto{\pgfqpoint{1.736560in}{1.245889in}}%
\pgfpathlineto{\pgfqpoint{1.741520in}{1.246800in}}%
\pgfpathlineto{\pgfqpoint{1.742760in}{1.249226in}}%
\pgfpathlineto{\pgfqpoint{1.745240in}{1.248079in}}%
\pgfpathlineto{\pgfqpoint{1.746480in}{1.247759in}}%
\pgfpathlineto{\pgfqpoint{1.747720in}{1.249534in}}%
\pgfpathlineto{\pgfqpoint{1.755160in}{1.247360in}}%
\pgfpathlineto{\pgfqpoint{1.756400in}{1.248823in}}%
\pgfpathlineto{\pgfqpoint{1.758880in}{1.247872in}}%
\pgfpathlineto{\pgfqpoint{1.760120in}{1.247993in}}%
\pgfpathlineto{\pgfqpoint{1.761360in}{1.246840in}}%
\pgfpathlineto{\pgfqpoint{1.763840in}{1.243160in}}%
\pgfpathlineto{\pgfqpoint{1.767560in}{1.244735in}}%
\pgfpathlineto{\pgfqpoint{1.772520in}{1.247145in}}%
\pgfpathlineto{\pgfqpoint{1.777480in}{1.242010in}}%
\pgfpathlineto{\pgfqpoint{1.781200in}{1.243944in}}%
\pgfpathlineto{\pgfqpoint{1.784920in}{1.241141in}}%
\pgfpathlineto{\pgfqpoint{1.786160in}{1.240007in}}%
\pgfpathlineto{\pgfqpoint{1.787400in}{1.240458in}}%
\pgfpathlineto{\pgfqpoint{1.789880in}{1.239599in}}%
\pgfpathlineto{\pgfqpoint{1.792360in}{1.243196in}}%
\pgfpathlineto{\pgfqpoint{1.794840in}{1.239851in}}%
\pgfpathlineto{\pgfqpoint{1.797320in}{1.243107in}}%
\pgfpathlineto{\pgfqpoint{1.802280in}{1.242428in}}%
\pgfpathlineto{\pgfqpoint{1.803520in}{1.243923in}}%
\pgfpathlineto{\pgfqpoint{1.806000in}{1.242728in}}%
\pgfpathlineto{\pgfqpoint{1.808480in}{1.245426in}}%
\pgfpathlineto{\pgfqpoint{1.814680in}{1.241180in}}%
\pgfpathlineto{\pgfqpoint{1.820880in}{1.246425in}}%
\pgfpathlineto{\pgfqpoint{1.824600in}{1.246165in}}%
\pgfpathlineto{\pgfqpoint{1.827080in}{1.242889in}}%
\pgfpathlineto{\pgfqpoint{1.830800in}{1.240272in}}%
\pgfpathlineto{\pgfqpoint{1.834520in}{1.235909in}}%
\pgfpathlineto{\pgfqpoint{1.838240in}{1.239264in}}%
\pgfpathlineto{\pgfqpoint{1.839480in}{1.240463in}}%
\pgfpathlineto{\pgfqpoint{1.841960in}{1.239853in}}%
\pgfpathlineto{\pgfqpoint{1.845680in}{1.241896in}}%
\pgfpathlineto{\pgfqpoint{1.850640in}{1.236754in}}%
\pgfpathlineto{\pgfqpoint{1.851880in}{1.236012in}}%
\pgfpathlineto{\pgfqpoint{1.854360in}{1.237810in}}%
\pgfpathlineto{\pgfqpoint{1.859320in}{1.238502in}}%
\pgfpathlineto{\pgfqpoint{1.861800in}{1.236403in}}%
\pgfpathlineto{\pgfqpoint{1.863040in}{1.235364in}}%
\pgfpathlineto{\pgfqpoint{1.865520in}{1.236746in}}%
\pgfpathlineto{\pgfqpoint{1.866760in}{1.238865in}}%
\pgfpathlineto{\pgfqpoint{1.868000in}{1.238146in}}%
\pgfpathlineto{\pgfqpoint{1.869240in}{1.240865in}}%
\pgfpathlineto{\pgfqpoint{1.870480in}{1.240487in}}%
\pgfpathlineto{\pgfqpoint{1.871720in}{1.241969in}}%
\pgfpathlineto{\pgfqpoint{1.879160in}{1.240467in}}%
\pgfpathlineto{\pgfqpoint{1.881640in}{1.241797in}}%
\pgfpathlineto{\pgfqpoint{1.885360in}{1.239732in}}%
\pgfpathlineto{\pgfqpoint{1.887840in}{1.236561in}}%
\pgfpathlineto{\pgfqpoint{1.895280in}{1.238935in}}%
\pgfpathlineto{\pgfqpoint{1.896520in}{1.240255in}}%
\pgfpathlineto{\pgfqpoint{1.899000in}{1.238385in}}%
\pgfpathlineto{\pgfqpoint{1.901480in}{1.235192in}}%
\pgfpathlineto{\pgfqpoint{1.905200in}{1.235671in}}%
\pgfpathlineto{\pgfqpoint{1.907680in}{1.234013in}}%
\pgfpathlineto{\pgfqpoint{1.910160in}{1.231238in}}%
\pgfpathlineto{\pgfqpoint{1.911400in}{1.231386in}}%
\pgfpathlineto{\pgfqpoint{1.913880in}{1.229884in}}%
\pgfpathlineto{\pgfqpoint{1.916360in}{1.231505in}}%
\pgfpathlineto{\pgfqpoint{1.918840in}{1.227671in}}%
\pgfpathlineto{\pgfqpoint{1.921320in}{1.231276in}}%
\pgfpathlineto{\pgfqpoint{1.926280in}{1.230418in}}%
\pgfpathlineto{\pgfqpoint{1.928760in}{1.231351in}}%
\pgfpathlineto{\pgfqpoint{1.930000in}{1.230441in}}%
\pgfpathlineto{\pgfqpoint{1.932480in}{1.233806in}}%
\pgfpathlineto{\pgfqpoint{1.938680in}{1.228742in}}%
\pgfpathlineto{\pgfqpoint{1.947360in}{1.235300in}}%
\pgfpathlineto{\pgfqpoint{1.953560in}{1.230998in}}%
\pgfpathlineto{\pgfqpoint{1.957280in}{1.226621in}}%
\pgfpathlineto{\pgfqpoint{1.958520in}{1.226192in}}%
\pgfpathlineto{\pgfqpoint{1.969680in}{1.233629in}}%
\pgfpathlineto{\pgfqpoint{1.974640in}{1.226480in}}%
\pgfpathlineto{\pgfqpoint{1.975880in}{1.225766in}}%
\pgfpathlineto{\pgfqpoint{1.978360in}{1.228568in}}%
\pgfpathlineto{\pgfqpoint{1.979600in}{1.229193in}}%
\pgfpathlineto{\pgfqpoint{1.982080in}{1.228249in}}%
\pgfpathlineto{\pgfqpoint{1.983320in}{1.228412in}}%
\pgfpathlineto{\pgfqpoint{1.987040in}{1.225706in}}%
\pgfpathlineto{\pgfqpoint{1.989520in}{1.226416in}}%
\pgfpathlineto{\pgfqpoint{1.990760in}{1.228283in}}%
\pgfpathlineto{\pgfqpoint{1.992000in}{1.226900in}}%
\pgfpathlineto{\pgfqpoint{1.993240in}{1.231891in}}%
\pgfpathlineto{\pgfqpoint{1.994480in}{1.231693in}}%
\pgfpathlineto{\pgfqpoint{1.995720in}{1.233710in}}%
\pgfpathlineto{\pgfqpoint{2.000680in}{1.231791in}}%
\pgfpathlineto{\pgfqpoint{2.003160in}{1.231729in}}%
\pgfpathlineto{\pgfqpoint{2.005640in}{1.232862in}}%
\pgfpathlineto{\pgfqpoint{2.011840in}{1.229031in}}%
\pgfpathlineto{\pgfqpoint{2.019280in}{1.230479in}}%
\pgfpathlineto{\pgfqpoint{2.020520in}{1.232080in}}%
\pgfpathlineto{\pgfqpoint{2.024240in}{1.226930in}}%
\pgfpathlineto{\pgfqpoint{2.029200in}{1.228171in}}%
\pgfpathlineto{\pgfqpoint{2.031680in}{1.226357in}}%
\pgfpathlineto{\pgfqpoint{2.034160in}{1.223773in}}%
\pgfpathlineto{\pgfqpoint{2.035400in}{1.224498in}}%
\pgfpathlineto{\pgfqpoint{2.037880in}{1.223117in}}%
\pgfpathlineto{\pgfqpoint{2.040360in}{1.224402in}}%
\pgfpathlineto{\pgfqpoint{2.042840in}{1.220692in}}%
\pgfpathlineto{\pgfqpoint{2.045320in}{1.223717in}}%
\pgfpathlineto{\pgfqpoint{2.047800in}{1.223193in}}%
\pgfpathlineto{\pgfqpoint{2.052760in}{1.224476in}}%
\pgfpathlineto{\pgfqpoint{2.054000in}{1.222876in}}%
\pgfpathlineto{\pgfqpoint{2.056480in}{1.225164in}}%
\pgfpathlineto{\pgfqpoint{2.062680in}{1.221634in}}%
\pgfpathlineto{\pgfqpoint{2.071360in}{1.226574in}}%
\pgfpathlineto{\pgfqpoint{2.072600in}{1.226366in}}%
\pgfpathlineto{\pgfqpoint{2.078800in}{1.219162in}}%
\pgfpathlineto{\pgfqpoint{2.083760in}{1.217223in}}%
\pgfpathlineto{\pgfqpoint{2.093680in}{1.223205in}}%
\pgfpathlineto{\pgfqpoint{2.098640in}{1.215285in}}%
\pgfpathlineto{\pgfqpoint{2.099880in}{1.214684in}}%
\pgfpathlineto{\pgfqpoint{2.103600in}{1.217873in}}%
\pgfpathlineto{\pgfqpoint{2.106080in}{1.216502in}}%
\pgfpathlineto{\pgfqpoint{2.107320in}{1.216362in}}%
\pgfpathlineto{\pgfqpoint{2.109800in}{1.215019in}}%
\pgfpathlineto{\pgfqpoint{2.112280in}{1.214693in}}%
\pgfpathlineto{\pgfqpoint{2.113520in}{1.214496in}}%
\pgfpathlineto{\pgfqpoint{2.114760in}{1.216149in}}%
\pgfpathlineto{\pgfqpoint{2.116000in}{1.214726in}}%
\pgfpathlineto{\pgfqpoint{2.117240in}{1.219650in}}%
\pgfpathlineto{\pgfqpoint{2.118480in}{1.219734in}}%
\pgfpathlineto{\pgfqpoint{2.119720in}{1.222033in}}%
\pgfpathlineto{\pgfqpoint{2.127160in}{1.220298in}}%
\pgfpathlineto{\pgfqpoint{2.129640in}{1.221517in}}%
\pgfpathlineto{\pgfqpoint{2.132120in}{1.220209in}}%
\pgfpathlineto{\pgfqpoint{2.133360in}{1.219971in}}%
\pgfpathlineto{\pgfqpoint{2.135840in}{1.217681in}}%
\pgfpathlineto{\pgfqpoint{2.138320in}{1.218528in}}%
\pgfpathlineto{\pgfqpoint{2.139560in}{1.219802in}}%
\pgfpathlineto{\pgfqpoint{2.142040in}{1.218673in}}%
\pgfpathlineto{\pgfqpoint{2.144520in}{1.221093in}}%
\pgfpathlineto{\pgfqpoint{2.148240in}{1.216304in}}%
\pgfpathlineto{\pgfqpoint{2.151960in}{1.218637in}}%
\pgfpathlineto{\pgfqpoint{2.155680in}{1.216805in}}%
\pgfpathlineto{\pgfqpoint{2.158160in}{1.214366in}}%
\pgfpathlineto{\pgfqpoint{2.159400in}{1.214852in}}%
\pgfpathlineto{\pgfqpoint{2.161880in}{1.214009in}}%
\pgfpathlineto{\pgfqpoint{2.164360in}{1.215395in}}%
\pgfpathlineto{\pgfqpoint{2.166840in}{1.211749in}}%
\pgfpathlineto{\pgfqpoint{2.169320in}{1.215084in}}%
\pgfpathlineto{\pgfqpoint{2.171800in}{1.214822in}}%
\pgfpathlineto{\pgfqpoint{2.175520in}{1.217412in}}%
\pgfpathlineto{\pgfqpoint{2.178000in}{1.215209in}}%
\pgfpathlineto{\pgfqpoint{2.180480in}{1.218077in}}%
\pgfpathlineto{\pgfqpoint{2.185440in}{1.214830in}}%
\pgfpathlineto{\pgfqpoint{2.189160in}{1.215648in}}%
\pgfpathlineto{\pgfqpoint{2.194120in}{1.218491in}}%
\pgfpathlineto{\pgfqpoint{2.196600in}{1.218973in}}%
\pgfpathlineto{\pgfqpoint{2.200320in}{1.214380in}}%
\pgfpathlineto{\pgfqpoint{2.201560in}{1.213629in}}%
\pgfpathlineto{\pgfqpoint{2.204040in}{1.210347in}}%
\pgfpathlineto{\pgfqpoint{2.206520in}{1.208201in}}%
\pgfpathlineto{\pgfqpoint{2.211480in}{1.212717in}}%
\pgfpathlineto{\pgfqpoint{2.213960in}{1.212232in}}%
\pgfpathlineto{\pgfqpoint{2.217680in}{1.214578in}}%
\pgfpathlineto{\pgfqpoint{2.222640in}{1.207108in}}%
\pgfpathlineto{\pgfqpoint{2.223880in}{1.206982in}}%
\pgfpathlineto{\pgfqpoint{2.226360in}{1.210172in}}%
\pgfpathlineto{\pgfqpoint{2.227600in}{1.210967in}}%
\pgfpathlineto{\pgfqpoint{2.230080in}{1.209355in}}%
\pgfpathlineto{\pgfqpoint{2.232560in}{1.207928in}}%
\pgfpathlineto{\pgfqpoint{2.237520in}{1.205669in}}%
\pgfpathlineto{\pgfqpoint{2.238760in}{1.207361in}}%
\pgfpathlineto{\pgfqpoint{2.240000in}{1.206151in}}%
\pgfpathlineto{\pgfqpoint{2.241240in}{1.211586in}}%
\pgfpathlineto{\pgfqpoint{2.242480in}{1.211874in}}%
\pgfpathlineto{\pgfqpoint{2.244960in}{1.214467in}}%
\pgfpathlineto{\pgfqpoint{2.249920in}{1.211270in}}%
\pgfpathlineto{\pgfqpoint{2.251160in}{1.211600in}}%
\pgfpathlineto{\pgfqpoint{2.253640in}{1.212962in}}%
\pgfpathlineto{\pgfqpoint{2.254880in}{1.211664in}}%
\pgfpathlineto{\pgfqpoint{2.257360in}{1.211926in}}%
\pgfpathlineto{\pgfqpoint{2.259840in}{1.209536in}}%
\pgfpathlineto{\pgfqpoint{2.262320in}{1.209513in}}%
\pgfpathlineto{\pgfqpoint{2.263560in}{1.210505in}}%
\pgfpathlineto{\pgfqpoint{2.266040in}{1.209609in}}%
\pgfpathlineto{\pgfqpoint{2.268520in}{1.211819in}}%
\pgfpathlineto{\pgfqpoint{2.272240in}{1.207316in}}%
\pgfpathlineto{\pgfqpoint{2.277200in}{1.209385in}}%
\pgfpathlineto{\pgfqpoint{2.279680in}{1.208682in}}%
\pgfpathlineto{\pgfqpoint{2.282160in}{1.207019in}}%
\pgfpathlineto{\pgfqpoint{2.283400in}{1.207036in}}%
\pgfpathlineto{\pgfqpoint{2.285880in}{1.206118in}}%
\pgfpathlineto{\pgfqpoint{2.288360in}{1.206872in}}%
\pgfpathlineto{\pgfqpoint{2.290840in}{1.202713in}}%
\pgfpathlineto{\pgfqpoint{2.293320in}{1.204962in}}%
\pgfpathlineto{\pgfqpoint{2.294560in}{1.204257in}}%
\pgfpathlineto{\pgfqpoint{2.300760in}{1.207056in}}%
\pgfpathlineto{\pgfqpoint{2.302000in}{1.205143in}}%
\pgfpathlineto{\pgfqpoint{2.304480in}{1.208416in}}%
\pgfpathlineto{\pgfqpoint{2.309440in}{1.205530in}}%
\pgfpathlineto{\pgfqpoint{2.313160in}{1.206507in}}%
\pgfpathlineto{\pgfqpoint{2.319360in}{1.210911in}}%
\pgfpathlineto{\pgfqpoint{2.320600in}{1.210481in}}%
\pgfpathlineto{\pgfqpoint{2.324320in}{1.205801in}}%
\pgfpathlineto{\pgfqpoint{2.325560in}{1.205345in}}%
\pgfpathlineto{\pgfqpoint{2.328040in}{1.202276in}}%
\pgfpathlineto{\pgfqpoint{2.330520in}{1.200603in}}%
\pgfpathlineto{\pgfqpoint{2.335480in}{1.204694in}}%
\pgfpathlineto{\pgfqpoint{2.336720in}{1.203731in}}%
\pgfpathlineto{\pgfqpoint{2.340440in}{1.206206in}}%
\pgfpathlineto{\pgfqpoint{2.341680in}{1.206962in}}%
\pgfpathlineto{\pgfqpoint{2.346640in}{1.201322in}}%
\pgfpathlineto{\pgfqpoint{2.347880in}{1.200844in}}%
\pgfpathlineto{\pgfqpoint{2.351600in}{1.205182in}}%
\pgfpathlineto{\pgfqpoint{2.359040in}{1.200746in}}%
\pgfpathlineto{\pgfqpoint{2.364000in}{1.200388in}}%
\pgfpathlineto{\pgfqpoint{2.365240in}{1.204971in}}%
\pgfpathlineto{\pgfqpoint{2.366480in}{1.205110in}}%
\pgfpathlineto{\pgfqpoint{2.367720in}{1.207199in}}%
\pgfpathlineto{\pgfqpoint{2.375160in}{1.205857in}}%
\pgfpathlineto{\pgfqpoint{2.377640in}{1.207187in}}%
\pgfpathlineto{\pgfqpoint{2.380120in}{1.206124in}}%
\pgfpathlineto{\pgfqpoint{2.381360in}{1.205972in}}%
\pgfpathlineto{\pgfqpoint{2.383840in}{1.203305in}}%
\pgfpathlineto{\pgfqpoint{2.386320in}{1.202978in}}%
\pgfpathlineto{\pgfqpoint{2.388800in}{1.203556in}}%
\pgfpathlineto{\pgfqpoint{2.390040in}{1.203202in}}%
\pgfpathlineto{\pgfqpoint{2.392520in}{1.204696in}}%
\pgfpathlineto{\pgfqpoint{2.397480in}{1.200735in}}%
\pgfpathlineto{\pgfqpoint{2.399960in}{1.203399in}}%
\pgfpathlineto{\pgfqpoint{2.403680in}{1.202513in}}%
\pgfpathlineto{\pgfqpoint{2.406160in}{1.201227in}}%
\pgfpathlineto{\pgfqpoint{2.407400in}{1.201419in}}%
\pgfpathlineto{\pgfqpoint{2.408640in}{1.200423in}}%
\pgfpathlineto{\pgfqpoint{2.412360in}{1.202062in}}%
\pgfpathlineto{\pgfqpoint{2.414840in}{1.197645in}}%
\pgfpathlineto{\pgfqpoint{2.424760in}{1.202110in}}%
\pgfpathlineto{\pgfqpoint{2.426000in}{1.200460in}}%
\pgfpathlineto{\pgfqpoint{2.428480in}{1.203387in}}%
\pgfpathlineto{\pgfqpoint{2.434680in}{1.201834in}}%
\pgfpathlineto{\pgfqpoint{2.437160in}{1.203162in}}%
\pgfpathlineto{\pgfqpoint{2.443360in}{1.207026in}}%
\pgfpathlineto{\pgfqpoint{2.445840in}{1.204646in}}%
\pgfpathlineto{\pgfqpoint{2.448320in}{1.202260in}}%
\pgfpathlineto{\pgfqpoint{2.449560in}{1.201983in}}%
\pgfpathlineto{\pgfqpoint{2.452040in}{1.199486in}}%
\pgfpathlineto{\pgfqpoint{2.454520in}{1.198255in}}%
\pgfpathlineto{\pgfqpoint{2.459480in}{1.201599in}}%
\pgfpathlineto{\pgfqpoint{2.460720in}{1.200562in}}%
\pgfpathlineto{\pgfqpoint{2.465680in}{1.203959in}}%
\pgfpathlineto{\pgfqpoint{2.470640in}{1.199089in}}%
\pgfpathlineto{\pgfqpoint{2.471880in}{1.198309in}}%
\pgfpathlineto{\pgfqpoint{2.475600in}{1.202476in}}%
\pgfpathlineto{\pgfqpoint{2.481800in}{1.200225in}}%
\pgfpathlineto{\pgfqpoint{2.485520in}{1.199314in}}%
\pgfpathlineto{\pgfqpoint{2.486760in}{1.200691in}}%
\pgfpathlineto{\pgfqpoint{2.488000in}{1.199911in}}%
\pgfpathlineto{\pgfqpoint{2.489240in}{1.203610in}}%
\pgfpathlineto{\pgfqpoint{2.490480in}{1.203568in}}%
\pgfpathlineto{\pgfqpoint{2.491720in}{1.205093in}}%
\pgfpathlineto{\pgfqpoint{2.499160in}{1.203608in}}%
\pgfpathlineto{\pgfqpoint{2.501640in}{1.204619in}}%
\pgfpathlineto{\pgfqpoint{2.504120in}{1.204490in}}%
\pgfpathlineto{\pgfqpoint{2.505360in}{1.204624in}}%
\pgfpathlineto{\pgfqpoint{2.509080in}{1.200802in}}%
\pgfpathlineto{\pgfqpoint{2.516520in}{1.202945in}}%
\pgfpathlineto{\pgfqpoint{2.521480in}{1.198611in}}%
\pgfpathlineto{\pgfqpoint{2.523960in}{1.201119in}}%
\pgfpathlineto{\pgfqpoint{2.527680in}{1.200346in}}%
\pgfpathlineto{\pgfqpoint{2.530160in}{1.198801in}}%
\pgfpathlineto{\pgfqpoint{2.533880in}{1.199157in}}%
\pgfpathlineto{\pgfqpoint{2.536360in}{1.200698in}}%
\pgfpathlineto{\pgfqpoint{2.538840in}{1.196238in}}%
\pgfpathlineto{\pgfqpoint{2.545040in}{1.198213in}}%
\pgfpathlineto{\pgfqpoint{2.546280in}{1.197749in}}%
\pgfpathlineto{\pgfqpoint{2.547520in}{1.199117in}}%
\pgfpathlineto{\pgfqpoint{2.548760in}{1.198625in}}%
\pgfpathlineto{\pgfqpoint{2.550000in}{1.196832in}}%
\pgfpathlineto{\pgfqpoint{2.552480in}{1.199865in}}%
\pgfpathlineto{\pgfqpoint{2.558680in}{1.198266in}}%
\pgfpathlineto{\pgfqpoint{2.561160in}{1.199974in}}%
\pgfpathlineto{\pgfqpoint{2.566120in}{1.203077in}}%
\pgfpathlineto{\pgfqpoint{2.568600in}{1.201994in}}%
\pgfpathlineto{\pgfqpoint{2.576040in}{1.194666in}}%
\pgfpathlineto{\pgfqpoint{2.578520in}{1.193511in}}%
\pgfpathlineto{\pgfqpoint{2.584720in}{1.196185in}}%
\pgfpathlineto{\pgfqpoint{2.588440in}{1.197583in}}%
\pgfpathlineto{\pgfqpoint{2.589680in}{1.198416in}}%
\pgfpathlineto{\pgfqpoint{2.595880in}{1.192812in}}%
\pgfpathlineto{\pgfqpoint{2.599600in}{1.197282in}}%
\pgfpathlineto{\pgfqpoint{2.603320in}{1.196458in}}%
\pgfpathlineto{\pgfqpoint{2.607040in}{1.195213in}}%
\pgfpathlineto{\pgfqpoint{2.610760in}{1.196327in}}%
\pgfpathlineto{\pgfqpoint{2.612000in}{1.195125in}}%
\pgfpathlineto{\pgfqpoint{2.613240in}{1.197286in}}%
\pgfpathlineto{\pgfqpoint{2.614480in}{1.197095in}}%
\pgfpathlineto{\pgfqpoint{2.616960in}{1.199066in}}%
\pgfpathlineto{\pgfqpoint{2.620680in}{1.199138in}}%
\pgfpathlineto{\pgfqpoint{2.623160in}{1.198919in}}%
\pgfpathlineto{\pgfqpoint{2.625640in}{1.199812in}}%
\pgfpathlineto{\pgfqpoint{2.630600in}{1.198690in}}%
\pgfpathlineto{\pgfqpoint{2.633080in}{1.195726in}}%
\pgfpathlineto{\pgfqpoint{2.636800in}{1.196404in}}%
\pgfpathlineto{\pgfqpoint{2.641760in}{1.194931in}}%
\pgfpathlineto{\pgfqpoint{2.644240in}{1.192147in}}%
\pgfpathlineto{\pgfqpoint{2.650440in}{1.196001in}}%
\pgfpathlineto{\pgfqpoint{2.652920in}{1.194697in}}%
\pgfpathlineto{\pgfqpoint{2.656640in}{1.192808in}}%
\pgfpathlineto{\pgfqpoint{2.660360in}{1.194731in}}%
\pgfpathlineto{\pgfqpoint{2.662840in}{1.190360in}}%
\pgfpathlineto{\pgfqpoint{2.666560in}{1.191038in}}%
\pgfpathlineto{\pgfqpoint{2.670280in}{1.191938in}}%
\pgfpathlineto{\pgfqpoint{2.672760in}{1.193567in}}%
\pgfpathlineto{\pgfqpoint{2.674000in}{1.192174in}}%
\pgfpathlineto{\pgfqpoint{2.676480in}{1.195588in}}%
\pgfpathlineto{\pgfqpoint{2.682680in}{1.194096in}}%
\pgfpathlineto{\pgfqpoint{2.685160in}{1.196018in}}%
\pgfpathlineto{\pgfqpoint{2.690120in}{1.199317in}}%
\pgfpathlineto{\pgfqpoint{2.692600in}{1.198676in}}%
\pgfpathlineto{\pgfqpoint{2.700040in}{1.191463in}}%
\pgfpathlineto{\pgfqpoint{2.702520in}{1.190009in}}%
\pgfpathlineto{\pgfqpoint{2.705000in}{1.191255in}}%
\pgfpathlineto{\pgfqpoint{2.707480in}{1.194240in}}%
\pgfpathlineto{\pgfqpoint{2.709960in}{1.193562in}}%
\pgfpathlineto{\pgfqpoint{2.712440in}{1.194090in}}%
\pgfpathlineto{\pgfqpoint{2.713680in}{1.194877in}}%
\pgfpathlineto{\pgfqpoint{2.719880in}{1.190004in}}%
\pgfpathlineto{\pgfqpoint{2.724840in}{1.193592in}}%
\pgfpathlineto{\pgfqpoint{2.727320in}{1.193930in}}%
\pgfpathlineto{\pgfqpoint{2.732280in}{1.192681in}}%
\pgfpathlineto{\pgfqpoint{2.733520in}{1.192723in}}%
\pgfpathlineto{\pgfqpoint{2.734760in}{1.193953in}}%
\pgfpathlineto{\pgfqpoint{2.737240in}{1.192493in}}%
\pgfpathlineto{\pgfqpoint{2.738480in}{1.192091in}}%
\pgfpathlineto{\pgfqpoint{2.740960in}{1.194226in}}%
\pgfpathlineto{\pgfqpoint{2.743440in}{1.194889in}}%
\pgfpathlineto{\pgfqpoint{2.750880in}{1.196267in}}%
\pgfpathlineto{\pgfqpoint{2.753360in}{1.197080in}}%
\pgfpathlineto{\pgfqpoint{2.757080in}{1.192151in}}%
\pgfpathlineto{\pgfqpoint{2.764520in}{1.193717in}}%
\pgfpathlineto{\pgfqpoint{2.768240in}{1.189444in}}%
\pgfpathlineto{\pgfqpoint{2.769480in}{1.189979in}}%
\pgfpathlineto{\pgfqpoint{2.771960in}{1.192777in}}%
\pgfpathlineto{\pgfqpoint{2.775680in}{1.192514in}}%
\pgfpathlineto{\pgfqpoint{2.778160in}{1.190817in}}%
\pgfpathlineto{\pgfqpoint{2.779400in}{1.190834in}}%
\pgfpathlineto{\pgfqpoint{2.780640in}{1.189669in}}%
\pgfpathlineto{\pgfqpoint{2.784360in}{1.191712in}}%
\pgfpathlineto{\pgfqpoint{2.786840in}{1.187378in}}%
\pgfpathlineto{\pgfqpoint{2.789320in}{1.189197in}}%
\pgfpathlineto{\pgfqpoint{2.791800in}{1.188557in}}%
\pgfpathlineto{\pgfqpoint{2.793040in}{1.189368in}}%
\pgfpathlineto{\pgfqpoint{2.794280in}{1.188268in}}%
\pgfpathlineto{\pgfqpoint{2.796760in}{1.189907in}}%
\pgfpathlineto{\pgfqpoint{2.798000in}{1.188950in}}%
\pgfpathlineto{\pgfqpoint{2.800480in}{1.192372in}}%
\pgfpathlineto{\pgfqpoint{2.802960in}{1.191518in}}%
\pgfpathlineto{\pgfqpoint{2.805440in}{1.189915in}}%
\pgfpathlineto{\pgfqpoint{2.806680in}{1.190237in}}%
\pgfpathlineto{\pgfqpoint{2.807920in}{1.192114in}}%
\pgfpathlineto{\pgfqpoint{2.809160in}{1.191886in}}%
\pgfpathlineto{\pgfqpoint{2.814120in}{1.195402in}}%
\pgfpathlineto{\pgfqpoint{2.816600in}{1.193901in}}%
\pgfpathlineto{\pgfqpoint{2.824040in}{1.186508in}}%
\pgfpathlineto{\pgfqpoint{2.827760in}{1.184903in}}%
\pgfpathlineto{\pgfqpoint{2.833960in}{1.188823in}}%
\pgfpathlineto{\pgfqpoint{2.836440in}{1.188643in}}%
\pgfpathlineto{\pgfqpoint{2.837680in}{1.188997in}}%
\pgfpathlineto{\pgfqpoint{2.840160in}{1.187757in}}%
\pgfpathlineto{\pgfqpoint{2.841400in}{1.187446in}}%
\pgfpathlineto{\pgfqpoint{2.843880in}{1.184053in}}%
\pgfpathlineto{\pgfqpoint{2.848840in}{1.187542in}}%
\pgfpathlineto{\pgfqpoint{2.862480in}{1.187463in}}%
\pgfpathlineto{\pgfqpoint{2.864960in}{1.189935in}}%
\pgfpathlineto{\pgfqpoint{2.868680in}{1.190871in}}%
\pgfpathlineto{\pgfqpoint{2.871160in}{1.190197in}}%
\pgfpathlineto{\pgfqpoint{2.876120in}{1.191680in}}%
\pgfpathlineto{\pgfqpoint{2.877360in}{1.191772in}}%
\pgfpathlineto{\pgfqpoint{2.882320in}{1.187173in}}%
\pgfpathlineto{\pgfqpoint{2.884800in}{1.188404in}}%
\pgfpathlineto{\pgfqpoint{2.889760in}{1.186533in}}%
\pgfpathlineto{\pgfqpoint{2.892240in}{1.184817in}}%
\pgfpathlineto{\pgfqpoint{2.897200in}{1.188349in}}%
\pgfpathlineto{\pgfqpoint{2.900920in}{1.186784in}}%
\pgfpathlineto{\pgfqpoint{2.905880in}{1.185026in}}%
\pgfpathlineto{\pgfqpoint{2.908360in}{1.186688in}}%
\pgfpathlineto{\pgfqpoint{2.910840in}{1.182751in}}%
\pgfpathlineto{\pgfqpoint{2.914560in}{1.183316in}}%
\pgfpathlineto{\pgfqpoint{2.919520in}{1.183939in}}%
\pgfpathlineto{\pgfqpoint{2.923240in}{1.185926in}}%
\pgfpathlineto{\pgfqpoint{2.925720in}{1.186451in}}%
\pgfpathlineto{\pgfqpoint{2.930680in}{1.184601in}}%
\pgfpathlineto{\pgfqpoint{2.931920in}{1.186344in}}%
\pgfpathlineto{\pgfqpoint{2.933160in}{1.185901in}}%
\pgfpathlineto{\pgfqpoint{2.938120in}{1.188895in}}%
\pgfpathlineto{\pgfqpoint{2.940600in}{1.186821in}}%
\pgfpathlineto{\pgfqpoint{2.946800in}{1.178562in}}%
\pgfpathlineto{\pgfqpoint{2.948040in}{1.178676in}}%
\pgfpathlineto{\pgfqpoint{2.951760in}{1.176250in}}%
\pgfpathlineto{\pgfqpoint{2.956720in}{1.179206in}}%
\pgfpathlineto{\pgfqpoint{2.960440in}{1.179348in}}%
\pgfpathlineto{\pgfqpoint{2.965400in}{1.179366in}}%
\pgfpathlineto{\pgfqpoint{2.967880in}{1.176012in}}%
\pgfpathlineto{\pgfqpoint{2.974080in}{1.180095in}}%
\pgfpathlineto{\pgfqpoint{2.977800in}{1.179487in}}%
\pgfpathlineto{\pgfqpoint{2.981520in}{1.179800in}}%
\pgfpathlineto{\pgfqpoint{2.982760in}{1.180417in}}%
\pgfpathlineto{\pgfqpoint{2.986480in}{1.177947in}}%
\pgfpathlineto{\pgfqpoint{2.988960in}{1.180260in}}%
\pgfpathlineto{\pgfqpoint{2.993920in}{1.180702in}}%
\pgfpathlineto{\pgfqpoint{2.996400in}{1.181424in}}%
\pgfpathlineto{\pgfqpoint{3.000120in}{1.182607in}}%
\pgfpathlineto{\pgfqpoint{3.001360in}{1.182506in}}%
\pgfpathlineto{\pgfqpoint{3.006320in}{1.177407in}}%
\pgfpathlineto{\pgfqpoint{3.008800in}{1.178370in}}%
\pgfpathlineto{\pgfqpoint{3.012520in}{1.179168in}}%
\pgfpathlineto{\pgfqpoint{3.015000in}{1.175919in}}%
\pgfpathlineto{\pgfqpoint{3.016240in}{1.175353in}}%
\pgfpathlineto{\pgfqpoint{3.021200in}{1.178080in}}%
\pgfpathlineto{\pgfqpoint{3.026160in}{1.175507in}}%
\pgfpathlineto{\pgfqpoint{3.029880in}{1.173489in}}%
\pgfpathlineto{\pgfqpoint{3.032360in}{1.174636in}}%
\pgfpathlineto{\pgfqpoint{3.034840in}{1.170640in}}%
\pgfpathlineto{\pgfqpoint{3.037320in}{1.172269in}}%
\pgfpathlineto{\pgfqpoint{3.039800in}{1.170771in}}%
\pgfpathlineto{\pgfqpoint{3.041040in}{1.171867in}}%
\pgfpathlineto{\pgfqpoint{3.042280in}{1.171424in}}%
\pgfpathlineto{\pgfqpoint{3.044760in}{1.172917in}}%
\pgfpathlineto{\pgfqpoint{3.046000in}{1.172409in}}%
\pgfpathlineto{\pgfqpoint{3.048480in}{1.175394in}}%
\pgfpathlineto{\pgfqpoint{3.050960in}{1.174798in}}%
\pgfpathlineto{\pgfqpoint{3.053440in}{1.173795in}}%
\pgfpathlineto{\pgfqpoint{3.054680in}{1.173944in}}%
\pgfpathlineto{\pgfqpoint{3.055920in}{1.175802in}}%
\pgfpathlineto{\pgfqpoint{3.057160in}{1.174985in}}%
\pgfpathlineto{\pgfqpoint{3.063360in}{1.177751in}}%
\pgfpathlineto{\pgfqpoint{3.065840in}{1.174323in}}%
\pgfpathlineto{\pgfqpoint{3.069560in}{1.168991in}}%
\pgfpathlineto{\pgfqpoint{3.070800in}{1.168157in}}%
\pgfpathlineto{\pgfqpoint{3.072040in}{1.168675in}}%
\pgfpathlineto{\pgfqpoint{3.075760in}{1.166141in}}%
\pgfpathlineto{\pgfqpoint{3.078240in}{1.168125in}}%
\pgfpathlineto{\pgfqpoint{3.079480in}{1.169945in}}%
\pgfpathlineto{\pgfqpoint{3.083200in}{1.169302in}}%
\pgfpathlineto{\pgfqpoint{3.085680in}{1.169732in}}%
\pgfpathlineto{\pgfqpoint{3.088160in}{1.168994in}}%
\pgfpathlineto{\pgfqpoint{3.089400in}{1.169291in}}%
\pgfpathlineto{\pgfqpoint{3.091880in}{1.166190in}}%
\pgfpathlineto{\pgfqpoint{3.093120in}{1.166761in}}%
\pgfpathlineto{\pgfqpoint{3.095600in}{1.170391in}}%
\pgfpathlineto{\pgfqpoint{3.104280in}{1.170179in}}%
\pgfpathlineto{\pgfqpoint{3.108000in}{1.169485in}}%
\pgfpathlineto{\pgfqpoint{3.110480in}{1.167267in}}%
\pgfpathlineto{\pgfqpoint{3.114200in}{1.169412in}}%
\pgfpathlineto{\pgfqpoint{3.121640in}{1.170290in}}%
\pgfpathlineto{\pgfqpoint{3.125360in}{1.170624in}}%
\pgfpathlineto{\pgfqpoint{3.130320in}{1.166387in}}%
\pgfpathlineto{\pgfqpoint{3.132800in}{1.167690in}}%
\pgfpathlineto{\pgfqpoint{3.136520in}{1.167866in}}%
\pgfpathlineto{\pgfqpoint{3.139000in}{1.164515in}}%
\pgfpathlineto{\pgfqpoint{3.140240in}{1.163963in}}%
\pgfpathlineto{\pgfqpoint{3.143960in}{1.165692in}}%
\pgfpathlineto{\pgfqpoint{3.150160in}{1.163314in}}%
\pgfpathlineto{\pgfqpoint{3.153880in}{1.161373in}}%
\pgfpathlineto{\pgfqpoint{3.156360in}{1.162698in}}%
\pgfpathlineto{\pgfqpoint{3.158840in}{1.159389in}}%
\pgfpathlineto{\pgfqpoint{3.161320in}{1.160529in}}%
\pgfpathlineto{\pgfqpoint{3.163800in}{1.158809in}}%
\pgfpathlineto{\pgfqpoint{3.165040in}{1.159885in}}%
\pgfpathlineto{\pgfqpoint{3.166280in}{1.159251in}}%
\pgfpathlineto{\pgfqpoint{3.168760in}{1.160281in}}%
\pgfpathlineto{\pgfqpoint{3.170000in}{1.159845in}}%
\pgfpathlineto{\pgfqpoint{3.172480in}{1.162760in}}%
\pgfpathlineto{\pgfqpoint{3.174960in}{1.161984in}}%
\pgfpathlineto{\pgfqpoint{3.177440in}{1.160562in}}%
\pgfpathlineto{\pgfqpoint{3.178680in}{1.160472in}}%
\pgfpathlineto{\pgfqpoint{3.179920in}{1.162394in}}%
\pgfpathlineto{\pgfqpoint{3.181160in}{1.161621in}}%
\pgfpathlineto{\pgfqpoint{3.186120in}{1.164178in}}%
\pgfpathlineto{\pgfqpoint{3.188600in}{1.161442in}}%
\pgfpathlineto{\pgfqpoint{3.192320in}{1.155023in}}%
\pgfpathlineto{\pgfqpoint{3.199760in}{1.151489in}}%
\pgfpathlineto{\pgfqpoint{3.205960in}{1.154258in}}%
\pgfpathlineto{\pgfqpoint{3.209680in}{1.154426in}}%
\pgfpathlineto{\pgfqpoint{3.212160in}{1.153528in}}%
\pgfpathlineto{\pgfqpoint{3.213400in}{1.154011in}}%
\pgfpathlineto{\pgfqpoint{3.215880in}{1.150956in}}%
\pgfpathlineto{\pgfqpoint{3.217120in}{1.151623in}}%
\pgfpathlineto{\pgfqpoint{3.219600in}{1.154962in}}%
\pgfpathlineto{\pgfqpoint{3.229520in}{1.154767in}}%
\pgfpathlineto{\pgfqpoint{3.230760in}{1.155913in}}%
\pgfpathlineto{\pgfqpoint{3.234480in}{1.151661in}}%
\pgfpathlineto{\pgfqpoint{3.238200in}{1.154860in}}%
\pgfpathlineto{\pgfqpoint{3.240680in}{1.155055in}}%
\pgfpathlineto{\pgfqpoint{3.243160in}{1.153427in}}%
\pgfpathlineto{\pgfqpoint{3.248120in}{1.155427in}}%
\pgfpathlineto{\pgfqpoint{3.254320in}{1.150189in}}%
\pgfpathlineto{\pgfqpoint{3.256800in}{1.151253in}}%
\pgfpathlineto{\pgfqpoint{3.260520in}{1.152183in}}%
\pgfpathlineto{\pgfqpoint{3.263000in}{1.148692in}}%
\pgfpathlineto{\pgfqpoint{3.265480in}{1.148171in}}%
\pgfpathlineto{\pgfqpoint{3.267960in}{1.149300in}}%
\pgfpathlineto{\pgfqpoint{3.275400in}{1.146464in}}%
\pgfpathlineto{\pgfqpoint{3.277880in}{1.145509in}}%
\pgfpathlineto{\pgfqpoint{3.279120in}{1.146746in}}%
\pgfpathlineto{\pgfqpoint{3.280360in}{1.146281in}}%
\pgfpathlineto{\pgfqpoint{3.282840in}{1.142248in}}%
\pgfpathlineto{\pgfqpoint{3.285320in}{1.142949in}}%
\pgfpathlineto{\pgfqpoint{3.287800in}{1.141096in}}%
\pgfpathlineto{\pgfqpoint{3.289040in}{1.142359in}}%
\pgfpathlineto{\pgfqpoint{3.291520in}{1.141982in}}%
\pgfpathlineto{\pgfqpoint{3.295240in}{1.143969in}}%
\pgfpathlineto{\pgfqpoint{3.297720in}{1.144823in}}%
\pgfpathlineto{\pgfqpoint{3.302680in}{1.142338in}}%
\pgfpathlineto{\pgfqpoint{3.303920in}{1.144143in}}%
\pgfpathlineto{\pgfqpoint{3.306400in}{1.144454in}}%
\pgfpathlineto{\pgfqpoint{3.307640in}{1.144177in}}%
\pgfpathlineto{\pgfqpoint{3.310120in}{1.145418in}}%
\pgfpathlineto{\pgfqpoint{3.312600in}{1.143382in}}%
\pgfpathlineto{\pgfqpoint{3.318800in}{1.135204in}}%
\pgfpathlineto{\pgfqpoint{3.321280in}{1.134910in}}%
\pgfpathlineto{\pgfqpoint{3.323760in}{1.133236in}}%
\pgfpathlineto{\pgfqpoint{3.329960in}{1.137240in}}%
\pgfpathlineto{\pgfqpoint{3.338640in}{1.136192in}}%
\pgfpathlineto{\pgfqpoint{3.339880in}{1.135019in}}%
\pgfpathlineto{\pgfqpoint{3.341120in}{1.135683in}}%
\pgfpathlineto{\pgfqpoint{3.343600in}{1.139073in}}%
\pgfpathlineto{\pgfqpoint{3.347320in}{1.139026in}}%
\pgfpathlineto{\pgfqpoint{3.352280in}{1.139346in}}%
\pgfpathlineto{\pgfqpoint{3.354760in}{1.140812in}}%
\pgfpathlineto{\pgfqpoint{3.358480in}{1.136084in}}%
\pgfpathlineto{\pgfqpoint{3.363440in}{1.140079in}}%
\pgfpathlineto{\pgfqpoint{3.365920in}{1.139106in}}%
\pgfpathlineto{\pgfqpoint{3.368400in}{1.138813in}}%
\pgfpathlineto{\pgfqpoint{3.372120in}{1.139849in}}%
\pgfpathlineto{\pgfqpoint{3.378320in}{1.135148in}}%
\pgfpathlineto{\pgfqpoint{3.380800in}{1.136509in}}%
\pgfpathlineto{\pgfqpoint{3.383280in}{1.136490in}}%
\pgfpathlineto{\pgfqpoint{3.384520in}{1.136720in}}%
\pgfpathlineto{\pgfqpoint{3.387000in}{1.134175in}}%
\pgfpathlineto{\pgfqpoint{3.388240in}{1.133211in}}%
\pgfpathlineto{\pgfqpoint{3.391960in}{1.134851in}}%
\pgfpathlineto{\pgfqpoint{3.394440in}{1.134469in}}%
\pgfpathlineto{\pgfqpoint{3.396920in}{1.134686in}}%
\pgfpathlineto{\pgfqpoint{3.399400in}{1.133364in}}%
\pgfpathlineto{\pgfqpoint{3.401880in}{1.131861in}}%
\pgfpathlineto{\pgfqpoint{3.404360in}{1.132709in}}%
\pgfpathlineto{\pgfqpoint{3.406840in}{1.129002in}}%
\pgfpathlineto{\pgfqpoint{3.409320in}{1.130519in}}%
\pgfpathlineto{\pgfqpoint{3.411800in}{1.128358in}}%
\pgfpathlineto{\pgfqpoint{3.413040in}{1.129289in}}%
\pgfpathlineto{\pgfqpoint{3.414280in}{1.128374in}}%
\pgfpathlineto{\pgfqpoint{3.419240in}{1.131182in}}%
\pgfpathlineto{\pgfqpoint{3.421720in}{1.132165in}}%
\pgfpathlineto{\pgfqpoint{3.426680in}{1.128711in}}%
\pgfpathlineto{\pgfqpoint{3.427920in}{1.130389in}}%
\pgfpathlineto{\pgfqpoint{3.431640in}{1.129812in}}%
\pgfpathlineto{\pgfqpoint{3.434120in}{1.130747in}}%
\pgfpathlineto{\pgfqpoint{3.437840in}{1.126762in}}%
\pgfpathlineto{\pgfqpoint{3.441560in}{1.122875in}}%
\pgfpathlineto{\pgfqpoint{3.444040in}{1.121751in}}%
\pgfpathlineto{\pgfqpoint{3.445280in}{1.121812in}}%
\pgfpathlineto{\pgfqpoint{3.447760in}{1.120381in}}%
\pgfpathlineto{\pgfqpoint{3.453960in}{1.125256in}}%
\pgfpathlineto{\pgfqpoint{3.461400in}{1.126125in}}%
\pgfpathlineto{\pgfqpoint{3.463880in}{1.123119in}}%
\pgfpathlineto{\pgfqpoint{3.465120in}{1.123760in}}%
\pgfpathlineto{\pgfqpoint{3.467600in}{1.127337in}}%
\pgfpathlineto{\pgfqpoint{3.477520in}{1.128611in}}%
\pgfpathlineto{\pgfqpoint{3.478760in}{1.129584in}}%
\pgfpathlineto{\pgfqpoint{3.480000in}{1.127867in}}%
\pgfpathlineto{\pgfqpoint{3.482480in}{1.128771in}}%
\pgfpathlineto{\pgfqpoint{3.488680in}{1.132212in}}%
\pgfpathlineto{\pgfqpoint{3.492400in}{1.130536in}}%
\pgfpathlineto{\pgfqpoint{3.496120in}{1.131709in}}%
\pgfpathlineto{\pgfqpoint{3.502320in}{1.126142in}}%
\pgfpathlineto{\pgfqpoint{3.506040in}{1.126456in}}%
\pgfpathlineto{\pgfqpoint{3.508520in}{1.126697in}}%
\pgfpathlineto{\pgfqpoint{3.511000in}{1.124217in}}%
\pgfpathlineto{\pgfqpoint{3.512240in}{1.123407in}}%
\pgfpathlineto{\pgfqpoint{3.517200in}{1.124990in}}%
\pgfpathlineto{\pgfqpoint{3.520920in}{1.125145in}}%
\pgfpathlineto{\pgfqpoint{3.525880in}{1.121814in}}%
\pgfpathlineto{\pgfqpoint{3.528360in}{1.123345in}}%
\pgfpathlineto{\pgfqpoint{3.530840in}{1.119807in}}%
\pgfpathlineto{\pgfqpoint{3.534560in}{1.120170in}}%
\pgfpathlineto{\pgfqpoint{3.535800in}{1.118921in}}%
\pgfpathlineto{\pgfqpoint{3.537040in}{1.119746in}}%
\pgfpathlineto{\pgfqpoint{3.539520in}{1.119175in}}%
\pgfpathlineto{\pgfqpoint{3.540760in}{1.119843in}}%
\pgfpathlineto{\pgfqpoint{3.542000in}{1.118896in}}%
\pgfpathlineto{\pgfqpoint{3.544480in}{1.122214in}}%
\pgfpathlineto{\pgfqpoint{3.546960in}{1.121885in}}%
\pgfpathlineto{\pgfqpoint{3.549440in}{1.119550in}}%
\pgfpathlineto{\pgfqpoint{3.550680in}{1.119182in}}%
\pgfpathlineto{\pgfqpoint{3.553160in}{1.121357in}}%
\pgfpathlineto{\pgfqpoint{3.556880in}{1.121869in}}%
\pgfpathlineto{\pgfqpoint{3.558120in}{1.121962in}}%
\pgfpathlineto{\pgfqpoint{3.561840in}{1.117722in}}%
\pgfpathlineto{\pgfqpoint{3.566800in}{1.112749in}}%
\pgfpathlineto{\pgfqpoint{3.569280in}{1.113157in}}%
\pgfpathlineto{\pgfqpoint{3.571760in}{1.111623in}}%
\pgfpathlineto{\pgfqpoint{3.574240in}{1.113329in}}%
\pgfpathlineto{\pgfqpoint{3.575480in}{1.114966in}}%
\pgfpathlineto{\pgfqpoint{3.576720in}{1.114719in}}%
\pgfpathlineto{\pgfqpoint{3.579200in}{1.116406in}}%
\pgfpathlineto{\pgfqpoint{3.581680in}{1.117043in}}%
\pgfpathlineto{\pgfqpoint{3.582920in}{1.115892in}}%
\pgfpathlineto{\pgfqpoint{3.585400in}{1.117322in}}%
\pgfpathlineto{\pgfqpoint{3.587880in}{1.114562in}}%
\pgfpathlineto{\pgfqpoint{3.589120in}{1.114823in}}%
\pgfpathlineto{\pgfqpoint{3.591600in}{1.118752in}}%
\pgfpathlineto{\pgfqpoint{3.601520in}{1.120172in}}%
\pgfpathlineto{\pgfqpoint{3.602760in}{1.120907in}}%
\pgfpathlineto{\pgfqpoint{3.604000in}{1.119310in}}%
\pgfpathlineto{\pgfqpoint{3.608960in}{1.121969in}}%
\pgfpathlineto{\pgfqpoint{3.611440in}{1.123185in}}%
\pgfpathlineto{\pgfqpoint{3.613920in}{1.122387in}}%
\pgfpathlineto{\pgfqpoint{3.616400in}{1.122117in}}%
\pgfpathlineto{\pgfqpoint{3.620120in}{1.123491in}}%
\pgfpathlineto{\pgfqpoint{3.626320in}{1.117404in}}%
\pgfpathlineto{\pgfqpoint{3.632520in}{1.118305in}}%
\pgfpathlineto{\pgfqpoint{3.635000in}{1.116218in}}%
\pgfpathlineto{\pgfqpoint{3.637480in}{1.115312in}}%
\pgfpathlineto{\pgfqpoint{3.641200in}{1.116113in}}%
\pgfpathlineto{\pgfqpoint{3.644920in}{1.116328in}}%
\pgfpathlineto{\pgfqpoint{3.649880in}{1.112607in}}%
\pgfpathlineto{\pgfqpoint{3.652360in}{1.113963in}}%
\pgfpathlineto{\pgfqpoint{3.654840in}{1.110786in}}%
\pgfpathlineto{\pgfqpoint{3.657320in}{1.111585in}}%
\pgfpathlineto{\pgfqpoint{3.659800in}{1.109535in}}%
\pgfpathlineto{\pgfqpoint{3.661040in}{1.110488in}}%
\pgfpathlineto{\pgfqpoint{3.663520in}{1.109477in}}%
\pgfpathlineto{\pgfqpoint{3.664760in}{1.109749in}}%
\pgfpathlineto{\pgfqpoint{3.666000in}{1.108670in}}%
\pgfpathlineto{\pgfqpoint{3.668480in}{1.111516in}}%
\pgfpathlineto{\pgfqpoint{3.670960in}{1.111695in}}%
\pgfpathlineto{\pgfqpoint{3.673440in}{1.109425in}}%
\pgfpathlineto{\pgfqpoint{3.674680in}{1.109317in}}%
\pgfpathlineto{\pgfqpoint{3.677160in}{1.111919in}}%
\pgfpathlineto{\pgfqpoint{3.682120in}{1.113767in}}%
\pgfpathlineto{\pgfqpoint{3.685840in}{1.110362in}}%
\pgfpathlineto{\pgfqpoint{3.690800in}{1.105563in}}%
\pgfpathlineto{\pgfqpoint{3.693280in}{1.106061in}}%
\pgfpathlineto{\pgfqpoint{3.695760in}{1.104289in}}%
\pgfpathlineto{\pgfqpoint{3.698240in}{1.106270in}}%
\pgfpathlineto{\pgfqpoint{3.700720in}{1.107697in}}%
\pgfpathlineto{\pgfqpoint{3.704440in}{1.108498in}}%
\pgfpathlineto{\pgfqpoint{3.710640in}{1.107083in}}%
\pgfpathlineto{\pgfqpoint{3.713120in}{1.106618in}}%
\pgfpathlineto{\pgfqpoint{3.716840in}{1.110426in}}%
\pgfpathlineto{\pgfqpoint{3.721800in}{1.111687in}}%
\pgfpathlineto{\pgfqpoint{3.726760in}{1.112583in}}%
\pgfpathlineto{\pgfqpoint{3.728000in}{1.111083in}}%
\pgfpathlineto{\pgfqpoint{3.729240in}{1.112423in}}%
\pgfpathlineto{\pgfqpoint{3.731720in}{1.112831in}}%
\pgfpathlineto{\pgfqpoint{3.735440in}{1.115058in}}%
\pgfpathlineto{\pgfqpoint{3.736680in}{1.115675in}}%
\pgfpathlineto{\pgfqpoint{3.740400in}{1.113904in}}%
\pgfpathlineto{\pgfqpoint{3.745360in}{1.113827in}}%
\pgfpathlineto{\pgfqpoint{3.750320in}{1.108945in}}%
\pgfpathlineto{\pgfqpoint{3.755280in}{1.110545in}}%
\pgfpathlineto{\pgfqpoint{3.756520in}{1.110752in}}%
\pgfpathlineto{\pgfqpoint{3.760240in}{1.108286in}}%
\pgfpathlineto{\pgfqpoint{3.768920in}{1.109864in}}%
\pgfpathlineto{\pgfqpoint{3.773880in}{1.106095in}}%
\pgfpathlineto{\pgfqpoint{3.776360in}{1.107460in}}%
\pgfpathlineto{\pgfqpoint{3.778840in}{1.103947in}}%
\pgfpathlineto{\pgfqpoint{3.781320in}{1.104118in}}%
\pgfpathlineto{\pgfqpoint{3.783800in}{1.102197in}}%
\pgfpathlineto{\pgfqpoint{3.785040in}{1.103510in}}%
\pgfpathlineto{\pgfqpoint{3.787520in}{1.102570in}}%
\pgfpathlineto{\pgfqpoint{3.788760in}{1.102731in}}%
\pgfpathlineto{\pgfqpoint{3.790000in}{1.101402in}}%
\pgfpathlineto{\pgfqpoint{3.792480in}{1.104046in}}%
\pgfpathlineto{\pgfqpoint{3.794960in}{1.103462in}}%
\pgfpathlineto{\pgfqpoint{3.798680in}{1.100965in}}%
\pgfpathlineto{\pgfqpoint{3.802400in}{1.104248in}}%
\pgfpathlineto{\pgfqpoint{3.804880in}{1.105305in}}%
\pgfpathlineto{\pgfqpoint{3.807360in}{1.104739in}}%
\pgfpathlineto{\pgfqpoint{3.816040in}{1.097632in}}%
\pgfpathlineto{\pgfqpoint{3.818520in}{1.096035in}}%
\pgfpathlineto{\pgfqpoint{3.819760in}{1.094848in}}%
\pgfpathlineto{\pgfqpoint{3.822240in}{1.096782in}}%
\pgfpathlineto{\pgfqpoint{3.824720in}{1.098158in}}%
\pgfpathlineto{\pgfqpoint{3.828440in}{1.099252in}}%
\pgfpathlineto{\pgfqpoint{3.832160in}{1.098965in}}%
\pgfpathlineto{\pgfqpoint{3.833400in}{1.099611in}}%
\pgfpathlineto{\pgfqpoint{3.835880in}{1.096739in}}%
\pgfpathlineto{\pgfqpoint{3.837120in}{1.097284in}}%
\pgfpathlineto{\pgfqpoint{3.840840in}{1.101699in}}%
\pgfpathlineto{\pgfqpoint{3.844560in}{1.102325in}}%
\pgfpathlineto{\pgfqpoint{3.849520in}{1.104375in}}%
\pgfpathlineto{\pgfqpoint{3.850760in}{1.104493in}}%
\pgfpathlineto{\pgfqpoint{3.852000in}{1.103022in}}%
\pgfpathlineto{\pgfqpoint{3.853240in}{1.104860in}}%
\pgfpathlineto{\pgfqpoint{3.854480in}{1.104286in}}%
\pgfpathlineto{\pgfqpoint{3.860680in}{1.108342in}}%
\pgfpathlineto{\pgfqpoint{3.864400in}{1.107188in}}%
\pgfpathlineto{\pgfqpoint{3.869360in}{1.106861in}}%
\pgfpathlineto{\pgfqpoint{3.873080in}{1.101983in}}%
\pgfpathlineto{\pgfqpoint{3.875560in}{1.101878in}}%
\pgfpathlineto{\pgfqpoint{3.878040in}{1.102853in}}%
\pgfpathlineto{\pgfqpoint{3.886720in}{1.100764in}}%
\pgfpathlineto{\pgfqpoint{3.892920in}{1.102446in}}%
\pgfpathlineto{\pgfqpoint{3.897880in}{1.099365in}}%
\pgfpathlineto{\pgfqpoint{3.900360in}{1.100247in}}%
\pgfpathlineto{\pgfqpoint{3.902840in}{1.096211in}}%
\pgfpathlineto{\pgfqpoint{3.905320in}{1.096680in}}%
\pgfpathlineto{\pgfqpoint{3.907800in}{1.094489in}}%
\pgfpathlineto{\pgfqpoint{3.909040in}{1.095593in}}%
\pgfpathlineto{\pgfqpoint{3.911520in}{1.094643in}}%
\pgfpathlineto{\pgfqpoint{3.912760in}{1.094581in}}%
\pgfpathlineto{\pgfqpoint{3.914000in}{1.092900in}}%
\pgfpathlineto{\pgfqpoint{3.916480in}{1.095340in}}%
\pgfpathlineto{\pgfqpoint{3.918960in}{1.094449in}}%
\pgfpathlineto{\pgfqpoint{3.922680in}{1.092197in}}%
\pgfpathlineto{\pgfqpoint{3.925160in}{1.095015in}}%
\pgfpathlineto{\pgfqpoint{3.928880in}{1.096869in}}%
\pgfpathlineto{\pgfqpoint{3.931360in}{1.096172in}}%
\pgfpathlineto{\pgfqpoint{3.940040in}{1.090076in}}%
\pgfpathlineto{\pgfqpoint{3.941280in}{1.090242in}}%
\pgfpathlineto{\pgfqpoint{3.943760in}{1.087955in}}%
\pgfpathlineto{\pgfqpoint{3.945000in}{1.088577in}}%
\pgfpathlineto{\pgfqpoint{3.947480in}{1.092188in}}%
\pgfpathlineto{\pgfqpoint{3.951200in}{1.092654in}}%
\pgfpathlineto{\pgfqpoint{3.953680in}{1.092761in}}%
\pgfpathlineto{\pgfqpoint{3.954920in}{1.091645in}}%
\pgfpathlineto{\pgfqpoint{3.957400in}{1.092993in}}%
\pgfpathlineto{\pgfqpoint{3.959880in}{1.090508in}}%
\pgfpathlineto{\pgfqpoint{3.961120in}{1.091189in}}%
\pgfpathlineto{\pgfqpoint{3.964840in}{1.095844in}}%
\pgfpathlineto{\pgfqpoint{3.968560in}{1.096216in}}%
\pgfpathlineto{\pgfqpoint{3.974760in}{1.099118in}}%
\pgfpathlineto{\pgfqpoint{3.976000in}{1.097307in}}%
\pgfpathlineto{\pgfqpoint{3.977240in}{1.098204in}}%
\pgfpathlineto{\pgfqpoint{3.978480in}{1.097502in}}%
\pgfpathlineto{\pgfqpoint{3.984680in}{1.102046in}}%
\pgfpathlineto{\pgfqpoint{3.988400in}{1.100601in}}%
\pgfpathlineto{\pgfqpoint{3.990880in}{1.102112in}}%
\pgfpathlineto{\pgfqpoint{3.992120in}{1.102700in}}%
\pgfpathlineto{\pgfqpoint{3.998320in}{1.095625in}}%
\pgfpathlineto{\pgfqpoint{4.007000in}{1.095645in}}%
\pgfpathlineto{\pgfqpoint{4.009480in}{1.094136in}}%
\pgfpathlineto{\pgfqpoint{4.016920in}{1.095714in}}%
\pgfpathlineto{\pgfqpoint{4.020640in}{1.093023in}}%
\pgfpathlineto{\pgfqpoint{4.024360in}{1.095226in}}%
\pgfpathlineto{\pgfqpoint{4.026840in}{1.092167in}}%
\pgfpathlineto{\pgfqpoint{4.029320in}{1.091737in}}%
\pgfpathlineto{\pgfqpoint{4.031800in}{1.089751in}}%
\pgfpathlineto{\pgfqpoint{4.033040in}{1.090355in}}%
\pgfpathlineto{\pgfqpoint{4.035520in}{1.089020in}}%
\pgfpathlineto{\pgfqpoint{4.036760in}{1.088832in}}%
\pgfpathlineto{\pgfqpoint{4.038000in}{1.087207in}}%
\pgfpathlineto{\pgfqpoint{4.040480in}{1.089168in}}%
\pgfpathlineto{\pgfqpoint{4.044200in}{1.088207in}}%
\pgfpathlineto{\pgfqpoint{4.046680in}{1.086079in}}%
\pgfpathlineto{\pgfqpoint{4.049160in}{1.088465in}}%
\pgfpathlineto{\pgfqpoint{4.051640in}{1.089153in}}%
\pgfpathlineto{\pgfqpoint{4.054120in}{1.089390in}}%
\pgfpathlineto{\pgfqpoint{4.057840in}{1.085726in}}%
\pgfpathlineto{\pgfqpoint{4.059080in}{1.085523in}}%
\pgfpathlineto{\pgfqpoint{4.061560in}{1.082971in}}%
\pgfpathlineto{\pgfqpoint{4.064040in}{1.082816in}}%
\pgfpathlineto{\pgfqpoint{4.065280in}{1.082909in}}%
\pgfpathlineto{\pgfqpoint{4.067760in}{1.080548in}}%
\pgfpathlineto{\pgfqpoint{4.069000in}{1.081303in}}%
\pgfpathlineto{\pgfqpoint{4.071480in}{1.085707in}}%
\pgfpathlineto{\pgfqpoint{4.080160in}{1.085371in}}%
\pgfpathlineto{\pgfqpoint{4.081400in}{1.085749in}}%
\pgfpathlineto{\pgfqpoint{4.083880in}{1.083448in}}%
\pgfpathlineto{\pgfqpoint{4.085120in}{1.083766in}}%
\pgfpathlineto{\pgfqpoint{4.088840in}{1.088224in}}%
\pgfpathlineto{\pgfqpoint{4.090080in}{1.088083in}}%
\pgfpathlineto{\pgfqpoint{4.092560in}{1.089115in}}%
\pgfpathlineto{\pgfqpoint{4.098760in}{1.092690in}}%
\pgfpathlineto{\pgfqpoint{4.100000in}{1.090884in}}%
\pgfpathlineto{\pgfqpoint{4.101240in}{1.091420in}}%
\pgfpathlineto{\pgfqpoint{4.103720in}{1.090664in}}%
\pgfpathlineto{\pgfqpoint{4.107440in}{1.092995in}}%
\pgfpathlineto{\pgfqpoint{4.108680in}{1.093444in}}%
\pgfpathlineto{\pgfqpoint{4.111160in}{1.090973in}}%
\pgfpathlineto{\pgfqpoint{4.113640in}{1.092113in}}%
\pgfpathlineto{\pgfqpoint{4.116120in}{1.094029in}}%
\pgfpathlineto{\pgfqpoint{4.121080in}{1.087551in}}%
\pgfpathlineto{\pgfqpoint{4.123560in}{1.086905in}}%
\pgfpathlineto{\pgfqpoint{4.126040in}{1.088167in}}%
\pgfpathlineto{\pgfqpoint{4.135960in}{1.086883in}}%
\pgfpathlineto{\pgfqpoint{4.139680in}{1.088292in}}%
\pgfpathlineto{\pgfqpoint{4.144640in}{1.083743in}}%
\pgfpathlineto{\pgfqpoint{4.148360in}{1.087236in}}%
\pgfpathlineto{\pgfqpoint{4.150840in}{1.084010in}}%
\pgfpathlineto{\pgfqpoint{4.153320in}{1.084082in}}%
\pgfpathlineto{\pgfqpoint{4.155800in}{1.082086in}}%
\pgfpathlineto{\pgfqpoint{4.157040in}{1.082641in}}%
\pgfpathlineto{\pgfqpoint{4.159520in}{1.081211in}}%
\pgfpathlineto{\pgfqpoint{4.160760in}{1.081071in}}%
\pgfpathlineto{\pgfqpoint{4.162000in}{1.079733in}}%
\pgfpathlineto{\pgfqpoint{4.164480in}{1.082250in}}%
\pgfpathlineto{\pgfqpoint{4.166960in}{1.081617in}}%
\pgfpathlineto{\pgfqpoint{4.169440in}{1.080431in}}%
\pgfpathlineto{\pgfqpoint{4.170680in}{1.079170in}}%
\pgfpathlineto{\pgfqpoint{4.173160in}{1.081392in}}%
\pgfpathlineto{\pgfqpoint{4.178120in}{1.081948in}}%
\pgfpathlineto{\pgfqpoint{4.181840in}{1.078623in}}%
\pgfpathlineto{\pgfqpoint{4.183080in}{1.078188in}}%
\pgfpathlineto{\pgfqpoint{4.185560in}{1.075209in}}%
\pgfpathlineto{\pgfqpoint{4.188040in}{1.075193in}}%
\pgfpathlineto{\pgfqpoint{4.189280in}{1.075708in}}%
\pgfpathlineto{\pgfqpoint{4.191760in}{1.073447in}}%
\pgfpathlineto{\pgfqpoint{4.193000in}{1.074172in}}%
\pgfpathlineto{\pgfqpoint{4.195480in}{1.078835in}}%
\pgfpathlineto{\pgfqpoint{4.204160in}{1.078005in}}%
\pgfpathlineto{\pgfqpoint{4.205400in}{1.078625in}}%
\pgfpathlineto{\pgfqpoint{4.207880in}{1.076357in}}%
\pgfpathlineto{\pgfqpoint{4.209120in}{1.076407in}}%
\pgfpathlineto{\pgfqpoint{4.211600in}{1.079940in}}%
\pgfpathlineto{\pgfqpoint{4.214080in}{1.079129in}}%
\pgfpathlineto{\pgfqpoint{4.215320in}{1.080529in}}%
\pgfpathlineto{\pgfqpoint{4.216560in}{1.080015in}}%
\pgfpathlineto{\pgfqpoint{4.221520in}{1.083646in}}%
\pgfpathlineto{\pgfqpoint{4.222760in}{1.084100in}}%
\pgfpathlineto{\pgfqpoint{4.225240in}{1.082277in}}%
\pgfpathlineto{\pgfqpoint{4.226480in}{1.080818in}}%
\pgfpathlineto{\pgfqpoint{4.233920in}{1.082161in}}%
\pgfpathlineto{\pgfqpoint{4.235160in}{1.081820in}}%
\pgfpathlineto{\pgfqpoint{4.240120in}{1.085573in}}%
\pgfpathlineto{\pgfqpoint{4.245080in}{1.079495in}}%
\pgfpathlineto{\pgfqpoint{4.247560in}{1.079208in}}%
\pgfpathlineto{\pgfqpoint{4.250040in}{1.080947in}}%
\pgfpathlineto{\pgfqpoint{4.258720in}{1.078502in}}%
\pgfpathlineto{\pgfqpoint{4.262440in}{1.079057in}}%
\pgfpathlineto{\pgfqpoint{4.264920in}{1.078667in}}%
\pgfpathlineto{\pgfqpoint{4.266160in}{1.078055in}}%
\pgfpathlineto{\pgfqpoint{4.268640in}{1.075586in}}%
\pgfpathlineto{\pgfqpoint{4.272360in}{1.079716in}}%
\pgfpathlineto{\pgfqpoint{4.274840in}{1.076268in}}%
\pgfpathlineto{\pgfqpoint{4.276080in}{1.076960in}}%
\pgfpathlineto{\pgfqpoint{4.282280in}{1.074975in}}%
\pgfpathlineto{\pgfqpoint{4.286000in}{1.073293in}}%
\pgfpathlineto{\pgfqpoint{4.287240in}{1.075489in}}%
\pgfpathlineto{\pgfqpoint{4.293440in}{1.074264in}}%
\pgfpathlineto{\pgfqpoint{4.294680in}{1.073059in}}%
\pgfpathlineto{\pgfqpoint{4.297160in}{1.075062in}}%
\pgfpathlineto{\pgfqpoint{4.300880in}{1.076055in}}%
\pgfpathlineto{\pgfqpoint{4.302120in}{1.075920in}}%
\pgfpathlineto{\pgfqpoint{4.308320in}{1.068629in}}%
\pgfpathlineto{\pgfqpoint{4.310800in}{1.067401in}}%
\pgfpathlineto{\pgfqpoint{4.313280in}{1.068140in}}%
\pgfpathlineto{\pgfqpoint{4.317000in}{1.066847in}}%
\pgfpathlineto{\pgfqpoint{4.320720in}{1.070588in}}%
\pgfpathlineto{\pgfqpoint{4.326920in}{1.068735in}}%
\pgfpathlineto{\pgfqpoint{4.329400in}{1.070347in}}%
\pgfpathlineto{\pgfqpoint{4.331880in}{1.068305in}}%
\pgfpathlineto{\pgfqpoint{4.333120in}{1.068502in}}%
\pgfpathlineto{\pgfqpoint{4.335600in}{1.071701in}}%
\pgfpathlineto{\pgfqpoint{4.338080in}{1.070642in}}%
\pgfpathlineto{\pgfqpoint{4.339320in}{1.072469in}}%
\pgfpathlineto{\pgfqpoint{4.340560in}{1.071895in}}%
\pgfpathlineto{\pgfqpoint{4.345520in}{1.075336in}}%
\pgfpathlineto{\pgfqpoint{4.346760in}{1.075751in}}%
\pgfpathlineto{\pgfqpoint{4.348000in}{1.073580in}}%
\pgfpathlineto{\pgfqpoint{4.349240in}{1.075598in}}%
\pgfpathlineto{\pgfqpoint{4.350480in}{1.073521in}}%
\pgfpathlineto{\pgfqpoint{4.351720in}{1.073830in}}%
\pgfpathlineto{\pgfqpoint{4.354200in}{1.076127in}}%
\pgfpathlineto{\pgfqpoint{4.361640in}{1.076907in}}%
\pgfpathlineto{\pgfqpoint{4.364120in}{1.079054in}}%
\pgfpathlineto{\pgfqpoint{4.369080in}{1.072470in}}%
\pgfpathlineto{\pgfqpoint{4.371560in}{1.073069in}}%
\pgfpathlineto{\pgfqpoint{4.374040in}{1.074372in}}%
\pgfpathlineto{\pgfqpoint{4.376520in}{1.074349in}}%
\pgfpathlineto{\pgfqpoint{4.379000in}{1.072952in}}%
\pgfpathlineto{\pgfqpoint{4.381480in}{1.071127in}}%
\pgfpathlineto{\pgfqpoint{4.383960in}{1.071894in}}%
\pgfpathlineto{\pgfqpoint{4.386440in}{1.070443in}}%
\pgfpathlineto{\pgfqpoint{4.388920in}{1.070081in}}%
\pgfpathlineto{\pgfqpoint{4.390160in}{1.069669in}}%
\pgfpathlineto{\pgfqpoint{4.391400in}{1.067886in}}%
\pgfpathlineto{\pgfqpoint{4.392640in}{1.068343in}}%
\pgfpathlineto{\pgfqpoint{4.396360in}{1.072756in}}%
\pgfpathlineto{\pgfqpoint{4.398840in}{1.069432in}}%
\pgfpathlineto{\pgfqpoint{4.401320in}{1.069908in}}%
\pgfpathlineto{\pgfqpoint{4.403800in}{1.067514in}}%
\pgfpathlineto{\pgfqpoint{4.405040in}{1.069290in}}%
\pgfpathlineto{\pgfqpoint{4.410000in}{1.065844in}}%
\pgfpathlineto{\pgfqpoint{4.411240in}{1.068002in}}%
\pgfpathlineto{\pgfqpoint{4.414960in}{1.067278in}}%
\pgfpathlineto{\pgfqpoint{4.417440in}{1.066654in}}%
\pgfpathlineto{\pgfqpoint{4.418680in}{1.065815in}}%
\pgfpathlineto{\pgfqpoint{4.421160in}{1.068063in}}%
\pgfpathlineto{\pgfqpoint{4.426120in}{1.068154in}}%
\pgfpathlineto{\pgfqpoint{4.433560in}{1.060169in}}%
\pgfpathlineto{\pgfqpoint{4.436040in}{1.060576in}}%
\pgfpathlineto{\pgfqpoint{4.437280in}{1.060940in}}%
\pgfpathlineto{\pgfqpoint{4.439760in}{1.059772in}}%
\pgfpathlineto{\pgfqpoint{4.441000in}{1.060431in}}%
\pgfpathlineto{\pgfqpoint{4.444720in}{1.064643in}}%
\pgfpathlineto{\pgfqpoint{4.452160in}{1.064112in}}%
\pgfpathlineto{\pgfqpoint{4.453400in}{1.065166in}}%
\pgfpathlineto{\pgfqpoint{4.455880in}{1.062623in}}%
\pgfpathlineto{\pgfqpoint{4.457120in}{1.062866in}}%
\pgfpathlineto{\pgfqpoint{4.459600in}{1.065805in}}%
\pgfpathlineto{\pgfqpoint{4.462080in}{1.065154in}}%
\pgfpathlineto{\pgfqpoint{4.463320in}{1.067271in}}%
\pgfpathlineto{\pgfqpoint{4.464560in}{1.066411in}}%
\pgfpathlineto{\pgfqpoint{4.469520in}{1.068971in}}%
\pgfpathlineto{\pgfqpoint{4.470760in}{1.069186in}}%
\pgfpathlineto{\pgfqpoint{4.472000in}{1.067422in}}%
\pgfpathlineto{\pgfqpoint{4.473240in}{1.068259in}}%
\pgfpathlineto{\pgfqpoint{4.475720in}{1.065992in}}%
\pgfpathlineto{\pgfqpoint{4.478200in}{1.067550in}}%
\pgfpathlineto{\pgfqpoint{4.485640in}{1.068536in}}%
\pgfpathlineto{\pgfqpoint{4.488120in}{1.070044in}}%
\pgfpathlineto{\pgfqpoint{4.493080in}{1.064224in}}%
\pgfpathlineto{\pgfqpoint{4.495560in}{1.065059in}}%
\pgfpathlineto{\pgfqpoint{4.499280in}{1.066977in}}%
\pgfpathlineto{\pgfqpoint{4.500520in}{1.066960in}}%
\pgfpathlineto{\pgfqpoint{4.501760in}{1.065360in}}%
\pgfpathlineto{\pgfqpoint{4.503000in}{1.065901in}}%
\pgfpathlineto{\pgfqpoint{4.505480in}{1.063863in}}%
\pgfpathlineto{\pgfqpoint{4.507960in}{1.065057in}}%
\pgfpathlineto{\pgfqpoint{4.510440in}{1.063490in}}%
\pgfpathlineto{\pgfqpoint{4.514160in}{1.062779in}}%
\pgfpathlineto{\pgfqpoint{4.515400in}{1.061090in}}%
\pgfpathlineto{\pgfqpoint{4.516640in}{1.061508in}}%
\pgfpathlineto{\pgfqpoint{4.519120in}{1.064589in}}%
\pgfpathlineto{\pgfqpoint{4.520360in}{1.065456in}}%
\pgfpathlineto{\pgfqpoint{4.522840in}{1.062110in}}%
\pgfpathlineto{\pgfqpoint{4.525320in}{1.062397in}}%
\pgfpathlineto{\pgfqpoint{4.527800in}{1.059545in}}%
\pgfpathlineto{\pgfqpoint{4.529040in}{1.061340in}}%
\pgfpathlineto{\pgfqpoint{4.534000in}{1.058032in}}%
\pgfpathlineto{\pgfqpoint{4.535240in}{1.060634in}}%
\pgfpathlineto{\pgfqpoint{4.538960in}{1.060134in}}%
\pgfpathlineto{\pgfqpoint{4.541440in}{1.059885in}}%
\pgfpathlineto{\pgfqpoint{4.542680in}{1.059040in}}%
\pgfpathlineto{\pgfqpoint{4.545160in}{1.061080in}}%
\pgfpathlineto{\pgfqpoint{4.547640in}{1.061393in}}%
\pgfpathlineto{\pgfqpoint{4.552600in}{1.058608in}}%
\pgfpathlineto{\pgfqpoint{4.557560in}{1.054146in}}%
\pgfpathlineto{\pgfqpoint{4.565000in}{1.053963in}}%
\pgfpathlineto{\pgfqpoint{4.568720in}{1.057548in}}%
\pgfpathlineto{\pgfqpoint{4.578640in}{1.056095in}}%
\pgfpathlineto{\pgfqpoint{4.581120in}{1.054829in}}%
\pgfpathlineto{\pgfqpoint{4.583600in}{1.057404in}}%
\pgfpathlineto{\pgfqpoint{4.586080in}{1.056739in}}%
\pgfpathlineto{\pgfqpoint{4.587320in}{1.058963in}}%
\pgfpathlineto{\pgfqpoint{4.588560in}{1.058004in}}%
\pgfpathlineto{\pgfqpoint{4.594760in}{1.060998in}}%
\pgfpathlineto{\pgfqpoint{4.596000in}{1.059157in}}%
\pgfpathlineto{\pgfqpoint{4.597240in}{1.060158in}}%
\pgfpathlineto{\pgfqpoint{4.599720in}{1.058296in}}%
\pgfpathlineto{\pgfqpoint{4.602200in}{1.059809in}}%
\pgfpathlineto{\pgfqpoint{4.607160in}{1.059021in}}%
\pgfpathlineto{\pgfqpoint{4.610880in}{1.061688in}}%
\pgfpathlineto{\pgfqpoint{4.612120in}{1.062409in}}%
\pgfpathlineto{\pgfqpoint{4.618320in}{1.055766in}}%
\pgfpathlineto{\pgfqpoint{4.623280in}{1.058771in}}%
\pgfpathlineto{\pgfqpoint{4.624520in}{1.058633in}}%
\pgfpathlineto{\pgfqpoint{4.625760in}{1.056528in}}%
\pgfpathlineto{\pgfqpoint{4.627000in}{1.057063in}}%
\pgfpathlineto{\pgfqpoint{4.629480in}{1.055407in}}%
\pgfpathlineto{\pgfqpoint{4.631960in}{1.056773in}}%
\pgfpathlineto{\pgfqpoint{4.635680in}{1.054946in}}%
\pgfpathlineto{\pgfqpoint{4.638160in}{1.054538in}}%
\pgfpathlineto{\pgfqpoint{4.639400in}{1.052948in}}%
\pgfpathlineto{\pgfqpoint{4.641880in}{1.055392in}}%
\pgfpathlineto{\pgfqpoint{4.644360in}{1.056929in}}%
\pgfpathlineto{\pgfqpoint{4.646840in}{1.053570in}}%
\pgfpathlineto{\pgfqpoint{4.649320in}{1.054281in}}%
\pgfpathlineto{\pgfqpoint{4.651800in}{1.051931in}}%
\pgfpathlineto{\pgfqpoint{4.653040in}{1.053618in}}%
\pgfpathlineto{\pgfqpoint{4.655520in}{1.051771in}}%
\pgfpathlineto{\pgfqpoint{4.656760in}{1.051643in}}%
\pgfpathlineto{\pgfqpoint{4.658000in}{1.050255in}}%
\pgfpathlineto{\pgfqpoint{4.659240in}{1.052857in}}%
\pgfpathlineto{\pgfqpoint{4.667920in}{1.053302in}}%
\pgfpathlineto{\pgfqpoint{4.671640in}{1.053463in}}%
\pgfpathlineto{\pgfqpoint{4.675360in}{1.051267in}}%
\pgfpathlineto{\pgfqpoint{4.681560in}{1.045131in}}%
\pgfpathlineto{\pgfqpoint{4.685280in}{1.045416in}}%
\pgfpathlineto{\pgfqpoint{4.687760in}{1.044742in}}%
\pgfpathlineto{\pgfqpoint{4.690240in}{1.047843in}}%
\pgfpathlineto{\pgfqpoint{4.691480in}{1.049757in}}%
\pgfpathlineto{\pgfqpoint{4.693960in}{1.049170in}}%
\pgfpathlineto{\pgfqpoint{4.695200in}{1.049098in}}%
\pgfpathlineto{\pgfqpoint{4.697680in}{1.050935in}}%
\pgfpathlineto{\pgfqpoint{4.698920in}{1.049381in}}%
\pgfpathlineto{\pgfqpoint{4.701400in}{1.050999in}}%
\pgfpathlineto{\pgfqpoint{4.703880in}{1.048360in}}%
\pgfpathlineto{\pgfqpoint{4.705120in}{1.048344in}}%
\pgfpathlineto{\pgfqpoint{4.708840in}{1.050601in}}%
\pgfpathlineto{\pgfqpoint{4.710080in}{1.049961in}}%
\pgfpathlineto{\pgfqpoint{4.711320in}{1.051953in}}%
\pgfpathlineto{\pgfqpoint{4.712560in}{1.051114in}}%
\pgfpathlineto{\pgfqpoint{4.718760in}{1.054503in}}%
\pgfpathlineto{\pgfqpoint{4.722480in}{1.050270in}}%
\pgfpathlineto{\pgfqpoint{4.723720in}{1.050162in}}%
\pgfpathlineto{\pgfqpoint{4.726200in}{1.051766in}}%
\pgfpathlineto{\pgfqpoint{4.732400in}{1.052599in}}%
\pgfpathlineto{\pgfqpoint{4.736120in}{1.055854in}}%
\pgfpathlineto{\pgfqpoint{4.742320in}{1.049556in}}%
\pgfpathlineto{\pgfqpoint{4.748520in}{1.051642in}}%
\pgfpathlineto{\pgfqpoint{4.749760in}{1.049671in}}%
\pgfpathlineto{\pgfqpoint{4.751000in}{1.050011in}}%
\pgfpathlineto{\pgfqpoint{4.753480in}{1.048455in}}%
\pgfpathlineto{\pgfqpoint{4.755960in}{1.050399in}}%
\pgfpathlineto{\pgfqpoint{4.759680in}{1.048068in}}%
\pgfpathlineto{\pgfqpoint{4.762160in}{1.047320in}}%
\pgfpathlineto{\pgfqpoint{4.763400in}{1.045311in}}%
\pgfpathlineto{\pgfqpoint{4.764640in}{1.045616in}}%
\pgfpathlineto{\pgfqpoint{4.767120in}{1.048150in}}%
\pgfpathlineto{\pgfqpoint{4.768360in}{1.048632in}}%
\pgfpathlineto{\pgfqpoint{4.770840in}{1.045936in}}%
\pgfpathlineto{\pgfqpoint{4.773320in}{1.047905in}}%
\pgfpathlineto{\pgfqpoint{4.775800in}{1.045782in}}%
\pgfpathlineto{\pgfqpoint{4.777040in}{1.047863in}}%
\pgfpathlineto{\pgfqpoint{4.782000in}{1.044163in}}%
\pgfpathlineto{\pgfqpoint{4.783240in}{1.046378in}}%
\pgfpathlineto{\pgfqpoint{4.790680in}{1.045629in}}%
\pgfpathlineto{\pgfqpoint{4.794400in}{1.048517in}}%
\pgfpathlineto{\pgfqpoint{4.798120in}{1.047116in}}%
\pgfpathlineto{\pgfqpoint{4.808040in}{1.038122in}}%
\pgfpathlineto{\pgfqpoint{4.813000in}{1.038616in}}%
\pgfpathlineto{\pgfqpoint{4.816720in}{1.041881in}}%
\pgfpathlineto{\pgfqpoint{4.825400in}{1.042947in}}%
\pgfpathlineto{\pgfqpoint{4.829120in}{1.039801in}}%
\pgfpathlineto{\pgfqpoint{4.832840in}{1.042175in}}%
\pgfpathlineto{\pgfqpoint{4.834080in}{1.041314in}}%
\pgfpathlineto{\pgfqpoint{4.835320in}{1.043227in}}%
\pgfpathlineto{\pgfqpoint{4.836560in}{1.042412in}}%
\pgfpathlineto{\pgfqpoint{4.842760in}{1.046079in}}%
\pgfpathlineto{\pgfqpoint{4.844000in}{1.043877in}}%
\pgfpathlineto{\pgfqpoint{4.845240in}{1.043946in}}%
\pgfpathlineto{\pgfqpoint{4.846480in}{1.041798in}}%
\pgfpathlineto{\pgfqpoint{4.855160in}{1.042470in}}%
\pgfpathlineto{\pgfqpoint{4.858880in}{1.046758in}}%
\pgfpathlineto{\pgfqpoint{4.860120in}{1.047609in}}%
\pgfpathlineto{\pgfqpoint{4.866320in}{1.042131in}}%
\pgfpathlineto{\pgfqpoint{4.868800in}{1.043835in}}%
\pgfpathlineto{\pgfqpoint{4.871280in}{1.044369in}}%
\pgfpathlineto{\pgfqpoint{4.872520in}{1.043849in}}%
\pgfpathlineto{\pgfqpoint{4.873760in}{1.041972in}}%
\pgfpathlineto{\pgfqpoint{4.876240in}{1.041619in}}%
\pgfpathlineto{\pgfqpoint{4.877480in}{1.041118in}}%
\pgfpathlineto{\pgfqpoint{4.879960in}{1.043312in}}%
\pgfpathlineto{\pgfqpoint{4.886160in}{1.039709in}}%
\pgfpathlineto{\pgfqpoint{4.887400in}{1.037503in}}%
\pgfpathlineto{\pgfqpoint{4.888640in}{1.037973in}}%
\pgfpathlineto{\pgfqpoint{4.892360in}{1.042113in}}%
\pgfpathlineto{\pgfqpoint{4.894840in}{1.039863in}}%
\pgfpathlineto{\pgfqpoint{4.897320in}{1.042136in}}%
\pgfpathlineto{\pgfqpoint{4.899800in}{1.039151in}}%
\pgfpathlineto{\pgfqpoint{4.901040in}{1.041168in}}%
\pgfpathlineto{\pgfqpoint{4.907240in}{1.040105in}}%
\pgfpathlineto{\pgfqpoint{4.913440in}{1.039180in}}%
\pgfpathlineto{\pgfqpoint{4.914680in}{1.038580in}}%
\pgfpathlineto{\pgfqpoint{4.918400in}{1.041865in}}%
\pgfpathlineto{\pgfqpoint{4.920880in}{1.041587in}}%
\pgfpathlineto{\pgfqpoint{4.922120in}{1.041184in}}%
\pgfpathlineto{\pgfqpoint{4.925840in}{1.036147in}}%
\pgfpathlineto{\pgfqpoint{4.932040in}{1.031440in}}%
\pgfpathlineto{\pgfqpoint{4.937000in}{1.032666in}}%
\pgfpathlineto{\pgfqpoint{4.940720in}{1.035398in}}%
\pgfpathlineto{\pgfqpoint{4.944440in}{1.034805in}}%
\pgfpathlineto{\pgfqpoint{4.945680in}{1.035306in}}%
\pgfpathlineto{\pgfqpoint{4.946920in}{1.034055in}}%
\pgfpathlineto{\pgfqpoint{4.949400in}{1.035732in}}%
\pgfpathlineto{\pgfqpoint{4.953120in}{1.032904in}}%
\pgfpathlineto{\pgfqpoint{4.955600in}{1.035462in}}%
\pgfpathlineto{\pgfqpoint{4.958080in}{1.032671in}}%
\pgfpathlineto{\pgfqpoint{4.959320in}{1.034471in}}%
\pgfpathlineto{\pgfqpoint{4.961800in}{1.034194in}}%
\pgfpathlineto{\pgfqpoint{4.964280in}{1.036621in}}%
\pgfpathlineto{\pgfqpoint{4.966760in}{1.037308in}}%
\pgfpathlineto{\pgfqpoint{4.970480in}{1.032026in}}%
\pgfpathlineto{\pgfqpoint{4.979160in}{1.032950in}}%
\pgfpathlineto{\pgfqpoint{4.982880in}{1.037006in}}%
\pgfpathlineto{\pgfqpoint{4.984120in}{1.038005in}}%
\pgfpathlineto{\pgfqpoint{4.990320in}{1.032430in}}%
\pgfpathlineto{\pgfqpoint{4.996520in}{1.035224in}}%
\pgfpathlineto{\pgfqpoint{4.997760in}{1.033683in}}%
\pgfpathlineto{\pgfqpoint{4.999000in}{1.034195in}}%
\pgfpathlineto{\pgfqpoint{5.001480in}{1.032720in}}%
\pgfpathlineto{\pgfqpoint{5.003960in}{1.035490in}}%
\pgfpathlineto{\pgfqpoint{5.010160in}{1.032239in}}%
\pgfpathlineto{\pgfqpoint{5.011400in}{1.029947in}}%
\pgfpathlineto{\pgfqpoint{5.016360in}{1.034284in}}%
\pgfpathlineto{\pgfqpoint{5.018840in}{1.031023in}}%
\pgfpathlineto{\pgfqpoint{5.021320in}{1.034042in}}%
\pgfpathlineto{\pgfqpoint{5.023800in}{1.031362in}}%
\pgfpathlineto{\pgfqpoint{5.025040in}{1.033561in}}%
\pgfpathlineto{\pgfqpoint{5.030000in}{1.032037in}}%
\pgfpathlineto{\pgfqpoint{5.032480in}{1.034082in}}%
\pgfpathlineto{\pgfqpoint{5.036200in}{1.033488in}}%
\pgfpathlineto{\pgfqpoint{5.038680in}{1.032480in}}%
\pgfpathlineto{\pgfqpoint{5.043640in}{1.036393in}}%
\pgfpathlineto{\pgfqpoint{5.046120in}{1.035895in}}%
\pgfpathlineto{\pgfqpoint{5.049840in}{1.030231in}}%
\pgfpathlineto{\pgfqpoint{5.053560in}{1.025732in}}%
\pgfpathlineto{\pgfqpoint{5.054800in}{1.026592in}}%
\pgfpathlineto{\pgfqpoint{5.056040in}{1.025136in}}%
\pgfpathlineto{\pgfqpoint{5.062240in}{1.027475in}}%
\pgfpathlineto{\pgfqpoint{5.064720in}{1.029279in}}%
\pgfpathlineto{\pgfqpoint{5.072160in}{1.028605in}}%
\pgfpathlineto{\pgfqpoint{5.073400in}{1.029049in}}%
\pgfpathlineto{\pgfqpoint{5.077120in}{1.026564in}}%
\pgfpathlineto{\pgfqpoint{5.079600in}{1.028883in}}%
\pgfpathlineto{\pgfqpoint{5.082080in}{1.026178in}}%
\pgfpathlineto{\pgfqpoint{5.083320in}{1.027315in}}%
\pgfpathlineto{\pgfqpoint{5.085800in}{1.026760in}}%
\pgfpathlineto{\pgfqpoint{5.089520in}{1.030007in}}%
\pgfpathlineto{\pgfqpoint{5.090760in}{1.029891in}}%
\pgfpathlineto{\pgfqpoint{5.094480in}{1.025720in}}%
\pgfpathlineto{\pgfqpoint{5.098200in}{1.026049in}}%
\pgfpathlineto{\pgfqpoint{5.103160in}{1.026426in}}%
\pgfpathlineto{\pgfqpoint{5.106880in}{1.031009in}}%
\pgfpathlineto{\pgfqpoint{5.108120in}{1.032009in}}%
\pgfpathlineto{\pgfqpoint{5.110600in}{1.030023in}}%
\pgfpathlineto{\pgfqpoint{5.113080in}{1.025762in}}%
\pgfpathlineto{\pgfqpoint{5.116800in}{1.029548in}}%
\pgfpathlineto{\pgfqpoint{5.120520in}{1.030662in}}%
\pgfpathlineto{\pgfqpoint{5.121760in}{1.029354in}}%
\pgfpathlineto{\pgfqpoint{5.123000in}{1.029675in}}%
\pgfpathlineto{\pgfqpoint{5.125480in}{1.027184in}}%
\pgfpathlineto{\pgfqpoint{5.127960in}{1.029060in}}%
\pgfpathlineto{\pgfqpoint{5.134160in}{1.027297in}}%
\pgfpathlineto{\pgfqpoint{5.135400in}{1.024971in}}%
\pgfpathlineto{\pgfqpoint{5.140360in}{1.029155in}}%
\pgfpathlineto{\pgfqpoint{5.142840in}{1.026045in}}%
\pgfpathlineto{\pgfqpoint{5.145320in}{1.028817in}}%
\pgfpathlineto{\pgfqpoint{5.147800in}{1.026277in}}%
\pgfpathlineto{\pgfqpoint{5.149040in}{1.028562in}}%
\pgfpathlineto{\pgfqpoint{5.151520in}{1.027745in}}%
\pgfpathlineto{\pgfqpoint{5.152760in}{1.027989in}}%
\pgfpathlineto{\pgfqpoint{5.154000in}{1.026387in}}%
\pgfpathlineto{\pgfqpoint{5.155240in}{1.029109in}}%
\pgfpathlineto{\pgfqpoint{5.158960in}{1.029766in}}%
\pgfpathlineto{\pgfqpoint{5.161440in}{1.029233in}}%
\pgfpathlineto{\pgfqpoint{5.162680in}{1.028865in}}%
\pgfpathlineto{\pgfqpoint{5.167640in}{1.031809in}}%
\pgfpathlineto{\pgfqpoint{5.170120in}{1.032235in}}%
\pgfpathlineto{\pgfqpoint{5.176320in}{1.021713in}}%
\pgfpathlineto{\pgfqpoint{5.177560in}{1.021208in}}%
\pgfpathlineto{\pgfqpoint{5.178800in}{1.021928in}}%
\pgfpathlineto{\pgfqpoint{5.180040in}{1.020446in}}%
\pgfpathlineto{\pgfqpoint{5.182520in}{1.020733in}}%
\pgfpathlineto{\pgfqpoint{5.186240in}{1.022164in}}%
\pgfpathlineto{\pgfqpoint{5.187480in}{1.024498in}}%
\pgfpathlineto{\pgfqpoint{5.192440in}{1.023181in}}%
\pgfpathlineto{\pgfqpoint{5.193680in}{1.023940in}}%
\pgfpathlineto{\pgfqpoint{5.194920in}{1.022977in}}%
\pgfpathlineto{\pgfqpoint{5.197400in}{1.023923in}}%
\pgfpathlineto{\pgfqpoint{5.199880in}{1.021845in}}%
\pgfpathlineto{\pgfqpoint{5.202360in}{1.023069in}}%
\pgfpathlineto{\pgfqpoint{5.203600in}{1.024308in}}%
\pgfpathlineto{\pgfqpoint{5.208560in}{1.021891in}}%
\pgfpathlineto{\pgfqpoint{5.211040in}{1.024295in}}%
\pgfpathlineto{\pgfqpoint{5.214760in}{1.026174in}}%
\pgfpathlineto{\pgfqpoint{5.217240in}{1.023837in}}%
\pgfpathlineto{\pgfqpoint{5.218480in}{1.022007in}}%
\pgfpathlineto{\pgfqpoint{5.222200in}{1.022898in}}%
\pgfpathlineto{\pgfqpoint{5.225920in}{1.022011in}}%
\pgfpathlineto{\pgfqpoint{5.227160in}{1.022824in}}%
\pgfpathlineto{\pgfqpoint{5.230880in}{1.027769in}}%
\pgfpathlineto{\pgfqpoint{5.232120in}{1.028535in}}%
\pgfpathlineto{\pgfqpoint{5.234600in}{1.026075in}}%
\pgfpathlineto{\pgfqpoint{5.237080in}{1.020879in}}%
\pgfpathlineto{\pgfqpoint{5.242040in}{1.026262in}}%
\pgfpathlineto{\pgfqpoint{5.243280in}{1.025582in}}%
\pgfpathlineto{\pgfqpoint{5.244520in}{1.026292in}}%
\pgfpathlineto{\pgfqpoint{5.245760in}{1.025415in}}%
\pgfpathlineto{\pgfqpoint{5.247000in}{1.025941in}}%
\pgfpathlineto{\pgfqpoint{5.249480in}{1.023439in}}%
\pgfpathlineto{\pgfqpoint{5.251960in}{1.026439in}}%
\pgfpathlineto{\pgfqpoint{5.256920in}{1.024969in}}%
\pgfpathlineto{\pgfqpoint{5.258160in}{1.024271in}}%
\pgfpathlineto{\pgfqpoint{5.259400in}{1.021654in}}%
\pgfpathlineto{\pgfqpoint{5.264360in}{1.026012in}}%
\pgfpathlineto{\pgfqpoint{5.266840in}{1.022030in}}%
\pgfpathlineto{\pgfqpoint{5.269320in}{1.024131in}}%
\pgfpathlineto{\pgfqpoint{5.271800in}{1.022253in}}%
\pgfpathlineto{\pgfqpoint{5.273040in}{1.024669in}}%
\pgfpathlineto{\pgfqpoint{5.275520in}{1.024019in}}%
\pgfpathlineto{\pgfqpoint{5.276760in}{1.024361in}}%
\pgfpathlineto{\pgfqpoint{5.278000in}{1.022566in}}%
\pgfpathlineto{\pgfqpoint{5.280480in}{1.025969in}}%
\pgfpathlineto{\pgfqpoint{5.284200in}{1.027430in}}%
\pgfpathlineto{\pgfqpoint{5.286680in}{1.026775in}}%
\pgfpathlineto{\pgfqpoint{5.294120in}{1.029058in}}%
\pgfpathlineto{\pgfqpoint{5.299080in}{1.022333in}}%
\pgfpathlineto{\pgfqpoint{5.300320in}{1.018411in}}%
\pgfpathlineto{\pgfqpoint{5.310240in}{1.020724in}}%
\pgfpathlineto{\pgfqpoint{5.312720in}{1.022719in}}%
\pgfpathlineto{\pgfqpoint{5.321400in}{1.023701in}}%
\pgfpathlineto{\pgfqpoint{5.323880in}{1.021976in}}%
\pgfpathlineto{\pgfqpoint{5.328840in}{1.023089in}}%
\pgfpathlineto{\pgfqpoint{5.330080in}{1.020677in}}%
\pgfpathlineto{\pgfqpoint{5.332560in}{1.021534in}}%
\pgfpathlineto{\pgfqpoint{5.337520in}{1.024312in}}%
\pgfpathlineto{\pgfqpoint{5.340000in}{1.023109in}}%
\pgfpathlineto{\pgfqpoint{5.342480in}{1.017414in}}%
\pgfpathlineto{\pgfqpoint{5.351160in}{1.019306in}}%
\pgfpathlineto{\pgfqpoint{5.354880in}{1.023842in}}%
\pgfpathlineto{\pgfqpoint{5.357360in}{1.023931in}}%
\pgfpathlineto{\pgfqpoint{5.359840in}{1.019874in}}%
\pgfpathlineto{\pgfqpoint{5.361080in}{1.016970in}}%
\pgfpathlineto{\pgfqpoint{5.364800in}{1.021321in}}%
\pgfpathlineto{\pgfqpoint{5.371000in}{1.021687in}}%
\pgfpathlineto{\pgfqpoint{5.373480in}{1.018596in}}%
\pgfpathlineto{\pgfqpoint{5.375960in}{1.022091in}}%
\pgfpathlineto{\pgfqpoint{5.382160in}{1.021534in}}%
\pgfpathlineto{\pgfqpoint{5.383400in}{1.019467in}}%
\pgfpathlineto{\pgfqpoint{5.388360in}{1.023170in}}%
\pgfpathlineto{\pgfqpoint{5.390840in}{1.019202in}}%
\pgfpathlineto{\pgfqpoint{5.393320in}{1.021280in}}%
\pgfpathlineto{\pgfqpoint{5.395800in}{1.019112in}}%
\pgfpathlineto{\pgfqpoint{5.397040in}{1.021276in}}%
\pgfpathlineto{\pgfqpoint{5.398280in}{1.019997in}}%
\pgfpathlineto{\pgfqpoint{5.400760in}{1.020801in}}%
\pgfpathlineto{\pgfqpoint{5.402000in}{1.019279in}}%
\pgfpathlineto{\pgfqpoint{5.403240in}{1.022901in}}%
\pgfpathlineto{\pgfqpoint{5.406960in}{1.023101in}}%
\pgfpathlineto{\pgfqpoint{5.409440in}{1.023770in}}%
\pgfpathlineto{\pgfqpoint{5.414400in}{1.025021in}}%
\pgfpathlineto{\pgfqpoint{5.419360in}{1.024120in}}%
\pgfpathlineto{\pgfqpoint{5.421840in}{1.021139in}}%
\pgfpathlineto{\pgfqpoint{5.425560in}{1.014725in}}%
\pgfpathlineto{\pgfqpoint{5.426800in}{1.015346in}}%
\pgfpathlineto{\pgfqpoint{5.428040in}{1.014205in}}%
\pgfpathlineto{\pgfqpoint{5.431760in}{1.015344in}}%
\pgfpathlineto{\pgfqpoint{5.441680in}{1.018727in}}%
\pgfpathlineto{\pgfqpoint{5.442920in}{1.017678in}}%
\pgfpathlineto{\pgfqpoint{5.445400in}{1.018530in}}%
\pgfpathlineto{\pgfqpoint{5.447880in}{1.016052in}}%
\pgfpathlineto{\pgfqpoint{5.451600in}{1.017435in}}%
\pgfpathlineto{\pgfqpoint{5.452840in}{1.016173in}}%
\pgfpathlineto{\pgfqpoint{5.454080in}{1.013220in}}%
\pgfpathlineto{\pgfqpoint{5.456560in}{1.014606in}}%
\pgfpathlineto{\pgfqpoint{5.461520in}{1.017280in}}%
\pgfpathlineto{\pgfqpoint{5.464000in}{1.016099in}}%
\pgfpathlineto{\pgfqpoint{5.467720in}{1.013161in}}%
\pgfpathlineto{\pgfqpoint{5.468960in}{1.013827in}}%
\pgfpathlineto{\pgfqpoint{5.471440in}{1.013238in}}%
\pgfpathlineto{\pgfqpoint{5.475160in}{1.013870in}}%
\pgfpathlineto{\pgfqpoint{5.480120in}{1.019934in}}%
\pgfpathlineto{\pgfqpoint{5.482600in}{1.018144in}}%
\pgfpathlineto{\pgfqpoint{5.485080in}{1.012885in}}%
\pgfpathlineto{\pgfqpoint{5.487560in}{1.015974in}}%
\pgfpathlineto{\pgfqpoint{5.488800in}{1.017446in}}%
\pgfpathlineto{\pgfqpoint{5.491280in}{1.017077in}}%
\pgfpathlineto{\pgfqpoint{5.492520in}{1.018597in}}%
\pgfpathlineto{\pgfqpoint{5.493760in}{1.017415in}}%
\pgfpathlineto{\pgfqpoint{5.495000in}{1.017977in}}%
\pgfpathlineto{\pgfqpoint{5.497480in}{1.015520in}}%
\pgfpathlineto{\pgfqpoint{5.499960in}{1.019515in}}%
\pgfpathlineto{\pgfqpoint{5.501200in}{1.019868in}}%
\pgfpathlineto{\pgfqpoint{5.502440in}{1.018942in}}%
\pgfpathlineto{\pgfqpoint{5.503680in}{1.019408in}}%
\pgfpathlineto{\pgfqpoint{5.504920in}{1.018422in}}%
\pgfpathlineto{\pgfqpoint{5.506160in}{1.019071in}}%
\pgfpathlineto{\pgfqpoint{5.507400in}{1.017335in}}%
\pgfpathlineto{\pgfqpoint{5.513600in}{1.019023in}}%
\pgfpathlineto{\pgfqpoint{5.514840in}{1.016367in}}%
\pgfpathlineto{\pgfqpoint{5.517320in}{1.018729in}}%
\pgfpathlineto{\pgfqpoint{5.519800in}{1.016013in}}%
\pgfpathlineto{\pgfqpoint{5.521040in}{1.017818in}}%
\pgfpathlineto{\pgfqpoint{5.523520in}{1.017359in}}%
\pgfpathlineto{\pgfqpoint{5.524760in}{1.018270in}}%
\pgfpathlineto{\pgfqpoint{5.526000in}{1.016432in}}%
\pgfpathlineto{\pgfqpoint{5.527240in}{1.019749in}}%
\pgfpathlineto{\pgfqpoint{5.530960in}{1.019175in}}%
\pgfpathlineto{\pgfqpoint{5.535920in}{1.021553in}}%
\pgfpathlineto{\pgfqpoint{5.542120in}{1.022995in}}%
\pgfpathlineto{\pgfqpoint{5.545840in}{1.018121in}}%
\pgfpathlineto{\pgfqpoint{5.548320in}{1.011559in}}%
\pgfpathlineto{\pgfqpoint{5.550800in}{1.011949in}}%
\pgfpathlineto{\pgfqpoint{5.552040in}{1.010344in}}%
\pgfpathlineto{\pgfqpoint{5.554520in}{1.011300in}}%
\pgfpathlineto{\pgfqpoint{5.558240in}{1.011629in}}%
\pgfpathlineto{\pgfqpoint{5.559480in}{1.013195in}}%
\pgfpathlineto{\pgfqpoint{5.561960in}{1.012926in}}%
\pgfpathlineto{\pgfqpoint{5.564440in}{1.014422in}}%
\pgfpathlineto{\pgfqpoint{5.565680in}{1.015203in}}%
\pgfpathlineto{\pgfqpoint{5.566920in}{1.013585in}}%
\pgfpathlineto{\pgfqpoint{5.569400in}{1.015258in}}%
\pgfpathlineto{\pgfqpoint{5.570640in}{1.013670in}}%
\pgfpathlineto{\pgfqpoint{5.575600in}{1.016524in}}%
\pgfpathlineto{\pgfqpoint{5.578080in}{1.013023in}}%
\pgfpathlineto{\pgfqpoint{5.579320in}{1.014315in}}%
\pgfpathlineto{\pgfqpoint{5.580560in}{1.013806in}}%
\pgfpathlineto{\pgfqpoint{5.585520in}{1.015933in}}%
\pgfpathlineto{\pgfqpoint{5.588000in}{1.014532in}}%
\pgfpathlineto{\pgfqpoint{5.589240in}{1.017363in}}%
\pgfpathlineto{\pgfqpoint{5.590480in}{1.015051in}}%
\pgfpathlineto{\pgfqpoint{5.595440in}{1.015800in}}%
\pgfpathlineto{\pgfqpoint{5.599160in}{1.016784in}}%
\pgfpathlineto{\pgfqpoint{5.604120in}{1.020359in}}%
\pgfpathlineto{\pgfqpoint{5.606600in}{1.018686in}}%
\pgfpathlineto{\pgfqpoint{5.609080in}{1.012950in}}%
\pgfpathlineto{\pgfqpoint{5.610320in}{1.015579in}}%
\pgfpathlineto{\pgfqpoint{5.611560in}{1.015394in}}%
\pgfpathlineto{\pgfqpoint{5.612800in}{1.016796in}}%
\pgfpathlineto{\pgfqpoint{5.615280in}{1.016375in}}%
\pgfpathlineto{\pgfqpoint{5.616520in}{1.017147in}}%
\pgfpathlineto{\pgfqpoint{5.617760in}{1.015694in}}%
\pgfpathlineto{\pgfqpoint{5.619000in}{1.016731in}}%
\pgfpathlineto{\pgfqpoint{5.621480in}{1.014739in}}%
\pgfpathlineto{\pgfqpoint{5.623960in}{1.018443in}}%
\pgfpathlineto{\pgfqpoint{5.625200in}{1.018818in}}%
\pgfpathlineto{\pgfqpoint{5.626440in}{1.017367in}}%
\pgfpathlineto{\pgfqpoint{5.627680in}{1.018795in}}%
\pgfpathlineto{\pgfqpoint{5.632640in}{1.018151in}}%
\pgfpathlineto{\pgfqpoint{5.636360in}{1.019009in}}%
\pgfpathlineto{\pgfqpoint{5.640080in}{1.015976in}}%
\pgfpathlineto{\pgfqpoint{5.641320in}{1.017483in}}%
\pgfpathlineto{\pgfqpoint{5.642560in}{1.014639in}}%
\pgfpathlineto{\pgfqpoint{5.643800in}{1.015215in}}%
\pgfpathlineto{\pgfqpoint{5.645040in}{1.017428in}}%
\pgfpathlineto{\pgfqpoint{5.648760in}{1.017311in}}%
\pgfpathlineto{\pgfqpoint{5.650000in}{1.015484in}}%
\pgfpathlineto{\pgfqpoint{5.651240in}{1.019310in}}%
\pgfpathlineto{\pgfqpoint{5.652480in}{1.018879in}}%
\pgfpathlineto{\pgfqpoint{5.656200in}{1.020672in}}%
\pgfpathlineto{\pgfqpoint{5.658680in}{1.021138in}}%
\pgfpathlineto{\pgfqpoint{5.661160in}{1.022306in}}%
\pgfpathlineto{\pgfqpoint{5.667360in}{1.022118in}}%
\pgfpathlineto{\pgfqpoint{5.671080in}{1.014438in}}%
\pgfpathlineto{\pgfqpoint{5.672320in}{1.011772in}}%
\pgfpathlineto{\pgfqpoint{5.674800in}{1.011734in}}%
\pgfpathlineto{\pgfqpoint{5.676040in}{1.010053in}}%
\pgfpathlineto{\pgfqpoint{5.679760in}{1.011267in}}%
\pgfpathlineto{\pgfqpoint{5.689680in}{1.013790in}}%
\pgfpathlineto{\pgfqpoint{5.692160in}{1.011715in}}%
\pgfpathlineto{\pgfqpoint{5.693400in}{1.013504in}}%
\pgfpathlineto{\pgfqpoint{5.694640in}{1.011769in}}%
\pgfpathlineto{\pgfqpoint{5.697120in}{1.013397in}}%
\pgfpathlineto{\pgfqpoint{5.698360in}{1.014144in}}%
\pgfpathlineto{\pgfqpoint{5.700840in}{1.011845in}}%
\pgfpathlineto{\pgfqpoint{5.702080in}{1.009449in}}%
\pgfpathlineto{\pgfqpoint{5.703320in}{1.010460in}}%
\pgfpathlineto{\pgfqpoint{5.704560in}{1.009206in}}%
\pgfpathlineto{\pgfqpoint{5.708280in}{1.010760in}}%
\pgfpathlineto{\pgfqpoint{5.713240in}{1.012482in}}%
\pgfpathlineto{\pgfqpoint{5.715720in}{1.009905in}}%
\pgfpathlineto{\pgfqpoint{5.718200in}{1.011104in}}%
\pgfpathlineto{\pgfqpoint{5.730600in}{1.013767in}}%
\pgfpathlineto{\pgfqpoint{5.733080in}{1.007655in}}%
\pgfpathlineto{\pgfqpoint{5.735560in}{1.011207in}}%
\pgfpathlineto{\pgfqpoint{5.736800in}{1.012819in}}%
\pgfpathlineto{\pgfqpoint{5.739280in}{1.012258in}}%
\pgfpathlineto{\pgfqpoint{5.740520in}{1.013071in}}%
\pgfpathlineto{\pgfqpoint{5.741760in}{1.011569in}}%
\pgfpathlineto{\pgfqpoint{5.743000in}{1.012552in}}%
\pgfpathlineto{\pgfqpoint{5.745480in}{1.010974in}}%
\pgfpathlineto{\pgfqpoint{5.747960in}{1.013939in}}%
\pgfpathlineto{\pgfqpoint{5.749200in}{1.014258in}}%
\pgfpathlineto{\pgfqpoint{5.750440in}{1.012700in}}%
\pgfpathlineto{\pgfqpoint{5.751680in}{1.013816in}}%
\pgfpathlineto{\pgfqpoint{5.752920in}{1.013272in}}%
\pgfpathlineto{\pgfqpoint{5.755400in}{1.014291in}}%
\pgfpathlineto{\pgfqpoint{5.760360in}{1.015267in}}%
\pgfpathlineto{\pgfqpoint{5.764080in}{1.011525in}}%
\pgfpathlineto{\pgfqpoint{5.765320in}{1.012638in}}%
\pgfpathlineto{\pgfqpoint{5.766560in}{1.009990in}}%
\pgfpathlineto{\pgfqpoint{5.770280in}{1.013246in}}%
\pgfpathlineto{\pgfqpoint{5.771520in}{1.012432in}}%
\pgfpathlineto{\pgfqpoint{5.772760in}{1.013298in}}%
\pgfpathlineto{\pgfqpoint{5.774000in}{1.011494in}}%
\pgfpathlineto{\pgfqpoint{5.776480in}{1.015521in}}%
\pgfpathlineto{\pgfqpoint{5.780200in}{1.018370in}}%
\pgfpathlineto{\pgfqpoint{5.782680in}{1.020373in}}%
\pgfpathlineto{\pgfqpoint{5.785160in}{1.022082in}}%
\pgfpathlineto{\pgfqpoint{5.790120in}{1.024317in}}%
\pgfpathlineto{\pgfqpoint{5.792600in}{1.021276in}}%
\pgfpathlineto{\pgfqpoint{5.795080in}{1.015506in}}%
\pgfpathlineto{\pgfqpoint{5.797560in}{1.013209in}}%
\pgfpathlineto{\pgfqpoint{5.802520in}{1.012927in}}%
\pgfpathlineto{\pgfqpoint{5.805000in}{1.013231in}}%
\pgfpathlineto{\pgfqpoint{5.809960in}{1.012937in}}%
\pgfpathlineto{\pgfqpoint{5.813680in}{1.014472in}}%
\pgfpathlineto{\pgfqpoint{5.816160in}{1.012704in}}%
\pgfpathlineto{\pgfqpoint{5.817400in}{1.014600in}}%
\pgfpathlineto{\pgfqpoint{5.818640in}{1.012477in}}%
\pgfpathlineto{\pgfqpoint{5.819880in}{1.012561in}}%
\pgfpathlineto{\pgfqpoint{5.822360in}{1.014873in}}%
\pgfpathlineto{\pgfqpoint{5.823600in}{1.014943in}}%
\pgfpathlineto{\pgfqpoint{5.826080in}{1.010853in}}%
\pgfpathlineto{\pgfqpoint{5.827320in}{1.012155in}}%
\pgfpathlineto{\pgfqpoint{5.828560in}{1.011088in}}%
\pgfpathlineto{\pgfqpoint{5.831040in}{1.012126in}}%
\pgfpathlineto{\pgfqpoint{5.836000in}{1.010819in}}%
\pgfpathlineto{\pgfqpoint{5.837240in}{1.013326in}}%
\pgfpathlineto{\pgfqpoint{5.838480in}{1.011359in}}%
\pgfpathlineto{\pgfqpoint{5.839720in}{1.011690in}}%
\pgfpathlineto{\pgfqpoint{5.842200in}{1.013688in}}%
\pgfpathlineto{\pgfqpoint{5.844680in}{1.013923in}}%
\pgfpathlineto{\pgfqpoint{5.848400in}{1.014221in}}%
\pgfpathlineto{\pgfqpoint{5.853360in}{1.016087in}}%
\pgfpathlineto{\pgfqpoint{5.855840in}{1.012296in}}%
\pgfpathlineto{\pgfqpoint{5.857080in}{1.008454in}}%
\pgfpathlineto{\pgfqpoint{5.859560in}{1.010932in}}%
\pgfpathlineto{\pgfqpoint{5.860800in}{1.012819in}}%
\pgfpathlineto{\pgfqpoint{5.863280in}{1.012634in}}%
\pgfpathlineto{\pgfqpoint{5.864520in}{1.014066in}}%
\pgfpathlineto{\pgfqpoint{5.865760in}{1.012473in}}%
\pgfpathlineto{\pgfqpoint{5.867000in}{1.013068in}}%
\pgfpathlineto{\pgfqpoint{5.869480in}{1.011306in}}%
\pgfpathlineto{\pgfqpoint{5.873200in}{1.014493in}}%
\pgfpathlineto{\pgfqpoint{5.874440in}{1.013571in}}%
\pgfpathlineto{\pgfqpoint{5.875680in}{1.014222in}}%
\pgfpathlineto{\pgfqpoint{5.876920in}{1.012781in}}%
\pgfpathlineto{\pgfqpoint{5.879400in}{1.013558in}}%
\pgfpathlineto{\pgfqpoint{5.883120in}{1.012244in}}%
\pgfpathlineto{\pgfqpoint{5.884360in}{1.012736in}}%
\pgfpathlineto{\pgfqpoint{5.886840in}{1.008251in}}%
\pgfpathlineto{\pgfqpoint{5.889320in}{1.010347in}}%
\pgfpathlineto{\pgfqpoint{5.890560in}{1.007940in}}%
\pgfpathlineto{\pgfqpoint{5.893040in}{1.012084in}}%
\pgfpathlineto{\pgfqpoint{5.895520in}{1.010830in}}%
\pgfpathlineto{\pgfqpoint{5.896760in}{1.012205in}}%
\pgfpathlineto{\pgfqpoint{5.898000in}{1.010220in}}%
\pgfpathlineto{\pgfqpoint{5.900480in}{1.015461in}}%
\pgfpathlineto{\pgfqpoint{5.904200in}{1.018741in}}%
\pgfpathlineto{\pgfqpoint{5.905440in}{1.018573in}}%
\pgfpathlineto{\pgfqpoint{5.910400in}{1.023414in}}%
\pgfpathlineto{\pgfqpoint{5.912880in}{1.023325in}}%
\pgfpathlineto{\pgfqpoint{5.914120in}{1.024665in}}%
\pgfpathlineto{\pgfqpoint{5.917840in}{1.018214in}}%
\pgfpathlineto{\pgfqpoint{5.920320in}{1.012997in}}%
\pgfpathlineto{\pgfqpoint{5.924040in}{1.012893in}}%
\pgfpathlineto{\pgfqpoint{5.931480in}{1.013120in}}%
\pgfpathlineto{\pgfqpoint{5.933960in}{1.011723in}}%
\pgfpathlineto{\pgfqpoint{5.941400in}{1.015202in}}%
\pgfpathlineto{\pgfqpoint{5.943880in}{1.012821in}}%
\pgfpathlineto{\pgfqpoint{5.947600in}{1.014896in}}%
\pgfpathlineto{\pgfqpoint{5.950080in}{1.011051in}}%
\pgfpathlineto{\pgfqpoint{5.951320in}{1.012834in}}%
\pgfpathlineto{\pgfqpoint{5.952560in}{1.011403in}}%
\pgfpathlineto{\pgfqpoint{5.955040in}{1.011970in}}%
\pgfpathlineto{\pgfqpoint{5.960000in}{1.010480in}}%
\pgfpathlineto{\pgfqpoint{5.961240in}{1.015450in}}%
\pgfpathlineto{\pgfqpoint{5.962480in}{1.013999in}}%
\pgfpathlineto{\pgfqpoint{5.963720in}{1.014524in}}%
\pgfpathlineto{\pgfqpoint{5.967440in}{1.018362in}}%
\pgfpathlineto{\pgfqpoint{5.973640in}{1.017930in}}%
\pgfpathlineto{\pgfqpoint{5.976120in}{1.019112in}}%
\pgfpathlineto{\pgfqpoint{5.977360in}{1.019074in}}%
\pgfpathlineto{\pgfqpoint{5.978600in}{1.017826in}}%
\pgfpathlineto{\pgfqpoint{5.981080in}{1.010350in}}%
\pgfpathlineto{\pgfqpoint{5.984800in}{1.014656in}}%
\pgfpathlineto{\pgfqpoint{5.987280in}{1.013997in}}%
\pgfpathlineto{\pgfqpoint{5.988520in}{1.015811in}}%
\pgfpathlineto{\pgfqpoint{5.989760in}{1.014621in}}%
\pgfpathlineto{\pgfqpoint{5.991000in}{1.015983in}}%
\pgfpathlineto{\pgfqpoint{5.993480in}{1.014424in}}%
\pgfpathlineto{\pgfqpoint{5.997200in}{1.017812in}}%
\pgfpathlineto{\pgfqpoint{5.998440in}{1.016611in}}%
\pgfpathlineto{\pgfqpoint{5.999680in}{1.017152in}}%
\pgfpathlineto{\pgfqpoint{6.000920in}{1.014997in}}%
\pgfpathlineto{\pgfqpoint{6.003400in}{1.015689in}}%
\pgfpathlineto{\pgfqpoint{6.005880in}{1.014332in}}%
\pgfpathlineto{\pgfqpoint{6.008360in}{1.015000in}}%
\pgfpathlineto{\pgfqpoint{6.010840in}{1.010720in}}%
\pgfpathlineto{\pgfqpoint{6.012080in}{1.011091in}}%
\pgfpathlineto{\pgfqpoint{6.013320in}{1.013356in}}%
\pgfpathlineto{\pgfqpoint{6.014560in}{1.011170in}}%
\pgfpathlineto{\pgfqpoint{6.020760in}{1.016426in}}%
\pgfpathlineto{\pgfqpoint{6.022000in}{1.013920in}}%
\pgfpathlineto{\pgfqpoint{6.024480in}{1.018298in}}%
\pgfpathlineto{\pgfqpoint{6.030680in}{1.023646in}}%
\pgfpathlineto{\pgfqpoint{6.031920in}{1.025991in}}%
\pgfpathlineto{\pgfqpoint{6.034400in}{1.024930in}}%
\pgfpathlineto{\pgfqpoint{6.036880in}{1.024519in}}%
\pgfpathlineto{\pgfqpoint{6.038120in}{1.026003in}}%
\pgfpathlineto{\pgfqpoint{6.043080in}{1.016527in}}%
\pgfpathlineto{\pgfqpoint{6.044320in}{1.013683in}}%
\pgfpathlineto{\pgfqpoint{6.055480in}{1.014879in}}%
\pgfpathlineto{\pgfqpoint{6.057960in}{1.012506in}}%
\pgfpathlineto{\pgfqpoint{6.060440in}{1.013923in}}%
\pgfpathlineto{\pgfqpoint{6.061680in}{1.015917in}}%
\pgfpathlineto{\pgfqpoint{6.064160in}{1.015329in}}%
\pgfpathlineto{\pgfqpoint{6.065400in}{1.017186in}}%
\pgfpathlineto{\pgfqpoint{6.066640in}{1.015010in}}%
\pgfpathlineto{\pgfqpoint{6.071600in}{1.015603in}}%
\pgfpathlineto{\pgfqpoint{6.074080in}{1.011578in}}%
\pgfpathlineto{\pgfqpoint{6.075320in}{1.014136in}}%
\pgfpathlineto{\pgfqpoint{6.076560in}{1.013204in}}%
\pgfpathlineto{\pgfqpoint{6.079040in}{1.014036in}}%
\pgfpathlineto{\pgfqpoint{6.082760in}{1.012360in}}%
\pgfpathlineto{\pgfqpoint{6.084000in}{1.012225in}}%
\pgfpathlineto{\pgfqpoint{6.085240in}{1.019209in}}%
\pgfpathlineto{\pgfqpoint{6.086480in}{1.017853in}}%
\pgfpathlineto{\pgfqpoint{6.092680in}{1.023675in}}%
\pgfpathlineto{\pgfqpoint{6.097640in}{1.024363in}}%
\pgfpathlineto{\pgfqpoint{6.100120in}{1.026369in}}%
\pgfpathlineto{\pgfqpoint{6.102600in}{1.024074in}}%
\pgfpathlineto{\pgfqpoint{6.105080in}{1.017198in}}%
\pgfpathlineto{\pgfqpoint{6.106320in}{1.018707in}}%
\pgfpathlineto{\pgfqpoint{6.107560in}{1.018388in}}%
\pgfpathlineto{\pgfqpoint{6.108800in}{1.019808in}}%
\pgfpathlineto{\pgfqpoint{6.111280in}{1.018611in}}%
\pgfpathlineto{\pgfqpoint{6.112520in}{1.020567in}}%
\pgfpathlineto{\pgfqpoint{6.113760in}{1.019024in}}%
\pgfpathlineto{\pgfqpoint{6.115000in}{1.020791in}}%
\pgfpathlineto{\pgfqpoint{6.117480in}{1.018581in}}%
\pgfpathlineto{\pgfqpoint{6.121200in}{1.023964in}}%
\pgfpathlineto{\pgfqpoint{6.122440in}{1.023256in}}%
\pgfpathlineto{\pgfqpoint{6.123680in}{1.024348in}}%
\pgfpathlineto{\pgfqpoint{6.126160in}{1.021810in}}%
\pgfpathlineto{\pgfqpoint{6.132360in}{1.021213in}}%
\pgfpathlineto{\pgfqpoint{6.134840in}{1.016716in}}%
\pgfpathlineto{\pgfqpoint{6.136080in}{1.017689in}}%
\pgfpathlineto{\pgfqpoint{6.137320in}{1.020418in}}%
\pgfpathlineto{\pgfqpoint{6.138560in}{1.016965in}}%
\pgfpathlineto{\pgfqpoint{6.142280in}{1.020532in}}%
\pgfpathlineto{\pgfqpoint{6.143520in}{1.020036in}}%
\pgfpathlineto{\pgfqpoint{6.144760in}{1.021178in}}%
\pgfpathlineto{\pgfqpoint{6.146000in}{1.019115in}}%
\pgfpathlineto{\pgfqpoint{6.148480in}{1.023314in}}%
\pgfpathlineto{\pgfqpoint{6.155920in}{1.031739in}}%
\pgfpathlineto{\pgfqpoint{6.158400in}{1.030049in}}%
\pgfpathlineto{\pgfqpoint{6.159640in}{1.029056in}}%
\pgfpathlineto{\pgfqpoint{6.162120in}{1.031334in}}%
\pgfpathlineto{\pgfqpoint{6.170800in}{1.018550in}}%
\pgfpathlineto{\pgfqpoint{6.175760in}{1.017381in}}%
\pgfpathlineto{\pgfqpoint{6.177000in}{1.015755in}}%
\pgfpathlineto{\pgfqpoint{6.179480in}{1.015914in}}%
\pgfpathlineto{\pgfqpoint{6.181960in}{1.014000in}}%
\pgfpathlineto{\pgfqpoint{6.184440in}{1.016305in}}%
\pgfpathlineto{\pgfqpoint{6.185680in}{1.019401in}}%
\pgfpathlineto{\pgfqpoint{6.188160in}{1.019052in}}%
\pgfpathlineto{\pgfqpoint{6.189400in}{1.020996in}}%
\pgfpathlineto{\pgfqpoint{6.190640in}{1.018664in}}%
\pgfpathlineto{\pgfqpoint{6.191880in}{1.019168in}}%
\pgfpathlineto{\pgfqpoint{6.193120in}{1.020919in}}%
\pgfpathlineto{\pgfqpoint{6.195600in}{1.020019in}}%
\pgfpathlineto{\pgfqpoint{6.198080in}{1.015144in}}%
\pgfpathlineto{\pgfqpoint{6.199320in}{1.017583in}}%
\pgfpathlineto{\pgfqpoint{6.201800in}{1.015747in}}%
\pgfpathlineto{\pgfqpoint{6.206760in}{1.015345in}}%
\pgfpathlineto{\pgfqpoint{6.208000in}{1.014025in}}%
\pgfpathlineto{\pgfqpoint{6.209240in}{1.018481in}}%
\pgfpathlineto{\pgfqpoint{6.210480in}{1.017383in}}%
\pgfpathlineto{\pgfqpoint{6.215440in}{1.023164in}}%
\pgfpathlineto{\pgfqpoint{6.221640in}{1.023306in}}%
\pgfpathlineto{\pgfqpoint{6.224120in}{1.024492in}}%
\pgfpathlineto{\pgfqpoint{6.225360in}{1.023907in}}%
\pgfpathlineto{\pgfqpoint{6.227840in}{1.018202in}}%
\pgfpathlineto{\pgfqpoint{6.229080in}{1.013828in}}%
\pgfpathlineto{\pgfqpoint{6.232800in}{1.016811in}}%
\pgfpathlineto{\pgfqpoint{6.235280in}{1.013993in}}%
\pgfpathlineto{\pgfqpoint{6.236520in}{1.016222in}}%
\pgfpathlineto{\pgfqpoint{6.237760in}{1.015800in}}%
\pgfpathlineto{\pgfqpoint{6.239000in}{1.019066in}}%
\pgfpathlineto{\pgfqpoint{6.241480in}{1.019630in}}%
\pgfpathlineto{\pgfqpoint{6.245200in}{1.026058in}}%
\pgfpathlineto{\pgfqpoint{6.246440in}{1.024834in}}%
\pgfpathlineto{\pgfqpoint{6.247680in}{1.025564in}}%
\pgfpathlineto{\pgfqpoint{6.248920in}{1.022591in}}%
\pgfpathlineto{\pgfqpoint{6.256360in}{1.028530in}}%
\pgfpathlineto{\pgfqpoint{6.258840in}{1.023680in}}%
\pgfpathlineto{\pgfqpoint{6.260080in}{1.023904in}}%
\pgfpathlineto{\pgfqpoint{6.261320in}{1.025334in}}%
\pgfpathlineto{\pgfqpoint{6.262560in}{1.021033in}}%
\pgfpathlineto{\pgfqpoint{6.266280in}{1.026072in}}%
\pgfpathlineto{\pgfqpoint{6.267520in}{1.025673in}}%
\pgfpathlineto{\pgfqpoint{6.268760in}{1.026621in}}%
\pgfpathlineto{\pgfqpoint{6.270000in}{1.023903in}}%
\pgfpathlineto{\pgfqpoint{6.272480in}{1.026966in}}%
\pgfpathlineto{\pgfqpoint{6.276200in}{1.030251in}}%
\pgfpathlineto{\pgfqpoint{6.277440in}{1.029492in}}%
\pgfpathlineto{\pgfqpoint{6.279920in}{1.033008in}}%
\pgfpathlineto{\pgfqpoint{6.283640in}{1.029306in}}%
\pgfpathlineto{\pgfqpoint{6.284880in}{1.029285in}}%
\pgfpathlineto{\pgfqpoint{6.286120in}{1.030661in}}%
\pgfpathlineto{\pgfqpoint{6.287360in}{1.028591in}}%
\pgfpathlineto{\pgfqpoint{6.289840in}{1.022778in}}%
\pgfpathlineto{\pgfqpoint{6.292320in}{1.018336in}}%
\pgfpathlineto{\pgfqpoint{6.296040in}{1.018799in}}%
\pgfpathlineto{\pgfqpoint{6.298520in}{1.018267in}}%
\pgfpathlineto{\pgfqpoint{6.307200in}{1.015433in}}%
\pgfpathlineto{\pgfqpoint{6.308440in}{1.016742in}}%
\pgfpathlineto{\pgfqpoint{6.309680in}{1.020367in}}%
\pgfpathlineto{\pgfqpoint{6.312160in}{1.020880in}}%
\pgfpathlineto{\pgfqpoint{6.313400in}{1.022935in}}%
\pgfpathlineto{\pgfqpoint{6.314640in}{1.021120in}}%
\pgfpathlineto{\pgfqpoint{6.319600in}{1.023116in}}%
\pgfpathlineto{\pgfqpoint{6.322080in}{1.020116in}}%
\pgfpathlineto{\pgfqpoint{6.323320in}{1.021986in}}%
\pgfpathlineto{\pgfqpoint{6.325800in}{1.019901in}}%
\pgfpathlineto{\pgfqpoint{6.329520in}{1.019320in}}%
\pgfpathlineto{\pgfqpoint{6.333240in}{1.017028in}}%
\pgfpathlineto{\pgfqpoint{6.334480in}{1.016651in}}%
\pgfpathlineto{\pgfqpoint{6.336960in}{1.020599in}}%
\pgfpathlineto{\pgfqpoint{6.341920in}{1.023198in}}%
\pgfpathlineto{\pgfqpoint{6.343160in}{1.021155in}}%
\pgfpathlineto{\pgfqpoint{6.348120in}{1.025235in}}%
\pgfpathlineto{\pgfqpoint{6.349360in}{1.024148in}}%
\pgfpathlineto{\pgfqpoint{6.353080in}{1.013345in}}%
\pgfpathlineto{\pgfqpoint{6.354320in}{1.015315in}}%
\pgfpathlineto{\pgfqpoint{6.355560in}{1.014244in}}%
\pgfpathlineto{\pgfqpoint{6.356800in}{1.015381in}}%
\pgfpathlineto{\pgfqpoint{6.359280in}{1.011954in}}%
\pgfpathlineto{\pgfqpoint{6.364240in}{1.019275in}}%
\pgfpathlineto{\pgfqpoint{6.366720in}{1.020842in}}%
\pgfpathlineto{\pgfqpoint{6.367960in}{1.022302in}}%
\pgfpathlineto{\pgfqpoint{6.369200in}{1.025518in}}%
\pgfpathlineto{\pgfqpoint{6.370440in}{1.024710in}}%
\pgfpathlineto{\pgfqpoint{6.371680in}{1.027180in}}%
\pgfpathlineto{\pgfqpoint{6.372920in}{1.024265in}}%
\pgfpathlineto{\pgfqpoint{6.380360in}{1.030067in}}%
\pgfpathlineto{\pgfqpoint{6.382840in}{1.026260in}}%
\pgfpathlineto{\pgfqpoint{6.385320in}{1.027364in}}%
\pgfpathlineto{\pgfqpoint{6.386560in}{1.023571in}}%
\pgfpathlineto{\pgfqpoint{6.387800in}{1.024198in}}%
\pgfpathlineto{\pgfqpoint{6.389040in}{1.027322in}}%
\pgfpathlineto{\pgfqpoint{6.390280in}{1.027249in}}%
\pgfpathlineto{\pgfqpoint{6.391520in}{1.025761in}}%
\pgfpathlineto{\pgfqpoint{6.392760in}{1.026401in}}%
\pgfpathlineto{\pgfqpoint{6.394000in}{1.023697in}}%
\pgfpathlineto{\pgfqpoint{6.396480in}{1.026973in}}%
\pgfpathlineto{\pgfqpoint{6.397720in}{1.026679in}}%
\pgfpathlineto{\pgfqpoint{6.400200in}{1.029656in}}%
\pgfpathlineto{\pgfqpoint{6.401440in}{1.028597in}}%
\pgfpathlineto{\pgfqpoint{6.403920in}{1.031940in}}%
\pgfpathlineto{\pgfqpoint{6.406400in}{1.030302in}}%
\pgfpathlineto{\pgfqpoint{6.410120in}{1.030955in}}%
\pgfpathlineto{\pgfqpoint{6.417560in}{1.021221in}}%
\pgfpathlineto{\pgfqpoint{6.425000in}{1.016763in}}%
\pgfpathlineto{\pgfqpoint{6.426240in}{1.018154in}}%
\pgfpathlineto{\pgfqpoint{6.427480in}{1.017862in}}%
\pgfpathlineto{\pgfqpoint{6.429960in}{1.014785in}}%
\pgfpathlineto{\pgfqpoint{6.432440in}{1.015053in}}%
\pgfpathlineto{\pgfqpoint{6.433680in}{1.017898in}}%
\pgfpathlineto{\pgfqpoint{6.434920in}{1.017072in}}%
\pgfpathlineto{\pgfqpoint{6.437400in}{1.019130in}}%
\pgfpathlineto{\pgfqpoint{6.439880in}{1.016681in}}%
\pgfpathlineto{\pgfqpoint{6.442360in}{1.016490in}}%
\pgfpathlineto{\pgfqpoint{6.444840in}{1.015463in}}%
\pgfpathlineto{\pgfqpoint{6.446080in}{1.014398in}}%
\pgfpathlineto{\pgfqpoint{6.447320in}{1.015081in}}%
\pgfpathlineto{\pgfqpoint{6.449800in}{1.011853in}}%
\pgfpathlineto{\pgfqpoint{6.452280in}{1.013509in}}%
\pgfpathlineto{\pgfqpoint{6.454760in}{1.012463in}}%
\pgfpathlineto{\pgfqpoint{6.456000in}{1.010176in}}%
\pgfpathlineto{\pgfqpoint{6.457240in}{1.015316in}}%
\pgfpathlineto{\pgfqpoint{6.458480in}{1.014732in}}%
\pgfpathlineto{\pgfqpoint{6.460960in}{1.018422in}}%
\pgfpathlineto{\pgfqpoint{6.464680in}{1.018719in}}%
\pgfpathlineto{\pgfqpoint{6.465920in}{1.020040in}}%
\pgfpathlineto{\pgfqpoint{6.467160in}{1.018061in}}%
\pgfpathlineto{\pgfqpoint{6.468400in}{1.019841in}}%
\pgfpathlineto{\pgfqpoint{6.469640in}{1.018948in}}%
\pgfpathlineto{\pgfqpoint{6.472120in}{1.020407in}}%
\pgfpathlineto{\pgfqpoint{6.474600in}{1.017793in}}%
\pgfpathlineto{\pgfqpoint{6.477080in}{1.011842in}}%
\pgfpathlineto{\pgfqpoint{6.480800in}{1.017193in}}%
\pgfpathlineto{\pgfqpoint{6.483280in}{1.012341in}}%
\pgfpathlineto{\pgfqpoint{6.488240in}{1.020645in}}%
\pgfpathlineto{\pgfqpoint{6.489480in}{1.020645in}}%
\pgfpathlineto{\pgfqpoint{6.491960in}{1.024299in}}%
\pgfpathlineto{\pgfqpoint{6.494440in}{1.028191in}}%
\pgfpathlineto{\pgfqpoint{6.495680in}{1.030620in}}%
\pgfpathlineto{\pgfqpoint{6.496920in}{1.028538in}}%
\pgfpathlineto{\pgfqpoint{6.499400in}{1.030324in}}%
\pgfpathlineto{\pgfqpoint{6.501880in}{1.030946in}}%
\pgfpathlineto{\pgfqpoint{6.504360in}{1.032229in}}%
\pgfpathlineto{\pgfqpoint{6.506840in}{1.027346in}}%
\pgfpathlineto{\pgfqpoint{6.509320in}{1.028407in}}%
\pgfpathlineto{\pgfqpoint{6.510560in}{1.023362in}}%
\pgfpathlineto{\pgfqpoint{6.511800in}{1.023784in}}%
\pgfpathlineto{\pgfqpoint{6.513040in}{1.027259in}}%
\pgfpathlineto{\pgfqpoint{6.515520in}{1.025582in}}%
\pgfpathlineto{\pgfqpoint{6.516760in}{1.026767in}}%
\pgfpathlineto{\pgfqpoint{6.518000in}{1.024504in}}%
\pgfpathlineto{\pgfqpoint{6.520480in}{1.028919in}}%
\pgfpathlineto{\pgfqpoint{6.522960in}{1.030773in}}%
\pgfpathlineto{\pgfqpoint{6.524200in}{1.033177in}}%
\pgfpathlineto{\pgfqpoint{6.525440in}{1.031410in}}%
\pgfpathlineto{\pgfqpoint{6.527920in}{1.034769in}}%
\pgfpathlineto{\pgfqpoint{6.529160in}{1.034261in}}%
\pgfpathlineto{\pgfqpoint{6.530400in}{1.035771in}}%
\pgfpathlineto{\pgfqpoint{6.532880in}{1.033106in}}%
\pgfpathlineto{\pgfqpoint{6.534120in}{1.033990in}}%
\pgfpathlineto{\pgfqpoint{6.545280in}{1.023616in}}%
\pgfpathlineto{\pgfqpoint{6.546520in}{1.024791in}}%
\pgfpathlineto{\pgfqpoint{6.549000in}{1.021200in}}%
\pgfpathlineto{\pgfqpoint{6.550240in}{1.022635in}}%
\pgfpathlineto{\pgfqpoint{6.551480in}{1.021888in}}%
\pgfpathlineto{\pgfqpoint{6.552720in}{1.019454in}}%
\pgfpathlineto{\pgfqpoint{6.556440in}{1.020413in}}%
\pgfpathlineto{\pgfqpoint{6.557680in}{1.022806in}}%
\pgfpathlineto{\pgfqpoint{6.558920in}{1.022189in}}%
\pgfpathlineto{\pgfqpoint{6.560160in}{1.022948in}}%
\pgfpathlineto{\pgfqpoint{6.561400in}{1.025064in}}%
\pgfpathlineto{\pgfqpoint{6.562640in}{1.022269in}}%
\pgfpathlineto{\pgfqpoint{6.565120in}{1.022767in}}%
\pgfpathlineto{\pgfqpoint{6.566360in}{1.022679in}}%
\pgfpathlineto{\pgfqpoint{6.570080in}{1.019347in}}%
\pgfpathlineto{\pgfqpoint{6.571320in}{1.019543in}}%
\pgfpathlineto{\pgfqpoint{6.573800in}{1.016878in}}%
\pgfpathlineto{\pgfqpoint{6.576280in}{1.017450in}}%
\pgfpathlineto{\pgfqpoint{6.578760in}{1.014183in}}%
\pgfpathlineto{\pgfqpoint{6.580000in}{1.010443in}}%
\pgfpathlineto{\pgfqpoint{6.581240in}{1.018497in}}%
\pgfpathlineto{\pgfqpoint{6.582480in}{1.018110in}}%
\pgfpathlineto{\pgfqpoint{6.584960in}{1.023432in}}%
\pgfpathlineto{\pgfqpoint{6.588680in}{1.021915in}}%
\pgfpathlineto{\pgfqpoint{6.589920in}{1.022950in}}%
\pgfpathlineto{\pgfqpoint{6.591160in}{1.018673in}}%
\pgfpathlineto{\pgfqpoint{6.594880in}{1.022055in}}%
\pgfpathlineto{\pgfqpoint{6.596120in}{1.024648in}}%
\pgfpathlineto{\pgfqpoint{6.598600in}{1.020890in}}%
\pgfpathlineto{\pgfqpoint{6.601080in}{1.012746in}}%
\pgfpathlineto{\pgfqpoint{6.604800in}{1.018970in}}%
\pgfpathlineto{\pgfqpoint{6.607280in}{1.014657in}}%
\pgfpathlineto{\pgfqpoint{6.613480in}{1.026546in}}%
\pgfpathlineto{\pgfqpoint{6.615960in}{1.029342in}}%
\pgfpathlineto{\pgfqpoint{6.617200in}{1.032629in}}%
\pgfpathlineto{\pgfqpoint{6.618440in}{1.032534in}}%
\pgfpathlineto{\pgfqpoint{6.619680in}{1.036030in}}%
\pgfpathlineto{\pgfqpoint{6.620920in}{1.032421in}}%
\pgfpathlineto{\pgfqpoint{6.623400in}{1.036245in}}%
\pgfpathlineto{\pgfqpoint{6.627120in}{1.036744in}}%
\pgfpathlineto{\pgfqpoint{6.628360in}{1.038780in}}%
\pgfpathlineto{\pgfqpoint{6.630840in}{1.036126in}}%
\pgfpathlineto{\pgfqpoint{6.633320in}{1.038034in}}%
\pgfpathlineto{\pgfqpoint{6.635800in}{1.032828in}}%
\pgfpathlineto{\pgfqpoint{6.637040in}{1.037172in}}%
\pgfpathlineto{\pgfqpoint{6.639520in}{1.034612in}}%
\pgfpathlineto{\pgfqpoint{6.640760in}{1.035802in}}%
\pgfpathlineto{\pgfqpoint{6.642000in}{1.032564in}}%
\pgfpathlineto{\pgfqpoint{6.645720in}{1.039330in}}%
\pgfpathlineto{\pgfqpoint{6.648200in}{1.042494in}}%
\pgfpathlineto{\pgfqpoint{6.649440in}{1.039980in}}%
\pgfpathlineto{\pgfqpoint{6.651920in}{1.044874in}}%
\pgfpathlineto{\pgfqpoint{6.653160in}{1.043974in}}%
\pgfpathlineto{\pgfqpoint{6.654400in}{1.046211in}}%
\pgfpathlineto{\pgfqpoint{6.660600in}{1.041448in}}%
\pgfpathlineto{\pgfqpoint{6.661840in}{1.043131in}}%
\pgfpathlineto{\pgfqpoint{6.665560in}{1.035496in}}%
\pgfpathlineto{\pgfqpoint{6.671760in}{1.029839in}}%
\pgfpathlineto{\pgfqpoint{6.673000in}{1.027230in}}%
\pgfpathlineto{\pgfqpoint{6.675480in}{1.028409in}}%
\pgfpathlineto{\pgfqpoint{6.677960in}{1.025061in}}%
\pgfpathlineto{\pgfqpoint{6.679200in}{1.023707in}}%
\pgfpathlineto{\pgfqpoint{6.682920in}{1.025469in}}%
\pgfpathlineto{\pgfqpoint{6.685400in}{1.031793in}}%
\pgfpathlineto{\pgfqpoint{6.686640in}{1.028804in}}%
\pgfpathlineto{\pgfqpoint{6.687880in}{1.028786in}}%
\pgfpathlineto{\pgfqpoint{6.691600in}{1.022401in}}%
\pgfpathlineto{\pgfqpoint{6.695320in}{1.020205in}}%
\pgfpathlineto{\pgfqpoint{6.697800in}{1.017021in}}%
\pgfpathlineto{\pgfqpoint{6.700280in}{1.017720in}}%
\pgfpathlineto{\pgfqpoint{6.701520in}{1.015757in}}%
\pgfpathlineto{\pgfqpoint{6.702760in}{1.016127in}}%
\pgfpathlineto{\pgfqpoint{6.704000in}{1.013401in}}%
\pgfpathlineto{\pgfqpoint{6.706480in}{1.022853in}}%
\pgfpathlineto{\pgfqpoint{6.708960in}{1.027985in}}%
\pgfpathlineto{\pgfqpoint{6.710200in}{1.025648in}}%
\pgfpathlineto{\pgfqpoint{6.711440in}{1.027731in}}%
\pgfpathlineto{\pgfqpoint{6.713920in}{1.027215in}}%
\pgfpathlineto{\pgfqpoint{6.715160in}{1.023281in}}%
\pgfpathlineto{\pgfqpoint{6.716400in}{1.024018in}}%
\pgfpathlineto{\pgfqpoint{6.718880in}{1.023273in}}%
\pgfpathlineto{\pgfqpoint{6.720120in}{1.025819in}}%
\pgfpathlineto{\pgfqpoint{6.721360in}{1.024502in}}%
\pgfpathlineto{\pgfqpoint{6.725080in}{1.015722in}}%
\pgfpathlineto{\pgfqpoint{6.727560in}{1.019677in}}%
\pgfpathlineto{\pgfqpoint{6.728800in}{1.021720in}}%
\pgfpathlineto{\pgfqpoint{6.731280in}{1.016665in}}%
\pgfpathlineto{\pgfqpoint{6.736240in}{1.026629in}}%
\pgfpathlineto{\pgfqpoint{6.737480in}{1.027592in}}%
\pgfpathlineto{\pgfqpoint{6.741200in}{1.036726in}}%
\pgfpathlineto{\pgfqpoint{6.742440in}{1.035231in}}%
\pgfpathlineto{\pgfqpoint{6.743680in}{1.038284in}}%
\pgfpathlineto{\pgfqpoint{6.744920in}{1.032505in}}%
\pgfpathlineto{\pgfqpoint{6.749880in}{1.037910in}}%
\pgfpathlineto{\pgfqpoint{6.754840in}{1.034614in}}%
\pgfpathlineto{\pgfqpoint{6.757320in}{1.037973in}}%
\pgfpathlineto{\pgfqpoint{6.759800in}{1.031796in}}%
\pgfpathlineto{\pgfqpoint{6.761040in}{1.036770in}}%
\pgfpathlineto{\pgfqpoint{6.763520in}{1.033710in}}%
\pgfpathlineto{\pgfqpoint{6.764760in}{1.034512in}}%
\pgfpathlineto{\pgfqpoint{6.766000in}{1.032423in}}%
\pgfpathlineto{\pgfqpoint{6.769720in}{1.040410in}}%
\pgfpathlineto{\pgfqpoint{6.772200in}{1.042544in}}%
\pgfpathlineto{\pgfqpoint{6.773440in}{1.039047in}}%
\pgfpathlineto{\pgfqpoint{6.778400in}{1.049385in}}%
\pgfpathlineto{\pgfqpoint{6.784600in}{1.039136in}}%
\pgfpathlineto{\pgfqpoint{6.785840in}{1.040799in}}%
\pgfpathlineto{\pgfqpoint{6.788320in}{1.034310in}}%
\pgfpathlineto{\pgfqpoint{6.790800in}{1.032418in}}%
\pgfpathlineto{\pgfqpoint{6.792040in}{1.033013in}}%
\pgfpathlineto{\pgfqpoint{6.794520in}{1.035617in}}%
\pgfpathlineto{\pgfqpoint{6.798240in}{1.030142in}}%
\pgfpathlineto{\pgfqpoint{6.799480in}{1.029626in}}%
\pgfpathlineto{\pgfqpoint{6.800720in}{1.026288in}}%
\pgfpathlineto{\pgfqpoint{6.801960in}{1.026879in}}%
\pgfpathlineto{\pgfqpoint{6.803200in}{1.024566in}}%
\pgfpathlineto{\pgfqpoint{6.804440in}{1.026561in}}%
\pgfpathlineto{\pgfqpoint{6.805680in}{1.025642in}}%
\pgfpathlineto{\pgfqpoint{6.806920in}{1.026474in}}%
\pgfpathlineto{\pgfqpoint{6.809400in}{1.034308in}}%
\pgfpathlineto{\pgfqpoint{6.811880in}{1.031504in}}%
\pgfpathlineto{\pgfqpoint{6.814360in}{1.029884in}}%
\pgfpathlineto{\pgfqpoint{6.816840in}{1.026903in}}%
\pgfpathlineto{\pgfqpoint{6.818080in}{1.025208in}}%
\pgfpathlineto{\pgfqpoint{6.819320in}{1.025485in}}%
\pgfpathlineto{\pgfqpoint{6.821800in}{1.019529in}}%
\pgfpathlineto{\pgfqpoint{6.824280in}{1.016724in}}%
\pgfpathlineto{\pgfqpoint{6.825520in}{1.014600in}}%
\pgfpathlineto{\pgfqpoint{6.826760in}{1.016599in}}%
\pgfpathlineto{\pgfqpoint{6.828000in}{1.014156in}}%
\pgfpathlineto{\pgfqpoint{6.832960in}{1.026355in}}%
\pgfpathlineto{\pgfqpoint{6.834200in}{1.024413in}}%
\pgfpathlineto{\pgfqpoint{6.836680in}{1.030851in}}%
\pgfpathlineto{\pgfqpoint{6.837920in}{1.032012in}}%
\pgfpathlineto{\pgfqpoint{6.839160in}{1.029008in}}%
\pgfpathlineto{\pgfqpoint{6.844120in}{1.032638in}}%
\pgfpathlineto{\pgfqpoint{6.847840in}{1.023374in}}%
\pgfpathlineto{\pgfqpoint{6.849080in}{1.019405in}}%
\pgfpathlineto{\pgfqpoint{6.851560in}{1.023428in}}%
\pgfpathlineto{\pgfqpoint{6.852800in}{1.024814in}}%
\pgfpathlineto{\pgfqpoint{6.855280in}{1.020288in}}%
\pgfpathlineto{\pgfqpoint{6.856520in}{1.024588in}}%
\pgfpathlineto{\pgfqpoint{6.857760in}{1.024141in}}%
\pgfpathlineto{\pgfqpoint{6.862720in}{1.037823in}}%
\pgfpathlineto{\pgfqpoint{6.865200in}{1.042681in}}%
\pgfpathlineto{\pgfqpoint{6.866440in}{1.039441in}}%
\pgfpathlineto{\pgfqpoint{6.867680in}{1.044409in}}%
\pgfpathlineto{\pgfqpoint{6.868920in}{1.039062in}}%
\pgfpathlineto{\pgfqpoint{6.870160in}{1.040604in}}%
\pgfpathlineto{\pgfqpoint{6.873880in}{1.052482in}}%
\pgfpathlineto{\pgfqpoint{6.875120in}{1.051563in}}%
\pgfpathlineto{\pgfqpoint{6.876360in}{1.053061in}}%
\pgfpathlineto{\pgfqpoint{6.878840in}{1.043584in}}%
\pgfpathlineto{\pgfqpoint{6.881320in}{1.049227in}}%
\pgfpathlineto{\pgfqpoint{6.883800in}{1.044428in}}%
\pgfpathlineto{\pgfqpoint{6.885040in}{1.048556in}}%
\pgfpathlineto{\pgfqpoint{6.887520in}{1.045115in}}%
\pgfpathlineto{\pgfqpoint{6.890000in}{1.049986in}}%
\pgfpathlineto{\pgfqpoint{6.893720in}{1.055011in}}%
\pgfpathlineto{\pgfqpoint{6.894960in}{1.054651in}}%
\pgfpathlineto{\pgfqpoint{6.896200in}{1.056204in}}%
\pgfpathlineto{\pgfqpoint{6.897440in}{1.053431in}}%
\pgfpathlineto{\pgfqpoint{6.901160in}{1.065935in}}%
\pgfpathlineto{\pgfqpoint{6.902400in}{1.065290in}}%
\pgfpathlineto{\pgfqpoint{6.907360in}{1.054318in}}%
\pgfpathlineto{\pgfqpoint{6.909840in}{1.053739in}}%
\pgfpathlineto{\pgfqpoint{6.913560in}{1.042794in}}%
\pgfpathlineto{\pgfqpoint{6.916040in}{1.040375in}}%
\pgfpathlineto{\pgfqpoint{6.918520in}{1.047153in}}%
\pgfpathlineto{\pgfqpoint{6.919760in}{1.046996in}}%
\pgfpathlineto{\pgfqpoint{6.921000in}{1.044222in}}%
\pgfpathlineto{\pgfqpoint{6.922240in}{1.045273in}}%
\pgfpathlineto{\pgfqpoint{6.923480in}{1.044787in}}%
\pgfpathlineto{\pgfqpoint{6.927200in}{1.038330in}}%
\pgfpathlineto{\pgfqpoint{6.929680in}{1.040144in}}%
\pgfpathlineto{\pgfqpoint{6.930920in}{1.040508in}}%
\pgfpathlineto{\pgfqpoint{6.933400in}{1.050282in}}%
\pgfpathlineto{\pgfqpoint{6.939600in}{1.040521in}}%
\pgfpathlineto{\pgfqpoint{6.940840in}{1.041468in}}%
\pgfpathlineto{\pgfqpoint{6.942080in}{1.040707in}}%
\pgfpathlineto{\pgfqpoint{6.943320in}{1.041272in}}%
\pgfpathlineto{\pgfqpoint{6.945800in}{1.035362in}}%
\pgfpathlineto{\pgfqpoint{6.948280in}{1.032245in}}%
\pgfpathlineto{\pgfqpoint{6.949520in}{1.029498in}}%
\pgfpathlineto{\pgfqpoint{6.950760in}{1.032532in}}%
\pgfpathlineto{\pgfqpoint{6.952000in}{1.029907in}}%
\pgfpathlineto{\pgfqpoint{6.954480in}{1.033220in}}%
\pgfpathlineto{\pgfqpoint{6.955720in}{1.038357in}}%
\pgfpathlineto{\pgfqpoint{6.958200in}{1.032276in}}%
\pgfpathlineto{\pgfqpoint{6.961920in}{1.041120in}}%
\pgfpathlineto{\pgfqpoint{6.963160in}{1.036929in}}%
\pgfpathlineto{\pgfqpoint{6.965640in}{1.039394in}}%
\pgfpathlineto{\pgfqpoint{6.968120in}{1.041477in}}%
\pgfpathlineto{\pgfqpoint{6.969360in}{1.039750in}}%
\pgfpathlineto{\pgfqpoint{6.973080in}{1.025080in}}%
\pgfpathlineto{\pgfqpoint{6.974320in}{1.027035in}}%
\pgfpathlineto{\pgfqpoint{6.976800in}{1.032658in}}%
\pgfpathlineto{\pgfqpoint{6.978040in}{1.026486in}}%
\pgfpathlineto{\pgfqpoint{6.979280in}{1.026454in}}%
\pgfpathlineto{\pgfqpoint{6.986720in}{1.055189in}}%
\pgfpathlineto{\pgfqpoint{6.987960in}{1.053207in}}%
\pgfpathlineto{\pgfqpoint{6.989200in}{1.055843in}}%
\pgfpathlineto{\pgfqpoint{6.990440in}{1.054766in}}%
\pgfpathlineto{\pgfqpoint{6.991680in}{1.061591in}}%
\pgfpathlineto{\pgfqpoint{6.994160in}{1.050513in}}%
\pgfpathlineto{\pgfqpoint{6.997880in}{1.059920in}}%
\pgfpathlineto{\pgfqpoint{6.999120in}{1.060294in}}%
\pgfpathlineto{\pgfqpoint{7.000360in}{1.065853in}}%
\pgfpathlineto{\pgfqpoint{7.002840in}{1.051668in}}%
\pgfpathlineto{\pgfqpoint{7.005320in}{1.057685in}}%
\pgfpathlineto{\pgfqpoint{7.006560in}{1.050506in}}%
\pgfpathlineto{\pgfqpoint{7.007800in}{1.051035in}}%
\pgfpathlineto{\pgfqpoint{7.009040in}{1.057859in}}%
\pgfpathlineto{\pgfqpoint{7.011520in}{1.049549in}}%
\pgfpathlineto{\pgfqpoint{7.012760in}{1.051360in}}%
\pgfpathlineto{\pgfqpoint{7.014000in}{1.048771in}}%
\pgfpathlineto{\pgfqpoint{7.016480in}{1.052430in}}%
\pgfpathlineto{\pgfqpoint{7.017720in}{1.058765in}}%
\pgfpathlineto{\pgfqpoint{7.020200in}{1.060332in}}%
\pgfpathlineto{\pgfqpoint{7.021440in}{1.057241in}}%
\pgfpathlineto{\pgfqpoint{7.026400in}{1.074266in}}%
\pgfpathlineto{\pgfqpoint{7.027640in}{1.069407in}}%
\pgfpathlineto{\pgfqpoint{7.028880in}{1.057526in}}%
\pgfpathlineto{\pgfqpoint{7.031360in}{1.056809in}}%
\pgfpathlineto{\pgfqpoint{7.032600in}{1.055408in}}%
\pgfpathlineto{\pgfqpoint{7.033840in}{1.057404in}}%
\pgfpathlineto{\pgfqpoint{7.036320in}{1.048505in}}%
\pgfpathlineto{\pgfqpoint{7.037560in}{1.048722in}}%
\pgfpathlineto{\pgfqpoint{7.041280in}{1.052817in}}%
\pgfpathlineto{\pgfqpoint{7.042520in}{1.052745in}}%
\pgfpathlineto{\pgfqpoint{7.043760in}{1.053900in}}%
\pgfpathlineto{\pgfqpoint{7.045000in}{1.053289in}}%
\pgfpathlineto{\pgfqpoint{7.051200in}{1.035148in}}%
\pgfpathlineto{\pgfqpoint{7.052440in}{1.035738in}}%
\pgfpathlineto{\pgfqpoint{7.054920in}{1.034339in}}%
\pgfpathlineto{\pgfqpoint{7.056160in}{1.037534in}}%
\pgfpathlineto{\pgfqpoint{7.057400in}{1.044474in}}%
\pgfpathlineto{\pgfqpoint{7.058640in}{1.043712in}}%
\pgfpathlineto{\pgfqpoint{7.059880in}{1.045313in}}%
\pgfpathlineto{\pgfqpoint{7.061120in}{1.044526in}}%
\pgfpathlineto{\pgfqpoint{7.062360in}{1.042269in}}%
\pgfpathlineto{\pgfqpoint{7.064840in}{1.043517in}}%
\pgfpathlineto{\pgfqpoint{7.066080in}{1.042856in}}%
\pgfpathlineto{\pgfqpoint{7.067320in}{1.047402in}}%
\pgfpathlineto{\pgfqpoint{7.068560in}{1.044146in}}%
\pgfpathlineto{\pgfqpoint{7.069800in}{1.045319in}}%
\pgfpathlineto{\pgfqpoint{7.071040in}{1.048110in}}%
\pgfpathlineto{\pgfqpoint{7.073520in}{1.041743in}}%
\pgfpathlineto{\pgfqpoint{7.074760in}{1.044578in}}%
\pgfpathlineto{\pgfqpoint{7.076000in}{1.043115in}}%
\pgfpathlineto{\pgfqpoint{7.077240in}{1.057124in}}%
\pgfpathlineto{\pgfqpoint{7.078480in}{1.058239in}}%
\pgfpathlineto{\pgfqpoint{7.079720in}{1.062943in}}%
\pgfpathlineto{\pgfqpoint{7.080960in}{1.061591in}}%
\pgfpathlineto{\pgfqpoint{7.082200in}{1.057177in}}%
\pgfpathlineto{\pgfqpoint{7.083440in}{1.059342in}}%
\pgfpathlineto{\pgfqpoint{7.084680in}{1.067665in}}%
\pgfpathlineto{\pgfqpoint{7.087160in}{1.058426in}}%
\pgfpathlineto{\pgfqpoint{7.089640in}{1.053936in}}%
\pgfpathlineto{\pgfqpoint{7.090880in}{1.051122in}}%
\pgfpathlineto{\pgfqpoint{7.093360in}{1.043653in}}%
\pgfpathlineto{\pgfqpoint{7.097080in}{1.031883in}}%
\pgfpathlineto{\pgfqpoint{7.098320in}{1.031248in}}%
\pgfpathlineto{\pgfqpoint{7.100800in}{1.040939in}}%
\pgfpathlineto{\pgfqpoint{7.103280in}{1.031674in}}%
\pgfpathlineto{\pgfqpoint{7.107000in}{1.044429in}}%
\pgfpathlineto{\pgfqpoint{7.110720in}{1.074224in}}%
\pgfpathlineto{\pgfqpoint{7.113200in}{1.065184in}}%
\pgfpathlineto{\pgfqpoint{7.114440in}{1.065912in}}%
\pgfpathlineto{\pgfqpoint{7.115680in}{1.071902in}}%
\pgfpathlineto{\pgfqpoint{7.118160in}{1.060960in}}%
\pgfpathlineto{\pgfqpoint{7.119400in}{1.066750in}}%
\pgfpathlineto{\pgfqpoint{7.120640in}{1.067153in}}%
\pgfpathlineto{\pgfqpoint{7.121880in}{1.070977in}}%
\pgfpathlineto{\pgfqpoint{7.123120in}{1.071108in}}%
\pgfpathlineto{\pgfqpoint{7.124360in}{1.076208in}}%
\pgfpathlineto{\pgfqpoint{7.126840in}{1.062455in}}%
\pgfpathlineto{\pgfqpoint{7.129320in}{1.076694in}}%
\pgfpathlineto{\pgfqpoint{7.130560in}{1.072150in}}%
\pgfpathlineto{\pgfqpoint{7.131800in}{1.074511in}}%
\pgfpathlineto{\pgfqpoint{7.133040in}{1.082765in}}%
\pgfpathlineto{\pgfqpoint{7.134280in}{1.081367in}}%
\pgfpathlineto{\pgfqpoint{7.136760in}{1.075483in}}%
\pgfpathlineto{\pgfqpoint{7.139240in}{1.080107in}}%
\pgfpathlineto{\pgfqpoint{7.140480in}{1.081099in}}%
\pgfpathlineto{\pgfqpoint{7.142960in}{1.092913in}}%
\pgfpathlineto{\pgfqpoint{7.144200in}{1.089595in}}%
\pgfpathlineto{\pgfqpoint{7.146680in}{1.078876in}}%
\pgfpathlineto{\pgfqpoint{7.150400in}{1.092706in}}%
\pgfpathlineto{\pgfqpoint{7.152880in}{1.075207in}}%
\pgfpathlineto{\pgfqpoint{7.154120in}{1.082717in}}%
\pgfpathlineto{\pgfqpoint{7.156600in}{1.078158in}}%
\pgfpathlineto{\pgfqpoint{7.157840in}{1.082637in}}%
\pgfpathlineto{\pgfqpoint{7.165280in}{1.062349in}}%
\pgfpathlineto{\pgfqpoint{7.167760in}{1.073980in}}%
\pgfpathlineto{\pgfqpoint{7.170240in}{1.071990in}}%
\pgfpathlineto{\pgfqpoint{7.172720in}{1.059627in}}%
\pgfpathlineto{\pgfqpoint{7.173960in}{1.062563in}}%
\pgfpathlineto{\pgfqpoint{7.176440in}{1.051674in}}%
\pgfpathlineto{\pgfqpoint{7.177680in}{1.050060in}}%
\pgfpathlineto{\pgfqpoint{7.180160in}{1.055342in}}%
\pgfpathlineto{\pgfqpoint{7.181400in}{1.065996in}}%
\pgfpathlineto{\pgfqpoint{7.182640in}{1.064541in}}%
\pgfpathlineto{\pgfqpoint{7.185120in}{1.068369in}}%
\pgfpathlineto{\pgfqpoint{7.186360in}{1.064758in}}%
\pgfpathlineto{\pgfqpoint{7.187600in}{1.065228in}}%
\pgfpathlineto{\pgfqpoint{7.190080in}{1.068180in}}%
\pgfpathlineto{\pgfqpoint{7.191320in}{1.070234in}}%
\pgfpathlineto{\pgfqpoint{7.193800in}{1.063590in}}%
\pgfpathlineto{\pgfqpoint{7.195040in}{1.063829in}}%
\pgfpathlineto{\pgfqpoint{7.196280in}{1.060047in}}%
\pgfpathlineto{\pgfqpoint{7.197520in}{1.061122in}}%
\pgfpathlineto{\pgfqpoint{7.200000in}{1.067147in}}%
\pgfpathlineto{\pgfqpoint{7.200000in}{1.067147in}}%
\pgfusepath{stroke}%
\end{pgfscope}%
\begin{pgfscope}%
\pgfpathrectangle{\pgfqpoint{1.000000in}{0.300000in}}{\pgfqpoint{6.200000in}{2.400000in}} %
\pgfusepath{clip}%
\pgfsetrectcap%
\pgfsetroundjoin%
\pgfsetlinewidth{2.007500pt}%
\definecolor{currentstroke}{rgb}{0.000000,0.000000,1.000000}%
\pgfsetstrokecolor{currentstroke}%
\pgfsetdash{}{0pt}%
\pgfpathmoveto{\pgfqpoint{2.241240in}{1.086016in}}%
\pgfpathlineto{\pgfqpoint{6.578760in}{1.086016in}}%
\pgfusepath{stroke}%
\end{pgfscope}%
\begin{pgfscope}%
\pgfpathrectangle{\pgfqpoint{1.000000in}{0.300000in}}{\pgfqpoint{6.200000in}{2.400000in}} %
\pgfusepath{clip}%
\pgfsetrectcap%
\pgfsetroundjoin%
\pgfsetlinewidth{1.003750pt}%
\definecolor{currentstroke}{rgb}{0.000000,0.500000,0.000000}%
\pgfsetstrokecolor{currentstroke}%
\pgfsetdash{}{0pt}%
\pgfpathmoveto{\pgfqpoint{1.001240in}{1.337138in}}%
\pgfpathlineto{\pgfqpoint{1.002480in}{1.729107in}}%
\pgfpathlineto{\pgfqpoint{1.003720in}{1.771831in}}%
\pgfpathlineto{\pgfqpoint{1.011160in}{1.152010in}}%
\pgfpathlineto{\pgfqpoint{1.016120in}{0.958250in}}%
\pgfpathlineto{\pgfqpoint{1.021080in}{0.847472in}}%
\pgfpathlineto{\pgfqpoint{1.027280in}{0.773010in}}%
\pgfpathlineto{\pgfqpoint{1.031000in}{0.743431in}}%
\pgfpathlineto{\pgfqpoint{1.032240in}{0.743706in}}%
\pgfpathlineto{\pgfqpoint{1.035960in}{0.734112in}}%
\pgfpathlineto{\pgfqpoint{1.037200in}{0.733846in}}%
\pgfpathlineto{\pgfqpoint{1.038440in}{0.731513in}}%
\pgfpathlineto{\pgfqpoint{1.042160in}{0.717786in}}%
\pgfpathlineto{\pgfqpoint{1.044640in}{0.720069in}}%
\pgfpathlineto{\pgfqpoint{1.047120in}{0.704306in}}%
\pgfpathlineto{\pgfqpoint{1.050840in}{0.684013in}}%
\pgfpathlineto{\pgfqpoint{1.053320in}{0.679370in}}%
\pgfpathlineto{\pgfqpoint{1.055800in}{0.670825in}}%
\pgfpathlineto{\pgfqpoint{1.059520in}{0.666126in}}%
\pgfpathlineto{\pgfqpoint{1.062000in}{0.657035in}}%
\pgfpathlineto{\pgfqpoint{1.066960in}{0.635592in}}%
\pgfpathlineto{\pgfqpoint{1.068200in}{0.634537in}}%
\pgfpathlineto{\pgfqpoint{1.073160in}{0.638264in}}%
\pgfpathlineto{\pgfqpoint{1.080600in}{0.632786in}}%
\pgfpathlineto{\pgfqpoint{1.081840in}{0.634842in}}%
\pgfpathlineto{\pgfqpoint{1.086800in}{0.620343in}}%
\pgfpathlineto{\pgfqpoint{1.089280in}{0.611148in}}%
\pgfpathlineto{\pgfqpoint{1.090520in}{0.610174in}}%
\pgfpathlineto{\pgfqpoint{1.091760in}{0.612192in}}%
\pgfpathlineto{\pgfqpoint{1.097960in}{0.598172in}}%
\pgfpathlineto{\pgfqpoint{1.100440in}{0.597745in}}%
\pgfpathlineto{\pgfqpoint{1.101680in}{0.602005in}}%
\pgfpathlineto{\pgfqpoint{1.109120in}{0.597646in}}%
\pgfpathlineto{\pgfqpoint{1.111600in}{0.600810in}}%
\pgfpathlineto{\pgfqpoint{1.112840in}{0.599438in}}%
\pgfpathlineto{\pgfqpoint{1.114080in}{0.600155in}}%
\pgfpathlineto{\pgfqpoint{1.115320in}{0.602385in}}%
\pgfpathlineto{\pgfqpoint{1.116560in}{0.602317in}}%
\pgfpathlineto{\pgfqpoint{1.117800in}{0.600753in}}%
\pgfpathlineto{\pgfqpoint{1.120280in}{0.602910in}}%
\pgfpathlineto{\pgfqpoint{1.122760in}{0.599798in}}%
\pgfpathlineto{\pgfqpoint{1.124000in}{0.600516in}}%
\pgfpathlineto{\pgfqpoint{1.126480in}{0.594510in}}%
\pgfpathlineto{\pgfqpoint{1.127720in}{0.595085in}}%
\pgfpathlineto{\pgfqpoint{1.131440in}{0.601090in}}%
\pgfpathlineto{\pgfqpoint{1.132680in}{0.600893in}}%
\pgfpathlineto{\pgfqpoint{1.133920in}{0.602305in}}%
\pgfpathlineto{\pgfqpoint{1.142600in}{0.590458in}}%
\pgfpathlineto{\pgfqpoint{1.143840in}{0.590253in}}%
\pgfpathlineto{\pgfqpoint{1.155000in}{0.567627in}}%
\pgfpathlineto{\pgfqpoint{1.157480in}{0.573421in}}%
\pgfpathlineto{\pgfqpoint{1.159960in}{0.577836in}}%
\pgfpathlineto{\pgfqpoint{1.164920in}{0.571425in}}%
\pgfpathlineto{\pgfqpoint{1.166160in}{0.570992in}}%
\pgfpathlineto{\pgfqpoint{1.168640in}{0.573712in}}%
\pgfpathlineto{\pgfqpoint{1.171120in}{0.569037in}}%
\pgfpathlineto{\pgfqpoint{1.172360in}{0.567767in}}%
\pgfpathlineto{\pgfqpoint{1.173600in}{0.569016in}}%
\pgfpathlineto{\pgfqpoint{1.176080in}{0.566710in}}%
\pgfpathlineto{\pgfqpoint{1.181040in}{0.567196in}}%
\pgfpathlineto{\pgfqpoint{1.183520in}{0.570553in}}%
\pgfpathlineto{\pgfqpoint{1.184760in}{0.570682in}}%
\pgfpathlineto{\pgfqpoint{1.187240in}{0.562987in}}%
\pgfpathlineto{\pgfqpoint{1.188480in}{0.563901in}}%
\pgfpathlineto{\pgfqpoint{1.193440in}{0.558621in}}%
\pgfpathlineto{\pgfqpoint{1.195920in}{0.559482in}}%
\pgfpathlineto{\pgfqpoint{1.197160in}{0.561904in}}%
\pgfpathlineto{\pgfqpoint{1.199640in}{0.561543in}}%
\pgfpathlineto{\pgfqpoint{1.200880in}{0.561849in}}%
\pgfpathlineto{\pgfqpoint{1.202120in}{0.564093in}}%
\pgfpathlineto{\pgfqpoint{1.207080in}{0.561738in}}%
\pgfpathlineto{\pgfqpoint{1.213280in}{0.555504in}}%
\pgfpathlineto{\pgfqpoint{1.214520in}{0.555188in}}%
\pgfpathlineto{\pgfqpoint{1.217000in}{0.557346in}}%
\pgfpathlineto{\pgfqpoint{1.221960in}{0.552509in}}%
\pgfpathlineto{\pgfqpoint{1.224440in}{0.554674in}}%
\pgfpathlineto{\pgfqpoint{1.225680in}{0.555297in}}%
\pgfpathlineto{\pgfqpoint{1.228160in}{0.550850in}}%
\pgfpathlineto{\pgfqpoint{1.229400in}{0.550249in}}%
\pgfpathlineto{\pgfqpoint{1.230640in}{0.552100in}}%
\pgfpathlineto{\pgfqpoint{1.231880in}{0.551608in}}%
\pgfpathlineto{\pgfqpoint{1.233120in}{0.552444in}}%
\pgfpathlineto{\pgfqpoint{1.235600in}{0.558143in}}%
\pgfpathlineto{\pgfqpoint{1.241800in}{0.556789in}}%
\pgfpathlineto{\pgfqpoint{1.244280in}{0.558707in}}%
\pgfpathlineto{\pgfqpoint{1.249240in}{0.550724in}}%
\pgfpathlineto{\pgfqpoint{1.251720in}{0.553495in}}%
\pgfpathlineto{\pgfqpoint{1.254200in}{0.555965in}}%
\pgfpathlineto{\pgfqpoint{1.256680in}{0.556090in}}%
\pgfpathlineto{\pgfqpoint{1.257920in}{0.558479in}}%
\pgfpathlineto{\pgfqpoint{1.262880in}{0.554243in}}%
\pgfpathlineto{\pgfqpoint{1.265360in}{0.555219in}}%
\pgfpathlineto{\pgfqpoint{1.267840in}{0.558465in}}%
\pgfpathlineto{\pgfqpoint{1.270320in}{0.555959in}}%
\pgfpathlineto{\pgfqpoint{1.275280in}{0.551944in}}%
\pgfpathlineto{\pgfqpoint{1.279000in}{0.545100in}}%
\pgfpathlineto{\pgfqpoint{1.283960in}{0.549681in}}%
\pgfpathlineto{\pgfqpoint{1.286440in}{0.546961in}}%
\pgfpathlineto{\pgfqpoint{1.288920in}{0.547073in}}%
\pgfpathlineto{\pgfqpoint{1.290160in}{0.546746in}}%
\pgfpathlineto{\pgfqpoint{1.292640in}{0.549307in}}%
\pgfpathlineto{\pgfqpoint{1.295120in}{0.547328in}}%
\pgfpathlineto{\pgfqpoint{1.296360in}{0.547349in}}%
\pgfpathlineto{\pgfqpoint{1.297600in}{0.549048in}}%
\pgfpathlineto{\pgfqpoint{1.300080in}{0.547365in}}%
\pgfpathlineto{\pgfqpoint{1.302560in}{0.549279in}}%
\pgfpathlineto{\pgfqpoint{1.307520in}{0.552262in}}%
\pgfpathlineto{\pgfqpoint{1.308760in}{0.552027in}}%
\pgfpathlineto{\pgfqpoint{1.311240in}{0.546807in}}%
\pgfpathlineto{\pgfqpoint{1.312480in}{0.547178in}}%
\pgfpathlineto{\pgfqpoint{1.316200in}{0.540028in}}%
\pgfpathlineto{\pgfqpoint{1.321160in}{0.543525in}}%
\pgfpathlineto{\pgfqpoint{1.323640in}{0.541720in}}%
\pgfpathlineto{\pgfqpoint{1.327360in}{0.545485in}}%
\pgfpathlineto{\pgfqpoint{1.328600in}{0.544086in}}%
\pgfpathlineto{\pgfqpoint{1.329840in}{0.545551in}}%
\pgfpathlineto{\pgfqpoint{1.336040in}{0.539554in}}%
\pgfpathlineto{\pgfqpoint{1.338520in}{0.539479in}}%
\pgfpathlineto{\pgfqpoint{1.341000in}{0.544005in}}%
\pgfpathlineto{\pgfqpoint{1.343480in}{0.542891in}}%
\pgfpathlineto{\pgfqpoint{1.344720in}{0.541797in}}%
\pgfpathlineto{\pgfqpoint{1.345960in}{0.538636in}}%
\pgfpathlineto{\pgfqpoint{1.349680in}{0.541220in}}%
\pgfpathlineto{\pgfqpoint{1.352160in}{0.538629in}}%
\pgfpathlineto{\pgfqpoint{1.357120in}{0.539315in}}%
\pgfpathlineto{\pgfqpoint{1.362080in}{0.544551in}}%
\pgfpathlineto{\pgfqpoint{1.363320in}{0.543516in}}%
\pgfpathlineto{\pgfqpoint{1.364560in}{0.543991in}}%
\pgfpathlineto{\pgfqpoint{1.365800in}{0.543215in}}%
\pgfpathlineto{\pgfqpoint{1.368280in}{0.544220in}}%
\pgfpathlineto{\pgfqpoint{1.372000in}{0.538948in}}%
\pgfpathlineto{\pgfqpoint{1.373240in}{0.538432in}}%
\pgfpathlineto{\pgfqpoint{1.376960in}{0.542480in}}%
\pgfpathlineto{\pgfqpoint{1.379440in}{0.542106in}}%
\pgfpathlineto{\pgfqpoint{1.381920in}{0.545573in}}%
\pgfpathlineto{\pgfqpoint{1.386880in}{0.540974in}}%
\pgfpathlineto{\pgfqpoint{1.391840in}{0.545395in}}%
\pgfpathlineto{\pgfqpoint{1.395560in}{0.543388in}}%
\pgfpathlineto{\pgfqpoint{1.396800in}{0.543714in}}%
\pgfpathlineto{\pgfqpoint{1.399280in}{0.542021in}}%
\pgfpathlineto{\pgfqpoint{1.404240in}{0.538615in}}%
\pgfpathlineto{\pgfqpoint{1.407960in}{0.542572in}}%
\pgfpathlineto{\pgfqpoint{1.412920in}{0.540075in}}%
\pgfpathlineto{\pgfqpoint{1.415400in}{0.540393in}}%
\pgfpathlineto{\pgfqpoint{1.416640in}{0.543193in}}%
\pgfpathlineto{\pgfqpoint{1.420360in}{0.541631in}}%
\pgfpathlineto{\pgfqpoint{1.421600in}{0.542473in}}%
\pgfpathlineto{\pgfqpoint{1.424080in}{0.541713in}}%
\pgfpathlineto{\pgfqpoint{1.425320in}{0.542693in}}%
\pgfpathlineto{\pgfqpoint{1.426560in}{0.541979in}}%
\pgfpathlineto{\pgfqpoint{1.432760in}{0.545122in}}%
\pgfpathlineto{\pgfqpoint{1.435240in}{0.539719in}}%
\pgfpathlineto{\pgfqpoint{1.436480in}{0.540611in}}%
\pgfpathlineto{\pgfqpoint{1.440200in}{0.535310in}}%
\pgfpathlineto{\pgfqpoint{1.442680in}{0.538225in}}%
\pgfpathlineto{\pgfqpoint{1.446400in}{0.539551in}}%
\pgfpathlineto{\pgfqpoint{1.447640in}{0.539138in}}%
\pgfpathlineto{\pgfqpoint{1.451360in}{0.541810in}}%
\pgfpathlineto{\pgfqpoint{1.452600in}{0.541062in}}%
\pgfpathlineto{\pgfqpoint{1.453840in}{0.542115in}}%
\pgfpathlineto{\pgfqpoint{1.456320in}{0.540287in}}%
\pgfpathlineto{\pgfqpoint{1.457560in}{0.540145in}}%
\pgfpathlineto{\pgfqpoint{1.460040in}{0.536222in}}%
\pgfpathlineto{\pgfqpoint{1.462520in}{0.537755in}}%
\pgfpathlineto{\pgfqpoint{1.465000in}{0.543172in}}%
\pgfpathlineto{\pgfqpoint{1.466240in}{0.542598in}}%
\pgfpathlineto{\pgfqpoint{1.467480in}{0.543339in}}%
\pgfpathlineto{\pgfqpoint{1.469960in}{0.539648in}}%
\pgfpathlineto{\pgfqpoint{1.473680in}{0.541787in}}%
\pgfpathlineto{\pgfqpoint{1.474920in}{0.540072in}}%
\pgfpathlineto{\pgfqpoint{1.482360in}{0.542388in}}%
\pgfpathlineto{\pgfqpoint{1.484840in}{0.545145in}}%
\pgfpathlineto{\pgfqpoint{1.486080in}{0.545270in}}%
\pgfpathlineto{\pgfqpoint{1.489800in}{0.543040in}}%
\pgfpathlineto{\pgfqpoint{1.492280in}{0.544939in}}%
\pgfpathlineto{\pgfqpoint{1.496000in}{0.539158in}}%
\pgfpathlineto{\pgfqpoint{1.497240in}{0.539577in}}%
\pgfpathlineto{\pgfqpoint{1.500960in}{0.544315in}}%
\pgfpathlineto{\pgfqpoint{1.503440in}{0.545434in}}%
\pgfpathlineto{\pgfqpoint{1.507160in}{0.547594in}}%
\pgfpathlineto{\pgfqpoint{1.510880in}{0.544649in}}%
\pgfpathlineto{\pgfqpoint{1.515840in}{0.546162in}}%
\pgfpathlineto{\pgfqpoint{1.519560in}{0.544896in}}%
\pgfpathlineto{\pgfqpoint{1.520800in}{0.545475in}}%
\pgfpathlineto{\pgfqpoint{1.523280in}{0.543212in}}%
\pgfpathlineto{\pgfqpoint{1.528240in}{0.539871in}}%
\pgfpathlineto{\pgfqpoint{1.531960in}{0.543601in}}%
\pgfpathlineto{\pgfqpoint{1.535680in}{0.543998in}}%
\pgfpathlineto{\pgfqpoint{1.538160in}{0.544571in}}%
\pgfpathlineto{\pgfqpoint{1.541880in}{0.545366in}}%
\pgfpathlineto{\pgfqpoint{1.544360in}{0.543434in}}%
\pgfpathlineto{\pgfqpoint{1.546840in}{0.543404in}}%
\pgfpathlineto{\pgfqpoint{1.548080in}{0.542912in}}%
\pgfpathlineto{\pgfqpoint{1.549320in}{0.543877in}}%
\pgfpathlineto{\pgfqpoint{1.550560in}{0.543359in}}%
\pgfpathlineto{\pgfqpoint{1.553040in}{0.544480in}}%
\pgfpathlineto{\pgfqpoint{1.555520in}{0.544538in}}%
\pgfpathlineto{\pgfqpoint{1.556760in}{0.545050in}}%
\pgfpathlineto{\pgfqpoint{1.559240in}{0.539799in}}%
\pgfpathlineto{\pgfqpoint{1.560480in}{0.540735in}}%
\pgfpathlineto{\pgfqpoint{1.564200in}{0.537189in}}%
\pgfpathlineto{\pgfqpoint{1.566680in}{0.539934in}}%
\pgfpathlineto{\pgfqpoint{1.571640in}{0.540094in}}%
\pgfpathlineto{\pgfqpoint{1.575360in}{0.542979in}}%
\pgfpathlineto{\pgfqpoint{1.576600in}{0.541476in}}%
\pgfpathlineto{\pgfqpoint{1.577840in}{0.542693in}}%
\pgfpathlineto{\pgfqpoint{1.580320in}{0.540851in}}%
\pgfpathlineto{\pgfqpoint{1.581560in}{0.540970in}}%
\pgfpathlineto{\pgfqpoint{1.584040in}{0.537754in}}%
\pgfpathlineto{\pgfqpoint{1.586520in}{0.539064in}}%
\pgfpathlineto{\pgfqpoint{1.589000in}{0.544562in}}%
\pgfpathlineto{\pgfqpoint{1.591480in}{0.544485in}}%
\pgfpathlineto{\pgfqpoint{1.592720in}{0.544043in}}%
\pgfpathlineto{\pgfqpoint{1.593960in}{0.542327in}}%
\pgfpathlineto{\pgfqpoint{1.595200in}{0.543033in}}%
\pgfpathlineto{\pgfqpoint{1.600160in}{0.540669in}}%
\pgfpathlineto{\pgfqpoint{1.606360in}{0.541136in}}%
\pgfpathlineto{\pgfqpoint{1.608840in}{0.544059in}}%
\pgfpathlineto{\pgfqpoint{1.610080in}{0.545180in}}%
\pgfpathlineto{\pgfqpoint{1.613800in}{0.543606in}}%
\pgfpathlineto{\pgfqpoint{1.616280in}{0.545113in}}%
\pgfpathlineto{\pgfqpoint{1.621240in}{0.540797in}}%
\pgfpathlineto{\pgfqpoint{1.624960in}{0.544808in}}%
\pgfpathlineto{\pgfqpoint{1.627440in}{0.544871in}}%
\pgfpathlineto{\pgfqpoint{1.629920in}{0.548954in}}%
\pgfpathlineto{\pgfqpoint{1.634880in}{0.545411in}}%
\pgfpathlineto{\pgfqpoint{1.638600in}{0.545679in}}%
\pgfpathlineto{\pgfqpoint{1.641080in}{0.544591in}}%
\pgfpathlineto{\pgfqpoint{1.643560in}{0.543787in}}%
\pgfpathlineto{\pgfqpoint{1.644800in}{0.544656in}}%
\pgfpathlineto{\pgfqpoint{1.647280in}{0.543392in}}%
\pgfpathlineto{\pgfqpoint{1.652240in}{0.540269in}}%
\pgfpathlineto{\pgfqpoint{1.655960in}{0.543127in}}%
\pgfpathlineto{\pgfqpoint{1.663400in}{0.544536in}}%
\pgfpathlineto{\pgfqpoint{1.664640in}{0.545831in}}%
\pgfpathlineto{\pgfqpoint{1.668360in}{0.543392in}}%
\pgfpathlineto{\pgfqpoint{1.669600in}{0.543702in}}%
\pgfpathlineto{\pgfqpoint{1.670840in}{0.542780in}}%
\pgfpathlineto{\pgfqpoint{1.674560in}{0.543938in}}%
\pgfpathlineto{\pgfqpoint{1.680760in}{0.544415in}}%
\pgfpathlineto{\pgfqpoint{1.683240in}{0.539303in}}%
\pgfpathlineto{\pgfqpoint{1.684480in}{0.540382in}}%
\pgfpathlineto{\pgfqpoint{1.688200in}{0.538442in}}%
\pgfpathlineto{\pgfqpoint{1.690680in}{0.540895in}}%
\pgfpathlineto{\pgfqpoint{1.701840in}{0.544225in}}%
\pgfpathlineto{\pgfqpoint{1.704320in}{0.542410in}}%
\pgfpathlineto{\pgfqpoint{1.705560in}{0.542076in}}%
\pgfpathlineto{\pgfqpoint{1.708040in}{0.538596in}}%
\pgfpathlineto{\pgfqpoint{1.710520in}{0.540226in}}%
\pgfpathlineto{\pgfqpoint{1.713000in}{0.544913in}}%
\pgfpathlineto{\pgfqpoint{1.714240in}{0.543866in}}%
\pgfpathlineto{\pgfqpoint{1.716720in}{0.544874in}}%
\pgfpathlineto{\pgfqpoint{1.717960in}{0.543420in}}%
\pgfpathlineto{\pgfqpoint{1.719200in}{0.544634in}}%
\pgfpathlineto{\pgfqpoint{1.725400in}{0.542699in}}%
\pgfpathlineto{\pgfqpoint{1.726640in}{0.543765in}}%
\pgfpathlineto{\pgfqpoint{1.729120in}{0.541970in}}%
\pgfpathlineto{\pgfqpoint{1.732840in}{0.545270in}}%
\pgfpathlineto{\pgfqpoint{1.734080in}{0.546474in}}%
\pgfpathlineto{\pgfqpoint{1.737800in}{0.544250in}}%
\pgfpathlineto{\pgfqpoint{1.740280in}{0.545381in}}%
\pgfpathlineto{\pgfqpoint{1.745240in}{0.542295in}}%
\pgfpathlineto{\pgfqpoint{1.748960in}{0.545794in}}%
\pgfpathlineto{\pgfqpoint{1.751440in}{0.545659in}}%
\pgfpathlineto{\pgfqpoint{1.753920in}{0.548341in}}%
\pgfpathlineto{\pgfqpoint{1.757640in}{0.545445in}}%
\pgfpathlineto{\pgfqpoint{1.760120in}{0.544645in}}%
\pgfpathlineto{\pgfqpoint{1.763840in}{0.544991in}}%
\pgfpathlineto{\pgfqpoint{1.766320in}{0.543141in}}%
\pgfpathlineto{\pgfqpoint{1.770040in}{0.544413in}}%
\pgfpathlineto{\pgfqpoint{1.772520in}{0.543721in}}%
\pgfpathlineto{\pgfqpoint{1.776240in}{0.541626in}}%
\pgfpathlineto{\pgfqpoint{1.778720in}{0.543731in}}%
\pgfpathlineto{\pgfqpoint{1.787400in}{0.544628in}}%
\pgfpathlineto{\pgfqpoint{1.788640in}{0.546696in}}%
\pgfpathlineto{\pgfqpoint{1.797320in}{0.544602in}}%
\pgfpathlineto{\pgfqpoint{1.801040in}{0.545285in}}%
\pgfpathlineto{\pgfqpoint{1.803520in}{0.544514in}}%
\pgfpathlineto{\pgfqpoint{1.804760in}{0.545281in}}%
\pgfpathlineto{\pgfqpoint{1.808480in}{0.540890in}}%
\pgfpathlineto{\pgfqpoint{1.812200in}{0.538696in}}%
\pgfpathlineto{\pgfqpoint{1.814680in}{0.541572in}}%
\pgfpathlineto{\pgfqpoint{1.819640in}{0.542322in}}%
\pgfpathlineto{\pgfqpoint{1.823360in}{0.545171in}}%
\pgfpathlineto{\pgfqpoint{1.824600in}{0.543747in}}%
\pgfpathlineto{\pgfqpoint{1.825840in}{0.545008in}}%
\pgfpathlineto{\pgfqpoint{1.828320in}{0.543070in}}%
\pgfpathlineto{\pgfqpoint{1.829560in}{0.542455in}}%
\pgfpathlineto{\pgfqpoint{1.832040in}{0.539361in}}%
\pgfpathlineto{\pgfqpoint{1.834520in}{0.541819in}}%
\pgfpathlineto{\pgfqpoint{1.837000in}{0.546432in}}%
\pgfpathlineto{\pgfqpoint{1.838240in}{0.545283in}}%
\pgfpathlineto{\pgfqpoint{1.840720in}{0.546450in}}%
\pgfpathlineto{\pgfqpoint{1.841960in}{0.545577in}}%
\pgfpathlineto{\pgfqpoint{1.843200in}{0.546418in}}%
\pgfpathlineto{\pgfqpoint{1.854360in}{0.542400in}}%
\pgfpathlineto{\pgfqpoint{1.856840in}{0.545197in}}%
\pgfpathlineto{\pgfqpoint{1.858080in}{0.546204in}}%
\pgfpathlineto{\pgfqpoint{1.861800in}{0.544412in}}%
\pgfpathlineto{\pgfqpoint{1.865520in}{0.544865in}}%
\pgfpathlineto{\pgfqpoint{1.868000in}{0.542362in}}%
\pgfpathlineto{\pgfqpoint{1.871720in}{0.546878in}}%
\pgfpathlineto{\pgfqpoint{1.874200in}{0.547401in}}%
\pgfpathlineto{\pgfqpoint{1.875440in}{0.546925in}}%
\pgfpathlineto{\pgfqpoint{1.877920in}{0.549944in}}%
\pgfpathlineto{\pgfqpoint{1.882880in}{0.547179in}}%
\pgfpathlineto{\pgfqpoint{1.887840in}{0.546927in}}%
\pgfpathlineto{\pgfqpoint{1.890320in}{0.544333in}}%
\pgfpathlineto{\pgfqpoint{1.891560in}{0.544624in}}%
\pgfpathlineto{\pgfqpoint{1.894040in}{0.546251in}}%
\pgfpathlineto{\pgfqpoint{1.900240in}{0.543881in}}%
\pgfpathlineto{\pgfqpoint{1.902720in}{0.546020in}}%
\pgfpathlineto{\pgfqpoint{1.907680in}{0.546485in}}%
\pgfpathlineto{\pgfqpoint{1.910160in}{0.546370in}}%
\pgfpathlineto{\pgfqpoint{1.911400in}{0.546579in}}%
\pgfpathlineto{\pgfqpoint{1.912640in}{0.548514in}}%
\pgfpathlineto{\pgfqpoint{1.916360in}{0.547445in}}%
\pgfpathlineto{\pgfqpoint{1.917600in}{0.548492in}}%
\pgfpathlineto{\pgfqpoint{1.920080in}{0.546960in}}%
\pgfpathlineto{\pgfqpoint{1.922560in}{0.546953in}}%
\pgfpathlineto{\pgfqpoint{1.925040in}{0.547971in}}%
\pgfpathlineto{\pgfqpoint{1.926280in}{0.546450in}}%
\pgfpathlineto{\pgfqpoint{1.928760in}{0.548082in}}%
\pgfpathlineto{\pgfqpoint{1.930000in}{0.546542in}}%
\pgfpathlineto{\pgfqpoint{1.931240in}{0.543030in}}%
\pgfpathlineto{\pgfqpoint{1.933720in}{0.543223in}}%
\pgfpathlineto{\pgfqpoint{1.936200in}{0.540718in}}%
\pgfpathlineto{\pgfqpoint{1.938680in}{0.542466in}}%
\pgfpathlineto{\pgfqpoint{1.944880in}{0.544619in}}%
\pgfpathlineto{\pgfqpoint{1.947360in}{0.546562in}}%
\pgfpathlineto{\pgfqpoint{1.948600in}{0.545508in}}%
\pgfpathlineto{\pgfqpoint{1.949840in}{0.546475in}}%
\pgfpathlineto{\pgfqpoint{1.952320in}{0.544873in}}%
\pgfpathlineto{\pgfqpoint{1.953560in}{0.544369in}}%
\pgfpathlineto{\pgfqpoint{1.956040in}{0.540957in}}%
\pgfpathlineto{\pgfqpoint{1.961000in}{0.547868in}}%
\pgfpathlineto{\pgfqpoint{1.962240in}{0.546403in}}%
\pgfpathlineto{\pgfqpoint{1.967200in}{0.547350in}}%
\pgfpathlineto{\pgfqpoint{1.972160in}{0.546168in}}%
\pgfpathlineto{\pgfqpoint{1.977120in}{0.543565in}}%
\pgfpathlineto{\pgfqpoint{1.978360in}{0.544292in}}%
\pgfpathlineto{\pgfqpoint{1.980840in}{0.547406in}}%
\pgfpathlineto{\pgfqpoint{1.982080in}{0.548036in}}%
\pgfpathlineto{\pgfqpoint{1.984560in}{0.546460in}}%
\pgfpathlineto{\pgfqpoint{1.985800in}{0.545993in}}%
\pgfpathlineto{\pgfqpoint{1.988280in}{0.548365in}}%
\pgfpathlineto{\pgfqpoint{1.992000in}{0.544900in}}%
\pgfpathlineto{\pgfqpoint{1.994480in}{0.547359in}}%
\pgfpathlineto{\pgfqpoint{1.998200in}{0.548900in}}%
\pgfpathlineto{\pgfqpoint{1.999440in}{0.548834in}}%
\pgfpathlineto{\pgfqpoint{2.001920in}{0.551914in}}%
\pgfpathlineto{\pgfqpoint{2.006880in}{0.548900in}}%
\pgfpathlineto{\pgfqpoint{2.011840in}{0.548377in}}%
\pgfpathlineto{\pgfqpoint{2.014320in}{0.545814in}}%
\pgfpathlineto{\pgfqpoint{2.020520in}{0.546696in}}%
\pgfpathlineto{\pgfqpoint{2.023000in}{0.544773in}}%
\pgfpathlineto{\pgfqpoint{2.030440in}{0.548022in}}%
\pgfpathlineto{\pgfqpoint{2.035400in}{0.547356in}}%
\pgfpathlineto{\pgfqpoint{2.036640in}{0.548963in}}%
\pgfpathlineto{\pgfqpoint{2.045320in}{0.546928in}}%
\pgfpathlineto{\pgfqpoint{2.049040in}{0.548237in}}%
\pgfpathlineto{\pgfqpoint{2.050280in}{0.546570in}}%
\pgfpathlineto{\pgfqpoint{2.052760in}{0.548062in}}%
\pgfpathlineto{\pgfqpoint{2.054000in}{0.546515in}}%
\pgfpathlineto{\pgfqpoint{2.055240in}{0.543060in}}%
\pgfpathlineto{\pgfqpoint{2.057720in}{0.543137in}}%
\pgfpathlineto{\pgfqpoint{2.060200in}{0.540998in}}%
\pgfpathlineto{\pgfqpoint{2.063920in}{0.543077in}}%
\pgfpathlineto{\pgfqpoint{2.066400in}{0.542987in}}%
\pgfpathlineto{\pgfqpoint{2.073840in}{0.544496in}}%
\pgfpathlineto{\pgfqpoint{2.076320in}{0.542888in}}%
\pgfpathlineto{\pgfqpoint{2.077560in}{0.542680in}}%
\pgfpathlineto{\pgfqpoint{2.080040in}{0.539469in}}%
\pgfpathlineto{\pgfqpoint{2.085000in}{0.546248in}}%
\pgfpathlineto{\pgfqpoint{2.086240in}{0.544668in}}%
\pgfpathlineto{\pgfqpoint{2.091200in}{0.545040in}}%
\pgfpathlineto{\pgfqpoint{2.096160in}{0.543972in}}%
\pgfpathlineto{\pgfqpoint{2.101120in}{0.541602in}}%
\pgfpathlineto{\pgfqpoint{2.106080in}{0.545830in}}%
\pgfpathlineto{\pgfqpoint{2.108560in}{0.544673in}}%
\pgfpathlineto{\pgfqpoint{2.109800in}{0.544330in}}%
\pgfpathlineto{\pgfqpoint{2.112280in}{0.546986in}}%
\pgfpathlineto{\pgfqpoint{2.116000in}{0.543852in}}%
\pgfpathlineto{\pgfqpoint{2.120960in}{0.548024in}}%
\pgfpathlineto{\pgfqpoint{2.123440in}{0.547250in}}%
\pgfpathlineto{\pgfqpoint{2.125920in}{0.550103in}}%
\pgfpathlineto{\pgfqpoint{2.130880in}{0.547661in}}%
\pgfpathlineto{\pgfqpoint{2.135840in}{0.547986in}}%
\pgfpathlineto{\pgfqpoint{2.139560in}{0.545397in}}%
\pgfpathlineto{\pgfqpoint{2.142040in}{0.545868in}}%
\pgfpathlineto{\pgfqpoint{2.148240in}{0.544896in}}%
\pgfpathlineto{\pgfqpoint{2.150720in}{0.546236in}}%
\pgfpathlineto{\pgfqpoint{2.154440in}{0.547829in}}%
\pgfpathlineto{\pgfqpoint{2.158160in}{0.547100in}}%
\pgfpathlineto{\pgfqpoint{2.159400in}{0.547349in}}%
\pgfpathlineto{\pgfqpoint{2.160640in}{0.549183in}}%
\pgfpathlineto{\pgfqpoint{2.175520in}{0.547773in}}%
\pgfpathlineto{\pgfqpoint{2.176760in}{0.548458in}}%
\pgfpathlineto{\pgfqpoint{2.178000in}{0.547308in}}%
\pgfpathlineto{\pgfqpoint{2.179240in}{0.543907in}}%
\pgfpathlineto{\pgfqpoint{2.181720in}{0.543889in}}%
\pgfpathlineto{\pgfqpoint{2.184200in}{0.542373in}}%
\pgfpathlineto{\pgfqpoint{2.187920in}{0.543452in}}%
\pgfpathlineto{\pgfqpoint{2.190400in}{0.543139in}}%
\pgfpathlineto{\pgfqpoint{2.197840in}{0.544076in}}%
\pgfpathlineto{\pgfqpoint{2.200320in}{0.542599in}}%
\pgfpathlineto{\pgfqpoint{2.202800in}{0.541026in}}%
\pgfpathlineto{\pgfqpoint{2.204040in}{0.540119in}}%
\pgfpathlineto{\pgfqpoint{2.209000in}{0.546582in}}%
\pgfpathlineto{\pgfqpoint{2.210240in}{0.545419in}}%
\pgfpathlineto{\pgfqpoint{2.213960in}{0.545974in}}%
\pgfpathlineto{\pgfqpoint{2.222640in}{0.544453in}}%
\pgfpathlineto{\pgfqpoint{2.225120in}{0.542465in}}%
\pgfpathlineto{\pgfqpoint{2.230080in}{0.545591in}}%
\pgfpathlineto{\pgfqpoint{2.232560in}{0.544096in}}%
\pgfpathlineto{\pgfqpoint{2.233800in}{0.544118in}}%
\pgfpathlineto{\pgfqpoint{2.236280in}{0.546348in}}%
\pgfpathlineto{\pgfqpoint{2.240000in}{0.543689in}}%
\pgfpathlineto{\pgfqpoint{2.243720in}{0.547560in}}%
\pgfpathlineto{\pgfqpoint{2.244960in}{0.548694in}}%
\pgfpathlineto{\pgfqpoint{2.247440in}{0.547758in}}%
\pgfpathlineto{\pgfqpoint{2.249920in}{0.550432in}}%
\pgfpathlineto{\pgfqpoint{2.254880in}{0.548950in}}%
\pgfpathlineto{\pgfqpoint{2.259840in}{0.548959in}}%
\pgfpathlineto{\pgfqpoint{2.262320in}{0.546933in}}%
\pgfpathlineto{\pgfqpoint{2.268520in}{0.546760in}}%
\pgfpathlineto{\pgfqpoint{2.271000in}{0.545377in}}%
\pgfpathlineto{\pgfqpoint{2.282160in}{0.548769in}}%
\pgfpathlineto{\pgfqpoint{2.283400in}{0.548442in}}%
\pgfpathlineto{\pgfqpoint{2.284640in}{0.550339in}}%
\pgfpathlineto{\pgfqpoint{2.292080in}{0.549170in}}%
\pgfpathlineto{\pgfqpoint{2.297040in}{0.550155in}}%
\pgfpathlineto{\pgfqpoint{2.298280in}{0.548388in}}%
\pgfpathlineto{\pgfqpoint{2.300760in}{0.550007in}}%
\pgfpathlineto{\pgfqpoint{2.302000in}{0.548796in}}%
\pgfpathlineto{\pgfqpoint{2.303240in}{0.545836in}}%
\pgfpathlineto{\pgfqpoint{2.304480in}{0.546498in}}%
\pgfpathlineto{\pgfqpoint{2.308200in}{0.544002in}}%
\pgfpathlineto{\pgfqpoint{2.311920in}{0.545469in}}%
\pgfpathlineto{\pgfqpoint{2.315640in}{0.544859in}}%
\pgfpathlineto{\pgfqpoint{2.319360in}{0.547343in}}%
\pgfpathlineto{\pgfqpoint{2.321840in}{0.545595in}}%
\pgfpathlineto{\pgfqpoint{2.326800in}{0.542683in}}%
\pgfpathlineto{\pgfqpoint{2.328040in}{0.542158in}}%
\pgfpathlineto{\pgfqpoint{2.333000in}{0.547417in}}%
\pgfpathlineto{\pgfqpoint{2.334240in}{0.546477in}}%
\pgfpathlineto{\pgfqpoint{2.337960in}{0.547563in}}%
\pgfpathlineto{\pgfqpoint{2.346640in}{0.545828in}}%
\pgfpathlineto{\pgfqpoint{2.349120in}{0.543783in}}%
\pgfpathlineto{\pgfqpoint{2.354080in}{0.547074in}}%
\pgfpathlineto{\pgfqpoint{2.356560in}{0.545899in}}%
\pgfpathlineto{\pgfqpoint{2.357800in}{0.545812in}}%
\pgfpathlineto{\pgfqpoint{2.361520in}{0.547634in}}%
\pgfpathlineto{\pgfqpoint{2.364000in}{0.545931in}}%
\pgfpathlineto{\pgfqpoint{2.368960in}{0.549718in}}%
\pgfpathlineto{\pgfqpoint{2.371440in}{0.548559in}}%
\pgfpathlineto{\pgfqpoint{2.375160in}{0.550941in}}%
\pgfpathlineto{\pgfqpoint{2.378880in}{0.551239in}}%
\pgfpathlineto{\pgfqpoint{2.383840in}{0.550569in}}%
\pgfpathlineto{\pgfqpoint{2.386320in}{0.548627in}}%
\pgfpathlineto{\pgfqpoint{2.390040in}{0.548873in}}%
\pgfpathlineto{\pgfqpoint{2.393760in}{0.547236in}}%
\pgfpathlineto{\pgfqpoint{2.396240in}{0.546907in}}%
\pgfpathlineto{\pgfqpoint{2.398720in}{0.548794in}}%
\pgfpathlineto{\pgfqpoint{2.404920in}{0.550348in}}%
\pgfpathlineto{\pgfqpoint{2.407400in}{0.549774in}}%
\pgfpathlineto{\pgfqpoint{2.408640in}{0.551643in}}%
\pgfpathlineto{\pgfqpoint{2.414840in}{0.550003in}}%
\pgfpathlineto{\pgfqpoint{2.421040in}{0.550944in}}%
\pgfpathlineto{\pgfqpoint{2.422280in}{0.549106in}}%
\pgfpathlineto{\pgfqpoint{2.424760in}{0.550862in}}%
\pgfpathlineto{\pgfqpoint{2.426000in}{0.550056in}}%
\pgfpathlineto{\pgfqpoint{2.427240in}{0.547344in}}%
\pgfpathlineto{\pgfqpoint{2.428480in}{0.548218in}}%
\pgfpathlineto{\pgfqpoint{2.433440in}{0.545790in}}%
\pgfpathlineto{\pgfqpoint{2.437160in}{0.546136in}}%
\pgfpathlineto{\pgfqpoint{2.440880in}{0.546846in}}%
\pgfpathlineto{\pgfqpoint{2.443360in}{0.547583in}}%
\pgfpathlineto{\pgfqpoint{2.444600in}{0.546180in}}%
\pgfpathlineto{\pgfqpoint{2.445840in}{0.546564in}}%
\pgfpathlineto{\pgfqpoint{2.449560in}{0.544996in}}%
\pgfpathlineto{\pgfqpoint{2.452040in}{0.543636in}}%
\pgfpathlineto{\pgfqpoint{2.457000in}{0.548132in}}%
\pgfpathlineto{\pgfqpoint{2.458240in}{0.546735in}}%
\pgfpathlineto{\pgfqpoint{2.464440in}{0.547165in}}%
\pgfpathlineto{\pgfqpoint{2.465680in}{0.547302in}}%
\pgfpathlineto{\pgfqpoint{2.466920in}{0.546026in}}%
\pgfpathlineto{\pgfqpoint{2.470640in}{0.546168in}}%
\pgfpathlineto{\pgfqpoint{2.473120in}{0.544444in}}%
\pgfpathlineto{\pgfqpoint{2.476840in}{0.546090in}}%
\pgfpathlineto{\pgfqpoint{2.478080in}{0.546861in}}%
\pgfpathlineto{\pgfqpoint{2.480560in}{0.545315in}}%
\pgfpathlineto{\pgfqpoint{2.483040in}{0.545983in}}%
\pgfpathlineto{\pgfqpoint{2.485520in}{0.547071in}}%
\pgfpathlineto{\pgfqpoint{2.490480in}{0.545051in}}%
\pgfpathlineto{\pgfqpoint{2.494200in}{0.546683in}}%
\pgfpathlineto{\pgfqpoint{2.495440in}{0.546383in}}%
\pgfpathlineto{\pgfqpoint{2.499160in}{0.548227in}}%
\pgfpathlineto{\pgfqpoint{2.500400in}{0.547780in}}%
\pgfpathlineto{\pgfqpoint{2.504120in}{0.548860in}}%
\pgfpathlineto{\pgfqpoint{2.506600in}{0.549006in}}%
\pgfpathlineto{\pgfqpoint{2.512800in}{0.546798in}}%
\pgfpathlineto{\pgfqpoint{2.520240in}{0.545177in}}%
\pgfpathlineto{\pgfqpoint{2.522720in}{0.546384in}}%
\pgfpathlineto{\pgfqpoint{2.530160in}{0.547880in}}%
\pgfpathlineto{\pgfqpoint{2.531400in}{0.547543in}}%
\pgfpathlineto{\pgfqpoint{2.532640in}{0.549601in}}%
\pgfpathlineto{\pgfqpoint{2.536360in}{0.548681in}}%
\pgfpathlineto{\pgfqpoint{2.541320in}{0.549840in}}%
\pgfpathlineto{\pgfqpoint{2.542560in}{0.550418in}}%
\pgfpathlineto{\pgfqpoint{2.546280in}{0.547741in}}%
\pgfpathlineto{\pgfqpoint{2.548760in}{0.549465in}}%
\pgfpathlineto{\pgfqpoint{2.550000in}{0.548825in}}%
\pgfpathlineto{\pgfqpoint{2.551240in}{0.546317in}}%
\pgfpathlineto{\pgfqpoint{2.552480in}{0.547519in}}%
\pgfpathlineto{\pgfqpoint{2.554960in}{0.545281in}}%
\pgfpathlineto{\pgfqpoint{2.556200in}{0.543981in}}%
\pgfpathlineto{\pgfqpoint{2.562400in}{0.545093in}}%
\pgfpathlineto{\pgfqpoint{2.569840in}{0.544838in}}%
\pgfpathlineto{\pgfqpoint{2.574800in}{0.542835in}}%
\pgfpathlineto{\pgfqpoint{2.576040in}{0.542238in}}%
\pgfpathlineto{\pgfqpoint{2.581000in}{0.547111in}}%
\pgfpathlineto{\pgfqpoint{2.583480in}{0.545617in}}%
\pgfpathlineto{\pgfqpoint{2.585960in}{0.546005in}}%
\pgfpathlineto{\pgfqpoint{2.589680in}{0.545713in}}%
\pgfpathlineto{\pgfqpoint{2.590920in}{0.544552in}}%
\pgfpathlineto{\pgfqpoint{2.593400in}{0.544773in}}%
\pgfpathlineto{\pgfqpoint{2.594640in}{0.545004in}}%
\pgfpathlineto{\pgfqpoint{2.597120in}{0.543112in}}%
\pgfpathlineto{\pgfqpoint{2.599600in}{0.545533in}}%
\pgfpathlineto{\pgfqpoint{2.600840in}{0.544558in}}%
\pgfpathlineto{\pgfqpoint{2.602080in}{0.545146in}}%
\pgfpathlineto{\pgfqpoint{2.604560in}{0.543771in}}%
\pgfpathlineto{\pgfqpoint{2.605800in}{0.543521in}}%
\pgfpathlineto{\pgfqpoint{2.609520in}{0.545746in}}%
\pgfpathlineto{\pgfqpoint{2.614480in}{0.544144in}}%
\pgfpathlineto{\pgfqpoint{2.616960in}{0.546128in}}%
\pgfpathlineto{\pgfqpoint{2.619440in}{0.545336in}}%
\pgfpathlineto{\pgfqpoint{2.623160in}{0.547144in}}%
\pgfpathlineto{\pgfqpoint{2.624400in}{0.546711in}}%
\pgfpathlineto{\pgfqpoint{2.628120in}{0.547907in}}%
\pgfpathlineto{\pgfqpoint{2.630600in}{0.548037in}}%
\pgfpathlineto{\pgfqpoint{2.636800in}{0.545542in}}%
\pgfpathlineto{\pgfqpoint{2.644240in}{0.544471in}}%
\pgfpathlineto{\pgfqpoint{2.646720in}{0.545535in}}%
\pgfpathlineto{\pgfqpoint{2.649200in}{0.546623in}}%
\pgfpathlineto{\pgfqpoint{2.650440in}{0.547847in}}%
\pgfpathlineto{\pgfqpoint{2.652920in}{0.548015in}}%
\pgfpathlineto{\pgfqpoint{2.655400in}{0.547551in}}%
\pgfpathlineto{\pgfqpoint{2.656640in}{0.549222in}}%
\pgfpathlineto{\pgfqpoint{2.660360in}{0.547774in}}%
\pgfpathlineto{\pgfqpoint{2.666560in}{0.549921in}}%
\pgfpathlineto{\pgfqpoint{2.671520in}{0.547887in}}%
\pgfpathlineto{\pgfqpoint{2.672760in}{0.548652in}}%
\pgfpathlineto{\pgfqpoint{2.674000in}{0.547987in}}%
\pgfpathlineto{\pgfqpoint{2.675240in}{0.545777in}}%
\pgfpathlineto{\pgfqpoint{2.676480in}{0.546864in}}%
\pgfpathlineto{\pgfqpoint{2.681440in}{0.543114in}}%
\pgfpathlineto{\pgfqpoint{2.686400in}{0.543830in}}%
\pgfpathlineto{\pgfqpoint{2.693840in}{0.542544in}}%
\pgfpathlineto{\pgfqpoint{2.696320in}{0.541650in}}%
\pgfpathlineto{\pgfqpoint{2.700040in}{0.540233in}}%
\pgfpathlineto{\pgfqpoint{2.705000in}{0.544823in}}%
\pgfpathlineto{\pgfqpoint{2.706240in}{0.543611in}}%
\pgfpathlineto{\pgfqpoint{2.711200in}{0.544399in}}%
\pgfpathlineto{\pgfqpoint{2.714920in}{0.543059in}}%
\pgfpathlineto{\pgfqpoint{2.716160in}{0.543725in}}%
\pgfpathlineto{\pgfqpoint{2.721120in}{0.541223in}}%
\pgfpathlineto{\pgfqpoint{2.724840in}{0.542729in}}%
\pgfpathlineto{\pgfqpoint{2.726080in}{0.543206in}}%
\pgfpathlineto{\pgfqpoint{2.728560in}{0.541970in}}%
\pgfpathlineto{\pgfqpoint{2.731040in}{0.542426in}}%
\pgfpathlineto{\pgfqpoint{2.733520in}{0.543853in}}%
\pgfpathlineto{\pgfqpoint{2.736000in}{0.542463in}}%
\pgfpathlineto{\pgfqpoint{2.737240in}{0.541485in}}%
\pgfpathlineto{\pgfqpoint{2.744680in}{0.545035in}}%
\pgfpathlineto{\pgfqpoint{2.745920in}{0.546572in}}%
\pgfpathlineto{\pgfqpoint{2.749640in}{0.546344in}}%
\pgfpathlineto{\pgfqpoint{2.755840in}{0.546102in}}%
\pgfpathlineto{\pgfqpoint{2.758320in}{0.543467in}}%
\pgfpathlineto{\pgfqpoint{2.765760in}{0.543502in}}%
\pgfpathlineto{\pgfqpoint{2.768240in}{0.542917in}}%
\pgfpathlineto{\pgfqpoint{2.770720in}{0.543971in}}%
\pgfpathlineto{\pgfqpoint{2.783120in}{0.545471in}}%
\pgfpathlineto{\pgfqpoint{2.789320in}{0.546604in}}%
\pgfpathlineto{\pgfqpoint{2.791800in}{0.546404in}}%
\pgfpathlineto{\pgfqpoint{2.793040in}{0.546627in}}%
\pgfpathlineto{\pgfqpoint{2.794280in}{0.544835in}}%
\pgfpathlineto{\pgfqpoint{2.798000in}{0.545750in}}%
\pgfpathlineto{\pgfqpoint{2.799240in}{0.543725in}}%
\pgfpathlineto{\pgfqpoint{2.800480in}{0.544934in}}%
\pgfpathlineto{\pgfqpoint{2.805440in}{0.540495in}}%
\pgfpathlineto{\pgfqpoint{2.807920in}{0.540695in}}%
\pgfpathlineto{\pgfqpoint{2.811640in}{0.540925in}}%
\pgfpathlineto{\pgfqpoint{2.814120in}{0.541896in}}%
\pgfpathlineto{\pgfqpoint{2.815360in}{0.542125in}}%
\pgfpathlineto{\pgfqpoint{2.819080in}{0.539703in}}%
\pgfpathlineto{\pgfqpoint{2.822800in}{0.539867in}}%
\pgfpathlineto{\pgfqpoint{2.824040in}{0.539347in}}%
\pgfpathlineto{\pgfqpoint{2.829000in}{0.543779in}}%
\pgfpathlineto{\pgfqpoint{2.830240in}{0.542758in}}%
\pgfpathlineto{\pgfqpoint{2.835200in}{0.544414in}}%
\pgfpathlineto{\pgfqpoint{2.838920in}{0.542895in}}%
\pgfpathlineto{\pgfqpoint{2.840160in}{0.543941in}}%
\pgfpathlineto{\pgfqpoint{2.845120in}{0.541410in}}%
\pgfpathlineto{\pgfqpoint{2.847600in}{0.543538in}}%
\pgfpathlineto{\pgfqpoint{2.851320in}{0.542185in}}%
\pgfpathlineto{\pgfqpoint{2.853800in}{0.541528in}}%
\pgfpathlineto{\pgfqpoint{2.858760in}{0.543410in}}%
\pgfpathlineto{\pgfqpoint{2.862480in}{0.541988in}}%
\pgfpathlineto{\pgfqpoint{2.866200in}{0.543827in}}%
\pgfpathlineto{\pgfqpoint{2.868680in}{0.544813in}}%
\pgfpathlineto{\pgfqpoint{2.869920in}{0.546183in}}%
\pgfpathlineto{\pgfqpoint{2.872400in}{0.545251in}}%
\pgfpathlineto{\pgfqpoint{2.874880in}{0.546196in}}%
\pgfpathlineto{\pgfqpoint{2.879840in}{0.546373in}}%
\pgfpathlineto{\pgfqpoint{2.882320in}{0.543783in}}%
\pgfpathlineto{\pgfqpoint{2.888520in}{0.544156in}}%
\pgfpathlineto{\pgfqpoint{2.892240in}{0.543145in}}%
\pgfpathlineto{\pgfqpoint{2.894720in}{0.544212in}}%
\pgfpathlineto{\pgfqpoint{2.897200in}{0.544770in}}%
\pgfpathlineto{\pgfqpoint{2.899680in}{0.545292in}}%
\pgfpathlineto{\pgfqpoint{2.902160in}{0.545667in}}%
\pgfpathlineto{\pgfqpoint{2.903400in}{0.545397in}}%
\pgfpathlineto{\pgfqpoint{2.904640in}{0.547014in}}%
\pgfpathlineto{\pgfqpoint{2.907120in}{0.544936in}}%
\pgfpathlineto{\pgfqpoint{2.913320in}{0.546913in}}%
\pgfpathlineto{\pgfqpoint{2.917040in}{0.547595in}}%
\pgfpathlineto{\pgfqpoint{2.918280in}{0.545792in}}%
\pgfpathlineto{\pgfqpoint{2.922000in}{0.546406in}}%
\pgfpathlineto{\pgfqpoint{2.923240in}{0.544591in}}%
\pgfpathlineto{\pgfqpoint{2.924480in}{0.545922in}}%
\pgfpathlineto{\pgfqpoint{2.929440in}{0.541937in}}%
\pgfpathlineto{\pgfqpoint{2.931920in}{0.541957in}}%
\pgfpathlineto{\pgfqpoint{2.935640in}{0.541612in}}%
\pgfpathlineto{\pgfqpoint{2.939360in}{0.542677in}}%
\pgfpathlineto{\pgfqpoint{2.943080in}{0.540414in}}%
\pgfpathlineto{\pgfqpoint{2.949280in}{0.541736in}}%
\pgfpathlineto{\pgfqpoint{2.953000in}{0.544492in}}%
\pgfpathlineto{\pgfqpoint{2.954240in}{0.543283in}}%
\pgfpathlineto{\pgfqpoint{2.959200in}{0.545300in}}%
\pgfpathlineto{\pgfqpoint{2.961680in}{0.544798in}}%
\pgfpathlineto{\pgfqpoint{2.962920in}{0.543953in}}%
\pgfpathlineto{\pgfqpoint{2.964160in}{0.545000in}}%
\pgfpathlineto{\pgfqpoint{2.969120in}{0.542340in}}%
\pgfpathlineto{\pgfqpoint{2.971600in}{0.544389in}}%
\pgfpathlineto{\pgfqpoint{2.972840in}{0.543598in}}%
\pgfpathlineto{\pgfqpoint{2.974080in}{0.544193in}}%
\pgfpathlineto{\pgfqpoint{2.977800in}{0.542408in}}%
\pgfpathlineto{\pgfqpoint{2.984000in}{0.543572in}}%
\pgfpathlineto{\pgfqpoint{2.985240in}{0.542376in}}%
\pgfpathlineto{\pgfqpoint{3.000120in}{0.547891in}}%
\pgfpathlineto{\pgfqpoint{3.003840in}{0.547712in}}%
\pgfpathlineto{\pgfqpoint{3.006320in}{0.545373in}}%
\pgfpathlineto{\pgfqpoint{3.008800in}{0.546354in}}%
\pgfpathlineto{\pgfqpoint{3.015000in}{0.543463in}}%
\pgfpathlineto{\pgfqpoint{3.024920in}{0.546652in}}%
\pgfpathlineto{\pgfqpoint{3.027400in}{0.545969in}}%
\pgfpathlineto{\pgfqpoint{3.028640in}{0.547243in}}%
\pgfpathlineto{\pgfqpoint{3.031120in}{0.545432in}}%
\pgfpathlineto{\pgfqpoint{3.041040in}{0.548063in}}%
\pgfpathlineto{\pgfqpoint{3.043520in}{0.546429in}}%
\pgfpathlineto{\pgfqpoint{3.046000in}{0.546976in}}%
\pgfpathlineto{\pgfqpoint{3.047240in}{0.545379in}}%
\pgfpathlineto{\pgfqpoint{3.048480in}{0.546780in}}%
\pgfpathlineto{\pgfqpoint{3.053440in}{0.542678in}}%
\pgfpathlineto{\pgfqpoint{3.057160in}{0.542416in}}%
\pgfpathlineto{\pgfqpoint{3.060880in}{0.542776in}}%
\pgfpathlineto{\pgfqpoint{3.063360in}{0.543275in}}%
\pgfpathlineto{\pgfqpoint{3.067080in}{0.541008in}}%
\pgfpathlineto{\pgfqpoint{3.070800in}{0.541352in}}%
\pgfpathlineto{\pgfqpoint{3.072040in}{0.540670in}}%
\pgfpathlineto{\pgfqpoint{3.077000in}{0.544580in}}%
\pgfpathlineto{\pgfqpoint{3.079480in}{0.543583in}}%
\pgfpathlineto{\pgfqpoint{3.081960in}{0.544172in}}%
\pgfpathlineto{\pgfqpoint{3.083200in}{0.544593in}}%
\pgfpathlineto{\pgfqpoint{3.085680in}{0.543884in}}%
\pgfpathlineto{\pgfqpoint{3.086920in}{0.543148in}}%
\pgfpathlineto{\pgfqpoint{3.088160in}{0.544170in}}%
\pgfpathlineto{\pgfqpoint{3.093120in}{0.541359in}}%
\pgfpathlineto{\pgfqpoint{3.098080in}{0.544124in}}%
\pgfpathlineto{\pgfqpoint{3.101800in}{0.542638in}}%
\pgfpathlineto{\pgfqpoint{3.106760in}{0.544175in}}%
\pgfpathlineto{\pgfqpoint{3.110480in}{0.543609in}}%
\pgfpathlineto{\pgfqpoint{3.117920in}{0.547963in}}%
\pgfpathlineto{\pgfqpoint{3.120400in}{0.547127in}}%
\pgfpathlineto{\pgfqpoint{3.122880in}{0.548302in}}%
\pgfpathlineto{\pgfqpoint{3.127840in}{0.548438in}}%
\pgfpathlineto{\pgfqpoint{3.130320in}{0.546007in}}%
\pgfpathlineto{\pgfqpoint{3.132800in}{0.547158in}}%
\pgfpathlineto{\pgfqpoint{3.139000in}{0.543422in}}%
\pgfpathlineto{\pgfqpoint{3.143960in}{0.545414in}}%
\pgfpathlineto{\pgfqpoint{3.152640in}{0.547614in}}%
\pgfpathlineto{\pgfqpoint{3.156360in}{0.545791in}}%
\pgfpathlineto{\pgfqpoint{3.160080in}{0.547843in}}%
\pgfpathlineto{\pgfqpoint{3.162560in}{0.548598in}}%
\pgfpathlineto{\pgfqpoint{3.167520in}{0.546550in}}%
\pgfpathlineto{\pgfqpoint{3.170000in}{0.547187in}}%
\pgfpathlineto{\pgfqpoint{3.171240in}{0.545706in}}%
\pgfpathlineto{\pgfqpoint{3.172480in}{0.547225in}}%
\pgfpathlineto{\pgfqpoint{3.177440in}{0.543554in}}%
\pgfpathlineto{\pgfqpoint{3.179920in}{0.543125in}}%
\pgfpathlineto{\pgfqpoint{3.183640in}{0.542160in}}%
\pgfpathlineto{\pgfqpoint{3.187360in}{0.542973in}}%
\pgfpathlineto{\pgfqpoint{3.191080in}{0.541212in}}%
\pgfpathlineto{\pgfqpoint{3.193560in}{0.541462in}}%
\pgfpathlineto{\pgfqpoint{3.197280in}{0.541821in}}%
\pgfpathlineto{\pgfqpoint{3.201000in}{0.545348in}}%
\pgfpathlineto{\pgfqpoint{3.202240in}{0.544214in}}%
\pgfpathlineto{\pgfqpoint{3.207200in}{0.545913in}}%
\pgfpathlineto{\pgfqpoint{3.210920in}{0.543899in}}%
\pgfpathlineto{\pgfqpoint{3.212160in}{0.544873in}}%
\pgfpathlineto{\pgfqpoint{3.213400in}{0.544059in}}%
\pgfpathlineto{\pgfqpoint{3.214640in}{0.544624in}}%
\pgfpathlineto{\pgfqpoint{3.217120in}{0.542257in}}%
\pgfpathlineto{\pgfqpoint{3.222080in}{0.544722in}}%
\pgfpathlineto{\pgfqpoint{3.224560in}{0.543771in}}%
\pgfpathlineto{\pgfqpoint{3.227040in}{0.543831in}}%
\pgfpathlineto{\pgfqpoint{3.229520in}{0.545433in}}%
\pgfpathlineto{\pgfqpoint{3.233240in}{0.543889in}}%
\pgfpathlineto{\pgfqpoint{3.235720in}{0.545433in}}%
\pgfpathlineto{\pgfqpoint{3.238200in}{0.547091in}}%
\pgfpathlineto{\pgfqpoint{3.248120in}{0.549311in}}%
\pgfpathlineto{\pgfqpoint{3.251840in}{0.548771in}}%
\pgfpathlineto{\pgfqpoint{3.254320in}{0.546498in}}%
\pgfpathlineto{\pgfqpoint{3.256800in}{0.547775in}}%
\pgfpathlineto{\pgfqpoint{3.263000in}{0.544228in}}%
\pgfpathlineto{\pgfqpoint{3.272920in}{0.548161in}}%
\pgfpathlineto{\pgfqpoint{3.277880in}{0.547863in}}%
\pgfpathlineto{\pgfqpoint{3.280360in}{0.547100in}}%
\pgfpathlineto{\pgfqpoint{3.284080in}{0.549299in}}%
\pgfpathlineto{\pgfqpoint{3.286560in}{0.550007in}}%
\pgfpathlineto{\pgfqpoint{3.291520in}{0.547489in}}%
\pgfpathlineto{\pgfqpoint{3.294000in}{0.548144in}}%
\pgfpathlineto{\pgfqpoint{3.295240in}{0.546521in}}%
\pgfpathlineto{\pgfqpoint{3.296480in}{0.547972in}}%
\pgfpathlineto{\pgfqpoint{3.301440in}{0.544871in}}%
\pgfpathlineto{\pgfqpoint{3.303920in}{0.544406in}}%
\pgfpathlineto{\pgfqpoint{3.307640in}{0.543415in}}%
\pgfpathlineto{\pgfqpoint{3.311360in}{0.544316in}}%
\pgfpathlineto{\pgfqpoint{3.315080in}{0.541985in}}%
\pgfpathlineto{\pgfqpoint{3.318800in}{0.542324in}}%
\pgfpathlineto{\pgfqpoint{3.320040in}{0.541093in}}%
\pgfpathlineto{\pgfqpoint{3.325000in}{0.545641in}}%
\pgfpathlineto{\pgfqpoint{3.326240in}{0.544539in}}%
\pgfpathlineto{\pgfqpoint{3.331200in}{0.546392in}}%
\pgfpathlineto{\pgfqpoint{3.334920in}{0.544532in}}%
\pgfpathlineto{\pgfqpoint{3.336160in}{0.545407in}}%
\pgfpathlineto{\pgfqpoint{3.341120in}{0.542811in}}%
\pgfpathlineto{\pgfqpoint{3.346080in}{0.545431in}}%
\pgfpathlineto{\pgfqpoint{3.349800in}{0.543818in}}%
\pgfpathlineto{\pgfqpoint{3.351040in}{0.544149in}}%
\pgfpathlineto{\pgfqpoint{3.352280in}{0.545881in}}%
\pgfpathlineto{\pgfqpoint{3.358480in}{0.546001in}}%
\pgfpathlineto{\pgfqpoint{3.362200in}{0.548819in}}%
\pgfpathlineto{\pgfqpoint{3.370880in}{0.551511in}}%
\pgfpathlineto{\pgfqpoint{3.375840in}{0.550647in}}%
\pgfpathlineto{\pgfqpoint{3.378320in}{0.548476in}}%
\pgfpathlineto{\pgfqpoint{3.380800in}{0.549889in}}%
\pgfpathlineto{\pgfqpoint{3.387000in}{0.546954in}}%
\pgfpathlineto{\pgfqpoint{3.391960in}{0.548912in}}%
\pgfpathlineto{\pgfqpoint{3.395680in}{0.549919in}}%
\pgfpathlineto{\pgfqpoint{3.398160in}{0.550589in}}%
\pgfpathlineto{\pgfqpoint{3.399400in}{0.550373in}}%
\pgfpathlineto{\pgfqpoint{3.400640in}{0.551469in}}%
\pgfpathlineto{\pgfqpoint{3.404360in}{0.549301in}}%
\pgfpathlineto{\pgfqpoint{3.408080in}{0.551448in}}%
\pgfpathlineto{\pgfqpoint{3.410560in}{0.552103in}}%
\pgfpathlineto{\pgfqpoint{3.415520in}{0.549596in}}%
\pgfpathlineto{\pgfqpoint{3.418000in}{0.550234in}}%
\pgfpathlineto{\pgfqpoint{3.419240in}{0.548661in}}%
\pgfpathlineto{\pgfqpoint{3.420480in}{0.550332in}}%
\pgfpathlineto{\pgfqpoint{3.425440in}{0.546426in}}%
\pgfpathlineto{\pgfqpoint{3.435360in}{0.545752in}}%
\pgfpathlineto{\pgfqpoint{3.437840in}{0.544164in}}%
\pgfpathlineto{\pgfqpoint{3.439080in}{0.543421in}}%
\pgfpathlineto{\pgfqpoint{3.441560in}{0.543995in}}%
\pgfpathlineto{\pgfqpoint{3.445280in}{0.544300in}}%
\pgfpathlineto{\pgfqpoint{3.447760in}{0.546115in}}%
\pgfpathlineto{\pgfqpoint{3.449000in}{0.547620in}}%
\pgfpathlineto{\pgfqpoint{3.451480in}{0.547239in}}%
\pgfpathlineto{\pgfqpoint{3.452720in}{0.548475in}}%
\pgfpathlineto{\pgfqpoint{3.453960in}{0.547811in}}%
\pgfpathlineto{\pgfqpoint{3.456440in}{0.548149in}}%
\pgfpathlineto{\pgfqpoint{3.463880in}{0.546934in}}%
\pgfpathlineto{\pgfqpoint{3.465120in}{0.545687in}}%
\pgfpathlineto{\pgfqpoint{3.470080in}{0.548443in}}%
\pgfpathlineto{\pgfqpoint{3.473800in}{0.546946in}}%
\pgfpathlineto{\pgfqpoint{3.475040in}{0.547214in}}%
\pgfpathlineto{\pgfqpoint{3.476280in}{0.548746in}}%
\pgfpathlineto{\pgfqpoint{3.480000in}{0.547427in}}%
\pgfpathlineto{\pgfqpoint{3.486200in}{0.552963in}}%
\pgfpathlineto{\pgfqpoint{3.488680in}{0.554066in}}%
\pgfpathlineto{\pgfqpoint{3.491160in}{0.554663in}}%
\pgfpathlineto{\pgfqpoint{3.498600in}{0.555652in}}%
\pgfpathlineto{\pgfqpoint{3.511000in}{0.551215in}}%
\pgfpathlineto{\pgfqpoint{3.522160in}{0.555763in}}%
\pgfpathlineto{\pgfqpoint{3.528360in}{0.554120in}}%
\pgfpathlineto{\pgfqpoint{3.532080in}{0.556041in}}%
\pgfpathlineto{\pgfqpoint{3.534560in}{0.557056in}}%
\pgfpathlineto{\pgfqpoint{3.543240in}{0.553951in}}%
\pgfpathlineto{\pgfqpoint{3.544480in}{0.555378in}}%
\pgfpathlineto{\pgfqpoint{3.549440in}{0.551232in}}%
\pgfpathlineto{\pgfqpoint{3.561840in}{0.549453in}}%
\pgfpathlineto{\pgfqpoint{3.563080in}{0.548700in}}%
\pgfpathlineto{\pgfqpoint{3.565560in}{0.549851in}}%
\pgfpathlineto{\pgfqpoint{3.568040in}{0.548425in}}%
\pgfpathlineto{\pgfqpoint{3.573000in}{0.553010in}}%
\pgfpathlineto{\pgfqpoint{3.575480in}{0.552587in}}%
\pgfpathlineto{\pgfqpoint{3.576720in}{0.553723in}}%
\pgfpathlineto{\pgfqpoint{3.577960in}{0.553254in}}%
\pgfpathlineto{\pgfqpoint{3.579200in}{0.554107in}}%
\pgfpathlineto{\pgfqpoint{3.582920in}{0.552316in}}%
\pgfpathlineto{\pgfqpoint{3.584160in}{0.553367in}}%
\pgfpathlineto{\pgfqpoint{3.589120in}{0.551341in}}%
\pgfpathlineto{\pgfqpoint{3.591600in}{0.553542in}}%
\pgfpathlineto{\pgfqpoint{3.592840in}{0.552949in}}%
\pgfpathlineto{\pgfqpoint{3.594080in}{0.553869in}}%
\pgfpathlineto{\pgfqpoint{3.597800in}{0.552470in}}%
\pgfpathlineto{\pgfqpoint{3.599040in}{0.552668in}}%
\pgfpathlineto{\pgfqpoint{3.601520in}{0.554354in}}%
\pgfpathlineto{\pgfqpoint{3.604000in}{0.553418in}}%
\pgfpathlineto{\pgfqpoint{3.607720in}{0.556806in}}%
\pgfpathlineto{\pgfqpoint{3.611440in}{0.558618in}}%
\pgfpathlineto{\pgfqpoint{3.620120in}{0.561581in}}%
\pgfpathlineto{\pgfqpoint{3.622600in}{0.561253in}}%
\pgfpathlineto{\pgfqpoint{3.631280in}{0.558109in}}%
\pgfpathlineto{\pgfqpoint{3.633760in}{0.557380in}}%
\pgfpathlineto{\pgfqpoint{3.635000in}{0.556389in}}%
\pgfpathlineto{\pgfqpoint{3.646160in}{0.561073in}}%
\pgfpathlineto{\pgfqpoint{3.649880in}{0.561035in}}%
\pgfpathlineto{\pgfqpoint{3.652360in}{0.559482in}}%
\pgfpathlineto{\pgfqpoint{3.656080in}{0.561641in}}%
\pgfpathlineto{\pgfqpoint{3.661040in}{0.562620in}}%
\pgfpathlineto{\pgfqpoint{3.663520in}{0.560315in}}%
\pgfpathlineto{\pgfqpoint{3.666000in}{0.561284in}}%
\pgfpathlineto{\pgfqpoint{3.667240in}{0.559654in}}%
\pgfpathlineto{\pgfqpoint{3.668480in}{0.560987in}}%
\pgfpathlineto{\pgfqpoint{3.673440in}{0.557196in}}%
\pgfpathlineto{\pgfqpoint{3.679640in}{0.556363in}}%
\pgfpathlineto{\pgfqpoint{3.692040in}{0.553892in}}%
\pgfpathlineto{\pgfqpoint{3.695760in}{0.557070in}}%
\pgfpathlineto{\pgfqpoint{3.697000in}{0.558468in}}%
\pgfpathlineto{\pgfqpoint{3.699480in}{0.558117in}}%
\pgfpathlineto{\pgfqpoint{3.703200in}{0.559413in}}%
\pgfpathlineto{\pgfqpoint{3.706920in}{0.557599in}}%
\pgfpathlineto{\pgfqpoint{3.708160in}{0.558812in}}%
\pgfpathlineto{\pgfqpoint{3.713120in}{0.556949in}}%
\pgfpathlineto{\pgfqpoint{3.715600in}{0.558918in}}%
\pgfpathlineto{\pgfqpoint{3.716840in}{0.558488in}}%
\pgfpathlineto{\pgfqpoint{3.718080in}{0.559485in}}%
\pgfpathlineto{\pgfqpoint{3.719320in}{0.558455in}}%
\pgfpathlineto{\pgfqpoint{3.720560in}{0.558936in}}%
\pgfpathlineto{\pgfqpoint{3.723040in}{0.558199in}}%
\pgfpathlineto{\pgfqpoint{3.725520in}{0.560047in}}%
\pgfpathlineto{\pgfqpoint{3.728000in}{0.558804in}}%
\pgfpathlineto{\pgfqpoint{3.731720in}{0.562079in}}%
\pgfpathlineto{\pgfqpoint{3.737920in}{0.565560in}}%
\pgfpathlineto{\pgfqpoint{3.740400in}{0.565919in}}%
\pgfpathlineto{\pgfqpoint{3.742880in}{0.567056in}}%
\pgfpathlineto{\pgfqpoint{3.746600in}{0.566944in}}%
\pgfpathlineto{\pgfqpoint{3.751560in}{0.564796in}}%
\pgfpathlineto{\pgfqpoint{3.754040in}{0.564517in}}%
\pgfpathlineto{\pgfqpoint{3.760240in}{0.562626in}}%
\pgfpathlineto{\pgfqpoint{3.762720in}{0.564724in}}%
\pgfpathlineto{\pgfqpoint{3.765200in}{0.564985in}}%
\pgfpathlineto{\pgfqpoint{3.768920in}{0.566396in}}%
\pgfpathlineto{\pgfqpoint{3.771400in}{0.566233in}}%
\pgfpathlineto{\pgfqpoint{3.772640in}{0.567287in}}%
\pgfpathlineto{\pgfqpoint{3.776360in}{0.564118in}}%
\pgfpathlineto{\pgfqpoint{3.781320in}{0.567840in}}%
\pgfpathlineto{\pgfqpoint{3.785040in}{0.568388in}}%
\pgfpathlineto{\pgfqpoint{3.787520in}{0.565856in}}%
\pgfpathlineto{\pgfqpoint{3.790000in}{0.567275in}}%
\pgfpathlineto{\pgfqpoint{3.791240in}{0.565581in}}%
\pgfpathlineto{\pgfqpoint{3.792480in}{0.566998in}}%
\pgfpathlineto{\pgfqpoint{3.796200in}{0.562882in}}%
\pgfpathlineto{\pgfqpoint{3.798680in}{0.562913in}}%
\pgfpathlineto{\pgfqpoint{3.802400in}{0.561223in}}%
\pgfpathlineto{\pgfqpoint{3.807360in}{0.562002in}}%
\pgfpathlineto{\pgfqpoint{3.809840in}{0.559841in}}%
\pgfpathlineto{\pgfqpoint{3.811080in}{0.559052in}}%
\pgfpathlineto{\pgfqpoint{3.813560in}{0.560587in}}%
\pgfpathlineto{\pgfqpoint{3.816040in}{0.559214in}}%
\pgfpathlineto{\pgfqpoint{3.818520in}{0.561497in}}%
\pgfpathlineto{\pgfqpoint{3.824720in}{0.564255in}}%
\pgfpathlineto{\pgfqpoint{3.825960in}{0.563812in}}%
\pgfpathlineto{\pgfqpoint{3.828440in}{0.564527in}}%
\pgfpathlineto{\pgfqpoint{3.833400in}{0.564239in}}%
\pgfpathlineto{\pgfqpoint{3.834640in}{0.565049in}}%
\pgfpathlineto{\pgfqpoint{3.837120in}{0.562929in}}%
\pgfpathlineto{\pgfqpoint{3.839600in}{0.564978in}}%
\pgfpathlineto{\pgfqpoint{3.843320in}{0.564790in}}%
\pgfpathlineto{\pgfqpoint{3.845800in}{0.565024in}}%
\pgfpathlineto{\pgfqpoint{3.847040in}{0.565126in}}%
\pgfpathlineto{\pgfqpoint{3.849520in}{0.566674in}}%
\pgfpathlineto{\pgfqpoint{3.852000in}{0.565322in}}%
\pgfpathlineto{\pgfqpoint{3.860680in}{0.569996in}}%
\pgfpathlineto{\pgfqpoint{3.866880in}{0.572145in}}%
\pgfpathlineto{\pgfqpoint{3.870600in}{0.571608in}}%
\pgfpathlineto{\pgfqpoint{3.874320in}{0.568319in}}%
\pgfpathlineto{\pgfqpoint{3.876800in}{0.569655in}}%
\pgfpathlineto{\pgfqpoint{3.883000in}{0.566441in}}%
\pgfpathlineto{\pgfqpoint{3.892920in}{0.570458in}}%
\pgfpathlineto{\pgfqpoint{3.895400in}{0.570123in}}%
\pgfpathlineto{\pgfqpoint{3.896640in}{0.571229in}}%
\pgfpathlineto{\pgfqpoint{3.900360in}{0.568267in}}%
\pgfpathlineto{\pgfqpoint{3.904080in}{0.570785in}}%
\pgfpathlineto{\pgfqpoint{3.906560in}{0.571923in}}%
\pgfpathlineto{\pgfqpoint{3.909040in}{0.572475in}}%
\pgfpathlineto{\pgfqpoint{3.911520in}{0.569956in}}%
\pgfpathlineto{\pgfqpoint{3.914000in}{0.571009in}}%
\pgfpathlineto{\pgfqpoint{3.915240in}{0.569362in}}%
\pgfpathlineto{\pgfqpoint{3.916480in}{0.570775in}}%
\pgfpathlineto{\pgfqpoint{3.920200in}{0.566976in}}%
\pgfpathlineto{\pgfqpoint{3.922680in}{0.567206in}}%
\pgfpathlineto{\pgfqpoint{3.926400in}{0.565571in}}%
\pgfpathlineto{\pgfqpoint{3.931360in}{0.565516in}}%
\pgfpathlineto{\pgfqpoint{3.933840in}{0.563496in}}%
\pgfpathlineto{\pgfqpoint{3.935080in}{0.562600in}}%
\pgfpathlineto{\pgfqpoint{3.938800in}{0.563572in}}%
\pgfpathlineto{\pgfqpoint{3.940040in}{0.562580in}}%
\pgfpathlineto{\pgfqpoint{3.942520in}{0.564970in}}%
\pgfpathlineto{\pgfqpoint{3.948720in}{0.566891in}}%
\pgfpathlineto{\pgfqpoint{3.957400in}{0.566644in}}%
\pgfpathlineto{\pgfqpoint{3.958640in}{0.566980in}}%
\pgfpathlineto{\pgfqpoint{3.961120in}{0.565088in}}%
\pgfpathlineto{\pgfqpoint{3.963600in}{0.567180in}}%
\pgfpathlineto{\pgfqpoint{3.967320in}{0.566942in}}%
\pgfpathlineto{\pgfqpoint{3.968560in}{0.567769in}}%
\pgfpathlineto{\pgfqpoint{3.971040in}{0.566834in}}%
\pgfpathlineto{\pgfqpoint{3.973520in}{0.568027in}}%
\pgfpathlineto{\pgfqpoint{3.977240in}{0.568166in}}%
\pgfpathlineto{\pgfqpoint{3.984680in}{0.572095in}}%
\pgfpathlineto{\pgfqpoint{3.992120in}{0.574240in}}%
\pgfpathlineto{\pgfqpoint{3.994600in}{0.573612in}}%
\pgfpathlineto{\pgfqpoint{3.998320in}{0.570957in}}%
\pgfpathlineto{\pgfqpoint{4.000800in}{0.572173in}}%
\pgfpathlineto{\pgfqpoint{4.008240in}{0.569180in}}%
\pgfpathlineto{\pgfqpoint{4.010720in}{0.570899in}}%
\pgfpathlineto{\pgfqpoint{4.013200in}{0.571312in}}%
\pgfpathlineto{\pgfqpoint{4.015680in}{0.571725in}}%
\pgfpathlineto{\pgfqpoint{4.020640in}{0.573461in}}%
\pgfpathlineto{\pgfqpoint{4.024360in}{0.570803in}}%
\pgfpathlineto{\pgfqpoint{4.029320in}{0.573915in}}%
\pgfpathlineto{\pgfqpoint{4.033040in}{0.574523in}}%
\pgfpathlineto{\pgfqpoint{4.035520in}{0.572622in}}%
\pgfpathlineto{\pgfqpoint{4.038000in}{0.574097in}}%
\pgfpathlineto{\pgfqpoint{4.039240in}{0.572491in}}%
\pgfpathlineto{\pgfqpoint{4.040480in}{0.573768in}}%
\pgfpathlineto{\pgfqpoint{4.044200in}{0.569263in}}%
\pgfpathlineto{\pgfqpoint{4.046680in}{0.569677in}}%
\pgfpathlineto{\pgfqpoint{4.050400in}{0.567646in}}%
\pgfpathlineto{\pgfqpoint{4.055360in}{0.568196in}}%
\pgfpathlineto{\pgfqpoint{4.057840in}{0.566252in}}%
\pgfpathlineto{\pgfqpoint{4.059080in}{0.565186in}}%
\pgfpathlineto{\pgfqpoint{4.062800in}{0.566613in}}%
\pgfpathlineto{\pgfqpoint{4.064040in}{0.565631in}}%
\pgfpathlineto{\pgfqpoint{4.067760in}{0.568589in}}%
\pgfpathlineto{\pgfqpoint{4.069000in}{0.569910in}}%
\pgfpathlineto{\pgfqpoint{4.071480in}{0.569613in}}%
\pgfpathlineto{\pgfqpoint{4.073960in}{0.570120in}}%
\pgfpathlineto{\pgfqpoint{4.076440in}{0.569465in}}%
\pgfpathlineto{\pgfqpoint{4.078920in}{0.567691in}}%
\pgfpathlineto{\pgfqpoint{4.081400in}{0.569167in}}%
\pgfpathlineto{\pgfqpoint{4.082640in}{0.569760in}}%
\pgfpathlineto{\pgfqpoint{4.086360in}{0.568456in}}%
\pgfpathlineto{\pgfqpoint{4.090080in}{0.570433in}}%
\pgfpathlineto{\pgfqpoint{4.091320in}{0.569252in}}%
\pgfpathlineto{\pgfqpoint{4.092560in}{0.570045in}}%
\pgfpathlineto{\pgfqpoint{4.095040in}{0.568907in}}%
\pgfpathlineto{\pgfqpoint{4.097520in}{0.569851in}}%
\pgfpathlineto{\pgfqpoint{4.101240in}{0.570086in}}%
\pgfpathlineto{\pgfqpoint{4.107440in}{0.573495in}}%
\pgfpathlineto{\pgfqpoint{4.118600in}{0.576202in}}%
\pgfpathlineto{\pgfqpoint{4.122320in}{0.573745in}}%
\pgfpathlineto{\pgfqpoint{4.126040in}{0.573969in}}%
\pgfpathlineto{\pgfqpoint{4.129760in}{0.572143in}}%
\pgfpathlineto{\pgfqpoint{4.132240in}{0.571245in}}%
\pgfpathlineto{\pgfqpoint{4.134720in}{0.573024in}}%
\pgfpathlineto{\pgfqpoint{4.137200in}{0.573103in}}%
\pgfpathlineto{\pgfqpoint{4.138440in}{0.573814in}}%
\pgfpathlineto{\pgfqpoint{4.142160in}{0.572826in}}%
\pgfpathlineto{\pgfqpoint{4.143400in}{0.572470in}}%
\pgfpathlineto{\pgfqpoint{4.144640in}{0.573435in}}%
\pgfpathlineto{\pgfqpoint{4.147120in}{0.570516in}}%
\pgfpathlineto{\pgfqpoint{4.150840in}{0.571694in}}%
\pgfpathlineto{\pgfqpoint{4.154560in}{0.573508in}}%
\pgfpathlineto{\pgfqpoint{4.159520in}{0.572088in}}%
\pgfpathlineto{\pgfqpoint{4.160760in}{0.574248in}}%
\pgfpathlineto{\pgfqpoint{4.162000in}{0.574172in}}%
\pgfpathlineto{\pgfqpoint{4.163240in}{0.572656in}}%
\pgfpathlineto{\pgfqpoint{4.164480in}{0.573915in}}%
\pgfpathlineto{\pgfqpoint{4.168200in}{0.569398in}}%
\pgfpathlineto{\pgfqpoint{4.170680in}{0.569649in}}%
\pgfpathlineto{\pgfqpoint{4.174400in}{0.567618in}}%
\pgfpathlineto{\pgfqpoint{4.179360in}{0.567870in}}%
\pgfpathlineto{\pgfqpoint{4.181840in}{0.565853in}}%
\pgfpathlineto{\pgfqpoint{4.183080in}{0.565039in}}%
\pgfpathlineto{\pgfqpoint{4.186800in}{0.566445in}}%
\pgfpathlineto{\pgfqpoint{4.188040in}{0.565623in}}%
\pgfpathlineto{\pgfqpoint{4.191760in}{0.568414in}}%
\pgfpathlineto{\pgfqpoint{4.193000in}{0.569487in}}%
\pgfpathlineto{\pgfqpoint{4.195480in}{0.569009in}}%
\pgfpathlineto{\pgfqpoint{4.197960in}{0.569920in}}%
\pgfpathlineto{\pgfqpoint{4.200440in}{0.570000in}}%
\pgfpathlineto{\pgfqpoint{4.202920in}{0.567833in}}%
\pgfpathlineto{\pgfqpoint{4.204160in}{0.569714in}}%
\pgfpathlineto{\pgfqpoint{4.210360in}{0.569290in}}%
\pgfpathlineto{\pgfqpoint{4.214080in}{0.571505in}}%
\pgfpathlineto{\pgfqpoint{4.215320in}{0.570270in}}%
\pgfpathlineto{\pgfqpoint{4.216560in}{0.571276in}}%
\pgfpathlineto{\pgfqpoint{4.219040in}{0.569916in}}%
\pgfpathlineto{\pgfqpoint{4.221520in}{0.570529in}}%
\pgfpathlineto{\pgfqpoint{4.225240in}{0.569650in}}%
\pgfpathlineto{\pgfqpoint{4.233920in}{0.573640in}}%
\pgfpathlineto{\pgfqpoint{4.236400in}{0.573407in}}%
\pgfpathlineto{\pgfqpoint{4.240120in}{0.575382in}}%
\pgfpathlineto{\pgfqpoint{4.242600in}{0.575056in}}%
\pgfpathlineto{\pgfqpoint{4.246320in}{0.573104in}}%
\pgfpathlineto{\pgfqpoint{4.250040in}{0.573263in}}%
\pgfpathlineto{\pgfqpoint{4.256240in}{0.570356in}}%
\pgfpathlineto{\pgfqpoint{4.261200in}{0.572679in}}%
\pgfpathlineto{\pgfqpoint{4.266160in}{0.571999in}}%
\pgfpathlineto{\pgfqpoint{4.267400in}{0.571629in}}%
\pgfpathlineto{\pgfqpoint{4.268640in}{0.572450in}}%
\pgfpathlineto{\pgfqpoint{4.271120in}{0.569649in}}%
\pgfpathlineto{\pgfqpoint{4.281040in}{0.572972in}}%
\pgfpathlineto{\pgfqpoint{4.283520in}{0.571927in}}%
\pgfpathlineto{\pgfqpoint{4.286000in}{0.574113in}}%
\pgfpathlineto{\pgfqpoint{4.287240in}{0.572632in}}%
\pgfpathlineto{\pgfqpoint{4.288480in}{0.573729in}}%
\pgfpathlineto{\pgfqpoint{4.292200in}{0.568672in}}%
\pgfpathlineto{\pgfqpoint{4.294680in}{0.568768in}}%
\pgfpathlineto{\pgfqpoint{4.298400in}{0.566931in}}%
\pgfpathlineto{\pgfqpoint{4.303360in}{0.568322in}}%
\pgfpathlineto{\pgfqpoint{4.305840in}{0.566010in}}%
\pgfpathlineto{\pgfqpoint{4.308320in}{0.565427in}}%
\pgfpathlineto{\pgfqpoint{4.310800in}{0.566310in}}%
\pgfpathlineto{\pgfqpoint{4.312040in}{0.565415in}}%
\pgfpathlineto{\pgfqpoint{4.314520in}{0.567122in}}%
\pgfpathlineto{\pgfqpoint{4.323200in}{0.569907in}}%
\pgfpathlineto{\pgfqpoint{4.326920in}{0.566620in}}%
\pgfpathlineto{\pgfqpoint{4.329400in}{0.568451in}}%
\pgfpathlineto{\pgfqpoint{4.331880in}{0.568294in}}%
\pgfpathlineto{\pgfqpoint{4.333120in}{0.567632in}}%
\pgfpathlineto{\pgfqpoint{4.338080in}{0.571045in}}%
\pgfpathlineto{\pgfqpoint{4.339320in}{0.569865in}}%
\pgfpathlineto{\pgfqpoint{4.340560in}{0.570897in}}%
\pgfpathlineto{\pgfqpoint{4.343040in}{0.570132in}}%
\pgfpathlineto{\pgfqpoint{4.344280in}{0.571238in}}%
\pgfpathlineto{\pgfqpoint{4.349240in}{0.569484in}}%
\pgfpathlineto{\pgfqpoint{4.355440in}{0.572274in}}%
\pgfpathlineto{\pgfqpoint{4.357920in}{0.572542in}}%
\pgfpathlineto{\pgfqpoint{4.360400in}{0.572489in}}%
\pgfpathlineto{\pgfqpoint{4.362880in}{0.574733in}}%
\pgfpathlineto{\pgfqpoint{4.366600in}{0.574760in}}%
\pgfpathlineto{\pgfqpoint{4.371560in}{0.572989in}}%
\pgfpathlineto{\pgfqpoint{4.374040in}{0.573096in}}%
\pgfpathlineto{\pgfqpoint{4.377760in}{0.571793in}}%
\pgfpathlineto{\pgfqpoint{4.380240in}{0.570381in}}%
\pgfpathlineto{\pgfqpoint{4.386440in}{0.572841in}}%
\pgfpathlineto{\pgfqpoint{4.388920in}{0.572232in}}%
\pgfpathlineto{\pgfqpoint{4.393880in}{0.570427in}}%
\pgfpathlineto{\pgfqpoint{4.396360in}{0.569720in}}%
\pgfpathlineto{\pgfqpoint{4.402560in}{0.571668in}}%
\pgfpathlineto{\pgfqpoint{4.406280in}{0.570459in}}%
\pgfpathlineto{\pgfqpoint{4.407520in}{0.570869in}}%
\pgfpathlineto{\pgfqpoint{4.410000in}{0.573183in}}%
\pgfpathlineto{\pgfqpoint{4.411240in}{0.571914in}}%
\pgfpathlineto{\pgfqpoint{4.412480in}{0.572671in}}%
\pgfpathlineto{\pgfqpoint{4.416200in}{0.568238in}}%
\pgfpathlineto{\pgfqpoint{4.418680in}{0.568375in}}%
\pgfpathlineto{\pgfqpoint{4.421160in}{0.566551in}}%
\pgfpathlineto{\pgfqpoint{4.423640in}{0.567001in}}%
\pgfpathlineto{\pgfqpoint{4.427360in}{0.567686in}}%
\pgfpathlineto{\pgfqpoint{4.429840in}{0.565131in}}%
\pgfpathlineto{\pgfqpoint{4.431080in}{0.564061in}}%
\pgfpathlineto{\pgfqpoint{4.434800in}{0.566118in}}%
\pgfpathlineto{\pgfqpoint{4.436040in}{0.565162in}}%
\pgfpathlineto{\pgfqpoint{4.439760in}{0.567211in}}%
\pgfpathlineto{\pgfqpoint{4.441000in}{0.568095in}}%
\pgfpathlineto{\pgfqpoint{4.443480in}{0.567112in}}%
\pgfpathlineto{\pgfqpoint{4.445960in}{0.568311in}}%
\pgfpathlineto{\pgfqpoint{4.448440in}{0.568000in}}%
\pgfpathlineto{\pgfqpoint{4.450920in}{0.565677in}}%
\pgfpathlineto{\pgfqpoint{4.453400in}{0.567398in}}%
\pgfpathlineto{\pgfqpoint{4.455880in}{0.567336in}}%
\pgfpathlineto{\pgfqpoint{4.457120in}{0.566976in}}%
\pgfpathlineto{\pgfqpoint{4.462080in}{0.569717in}}%
\pgfpathlineto{\pgfqpoint{4.463320in}{0.568602in}}%
\pgfpathlineto{\pgfqpoint{4.465800in}{0.569154in}}%
\pgfpathlineto{\pgfqpoint{4.470760in}{0.569328in}}%
\pgfpathlineto{\pgfqpoint{4.472000in}{0.568121in}}%
\pgfpathlineto{\pgfqpoint{4.479440in}{0.570715in}}%
\pgfpathlineto{\pgfqpoint{4.481920in}{0.571538in}}%
\pgfpathlineto{\pgfqpoint{4.484400in}{0.572304in}}%
\pgfpathlineto{\pgfqpoint{4.486880in}{0.574209in}}%
\pgfpathlineto{\pgfqpoint{4.490600in}{0.574864in}}%
\pgfpathlineto{\pgfqpoint{4.494320in}{0.573098in}}%
\pgfpathlineto{\pgfqpoint{4.499280in}{0.573196in}}%
\pgfpathlineto{\pgfqpoint{4.501760in}{0.572150in}}%
\pgfpathlineto{\pgfqpoint{4.504240in}{0.571492in}}%
\pgfpathlineto{\pgfqpoint{4.510440in}{0.573477in}}%
\pgfpathlineto{\pgfqpoint{4.512920in}{0.573315in}}%
\pgfpathlineto{\pgfqpoint{4.516640in}{0.573343in}}%
\pgfpathlineto{\pgfqpoint{4.519120in}{0.570666in}}%
\pgfpathlineto{\pgfqpoint{4.529040in}{0.573226in}}%
\pgfpathlineto{\pgfqpoint{4.530280in}{0.571828in}}%
\pgfpathlineto{\pgfqpoint{4.531520in}{0.572378in}}%
\pgfpathlineto{\pgfqpoint{4.532760in}{0.574582in}}%
\pgfpathlineto{\pgfqpoint{4.536480in}{0.573523in}}%
\pgfpathlineto{\pgfqpoint{4.538960in}{0.570223in}}%
\pgfpathlineto{\pgfqpoint{4.541440in}{0.569950in}}%
\pgfpathlineto{\pgfqpoint{4.542680in}{0.569834in}}%
\pgfpathlineto{\pgfqpoint{4.545160in}{0.567581in}}%
\pgfpathlineto{\pgfqpoint{4.547640in}{0.568055in}}%
\pgfpathlineto{\pgfqpoint{4.551360in}{0.569091in}}%
\pgfpathlineto{\pgfqpoint{4.553840in}{0.566174in}}%
\pgfpathlineto{\pgfqpoint{4.555080in}{0.565271in}}%
\pgfpathlineto{\pgfqpoint{4.558800in}{0.566979in}}%
\pgfpathlineto{\pgfqpoint{4.560040in}{0.566255in}}%
\pgfpathlineto{\pgfqpoint{4.565000in}{0.569313in}}%
\pgfpathlineto{\pgfqpoint{4.567480in}{0.568508in}}%
\pgfpathlineto{\pgfqpoint{4.569960in}{0.569470in}}%
\pgfpathlineto{\pgfqpoint{4.572440in}{0.569221in}}%
\pgfpathlineto{\pgfqpoint{4.574920in}{0.566759in}}%
\pgfpathlineto{\pgfqpoint{4.577400in}{0.568349in}}%
\pgfpathlineto{\pgfqpoint{4.582360in}{0.569060in}}%
\pgfpathlineto{\pgfqpoint{4.586080in}{0.570582in}}%
\pgfpathlineto{\pgfqpoint{4.587320in}{0.569982in}}%
\pgfpathlineto{\pgfqpoint{4.589800in}{0.570339in}}%
\pgfpathlineto{\pgfqpoint{4.591040in}{0.570497in}}%
\pgfpathlineto{\pgfqpoint{4.593520in}{0.571629in}}%
\pgfpathlineto{\pgfqpoint{4.598480in}{0.569025in}}%
\pgfpathlineto{\pgfqpoint{4.603440in}{0.570726in}}%
\pgfpathlineto{\pgfqpoint{4.607160in}{0.571568in}}%
\pgfpathlineto{\pgfqpoint{4.614600in}{0.574714in}}%
\pgfpathlineto{\pgfqpoint{4.619560in}{0.573226in}}%
\pgfpathlineto{\pgfqpoint{4.622040in}{0.573578in}}%
\pgfpathlineto{\pgfqpoint{4.627000in}{0.570866in}}%
\pgfpathlineto{\pgfqpoint{4.629480in}{0.571064in}}%
\pgfpathlineto{\pgfqpoint{4.631960in}{0.572627in}}%
\pgfpathlineto{\pgfqpoint{4.635680in}{0.572563in}}%
\pgfpathlineto{\pgfqpoint{4.638160in}{0.572156in}}%
\pgfpathlineto{\pgfqpoint{4.639400in}{0.571291in}}%
\pgfpathlineto{\pgfqpoint{4.640640in}{0.572180in}}%
\pgfpathlineto{\pgfqpoint{4.643120in}{0.569250in}}%
\pgfpathlineto{\pgfqpoint{4.653040in}{0.571570in}}%
\pgfpathlineto{\pgfqpoint{4.654280in}{0.569749in}}%
\pgfpathlineto{\pgfqpoint{4.655520in}{0.570237in}}%
\pgfpathlineto{\pgfqpoint{4.658000in}{0.572530in}}%
\pgfpathlineto{\pgfqpoint{4.659240in}{0.571486in}}%
\pgfpathlineto{\pgfqpoint{4.660480in}{0.572162in}}%
\pgfpathlineto{\pgfqpoint{4.662960in}{0.569116in}}%
\pgfpathlineto{\pgfqpoint{4.665440in}{0.568435in}}%
\pgfpathlineto{\pgfqpoint{4.666680in}{0.568101in}}%
\pgfpathlineto{\pgfqpoint{4.669160in}{0.566553in}}%
\pgfpathlineto{\pgfqpoint{4.675360in}{0.567905in}}%
\pgfpathlineto{\pgfqpoint{4.677840in}{0.565441in}}%
\pgfpathlineto{\pgfqpoint{4.679080in}{0.564384in}}%
\pgfpathlineto{\pgfqpoint{4.682800in}{0.566622in}}%
\pgfpathlineto{\pgfqpoint{4.684040in}{0.565934in}}%
\pgfpathlineto{\pgfqpoint{4.689000in}{0.568576in}}%
\pgfpathlineto{\pgfqpoint{4.690240in}{0.567348in}}%
\pgfpathlineto{\pgfqpoint{4.695200in}{0.568623in}}%
\pgfpathlineto{\pgfqpoint{4.697680in}{0.566846in}}%
\pgfpathlineto{\pgfqpoint{4.698920in}{0.566035in}}%
\pgfpathlineto{\pgfqpoint{4.700160in}{0.567542in}}%
\pgfpathlineto{\pgfqpoint{4.705120in}{0.566432in}}%
\pgfpathlineto{\pgfqpoint{4.710080in}{0.569721in}}%
\pgfpathlineto{\pgfqpoint{4.712560in}{0.569650in}}%
\pgfpathlineto{\pgfqpoint{4.715040in}{0.569672in}}%
\pgfpathlineto{\pgfqpoint{4.717520in}{0.571167in}}%
\pgfpathlineto{\pgfqpoint{4.722480in}{0.567890in}}%
\pgfpathlineto{\pgfqpoint{4.727440in}{0.569532in}}%
\pgfpathlineto{\pgfqpoint{4.729920in}{0.569882in}}%
\pgfpathlineto{\pgfqpoint{4.732400in}{0.571197in}}%
\pgfpathlineto{\pgfqpoint{4.734880in}{0.573351in}}%
\pgfpathlineto{\pgfqpoint{4.738600in}{0.573766in}}%
\pgfpathlineto{\pgfqpoint{4.743560in}{0.571534in}}%
\pgfpathlineto{\pgfqpoint{4.746040in}{0.571366in}}%
\pgfpathlineto{\pgfqpoint{4.748520in}{0.571086in}}%
\pgfpathlineto{\pgfqpoint{4.753480in}{0.570073in}}%
\pgfpathlineto{\pgfqpoint{4.754720in}{0.571593in}}%
\pgfpathlineto{\pgfqpoint{4.765880in}{0.568953in}}%
\pgfpathlineto{\pgfqpoint{4.768360in}{0.567155in}}%
\pgfpathlineto{\pgfqpoint{4.777040in}{0.570005in}}%
\pgfpathlineto{\pgfqpoint{4.778280in}{0.568376in}}%
\pgfpathlineto{\pgfqpoint{4.779520in}{0.569125in}}%
\pgfpathlineto{\pgfqpoint{4.780760in}{0.571200in}}%
\pgfpathlineto{\pgfqpoint{4.784480in}{0.570523in}}%
\pgfpathlineto{\pgfqpoint{4.788200in}{0.565945in}}%
\pgfpathlineto{\pgfqpoint{4.790680in}{0.565579in}}%
\pgfpathlineto{\pgfqpoint{4.793160in}{0.563924in}}%
\pgfpathlineto{\pgfqpoint{4.795640in}{0.564550in}}%
\pgfpathlineto{\pgfqpoint{4.799360in}{0.565109in}}%
\pgfpathlineto{\pgfqpoint{4.801840in}{0.563234in}}%
\pgfpathlineto{\pgfqpoint{4.803080in}{0.562606in}}%
\pgfpathlineto{\pgfqpoint{4.806800in}{0.564817in}}%
\pgfpathlineto{\pgfqpoint{4.808040in}{0.564001in}}%
\pgfpathlineto{\pgfqpoint{4.813000in}{0.566788in}}%
\pgfpathlineto{\pgfqpoint{4.814240in}{0.565967in}}%
\pgfpathlineto{\pgfqpoint{4.817960in}{0.567642in}}%
\pgfpathlineto{\pgfqpoint{4.824160in}{0.565384in}}%
\pgfpathlineto{\pgfqpoint{4.829120in}{0.563941in}}%
\pgfpathlineto{\pgfqpoint{4.832840in}{0.566644in}}%
\pgfpathlineto{\pgfqpoint{4.835320in}{0.566876in}}%
\pgfpathlineto{\pgfqpoint{4.837800in}{0.567258in}}%
\pgfpathlineto{\pgfqpoint{4.839040in}{0.567284in}}%
\pgfpathlineto{\pgfqpoint{4.841520in}{0.568939in}}%
\pgfpathlineto{\pgfqpoint{4.845240in}{0.566297in}}%
\pgfpathlineto{\pgfqpoint{4.847720in}{0.566910in}}%
\pgfpathlineto{\pgfqpoint{4.851440in}{0.567605in}}%
\pgfpathlineto{\pgfqpoint{4.853920in}{0.568168in}}%
\pgfpathlineto{\pgfqpoint{4.856400in}{0.569451in}}%
\pgfpathlineto{\pgfqpoint{4.858880in}{0.572108in}}%
\pgfpathlineto{\pgfqpoint{4.863840in}{0.571402in}}%
\pgfpathlineto{\pgfqpoint{4.867560in}{0.570044in}}%
\pgfpathlineto{\pgfqpoint{4.870040in}{0.570460in}}%
\pgfpathlineto{\pgfqpoint{4.872520in}{0.570162in}}%
\pgfpathlineto{\pgfqpoint{4.877480in}{0.568821in}}%
\pgfpathlineto{\pgfqpoint{4.879960in}{0.569843in}}%
\pgfpathlineto{\pgfqpoint{4.882440in}{0.570105in}}%
\pgfpathlineto{\pgfqpoint{4.887400in}{0.568206in}}%
\pgfpathlineto{\pgfqpoint{4.888640in}{0.569274in}}%
\pgfpathlineto{\pgfqpoint{4.892360in}{0.566100in}}%
\pgfpathlineto{\pgfqpoint{4.894840in}{0.566350in}}%
\pgfpathlineto{\pgfqpoint{4.896080in}{0.566056in}}%
\pgfpathlineto{\pgfqpoint{4.898560in}{0.567905in}}%
\pgfpathlineto{\pgfqpoint{4.903520in}{0.567251in}}%
\pgfpathlineto{\pgfqpoint{4.906000in}{0.569373in}}%
\pgfpathlineto{\pgfqpoint{4.907240in}{0.568503in}}%
\pgfpathlineto{\pgfqpoint{4.908480in}{0.569385in}}%
\pgfpathlineto{\pgfqpoint{4.912200in}{0.565030in}}%
\pgfpathlineto{\pgfqpoint{4.914680in}{0.564470in}}%
\pgfpathlineto{\pgfqpoint{4.917160in}{0.562204in}}%
\pgfpathlineto{\pgfqpoint{4.925840in}{0.561015in}}%
\pgfpathlineto{\pgfqpoint{4.927080in}{0.560335in}}%
\pgfpathlineto{\pgfqpoint{4.930800in}{0.562775in}}%
\pgfpathlineto{\pgfqpoint{4.932040in}{0.562154in}}%
\pgfpathlineto{\pgfqpoint{4.937000in}{0.564211in}}%
\pgfpathlineto{\pgfqpoint{4.939480in}{0.563410in}}%
\pgfpathlineto{\pgfqpoint{4.940720in}{0.564714in}}%
\pgfpathlineto{\pgfqpoint{4.945680in}{0.562381in}}%
\pgfpathlineto{\pgfqpoint{4.946920in}{0.561126in}}%
\pgfpathlineto{\pgfqpoint{4.948160in}{0.562613in}}%
\pgfpathlineto{\pgfqpoint{4.953120in}{0.561355in}}%
\pgfpathlineto{\pgfqpoint{4.955600in}{0.563900in}}%
\pgfpathlineto{\pgfqpoint{4.956840in}{0.563433in}}%
\pgfpathlineto{\pgfqpoint{4.958080in}{0.564177in}}%
\pgfpathlineto{\pgfqpoint{4.960560in}{0.564123in}}%
\pgfpathlineto{\pgfqpoint{4.963040in}{0.564208in}}%
\pgfpathlineto{\pgfqpoint{4.965520in}{0.565883in}}%
\pgfpathlineto{\pgfqpoint{4.970480in}{0.561678in}}%
\pgfpathlineto{\pgfqpoint{4.975440in}{0.562921in}}%
\pgfpathlineto{\pgfqpoint{4.977920in}{0.562494in}}%
\pgfpathlineto{\pgfqpoint{4.979160in}{0.562575in}}%
\pgfpathlineto{\pgfqpoint{4.984120in}{0.567245in}}%
\pgfpathlineto{\pgfqpoint{5.001480in}{0.564353in}}%
\pgfpathlineto{\pgfqpoint{5.003960in}{0.565040in}}%
\pgfpathlineto{\pgfqpoint{5.006440in}{0.564933in}}%
\pgfpathlineto{\pgfqpoint{5.007680in}{0.563776in}}%
\pgfpathlineto{\pgfqpoint{5.010160in}{0.564506in}}%
\pgfpathlineto{\pgfqpoint{5.011400in}{0.563617in}}%
\pgfpathlineto{\pgfqpoint{5.012640in}{0.564813in}}%
\pgfpathlineto{\pgfqpoint{5.016360in}{0.561055in}}%
\pgfpathlineto{\pgfqpoint{5.018840in}{0.561744in}}%
\pgfpathlineto{\pgfqpoint{5.020080in}{0.561413in}}%
\pgfpathlineto{\pgfqpoint{5.022560in}{0.562762in}}%
\pgfpathlineto{\pgfqpoint{5.027520in}{0.561954in}}%
\pgfpathlineto{\pgfqpoint{5.030000in}{0.564681in}}%
\pgfpathlineto{\pgfqpoint{5.031240in}{0.564159in}}%
\pgfpathlineto{\pgfqpoint{5.032480in}{0.564972in}}%
\pgfpathlineto{\pgfqpoint{5.036200in}{0.560394in}}%
\pgfpathlineto{\pgfqpoint{5.038680in}{0.560679in}}%
\pgfpathlineto{\pgfqpoint{5.041160in}{0.558787in}}%
\pgfpathlineto{\pgfqpoint{5.048600in}{0.558162in}}%
\pgfpathlineto{\pgfqpoint{5.051080in}{0.556265in}}%
\pgfpathlineto{\pgfqpoint{5.059760in}{0.559809in}}%
\pgfpathlineto{\pgfqpoint{5.062240in}{0.558500in}}%
\pgfpathlineto{\pgfqpoint{5.063480in}{0.558350in}}%
\pgfpathlineto{\pgfqpoint{5.065960in}{0.559106in}}%
\pgfpathlineto{\pgfqpoint{5.069680in}{0.558089in}}%
\pgfpathlineto{\pgfqpoint{5.070920in}{0.556703in}}%
\pgfpathlineto{\pgfqpoint{5.073400in}{0.558016in}}%
\pgfpathlineto{\pgfqpoint{5.075880in}{0.557862in}}%
\pgfpathlineto{\pgfqpoint{5.077120in}{0.556502in}}%
\pgfpathlineto{\pgfqpoint{5.080840in}{0.558656in}}%
\pgfpathlineto{\pgfqpoint{5.082080in}{0.559803in}}%
\pgfpathlineto{\pgfqpoint{5.084560in}{0.558987in}}%
\pgfpathlineto{\pgfqpoint{5.087040in}{0.559426in}}%
\pgfpathlineto{\pgfqpoint{5.089520in}{0.561637in}}%
\pgfpathlineto{\pgfqpoint{5.092000in}{0.559628in}}%
\pgfpathlineto{\pgfqpoint{5.098200in}{0.561839in}}%
\pgfpathlineto{\pgfqpoint{5.101920in}{0.561373in}}%
\pgfpathlineto{\pgfqpoint{5.103160in}{0.561179in}}%
\pgfpathlineto{\pgfqpoint{5.106880in}{0.565958in}}%
\pgfpathlineto{\pgfqpoint{5.110600in}{0.565728in}}%
\pgfpathlineto{\pgfqpoint{5.119280in}{0.563025in}}%
\pgfpathlineto{\pgfqpoint{5.121760in}{0.562950in}}%
\pgfpathlineto{\pgfqpoint{5.124240in}{0.561311in}}%
\pgfpathlineto{\pgfqpoint{5.129200in}{0.563079in}}%
\pgfpathlineto{\pgfqpoint{5.130440in}{0.563424in}}%
\pgfpathlineto{\pgfqpoint{5.131680in}{0.562468in}}%
\pgfpathlineto{\pgfqpoint{5.134160in}{0.562897in}}%
\pgfpathlineto{\pgfqpoint{5.135400in}{0.561816in}}%
\pgfpathlineto{\pgfqpoint{5.136640in}{0.563008in}}%
\pgfpathlineto{\pgfqpoint{5.140360in}{0.559428in}}%
\pgfpathlineto{\pgfqpoint{5.146560in}{0.561483in}}%
\pgfpathlineto{\pgfqpoint{5.151520in}{0.560244in}}%
\pgfpathlineto{\pgfqpoint{5.154000in}{0.562681in}}%
\pgfpathlineto{\pgfqpoint{5.156480in}{0.563052in}}%
\pgfpathlineto{\pgfqpoint{5.158960in}{0.559728in}}%
\pgfpathlineto{\pgfqpoint{5.161440in}{0.559452in}}%
\pgfpathlineto{\pgfqpoint{5.162680in}{0.559594in}}%
\pgfpathlineto{\pgfqpoint{5.163920in}{0.557982in}}%
\pgfpathlineto{\pgfqpoint{5.168880in}{0.558267in}}%
\pgfpathlineto{\pgfqpoint{5.171360in}{0.557939in}}%
\pgfpathlineto{\pgfqpoint{5.177560in}{0.555548in}}%
\pgfpathlineto{\pgfqpoint{5.178800in}{0.556865in}}%
\pgfpathlineto{\pgfqpoint{5.180040in}{0.556533in}}%
\pgfpathlineto{\pgfqpoint{5.183760in}{0.558874in}}%
\pgfpathlineto{\pgfqpoint{5.185000in}{0.559105in}}%
\pgfpathlineto{\pgfqpoint{5.187480in}{0.557775in}}%
\pgfpathlineto{\pgfqpoint{5.189960in}{0.558483in}}%
\pgfpathlineto{\pgfqpoint{5.193680in}{0.558493in}}%
\pgfpathlineto{\pgfqpoint{5.194920in}{0.556984in}}%
\pgfpathlineto{\pgfqpoint{5.196160in}{0.558371in}}%
\pgfpathlineto{\pgfqpoint{5.198640in}{0.558040in}}%
\pgfpathlineto{\pgfqpoint{5.199880in}{0.558494in}}%
\pgfpathlineto{\pgfqpoint{5.201120in}{0.557455in}}%
\pgfpathlineto{\pgfqpoint{5.206080in}{0.560448in}}%
\pgfpathlineto{\pgfqpoint{5.208560in}{0.559402in}}%
\pgfpathlineto{\pgfqpoint{5.211040in}{0.559299in}}%
\pgfpathlineto{\pgfqpoint{5.213520in}{0.560956in}}%
\pgfpathlineto{\pgfqpoint{5.217240in}{0.559113in}}%
\pgfpathlineto{\pgfqpoint{5.223440in}{0.560468in}}%
\pgfpathlineto{\pgfqpoint{5.227160in}{0.560526in}}%
\pgfpathlineto{\pgfqpoint{5.230880in}{0.564770in}}%
\pgfpathlineto{\pgfqpoint{5.233360in}{0.564433in}}%
\pgfpathlineto{\pgfqpoint{5.248240in}{0.559755in}}%
\pgfpathlineto{\pgfqpoint{5.251960in}{0.560686in}}%
\pgfpathlineto{\pgfqpoint{5.254440in}{0.560428in}}%
\pgfpathlineto{\pgfqpoint{5.256920in}{0.559901in}}%
\pgfpathlineto{\pgfqpoint{5.261880in}{0.559278in}}%
\pgfpathlineto{\pgfqpoint{5.263120in}{0.557729in}}%
\pgfpathlineto{\pgfqpoint{5.273040in}{0.560032in}}%
\pgfpathlineto{\pgfqpoint{5.274280in}{0.558304in}}%
\pgfpathlineto{\pgfqpoint{5.275520in}{0.558546in}}%
\pgfpathlineto{\pgfqpoint{5.278000in}{0.561104in}}%
\pgfpathlineto{\pgfqpoint{5.280480in}{0.561287in}}%
\pgfpathlineto{\pgfqpoint{5.282960in}{0.557533in}}%
\pgfpathlineto{\pgfqpoint{5.285440in}{0.557384in}}%
\pgfpathlineto{\pgfqpoint{5.286680in}{0.557631in}}%
\pgfpathlineto{\pgfqpoint{5.287920in}{0.555883in}}%
\pgfpathlineto{\pgfqpoint{5.296600in}{0.556182in}}%
\pgfpathlineto{\pgfqpoint{5.300320in}{0.554031in}}%
\pgfpathlineto{\pgfqpoint{5.301560in}{0.553962in}}%
\pgfpathlineto{\pgfqpoint{5.306520in}{0.557327in}}%
\pgfpathlineto{\pgfqpoint{5.309000in}{0.557806in}}%
\pgfpathlineto{\pgfqpoint{5.311480in}{0.556801in}}%
\pgfpathlineto{\pgfqpoint{5.312720in}{0.557323in}}%
\pgfpathlineto{\pgfqpoint{5.315200in}{0.557119in}}%
\pgfpathlineto{\pgfqpoint{5.316440in}{0.557765in}}%
\pgfpathlineto{\pgfqpoint{5.317680in}{0.557119in}}%
\pgfpathlineto{\pgfqpoint{5.318920in}{0.555093in}}%
\pgfpathlineto{\pgfqpoint{5.323880in}{0.556363in}}%
\pgfpathlineto{\pgfqpoint{5.325120in}{0.555525in}}%
\pgfpathlineto{\pgfqpoint{5.330080in}{0.559315in}}%
\pgfpathlineto{\pgfqpoint{5.333800in}{0.558058in}}%
\pgfpathlineto{\pgfqpoint{5.337520in}{0.559485in}}%
\pgfpathlineto{\pgfqpoint{5.340000in}{0.558047in}}%
\pgfpathlineto{\pgfqpoint{5.347440in}{0.560306in}}%
\pgfpathlineto{\pgfqpoint{5.351160in}{0.560653in}}%
\pgfpathlineto{\pgfqpoint{5.356120in}{0.565710in}}%
\pgfpathlineto{\pgfqpoint{5.363560in}{0.561569in}}%
\pgfpathlineto{\pgfqpoint{5.366040in}{0.561509in}}%
\pgfpathlineto{\pgfqpoint{5.368520in}{0.560412in}}%
\pgfpathlineto{\pgfqpoint{5.379680in}{0.558513in}}%
\pgfpathlineto{\pgfqpoint{5.385880in}{0.560012in}}%
\pgfpathlineto{\pgfqpoint{5.388360in}{0.558697in}}%
\pgfpathlineto{\pgfqpoint{5.392080in}{0.558972in}}%
\pgfpathlineto{\pgfqpoint{5.394560in}{0.561402in}}%
\pgfpathlineto{\pgfqpoint{5.399520in}{0.560100in}}%
\pgfpathlineto{\pgfqpoint{5.402000in}{0.562387in}}%
\pgfpathlineto{\pgfqpoint{5.404480in}{0.563233in}}%
\pgfpathlineto{\pgfqpoint{5.406960in}{0.560044in}}%
\pgfpathlineto{\pgfqpoint{5.409440in}{0.559818in}}%
\pgfpathlineto{\pgfqpoint{5.410680in}{0.560379in}}%
\pgfpathlineto{\pgfqpoint{5.413160in}{0.558714in}}%
\pgfpathlineto{\pgfqpoint{5.418120in}{0.558142in}}%
\pgfpathlineto{\pgfqpoint{5.419360in}{0.559319in}}%
\pgfpathlineto{\pgfqpoint{5.421840in}{0.557476in}}%
\pgfpathlineto{\pgfqpoint{5.424320in}{0.556061in}}%
\pgfpathlineto{\pgfqpoint{5.425560in}{0.555611in}}%
\pgfpathlineto{\pgfqpoint{5.430520in}{0.559295in}}%
\pgfpathlineto{\pgfqpoint{5.433000in}{0.560371in}}%
\pgfpathlineto{\pgfqpoint{5.434240in}{0.559615in}}%
\pgfpathlineto{\pgfqpoint{5.436720in}{0.560501in}}%
\pgfpathlineto{\pgfqpoint{5.437960in}{0.559042in}}%
\pgfpathlineto{\pgfqpoint{5.441680in}{0.560014in}}%
\pgfpathlineto{\pgfqpoint{5.442920in}{0.558142in}}%
\pgfpathlineto{\pgfqpoint{5.447880in}{0.559391in}}%
\pgfpathlineto{\pgfqpoint{5.449120in}{0.558927in}}%
\pgfpathlineto{\pgfqpoint{5.454080in}{0.562707in}}%
\pgfpathlineto{\pgfqpoint{5.456560in}{0.562131in}}%
\pgfpathlineto{\pgfqpoint{5.461520in}{0.563015in}}%
\pgfpathlineto{\pgfqpoint{5.464000in}{0.561546in}}%
\pgfpathlineto{\pgfqpoint{5.473920in}{0.565728in}}%
\pgfpathlineto{\pgfqpoint{5.475160in}{0.565327in}}%
\pgfpathlineto{\pgfqpoint{5.480120in}{0.570313in}}%
\pgfpathlineto{\pgfqpoint{5.483840in}{0.567429in}}%
\pgfpathlineto{\pgfqpoint{5.486320in}{0.565780in}}%
\pgfpathlineto{\pgfqpoint{5.488800in}{0.565260in}}%
\pgfpathlineto{\pgfqpoint{5.490040in}{0.565169in}}%
\pgfpathlineto{\pgfqpoint{5.492520in}{0.564272in}}%
\pgfpathlineto{\pgfqpoint{5.503680in}{0.561564in}}%
\pgfpathlineto{\pgfqpoint{5.509880in}{0.561493in}}%
\pgfpathlineto{\pgfqpoint{5.511120in}{0.560119in}}%
\pgfpathlineto{\pgfqpoint{5.514840in}{0.561030in}}%
\pgfpathlineto{\pgfqpoint{5.516080in}{0.560260in}}%
\pgfpathlineto{\pgfqpoint{5.518560in}{0.562547in}}%
\pgfpathlineto{\pgfqpoint{5.523520in}{0.560899in}}%
\pgfpathlineto{\pgfqpoint{5.526000in}{0.563715in}}%
\pgfpathlineto{\pgfqpoint{5.528480in}{0.564790in}}%
\pgfpathlineto{\pgfqpoint{5.530960in}{0.561261in}}%
\pgfpathlineto{\pgfqpoint{5.533440in}{0.561196in}}%
\pgfpathlineto{\pgfqpoint{5.534680in}{0.562255in}}%
\pgfpathlineto{\pgfqpoint{5.537160in}{0.559940in}}%
\pgfpathlineto{\pgfqpoint{5.542120in}{0.560285in}}%
\pgfpathlineto{\pgfqpoint{5.543360in}{0.561553in}}%
\pgfpathlineto{\pgfqpoint{5.549560in}{0.557790in}}%
\pgfpathlineto{\pgfqpoint{5.554520in}{0.562325in}}%
\pgfpathlineto{\pgfqpoint{5.557000in}{0.563435in}}%
\pgfpathlineto{\pgfqpoint{5.559480in}{0.563299in}}%
\pgfpathlineto{\pgfqpoint{5.560720in}{0.563363in}}%
\pgfpathlineto{\pgfqpoint{5.563200in}{0.561652in}}%
\pgfpathlineto{\pgfqpoint{5.565680in}{0.562135in}}%
\pgfpathlineto{\pgfqpoint{5.566920in}{0.560978in}}%
\pgfpathlineto{\pgfqpoint{5.571880in}{0.561982in}}%
\pgfpathlineto{\pgfqpoint{5.573120in}{0.561347in}}%
\pgfpathlineto{\pgfqpoint{5.576840in}{0.563430in}}%
\pgfpathlineto{\pgfqpoint{5.578080in}{0.563980in}}%
\pgfpathlineto{\pgfqpoint{5.579320in}{0.563244in}}%
\pgfpathlineto{\pgfqpoint{5.585520in}{0.565253in}}%
\pgfpathlineto{\pgfqpoint{5.588000in}{0.562880in}}%
\pgfpathlineto{\pgfqpoint{5.595440in}{0.566964in}}%
\pgfpathlineto{\pgfqpoint{5.597920in}{0.568378in}}%
\pgfpathlineto{\pgfqpoint{5.599160in}{0.567390in}}%
\pgfpathlineto{\pgfqpoint{5.600400in}{0.568574in}}%
\pgfpathlineto{\pgfqpoint{5.602880in}{0.572340in}}%
\pgfpathlineto{\pgfqpoint{5.604120in}{0.572790in}}%
\pgfpathlineto{\pgfqpoint{5.607840in}{0.569566in}}%
\pgfpathlineto{\pgfqpoint{5.620240in}{0.564591in}}%
\pgfpathlineto{\pgfqpoint{5.622720in}{0.564626in}}%
\pgfpathlineto{\pgfqpoint{5.625200in}{0.563218in}}%
\pgfpathlineto{\pgfqpoint{5.627680in}{0.563223in}}%
\pgfpathlineto{\pgfqpoint{5.633880in}{0.563301in}}%
\pgfpathlineto{\pgfqpoint{5.635120in}{0.562137in}}%
\pgfpathlineto{\pgfqpoint{5.638840in}{0.563687in}}%
\pgfpathlineto{\pgfqpoint{5.640080in}{0.563032in}}%
\pgfpathlineto{\pgfqpoint{5.642560in}{0.565311in}}%
\pgfpathlineto{\pgfqpoint{5.647520in}{0.563055in}}%
\pgfpathlineto{\pgfqpoint{5.650000in}{0.565965in}}%
\pgfpathlineto{\pgfqpoint{5.652480in}{0.567598in}}%
\pgfpathlineto{\pgfqpoint{5.654960in}{0.564237in}}%
\pgfpathlineto{\pgfqpoint{5.656200in}{0.563450in}}%
\pgfpathlineto{\pgfqpoint{5.658680in}{0.565583in}}%
\pgfpathlineto{\pgfqpoint{5.661160in}{0.563458in}}%
\pgfpathlineto{\pgfqpoint{5.667360in}{0.563460in}}%
\pgfpathlineto{\pgfqpoint{5.673560in}{0.557603in}}%
\pgfpathlineto{\pgfqpoint{5.681000in}{0.562586in}}%
\pgfpathlineto{\pgfqpoint{5.683480in}{0.562360in}}%
\pgfpathlineto{\pgfqpoint{5.684720in}{0.562902in}}%
\pgfpathlineto{\pgfqpoint{5.687200in}{0.561253in}}%
\pgfpathlineto{\pgfqpoint{5.689680in}{0.562260in}}%
\pgfpathlineto{\pgfqpoint{5.690920in}{0.561152in}}%
\pgfpathlineto{\pgfqpoint{5.694640in}{0.561844in}}%
\pgfpathlineto{\pgfqpoint{5.698360in}{0.561341in}}%
\pgfpathlineto{\pgfqpoint{5.699600in}{0.562982in}}%
\pgfpathlineto{\pgfqpoint{5.703320in}{0.562741in}}%
\pgfpathlineto{\pgfqpoint{5.709520in}{0.565061in}}%
\pgfpathlineto{\pgfqpoint{5.712000in}{0.563242in}}%
\pgfpathlineto{\pgfqpoint{5.716960in}{0.564564in}}%
\pgfpathlineto{\pgfqpoint{5.718200in}{0.564548in}}%
\pgfpathlineto{\pgfqpoint{5.721920in}{0.567381in}}%
\pgfpathlineto{\pgfqpoint{5.723160in}{0.566039in}}%
\pgfpathlineto{\pgfqpoint{5.724400in}{0.567637in}}%
\pgfpathlineto{\pgfqpoint{5.726880in}{0.572760in}}%
\pgfpathlineto{\pgfqpoint{5.728120in}{0.573147in}}%
\pgfpathlineto{\pgfqpoint{5.730600in}{0.571408in}}%
\pgfpathlineto{\pgfqpoint{5.735560in}{0.568517in}}%
\pgfpathlineto{\pgfqpoint{5.738040in}{0.568044in}}%
\pgfpathlineto{\pgfqpoint{5.740520in}{0.567606in}}%
\pgfpathlineto{\pgfqpoint{5.741760in}{0.567980in}}%
\pgfpathlineto{\pgfqpoint{5.745480in}{0.566228in}}%
\pgfpathlineto{\pgfqpoint{5.747960in}{0.565761in}}%
\pgfpathlineto{\pgfqpoint{5.752920in}{0.566047in}}%
\pgfpathlineto{\pgfqpoint{5.754160in}{0.567353in}}%
\pgfpathlineto{\pgfqpoint{5.759120in}{0.564853in}}%
\pgfpathlineto{\pgfqpoint{5.762840in}{0.567032in}}%
\pgfpathlineto{\pgfqpoint{5.764080in}{0.566394in}}%
\pgfpathlineto{\pgfqpoint{5.766560in}{0.568299in}}%
\pgfpathlineto{\pgfqpoint{5.769040in}{0.567094in}}%
\pgfpathlineto{\pgfqpoint{5.770280in}{0.564987in}}%
\pgfpathlineto{\pgfqpoint{5.771520in}{0.565521in}}%
\pgfpathlineto{\pgfqpoint{5.774000in}{0.568428in}}%
\pgfpathlineto{\pgfqpoint{5.776480in}{0.570753in}}%
\pgfpathlineto{\pgfqpoint{5.780200in}{0.565433in}}%
\pgfpathlineto{\pgfqpoint{5.782680in}{0.566770in}}%
\pgfpathlineto{\pgfqpoint{5.785160in}{0.564119in}}%
\pgfpathlineto{\pgfqpoint{5.788880in}{0.564080in}}%
\pgfpathlineto{\pgfqpoint{5.793840in}{0.560817in}}%
\pgfpathlineto{\pgfqpoint{5.795080in}{0.558791in}}%
\pgfpathlineto{\pgfqpoint{5.798800in}{0.560548in}}%
\pgfpathlineto{\pgfqpoint{5.800040in}{0.560003in}}%
\pgfpathlineto{\pgfqpoint{5.803760in}{0.564462in}}%
\pgfpathlineto{\pgfqpoint{5.805000in}{0.565017in}}%
\pgfpathlineto{\pgfqpoint{5.807480in}{0.564539in}}%
\pgfpathlineto{\pgfqpoint{5.809960in}{0.564293in}}%
\pgfpathlineto{\pgfqpoint{5.811200in}{0.563450in}}%
\pgfpathlineto{\pgfqpoint{5.813680in}{0.564452in}}%
\pgfpathlineto{\pgfqpoint{5.814920in}{0.563533in}}%
\pgfpathlineto{\pgfqpoint{5.816160in}{0.564353in}}%
\pgfpathlineto{\pgfqpoint{5.818640in}{0.564263in}}%
\pgfpathlineto{\pgfqpoint{5.819880in}{0.564302in}}%
\pgfpathlineto{\pgfqpoint{5.822360in}{0.563423in}}%
\pgfpathlineto{\pgfqpoint{5.824840in}{0.564586in}}%
\pgfpathlineto{\pgfqpoint{5.833520in}{0.567374in}}%
\pgfpathlineto{\pgfqpoint{5.836000in}{0.564928in}}%
\pgfpathlineto{\pgfqpoint{5.840960in}{0.567703in}}%
\pgfpathlineto{\pgfqpoint{5.842200in}{0.567900in}}%
\pgfpathlineto{\pgfqpoint{5.845920in}{0.570724in}}%
\pgfpathlineto{\pgfqpoint{5.847160in}{0.569684in}}%
\pgfpathlineto{\pgfqpoint{5.848400in}{0.570702in}}%
\pgfpathlineto{\pgfqpoint{5.850880in}{0.576419in}}%
\pgfpathlineto{\pgfqpoint{5.852120in}{0.577180in}}%
\pgfpathlineto{\pgfqpoint{5.855840in}{0.574112in}}%
\pgfpathlineto{\pgfqpoint{5.857080in}{0.574339in}}%
\pgfpathlineto{\pgfqpoint{5.859560in}{0.572205in}}%
\pgfpathlineto{\pgfqpoint{5.860800in}{0.572512in}}%
\pgfpathlineto{\pgfqpoint{5.863280in}{0.570690in}}%
\pgfpathlineto{\pgfqpoint{5.865760in}{0.572493in}}%
\pgfpathlineto{\pgfqpoint{5.868240in}{0.570092in}}%
\pgfpathlineto{\pgfqpoint{5.870720in}{0.571632in}}%
\pgfpathlineto{\pgfqpoint{5.873200in}{0.571012in}}%
\pgfpathlineto{\pgfqpoint{5.878160in}{0.572015in}}%
\pgfpathlineto{\pgfqpoint{5.879400in}{0.571214in}}%
\pgfpathlineto{\pgfqpoint{5.880640in}{0.572012in}}%
\pgfpathlineto{\pgfqpoint{5.883120in}{0.569951in}}%
\pgfpathlineto{\pgfqpoint{5.884360in}{0.570812in}}%
\pgfpathlineto{\pgfqpoint{5.888080in}{0.569296in}}%
\pgfpathlineto{\pgfqpoint{5.890560in}{0.570816in}}%
\pgfpathlineto{\pgfqpoint{5.893040in}{0.570001in}}%
\pgfpathlineto{\pgfqpoint{5.894280in}{0.567847in}}%
\pgfpathlineto{\pgfqpoint{5.895520in}{0.568432in}}%
\pgfpathlineto{\pgfqpoint{5.898000in}{0.570609in}}%
\pgfpathlineto{\pgfqpoint{5.900480in}{0.572560in}}%
\pgfpathlineto{\pgfqpoint{5.904200in}{0.567381in}}%
\pgfpathlineto{\pgfqpoint{5.906680in}{0.568630in}}%
\pgfpathlineto{\pgfqpoint{5.909160in}{0.565796in}}%
\pgfpathlineto{\pgfqpoint{5.911640in}{0.565643in}}%
\pgfpathlineto{\pgfqpoint{5.916600in}{0.564270in}}%
\pgfpathlineto{\pgfqpoint{5.919080in}{0.560460in}}%
\pgfpathlineto{\pgfqpoint{5.922800in}{0.562970in}}%
\pgfpathlineto{\pgfqpoint{5.924040in}{0.562604in}}%
\pgfpathlineto{\pgfqpoint{5.929000in}{0.567767in}}%
\pgfpathlineto{\pgfqpoint{5.931480in}{0.566910in}}%
\pgfpathlineto{\pgfqpoint{5.936440in}{0.566057in}}%
\pgfpathlineto{\pgfqpoint{5.940160in}{0.567423in}}%
\pgfpathlineto{\pgfqpoint{5.941400in}{0.567010in}}%
\pgfpathlineto{\pgfqpoint{5.943880in}{0.567373in}}%
\pgfpathlineto{\pgfqpoint{5.946360in}{0.566907in}}%
\pgfpathlineto{\pgfqpoint{5.948840in}{0.567885in}}%
\pgfpathlineto{\pgfqpoint{5.955040in}{0.567608in}}%
\pgfpathlineto{\pgfqpoint{5.956280in}{0.569749in}}%
\pgfpathlineto{\pgfqpoint{5.957520in}{0.569613in}}%
\pgfpathlineto{\pgfqpoint{5.960000in}{0.567509in}}%
\pgfpathlineto{\pgfqpoint{5.964960in}{0.571704in}}%
\pgfpathlineto{\pgfqpoint{5.967440in}{0.572997in}}%
\pgfpathlineto{\pgfqpoint{5.972400in}{0.574019in}}%
\pgfpathlineto{\pgfqpoint{5.974880in}{0.579962in}}%
\pgfpathlineto{\pgfqpoint{5.976120in}{0.581228in}}%
\pgfpathlineto{\pgfqpoint{5.979840in}{0.576726in}}%
\pgfpathlineto{\pgfqpoint{5.982320in}{0.575832in}}%
\pgfpathlineto{\pgfqpoint{5.983560in}{0.574912in}}%
\pgfpathlineto{\pgfqpoint{5.984800in}{0.575613in}}%
\pgfpathlineto{\pgfqpoint{5.987280in}{0.573361in}}%
\pgfpathlineto{\pgfqpoint{5.989760in}{0.576059in}}%
\pgfpathlineto{\pgfqpoint{5.992240in}{0.574518in}}%
\pgfpathlineto{\pgfqpoint{5.997200in}{0.576824in}}%
\pgfpathlineto{\pgfqpoint{6.000920in}{0.576653in}}%
\pgfpathlineto{\pgfqpoint{6.002160in}{0.577733in}}%
\pgfpathlineto{\pgfqpoint{6.003400in}{0.576767in}}%
\pgfpathlineto{\pgfqpoint{6.004640in}{0.577694in}}%
\pgfpathlineto{\pgfqpoint{6.008360in}{0.576801in}}%
\pgfpathlineto{\pgfqpoint{6.012080in}{0.576495in}}%
\pgfpathlineto{\pgfqpoint{6.014560in}{0.577899in}}%
\pgfpathlineto{\pgfqpoint{6.018280in}{0.573807in}}%
\pgfpathlineto{\pgfqpoint{6.024480in}{0.578695in}}%
\pgfpathlineto{\pgfqpoint{6.028200in}{0.572701in}}%
\pgfpathlineto{\pgfqpoint{6.030680in}{0.573693in}}%
\pgfpathlineto{\pgfqpoint{6.033160in}{0.571539in}}%
\pgfpathlineto{\pgfqpoint{6.034400in}{0.572190in}}%
\pgfpathlineto{\pgfqpoint{6.038120in}{0.570837in}}%
\pgfpathlineto{\pgfqpoint{6.039360in}{0.571879in}}%
\pgfpathlineto{\pgfqpoint{6.043080in}{0.567178in}}%
\pgfpathlineto{\pgfqpoint{6.045560in}{0.568462in}}%
\pgfpathlineto{\pgfqpoint{6.046800in}{0.569305in}}%
\pgfpathlineto{\pgfqpoint{6.048040in}{0.568398in}}%
\pgfpathlineto{\pgfqpoint{6.055480in}{0.572449in}}%
\pgfpathlineto{\pgfqpoint{6.056720in}{0.572872in}}%
\pgfpathlineto{\pgfqpoint{6.059200in}{0.571996in}}%
\pgfpathlineto{\pgfqpoint{6.064160in}{0.574611in}}%
\pgfpathlineto{\pgfqpoint{6.069120in}{0.572429in}}%
\pgfpathlineto{\pgfqpoint{6.075320in}{0.573718in}}%
\pgfpathlineto{\pgfqpoint{6.079040in}{0.573108in}}%
\pgfpathlineto{\pgfqpoint{6.080280in}{0.575716in}}%
\pgfpathlineto{\pgfqpoint{6.082760in}{0.573990in}}%
\pgfpathlineto{\pgfqpoint{6.085240in}{0.573813in}}%
\pgfpathlineto{\pgfqpoint{6.086480in}{0.574117in}}%
\pgfpathlineto{\pgfqpoint{6.088960in}{0.576177in}}%
\pgfpathlineto{\pgfqpoint{6.093920in}{0.578001in}}%
\pgfpathlineto{\pgfqpoint{6.095160in}{0.577704in}}%
\pgfpathlineto{\pgfqpoint{6.096400in}{0.578637in}}%
\pgfpathlineto{\pgfqpoint{6.098880in}{0.583348in}}%
\pgfpathlineto{\pgfqpoint{6.100120in}{0.584429in}}%
\pgfpathlineto{\pgfqpoint{6.102600in}{0.580837in}}%
\pgfpathlineto{\pgfqpoint{6.106320in}{0.578284in}}%
\pgfpathlineto{\pgfqpoint{6.111280in}{0.576681in}}%
\pgfpathlineto{\pgfqpoint{6.113760in}{0.578493in}}%
\pgfpathlineto{\pgfqpoint{6.117480in}{0.576887in}}%
\pgfpathlineto{\pgfqpoint{6.122440in}{0.578902in}}%
\pgfpathlineto{\pgfqpoint{6.124920in}{0.578343in}}%
\pgfpathlineto{\pgfqpoint{6.126160in}{0.579751in}}%
\pgfpathlineto{\pgfqpoint{6.127400in}{0.578561in}}%
\pgfpathlineto{\pgfqpoint{6.128640in}{0.579884in}}%
\pgfpathlineto{\pgfqpoint{6.131120in}{0.577330in}}%
\pgfpathlineto{\pgfqpoint{6.132360in}{0.579260in}}%
\pgfpathlineto{\pgfqpoint{6.136080in}{0.578234in}}%
\pgfpathlineto{\pgfqpoint{6.138560in}{0.580120in}}%
\pgfpathlineto{\pgfqpoint{6.143520in}{0.576599in}}%
\pgfpathlineto{\pgfqpoint{6.147240in}{0.580556in}}%
\pgfpathlineto{\pgfqpoint{6.148480in}{0.581591in}}%
\pgfpathlineto{\pgfqpoint{6.152200in}{0.575111in}}%
\pgfpathlineto{\pgfqpoint{6.154680in}{0.577131in}}%
\pgfpathlineto{\pgfqpoint{6.157160in}{0.574840in}}%
\pgfpathlineto{\pgfqpoint{6.159640in}{0.574693in}}%
\pgfpathlineto{\pgfqpoint{6.163360in}{0.574379in}}%
\pgfpathlineto{\pgfqpoint{6.165840in}{0.572014in}}%
\pgfpathlineto{\pgfqpoint{6.167080in}{0.569271in}}%
\pgfpathlineto{\pgfqpoint{6.174520in}{0.571800in}}%
\pgfpathlineto{\pgfqpoint{6.177000in}{0.575051in}}%
\pgfpathlineto{\pgfqpoint{6.178240in}{0.574240in}}%
\pgfpathlineto{\pgfqpoint{6.180720in}{0.574964in}}%
\pgfpathlineto{\pgfqpoint{6.183200in}{0.575256in}}%
\pgfpathlineto{\pgfqpoint{6.190640in}{0.576711in}}%
\pgfpathlineto{\pgfqpoint{6.193120in}{0.573722in}}%
\pgfpathlineto{\pgfqpoint{6.200560in}{0.574846in}}%
\pgfpathlineto{\pgfqpoint{6.203040in}{0.573513in}}%
\pgfpathlineto{\pgfqpoint{6.204280in}{0.575714in}}%
\pgfpathlineto{\pgfqpoint{6.206760in}{0.574275in}}%
\pgfpathlineto{\pgfqpoint{6.208000in}{0.572637in}}%
\pgfpathlineto{\pgfqpoint{6.211720in}{0.577042in}}%
\pgfpathlineto{\pgfqpoint{6.214200in}{0.577348in}}%
\pgfpathlineto{\pgfqpoint{6.220400in}{0.578516in}}%
\pgfpathlineto{\pgfqpoint{6.222880in}{0.583807in}}%
\pgfpathlineto{\pgfqpoint{6.224120in}{0.585320in}}%
\pgfpathlineto{\pgfqpoint{6.226600in}{0.580781in}}%
\pgfpathlineto{\pgfqpoint{6.227840in}{0.579766in}}%
\pgfpathlineto{\pgfqpoint{6.230320in}{0.580135in}}%
\pgfpathlineto{\pgfqpoint{6.234040in}{0.579862in}}%
\pgfpathlineto{\pgfqpoint{6.235280in}{0.577137in}}%
\pgfpathlineto{\pgfqpoint{6.236520in}{0.577132in}}%
\pgfpathlineto{\pgfqpoint{6.237760in}{0.578598in}}%
\pgfpathlineto{\pgfqpoint{6.241480in}{0.576283in}}%
\pgfpathlineto{\pgfqpoint{6.246440in}{0.578621in}}%
\pgfpathlineto{\pgfqpoint{6.248920in}{0.577246in}}%
\pgfpathlineto{\pgfqpoint{6.250160in}{0.579347in}}%
\pgfpathlineto{\pgfqpoint{6.251400in}{0.578628in}}%
\pgfpathlineto{\pgfqpoint{6.252640in}{0.580755in}}%
\pgfpathlineto{\pgfqpoint{6.255120in}{0.577439in}}%
\pgfpathlineto{\pgfqpoint{6.256360in}{0.580022in}}%
\pgfpathlineto{\pgfqpoint{6.260080in}{0.578510in}}%
\pgfpathlineto{\pgfqpoint{6.262560in}{0.580545in}}%
\pgfpathlineto{\pgfqpoint{6.267520in}{0.577922in}}%
\pgfpathlineto{\pgfqpoint{6.272480in}{0.582611in}}%
\pgfpathlineto{\pgfqpoint{6.276200in}{0.575714in}}%
\pgfpathlineto{\pgfqpoint{6.278680in}{0.576422in}}%
\pgfpathlineto{\pgfqpoint{6.281160in}{0.574909in}}%
\pgfpathlineto{\pgfqpoint{6.282400in}{0.576227in}}%
\pgfpathlineto{\pgfqpoint{6.284880in}{0.575597in}}%
\pgfpathlineto{\pgfqpoint{6.286120in}{0.573737in}}%
\pgfpathlineto{\pgfqpoint{6.287360in}{0.574290in}}%
\pgfpathlineto{\pgfqpoint{6.292320in}{0.569920in}}%
\pgfpathlineto{\pgfqpoint{6.296040in}{0.570325in}}%
\pgfpathlineto{\pgfqpoint{6.298520in}{0.571081in}}%
\pgfpathlineto{\pgfqpoint{6.301000in}{0.575000in}}%
\pgfpathlineto{\pgfqpoint{6.302240in}{0.574793in}}%
\pgfpathlineto{\pgfqpoint{6.304720in}{0.576189in}}%
\pgfpathlineto{\pgfqpoint{6.308440in}{0.576228in}}%
\pgfpathlineto{\pgfqpoint{6.309680in}{0.577827in}}%
\pgfpathlineto{\pgfqpoint{6.310920in}{0.577235in}}%
\pgfpathlineto{\pgfqpoint{6.314640in}{0.579087in}}%
\pgfpathlineto{\pgfqpoint{6.318360in}{0.574876in}}%
\pgfpathlineto{\pgfqpoint{6.322080in}{0.576599in}}%
\pgfpathlineto{\pgfqpoint{6.323320in}{0.575226in}}%
\pgfpathlineto{\pgfqpoint{6.325800in}{0.576029in}}%
\pgfpathlineto{\pgfqpoint{6.327040in}{0.575843in}}%
\pgfpathlineto{\pgfqpoint{6.328280in}{0.578385in}}%
\pgfpathlineto{\pgfqpoint{6.332000in}{0.575616in}}%
\pgfpathlineto{\pgfqpoint{6.335720in}{0.580441in}}%
\pgfpathlineto{\pgfqpoint{6.338200in}{0.581426in}}%
\pgfpathlineto{\pgfqpoint{6.341920in}{0.583642in}}%
\pgfpathlineto{\pgfqpoint{6.344400in}{0.583378in}}%
\pgfpathlineto{\pgfqpoint{6.346880in}{0.587227in}}%
\pgfpathlineto{\pgfqpoint{6.348120in}{0.589275in}}%
\pgfpathlineto{\pgfqpoint{6.350600in}{0.584026in}}%
\pgfpathlineto{\pgfqpoint{6.353080in}{0.582381in}}%
\pgfpathlineto{\pgfqpoint{6.358040in}{0.581877in}}%
\pgfpathlineto{\pgfqpoint{6.359280in}{0.579155in}}%
\pgfpathlineto{\pgfqpoint{6.360520in}{0.579948in}}%
\pgfpathlineto{\pgfqpoint{6.361760in}{0.582130in}}%
\pgfpathlineto{\pgfqpoint{6.367960in}{0.579545in}}%
\pgfpathlineto{\pgfqpoint{6.370440in}{0.580606in}}%
\pgfpathlineto{\pgfqpoint{6.372920in}{0.577804in}}%
\pgfpathlineto{\pgfqpoint{6.374160in}{0.578956in}}%
\pgfpathlineto{\pgfqpoint{6.375400in}{0.578172in}}%
\pgfpathlineto{\pgfqpoint{6.376640in}{0.580496in}}%
\pgfpathlineto{\pgfqpoint{6.379120in}{0.577627in}}%
\pgfpathlineto{\pgfqpoint{6.380360in}{0.580132in}}%
\pgfpathlineto{\pgfqpoint{6.382840in}{0.578803in}}%
\pgfpathlineto{\pgfqpoint{6.389040in}{0.580708in}}%
\pgfpathlineto{\pgfqpoint{6.390280in}{0.578529in}}%
\pgfpathlineto{\pgfqpoint{6.394000in}{0.581504in}}%
\pgfpathlineto{\pgfqpoint{6.396480in}{0.584485in}}%
\pgfpathlineto{\pgfqpoint{6.400200in}{0.575823in}}%
\pgfpathlineto{\pgfqpoint{6.405160in}{0.574518in}}%
\pgfpathlineto{\pgfqpoint{6.406400in}{0.576850in}}%
\pgfpathlineto{\pgfqpoint{6.408880in}{0.575678in}}%
\pgfpathlineto{\pgfqpoint{6.410120in}{0.574032in}}%
\pgfpathlineto{\pgfqpoint{6.411360in}{0.574741in}}%
\pgfpathlineto{\pgfqpoint{6.415080in}{0.569329in}}%
\pgfpathlineto{\pgfqpoint{6.427480in}{0.578137in}}%
\pgfpathlineto{\pgfqpoint{6.434920in}{0.578581in}}%
\pgfpathlineto{\pgfqpoint{6.438640in}{0.577979in}}%
\pgfpathlineto{\pgfqpoint{6.441120in}{0.573558in}}%
\pgfpathlineto{\pgfqpoint{6.442360in}{0.573749in}}%
\pgfpathlineto{\pgfqpoint{6.444840in}{0.575614in}}%
\pgfpathlineto{\pgfqpoint{6.446080in}{0.576005in}}%
\pgfpathlineto{\pgfqpoint{6.447320in}{0.574598in}}%
\pgfpathlineto{\pgfqpoint{6.449800in}{0.575337in}}%
\pgfpathlineto{\pgfqpoint{6.451040in}{0.575272in}}%
\pgfpathlineto{\pgfqpoint{6.452280in}{0.577971in}}%
\pgfpathlineto{\pgfqpoint{6.456000in}{0.574291in}}%
\pgfpathlineto{\pgfqpoint{6.457240in}{0.571254in}}%
\pgfpathlineto{\pgfqpoint{6.462200in}{0.574450in}}%
\pgfpathlineto{\pgfqpoint{6.464680in}{0.577076in}}%
\pgfpathlineto{\pgfqpoint{6.468400in}{0.575953in}}%
\pgfpathlineto{\pgfqpoint{6.472120in}{0.583580in}}%
\pgfpathlineto{\pgfqpoint{6.474600in}{0.580088in}}%
\pgfpathlineto{\pgfqpoint{6.475840in}{0.579422in}}%
\pgfpathlineto{\pgfqpoint{6.480800in}{0.581768in}}%
\pgfpathlineto{\pgfqpoint{6.482040in}{0.581127in}}%
\pgfpathlineto{\pgfqpoint{6.483280in}{0.578189in}}%
\pgfpathlineto{\pgfqpoint{6.485760in}{0.581563in}}%
\pgfpathlineto{\pgfqpoint{6.491960in}{0.577091in}}%
\pgfpathlineto{\pgfqpoint{6.494440in}{0.578593in}}%
\pgfpathlineto{\pgfqpoint{6.495680in}{0.576005in}}%
\pgfpathlineto{\pgfqpoint{6.499400in}{0.575824in}}%
\pgfpathlineto{\pgfqpoint{6.501880in}{0.578484in}}%
\pgfpathlineto{\pgfqpoint{6.503120in}{0.575644in}}%
\pgfpathlineto{\pgfqpoint{6.504360in}{0.577212in}}%
\pgfpathlineto{\pgfqpoint{6.508080in}{0.574829in}}%
\pgfpathlineto{\pgfqpoint{6.511800in}{0.579149in}}%
\pgfpathlineto{\pgfqpoint{6.513040in}{0.578940in}}%
\pgfpathlineto{\pgfqpoint{6.514280in}{0.576601in}}%
\pgfpathlineto{\pgfqpoint{6.518000in}{0.578104in}}%
\pgfpathlineto{\pgfqpoint{6.520480in}{0.580892in}}%
\pgfpathlineto{\pgfqpoint{6.524200in}{0.570889in}}%
\pgfpathlineto{\pgfqpoint{6.529160in}{0.567057in}}%
\pgfpathlineto{\pgfqpoint{6.531640in}{0.569564in}}%
\pgfpathlineto{\pgfqpoint{6.537840in}{0.564915in}}%
\pgfpathlineto{\pgfqpoint{6.539080in}{0.563560in}}%
\pgfpathlineto{\pgfqpoint{6.541560in}{0.565239in}}%
\pgfpathlineto{\pgfqpoint{6.542800in}{0.567297in}}%
\pgfpathlineto{\pgfqpoint{6.544040in}{0.567001in}}%
\pgfpathlineto{\pgfqpoint{6.549000in}{0.573748in}}%
\pgfpathlineto{\pgfqpoint{6.550240in}{0.573244in}}%
\pgfpathlineto{\pgfqpoint{6.552720in}{0.575163in}}%
\pgfpathlineto{\pgfqpoint{6.556440in}{0.574822in}}%
\pgfpathlineto{\pgfqpoint{6.558920in}{0.575889in}}%
\pgfpathlineto{\pgfqpoint{6.562640in}{0.575304in}}%
\pgfpathlineto{\pgfqpoint{6.566360in}{0.571455in}}%
\pgfpathlineto{\pgfqpoint{6.568840in}{0.572553in}}%
\pgfpathlineto{\pgfqpoint{6.570080in}{0.572906in}}%
\pgfpathlineto{\pgfqpoint{6.571320in}{0.571621in}}%
\pgfpathlineto{\pgfqpoint{6.575040in}{0.572270in}}%
\pgfpathlineto{\pgfqpoint{6.576280in}{0.574721in}}%
\pgfpathlineto{\pgfqpoint{6.580000in}{0.570153in}}%
\pgfpathlineto{\pgfqpoint{6.581240in}{0.569980in}}%
\pgfpathlineto{\pgfqpoint{6.584960in}{0.572429in}}%
\pgfpathlineto{\pgfqpoint{6.587440in}{0.574017in}}%
\pgfpathlineto{\pgfqpoint{6.589920in}{0.576084in}}%
\pgfpathlineto{\pgfqpoint{6.592400in}{0.577040in}}%
\pgfpathlineto{\pgfqpoint{6.594880in}{0.580768in}}%
\pgfpathlineto{\pgfqpoint{6.596120in}{0.582315in}}%
\pgfpathlineto{\pgfqpoint{6.598600in}{0.576808in}}%
\pgfpathlineto{\pgfqpoint{6.601080in}{0.578033in}}%
\pgfpathlineto{\pgfqpoint{6.603560in}{0.579613in}}%
\pgfpathlineto{\pgfqpoint{6.606040in}{0.579303in}}%
\pgfpathlineto{\pgfqpoint{6.607280in}{0.576479in}}%
\pgfpathlineto{\pgfqpoint{6.609760in}{0.581674in}}%
\pgfpathlineto{\pgfqpoint{6.614720in}{0.577197in}}%
\pgfpathlineto{\pgfqpoint{6.615960in}{0.577016in}}%
\pgfpathlineto{\pgfqpoint{6.618440in}{0.578847in}}%
\pgfpathlineto{\pgfqpoint{6.619680in}{0.576150in}}%
\pgfpathlineto{\pgfqpoint{6.622160in}{0.577381in}}%
\pgfpathlineto{\pgfqpoint{6.623400in}{0.575385in}}%
\pgfpathlineto{\pgfqpoint{6.625880in}{0.579288in}}%
\pgfpathlineto{\pgfqpoint{6.627120in}{0.575932in}}%
\pgfpathlineto{\pgfqpoint{6.628360in}{0.576668in}}%
\pgfpathlineto{\pgfqpoint{6.632080in}{0.571567in}}%
\pgfpathlineto{\pgfqpoint{6.634560in}{0.574206in}}%
\pgfpathlineto{\pgfqpoint{6.637040in}{0.573383in}}%
\pgfpathlineto{\pgfqpoint{6.638280in}{0.570807in}}%
\pgfpathlineto{\pgfqpoint{6.643240in}{0.575362in}}%
\pgfpathlineto{\pgfqpoint{6.644480in}{0.574608in}}%
\pgfpathlineto{\pgfqpoint{6.648200in}{0.564387in}}%
\pgfpathlineto{\pgfqpoint{6.651920in}{0.564206in}}%
\pgfpathlineto{\pgfqpoint{6.653160in}{0.563023in}}%
\pgfpathlineto{\pgfqpoint{6.654400in}{0.565347in}}%
\pgfpathlineto{\pgfqpoint{6.656880in}{0.564334in}}%
\pgfpathlineto{\pgfqpoint{6.658120in}{0.562295in}}%
\pgfpathlineto{\pgfqpoint{6.659360in}{0.562902in}}%
\pgfpathlineto{\pgfqpoint{6.664320in}{0.558134in}}%
\pgfpathlineto{\pgfqpoint{6.666800in}{0.561184in}}%
\pgfpathlineto{\pgfqpoint{6.668040in}{0.560137in}}%
\pgfpathlineto{\pgfqpoint{6.670520in}{0.562909in}}%
\pgfpathlineto{\pgfqpoint{6.673000in}{0.568472in}}%
\pgfpathlineto{\pgfqpoint{6.682920in}{0.572439in}}%
\pgfpathlineto{\pgfqpoint{6.687880in}{0.568851in}}%
\pgfpathlineto{\pgfqpoint{6.689120in}{0.565293in}}%
\pgfpathlineto{\pgfqpoint{6.690360in}{0.565483in}}%
\pgfpathlineto{\pgfqpoint{6.692840in}{0.567861in}}%
\pgfpathlineto{\pgfqpoint{6.694080in}{0.568521in}}%
\pgfpathlineto{\pgfqpoint{6.695320in}{0.565666in}}%
\pgfpathlineto{\pgfqpoint{6.699040in}{0.565702in}}%
\pgfpathlineto{\pgfqpoint{6.700280in}{0.568422in}}%
\pgfpathlineto{\pgfqpoint{6.704000in}{0.563654in}}%
\pgfpathlineto{\pgfqpoint{6.706480in}{0.566331in}}%
\pgfpathlineto{\pgfqpoint{6.708960in}{0.565825in}}%
\pgfpathlineto{\pgfqpoint{6.715160in}{0.571359in}}%
\pgfpathlineto{\pgfqpoint{6.716400in}{0.570554in}}%
\pgfpathlineto{\pgfqpoint{6.718880in}{0.575700in}}%
\pgfpathlineto{\pgfqpoint{6.720120in}{0.577128in}}%
\pgfpathlineto{\pgfqpoint{6.722600in}{0.571730in}}%
\pgfpathlineto{\pgfqpoint{6.723840in}{0.572546in}}%
\pgfpathlineto{\pgfqpoint{6.727560in}{0.577093in}}%
\pgfpathlineto{\pgfqpoint{6.730040in}{0.578908in}}%
\pgfpathlineto{\pgfqpoint{6.731280in}{0.577835in}}%
\pgfpathlineto{\pgfqpoint{6.733760in}{0.584527in}}%
\pgfpathlineto{\pgfqpoint{6.735000in}{0.583508in}}%
\pgfpathlineto{\pgfqpoint{6.737480in}{0.579626in}}%
\pgfpathlineto{\pgfqpoint{6.742440in}{0.580441in}}%
\pgfpathlineto{\pgfqpoint{6.743680in}{0.578050in}}%
\pgfpathlineto{\pgfqpoint{6.746160in}{0.582939in}}%
\pgfpathlineto{\pgfqpoint{6.747400in}{0.582024in}}%
\pgfpathlineto{\pgfqpoint{6.749880in}{0.586760in}}%
\pgfpathlineto{\pgfqpoint{6.751120in}{0.584099in}}%
\pgfpathlineto{\pgfqpoint{6.752360in}{0.584352in}}%
\pgfpathlineto{\pgfqpoint{6.756080in}{0.577015in}}%
\pgfpathlineto{\pgfqpoint{6.757320in}{0.576974in}}%
\pgfpathlineto{\pgfqpoint{6.758560in}{0.578781in}}%
\pgfpathlineto{\pgfqpoint{6.761040in}{0.577507in}}%
\pgfpathlineto{\pgfqpoint{6.762280in}{0.574955in}}%
\pgfpathlineto{\pgfqpoint{6.766000in}{0.577314in}}%
\pgfpathlineto{\pgfqpoint{6.767240in}{0.579305in}}%
\pgfpathlineto{\pgfqpoint{6.768480in}{0.576788in}}%
\pgfpathlineto{\pgfqpoint{6.770960in}{0.568391in}}%
\pgfpathlineto{\pgfqpoint{6.773440in}{0.566422in}}%
\pgfpathlineto{\pgfqpoint{6.774680in}{0.566413in}}%
\pgfpathlineto{\pgfqpoint{6.777160in}{0.562807in}}%
\pgfpathlineto{\pgfqpoint{6.779640in}{0.565303in}}%
\pgfpathlineto{\pgfqpoint{6.782120in}{0.564330in}}%
\pgfpathlineto{\pgfqpoint{6.785840in}{0.560043in}}%
\pgfpathlineto{\pgfqpoint{6.787080in}{0.557376in}}%
\pgfpathlineto{\pgfqpoint{6.790800in}{0.561491in}}%
\pgfpathlineto{\pgfqpoint{6.792040in}{0.559492in}}%
\pgfpathlineto{\pgfqpoint{6.795760in}{0.565387in}}%
\pgfpathlineto{\pgfqpoint{6.798240in}{0.567870in}}%
\pgfpathlineto{\pgfqpoint{6.799480in}{0.567756in}}%
\pgfpathlineto{\pgfqpoint{6.801960in}{0.570069in}}%
\pgfpathlineto{\pgfqpoint{6.803200in}{0.569664in}}%
\pgfpathlineto{\pgfqpoint{6.805680in}{0.572372in}}%
\pgfpathlineto{\pgfqpoint{6.806920in}{0.572798in}}%
\pgfpathlineto{\pgfqpoint{6.809400in}{0.570120in}}%
\pgfpathlineto{\pgfqpoint{6.810640in}{0.569685in}}%
\pgfpathlineto{\pgfqpoint{6.814360in}{0.564134in}}%
\pgfpathlineto{\pgfqpoint{6.815600in}{0.566305in}}%
\pgfpathlineto{\pgfqpoint{6.816840in}{0.566199in}}%
\pgfpathlineto{\pgfqpoint{6.818080in}{0.567930in}}%
\pgfpathlineto{\pgfqpoint{6.820560in}{0.567490in}}%
\pgfpathlineto{\pgfqpoint{6.823040in}{0.566277in}}%
\pgfpathlineto{\pgfqpoint{6.824280in}{0.568039in}}%
\pgfpathlineto{\pgfqpoint{6.825520in}{0.567486in}}%
\pgfpathlineto{\pgfqpoint{6.828000in}{0.563276in}}%
\pgfpathlineto{\pgfqpoint{6.829240in}{0.568036in}}%
\pgfpathlineto{\pgfqpoint{6.832960in}{0.562674in}}%
\pgfpathlineto{\pgfqpoint{6.839160in}{0.571050in}}%
\pgfpathlineto{\pgfqpoint{6.840400in}{0.569431in}}%
\pgfpathlineto{\pgfqpoint{6.842880in}{0.575876in}}%
\pgfpathlineto{\pgfqpoint{6.844120in}{0.577881in}}%
\pgfpathlineto{\pgfqpoint{6.845360in}{0.572767in}}%
\pgfpathlineto{\pgfqpoint{6.847840in}{0.573624in}}%
\pgfpathlineto{\pgfqpoint{6.850320in}{0.579099in}}%
\pgfpathlineto{\pgfqpoint{6.851560in}{0.578004in}}%
\pgfpathlineto{\pgfqpoint{6.855280in}{0.576922in}}%
\pgfpathlineto{\pgfqpoint{6.857760in}{0.585058in}}%
\pgfpathlineto{\pgfqpoint{6.859000in}{0.583839in}}%
\pgfpathlineto{\pgfqpoint{6.861480in}{0.579777in}}%
\pgfpathlineto{\pgfqpoint{6.863960in}{0.582025in}}%
\pgfpathlineto{\pgfqpoint{6.865200in}{0.583988in}}%
\pgfpathlineto{\pgfqpoint{6.867680in}{0.578570in}}%
\pgfpathlineto{\pgfqpoint{6.870160in}{0.583921in}}%
\pgfpathlineto{\pgfqpoint{6.871400in}{0.584090in}}%
\pgfpathlineto{\pgfqpoint{6.873880in}{0.590950in}}%
\pgfpathlineto{\pgfqpoint{6.880080in}{0.578172in}}%
\pgfpathlineto{\pgfqpoint{6.881320in}{0.577819in}}%
\pgfpathlineto{\pgfqpoint{6.882560in}{0.578975in}}%
\pgfpathlineto{\pgfqpoint{6.886280in}{0.569929in}}%
\pgfpathlineto{\pgfqpoint{6.888760in}{0.573384in}}%
\pgfpathlineto{\pgfqpoint{6.891240in}{0.577891in}}%
\pgfpathlineto{\pgfqpoint{6.892480in}{0.576637in}}%
\pgfpathlineto{\pgfqpoint{6.894960in}{0.572035in}}%
\pgfpathlineto{\pgfqpoint{6.897440in}{0.571089in}}%
\pgfpathlineto{\pgfqpoint{6.898680in}{0.572721in}}%
\pgfpathlineto{\pgfqpoint{6.901160in}{0.568159in}}%
\pgfpathlineto{\pgfqpoint{6.902400in}{0.569933in}}%
\pgfpathlineto{\pgfqpoint{6.907360in}{0.565536in}}%
\pgfpathlineto{\pgfqpoint{6.909840in}{0.565909in}}%
\pgfpathlineto{\pgfqpoint{6.911080in}{0.563031in}}%
\pgfpathlineto{\pgfqpoint{6.912320in}{0.564079in}}%
\pgfpathlineto{\pgfqpoint{6.913560in}{0.562782in}}%
\pgfpathlineto{\pgfqpoint{6.914800in}{0.565206in}}%
\pgfpathlineto{\pgfqpoint{6.916040in}{0.564319in}}%
\pgfpathlineto{\pgfqpoint{6.923480in}{0.572629in}}%
\pgfpathlineto{\pgfqpoint{6.927200in}{0.570137in}}%
\pgfpathlineto{\pgfqpoint{6.929680in}{0.572691in}}%
\pgfpathlineto{\pgfqpoint{6.933400in}{0.566444in}}%
\pgfpathlineto{\pgfqpoint{6.935880in}{0.565216in}}%
\pgfpathlineto{\pgfqpoint{6.938360in}{0.562325in}}%
\pgfpathlineto{\pgfqpoint{6.939600in}{0.563534in}}%
\pgfpathlineto{\pgfqpoint{6.940840in}{0.561416in}}%
\pgfpathlineto{\pgfqpoint{6.942080in}{0.562402in}}%
\pgfpathlineto{\pgfqpoint{6.944560in}{0.561508in}}%
\pgfpathlineto{\pgfqpoint{6.947040in}{0.564754in}}%
\pgfpathlineto{\pgfqpoint{6.948280in}{0.567958in}}%
\pgfpathlineto{\pgfqpoint{6.949520in}{0.567936in}}%
\pgfpathlineto{\pgfqpoint{6.952000in}{0.560954in}}%
\pgfpathlineto{\pgfqpoint{6.953240in}{0.568909in}}%
\pgfpathlineto{\pgfqpoint{6.955720in}{0.563282in}}%
\pgfpathlineto{\pgfqpoint{6.956960in}{0.559391in}}%
\pgfpathlineto{\pgfqpoint{6.960680in}{0.569415in}}%
\pgfpathlineto{\pgfqpoint{6.961920in}{0.569286in}}%
\pgfpathlineto{\pgfqpoint{6.963160in}{0.572447in}}%
\pgfpathlineto{\pgfqpoint{6.964400in}{0.572553in}}%
\pgfpathlineto{\pgfqpoint{6.966880in}{0.576352in}}%
\pgfpathlineto{\pgfqpoint{6.968120in}{0.577594in}}%
\pgfpathlineto{\pgfqpoint{6.970600in}{0.569373in}}%
\pgfpathlineto{\pgfqpoint{6.973080in}{0.572299in}}%
\pgfpathlineto{\pgfqpoint{6.974320in}{0.575225in}}%
\pgfpathlineto{\pgfqpoint{6.976800in}{0.569610in}}%
\pgfpathlineto{\pgfqpoint{6.979280in}{0.571718in}}%
\pgfpathlineto{\pgfqpoint{6.981760in}{0.578445in}}%
\pgfpathlineto{\pgfqpoint{6.985480in}{0.568123in}}%
\pgfpathlineto{\pgfqpoint{6.987960in}{0.569212in}}%
\pgfpathlineto{\pgfqpoint{6.989200in}{0.574391in}}%
\pgfpathlineto{\pgfqpoint{6.991680in}{0.570419in}}%
\pgfpathlineto{\pgfqpoint{6.997880in}{0.582447in}}%
\pgfpathlineto{\pgfqpoint{6.999120in}{0.579958in}}%
\pgfpathlineto{\pgfqpoint{7.000360in}{0.581977in}}%
\pgfpathlineto{\pgfqpoint{7.002840in}{0.575003in}}%
\pgfpathlineto{\pgfqpoint{7.005320in}{0.572147in}}%
\pgfpathlineto{\pgfqpoint{7.006560in}{0.572546in}}%
\pgfpathlineto{\pgfqpoint{7.007800in}{0.569013in}}%
\pgfpathlineto{\pgfqpoint{7.009040in}{0.570420in}}%
\pgfpathlineto{\pgfqpoint{7.010280in}{0.565849in}}%
\pgfpathlineto{\pgfqpoint{7.014000in}{0.573171in}}%
\pgfpathlineto{\pgfqpoint{7.015240in}{0.576946in}}%
\pgfpathlineto{\pgfqpoint{7.016480in}{0.574563in}}%
\pgfpathlineto{\pgfqpoint{7.018960in}{0.568068in}}%
\pgfpathlineto{\pgfqpoint{7.022680in}{0.572992in}}%
\pgfpathlineto{\pgfqpoint{7.023920in}{0.570610in}}%
\pgfpathlineto{\pgfqpoint{7.026400in}{0.573573in}}%
\pgfpathlineto{\pgfqpoint{7.030120in}{0.566959in}}%
\pgfpathlineto{\pgfqpoint{7.033840in}{0.563919in}}%
\pgfpathlineto{\pgfqpoint{7.037560in}{0.563556in}}%
\pgfpathlineto{\pgfqpoint{7.040040in}{0.570937in}}%
\pgfpathlineto{\pgfqpoint{7.041280in}{0.573523in}}%
\pgfpathlineto{\pgfqpoint{7.042520in}{0.573405in}}%
\pgfpathlineto{\pgfqpoint{7.048720in}{0.580895in}}%
\pgfpathlineto{\pgfqpoint{7.052440in}{0.577979in}}%
\pgfpathlineto{\pgfqpoint{7.053680in}{0.579558in}}%
\pgfpathlineto{\pgfqpoint{7.058640in}{0.567782in}}%
\pgfpathlineto{\pgfqpoint{7.059880in}{0.568395in}}%
\pgfpathlineto{\pgfqpoint{7.062360in}{0.563523in}}%
\pgfpathlineto{\pgfqpoint{7.063600in}{0.568032in}}%
\pgfpathlineto{\pgfqpoint{7.066080in}{0.564782in}}%
\pgfpathlineto{\pgfqpoint{7.067320in}{0.565655in}}%
\pgfpathlineto{\pgfqpoint{7.069800in}{0.562599in}}%
\pgfpathlineto{\pgfqpoint{7.071040in}{0.564397in}}%
\pgfpathlineto{\pgfqpoint{7.073520in}{0.570839in}}%
\pgfpathlineto{\pgfqpoint{7.074760in}{0.569699in}}%
\pgfpathlineto{\pgfqpoint{7.076000in}{0.565342in}}%
\pgfpathlineto{\pgfqpoint{7.077240in}{0.580177in}}%
\pgfpathlineto{\pgfqpoint{7.080960in}{0.568593in}}%
\pgfpathlineto{\pgfqpoint{7.083440in}{0.571294in}}%
\pgfpathlineto{\pgfqpoint{7.088400in}{0.581423in}}%
\pgfpathlineto{\pgfqpoint{7.090880in}{0.581433in}}%
\pgfpathlineto{\pgfqpoint{7.092120in}{0.586666in}}%
\pgfpathlineto{\pgfqpoint{7.094600in}{0.575103in}}%
\pgfpathlineto{\pgfqpoint{7.097080in}{0.578631in}}%
\pgfpathlineto{\pgfqpoint{7.098320in}{0.587615in}}%
\pgfpathlineto{\pgfqpoint{7.099560in}{0.586761in}}%
\pgfpathlineto{\pgfqpoint{7.100800in}{0.588365in}}%
\pgfpathlineto{\pgfqpoint{7.103280in}{0.588300in}}%
\pgfpathlineto{\pgfqpoint{7.104520in}{0.594651in}}%
\pgfpathlineto{\pgfqpoint{7.105760in}{0.593411in}}%
\pgfpathlineto{\pgfqpoint{7.109480in}{0.577758in}}%
\pgfpathlineto{\pgfqpoint{7.111960in}{0.581900in}}%
\pgfpathlineto{\pgfqpoint{7.113200in}{0.590427in}}%
\pgfpathlineto{\pgfqpoint{7.116920in}{0.579617in}}%
\pgfpathlineto{\pgfqpoint{7.118160in}{0.580791in}}%
\pgfpathlineto{\pgfqpoint{7.119400in}{0.578897in}}%
\pgfpathlineto{\pgfqpoint{7.121880in}{0.582107in}}%
\pgfpathlineto{\pgfqpoint{7.123120in}{0.581557in}}%
\pgfpathlineto{\pgfqpoint{7.124360in}{0.583818in}}%
\pgfpathlineto{\pgfqpoint{7.125600in}{0.576437in}}%
\pgfpathlineto{\pgfqpoint{7.128080in}{0.583461in}}%
\pgfpathlineto{\pgfqpoint{7.130560in}{0.584730in}}%
\pgfpathlineto{\pgfqpoint{7.134280in}{0.574089in}}%
\pgfpathlineto{\pgfqpoint{7.135520in}{0.574853in}}%
\pgfpathlineto{\pgfqpoint{7.139240in}{0.594400in}}%
\pgfpathlineto{\pgfqpoint{7.142960in}{0.576026in}}%
\pgfpathlineto{\pgfqpoint{7.146680in}{0.572148in}}%
\pgfpathlineto{\pgfqpoint{7.147920in}{0.565975in}}%
\pgfpathlineto{\pgfqpoint{7.150400in}{0.568341in}}%
\pgfpathlineto{\pgfqpoint{7.151640in}{0.566444in}}%
\pgfpathlineto{\pgfqpoint{7.152880in}{0.561357in}}%
\pgfpathlineto{\pgfqpoint{7.154120in}{0.561276in}}%
\pgfpathlineto{\pgfqpoint{7.156600in}{0.572542in}}%
\pgfpathlineto{\pgfqpoint{7.157840in}{0.574691in}}%
\pgfpathlineto{\pgfqpoint{7.159080in}{0.573555in}}%
\pgfpathlineto{\pgfqpoint{7.164040in}{0.589746in}}%
\pgfpathlineto{\pgfqpoint{7.165280in}{0.591567in}}%
\pgfpathlineto{\pgfqpoint{7.167760in}{0.588272in}}%
\pgfpathlineto{\pgfqpoint{7.169000in}{0.588432in}}%
\pgfpathlineto{\pgfqpoint{7.170240in}{0.594836in}}%
\pgfpathlineto{\pgfqpoint{7.171480in}{0.593080in}}%
\pgfpathlineto{\pgfqpoint{7.175200in}{0.598749in}}%
\pgfpathlineto{\pgfqpoint{7.181400in}{0.577817in}}%
\pgfpathlineto{\pgfqpoint{7.182640in}{0.575998in}}%
\pgfpathlineto{\pgfqpoint{7.183880in}{0.583065in}}%
\pgfpathlineto{\pgfqpoint{7.185120in}{0.581884in}}%
\pgfpathlineto{\pgfqpoint{7.186360in}{0.579085in}}%
\pgfpathlineto{\pgfqpoint{7.188840in}{0.589686in}}%
\pgfpathlineto{\pgfqpoint{7.191320in}{0.588228in}}%
\pgfpathlineto{\pgfqpoint{7.193800in}{0.583056in}}%
\pgfpathlineto{\pgfqpoint{7.197520in}{0.590614in}}%
\pgfpathlineto{\pgfqpoint{7.198760in}{0.591301in}}%
\pgfpathlineto{\pgfqpoint{7.200000in}{0.588987in}}%
\pgfpathlineto{\pgfqpoint{7.200000in}{0.588987in}}%
\pgfusepath{stroke}%
\end{pgfscope}%
\begin{pgfscope}%
\pgfpathrectangle{\pgfqpoint{1.000000in}{0.300000in}}{\pgfqpoint{6.200000in}{2.400000in}} %
\pgfusepath{clip}%
\pgfsetrectcap%
\pgfsetroundjoin%
\pgfsetlinewidth{2.007500pt}%
\definecolor{currentstroke}{rgb}{0.000000,0.500000,0.000000}%
\pgfsetstrokecolor{currentstroke}%
\pgfsetdash{}{0pt}%
\pgfpathmoveto{\pgfqpoint{2.241240in}{0.561243in}}%
\pgfpathlineto{\pgfqpoint{6.578760in}{0.561243in}}%
\pgfusepath{stroke}%
\end{pgfscope}%
\begin{pgfscope}%
\pgfpathrectangle{\pgfqpoint{1.000000in}{0.300000in}}{\pgfqpoint{6.200000in}{2.400000in}} %
\pgfusepath{clip}%
\pgfsetrectcap%
\pgfsetroundjoin%
\pgfsetlinewidth{1.003750pt}%
\definecolor{currentstroke}{rgb}{1.000000,0.000000,0.000000}%
\pgfsetstrokecolor{currentstroke}%
\pgfsetdash{}{0pt}%
\pgfpathmoveto{\pgfqpoint{1.001240in}{1.520756in}}%
\pgfpathlineto{\pgfqpoint{1.002480in}{2.151600in}}%
\pgfpathlineto{\pgfqpoint{1.003720in}{2.421570in}}%
\pgfpathlineto{\pgfqpoint{1.004960in}{2.474556in}}%
\pgfpathlineto{\pgfqpoint{1.008680in}{2.422400in}}%
\pgfpathlineto{\pgfqpoint{1.014880in}{2.305800in}}%
\pgfpathlineto{\pgfqpoint{1.017360in}{2.279517in}}%
\pgfpathlineto{\pgfqpoint{1.023560in}{2.216103in}}%
\pgfpathlineto{\pgfqpoint{1.024800in}{2.217151in}}%
\pgfpathlineto{\pgfqpoint{1.027280in}{2.214651in}}%
\pgfpathlineto{\pgfqpoint{1.029760in}{2.199063in}}%
\pgfpathlineto{\pgfqpoint{1.031000in}{2.199534in}}%
\pgfpathlineto{\pgfqpoint{1.032240in}{2.198756in}}%
\pgfpathlineto{\pgfqpoint{1.033480in}{2.195888in}}%
\pgfpathlineto{\pgfqpoint{1.038440in}{2.146624in}}%
\pgfpathlineto{\pgfqpoint{1.042160in}{2.128131in}}%
\pgfpathlineto{\pgfqpoint{1.044640in}{2.134889in}}%
\pgfpathlineto{\pgfqpoint{1.045880in}{2.135387in}}%
\pgfpathlineto{\pgfqpoint{1.050840in}{2.112683in}}%
\pgfpathlineto{\pgfqpoint{1.052080in}{2.111175in}}%
\pgfpathlineto{\pgfqpoint{1.057040in}{2.081005in}}%
\pgfpathlineto{\pgfqpoint{1.059520in}{2.067852in}}%
\pgfpathlineto{\pgfqpoint{1.060760in}{2.069741in}}%
\pgfpathlineto{\pgfqpoint{1.062000in}{2.068107in}}%
\pgfpathlineto{\pgfqpoint{1.066960in}{2.050979in}}%
\pgfpathlineto{\pgfqpoint{1.068200in}{2.052679in}}%
\pgfpathlineto{\pgfqpoint{1.069440in}{2.056667in}}%
\pgfpathlineto{\pgfqpoint{1.070680in}{2.054558in}}%
\pgfpathlineto{\pgfqpoint{1.073160in}{2.040640in}}%
\pgfpathlineto{\pgfqpoint{1.074400in}{2.039363in}}%
\pgfpathlineto{\pgfqpoint{1.075640in}{2.036423in}}%
\pgfpathlineto{\pgfqpoint{1.076880in}{2.037233in}}%
\pgfpathlineto{\pgfqpoint{1.078120in}{2.033545in}}%
\pgfpathlineto{\pgfqpoint{1.079360in}{2.036306in}}%
\pgfpathlineto{\pgfqpoint{1.081840in}{2.030061in}}%
\pgfpathlineto{\pgfqpoint{1.084320in}{2.036471in}}%
\pgfpathlineto{\pgfqpoint{1.085560in}{2.035317in}}%
\pgfpathlineto{\pgfqpoint{1.090520in}{2.026234in}}%
\pgfpathlineto{\pgfqpoint{1.091760in}{2.026078in}}%
\pgfpathlineto{\pgfqpoint{1.093000in}{2.027099in}}%
\pgfpathlineto{\pgfqpoint{1.094240in}{2.031135in}}%
\pgfpathlineto{\pgfqpoint{1.096720in}{2.020080in}}%
\pgfpathlineto{\pgfqpoint{1.100440in}{2.011638in}}%
\pgfpathlineto{\pgfqpoint{1.101680in}{2.012650in}}%
\pgfpathlineto{\pgfqpoint{1.102920in}{2.013960in}}%
\pgfpathlineto{\pgfqpoint{1.104160in}{2.012974in}}%
\pgfpathlineto{\pgfqpoint{1.105400in}{2.016486in}}%
\pgfpathlineto{\pgfqpoint{1.106640in}{2.014088in}}%
\pgfpathlineto{\pgfqpoint{1.110360in}{2.002342in}}%
\pgfpathlineto{\pgfqpoint{1.114080in}{2.002454in}}%
\pgfpathlineto{\pgfqpoint{1.119040in}{2.014454in}}%
\pgfpathlineto{\pgfqpoint{1.120280in}{2.015611in}}%
\pgfpathlineto{\pgfqpoint{1.122760in}{2.004709in}}%
\pgfpathlineto{\pgfqpoint{1.126480in}{1.987451in}}%
\pgfpathlineto{\pgfqpoint{1.128960in}{1.988641in}}%
\pgfpathlineto{\pgfqpoint{1.131440in}{1.987572in}}%
\pgfpathlineto{\pgfqpoint{1.133920in}{1.983136in}}%
\pgfpathlineto{\pgfqpoint{1.135160in}{1.983295in}}%
\pgfpathlineto{\pgfqpoint{1.137640in}{1.980144in}}%
\pgfpathlineto{\pgfqpoint{1.140120in}{1.987418in}}%
\pgfpathlineto{\pgfqpoint{1.141360in}{1.986292in}}%
\pgfpathlineto{\pgfqpoint{1.142600in}{1.982899in}}%
\pgfpathlineto{\pgfqpoint{1.145080in}{1.982613in}}%
\pgfpathlineto{\pgfqpoint{1.146320in}{1.988158in}}%
\pgfpathlineto{\pgfqpoint{1.148800in}{1.985259in}}%
\pgfpathlineto{\pgfqpoint{1.152520in}{1.986706in}}%
\pgfpathlineto{\pgfqpoint{1.155000in}{1.986492in}}%
\pgfpathlineto{\pgfqpoint{1.157480in}{1.993858in}}%
\pgfpathlineto{\pgfqpoint{1.161200in}{1.990386in}}%
\pgfpathlineto{\pgfqpoint{1.162440in}{1.991033in}}%
\pgfpathlineto{\pgfqpoint{1.166160in}{1.983045in}}%
\pgfpathlineto{\pgfqpoint{1.167400in}{1.988145in}}%
\pgfpathlineto{\pgfqpoint{1.168640in}{1.988477in}}%
\pgfpathlineto{\pgfqpoint{1.169880in}{1.992308in}}%
\pgfpathlineto{\pgfqpoint{1.172360in}{1.988978in}}%
\pgfpathlineto{\pgfqpoint{1.174840in}{1.986216in}}%
\pgfpathlineto{\pgfqpoint{1.176080in}{1.987092in}}%
\pgfpathlineto{\pgfqpoint{1.177320in}{1.985694in}}%
\pgfpathlineto{\pgfqpoint{1.179800in}{1.977940in}}%
\pgfpathlineto{\pgfqpoint{1.182280in}{1.977043in}}%
\pgfpathlineto{\pgfqpoint{1.183520in}{1.972120in}}%
\pgfpathlineto{\pgfqpoint{1.188480in}{1.972000in}}%
\pgfpathlineto{\pgfqpoint{1.189720in}{1.974306in}}%
\pgfpathlineto{\pgfqpoint{1.190960in}{1.973345in}}%
\pgfpathlineto{\pgfqpoint{1.192200in}{1.978010in}}%
\pgfpathlineto{\pgfqpoint{1.193440in}{1.978302in}}%
\pgfpathlineto{\pgfqpoint{1.194680in}{1.980091in}}%
\pgfpathlineto{\pgfqpoint{1.195920in}{1.978442in}}%
\pgfpathlineto{\pgfqpoint{1.197160in}{1.979010in}}%
\pgfpathlineto{\pgfqpoint{1.199640in}{1.976240in}}%
\pgfpathlineto{\pgfqpoint{1.200880in}{1.978343in}}%
\pgfpathlineto{\pgfqpoint{1.205840in}{1.976027in}}%
\pgfpathlineto{\pgfqpoint{1.207080in}{1.979808in}}%
\pgfpathlineto{\pgfqpoint{1.209560in}{1.979890in}}%
\pgfpathlineto{\pgfqpoint{1.213280in}{1.983638in}}%
\pgfpathlineto{\pgfqpoint{1.214520in}{1.980925in}}%
\pgfpathlineto{\pgfqpoint{1.217000in}{1.983662in}}%
\pgfpathlineto{\pgfqpoint{1.218240in}{1.986722in}}%
\pgfpathlineto{\pgfqpoint{1.219480in}{1.984620in}}%
\pgfpathlineto{\pgfqpoint{1.220720in}{1.985378in}}%
\pgfpathlineto{\pgfqpoint{1.223200in}{1.980909in}}%
\pgfpathlineto{\pgfqpoint{1.224440in}{1.976513in}}%
\pgfpathlineto{\pgfqpoint{1.225680in}{1.976337in}}%
\pgfpathlineto{\pgfqpoint{1.228160in}{1.973899in}}%
\pgfpathlineto{\pgfqpoint{1.229400in}{1.975080in}}%
\pgfpathlineto{\pgfqpoint{1.230640in}{1.974246in}}%
\pgfpathlineto{\pgfqpoint{1.234360in}{1.962994in}}%
\pgfpathlineto{\pgfqpoint{1.235600in}{1.963111in}}%
\pgfpathlineto{\pgfqpoint{1.236840in}{1.965345in}}%
\pgfpathlineto{\pgfqpoint{1.240560in}{1.961842in}}%
\pgfpathlineto{\pgfqpoint{1.244280in}{1.975389in}}%
\pgfpathlineto{\pgfqpoint{1.245520in}{1.974846in}}%
\pgfpathlineto{\pgfqpoint{1.250480in}{1.953127in}}%
\pgfpathlineto{\pgfqpoint{1.255440in}{1.955115in}}%
\pgfpathlineto{\pgfqpoint{1.256680in}{1.954304in}}%
\pgfpathlineto{\pgfqpoint{1.261640in}{1.958224in}}%
\pgfpathlineto{\pgfqpoint{1.264120in}{1.964359in}}%
\pgfpathlineto{\pgfqpoint{1.265360in}{1.963580in}}%
\pgfpathlineto{\pgfqpoint{1.269080in}{1.957610in}}%
\pgfpathlineto{\pgfqpoint{1.270320in}{1.959796in}}%
\pgfpathlineto{\pgfqpoint{1.274040in}{1.958840in}}%
\pgfpathlineto{\pgfqpoint{1.275280in}{1.960320in}}%
\pgfpathlineto{\pgfqpoint{1.277760in}{1.958012in}}%
\pgfpathlineto{\pgfqpoint{1.279000in}{1.955040in}}%
\pgfpathlineto{\pgfqpoint{1.280240in}{1.955827in}}%
\pgfpathlineto{\pgfqpoint{1.281480in}{1.953866in}}%
\pgfpathlineto{\pgfqpoint{1.283960in}{1.956008in}}%
\pgfpathlineto{\pgfqpoint{1.285200in}{1.955204in}}%
\pgfpathlineto{\pgfqpoint{1.286440in}{1.956038in}}%
\pgfpathlineto{\pgfqpoint{1.287680in}{1.953437in}}%
\pgfpathlineto{\pgfqpoint{1.290160in}{1.954245in}}%
\pgfpathlineto{\pgfqpoint{1.291400in}{1.956961in}}%
\pgfpathlineto{\pgfqpoint{1.292640in}{1.954885in}}%
\pgfpathlineto{\pgfqpoint{1.293880in}{1.956090in}}%
\pgfpathlineto{\pgfqpoint{1.297600in}{1.953489in}}%
\pgfpathlineto{\pgfqpoint{1.301320in}{1.954540in}}%
\pgfpathlineto{\pgfqpoint{1.302560in}{1.953268in}}%
\pgfpathlineto{\pgfqpoint{1.305040in}{1.955577in}}%
\pgfpathlineto{\pgfqpoint{1.306280in}{1.954334in}}%
\pgfpathlineto{\pgfqpoint{1.308760in}{1.947118in}}%
\pgfpathlineto{\pgfqpoint{1.312480in}{1.948684in}}%
\pgfpathlineto{\pgfqpoint{1.318680in}{1.961555in}}%
\pgfpathlineto{\pgfqpoint{1.319920in}{1.959535in}}%
\pgfpathlineto{\pgfqpoint{1.321160in}{1.960551in}}%
\pgfpathlineto{\pgfqpoint{1.323640in}{1.958582in}}%
\pgfpathlineto{\pgfqpoint{1.326120in}{1.959069in}}%
\pgfpathlineto{\pgfqpoint{1.327360in}{1.958566in}}%
\pgfpathlineto{\pgfqpoint{1.328600in}{1.955849in}}%
\pgfpathlineto{\pgfqpoint{1.329840in}{1.956608in}}%
\pgfpathlineto{\pgfqpoint{1.332320in}{1.959174in}}%
\pgfpathlineto{\pgfqpoint{1.333560in}{1.957008in}}%
\pgfpathlineto{\pgfqpoint{1.337280in}{1.960474in}}%
\pgfpathlineto{\pgfqpoint{1.338520in}{1.958077in}}%
\pgfpathlineto{\pgfqpoint{1.342240in}{1.962071in}}%
\pgfpathlineto{\pgfqpoint{1.343480in}{1.959135in}}%
\pgfpathlineto{\pgfqpoint{1.344720in}{1.959626in}}%
\pgfpathlineto{\pgfqpoint{1.345960in}{1.958896in}}%
\pgfpathlineto{\pgfqpoint{1.347200in}{1.956692in}}%
\pgfpathlineto{\pgfqpoint{1.349680in}{1.949466in}}%
\pgfpathlineto{\pgfqpoint{1.352160in}{1.946782in}}%
\pgfpathlineto{\pgfqpoint{1.354640in}{1.948485in}}%
\pgfpathlineto{\pgfqpoint{1.357120in}{1.945435in}}%
\pgfpathlineto{\pgfqpoint{1.358360in}{1.942153in}}%
\pgfpathlineto{\pgfqpoint{1.359600in}{1.942262in}}%
\pgfpathlineto{\pgfqpoint{1.360840in}{1.945819in}}%
\pgfpathlineto{\pgfqpoint{1.362080in}{1.945044in}}%
\pgfpathlineto{\pgfqpoint{1.364560in}{1.941533in}}%
\pgfpathlineto{\pgfqpoint{1.368280in}{1.951361in}}%
\pgfpathlineto{\pgfqpoint{1.369520in}{1.951093in}}%
\pgfpathlineto{\pgfqpoint{1.373240in}{1.938884in}}%
\pgfpathlineto{\pgfqpoint{1.378200in}{1.940041in}}%
\pgfpathlineto{\pgfqpoint{1.381920in}{1.938329in}}%
\pgfpathlineto{\pgfqpoint{1.383160in}{1.939862in}}%
\pgfpathlineto{\pgfqpoint{1.384400in}{1.938750in}}%
\pgfpathlineto{\pgfqpoint{1.388120in}{1.944618in}}%
\pgfpathlineto{\pgfqpoint{1.389360in}{1.944914in}}%
\pgfpathlineto{\pgfqpoint{1.393080in}{1.940283in}}%
\pgfpathlineto{\pgfqpoint{1.394320in}{1.942379in}}%
\pgfpathlineto{\pgfqpoint{1.396800in}{1.937936in}}%
\pgfpathlineto{\pgfqpoint{1.400520in}{1.940897in}}%
\pgfpathlineto{\pgfqpoint{1.401760in}{1.940332in}}%
\pgfpathlineto{\pgfqpoint{1.403000in}{1.937544in}}%
\pgfpathlineto{\pgfqpoint{1.404240in}{1.938893in}}%
\pgfpathlineto{\pgfqpoint{1.405480in}{1.937195in}}%
\pgfpathlineto{\pgfqpoint{1.407960in}{1.939608in}}%
\pgfpathlineto{\pgfqpoint{1.412920in}{1.933313in}}%
\pgfpathlineto{\pgfqpoint{1.415400in}{1.937625in}}%
\pgfpathlineto{\pgfqpoint{1.417880in}{1.935160in}}%
\pgfpathlineto{\pgfqpoint{1.421600in}{1.931634in}}%
\pgfpathlineto{\pgfqpoint{1.424080in}{1.931735in}}%
\pgfpathlineto{\pgfqpoint{1.425320in}{1.933664in}}%
\pgfpathlineto{\pgfqpoint{1.426560in}{1.933307in}}%
\pgfpathlineto{\pgfqpoint{1.429040in}{1.935890in}}%
\pgfpathlineto{\pgfqpoint{1.430280in}{1.935069in}}%
\pgfpathlineto{\pgfqpoint{1.432760in}{1.930003in}}%
\pgfpathlineto{\pgfqpoint{1.434000in}{1.929663in}}%
\pgfpathlineto{\pgfqpoint{1.436480in}{1.927854in}}%
\pgfpathlineto{\pgfqpoint{1.440200in}{1.935881in}}%
\pgfpathlineto{\pgfqpoint{1.442680in}{1.939988in}}%
\pgfpathlineto{\pgfqpoint{1.445160in}{1.937611in}}%
\pgfpathlineto{\pgfqpoint{1.447640in}{1.937156in}}%
\pgfpathlineto{\pgfqpoint{1.450120in}{1.936902in}}%
\pgfpathlineto{\pgfqpoint{1.453840in}{1.933030in}}%
\pgfpathlineto{\pgfqpoint{1.455080in}{1.934466in}}%
\pgfpathlineto{\pgfqpoint{1.458800in}{1.932505in}}%
\pgfpathlineto{\pgfqpoint{1.460040in}{1.934190in}}%
\pgfpathlineto{\pgfqpoint{1.461280in}{1.933429in}}%
\pgfpathlineto{\pgfqpoint{1.462520in}{1.929884in}}%
\pgfpathlineto{\pgfqpoint{1.466240in}{1.933097in}}%
\pgfpathlineto{\pgfqpoint{1.468720in}{1.929136in}}%
\pgfpathlineto{\pgfqpoint{1.469960in}{1.929223in}}%
\pgfpathlineto{\pgfqpoint{1.476160in}{1.921412in}}%
\pgfpathlineto{\pgfqpoint{1.477400in}{1.921841in}}%
\pgfpathlineto{\pgfqpoint{1.478640in}{1.923489in}}%
\pgfpathlineto{\pgfqpoint{1.481120in}{1.921214in}}%
\pgfpathlineto{\pgfqpoint{1.482360in}{1.918006in}}%
\pgfpathlineto{\pgfqpoint{1.483600in}{1.919282in}}%
\pgfpathlineto{\pgfqpoint{1.486080in}{1.923220in}}%
\pgfpathlineto{\pgfqpoint{1.488560in}{1.919041in}}%
\pgfpathlineto{\pgfqpoint{1.491040in}{1.922828in}}%
\pgfpathlineto{\pgfqpoint{1.492280in}{1.927092in}}%
\pgfpathlineto{\pgfqpoint{1.493520in}{1.925282in}}%
\pgfpathlineto{\pgfqpoint{1.497240in}{1.912059in}}%
\pgfpathlineto{\pgfqpoint{1.499720in}{1.914315in}}%
\pgfpathlineto{\pgfqpoint{1.500960in}{1.914059in}}%
\pgfpathlineto{\pgfqpoint{1.503440in}{1.911450in}}%
\pgfpathlineto{\pgfqpoint{1.505920in}{1.912745in}}%
\pgfpathlineto{\pgfqpoint{1.507160in}{1.914561in}}%
\pgfpathlineto{\pgfqpoint{1.508400in}{1.912748in}}%
\pgfpathlineto{\pgfqpoint{1.513360in}{1.915545in}}%
\pgfpathlineto{\pgfqpoint{1.515840in}{1.911731in}}%
\pgfpathlineto{\pgfqpoint{1.517080in}{1.911852in}}%
\pgfpathlineto{\pgfqpoint{1.518320in}{1.913695in}}%
\pgfpathlineto{\pgfqpoint{1.520800in}{1.909862in}}%
\pgfpathlineto{\pgfqpoint{1.524520in}{1.911726in}}%
\pgfpathlineto{\pgfqpoint{1.525760in}{1.911698in}}%
\pgfpathlineto{\pgfqpoint{1.527000in}{1.909746in}}%
\pgfpathlineto{\pgfqpoint{1.528240in}{1.911184in}}%
\pgfpathlineto{\pgfqpoint{1.529480in}{1.910804in}}%
\pgfpathlineto{\pgfqpoint{1.531960in}{1.913136in}}%
\pgfpathlineto{\pgfqpoint{1.533200in}{1.912320in}}%
\pgfpathlineto{\pgfqpoint{1.535680in}{1.909403in}}%
\pgfpathlineto{\pgfqpoint{1.538160in}{1.909606in}}%
\pgfpathlineto{\pgfqpoint{1.539400in}{1.911709in}}%
\pgfpathlineto{\pgfqpoint{1.540640in}{1.909664in}}%
\pgfpathlineto{\pgfqpoint{1.541880in}{1.910435in}}%
\pgfpathlineto{\pgfqpoint{1.544360in}{1.908342in}}%
\pgfpathlineto{\pgfqpoint{1.546840in}{1.909070in}}%
\pgfpathlineto{\pgfqpoint{1.548080in}{1.909633in}}%
\pgfpathlineto{\pgfqpoint{1.549320in}{1.911758in}}%
\pgfpathlineto{\pgfqpoint{1.550560in}{1.911091in}}%
\pgfpathlineto{\pgfqpoint{1.553040in}{1.912377in}}%
\pgfpathlineto{\pgfqpoint{1.560480in}{1.907355in}}%
\pgfpathlineto{\pgfqpoint{1.564200in}{1.915980in}}%
\pgfpathlineto{\pgfqpoint{1.566680in}{1.920170in}}%
\pgfpathlineto{\pgfqpoint{1.569160in}{1.915221in}}%
\pgfpathlineto{\pgfqpoint{1.574120in}{1.913957in}}%
\pgfpathlineto{\pgfqpoint{1.577840in}{1.910224in}}%
\pgfpathlineto{\pgfqpoint{1.580320in}{1.912726in}}%
\pgfpathlineto{\pgfqpoint{1.581560in}{1.910519in}}%
\pgfpathlineto{\pgfqpoint{1.585280in}{1.912768in}}%
\pgfpathlineto{\pgfqpoint{1.586520in}{1.909791in}}%
\pgfpathlineto{\pgfqpoint{1.589000in}{1.911621in}}%
\pgfpathlineto{\pgfqpoint{1.590240in}{1.913819in}}%
\pgfpathlineto{\pgfqpoint{1.592720in}{1.910060in}}%
\pgfpathlineto{\pgfqpoint{1.595200in}{1.910463in}}%
\pgfpathlineto{\pgfqpoint{1.597680in}{1.905459in}}%
\pgfpathlineto{\pgfqpoint{1.598920in}{1.903902in}}%
\pgfpathlineto{\pgfqpoint{1.603880in}{1.905293in}}%
\pgfpathlineto{\pgfqpoint{1.606360in}{1.900691in}}%
\pgfpathlineto{\pgfqpoint{1.607600in}{1.901977in}}%
\pgfpathlineto{\pgfqpoint{1.610080in}{1.905294in}}%
\pgfpathlineto{\pgfqpoint{1.612560in}{1.899676in}}%
\pgfpathlineto{\pgfqpoint{1.613800in}{1.900431in}}%
\pgfpathlineto{\pgfqpoint{1.616280in}{1.906135in}}%
\pgfpathlineto{\pgfqpoint{1.617520in}{1.904248in}}%
\pgfpathlineto{\pgfqpoint{1.620000in}{1.895733in}}%
\pgfpathlineto{\pgfqpoint{1.624960in}{1.896702in}}%
\pgfpathlineto{\pgfqpoint{1.628680in}{1.891865in}}%
\pgfpathlineto{\pgfqpoint{1.629920in}{1.891969in}}%
\pgfpathlineto{\pgfqpoint{1.631160in}{1.893608in}}%
\pgfpathlineto{\pgfqpoint{1.632400in}{1.892122in}}%
\pgfpathlineto{\pgfqpoint{1.636120in}{1.895304in}}%
\pgfpathlineto{\pgfqpoint{1.637360in}{1.895694in}}%
\pgfpathlineto{\pgfqpoint{1.639840in}{1.891869in}}%
\pgfpathlineto{\pgfqpoint{1.641080in}{1.891300in}}%
\pgfpathlineto{\pgfqpoint{1.642320in}{1.893119in}}%
\pgfpathlineto{\pgfqpoint{1.646040in}{1.889013in}}%
\pgfpathlineto{\pgfqpoint{1.648520in}{1.889123in}}%
\pgfpathlineto{\pgfqpoint{1.649760in}{1.889924in}}%
\pgfpathlineto{\pgfqpoint{1.651000in}{1.888842in}}%
\pgfpathlineto{\pgfqpoint{1.655960in}{1.894573in}}%
\pgfpathlineto{\pgfqpoint{1.658440in}{1.890927in}}%
\pgfpathlineto{\pgfqpoint{1.660920in}{1.891395in}}%
\pgfpathlineto{\pgfqpoint{1.662160in}{1.891827in}}%
\pgfpathlineto{\pgfqpoint{1.663400in}{1.893936in}}%
\pgfpathlineto{\pgfqpoint{1.664640in}{1.892504in}}%
\pgfpathlineto{\pgfqpoint{1.665880in}{1.893048in}}%
\pgfpathlineto{\pgfqpoint{1.668360in}{1.890616in}}%
\pgfpathlineto{\pgfqpoint{1.677040in}{1.896011in}}%
\pgfpathlineto{\pgfqpoint{1.683240in}{1.890595in}}%
\pgfpathlineto{\pgfqpoint{1.684480in}{1.891514in}}%
\pgfpathlineto{\pgfqpoint{1.688200in}{1.899135in}}%
\pgfpathlineto{\pgfqpoint{1.690680in}{1.902617in}}%
\pgfpathlineto{\pgfqpoint{1.694400in}{1.897238in}}%
\pgfpathlineto{\pgfqpoint{1.700600in}{1.896696in}}%
\pgfpathlineto{\pgfqpoint{1.701840in}{1.895351in}}%
\pgfpathlineto{\pgfqpoint{1.704320in}{1.896729in}}%
\pgfpathlineto{\pgfqpoint{1.705560in}{1.895032in}}%
\pgfpathlineto{\pgfqpoint{1.709280in}{1.898570in}}%
\pgfpathlineto{\pgfqpoint{1.710520in}{1.895858in}}%
\pgfpathlineto{\pgfqpoint{1.713000in}{1.897511in}}%
\pgfpathlineto{\pgfqpoint{1.714240in}{1.899861in}}%
\pgfpathlineto{\pgfqpoint{1.715480in}{1.897443in}}%
\pgfpathlineto{\pgfqpoint{1.719200in}{1.898743in}}%
\pgfpathlineto{\pgfqpoint{1.721680in}{1.894441in}}%
\pgfpathlineto{\pgfqpoint{1.724160in}{1.892740in}}%
\pgfpathlineto{\pgfqpoint{1.725400in}{1.892858in}}%
\pgfpathlineto{\pgfqpoint{1.727880in}{1.894821in}}%
\pgfpathlineto{\pgfqpoint{1.731600in}{1.893235in}}%
\pgfpathlineto{\pgfqpoint{1.734080in}{1.895916in}}%
\pgfpathlineto{\pgfqpoint{1.736560in}{1.891650in}}%
\pgfpathlineto{\pgfqpoint{1.737800in}{1.892919in}}%
\pgfpathlineto{\pgfqpoint{1.740280in}{1.899017in}}%
\pgfpathlineto{\pgfqpoint{1.741520in}{1.897755in}}%
\pgfpathlineto{\pgfqpoint{1.745240in}{1.885106in}}%
\pgfpathlineto{\pgfqpoint{1.747720in}{1.886615in}}%
\pgfpathlineto{\pgfqpoint{1.750200in}{1.883412in}}%
\pgfpathlineto{\pgfqpoint{1.752680in}{1.880480in}}%
\pgfpathlineto{\pgfqpoint{1.753920in}{1.879488in}}%
\pgfpathlineto{\pgfqpoint{1.755160in}{1.881084in}}%
\pgfpathlineto{\pgfqpoint{1.756400in}{1.880066in}}%
\pgfpathlineto{\pgfqpoint{1.758880in}{1.883355in}}%
\pgfpathlineto{\pgfqpoint{1.761360in}{1.882830in}}%
\pgfpathlineto{\pgfqpoint{1.763840in}{1.878038in}}%
\pgfpathlineto{\pgfqpoint{1.765080in}{1.877972in}}%
\pgfpathlineto{\pgfqpoint{1.766320in}{1.879346in}}%
\pgfpathlineto{\pgfqpoint{1.770040in}{1.875051in}}%
\pgfpathlineto{\pgfqpoint{1.776240in}{1.878388in}}%
\pgfpathlineto{\pgfqpoint{1.778720in}{1.881132in}}%
\pgfpathlineto{\pgfqpoint{1.779960in}{1.882533in}}%
\pgfpathlineto{\pgfqpoint{1.783680in}{1.878987in}}%
\pgfpathlineto{\pgfqpoint{1.786160in}{1.878689in}}%
\pgfpathlineto{\pgfqpoint{1.787400in}{1.881130in}}%
\pgfpathlineto{\pgfqpoint{1.788640in}{1.879553in}}%
\pgfpathlineto{\pgfqpoint{1.789880in}{1.880269in}}%
\pgfpathlineto{\pgfqpoint{1.792360in}{1.878745in}}%
\pgfpathlineto{\pgfqpoint{1.799800in}{1.885778in}}%
\pgfpathlineto{\pgfqpoint{1.801040in}{1.886777in}}%
\pgfpathlineto{\pgfqpoint{1.808480in}{1.880068in}}%
\pgfpathlineto{\pgfqpoint{1.812200in}{1.886185in}}%
\pgfpathlineto{\pgfqpoint{1.814680in}{1.890421in}}%
\pgfpathlineto{\pgfqpoint{1.817160in}{1.887604in}}%
\pgfpathlineto{\pgfqpoint{1.820880in}{1.886882in}}%
\pgfpathlineto{\pgfqpoint{1.822120in}{1.887598in}}%
\pgfpathlineto{\pgfqpoint{1.825840in}{1.883290in}}%
\pgfpathlineto{\pgfqpoint{1.827080in}{1.883275in}}%
\pgfpathlineto{\pgfqpoint{1.828320in}{1.884538in}}%
\pgfpathlineto{\pgfqpoint{1.829560in}{1.883219in}}%
\pgfpathlineto{\pgfqpoint{1.833280in}{1.886217in}}%
\pgfpathlineto{\pgfqpoint{1.834520in}{1.884243in}}%
\pgfpathlineto{\pgfqpoint{1.837000in}{1.885712in}}%
\pgfpathlineto{\pgfqpoint{1.838240in}{1.887587in}}%
\pgfpathlineto{\pgfqpoint{1.839480in}{1.885767in}}%
\pgfpathlineto{\pgfqpoint{1.841960in}{1.888532in}}%
\pgfpathlineto{\pgfqpoint{1.843200in}{1.888392in}}%
\pgfpathlineto{\pgfqpoint{1.846920in}{1.882483in}}%
\pgfpathlineto{\pgfqpoint{1.849400in}{1.881781in}}%
\pgfpathlineto{\pgfqpoint{1.851880in}{1.883827in}}%
\pgfpathlineto{\pgfqpoint{1.855600in}{1.882570in}}%
\pgfpathlineto{\pgfqpoint{1.858080in}{1.884727in}}%
\pgfpathlineto{\pgfqpoint{1.860560in}{1.879468in}}%
\pgfpathlineto{\pgfqpoint{1.861800in}{1.880022in}}%
\pgfpathlineto{\pgfqpoint{1.864280in}{1.884735in}}%
\pgfpathlineto{\pgfqpoint{1.865520in}{1.883903in}}%
\pgfpathlineto{\pgfqpoint{1.869240in}{1.872141in}}%
\pgfpathlineto{\pgfqpoint{1.872960in}{1.872412in}}%
\pgfpathlineto{\pgfqpoint{1.875440in}{1.869065in}}%
\pgfpathlineto{\pgfqpoint{1.877920in}{1.869002in}}%
\pgfpathlineto{\pgfqpoint{1.879160in}{1.870575in}}%
\pgfpathlineto{\pgfqpoint{1.880400in}{1.870041in}}%
\pgfpathlineto{\pgfqpoint{1.882880in}{1.873408in}}%
\pgfpathlineto{\pgfqpoint{1.885360in}{1.872263in}}%
\pgfpathlineto{\pgfqpoint{1.887840in}{1.867449in}}%
\pgfpathlineto{\pgfqpoint{1.889080in}{1.867135in}}%
\pgfpathlineto{\pgfqpoint{1.890320in}{1.869260in}}%
\pgfpathlineto{\pgfqpoint{1.894040in}{1.865381in}}%
\pgfpathlineto{\pgfqpoint{1.897760in}{1.867268in}}%
\pgfpathlineto{\pgfqpoint{1.899000in}{1.866566in}}%
\pgfpathlineto{\pgfqpoint{1.903960in}{1.871248in}}%
\pgfpathlineto{\pgfqpoint{1.908920in}{1.865669in}}%
\pgfpathlineto{\pgfqpoint{1.910160in}{1.865873in}}%
\pgfpathlineto{\pgfqpoint{1.911400in}{1.868278in}}%
\pgfpathlineto{\pgfqpoint{1.912640in}{1.867681in}}%
\pgfpathlineto{\pgfqpoint{1.913880in}{1.869052in}}%
\pgfpathlineto{\pgfqpoint{1.917600in}{1.868805in}}%
\pgfpathlineto{\pgfqpoint{1.920080in}{1.871596in}}%
\pgfpathlineto{\pgfqpoint{1.922560in}{1.873393in}}%
\pgfpathlineto{\pgfqpoint{1.925040in}{1.874346in}}%
\pgfpathlineto{\pgfqpoint{1.928760in}{1.871150in}}%
\pgfpathlineto{\pgfqpoint{1.931240in}{1.867794in}}%
\pgfpathlineto{\pgfqpoint{1.932480in}{1.868286in}}%
\pgfpathlineto{\pgfqpoint{1.936200in}{1.873757in}}%
\pgfpathlineto{\pgfqpoint{1.938680in}{1.877795in}}%
\pgfpathlineto{\pgfqpoint{1.941160in}{1.874680in}}%
\pgfpathlineto{\pgfqpoint{1.946120in}{1.873333in}}%
\pgfpathlineto{\pgfqpoint{1.948600in}{1.871056in}}%
\pgfpathlineto{\pgfqpoint{1.951080in}{1.870077in}}%
\pgfpathlineto{\pgfqpoint{1.952320in}{1.870995in}}%
\pgfpathlineto{\pgfqpoint{1.953560in}{1.869704in}}%
\pgfpathlineto{\pgfqpoint{1.957280in}{1.872792in}}%
\pgfpathlineto{\pgfqpoint{1.958520in}{1.870643in}}%
\pgfpathlineto{\pgfqpoint{1.961000in}{1.872909in}}%
\pgfpathlineto{\pgfqpoint{1.962240in}{1.874868in}}%
\pgfpathlineto{\pgfqpoint{1.963480in}{1.872876in}}%
\pgfpathlineto{\pgfqpoint{1.967200in}{1.875932in}}%
\pgfpathlineto{\pgfqpoint{1.969680in}{1.871537in}}%
\pgfpathlineto{\pgfqpoint{1.972160in}{1.869909in}}%
\pgfpathlineto{\pgfqpoint{1.973400in}{1.869438in}}%
\pgfpathlineto{\pgfqpoint{1.975880in}{1.870939in}}%
\pgfpathlineto{\pgfqpoint{1.979600in}{1.870194in}}%
\pgfpathlineto{\pgfqpoint{1.980840in}{1.872659in}}%
\pgfpathlineto{\pgfqpoint{1.982080in}{1.872536in}}%
\pgfpathlineto{\pgfqpoint{1.984560in}{1.869325in}}%
\pgfpathlineto{\pgfqpoint{1.985800in}{1.869923in}}%
\pgfpathlineto{\pgfqpoint{1.988280in}{1.874649in}}%
\pgfpathlineto{\pgfqpoint{1.989520in}{1.872988in}}%
\pgfpathlineto{\pgfqpoint{1.993240in}{1.862824in}}%
\pgfpathlineto{\pgfqpoint{1.995720in}{1.863194in}}%
\pgfpathlineto{\pgfqpoint{1.999440in}{1.861412in}}%
\pgfpathlineto{\pgfqpoint{2.001920in}{1.861309in}}%
\pgfpathlineto{\pgfqpoint{2.003160in}{1.863192in}}%
\pgfpathlineto{\pgfqpoint{2.004400in}{1.862546in}}%
\pgfpathlineto{\pgfqpoint{2.006880in}{1.865703in}}%
\pgfpathlineto{\pgfqpoint{2.009360in}{1.864292in}}%
\pgfpathlineto{\pgfqpoint{2.011840in}{1.859194in}}%
\pgfpathlineto{\pgfqpoint{2.013080in}{1.859059in}}%
\pgfpathlineto{\pgfqpoint{2.014320in}{1.860913in}}%
\pgfpathlineto{\pgfqpoint{2.018040in}{1.857786in}}%
\pgfpathlineto{\pgfqpoint{2.021760in}{1.859881in}}%
\pgfpathlineto{\pgfqpoint{2.023000in}{1.860070in}}%
\pgfpathlineto{\pgfqpoint{2.027960in}{1.864981in}}%
\pgfpathlineto{\pgfqpoint{2.032920in}{1.860507in}}%
\pgfpathlineto{\pgfqpoint{2.034160in}{1.860608in}}%
\pgfpathlineto{\pgfqpoint{2.036640in}{1.863360in}}%
\pgfpathlineto{\pgfqpoint{2.039120in}{1.865442in}}%
\pgfpathlineto{\pgfqpoint{2.041600in}{1.866286in}}%
\pgfpathlineto{\pgfqpoint{2.049040in}{1.871095in}}%
\pgfpathlineto{\pgfqpoint{2.051520in}{1.869948in}}%
\pgfpathlineto{\pgfqpoint{2.055240in}{1.866408in}}%
\pgfpathlineto{\pgfqpoint{2.056480in}{1.867387in}}%
\pgfpathlineto{\pgfqpoint{2.058960in}{1.871501in}}%
\pgfpathlineto{\pgfqpoint{2.060200in}{1.872013in}}%
\pgfpathlineto{\pgfqpoint{2.062680in}{1.875978in}}%
\pgfpathlineto{\pgfqpoint{2.065160in}{1.872899in}}%
\pgfpathlineto{\pgfqpoint{2.068880in}{1.871518in}}%
\pgfpathlineto{\pgfqpoint{2.070120in}{1.871067in}}%
\pgfpathlineto{\pgfqpoint{2.072600in}{1.868627in}}%
\pgfpathlineto{\pgfqpoint{2.073840in}{1.868170in}}%
\pgfpathlineto{\pgfqpoint{2.076320in}{1.870097in}}%
\pgfpathlineto{\pgfqpoint{2.077560in}{1.868925in}}%
\pgfpathlineto{\pgfqpoint{2.081280in}{1.870561in}}%
\pgfpathlineto{\pgfqpoint{2.083760in}{1.868513in}}%
\pgfpathlineto{\pgfqpoint{2.086240in}{1.871314in}}%
\pgfpathlineto{\pgfqpoint{2.087480in}{1.869522in}}%
\pgfpathlineto{\pgfqpoint{2.091200in}{1.873657in}}%
\pgfpathlineto{\pgfqpoint{2.093680in}{1.870209in}}%
\pgfpathlineto{\pgfqpoint{2.096160in}{1.868879in}}%
\pgfpathlineto{\pgfqpoint{2.097400in}{1.868482in}}%
\pgfpathlineto{\pgfqpoint{2.101120in}{1.869940in}}%
\pgfpathlineto{\pgfqpoint{2.102360in}{1.868064in}}%
\pgfpathlineto{\pgfqpoint{2.103600in}{1.868400in}}%
\pgfpathlineto{\pgfqpoint{2.104840in}{1.870540in}}%
\pgfpathlineto{\pgfqpoint{2.106080in}{1.870154in}}%
\pgfpathlineto{\pgfqpoint{2.108560in}{1.867487in}}%
\pgfpathlineto{\pgfqpoint{2.109800in}{1.867374in}}%
\pgfpathlineto{\pgfqpoint{2.112280in}{1.871708in}}%
\pgfpathlineto{\pgfqpoint{2.113520in}{1.870235in}}%
\pgfpathlineto{\pgfqpoint{2.116000in}{1.863299in}}%
\pgfpathlineto{\pgfqpoint{2.119720in}{1.863037in}}%
\pgfpathlineto{\pgfqpoint{2.124680in}{1.861534in}}%
\pgfpathlineto{\pgfqpoint{2.125920in}{1.861772in}}%
\pgfpathlineto{\pgfqpoint{2.127160in}{1.863877in}}%
\pgfpathlineto{\pgfqpoint{2.128400in}{1.863604in}}%
\pgfpathlineto{\pgfqpoint{2.132120in}{1.866654in}}%
\pgfpathlineto{\pgfqpoint{2.133360in}{1.864843in}}%
\pgfpathlineto{\pgfqpoint{2.134600in}{1.860782in}}%
\pgfpathlineto{\pgfqpoint{2.137080in}{1.860645in}}%
\pgfpathlineto{\pgfqpoint{2.138320in}{1.861878in}}%
\pgfpathlineto{\pgfqpoint{2.142040in}{1.858750in}}%
\pgfpathlineto{\pgfqpoint{2.145760in}{1.861154in}}%
\pgfpathlineto{\pgfqpoint{2.149480in}{1.863486in}}%
\pgfpathlineto{\pgfqpoint{2.151960in}{1.866626in}}%
\pgfpathlineto{\pgfqpoint{2.156920in}{1.861424in}}%
\pgfpathlineto{\pgfqpoint{2.158160in}{1.861770in}}%
\pgfpathlineto{\pgfqpoint{2.160640in}{1.864046in}}%
\pgfpathlineto{\pgfqpoint{2.161880in}{1.865953in}}%
\pgfpathlineto{\pgfqpoint{2.164360in}{1.865722in}}%
\pgfpathlineto{\pgfqpoint{2.166840in}{1.867590in}}%
\pgfpathlineto{\pgfqpoint{2.169320in}{1.870053in}}%
\pgfpathlineto{\pgfqpoint{2.171800in}{1.870051in}}%
\pgfpathlineto{\pgfqpoint{2.175520in}{1.870220in}}%
\pgfpathlineto{\pgfqpoint{2.179240in}{1.865594in}}%
\pgfpathlineto{\pgfqpoint{2.186680in}{1.874569in}}%
\pgfpathlineto{\pgfqpoint{2.189160in}{1.872786in}}%
\pgfpathlineto{\pgfqpoint{2.191640in}{1.872464in}}%
\pgfpathlineto{\pgfqpoint{2.195360in}{1.867838in}}%
\pgfpathlineto{\pgfqpoint{2.199080in}{1.867349in}}%
\pgfpathlineto{\pgfqpoint{2.200320in}{1.868531in}}%
\pgfpathlineto{\pgfqpoint{2.201560in}{1.867187in}}%
\pgfpathlineto{\pgfqpoint{2.205280in}{1.867865in}}%
\pgfpathlineto{\pgfqpoint{2.207760in}{1.865768in}}%
\pgfpathlineto{\pgfqpoint{2.210240in}{1.868761in}}%
\pgfpathlineto{\pgfqpoint{2.211480in}{1.867192in}}%
\pgfpathlineto{\pgfqpoint{2.215200in}{1.871501in}}%
\pgfpathlineto{\pgfqpoint{2.218920in}{1.867076in}}%
\pgfpathlineto{\pgfqpoint{2.221400in}{1.866430in}}%
\pgfpathlineto{\pgfqpoint{2.225120in}{1.868745in}}%
\pgfpathlineto{\pgfqpoint{2.226360in}{1.866763in}}%
\pgfpathlineto{\pgfqpoint{2.230080in}{1.869005in}}%
\pgfpathlineto{\pgfqpoint{2.232560in}{1.866385in}}%
\pgfpathlineto{\pgfqpoint{2.233800in}{1.866133in}}%
\pgfpathlineto{\pgfqpoint{2.236280in}{1.871194in}}%
\pgfpathlineto{\pgfqpoint{2.237520in}{1.869708in}}%
\pgfpathlineto{\pgfqpoint{2.240000in}{1.862471in}}%
\pgfpathlineto{\pgfqpoint{2.242480in}{1.861819in}}%
\pgfpathlineto{\pgfqpoint{2.247440in}{1.860486in}}%
\pgfpathlineto{\pgfqpoint{2.256120in}{1.866673in}}%
\pgfpathlineto{\pgfqpoint{2.257360in}{1.864944in}}%
\pgfpathlineto{\pgfqpoint{2.258600in}{1.860687in}}%
\pgfpathlineto{\pgfqpoint{2.262320in}{1.861877in}}%
\pgfpathlineto{\pgfqpoint{2.266040in}{1.858681in}}%
\pgfpathlineto{\pgfqpoint{2.274720in}{1.866010in}}%
\pgfpathlineto{\pgfqpoint{2.275960in}{1.867381in}}%
\pgfpathlineto{\pgfqpoint{2.282160in}{1.863060in}}%
\pgfpathlineto{\pgfqpoint{2.285880in}{1.869013in}}%
\pgfpathlineto{\pgfqpoint{2.288360in}{1.868234in}}%
\pgfpathlineto{\pgfqpoint{2.297040in}{1.874871in}}%
\pgfpathlineto{\pgfqpoint{2.298280in}{1.875167in}}%
\pgfpathlineto{\pgfqpoint{2.302000in}{1.870853in}}%
\pgfpathlineto{\pgfqpoint{2.303240in}{1.868775in}}%
\pgfpathlineto{\pgfqpoint{2.305720in}{1.871729in}}%
\pgfpathlineto{\pgfqpoint{2.309440in}{1.875606in}}%
\pgfpathlineto{\pgfqpoint{2.310680in}{1.877050in}}%
\pgfpathlineto{\pgfqpoint{2.313160in}{1.875642in}}%
\pgfpathlineto{\pgfqpoint{2.315640in}{1.876442in}}%
\pgfpathlineto{\pgfqpoint{2.320600in}{1.871330in}}%
\pgfpathlineto{\pgfqpoint{2.321840in}{1.870590in}}%
\pgfpathlineto{\pgfqpoint{2.324320in}{1.873242in}}%
\pgfpathlineto{\pgfqpoint{2.325560in}{1.872044in}}%
\pgfpathlineto{\pgfqpoint{2.328040in}{1.873076in}}%
\pgfpathlineto{\pgfqpoint{2.329280in}{1.872570in}}%
\pgfpathlineto{\pgfqpoint{2.330520in}{1.870368in}}%
\pgfpathlineto{\pgfqpoint{2.331760in}{1.870875in}}%
\pgfpathlineto{\pgfqpoint{2.334240in}{1.874614in}}%
\pgfpathlineto{\pgfqpoint{2.335480in}{1.873639in}}%
\pgfpathlineto{\pgfqpoint{2.339200in}{1.877588in}}%
\pgfpathlineto{\pgfqpoint{2.342920in}{1.873284in}}%
\pgfpathlineto{\pgfqpoint{2.346640in}{1.873448in}}%
\pgfpathlineto{\pgfqpoint{2.347880in}{1.874830in}}%
\pgfpathlineto{\pgfqpoint{2.349120in}{1.874126in}}%
\pgfpathlineto{\pgfqpoint{2.350360in}{1.871807in}}%
\pgfpathlineto{\pgfqpoint{2.354080in}{1.873105in}}%
\pgfpathlineto{\pgfqpoint{2.356560in}{1.871042in}}%
\pgfpathlineto{\pgfqpoint{2.357800in}{1.870849in}}%
\pgfpathlineto{\pgfqpoint{2.360280in}{1.875497in}}%
\pgfpathlineto{\pgfqpoint{2.361520in}{1.873746in}}%
\pgfpathlineto{\pgfqpoint{2.365240in}{1.864162in}}%
\pgfpathlineto{\pgfqpoint{2.367720in}{1.863840in}}%
\pgfpathlineto{\pgfqpoint{2.371440in}{1.863769in}}%
\pgfpathlineto{\pgfqpoint{2.373920in}{1.865142in}}%
\pgfpathlineto{\pgfqpoint{2.375160in}{1.868107in}}%
\pgfpathlineto{\pgfqpoint{2.377640in}{1.869245in}}%
\pgfpathlineto{\pgfqpoint{2.378880in}{1.871263in}}%
\pgfpathlineto{\pgfqpoint{2.380120in}{1.870834in}}%
\pgfpathlineto{\pgfqpoint{2.382600in}{1.864947in}}%
\pgfpathlineto{\pgfqpoint{2.383840in}{1.865878in}}%
\pgfpathlineto{\pgfqpoint{2.386320in}{1.866989in}}%
\pgfpathlineto{\pgfqpoint{2.390040in}{1.862740in}}%
\pgfpathlineto{\pgfqpoint{2.397480in}{1.868396in}}%
\pgfpathlineto{\pgfqpoint{2.399960in}{1.871806in}}%
\pgfpathlineto{\pgfqpoint{2.406160in}{1.866921in}}%
\pgfpathlineto{\pgfqpoint{2.409880in}{1.872020in}}%
\pgfpathlineto{\pgfqpoint{2.412360in}{1.870844in}}%
\pgfpathlineto{\pgfqpoint{2.414840in}{1.872515in}}%
\pgfpathlineto{\pgfqpoint{2.416080in}{1.875256in}}%
\pgfpathlineto{\pgfqpoint{2.424760in}{1.873194in}}%
\pgfpathlineto{\pgfqpoint{2.427240in}{1.870553in}}%
\pgfpathlineto{\pgfqpoint{2.428480in}{1.870980in}}%
\pgfpathlineto{\pgfqpoint{2.434680in}{1.879295in}}%
\pgfpathlineto{\pgfqpoint{2.437160in}{1.877743in}}%
\pgfpathlineto{\pgfqpoint{2.439640in}{1.878440in}}%
\pgfpathlineto{\pgfqpoint{2.444600in}{1.873196in}}%
\pgfpathlineto{\pgfqpoint{2.447080in}{1.873196in}}%
\pgfpathlineto{\pgfqpoint{2.448320in}{1.874186in}}%
\pgfpathlineto{\pgfqpoint{2.449560in}{1.872968in}}%
\pgfpathlineto{\pgfqpoint{2.452040in}{1.873720in}}%
\pgfpathlineto{\pgfqpoint{2.453280in}{1.873227in}}%
\pgfpathlineto{\pgfqpoint{2.454520in}{1.871287in}}%
\pgfpathlineto{\pgfqpoint{2.455760in}{1.872129in}}%
\pgfpathlineto{\pgfqpoint{2.458240in}{1.875714in}}%
\pgfpathlineto{\pgfqpoint{2.459480in}{1.874919in}}%
\pgfpathlineto{\pgfqpoint{2.464440in}{1.878391in}}%
\pgfpathlineto{\pgfqpoint{2.465680in}{1.877537in}}%
\pgfpathlineto{\pgfqpoint{2.468160in}{1.875225in}}%
\pgfpathlineto{\pgfqpoint{2.470640in}{1.874735in}}%
\pgfpathlineto{\pgfqpoint{2.471880in}{1.876014in}}%
\pgfpathlineto{\pgfqpoint{2.473120in}{1.875301in}}%
\pgfpathlineto{\pgfqpoint{2.474360in}{1.873259in}}%
\pgfpathlineto{\pgfqpoint{2.478080in}{1.874720in}}%
\pgfpathlineto{\pgfqpoint{2.480560in}{1.872321in}}%
\pgfpathlineto{\pgfqpoint{2.481800in}{1.871749in}}%
\pgfpathlineto{\pgfqpoint{2.484280in}{1.876125in}}%
\pgfpathlineto{\pgfqpoint{2.485520in}{1.874491in}}%
\pgfpathlineto{\pgfqpoint{2.488000in}{1.867578in}}%
\pgfpathlineto{\pgfqpoint{2.489240in}{1.866199in}}%
\pgfpathlineto{\pgfqpoint{2.490480in}{1.866648in}}%
\pgfpathlineto{\pgfqpoint{2.492960in}{1.866157in}}%
\pgfpathlineto{\pgfqpoint{2.497920in}{1.868102in}}%
\pgfpathlineto{\pgfqpoint{2.500400in}{1.871438in}}%
\pgfpathlineto{\pgfqpoint{2.504120in}{1.875004in}}%
\pgfpathlineto{\pgfqpoint{2.507840in}{1.870687in}}%
\pgfpathlineto{\pgfqpoint{2.510320in}{1.871487in}}%
\pgfpathlineto{\pgfqpoint{2.514040in}{1.867386in}}%
\pgfpathlineto{\pgfqpoint{2.525200in}{1.875126in}}%
\pgfpathlineto{\pgfqpoint{2.527680in}{1.872879in}}%
\pgfpathlineto{\pgfqpoint{2.530160in}{1.871799in}}%
\pgfpathlineto{\pgfqpoint{2.533880in}{1.876517in}}%
\pgfpathlineto{\pgfqpoint{2.536360in}{1.875220in}}%
\pgfpathlineto{\pgfqpoint{2.538840in}{1.876736in}}%
\pgfpathlineto{\pgfqpoint{2.540080in}{1.879846in}}%
\pgfpathlineto{\pgfqpoint{2.547520in}{1.878412in}}%
\pgfpathlineto{\pgfqpoint{2.552480in}{1.875023in}}%
\pgfpathlineto{\pgfqpoint{2.558680in}{1.881558in}}%
\pgfpathlineto{\pgfqpoint{2.561160in}{1.879822in}}%
\pgfpathlineto{\pgfqpoint{2.563640in}{1.880481in}}%
\pgfpathlineto{\pgfqpoint{2.567360in}{1.875774in}}%
\pgfpathlineto{\pgfqpoint{2.571080in}{1.874821in}}%
\pgfpathlineto{\pgfqpoint{2.572320in}{1.875689in}}%
\pgfpathlineto{\pgfqpoint{2.573560in}{1.873978in}}%
\pgfpathlineto{\pgfqpoint{2.576040in}{1.875061in}}%
\pgfpathlineto{\pgfqpoint{2.579760in}{1.873815in}}%
\pgfpathlineto{\pgfqpoint{2.582240in}{1.878098in}}%
\pgfpathlineto{\pgfqpoint{2.583480in}{1.877043in}}%
\pgfpathlineto{\pgfqpoint{2.584720in}{1.877993in}}%
\pgfpathlineto{\pgfqpoint{2.587200in}{1.882097in}}%
\pgfpathlineto{\pgfqpoint{2.589680in}{1.880731in}}%
\pgfpathlineto{\pgfqpoint{2.592160in}{1.877972in}}%
\pgfpathlineto{\pgfqpoint{2.594640in}{1.877218in}}%
\pgfpathlineto{\pgfqpoint{2.595880in}{1.878313in}}%
\pgfpathlineto{\pgfqpoint{2.599600in}{1.876262in}}%
\pgfpathlineto{\pgfqpoint{2.600840in}{1.877602in}}%
\pgfpathlineto{\pgfqpoint{2.603320in}{1.875409in}}%
\pgfpathlineto{\pgfqpoint{2.605800in}{1.873395in}}%
\pgfpathlineto{\pgfqpoint{2.608280in}{1.877685in}}%
\pgfpathlineto{\pgfqpoint{2.609520in}{1.875703in}}%
\pgfpathlineto{\pgfqpoint{2.612000in}{1.868862in}}%
\pgfpathlineto{\pgfqpoint{2.614480in}{1.870814in}}%
\pgfpathlineto{\pgfqpoint{2.615720in}{1.869786in}}%
\pgfpathlineto{\pgfqpoint{2.621920in}{1.872756in}}%
\pgfpathlineto{\pgfqpoint{2.624400in}{1.876163in}}%
\pgfpathlineto{\pgfqpoint{2.628120in}{1.880117in}}%
\pgfpathlineto{\pgfqpoint{2.629360in}{1.878813in}}%
\pgfpathlineto{\pgfqpoint{2.630600in}{1.875550in}}%
\pgfpathlineto{\pgfqpoint{2.634320in}{1.877043in}}%
\pgfpathlineto{\pgfqpoint{2.638040in}{1.873683in}}%
\pgfpathlineto{\pgfqpoint{2.640520in}{1.875188in}}%
\pgfpathlineto{\pgfqpoint{2.643000in}{1.877513in}}%
\pgfpathlineto{\pgfqpoint{2.645480in}{1.877748in}}%
\pgfpathlineto{\pgfqpoint{2.647960in}{1.882396in}}%
\pgfpathlineto{\pgfqpoint{2.649200in}{1.881781in}}%
\pgfpathlineto{\pgfqpoint{2.651680in}{1.879119in}}%
\pgfpathlineto{\pgfqpoint{2.654160in}{1.877405in}}%
\pgfpathlineto{\pgfqpoint{2.657880in}{1.881245in}}%
\pgfpathlineto{\pgfqpoint{2.660360in}{1.880051in}}%
\pgfpathlineto{\pgfqpoint{2.665320in}{1.884208in}}%
\pgfpathlineto{\pgfqpoint{2.667800in}{1.884707in}}%
\pgfpathlineto{\pgfqpoint{2.670280in}{1.884585in}}%
\pgfpathlineto{\pgfqpoint{2.674000in}{1.880458in}}%
\pgfpathlineto{\pgfqpoint{2.675240in}{1.878215in}}%
\pgfpathlineto{\pgfqpoint{2.676480in}{1.878646in}}%
\pgfpathlineto{\pgfqpoint{2.682680in}{1.885032in}}%
\pgfpathlineto{\pgfqpoint{2.685160in}{1.882405in}}%
\pgfpathlineto{\pgfqpoint{2.687640in}{1.882941in}}%
\pgfpathlineto{\pgfqpoint{2.690120in}{1.881355in}}%
\pgfpathlineto{\pgfqpoint{2.693840in}{1.878383in}}%
\pgfpathlineto{\pgfqpoint{2.701280in}{1.878484in}}%
\pgfpathlineto{\pgfqpoint{2.703760in}{1.877331in}}%
\pgfpathlineto{\pgfqpoint{2.706240in}{1.881421in}}%
\pgfpathlineto{\pgfqpoint{2.707480in}{1.880880in}}%
\pgfpathlineto{\pgfqpoint{2.713680in}{1.885483in}}%
\pgfpathlineto{\pgfqpoint{2.716160in}{1.882755in}}%
\pgfpathlineto{\pgfqpoint{2.718640in}{1.881472in}}%
\pgfpathlineto{\pgfqpoint{2.719880in}{1.882190in}}%
\pgfpathlineto{\pgfqpoint{2.722360in}{1.879886in}}%
\pgfpathlineto{\pgfqpoint{2.724840in}{1.883085in}}%
\pgfpathlineto{\pgfqpoint{2.726080in}{1.882439in}}%
\pgfpathlineto{\pgfqpoint{2.729800in}{1.877628in}}%
\pgfpathlineto{\pgfqpoint{2.732280in}{1.881083in}}%
\pgfpathlineto{\pgfqpoint{2.733520in}{1.879361in}}%
\pgfpathlineto{\pgfqpoint{2.736000in}{1.873429in}}%
\pgfpathlineto{\pgfqpoint{2.738480in}{1.874709in}}%
\pgfpathlineto{\pgfqpoint{2.740960in}{1.873416in}}%
\pgfpathlineto{\pgfqpoint{2.744680in}{1.874723in}}%
\pgfpathlineto{\pgfqpoint{2.745920in}{1.875808in}}%
\pgfpathlineto{\pgfqpoint{2.748400in}{1.880035in}}%
\pgfpathlineto{\pgfqpoint{2.752120in}{1.883417in}}%
\pgfpathlineto{\pgfqpoint{2.753360in}{1.882423in}}%
\pgfpathlineto{\pgfqpoint{2.754600in}{1.879226in}}%
\pgfpathlineto{\pgfqpoint{2.758320in}{1.880248in}}%
\pgfpathlineto{\pgfqpoint{2.762040in}{1.877011in}}%
\pgfpathlineto{\pgfqpoint{2.763280in}{1.876678in}}%
\pgfpathlineto{\pgfqpoint{2.773200in}{1.883944in}}%
\pgfpathlineto{\pgfqpoint{2.778160in}{1.879873in}}%
\pgfpathlineto{\pgfqpoint{2.781880in}{1.885575in}}%
\pgfpathlineto{\pgfqpoint{2.784360in}{1.883402in}}%
\pgfpathlineto{\pgfqpoint{2.793040in}{1.888376in}}%
\pgfpathlineto{\pgfqpoint{2.795520in}{1.886593in}}%
\pgfpathlineto{\pgfqpoint{2.800480in}{1.881414in}}%
\pgfpathlineto{\pgfqpoint{2.802960in}{1.885016in}}%
\pgfpathlineto{\pgfqpoint{2.805440in}{1.886245in}}%
\pgfpathlineto{\pgfqpoint{2.806680in}{1.887512in}}%
\pgfpathlineto{\pgfqpoint{2.809160in}{1.885335in}}%
\pgfpathlineto{\pgfqpoint{2.811640in}{1.886092in}}%
\pgfpathlineto{\pgfqpoint{2.815360in}{1.882811in}}%
\pgfpathlineto{\pgfqpoint{2.819080in}{1.881198in}}%
\pgfpathlineto{\pgfqpoint{2.820320in}{1.882240in}}%
\pgfpathlineto{\pgfqpoint{2.821560in}{1.881284in}}%
\pgfpathlineto{\pgfqpoint{2.824040in}{1.882790in}}%
\pgfpathlineto{\pgfqpoint{2.827760in}{1.880865in}}%
\pgfpathlineto{\pgfqpoint{2.830240in}{1.884113in}}%
\pgfpathlineto{\pgfqpoint{2.831480in}{1.883324in}}%
\pgfpathlineto{\pgfqpoint{2.837680in}{1.887236in}}%
\pgfpathlineto{\pgfqpoint{2.840160in}{1.884540in}}%
\pgfpathlineto{\pgfqpoint{2.842640in}{1.883700in}}%
\pgfpathlineto{\pgfqpoint{2.843880in}{1.884344in}}%
\pgfpathlineto{\pgfqpoint{2.846360in}{1.881148in}}%
\pgfpathlineto{\pgfqpoint{2.850080in}{1.883950in}}%
\pgfpathlineto{\pgfqpoint{2.853800in}{1.879981in}}%
\pgfpathlineto{\pgfqpoint{2.856280in}{1.883054in}}%
\pgfpathlineto{\pgfqpoint{2.857520in}{1.881356in}}%
\pgfpathlineto{\pgfqpoint{2.860000in}{1.875757in}}%
\pgfpathlineto{\pgfqpoint{2.862480in}{1.877843in}}%
\pgfpathlineto{\pgfqpoint{2.863720in}{1.876450in}}%
\pgfpathlineto{\pgfqpoint{2.869920in}{1.879752in}}%
\pgfpathlineto{\pgfqpoint{2.872400in}{1.883682in}}%
\pgfpathlineto{\pgfqpoint{2.876120in}{1.887116in}}%
\pgfpathlineto{\pgfqpoint{2.877360in}{1.886367in}}%
\pgfpathlineto{\pgfqpoint{2.878600in}{1.883789in}}%
\pgfpathlineto{\pgfqpoint{2.882320in}{1.885929in}}%
\pgfpathlineto{\pgfqpoint{2.887280in}{1.882397in}}%
\pgfpathlineto{\pgfqpoint{2.895960in}{1.889693in}}%
\pgfpathlineto{\pgfqpoint{2.898440in}{1.888266in}}%
\pgfpathlineto{\pgfqpoint{2.902160in}{1.886064in}}%
\pgfpathlineto{\pgfqpoint{2.905880in}{1.892418in}}%
\pgfpathlineto{\pgfqpoint{2.908360in}{1.889875in}}%
\pgfpathlineto{\pgfqpoint{2.910840in}{1.891353in}}%
\pgfpathlineto{\pgfqpoint{2.913320in}{1.894758in}}%
\pgfpathlineto{\pgfqpoint{2.919520in}{1.894361in}}%
\pgfpathlineto{\pgfqpoint{2.924480in}{1.888872in}}%
\pgfpathlineto{\pgfqpoint{2.926960in}{1.892518in}}%
\pgfpathlineto{\pgfqpoint{2.929440in}{1.892728in}}%
\pgfpathlineto{\pgfqpoint{2.930680in}{1.893526in}}%
\pgfpathlineto{\pgfqpoint{2.933160in}{1.891123in}}%
\pgfpathlineto{\pgfqpoint{2.935640in}{1.891794in}}%
\pgfpathlineto{\pgfqpoint{2.939360in}{1.887542in}}%
\pgfpathlineto{\pgfqpoint{2.941840in}{1.886048in}}%
\pgfpathlineto{\pgfqpoint{2.948040in}{1.888876in}}%
\pgfpathlineto{\pgfqpoint{2.951760in}{1.887254in}}%
\pgfpathlineto{\pgfqpoint{2.954240in}{1.890739in}}%
\pgfpathlineto{\pgfqpoint{2.955480in}{1.890108in}}%
\pgfpathlineto{\pgfqpoint{2.960440in}{1.893697in}}%
\pgfpathlineto{\pgfqpoint{2.961680in}{1.893095in}}%
\pgfpathlineto{\pgfqpoint{2.962920in}{1.890190in}}%
\pgfpathlineto{\pgfqpoint{2.964160in}{1.890370in}}%
\pgfpathlineto{\pgfqpoint{2.966640in}{1.889155in}}%
\pgfpathlineto{\pgfqpoint{2.967880in}{1.889218in}}%
\pgfpathlineto{\pgfqpoint{2.971600in}{1.887175in}}%
\pgfpathlineto{\pgfqpoint{2.972840in}{1.888943in}}%
\pgfpathlineto{\pgfqpoint{2.974080in}{1.888738in}}%
\pgfpathlineto{\pgfqpoint{2.976560in}{1.885462in}}%
\pgfpathlineto{\pgfqpoint{2.977800in}{1.885321in}}%
\pgfpathlineto{\pgfqpoint{2.980280in}{1.888879in}}%
\pgfpathlineto{\pgfqpoint{2.981520in}{1.887395in}}%
\pgfpathlineto{\pgfqpoint{2.984000in}{1.881989in}}%
\pgfpathlineto{\pgfqpoint{2.986480in}{1.884408in}}%
\pgfpathlineto{\pgfqpoint{2.988960in}{1.883170in}}%
\pgfpathlineto{\pgfqpoint{2.993920in}{1.885710in}}%
\pgfpathlineto{\pgfqpoint{2.996400in}{1.889836in}}%
\pgfpathlineto{\pgfqpoint{3.000120in}{1.893817in}}%
\pgfpathlineto{\pgfqpoint{3.003840in}{1.891073in}}%
\pgfpathlineto{\pgfqpoint{3.007560in}{1.891688in}}%
\pgfpathlineto{\pgfqpoint{3.010040in}{1.888512in}}%
\pgfpathlineto{\pgfqpoint{3.011280in}{1.888015in}}%
\pgfpathlineto{\pgfqpoint{3.016240in}{1.891548in}}%
\pgfpathlineto{\pgfqpoint{3.017480in}{1.891417in}}%
\pgfpathlineto{\pgfqpoint{3.019960in}{1.894500in}}%
\pgfpathlineto{\pgfqpoint{3.023680in}{1.893132in}}%
\pgfpathlineto{\pgfqpoint{3.026160in}{1.891340in}}%
\pgfpathlineto{\pgfqpoint{3.029880in}{1.897600in}}%
\pgfpathlineto{\pgfqpoint{3.032360in}{1.894973in}}%
\pgfpathlineto{\pgfqpoint{3.034840in}{1.896593in}}%
\pgfpathlineto{\pgfqpoint{3.037320in}{1.900130in}}%
\pgfpathlineto{\pgfqpoint{3.041040in}{1.900713in}}%
\pgfpathlineto{\pgfqpoint{3.043520in}{1.898883in}}%
\pgfpathlineto{\pgfqpoint{3.047240in}{1.892672in}}%
\pgfpathlineto{\pgfqpoint{3.048480in}{1.892779in}}%
\pgfpathlineto{\pgfqpoint{3.050960in}{1.895582in}}%
\pgfpathlineto{\pgfqpoint{3.053440in}{1.895556in}}%
\pgfpathlineto{\pgfqpoint{3.054680in}{1.896653in}}%
\pgfpathlineto{\pgfqpoint{3.057160in}{1.894369in}}%
\pgfpathlineto{\pgfqpoint{3.059640in}{1.895585in}}%
\pgfpathlineto{\pgfqpoint{3.064600in}{1.890556in}}%
\pgfpathlineto{\pgfqpoint{3.065840in}{1.890055in}}%
\pgfpathlineto{\pgfqpoint{3.069560in}{1.892783in}}%
\pgfpathlineto{\pgfqpoint{3.072040in}{1.893872in}}%
\pgfpathlineto{\pgfqpoint{3.075760in}{1.892040in}}%
\pgfpathlineto{\pgfqpoint{3.078240in}{1.894675in}}%
\pgfpathlineto{\pgfqpoint{3.079480in}{1.894048in}}%
\pgfpathlineto{\pgfqpoint{3.084440in}{1.898958in}}%
\pgfpathlineto{\pgfqpoint{3.085680in}{1.898277in}}%
\pgfpathlineto{\pgfqpoint{3.086920in}{1.895574in}}%
\pgfpathlineto{\pgfqpoint{3.088160in}{1.895598in}}%
\pgfpathlineto{\pgfqpoint{3.090640in}{1.893921in}}%
\pgfpathlineto{\pgfqpoint{3.093120in}{1.892749in}}%
\pgfpathlineto{\pgfqpoint{3.094360in}{1.890616in}}%
\pgfpathlineto{\pgfqpoint{3.098080in}{1.893680in}}%
\pgfpathlineto{\pgfqpoint{3.101800in}{1.890560in}}%
\pgfpathlineto{\pgfqpoint{3.104280in}{1.894306in}}%
\pgfpathlineto{\pgfqpoint{3.108000in}{1.885877in}}%
\pgfpathlineto{\pgfqpoint{3.109240in}{1.887223in}}%
\pgfpathlineto{\pgfqpoint{3.110480in}{1.887795in}}%
\pgfpathlineto{\pgfqpoint{3.111720in}{1.886753in}}%
\pgfpathlineto{\pgfqpoint{3.117920in}{1.889774in}}%
\pgfpathlineto{\pgfqpoint{3.120400in}{1.893497in}}%
\pgfpathlineto{\pgfqpoint{3.124120in}{1.896669in}}%
\pgfpathlineto{\pgfqpoint{3.127840in}{1.894286in}}%
\pgfpathlineto{\pgfqpoint{3.131560in}{1.895227in}}%
\pgfpathlineto{\pgfqpoint{3.134040in}{1.892764in}}%
\pgfpathlineto{\pgfqpoint{3.135280in}{1.891989in}}%
\pgfpathlineto{\pgfqpoint{3.145200in}{1.897570in}}%
\pgfpathlineto{\pgfqpoint{3.150160in}{1.894812in}}%
\pgfpathlineto{\pgfqpoint{3.153880in}{1.900839in}}%
\pgfpathlineto{\pgfqpoint{3.156360in}{1.898134in}}%
\pgfpathlineto{\pgfqpoint{3.158840in}{1.899535in}}%
\pgfpathlineto{\pgfqpoint{3.160080in}{1.902434in}}%
\pgfpathlineto{\pgfqpoint{3.166280in}{1.901532in}}%
\pgfpathlineto{\pgfqpoint{3.168760in}{1.897835in}}%
\pgfpathlineto{\pgfqpoint{3.172480in}{1.894531in}}%
\pgfpathlineto{\pgfqpoint{3.174960in}{1.897593in}}%
\pgfpathlineto{\pgfqpoint{3.177440in}{1.897692in}}%
\pgfpathlineto{\pgfqpoint{3.178680in}{1.899054in}}%
\pgfpathlineto{\pgfqpoint{3.181160in}{1.897064in}}%
\pgfpathlineto{\pgfqpoint{3.183640in}{1.897850in}}%
\pgfpathlineto{\pgfqpoint{3.188600in}{1.892647in}}%
\pgfpathlineto{\pgfqpoint{3.189840in}{1.892123in}}%
\pgfpathlineto{\pgfqpoint{3.192320in}{1.895510in}}%
\pgfpathlineto{\pgfqpoint{3.193560in}{1.894513in}}%
\pgfpathlineto{\pgfqpoint{3.196040in}{1.894931in}}%
\pgfpathlineto{\pgfqpoint{3.199760in}{1.892679in}}%
\pgfpathlineto{\pgfqpoint{3.202240in}{1.895055in}}%
\pgfpathlineto{\pgfqpoint{3.203480in}{1.894518in}}%
\pgfpathlineto{\pgfqpoint{3.207200in}{1.899001in}}%
\pgfpathlineto{\pgfqpoint{3.209680in}{1.898943in}}%
\pgfpathlineto{\pgfqpoint{3.210920in}{1.896570in}}%
\pgfpathlineto{\pgfqpoint{3.212160in}{1.896781in}}%
\pgfpathlineto{\pgfqpoint{3.214640in}{1.894963in}}%
\pgfpathlineto{\pgfqpoint{3.217120in}{1.894064in}}%
\pgfpathlineto{\pgfqpoint{3.218360in}{1.892120in}}%
\pgfpathlineto{\pgfqpoint{3.220840in}{1.894203in}}%
\pgfpathlineto{\pgfqpoint{3.222080in}{1.893993in}}%
\pgfpathlineto{\pgfqpoint{3.225800in}{1.890319in}}%
\pgfpathlineto{\pgfqpoint{3.228280in}{1.894124in}}%
\pgfpathlineto{\pgfqpoint{3.232000in}{1.885891in}}%
\pgfpathlineto{\pgfqpoint{3.233240in}{1.886763in}}%
\pgfpathlineto{\pgfqpoint{3.234480in}{1.887254in}}%
\pgfpathlineto{\pgfqpoint{3.235720in}{1.886432in}}%
\pgfpathlineto{\pgfqpoint{3.238200in}{1.887418in}}%
\pgfpathlineto{\pgfqpoint{3.239440in}{1.886598in}}%
\pgfpathlineto{\pgfqpoint{3.243160in}{1.892394in}}%
\pgfpathlineto{\pgfqpoint{3.244400in}{1.892516in}}%
\pgfpathlineto{\pgfqpoint{3.248120in}{1.896838in}}%
\pgfpathlineto{\pgfqpoint{3.251840in}{1.894425in}}%
\pgfpathlineto{\pgfqpoint{3.255560in}{1.894307in}}%
\pgfpathlineto{\pgfqpoint{3.259280in}{1.890989in}}%
\pgfpathlineto{\pgfqpoint{3.265480in}{1.893271in}}%
\pgfpathlineto{\pgfqpoint{3.267960in}{1.896180in}}%
\pgfpathlineto{\pgfqpoint{3.272920in}{1.894719in}}%
\pgfpathlineto{\pgfqpoint{3.274160in}{1.894039in}}%
\pgfpathlineto{\pgfqpoint{3.277880in}{1.900281in}}%
\pgfpathlineto{\pgfqpoint{3.280360in}{1.897641in}}%
\pgfpathlineto{\pgfqpoint{3.282840in}{1.899107in}}%
\pgfpathlineto{\pgfqpoint{3.284080in}{1.902196in}}%
\pgfpathlineto{\pgfqpoint{3.286560in}{1.902536in}}%
\pgfpathlineto{\pgfqpoint{3.289040in}{1.902769in}}%
\pgfpathlineto{\pgfqpoint{3.291520in}{1.900019in}}%
\pgfpathlineto{\pgfqpoint{3.295240in}{1.893267in}}%
\pgfpathlineto{\pgfqpoint{3.296480in}{1.893948in}}%
\pgfpathlineto{\pgfqpoint{3.298960in}{1.897742in}}%
\pgfpathlineto{\pgfqpoint{3.301440in}{1.897237in}}%
\pgfpathlineto{\pgfqpoint{3.302680in}{1.898684in}}%
\pgfpathlineto{\pgfqpoint{3.305160in}{1.897303in}}%
\pgfpathlineto{\pgfqpoint{3.307640in}{1.898051in}}%
\pgfpathlineto{\pgfqpoint{3.313840in}{1.892278in}}%
\pgfpathlineto{\pgfqpoint{3.316320in}{1.894970in}}%
\pgfpathlineto{\pgfqpoint{3.317560in}{1.893741in}}%
\pgfpathlineto{\pgfqpoint{3.321280in}{1.894848in}}%
\pgfpathlineto{\pgfqpoint{3.323760in}{1.893677in}}%
\pgfpathlineto{\pgfqpoint{3.326240in}{1.895901in}}%
\pgfpathlineto{\pgfqpoint{3.327480in}{1.895468in}}%
\pgfpathlineto{\pgfqpoint{3.331200in}{1.899574in}}%
\pgfpathlineto{\pgfqpoint{3.333680in}{1.899863in}}%
\pgfpathlineto{\pgfqpoint{3.336160in}{1.897193in}}%
\pgfpathlineto{\pgfqpoint{3.338640in}{1.894948in}}%
\pgfpathlineto{\pgfqpoint{3.341120in}{1.894181in}}%
\pgfpathlineto{\pgfqpoint{3.342360in}{1.892400in}}%
\pgfpathlineto{\pgfqpoint{3.344840in}{1.894162in}}%
\pgfpathlineto{\pgfqpoint{3.347320in}{1.892656in}}%
\pgfpathlineto{\pgfqpoint{3.349800in}{1.891294in}}%
\pgfpathlineto{\pgfqpoint{3.352280in}{1.894743in}}%
\pgfpathlineto{\pgfqpoint{3.357240in}{1.887183in}}%
\pgfpathlineto{\pgfqpoint{3.363440in}{1.886979in}}%
\pgfpathlineto{\pgfqpoint{3.367160in}{1.892382in}}%
\pgfpathlineto{\pgfqpoint{3.368400in}{1.892321in}}%
\pgfpathlineto{\pgfqpoint{3.372120in}{1.896018in}}%
\pgfpathlineto{\pgfqpoint{3.373360in}{1.895207in}}%
\pgfpathlineto{\pgfqpoint{3.374600in}{1.892466in}}%
\pgfpathlineto{\pgfqpoint{3.378320in}{1.895342in}}%
\pgfpathlineto{\pgfqpoint{3.383280in}{1.890958in}}%
\pgfpathlineto{\pgfqpoint{3.395680in}{1.895284in}}%
\pgfpathlineto{\pgfqpoint{3.398160in}{1.894485in}}%
\pgfpathlineto{\pgfqpoint{3.401880in}{1.900863in}}%
\pgfpathlineto{\pgfqpoint{3.404360in}{1.897969in}}%
\pgfpathlineto{\pgfqpoint{3.406840in}{1.900005in}}%
\pgfpathlineto{\pgfqpoint{3.408080in}{1.904028in}}%
\pgfpathlineto{\pgfqpoint{3.413040in}{1.903983in}}%
\pgfpathlineto{\pgfqpoint{3.416760in}{1.898285in}}%
\pgfpathlineto{\pgfqpoint{3.419240in}{1.894004in}}%
\pgfpathlineto{\pgfqpoint{3.420480in}{1.894900in}}%
\pgfpathlineto{\pgfqpoint{3.422960in}{1.898399in}}%
\pgfpathlineto{\pgfqpoint{3.425440in}{1.897496in}}%
\pgfpathlineto{\pgfqpoint{3.426680in}{1.899223in}}%
\pgfpathlineto{\pgfqpoint{3.429160in}{1.897922in}}%
\pgfpathlineto{\pgfqpoint{3.431640in}{1.899801in}}%
\pgfpathlineto{\pgfqpoint{3.437840in}{1.894308in}}%
\pgfpathlineto{\pgfqpoint{3.441560in}{1.896682in}}%
\pgfpathlineto{\pgfqpoint{3.445280in}{1.896962in}}%
\pgfpathlineto{\pgfqpoint{3.447760in}{1.895947in}}%
\pgfpathlineto{\pgfqpoint{3.452720in}{1.900036in}}%
\pgfpathlineto{\pgfqpoint{3.455200in}{1.901791in}}%
\pgfpathlineto{\pgfqpoint{3.457680in}{1.900901in}}%
\pgfpathlineto{\pgfqpoint{3.458920in}{1.898705in}}%
\pgfpathlineto{\pgfqpoint{3.460160in}{1.899158in}}%
\pgfpathlineto{\pgfqpoint{3.462640in}{1.897619in}}%
\pgfpathlineto{\pgfqpoint{3.465120in}{1.897188in}}%
\pgfpathlineto{\pgfqpoint{3.466360in}{1.895652in}}%
\pgfpathlineto{\pgfqpoint{3.468840in}{1.897411in}}%
\pgfpathlineto{\pgfqpoint{3.471320in}{1.895634in}}%
\pgfpathlineto{\pgfqpoint{3.473800in}{1.894751in}}%
\pgfpathlineto{\pgfqpoint{3.476280in}{1.898429in}}%
\pgfpathlineto{\pgfqpoint{3.478760in}{1.893348in}}%
\pgfpathlineto{\pgfqpoint{3.480000in}{1.890907in}}%
\pgfpathlineto{\pgfqpoint{3.482480in}{1.892172in}}%
\pgfpathlineto{\pgfqpoint{3.483720in}{1.891511in}}%
\pgfpathlineto{\pgfqpoint{3.486200in}{1.892366in}}%
\pgfpathlineto{\pgfqpoint{3.487440in}{1.891435in}}%
\pgfpathlineto{\pgfqpoint{3.491160in}{1.896770in}}%
\pgfpathlineto{\pgfqpoint{3.492400in}{1.896444in}}%
\pgfpathlineto{\pgfqpoint{3.496120in}{1.900380in}}%
\pgfpathlineto{\pgfqpoint{3.497360in}{1.899916in}}%
\pgfpathlineto{\pgfqpoint{3.498600in}{1.897119in}}%
\pgfpathlineto{\pgfqpoint{3.502320in}{1.899476in}}%
\pgfpathlineto{\pgfqpoint{3.508520in}{1.895000in}}%
\pgfpathlineto{\pgfqpoint{3.511000in}{1.896778in}}%
\pgfpathlineto{\pgfqpoint{3.513480in}{1.896720in}}%
\pgfpathlineto{\pgfqpoint{3.515960in}{1.899338in}}%
\pgfpathlineto{\pgfqpoint{3.517200in}{1.899558in}}%
\pgfpathlineto{\pgfqpoint{3.519680in}{1.898307in}}%
\pgfpathlineto{\pgfqpoint{3.522160in}{1.898143in}}%
\pgfpathlineto{\pgfqpoint{3.525880in}{1.904368in}}%
\pgfpathlineto{\pgfqpoint{3.528360in}{1.901020in}}%
\pgfpathlineto{\pgfqpoint{3.530840in}{1.903907in}}%
\pgfpathlineto{\pgfqpoint{3.532080in}{1.908197in}}%
\pgfpathlineto{\pgfqpoint{3.537040in}{1.907872in}}%
\pgfpathlineto{\pgfqpoint{3.540760in}{1.900933in}}%
\pgfpathlineto{\pgfqpoint{3.543240in}{1.896338in}}%
\pgfpathlineto{\pgfqpoint{3.544480in}{1.897470in}}%
\pgfpathlineto{\pgfqpoint{3.546960in}{1.901034in}}%
\pgfpathlineto{\pgfqpoint{3.549440in}{1.901146in}}%
\pgfpathlineto{\pgfqpoint{3.550680in}{1.903613in}}%
\pgfpathlineto{\pgfqpoint{3.553160in}{1.901994in}}%
\pgfpathlineto{\pgfqpoint{3.555640in}{1.903623in}}%
\pgfpathlineto{\pgfqpoint{3.561840in}{1.897801in}}%
\pgfpathlineto{\pgfqpoint{3.565560in}{1.900636in}}%
\pgfpathlineto{\pgfqpoint{3.573000in}{1.902549in}}%
\pgfpathlineto{\pgfqpoint{3.579200in}{1.908350in}}%
\pgfpathlineto{\pgfqpoint{3.580440in}{1.908731in}}%
\pgfpathlineto{\pgfqpoint{3.581680in}{1.907659in}}%
\pgfpathlineto{\pgfqpoint{3.582920in}{1.905061in}}%
\pgfpathlineto{\pgfqpoint{3.584160in}{1.905251in}}%
\pgfpathlineto{\pgfqpoint{3.586640in}{1.903814in}}%
\pgfpathlineto{\pgfqpoint{3.591600in}{1.903853in}}%
\pgfpathlineto{\pgfqpoint{3.594080in}{1.904011in}}%
\pgfpathlineto{\pgfqpoint{3.597800in}{1.901782in}}%
\pgfpathlineto{\pgfqpoint{3.600280in}{1.906176in}}%
\pgfpathlineto{\pgfqpoint{3.601520in}{1.904716in}}%
\pgfpathlineto{\pgfqpoint{3.604000in}{1.898921in}}%
\pgfpathlineto{\pgfqpoint{3.605240in}{1.901354in}}%
\pgfpathlineto{\pgfqpoint{3.611440in}{1.900730in}}%
\pgfpathlineto{\pgfqpoint{3.613920in}{1.904145in}}%
\pgfpathlineto{\pgfqpoint{3.615160in}{1.905622in}}%
\pgfpathlineto{\pgfqpoint{3.616400in}{1.904963in}}%
\pgfpathlineto{\pgfqpoint{3.620120in}{1.908733in}}%
\pgfpathlineto{\pgfqpoint{3.621360in}{1.907749in}}%
\pgfpathlineto{\pgfqpoint{3.622600in}{1.905291in}}%
\pgfpathlineto{\pgfqpoint{3.623840in}{1.906494in}}%
\pgfpathlineto{\pgfqpoint{3.625080in}{1.905946in}}%
\pgfpathlineto{\pgfqpoint{3.626320in}{1.907093in}}%
\pgfpathlineto{\pgfqpoint{3.632520in}{1.902006in}}%
\pgfpathlineto{\pgfqpoint{3.635000in}{1.903777in}}%
\pgfpathlineto{\pgfqpoint{3.637480in}{1.903535in}}%
\pgfpathlineto{\pgfqpoint{3.638720in}{1.905902in}}%
\pgfpathlineto{\pgfqpoint{3.642440in}{1.904130in}}%
\pgfpathlineto{\pgfqpoint{3.646160in}{1.903645in}}%
\pgfpathlineto{\pgfqpoint{3.649880in}{1.910087in}}%
\pgfpathlineto{\pgfqpoint{3.652360in}{1.906416in}}%
\pgfpathlineto{\pgfqpoint{3.654840in}{1.909400in}}%
\pgfpathlineto{\pgfqpoint{3.656080in}{1.913939in}}%
\pgfpathlineto{\pgfqpoint{3.661040in}{1.914617in}}%
\pgfpathlineto{\pgfqpoint{3.664760in}{1.908326in}}%
\pgfpathlineto{\pgfqpoint{3.667240in}{1.904062in}}%
\pgfpathlineto{\pgfqpoint{3.668480in}{1.905196in}}%
\pgfpathlineto{\pgfqpoint{3.670960in}{1.908303in}}%
\pgfpathlineto{\pgfqpoint{3.673440in}{1.908893in}}%
\pgfpathlineto{\pgfqpoint{3.674680in}{1.911370in}}%
\pgfpathlineto{\pgfqpoint{3.677160in}{1.910235in}}%
\pgfpathlineto{\pgfqpoint{3.679640in}{1.912776in}}%
\pgfpathlineto{\pgfqpoint{3.685840in}{1.907600in}}%
\pgfpathlineto{\pgfqpoint{3.690800in}{1.910536in}}%
\pgfpathlineto{\pgfqpoint{3.694520in}{1.910370in}}%
\pgfpathlineto{\pgfqpoint{3.699480in}{1.912960in}}%
\pgfpathlineto{\pgfqpoint{3.703200in}{1.917570in}}%
\pgfpathlineto{\pgfqpoint{3.704440in}{1.917892in}}%
\pgfpathlineto{\pgfqpoint{3.709400in}{1.913120in}}%
\pgfpathlineto{\pgfqpoint{3.713120in}{1.912850in}}%
\pgfpathlineto{\pgfqpoint{3.714360in}{1.911260in}}%
\pgfpathlineto{\pgfqpoint{3.718080in}{1.913406in}}%
\pgfpathlineto{\pgfqpoint{3.721800in}{1.910845in}}%
\pgfpathlineto{\pgfqpoint{3.724280in}{1.915229in}}%
\pgfpathlineto{\pgfqpoint{3.726760in}{1.910833in}}%
\pgfpathlineto{\pgfqpoint{3.728000in}{1.908788in}}%
\pgfpathlineto{\pgfqpoint{3.729240in}{1.910222in}}%
\pgfpathlineto{\pgfqpoint{3.732960in}{1.910272in}}%
\pgfpathlineto{\pgfqpoint{3.735440in}{1.909356in}}%
\pgfpathlineto{\pgfqpoint{3.737920in}{1.913200in}}%
\pgfpathlineto{\pgfqpoint{3.739160in}{1.914450in}}%
\pgfpathlineto{\pgfqpoint{3.740400in}{1.913703in}}%
\pgfpathlineto{\pgfqpoint{3.742880in}{1.918188in}}%
\pgfpathlineto{\pgfqpoint{3.745360in}{1.917292in}}%
\pgfpathlineto{\pgfqpoint{3.746600in}{1.914977in}}%
\pgfpathlineto{\pgfqpoint{3.747840in}{1.915576in}}%
\pgfpathlineto{\pgfqpoint{3.749080in}{1.914923in}}%
\pgfpathlineto{\pgfqpoint{3.750320in}{1.915782in}}%
\pgfpathlineto{\pgfqpoint{3.754040in}{1.913351in}}%
\pgfpathlineto{\pgfqpoint{3.756520in}{1.911732in}}%
\pgfpathlineto{\pgfqpoint{3.759000in}{1.913595in}}%
\pgfpathlineto{\pgfqpoint{3.761480in}{1.912888in}}%
\pgfpathlineto{\pgfqpoint{3.763960in}{1.915756in}}%
\pgfpathlineto{\pgfqpoint{3.765200in}{1.915936in}}%
\pgfpathlineto{\pgfqpoint{3.767680in}{1.914464in}}%
\pgfpathlineto{\pgfqpoint{3.770160in}{1.914674in}}%
\pgfpathlineto{\pgfqpoint{3.773880in}{1.921688in}}%
\pgfpathlineto{\pgfqpoint{3.776360in}{1.917428in}}%
\pgfpathlineto{\pgfqpoint{3.778840in}{1.920447in}}%
\pgfpathlineto{\pgfqpoint{3.781320in}{1.924871in}}%
\pgfpathlineto{\pgfqpoint{3.785040in}{1.925527in}}%
\pgfpathlineto{\pgfqpoint{3.791240in}{1.914216in}}%
\pgfpathlineto{\pgfqpoint{3.796200in}{1.918661in}}%
\pgfpathlineto{\pgfqpoint{3.797440in}{1.918952in}}%
\pgfpathlineto{\pgfqpoint{3.798680in}{1.920865in}}%
\pgfpathlineto{\pgfqpoint{3.801160in}{1.919179in}}%
\pgfpathlineto{\pgfqpoint{3.803640in}{1.921461in}}%
\pgfpathlineto{\pgfqpoint{3.807360in}{1.918363in}}%
\pgfpathlineto{\pgfqpoint{3.809840in}{1.917311in}}%
\pgfpathlineto{\pgfqpoint{3.813560in}{1.920093in}}%
\pgfpathlineto{\pgfqpoint{3.819760in}{1.920491in}}%
\pgfpathlineto{\pgfqpoint{3.822240in}{1.922425in}}%
\pgfpathlineto{\pgfqpoint{3.823480in}{1.922513in}}%
\pgfpathlineto{\pgfqpoint{3.827200in}{1.927216in}}%
\pgfpathlineto{\pgfqpoint{3.828440in}{1.927840in}}%
\pgfpathlineto{\pgfqpoint{3.833400in}{1.922617in}}%
\pgfpathlineto{\pgfqpoint{3.837120in}{1.922385in}}%
\pgfpathlineto{\pgfqpoint{3.838360in}{1.920737in}}%
\pgfpathlineto{\pgfqpoint{3.842080in}{1.923083in}}%
\pgfpathlineto{\pgfqpoint{3.843320in}{1.922554in}}%
\pgfpathlineto{\pgfqpoint{3.845800in}{1.920718in}}%
\pgfpathlineto{\pgfqpoint{3.848280in}{1.925722in}}%
\pgfpathlineto{\pgfqpoint{3.850760in}{1.921312in}}%
\pgfpathlineto{\pgfqpoint{3.852000in}{1.918910in}}%
\pgfpathlineto{\pgfqpoint{3.853240in}{1.922835in}}%
\pgfpathlineto{\pgfqpoint{3.859440in}{1.920183in}}%
\pgfpathlineto{\pgfqpoint{3.861920in}{1.924383in}}%
\pgfpathlineto{\pgfqpoint{3.863160in}{1.925631in}}%
\pgfpathlineto{\pgfqpoint{3.864400in}{1.924477in}}%
\pgfpathlineto{\pgfqpoint{3.866880in}{1.928451in}}%
\pgfpathlineto{\pgfqpoint{3.869360in}{1.927477in}}%
\pgfpathlineto{\pgfqpoint{3.870600in}{1.925315in}}%
\pgfpathlineto{\pgfqpoint{3.871840in}{1.926505in}}%
\pgfpathlineto{\pgfqpoint{3.874320in}{1.926398in}}%
\pgfpathlineto{\pgfqpoint{3.878040in}{1.924445in}}%
\pgfpathlineto{\pgfqpoint{3.879280in}{1.923447in}}%
\pgfpathlineto{\pgfqpoint{3.883000in}{1.925030in}}%
\pgfpathlineto{\pgfqpoint{3.885480in}{1.923867in}}%
\pgfpathlineto{\pgfqpoint{3.887960in}{1.926831in}}%
\pgfpathlineto{\pgfqpoint{3.891680in}{1.925695in}}%
\pgfpathlineto{\pgfqpoint{3.894160in}{1.925174in}}%
\pgfpathlineto{\pgfqpoint{3.897880in}{1.931155in}}%
\pgfpathlineto{\pgfqpoint{3.900360in}{1.927586in}}%
\pgfpathlineto{\pgfqpoint{3.902840in}{1.930521in}}%
\pgfpathlineto{\pgfqpoint{3.905320in}{1.935480in}}%
\pgfpathlineto{\pgfqpoint{3.909040in}{1.936131in}}%
\pgfpathlineto{\pgfqpoint{3.915240in}{1.925222in}}%
\pgfpathlineto{\pgfqpoint{3.922680in}{1.931788in}}%
\pgfpathlineto{\pgfqpoint{3.925160in}{1.929493in}}%
\pgfpathlineto{\pgfqpoint{3.927640in}{1.932864in}}%
\pgfpathlineto{\pgfqpoint{3.935080in}{1.929923in}}%
\pgfpathlineto{\pgfqpoint{3.936320in}{1.931877in}}%
\pgfpathlineto{\pgfqpoint{3.942520in}{1.930896in}}%
\pgfpathlineto{\pgfqpoint{3.946240in}{1.933913in}}%
\pgfpathlineto{\pgfqpoint{3.947480in}{1.934011in}}%
\pgfpathlineto{\pgfqpoint{3.951200in}{1.939429in}}%
\pgfpathlineto{\pgfqpoint{3.952440in}{1.939800in}}%
\pgfpathlineto{\pgfqpoint{3.957400in}{1.934091in}}%
\pgfpathlineto{\pgfqpoint{3.959880in}{1.934107in}}%
\pgfpathlineto{\pgfqpoint{3.963600in}{1.932186in}}%
\pgfpathlineto{\pgfqpoint{3.966080in}{1.933073in}}%
\pgfpathlineto{\pgfqpoint{3.969800in}{1.931202in}}%
\pgfpathlineto{\pgfqpoint{3.972280in}{1.936831in}}%
\pgfpathlineto{\pgfqpoint{3.976000in}{1.930485in}}%
\pgfpathlineto{\pgfqpoint{3.977240in}{1.933431in}}%
\pgfpathlineto{\pgfqpoint{3.980960in}{1.931669in}}%
\pgfpathlineto{\pgfqpoint{3.983440in}{1.930091in}}%
\pgfpathlineto{\pgfqpoint{3.987160in}{1.935356in}}%
\pgfpathlineto{\pgfqpoint{3.988400in}{1.933914in}}%
\pgfpathlineto{\pgfqpoint{3.992120in}{1.937095in}}%
\pgfpathlineto{\pgfqpoint{3.993360in}{1.936929in}}%
\pgfpathlineto{\pgfqpoint{3.994600in}{1.935033in}}%
\pgfpathlineto{\pgfqpoint{3.995840in}{1.936718in}}%
\pgfpathlineto{\pgfqpoint{3.997080in}{1.936236in}}%
\pgfpathlineto{\pgfqpoint{3.999560in}{1.936959in}}%
\pgfpathlineto{\pgfqpoint{4.004520in}{1.934503in}}%
\pgfpathlineto{\pgfqpoint{4.007000in}{1.935523in}}%
\pgfpathlineto{\pgfqpoint{4.009480in}{1.935237in}}%
\pgfpathlineto{\pgfqpoint{4.011960in}{1.938359in}}%
\pgfpathlineto{\pgfqpoint{4.015680in}{1.936184in}}%
\pgfpathlineto{\pgfqpoint{4.016920in}{1.935311in}}%
\pgfpathlineto{\pgfqpoint{4.018160in}{1.935870in}}%
\pgfpathlineto{\pgfqpoint{4.021880in}{1.942012in}}%
\pgfpathlineto{\pgfqpoint{4.024360in}{1.938420in}}%
\pgfpathlineto{\pgfqpoint{4.026840in}{1.941830in}}%
\pgfpathlineto{\pgfqpoint{4.029320in}{1.947148in}}%
\pgfpathlineto{\pgfqpoint{4.031800in}{1.948451in}}%
\pgfpathlineto{\pgfqpoint{4.033040in}{1.948122in}}%
\pgfpathlineto{\pgfqpoint{4.036760in}{1.941098in}}%
\pgfpathlineto{\pgfqpoint{4.039240in}{1.937869in}}%
\pgfpathlineto{\pgfqpoint{4.040480in}{1.938441in}}%
\pgfpathlineto{\pgfqpoint{4.042960in}{1.941108in}}%
\pgfpathlineto{\pgfqpoint{4.046680in}{1.943384in}}%
\pgfpathlineto{\pgfqpoint{4.049160in}{1.941043in}}%
\pgfpathlineto{\pgfqpoint{4.051640in}{1.944907in}}%
\pgfpathlineto{\pgfqpoint{4.059080in}{1.941584in}}%
\pgfpathlineto{\pgfqpoint{4.060320in}{1.943400in}}%
\pgfpathlineto{\pgfqpoint{4.064040in}{1.942139in}}%
\pgfpathlineto{\pgfqpoint{4.071480in}{1.945334in}}%
\pgfpathlineto{\pgfqpoint{4.075200in}{1.950301in}}%
\pgfpathlineto{\pgfqpoint{4.076440in}{1.950034in}}%
\pgfpathlineto{\pgfqpoint{4.081400in}{1.943028in}}%
\pgfpathlineto{\pgfqpoint{4.082640in}{1.943129in}}%
\pgfpathlineto{\pgfqpoint{4.087600in}{1.940221in}}%
\pgfpathlineto{\pgfqpoint{4.090080in}{1.940970in}}%
\pgfpathlineto{\pgfqpoint{4.093800in}{1.939364in}}%
\pgfpathlineto{\pgfqpoint{4.096280in}{1.945210in}}%
\pgfpathlineto{\pgfqpoint{4.100000in}{1.939413in}}%
\pgfpathlineto{\pgfqpoint{4.101240in}{1.940823in}}%
\pgfpathlineto{\pgfqpoint{4.107440in}{1.937836in}}%
\pgfpathlineto{\pgfqpoint{4.111160in}{1.943509in}}%
\pgfpathlineto{\pgfqpoint{4.112400in}{1.942451in}}%
\pgfpathlineto{\pgfqpoint{4.114880in}{1.945906in}}%
\pgfpathlineto{\pgfqpoint{4.118600in}{1.941292in}}%
\pgfpathlineto{\pgfqpoint{4.121080in}{1.942004in}}%
\pgfpathlineto{\pgfqpoint{4.123560in}{1.942087in}}%
\pgfpathlineto{\pgfqpoint{4.128520in}{1.939364in}}%
\pgfpathlineto{\pgfqpoint{4.131000in}{1.940569in}}%
\pgfpathlineto{\pgfqpoint{4.133480in}{1.940832in}}%
\pgfpathlineto{\pgfqpoint{4.135960in}{1.943439in}}%
\pgfpathlineto{\pgfqpoint{4.138440in}{1.942797in}}%
\pgfpathlineto{\pgfqpoint{4.142160in}{1.941859in}}%
\pgfpathlineto{\pgfqpoint{4.145880in}{1.947913in}}%
\pgfpathlineto{\pgfqpoint{4.148360in}{1.943611in}}%
\pgfpathlineto{\pgfqpoint{4.150840in}{1.947439in}}%
\pgfpathlineto{\pgfqpoint{4.153320in}{1.952648in}}%
\pgfpathlineto{\pgfqpoint{4.155800in}{1.953822in}}%
\pgfpathlineto{\pgfqpoint{4.159520in}{1.951394in}}%
\pgfpathlineto{\pgfqpoint{4.162000in}{1.944828in}}%
\pgfpathlineto{\pgfqpoint{4.163240in}{1.944180in}}%
\pgfpathlineto{\pgfqpoint{4.164480in}{1.944810in}}%
\pgfpathlineto{\pgfqpoint{4.166960in}{1.947789in}}%
\pgfpathlineto{\pgfqpoint{4.170680in}{1.951056in}}%
\pgfpathlineto{\pgfqpoint{4.173160in}{1.948849in}}%
\pgfpathlineto{\pgfqpoint{4.175640in}{1.952750in}}%
\pgfpathlineto{\pgfqpoint{4.176880in}{1.952390in}}%
\pgfpathlineto{\pgfqpoint{4.180600in}{1.949469in}}%
\pgfpathlineto{\pgfqpoint{4.183080in}{1.949767in}}%
\pgfpathlineto{\pgfqpoint{4.184320in}{1.951173in}}%
\pgfpathlineto{\pgfqpoint{4.188040in}{1.949444in}}%
\pgfpathlineto{\pgfqpoint{4.190520in}{1.949246in}}%
\pgfpathlineto{\pgfqpoint{4.193000in}{1.952854in}}%
\pgfpathlineto{\pgfqpoint{4.195480in}{1.954782in}}%
\pgfpathlineto{\pgfqpoint{4.197960in}{1.959403in}}%
\pgfpathlineto{\pgfqpoint{4.199200in}{1.959959in}}%
\pgfpathlineto{\pgfqpoint{4.200440in}{1.959233in}}%
\pgfpathlineto{\pgfqpoint{4.205400in}{1.950775in}}%
\pgfpathlineto{\pgfqpoint{4.206640in}{1.950645in}}%
\pgfpathlineto{\pgfqpoint{4.210360in}{1.947174in}}%
\pgfpathlineto{\pgfqpoint{4.214080in}{1.948721in}}%
\pgfpathlineto{\pgfqpoint{4.217800in}{1.946322in}}%
\pgfpathlineto{\pgfqpoint{4.220280in}{1.953172in}}%
\pgfpathlineto{\pgfqpoint{4.222760in}{1.949127in}}%
\pgfpathlineto{\pgfqpoint{4.224000in}{1.947429in}}%
\pgfpathlineto{\pgfqpoint{4.226480in}{1.948667in}}%
\pgfpathlineto{\pgfqpoint{4.227720in}{1.948648in}}%
\pgfpathlineto{\pgfqpoint{4.231440in}{1.945263in}}%
\pgfpathlineto{\pgfqpoint{4.235160in}{1.951071in}}%
\pgfpathlineto{\pgfqpoint{4.236400in}{1.950156in}}%
\pgfpathlineto{\pgfqpoint{4.238880in}{1.954497in}}%
\pgfpathlineto{\pgfqpoint{4.242600in}{1.948768in}}%
\pgfpathlineto{\pgfqpoint{4.245080in}{1.949886in}}%
\pgfpathlineto{\pgfqpoint{4.247560in}{1.949332in}}%
\pgfpathlineto{\pgfqpoint{4.250040in}{1.947738in}}%
\pgfpathlineto{\pgfqpoint{4.252520in}{1.945824in}}%
\pgfpathlineto{\pgfqpoint{4.255000in}{1.947407in}}%
\pgfpathlineto{\pgfqpoint{4.257480in}{1.947486in}}%
\pgfpathlineto{\pgfqpoint{4.259960in}{1.949895in}}%
\pgfpathlineto{\pgfqpoint{4.261200in}{1.949957in}}%
\pgfpathlineto{\pgfqpoint{4.264920in}{1.947446in}}%
\pgfpathlineto{\pgfqpoint{4.266160in}{1.947799in}}%
\pgfpathlineto{\pgfqpoint{4.268640in}{1.952686in}}%
\pgfpathlineto{\pgfqpoint{4.269880in}{1.953488in}}%
\pgfpathlineto{\pgfqpoint{4.272360in}{1.948736in}}%
\pgfpathlineto{\pgfqpoint{4.274840in}{1.952882in}}%
\pgfpathlineto{\pgfqpoint{4.277320in}{1.958062in}}%
\pgfpathlineto{\pgfqpoint{4.279800in}{1.959040in}}%
\pgfpathlineto{\pgfqpoint{4.283520in}{1.956291in}}%
\pgfpathlineto{\pgfqpoint{4.286000in}{1.950036in}}%
\pgfpathlineto{\pgfqpoint{4.288480in}{1.950478in}}%
\pgfpathlineto{\pgfqpoint{4.290960in}{1.954285in}}%
\pgfpathlineto{\pgfqpoint{4.294680in}{1.957408in}}%
\pgfpathlineto{\pgfqpoint{4.297160in}{1.955433in}}%
\pgfpathlineto{\pgfqpoint{4.299640in}{1.957796in}}%
\pgfpathlineto{\pgfqpoint{4.305840in}{1.954807in}}%
\pgfpathlineto{\pgfqpoint{4.308320in}{1.957321in}}%
\pgfpathlineto{\pgfqpoint{4.309560in}{1.956960in}}%
\pgfpathlineto{\pgfqpoint{4.310800in}{1.958184in}}%
\pgfpathlineto{\pgfqpoint{4.314520in}{1.956898in}}%
\pgfpathlineto{\pgfqpoint{4.318240in}{1.962538in}}%
\pgfpathlineto{\pgfqpoint{4.319480in}{1.963726in}}%
\pgfpathlineto{\pgfqpoint{4.321960in}{1.968536in}}%
\pgfpathlineto{\pgfqpoint{4.324440in}{1.969138in}}%
\pgfpathlineto{\pgfqpoint{4.329400in}{1.960825in}}%
\pgfpathlineto{\pgfqpoint{4.336840in}{1.955652in}}%
\pgfpathlineto{\pgfqpoint{4.339320in}{1.955116in}}%
\pgfpathlineto{\pgfqpoint{4.341800in}{1.953795in}}%
\pgfpathlineto{\pgfqpoint{4.344280in}{1.959757in}}%
\pgfpathlineto{\pgfqpoint{4.349240in}{1.953791in}}%
\pgfpathlineto{\pgfqpoint{4.351720in}{1.953909in}}%
\pgfpathlineto{\pgfqpoint{4.355440in}{1.950643in}}%
\pgfpathlineto{\pgfqpoint{4.359160in}{1.957071in}}%
\pgfpathlineto{\pgfqpoint{4.360400in}{1.955813in}}%
\pgfpathlineto{\pgfqpoint{4.362880in}{1.959462in}}%
\pgfpathlineto{\pgfqpoint{4.366600in}{1.953906in}}%
\pgfpathlineto{\pgfqpoint{4.367840in}{1.955140in}}%
\pgfpathlineto{\pgfqpoint{4.371560in}{1.954023in}}%
\pgfpathlineto{\pgfqpoint{4.374040in}{1.952716in}}%
\pgfpathlineto{\pgfqpoint{4.375280in}{1.950740in}}%
\pgfpathlineto{\pgfqpoint{4.376520in}{1.951068in}}%
\pgfpathlineto{\pgfqpoint{4.379000in}{1.952746in}}%
\pgfpathlineto{\pgfqpoint{4.381480in}{1.953043in}}%
\pgfpathlineto{\pgfqpoint{4.383960in}{1.955330in}}%
\pgfpathlineto{\pgfqpoint{4.385200in}{1.955521in}}%
\pgfpathlineto{\pgfqpoint{4.387680in}{1.953681in}}%
\pgfpathlineto{\pgfqpoint{4.390160in}{1.952475in}}%
\pgfpathlineto{\pgfqpoint{4.392640in}{1.957641in}}%
\pgfpathlineto{\pgfqpoint{4.393880in}{1.958611in}}%
\pgfpathlineto{\pgfqpoint{4.396360in}{1.954161in}}%
\pgfpathlineto{\pgfqpoint{4.398840in}{1.958486in}}%
\pgfpathlineto{\pgfqpoint{4.401320in}{1.964154in}}%
\pgfpathlineto{\pgfqpoint{4.403800in}{1.964696in}}%
\pgfpathlineto{\pgfqpoint{4.407520in}{1.963083in}}%
\pgfpathlineto{\pgfqpoint{4.410000in}{1.957021in}}%
\pgfpathlineto{\pgfqpoint{4.412480in}{1.957261in}}%
\pgfpathlineto{\pgfqpoint{4.414960in}{1.960453in}}%
\pgfpathlineto{\pgfqpoint{4.418680in}{1.963660in}}%
\pgfpathlineto{\pgfqpoint{4.421160in}{1.961890in}}%
\pgfpathlineto{\pgfqpoint{4.423640in}{1.964634in}}%
\pgfpathlineto{\pgfqpoint{4.429840in}{1.961939in}}%
\pgfpathlineto{\pgfqpoint{4.432320in}{1.964541in}}%
\pgfpathlineto{\pgfqpoint{4.434800in}{1.964854in}}%
\pgfpathlineto{\pgfqpoint{4.438520in}{1.964957in}}%
\pgfpathlineto{\pgfqpoint{4.441000in}{1.968715in}}%
\pgfpathlineto{\pgfqpoint{4.448440in}{1.975442in}}%
\pgfpathlineto{\pgfqpoint{4.454640in}{1.965480in}}%
\pgfpathlineto{\pgfqpoint{4.458360in}{1.961292in}}%
\pgfpathlineto{\pgfqpoint{4.462080in}{1.962241in}}%
\pgfpathlineto{\pgfqpoint{4.465800in}{1.960607in}}%
\pgfpathlineto{\pgfqpoint{4.468280in}{1.966252in}}%
\pgfpathlineto{\pgfqpoint{4.472000in}{1.960263in}}%
\pgfpathlineto{\pgfqpoint{4.475720in}{1.962701in}}%
\pgfpathlineto{\pgfqpoint{4.479440in}{1.960287in}}%
\pgfpathlineto{\pgfqpoint{4.483160in}{1.967557in}}%
\pgfpathlineto{\pgfqpoint{4.484400in}{1.966138in}}%
\pgfpathlineto{\pgfqpoint{4.486880in}{1.970267in}}%
\pgfpathlineto{\pgfqpoint{4.490600in}{1.965513in}}%
\pgfpathlineto{\pgfqpoint{4.491840in}{1.966089in}}%
\pgfpathlineto{\pgfqpoint{4.493080in}{1.964781in}}%
\pgfpathlineto{\pgfqpoint{4.495560in}{1.965224in}}%
\pgfpathlineto{\pgfqpoint{4.498040in}{1.963202in}}%
\pgfpathlineto{\pgfqpoint{4.499280in}{1.961129in}}%
\pgfpathlineto{\pgfqpoint{4.505480in}{1.962502in}}%
\pgfpathlineto{\pgfqpoint{4.507960in}{1.964274in}}%
\pgfpathlineto{\pgfqpoint{4.509200in}{1.963874in}}%
\pgfpathlineto{\pgfqpoint{4.511680in}{1.961497in}}%
\pgfpathlineto{\pgfqpoint{4.512920in}{1.960396in}}%
\pgfpathlineto{\pgfqpoint{4.514160in}{1.960899in}}%
\pgfpathlineto{\pgfqpoint{4.516640in}{1.966381in}}%
\pgfpathlineto{\pgfqpoint{4.517880in}{1.968247in}}%
\pgfpathlineto{\pgfqpoint{4.520360in}{1.963259in}}%
\pgfpathlineto{\pgfqpoint{4.522840in}{1.966637in}}%
\pgfpathlineto{\pgfqpoint{4.525320in}{1.972875in}}%
\pgfpathlineto{\pgfqpoint{4.529040in}{1.972598in}}%
\pgfpathlineto{\pgfqpoint{4.531520in}{1.971192in}}%
\pgfpathlineto{\pgfqpoint{4.534000in}{1.965067in}}%
\pgfpathlineto{\pgfqpoint{4.535240in}{1.964536in}}%
\pgfpathlineto{\pgfqpoint{4.541440in}{1.969328in}}%
\pgfpathlineto{\pgfqpoint{4.542680in}{1.971245in}}%
\pgfpathlineto{\pgfqpoint{4.545160in}{1.969075in}}%
\pgfpathlineto{\pgfqpoint{4.547640in}{1.971782in}}%
\pgfpathlineto{\pgfqpoint{4.552600in}{1.968542in}}%
\pgfpathlineto{\pgfqpoint{4.553840in}{1.968583in}}%
\pgfpathlineto{\pgfqpoint{4.556320in}{1.971285in}}%
\pgfpathlineto{\pgfqpoint{4.557560in}{1.970969in}}%
\pgfpathlineto{\pgfqpoint{4.560040in}{1.972223in}}%
\pgfpathlineto{\pgfqpoint{4.562520in}{1.971352in}}%
\pgfpathlineto{\pgfqpoint{4.565000in}{1.975709in}}%
\pgfpathlineto{\pgfqpoint{4.568720in}{1.981473in}}%
\pgfpathlineto{\pgfqpoint{4.571200in}{1.983340in}}%
\pgfpathlineto{\pgfqpoint{4.572440in}{1.982937in}}%
\pgfpathlineto{\pgfqpoint{4.578640in}{1.973658in}}%
\pgfpathlineto{\pgfqpoint{4.583600in}{1.969389in}}%
\pgfpathlineto{\pgfqpoint{4.587320in}{1.969563in}}%
\pgfpathlineto{\pgfqpoint{4.588560in}{1.968301in}}%
\pgfpathlineto{\pgfqpoint{4.589800in}{1.968983in}}%
\pgfpathlineto{\pgfqpoint{4.592280in}{1.974008in}}%
\pgfpathlineto{\pgfqpoint{4.596000in}{1.967537in}}%
\pgfpathlineto{\pgfqpoint{4.597240in}{1.970238in}}%
\pgfpathlineto{\pgfqpoint{4.602200in}{1.969642in}}%
\pgfpathlineto{\pgfqpoint{4.603440in}{1.967621in}}%
\pgfpathlineto{\pgfqpoint{4.607160in}{1.975988in}}%
\pgfpathlineto{\pgfqpoint{4.608400in}{1.975091in}}%
\pgfpathlineto{\pgfqpoint{4.610880in}{1.979905in}}%
\pgfpathlineto{\pgfqpoint{4.614600in}{1.974749in}}%
\pgfpathlineto{\pgfqpoint{4.618320in}{1.973752in}}%
\pgfpathlineto{\pgfqpoint{4.620800in}{1.972414in}}%
\pgfpathlineto{\pgfqpoint{4.624520in}{1.968973in}}%
\pgfpathlineto{\pgfqpoint{4.627000in}{1.969671in}}%
\pgfpathlineto{\pgfqpoint{4.629480in}{1.970510in}}%
\pgfpathlineto{\pgfqpoint{4.630720in}{1.972253in}}%
\pgfpathlineto{\pgfqpoint{4.638160in}{1.970030in}}%
\pgfpathlineto{\pgfqpoint{4.640640in}{1.975798in}}%
\pgfpathlineto{\pgfqpoint{4.641880in}{1.977521in}}%
\pgfpathlineto{\pgfqpoint{4.644360in}{1.972176in}}%
\pgfpathlineto{\pgfqpoint{4.646840in}{1.975584in}}%
\pgfpathlineto{\pgfqpoint{4.649320in}{1.982093in}}%
\pgfpathlineto{\pgfqpoint{4.651800in}{1.982599in}}%
\pgfpathlineto{\pgfqpoint{4.655520in}{1.979955in}}%
\pgfpathlineto{\pgfqpoint{4.658000in}{1.973407in}}%
\pgfpathlineto{\pgfqpoint{4.659240in}{1.972214in}}%
\pgfpathlineto{\pgfqpoint{4.662960in}{1.974307in}}%
\pgfpathlineto{\pgfqpoint{4.667920in}{1.976821in}}%
\pgfpathlineto{\pgfqpoint{4.669160in}{1.975467in}}%
\pgfpathlineto{\pgfqpoint{4.672880in}{1.977110in}}%
\pgfpathlineto{\pgfqpoint{4.675360in}{1.976071in}}%
\pgfpathlineto{\pgfqpoint{4.677840in}{1.976637in}}%
\pgfpathlineto{\pgfqpoint{4.680320in}{1.979841in}}%
\pgfpathlineto{\pgfqpoint{4.682800in}{1.979348in}}%
\pgfpathlineto{\pgfqpoint{4.684040in}{1.980075in}}%
\pgfpathlineto{\pgfqpoint{4.686520in}{1.978033in}}%
\pgfpathlineto{\pgfqpoint{4.690240in}{1.983886in}}%
\pgfpathlineto{\pgfqpoint{4.691480in}{1.985258in}}%
\pgfpathlineto{\pgfqpoint{4.693960in}{1.990822in}}%
\pgfpathlineto{\pgfqpoint{4.695200in}{1.991514in}}%
\pgfpathlineto{\pgfqpoint{4.697680in}{1.987196in}}%
\pgfpathlineto{\pgfqpoint{4.701400in}{1.981629in}}%
\pgfpathlineto{\pgfqpoint{4.707600in}{1.975279in}}%
\pgfpathlineto{\pgfqpoint{4.710080in}{1.975407in}}%
\pgfpathlineto{\pgfqpoint{4.713800in}{1.974210in}}%
\pgfpathlineto{\pgfqpoint{4.716280in}{1.979107in}}%
\pgfpathlineto{\pgfqpoint{4.721240in}{1.972897in}}%
\pgfpathlineto{\pgfqpoint{4.723720in}{1.974731in}}%
\pgfpathlineto{\pgfqpoint{4.726200in}{1.973874in}}%
\pgfpathlineto{\pgfqpoint{4.727440in}{1.971960in}}%
\pgfpathlineto{\pgfqpoint{4.731160in}{1.980476in}}%
\pgfpathlineto{\pgfqpoint{4.732400in}{1.979914in}}%
\pgfpathlineto{\pgfqpoint{4.734880in}{1.985657in}}%
\pgfpathlineto{\pgfqpoint{4.739840in}{1.981760in}}%
\pgfpathlineto{\pgfqpoint{4.742320in}{1.981252in}}%
\pgfpathlineto{\pgfqpoint{4.743560in}{1.981351in}}%
\pgfpathlineto{\pgfqpoint{4.748520in}{1.976063in}}%
\pgfpathlineto{\pgfqpoint{4.751000in}{1.977814in}}%
\pgfpathlineto{\pgfqpoint{4.753480in}{1.978394in}}%
\pgfpathlineto{\pgfqpoint{4.754720in}{1.979912in}}%
\pgfpathlineto{\pgfqpoint{4.759680in}{1.978297in}}%
\pgfpathlineto{\pgfqpoint{4.760920in}{1.976826in}}%
\pgfpathlineto{\pgfqpoint{4.762160in}{1.977202in}}%
\pgfpathlineto{\pgfqpoint{4.764640in}{1.982421in}}%
\pgfpathlineto{\pgfqpoint{4.765880in}{1.983743in}}%
\pgfpathlineto{\pgfqpoint{4.768360in}{1.978663in}}%
\pgfpathlineto{\pgfqpoint{4.770840in}{1.982748in}}%
\pgfpathlineto{\pgfqpoint{4.773320in}{1.988010in}}%
\pgfpathlineto{\pgfqpoint{4.775800in}{1.988524in}}%
\pgfpathlineto{\pgfqpoint{4.777040in}{1.988051in}}%
\pgfpathlineto{\pgfqpoint{4.783240in}{1.977856in}}%
\pgfpathlineto{\pgfqpoint{4.790680in}{1.981913in}}%
\pgfpathlineto{\pgfqpoint{4.793160in}{1.979143in}}%
\pgfpathlineto{\pgfqpoint{4.796880in}{1.981318in}}%
\pgfpathlineto{\pgfqpoint{4.799360in}{1.980641in}}%
\pgfpathlineto{\pgfqpoint{4.805560in}{1.983513in}}%
\pgfpathlineto{\pgfqpoint{4.809280in}{1.982613in}}%
\pgfpathlineto{\pgfqpoint{4.810520in}{1.981195in}}%
\pgfpathlineto{\pgfqpoint{4.813000in}{1.983986in}}%
\pgfpathlineto{\pgfqpoint{4.816720in}{1.990341in}}%
\pgfpathlineto{\pgfqpoint{4.819200in}{1.993668in}}%
\pgfpathlineto{\pgfqpoint{4.821680in}{1.989509in}}%
\pgfpathlineto{\pgfqpoint{4.822920in}{1.987764in}}%
\pgfpathlineto{\pgfqpoint{4.824160in}{1.988240in}}%
\pgfpathlineto{\pgfqpoint{4.829120in}{1.982124in}}%
\pgfpathlineto{\pgfqpoint{4.834080in}{1.979284in}}%
\pgfpathlineto{\pgfqpoint{4.837800in}{1.976614in}}%
\pgfpathlineto{\pgfqpoint{4.840280in}{1.982474in}}%
\pgfpathlineto{\pgfqpoint{4.844000in}{1.975894in}}%
\pgfpathlineto{\pgfqpoint{4.845240in}{1.973338in}}%
\pgfpathlineto{\pgfqpoint{4.847720in}{1.974268in}}%
\pgfpathlineto{\pgfqpoint{4.851440in}{1.971206in}}%
\pgfpathlineto{\pgfqpoint{4.855160in}{1.979002in}}%
\pgfpathlineto{\pgfqpoint{4.856400in}{1.979280in}}%
\pgfpathlineto{\pgfqpoint{4.858880in}{1.985190in}}%
\pgfpathlineto{\pgfqpoint{4.860120in}{1.983866in}}%
\pgfpathlineto{\pgfqpoint{4.861360in}{1.983868in}}%
\pgfpathlineto{\pgfqpoint{4.863840in}{1.981392in}}%
\pgfpathlineto{\pgfqpoint{4.865080in}{1.981064in}}%
\pgfpathlineto{\pgfqpoint{4.867560in}{1.982269in}}%
\pgfpathlineto{\pgfqpoint{4.871280in}{1.977913in}}%
\pgfpathlineto{\pgfqpoint{4.872520in}{1.978130in}}%
\pgfpathlineto{\pgfqpoint{4.875000in}{1.979488in}}%
\pgfpathlineto{\pgfqpoint{4.877480in}{1.979221in}}%
\pgfpathlineto{\pgfqpoint{4.879960in}{1.980153in}}%
\pgfpathlineto{\pgfqpoint{4.882440in}{1.979086in}}%
\pgfpathlineto{\pgfqpoint{4.883680in}{1.978856in}}%
\pgfpathlineto{\pgfqpoint{4.886160in}{1.977629in}}%
\pgfpathlineto{\pgfqpoint{4.888640in}{1.982846in}}%
\pgfpathlineto{\pgfqpoint{4.889880in}{1.984392in}}%
\pgfpathlineto{\pgfqpoint{4.892360in}{1.979754in}}%
\pgfpathlineto{\pgfqpoint{4.897320in}{1.989172in}}%
\pgfpathlineto{\pgfqpoint{4.899800in}{1.989700in}}%
\pgfpathlineto{\pgfqpoint{4.903520in}{1.986308in}}%
\pgfpathlineto{\pgfqpoint{4.907240in}{1.977190in}}%
\pgfpathlineto{\pgfqpoint{4.910960in}{1.977931in}}%
\pgfpathlineto{\pgfqpoint{4.913440in}{1.979031in}}%
\pgfpathlineto{\pgfqpoint{4.914680in}{1.980586in}}%
\pgfpathlineto{\pgfqpoint{4.917160in}{1.978686in}}%
\pgfpathlineto{\pgfqpoint{4.919640in}{1.981310in}}%
\pgfpathlineto{\pgfqpoint{4.923360in}{1.980155in}}%
\pgfpathlineto{\pgfqpoint{4.928320in}{1.984655in}}%
\pgfpathlineto{\pgfqpoint{4.930800in}{1.983312in}}%
\pgfpathlineto{\pgfqpoint{4.932040in}{1.984286in}}%
\pgfpathlineto{\pgfqpoint{4.934520in}{1.982076in}}%
\pgfpathlineto{\pgfqpoint{4.937000in}{1.985762in}}%
\pgfpathlineto{\pgfqpoint{4.939480in}{1.987560in}}%
\pgfpathlineto{\pgfqpoint{4.941960in}{1.994115in}}%
\pgfpathlineto{\pgfqpoint{4.943200in}{1.995022in}}%
\pgfpathlineto{\pgfqpoint{4.944440in}{1.993740in}}%
\pgfpathlineto{\pgfqpoint{4.946920in}{1.988867in}}%
\pgfpathlineto{\pgfqpoint{4.948160in}{1.989350in}}%
\pgfpathlineto{\pgfqpoint{4.951880in}{1.984613in}}%
\pgfpathlineto{\pgfqpoint{4.955600in}{1.982728in}}%
\pgfpathlineto{\pgfqpoint{4.959320in}{1.980610in}}%
\pgfpathlineto{\pgfqpoint{4.960560in}{1.978316in}}%
\pgfpathlineto{\pgfqpoint{4.961800in}{1.978475in}}%
\pgfpathlineto{\pgfqpoint{4.964280in}{1.983923in}}%
\pgfpathlineto{\pgfqpoint{4.966760in}{1.979403in}}%
\pgfpathlineto{\pgfqpoint{4.969240in}{1.975099in}}%
\pgfpathlineto{\pgfqpoint{4.971720in}{1.976445in}}%
\pgfpathlineto{\pgfqpoint{4.974200in}{1.974938in}}%
\pgfpathlineto{\pgfqpoint{4.975440in}{1.972583in}}%
\pgfpathlineto{\pgfqpoint{4.979160in}{1.979127in}}%
\pgfpathlineto{\pgfqpoint{4.980400in}{1.979560in}}%
\pgfpathlineto{\pgfqpoint{4.982880in}{1.984539in}}%
\pgfpathlineto{\pgfqpoint{4.984120in}{1.983223in}}%
\pgfpathlineto{\pgfqpoint{4.985360in}{1.984408in}}%
\pgfpathlineto{\pgfqpoint{4.989080in}{1.982053in}}%
\pgfpathlineto{\pgfqpoint{4.991560in}{1.982759in}}%
\pgfpathlineto{\pgfqpoint{4.995280in}{1.977807in}}%
\pgfpathlineto{\pgfqpoint{5.001480in}{1.976989in}}%
\pgfpathlineto{\pgfqpoint{5.003960in}{1.978593in}}%
\pgfpathlineto{\pgfqpoint{5.006440in}{1.978380in}}%
\pgfpathlineto{\pgfqpoint{5.010160in}{1.976856in}}%
\pgfpathlineto{\pgfqpoint{5.013880in}{1.983671in}}%
\pgfpathlineto{\pgfqpoint{5.016360in}{1.978653in}}%
\pgfpathlineto{\pgfqpoint{5.020080in}{1.988861in}}%
\pgfpathlineto{\pgfqpoint{5.025040in}{1.987410in}}%
\pgfpathlineto{\pgfqpoint{5.028760in}{1.981092in}}%
\pgfpathlineto{\pgfqpoint{5.031240in}{1.976394in}}%
\pgfpathlineto{\pgfqpoint{5.033720in}{1.977658in}}%
\pgfpathlineto{\pgfqpoint{5.034960in}{1.977389in}}%
\pgfpathlineto{\pgfqpoint{5.038680in}{1.982312in}}%
\pgfpathlineto{\pgfqpoint{5.041160in}{1.980434in}}%
\pgfpathlineto{\pgfqpoint{5.043640in}{1.983410in}}%
\pgfpathlineto{\pgfqpoint{5.046120in}{1.981125in}}%
\pgfpathlineto{\pgfqpoint{5.047360in}{1.980910in}}%
\pgfpathlineto{\pgfqpoint{5.048600in}{1.981892in}}%
\pgfpathlineto{\pgfqpoint{5.049840in}{1.981404in}}%
\pgfpathlineto{\pgfqpoint{5.052320in}{1.985870in}}%
\pgfpathlineto{\pgfqpoint{5.054800in}{1.985437in}}%
\pgfpathlineto{\pgfqpoint{5.056040in}{1.986498in}}%
\pgfpathlineto{\pgfqpoint{5.058520in}{1.984272in}}%
\pgfpathlineto{\pgfqpoint{5.059760in}{1.986635in}}%
\pgfpathlineto{\pgfqpoint{5.062240in}{1.986664in}}%
\pgfpathlineto{\pgfqpoint{5.063480in}{1.987967in}}%
\pgfpathlineto{\pgfqpoint{5.067200in}{1.996341in}}%
\pgfpathlineto{\pgfqpoint{5.077120in}{1.986212in}}%
\pgfpathlineto{\pgfqpoint{5.083320in}{1.984813in}}%
\pgfpathlineto{\pgfqpoint{5.085800in}{1.981983in}}%
\pgfpathlineto{\pgfqpoint{5.088280in}{1.986651in}}%
\pgfpathlineto{\pgfqpoint{5.090760in}{1.982800in}}%
\pgfpathlineto{\pgfqpoint{5.093240in}{1.978602in}}%
\pgfpathlineto{\pgfqpoint{5.095720in}{1.979389in}}%
\pgfpathlineto{\pgfqpoint{5.098200in}{1.977862in}}%
\pgfpathlineto{\pgfqpoint{5.099440in}{1.975096in}}%
\pgfpathlineto{\pgfqpoint{5.103160in}{1.981287in}}%
\pgfpathlineto{\pgfqpoint{5.104400in}{1.981477in}}%
\pgfpathlineto{\pgfqpoint{5.106880in}{1.988057in}}%
\pgfpathlineto{\pgfqpoint{5.108120in}{1.987360in}}%
\pgfpathlineto{\pgfqpoint{5.109360in}{1.989023in}}%
\pgfpathlineto{\pgfqpoint{5.114320in}{1.986424in}}%
\pgfpathlineto{\pgfqpoint{5.115560in}{1.986559in}}%
\pgfpathlineto{\pgfqpoint{5.119280in}{1.981959in}}%
\pgfpathlineto{\pgfqpoint{5.125480in}{1.982076in}}%
\pgfpathlineto{\pgfqpoint{5.127960in}{1.983602in}}%
\pgfpathlineto{\pgfqpoint{5.130440in}{1.984379in}}%
\pgfpathlineto{\pgfqpoint{5.134160in}{1.984250in}}%
\pgfpathlineto{\pgfqpoint{5.137880in}{1.991073in}}%
\pgfpathlineto{\pgfqpoint{5.140360in}{1.986440in}}%
\pgfpathlineto{\pgfqpoint{5.144080in}{1.993969in}}%
\pgfpathlineto{\pgfqpoint{5.149040in}{1.992449in}}%
\pgfpathlineto{\pgfqpoint{5.152760in}{1.986420in}}%
\pgfpathlineto{\pgfqpoint{5.155240in}{1.981064in}}%
\pgfpathlineto{\pgfqpoint{5.160200in}{1.983272in}}%
\pgfpathlineto{\pgfqpoint{5.162680in}{1.986453in}}%
\pgfpathlineto{\pgfqpoint{5.165160in}{1.985025in}}%
\pgfpathlineto{\pgfqpoint{5.167640in}{1.987725in}}%
\pgfpathlineto{\pgfqpoint{5.170120in}{1.984818in}}%
\pgfpathlineto{\pgfqpoint{5.171360in}{1.984397in}}%
\pgfpathlineto{\pgfqpoint{5.175080in}{1.986566in}}%
\pgfpathlineto{\pgfqpoint{5.176320in}{1.989601in}}%
\pgfpathlineto{\pgfqpoint{5.178800in}{1.988599in}}%
\pgfpathlineto{\pgfqpoint{5.180040in}{1.990181in}}%
\pgfpathlineto{\pgfqpoint{5.182520in}{1.987123in}}%
\pgfpathlineto{\pgfqpoint{5.183760in}{1.989758in}}%
\pgfpathlineto{\pgfqpoint{5.186240in}{1.990284in}}%
\pgfpathlineto{\pgfqpoint{5.188720in}{1.995941in}}%
\pgfpathlineto{\pgfqpoint{5.191200in}{2.000302in}}%
\pgfpathlineto{\pgfqpoint{5.194920in}{1.994827in}}%
\pgfpathlineto{\pgfqpoint{5.196160in}{1.996323in}}%
\pgfpathlineto{\pgfqpoint{5.198640in}{1.991593in}}%
\pgfpathlineto{\pgfqpoint{5.201120in}{1.989991in}}%
\pgfpathlineto{\pgfqpoint{5.203600in}{1.989361in}}%
\pgfpathlineto{\pgfqpoint{5.206080in}{1.988810in}}%
\pgfpathlineto{\pgfqpoint{5.209800in}{1.984333in}}%
\pgfpathlineto{\pgfqpoint{5.213520in}{1.988209in}}%
\pgfpathlineto{\pgfqpoint{5.216000in}{1.981673in}}%
\pgfpathlineto{\pgfqpoint{5.218480in}{1.987661in}}%
\pgfpathlineto{\pgfqpoint{5.220960in}{1.986460in}}%
\pgfpathlineto{\pgfqpoint{5.223440in}{1.981529in}}%
\pgfpathlineto{\pgfqpoint{5.227160in}{1.989071in}}%
\pgfpathlineto{\pgfqpoint{5.228400in}{1.988498in}}%
\pgfpathlineto{\pgfqpoint{5.233360in}{1.994955in}}%
\pgfpathlineto{\pgfqpoint{5.237080in}{1.992375in}}%
\pgfpathlineto{\pgfqpoint{5.239560in}{1.994063in}}%
\pgfpathlineto{\pgfqpoint{5.242040in}{1.991046in}}%
\pgfpathlineto{\pgfqpoint{5.244520in}{1.989720in}}%
\pgfpathlineto{\pgfqpoint{5.248240in}{1.989953in}}%
\pgfpathlineto{\pgfqpoint{5.251960in}{1.993011in}}%
\pgfpathlineto{\pgfqpoint{5.258160in}{1.994040in}}%
\pgfpathlineto{\pgfqpoint{5.261880in}{2.001794in}}%
\pgfpathlineto{\pgfqpoint{5.264360in}{1.995816in}}%
\pgfpathlineto{\pgfqpoint{5.268080in}{2.001569in}}%
\pgfpathlineto{\pgfqpoint{5.274280in}{1.999204in}}%
\pgfpathlineto{\pgfqpoint{5.275520in}{1.998041in}}%
\pgfpathlineto{\pgfqpoint{5.279240in}{1.988039in}}%
\pgfpathlineto{\pgfqpoint{5.282960in}{1.989764in}}%
\pgfpathlineto{\pgfqpoint{5.286680in}{1.994186in}}%
\pgfpathlineto{\pgfqpoint{5.289160in}{1.993078in}}%
\pgfpathlineto{\pgfqpoint{5.291640in}{1.995618in}}%
\pgfpathlineto{\pgfqpoint{5.294120in}{1.990763in}}%
\pgfpathlineto{\pgfqpoint{5.295360in}{1.990121in}}%
\pgfpathlineto{\pgfqpoint{5.296600in}{1.991249in}}%
\pgfpathlineto{\pgfqpoint{5.297840in}{1.990864in}}%
\pgfpathlineto{\pgfqpoint{5.300320in}{1.995092in}}%
\pgfpathlineto{\pgfqpoint{5.302800in}{1.993705in}}%
\pgfpathlineto{\pgfqpoint{5.304040in}{1.995470in}}%
\pgfpathlineto{\pgfqpoint{5.306520in}{1.991551in}}%
\pgfpathlineto{\pgfqpoint{5.307760in}{1.993648in}}%
\pgfpathlineto{\pgfqpoint{5.309000in}{1.993033in}}%
\pgfpathlineto{\pgfqpoint{5.311480in}{1.995178in}}%
\pgfpathlineto{\pgfqpoint{5.315200in}{2.003218in}}%
\pgfpathlineto{\pgfqpoint{5.318920in}{1.999378in}}%
\pgfpathlineto{\pgfqpoint{5.320160in}{2.001595in}}%
\pgfpathlineto{\pgfqpoint{5.322640in}{1.998145in}}%
\pgfpathlineto{\pgfqpoint{5.325120in}{1.995999in}}%
\pgfpathlineto{\pgfqpoint{5.328840in}{1.994134in}}%
\pgfpathlineto{\pgfqpoint{5.331320in}{1.992731in}}%
\pgfpathlineto{\pgfqpoint{5.333800in}{1.989957in}}%
\pgfpathlineto{\pgfqpoint{5.336280in}{1.993464in}}%
\pgfpathlineto{\pgfqpoint{5.337520in}{1.992574in}}%
\pgfpathlineto{\pgfqpoint{5.340000in}{1.987111in}}%
\pgfpathlineto{\pgfqpoint{5.342480in}{1.990615in}}%
\pgfpathlineto{\pgfqpoint{5.344960in}{1.990335in}}%
\pgfpathlineto{\pgfqpoint{5.347440in}{1.984427in}}%
\pgfpathlineto{\pgfqpoint{5.351160in}{1.991726in}}%
\pgfpathlineto{\pgfqpoint{5.352400in}{1.991472in}}%
\pgfpathlineto{\pgfqpoint{5.356120in}{1.997492in}}%
\pgfpathlineto{\pgfqpoint{5.357360in}{1.998704in}}%
\pgfpathlineto{\pgfqpoint{5.361080in}{1.994272in}}%
\pgfpathlineto{\pgfqpoint{5.363560in}{1.994673in}}%
\pgfpathlineto{\pgfqpoint{5.364800in}{1.992702in}}%
\pgfpathlineto{\pgfqpoint{5.366040in}{1.992979in}}%
\pgfpathlineto{\pgfqpoint{5.367280in}{1.991110in}}%
\pgfpathlineto{\pgfqpoint{5.371000in}{1.990906in}}%
\pgfpathlineto{\pgfqpoint{5.372240in}{1.990855in}}%
\pgfpathlineto{\pgfqpoint{5.379680in}{1.997244in}}%
\pgfpathlineto{\pgfqpoint{5.382160in}{1.997466in}}%
\pgfpathlineto{\pgfqpoint{5.385880in}{2.004068in}}%
\pgfpathlineto{\pgfqpoint{5.388360in}{1.999413in}}%
\pgfpathlineto{\pgfqpoint{5.389600in}{2.002021in}}%
\pgfpathlineto{\pgfqpoint{5.390840in}{2.001817in}}%
\pgfpathlineto{\pgfqpoint{5.392080in}{2.005728in}}%
\pgfpathlineto{\pgfqpoint{5.398280in}{2.004376in}}%
\pgfpathlineto{\pgfqpoint{5.399520in}{2.003320in}}%
\pgfpathlineto{\pgfqpoint{5.403240in}{1.993169in}}%
\pgfpathlineto{\pgfqpoint{5.405720in}{1.994218in}}%
\pgfpathlineto{\pgfqpoint{5.409440in}{1.996863in}}%
\pgfpathlineto{\pgfqpoint{5.410680in}{1.998135in}}%
\pgfpathlineto{\pgfqpoint{5.413160in}{1.995058in}}%
\pgfpathlineto{\pgfqpoint{5.415640in}{1.998449in}}%
\pgfpathlineto{\pgfqpoint{5.419360in}{1.992854in}}%
\pgfpathlineto{\pgfqpoint{5.420600in}{1.993527in}}%
\pgfpathlineto{\pgfqpoint{5.421840in}{1.992154in}}%
\pgfpathlineto{\pgfqpoint{5.425560in}{1.996211in}}%
\pgfpathlineto{\pgfqpoint{5.426800in}{1.995811in}}%
\pgfpathlineto{\pgfqpoint{5.428040in}{1.997726in}}%
\pgfpathlineto{\pgfqpoint{5.430520in}{1.993317in}}%
\pgfpathlineto{\pgfqpoint{5.433000in}{1.995179in}}%
\pgfpathlineto{\pgfqpoint{5.435480in}{1.997586in}}%
\pgfpathlineto{\pgfqpoint{5.439200in}{2.004953in}}%
\pgfpathlineto{\pgfqpoint{5.442920in}{2.000648in}}%
\pgfpathlineto{\pgfqpoint{5.444160in}{2.003175in}}%
\pgfpathlineto{\pgfqpoint{5.447880in}{1.998504in}}%
\pgfpathlineto{\pgfqpoint{5.450360in}{1.995427in}}%
\pgfpathlineto{\pgfqpoint{5.455320in}{1.993738in}}%
\pgfpathlineto{\pgfqpoint{5.457800in}{1.990570in}}%
\pgfpathlineto{\pgfqpoint{5.461520in}{1.993533in}}%
\pgfpathlineto{\pgfqpoint{5.464000in}{1.989383in}}%
\pgfpathlineto{\pgfqpoint{5.466480in}{1.993775in}}%
\pgfpathlineto{\pgfqpoint{5.467720in}{1.994292in}}%
\pgfpathlineto{\pgfqpoint{5.470200in}{1.991231in}}%
\pgfpathlineto{\pgfqpoint{5.471440in}{1.987089in}}%
\pgfpathlineto{\pgfqpoint{5.475160in}{1.994160in}}%
\pgfpathlineto{\pgfqpoint{5.476400in}{1.994404in}}%
\pgfpathlineto{\pgfqpoint{5.477640in}{1.996151in}}%
\pgfpathlineto{\pgfqpoint{5.478880in}{2.000504in}}%
\pgfpathlineto{\pgfqpoint{5.480120in}{2.000378in}}%
\pgfpathlineto{\pgfqpoint{5.481360in}{2.001752in}}%
\pgfpathlineto{\pgfqpoint{5.483840in}{1.997148in}}%
\pgfpathlineto{\pgfqpoint{5.485080in}{1.995672in}}%
\pgfpathlineto{\pgfqpoint{5.487560in}{1.996135in}}%
\pgfpathlineto{\pgfqpoint{5.488800in}{1.994193in}}%
\pgfpathlineto{\pgfqpoint{5.490040in}{1.994461in}}%
\pgfpathlineto{\pgfqpoint{5.491280in}{1.992889in}}%
\pgfpathlineto{\pgfqpoint{5.492520in}{1.994017in}}%
\pgfpathlineto{\pgfqpoint{5.496240in}{1.992517in}}%
\pgfpathlineto{\pgfqpoint{5.498720in}{1.995529in}}%
\pgfpathlineto{\pgfqpoint{5.501200in}{1.995305in}}%
\pgfpathlineto{\pgfqpoint{5.503680in}{1.995056in}}%
\pgfpathlineto{\pgfqpoint{5.506160in}{1.994784in}}%
\pgfpathlineto{\pgfqpoint{5.509880in}{2.000425in}}%
\pgfpathlineto{\pgfqpoint{5.512360in}{1.996534in}}%
\pgfpathlineto{\pgfqpoint{5.513600in}{1.999023in}}%
\pgfpathlineto{\pgfqpoint{5.514840in}{1.998386in}}%
\pgfpathlineto{\pgfqpoint{5.517320in}{2.002675in}}%
\pgfpathlineto{\pgfqpoint{5.518560in}{2.002354in}}%
\pgfpathlineto{\pgfqpoint{5.521040in}{2.000323in}}%
\pgfpathlineto{\pgfqpoint{5.523520in}{1.999610in}}%
\pgfpathlineto{\pgfqpoint{5.527240in}{1.990983in}}%
\pgfpathlineto{\pgfqpoint{5.528480in}{1.990812in}}%
\pgfpathlineto{\pgfqpoint{5.534680in}{1.997591in}}%
\pgfpathlineto{\pgfqpoint{5.537160in}{1.993966in}}%
\pgfpathlineto{\pgfqpoint{5.539640in}{1.996901in}}%
\pgfpathlineto{\pgfqpoint{5.543360in}{1.991121in}}%
\pgfpathlineto{\pgfqpoint{5.544600in}{1.991217in}}%
\pgfpathlineto{\pgfqpoint{5.545840in}{1.989350in}}%
\pgfpathlineto{\pgfqpoint{5.549560in}{1.994170in}}%
\pgfpathlineto{\pgfqpoint{5.550800in}{1.993345in}}%
\pgfpathlineto{\pgfqpoint{5.552040in}{1.994749in}}%
\pgfpathlineto{\pgfqpoint{5.554520in}{1.989688in}}%
\pgfpathlineto{\pgfqpoint{5.557000in}{1.991961in}}%
\pgfpathlineto{\pgfqpoint{5.558240in}{1.992081in}}%
\pgfpathlineto{\pgfqpoint{5.559480in}{1.993434in}}%
\pgfpathlineto{\pgfqpoint{5.561960in}{1.998771in}}%
\pgfpathlineto{\pgfqpoint{5.563200in}{1.999565in}}%
\pgfpathlineto{\pgfqpoint{5.566920in}{1.997143in}}%
\pgfpathlineto{\pgfqpoint{5.568160in}{1.999433in}}%
\pgfpathlineto{\pgfqpoint{5.570640in}{1.995593in}}%
\pgfpathlineto{\pgfqpoint{5.576840in}{1.991634in}}%
\pgfpathlineto{\pgfqpoint{5.578080in}{1.992043in}}%
\pgfpathlineto{\pgfqpoint{5.579320in}{1.991200in}}%
\pgfpathlineto{\pgfqpoint{5.581800in}{1.988696in}}%
\pgfpathlineto{\pgfqpoint{5.584280in}{1.991606in}}%
\pgfpathlineto{\pgfqpoint{5.585520in}{1.990620in}}%
\pgfpathlineto{\pgfqpoint{5.588000in}{1.985405in}}%
\pgfpathlineto{\pgfqpoint{5.589240in}{1.985671in}}%
\pgfpathlineto{\pgfqpoint{5.591720in}{1.988601in}}%
\pgfpathlineto{\pgfqpoint{5.592960in}{1.987569in}}%
\pgfpathlineto{\pgfqpoint{5.595440in}{1.980830in}}%
\pgfpathlineto{\pgfqpoint{5.599160in}{1.987595in}}%
\pgfpathlineto{\pgfqpoint{5.601640in}{1.989895in}}%
\pgfpathlineto{\pgfqpoint{5.602880in}{1.994277in}}%
\pgfpathlineto{\pgfqpoint{5.604120in}{1.994204in}}%
\pgfpathlineto{\pgfqpoint{5.605360in}{1.995997in}}%
\pgfpathlineto{\pgfqpoint{5.607840in}{1.991305in}}%
\pgfpathlineto{\pgfqpoint{5.609080in}{1.990443in}}%
\pgfpathlineto{\pgfqpoint{5.611560in}{1.990705in}}%
\pgfpathlineto{\pgfqpoint{5.615280in}{1.986238in}}%
\pgfpathlineto{\pgfqpoint{5.616520in}{1.986907in}}%
\pgfpathlineto{\pgfqpoint{5.620240in}{1.985143in}}%
\pgfpathlineto{\pgfqpoint{5.622720in}{1.989318in}}%
\pgfpathlineto{\pgfqpoint{5.625200in}{1.990819in}}%
\pgfpathlineto{\pgfqpoint{5.627680in}{1.991175in}}%
\pgfpathlineto{\pgfqpoint{5.630160in}{1.991865in}}%
\pgfpathlineto{\pgfqpoint{5.633880in}{1.995192in}}%
\pgfpathlineto{\pgfqpoint{5.636360in}{1.990838in}}%
\pgfpathlineto{\pgfqpoint{5.637600in}{1.993500in}}%
\pgfpathlineto{\pgfqpoint{5.638840in}{1.992904in}}%
\pgfpathlineto{\pgfqpoint{5.640080in}{1.998058in}}%
\pgfpathlineto{\pgfqpoint{5.641320in}{1.998222in}}%
\pgfpathlineto{\pgfqpoint{5.645040in}{1.994916in}}%
\pgfpathlineto{\pgfqpoint{5.647520in}{1.994296in}}%
\pgfpathlineto{\pgfqpoint{5.651240in}{1.986235in}}%
\pgfpathlineto{\pgfqpoint{5.652480in}{1.985501in}}%
\pgfpathlineto{\pgfqpoint{5.658680in}{1.992367in}}%
\pgfpathlineto{\pgfqpoint{5.661160in}{1.988345in}}%
\pgfpathlineto{\pgfqpoint{5.663640in}{1.992032in}}%
\pgfpathlineto{\pgfqpoint{5.667360in}{1.985701in}}%
\pgfpathlineto{\pgfqpoint{5.668600in}{1.986882in}}%
\pgfpathlineto{\pgfqpoint{5.669840in}{1.985466in}}%
\pgfpathlineto{\pgfqpoint{5.673560in}{1.989620in}}%
\pgfpathlineto{\pgfqpoint{5.674800in}{1.987639in}}%
\pgfpathlineto{\pgfqpoint{5.676040in}{1.988535in}}%
\pgfpathlineto{\pgfqpoint{5.678520in}{1.982976in}}%
\pgfpathlineto{\pgfqpoint{5.681000in}{1.986958in}}%
\pgfpathlineto{\pgfqpoint{5.683480in}{1.988277in}}%
\pgfpathlineto{\pgfqpoint{5.687200in}{1.993851in}}%
\pgfpathlineto{\pgfqpoint{5.688440in}{1.993614in}}%
\pgfpathlineto{\pgfqpoint{5.690920in}{1.991709in}}%
\pgfpathlineto{\pgfqpoint{5.692160in}{1.993141in}}%
\pgfpathlineto{\pgfqpoint{5.694640in}{1.987856in}}%
\pgfpathlineto{\pgfqpoint{5.695880in}{1.987269in}}%
\pgfpathlineto{\pgfqpoint{5.698360in}{1.984080in}}%
\pgfpathlineto{\pgfqpoint{5.703320in}{1.982163in}}%
\pgfpathlineto{\pgfqpoint{5.704560in}{1.979681in}}%
\pgfpathlineto{\pgfqpoint{5.705800in}{1.980059in}}%
\pgfpathlineto{\pgfqpoint{5.708280in}{1.982826in}}%
\pgfpathlineto{\pgfqpoint{5.709520in}{1.982710in}}%
\pgfpathlineto{\pgfqpoint{5.712000in}{1.978737in}}%
\pgfpathlineto{\pgfqpoint{5.714480in}{1.984985in}}%
\pgfpathlineto{\pgfqpoint{5.715720in}{1.987136in}}%
\pgfpathlineto{\pgfqpoint{5.716960in}{1.986108in}}%
\pgfpathlineto{\pgfqpoint{5.719440in}{1.978496in}}%
\pgfpathlineto{\pgfqpoint{5.723160in}{1.985622in}}%
\pgfpathlineto{\pgfqpoint{5.724400in}{1.985249in}}%
\pgfpathlineto{\pgfqpoint{5.725640in}{1.986603in}}%
\pgfpathlineto{\pgfqpoint{5.726880in}{1.990824in}}%
\pgfpathlineto{\pgfqpoint{5.729360in}{1.989611in}}%
\pgfpathlineto{\pgfqpoint{5.731840in}{1.984836in}}%
\pgfpathlineto{\pgfqpoint{5.734320in}{1.984396in}}%
\pgfpathlineto{\pgfqpoint{5.740520in}{1.981077in}}%
\pgfpathlineto{\pgfqpoint{5.744240in}{1.980223in}}%
\pgfpathlineto{\pgfqpoint{5.747960in}{1.986823in}}%
\pgfpathlineto{\pgfqpoint{5.750440in}{1.988170in}}%
\pgfpathlineto{\pgfqpoint{5.755400in}{1.988286in}}%
\pgfpathlineto{\pgfqpoint{5.757880in}{1.990742in}}%
\pgfpathlineto{\pgfqpoint{5.760360in}{1.986041in}}%
\pgfpathlineto{\pgfqpoint{5.761600in}{1.987083in}}%
\pgfpathlineto{\pgfqpoint{5.762840in}{1.985904in}}%
\pgfpathlineto{\pgfqpoint{5.765320in}{1.992590in}}%
\pgfpathlineto{\pgfqpoint{5.767800in}{1.990230in}}%
\pgfpathlineto{\pgfqpoint{5.771520in}{1.991799in}}%
\pgfpathlineto{\pgfqpoint{5.775240in}{1.982657in}}%
\pgfpathlineto{\pgfqpoint{5.776480in}{1.982428in}}%
\pgfpathlineto{\pgfqpoint{5.777720in}{1.984367in}}%
\pgfpathlineto{\pgfqpoint{5.778960in}{1.983815in}}%
\pgfpathlineto{\pgfqpoint{5.782680in}{1.989419in}}%
\pgfpathlineto{\pgfqpoint{5.785160in}{1.985595in}}%
\pgfpathlineto{\pgfqpoint{5.787640in}{1.989557in}}%
\pgfpathlineto{\pgfqpoint{5.791360in}{1.981113in}}%
\pgfpathlineto{\pgfqpoint{5.792600in}{1.982150in}}%
\pgfpathlineto{\pgfqpoint{5.793840in}{1.980106in}}%
\pgfpathlineto{\pgfqpoint{5.797560in}{1.984225in}}%
\pgfpathlineto{\pgfqpoint{5.802520in}{1.977184in}}%
\pgfpathlineto{\pgfqpoint{5.805000in}{1.980654in}}%
\pgfpathlineto{\pgfqpoint{5.807480in}{1.981804in}}%
\pgfpathlineto{\pgfqpoint{5.811200in}{1.988545in}}%
\pgfpathlineto{\pgfqpoint{5.814920in}{1.985020in}}%
\pgfpathlineto{\pgfqpoint{5.816160in}{1.987150in}}%
\pgfpathlineto{\pgfqpoint{5.821120in}{1.978967in}}%
\pgfpathlineto{\pgfqpoint{5.829800in}{1.975470in}}%
\pgfpathlineto{\pgfqpoint{5.832280in}{1.978990in}}%
\pgfpathlineto{\pgfqpoint{5.833520in}{1.979562in}}%
\pgfpathlineto{\pgfqpoint{5.836000in}{1.974758in}}%
\pgfpathlineto{\pgfqpoint{5.838480in}{1.981736in}}%
\pgfpathlineto{\pgfqpoint{5.839720in}{1.983878in}}%
\pgfpathlineto{\pgfqpoint{5.840960in}{1.982514in}}%
\pgfpathlineto{\pgfqpoint{5.843440in}{1.974643in}}%
\pgfpathlineto{\pgfqpoint{5.847160in}{1.982522in}}%
\pgfpathlineto{\pgfqpoint{5.848400in}{1.981296in}}%
\pgfpathlineto{\pgfqpoint{5.849640in}{1.982093in}}%
\pgfpathlineto{\pgfqpoint{5.850880in}{1.986503in}}%
\pgfpathlineto{\pgfqpoint{5.853360in}{1.984528in}}%
\pgfpathlineto{\pgfqpoint{5.855840in}{1.979809in}}%
\pgfpathlineto{\pgfqpoint{5.858320in}{1.980177in}}%
\pgfpathlineto{\pgfqpoint{5.859560in}{1.979544in}}%
\pgfpathlineto{\pgfqpoint{5.863280in}{1.974384in}}%
\pgfpathlineto{\pgfqpoint{5.864520in}{1.975249in}}%
\pgfpathlineto{\pgfqpoint{5.867000in}{1.973534in}}%
\pgfpathlineto{\pgfqpoint{5.868240in}{1.974851in}}%
\pgfpathlineto{\pgfqpoint{5.871960in}{1.981921in}}%
\pgfpathlineto{\pgfqpoint{5.874440in}{1.983552in}}%
\pgfpathlineto{\pgfqpoint{5.875680in}{1.982921in}}%
\pgfpathlineto{\pgfqpoint{5.878160in}{1.981003in}}%
\pgfpathlineto{\pgfqpoint{5.881880in}{1.985863in}}%
\pgfpathlineto{\pgfqpoint{5.884360in}{1.981532in}}%
\pgfpathlineto{\pgfqpoint{5.885600in}{1.982549in}}%
\pgfpathlineto{\pgfqpoint{5.886840in}{1.979732in}}%
\pgfpathlineto{\pgfqpoint{5.889320in}{1.984809in}}%
\pgfpathlineto{\pgfqpoint{5.891800in}{1.982760in}}%
\pgfpathlineto{\pgfqpoint{5.894280in}{1.983304in}}%
\pgfpathlineto{\pgfqpoint{5.895520in}{1.982877in}}%
\pgfpathlineto{\pgfqpoint{5.898000in}{1.975679in}}%
\pgfpathlineto{\pgfqpoint{5.900480in}{1.973707in}}%
\pgfpathlineto{\pgfqpoint{5.906680in}{1.980128in}}%
\pgfpathlineto{\pgfqpoint{5.909160in}{1.976293in}}%
\pgfpathlineto{\pgfqpoint{5.911640in}{1.979037in}}%
\pgfpathlineto{\pgfqpoint{5.914120in}{1.970452in}}%
\pgfpathlineto{\pgfqpoint{5.915360in}{1.969604in}}%
\pgfpathlineto{\pgfqpoint{5.916600in}{1.971111in}}%
\pgfpathlineto{\pgfqpoint{5.917840in}{1.969323in}}%
\pgfpathlineto{\pgfqpoint{5.921560in}{1.972965in}}%
\pgfpathlineto{\pgfqpoint{5.926520in}{1.964457in}}%
\pgfpathlineto{\pgfqpoint{5.929000in}{1.967198in}}%
\pgfpathlineto{\pgfqpoint{5.931480in}{1.968259in}}%
\pgfpathlineto{\pgfqpoint{5.935200in}{1.976427in}}%
\pgfpathlineto{\pgfqpoint{5.937680in}{1.974937in}}%
\pgfpathlineto{\pgfqpoint{5.938920in}{1.973602in}}%
\pgfpathlineto{\pgfqpoint{5.940160in}{1.975435in}}%
\pgfpathlineto{\pgfqpoint{5.943880in}{1.970540in}}%
\pgfpathlineto{\pgfqpoint{5.947600in}{1.966232in}}%
\pgfpathlineto{\pgfqpoint{5.948840in}{1.966260in}}%
\pgfpathlineto{\pgfqpoint{5.950080in}{1.968559in}}%
\pgfpathlineto{\pgfqpoint{5.953800in}{1.968263in}}%
\pgfpathlineto{\pgfqpoint{5.956280in}{1.972699in}}%
\pgfpathlineto{\pgfqpoint{5.957520in}{1.973548in}}%
\pgfpathlineto{\pgfqpoint{5.960000in}{1.967356in}}%
\pgfpathlineto{\pgfqpoint{5.963720in}{1.973145in}}%
\pgfpathlineto{\pgfqpoint{5.966200in}{1.967050in}}%
\pgfpathlineto{\pgfqpoint{5.967440in}{1.964659in}}%
\pgfpathlineto{\pgfqpoint{5.969920in}{1.969705in}}%
\pgfpathlineto{\pgfqpoint{5.971160in}{1.972060in}}%
\pgfpathlineto{\pgfqpoint{5.973640in}{1.970584in}}%
\pgfpathlineto{\pgfqpoint{5.974880in}{1.974776in}}%
\pgfpathlineto{\pgfqpoint{5.977360in}{1.973255in}}%
\pgfpathlineto{\pgfqpoint{5.979840in}{1.969987in}}%
\pgfpathlineto{\pgfqpoint{5.982320in}{1.971080in}}%
\pgfpathlineto{\pgfqpoint{5.986040in}{1.965545in}}%
\pgfpathlineto{\pgfqpoint{5.987280in}{1.963893in}}%
\pgfpathlineto{\pgfqpoint{5.988520in}{1.965004in}}%
\pgfpathlineto{\pgfqpoint{5.991000in}{1.961438in}}%
\pgfpathlineto{\pgfqpoint{5.993480in}{1.966087in}}%
\pgfpathlineto{\pgfqpoint{5.995960in}{1.969629in}}%
\pgfpathlineto{\pgfqpoint{5.998440in}{1.974335in}}%
\pgfpathlineto{\pgfqpoint{5.999680in}{1.974052in}}%
\pgfpathlineto{\pgfqpoint{6.002160in}{1.971063in}}%
\pgfpathlineto{\pgfqpoint{6.004640in}{1.973472in}}%
\pgfpathlineto{\pgfqpoint{6.005880in}{1.975656in}}%
\pgfpathlineto{\pgfqpoint{6.008360in}{1.971327in}}%
\pgfpathlineto{\pgfqpoint{6.009600in}{1.972233in}}%
\pgfpathlineto{\pgfqpoint{6.010840in}{1.968652in}}%
\pgfpathlineto{\pgfqpoint{6.013320in}{1.974620in}}%
\pgfpathlineto{\pgfqpoint{6.015800in}{1.972232in}}%
\pgfpathlineto{\pgfqpoint{6.018280in}{1.972591in}}%
\pgfpathlineto{\pgfqpoint{6.019520in}{1.971322in}}%
\pgfpathlineto{\pgfqpoint{6.022000in}{1.964878in}}%
\pgfpathlineto{\pgfqpoint{6.024480in}{1.962850in}}%
\pgfpathlineto{\pgfqpoint{6.026960in}{1.963257in}}%
\pgfpathlineto{\pgfqpoint{6.030680in}{1.969060in}}%
\pgfpathlineto{\pgfqpoint{6.035640in}{1.965950in}}%
\pgfpathlineto{\pgfqpoint{6.038120in}{1.955482in}}%
\pgfpathlineto{\pgfqpoint{6.039360in}{1.954615in}}%
\pgfpathlineto{\pgfqpoint{6.040600in}{1.956613in}}%
\pgfpathlineto{\pgfqpoint{6.041840in}{1.954763in}}%
\pgfpathlineto{\pgfqpoint{6.045560in}{1.958873in}}%
\pgfpathlineto{\pgfqpoint{6.046800in}{1.956915in}}%
\pgfpathlineto{\pgfqpoint{6.048040in}{1.957360in}}%
\pgfpathlineto{\pgfqpoint{6.050520in}{1.951657in}}%
\pgfpathlineto{\pgfqpoint{6.053000in}{1.954729in}}%
\pgfpathlineto{\pgfqpoint{6.055480in}{1.957319in}}%
\pgfpathlineto{\pgfqpoint{6.059200in}{1.967887in}}%
\pgfpathlineto{\pgfqpoint{6.060440in}{1.967914in}}%
\pgfpathlineto{\pgfqpoint{6.062920in}{1.964027in}}%
\pgfpathlineto{\pgfqpoint{6.064160in}{1.966797in}}%
\pgfpathlineto{\pgfqpoint{6.066640in}{1.962479in}}%
\pgfpathlineto{\pgfqpoint{6.067880in}{1.962224in}}%
\pgfpathlineto{\pgfqpoint{6.069120in}{1.958926in}}%
\pgfpathlineto{\pgfqpoint{6.070360in}{1.959433in}}%
\pgfpathlineto{\pgfqpoint{6.072840in}{1.957079in}}%
\pgfpathlineto{\pgfqpoint{6.074080in}{1.958977in}}%
\pgfpathlineto{\pgfqpoint{6.076560in}{1.958297in}}%
\pgfpathlineto{\pgfqpoint{6.081520in}{1.966374in}}%
\pgfpathlineto{\pgfqpoint{6.085240in}{1.958739in}}%
\pgfpathlineto{\pgfqpoint{6.087720in}{1.962892in}}%
\pgfpathlineto{\pgfqpoint{6.091440in}{1.957001in}}%
\pgfpathlineto{\pgfqpoint{6.093920in}{1.960695in}}%
\pgfpathlineto{\pgfqpoint{6.095160in}{1.962281in}}%
\pgfpathlineto{\pgfqpoint{6.097640in}{1.959310in}}%
\pgfpathlineto{\pgfqpoint{6.098880in}{1.963424in}}%
\pgfpathlineto{\pgfqpoint{6.101360in}{1.961017in}}%
\pgfpathlineto{\pgfqpoint{6.103840in}{1.957553in}}%
\pgfpathlineto{\pgfqpoint{6.106320in}{1.957396in}}%
\pgfpathlineto{\pgfqpoint{6.111280in}{1.951970in}}%
\pgfpathlineto{\pgfqpoint{6.112520in}{1.953445in}}%
\pgfpathlineto{\pgfqpoint{6.115000in}{1.951588in}}%
\pgfpathlineto{\pgfqpoint{6.116240in}{1.951855in}}%
\pgfpathlineto{\pgfqpoint{6.122440in}{1.965607in}}%
\pgfpathlineto{\pgfqpoint{6.123680in}{1.965161in}}%
\pgfpathlineto{\pgfqpoint{6.126160in}{1.962291in}}%
\pgfpathlineto{\pgfqpoint{6.127400in}{1.962555in}}%
\pgfpathlineto{\pgfqpoint{6.129880in}{1.967123in}}%
\pgfpathlineto{\pgfqpoint{6.132360in}{1.962108in}}%
\pgfpathlineto{\pgfqpoint{6.133600in}{1.962179in}}%
\pgfpathlineto{\pgfqpoint{6.134840in}{1.958365in}}%
\pgfpathlineto{\pgfqpoint{6.137320in}{1.964430in}}%
\pgfpathlineto{\pgfqpoint{6.139800in}{1.962147in}}%
\pgfpathlineto{\pgfqpoint{6.142280in}{1.963639in}}%
\pgfpathlineto{\pgfqpoint{6.143520in}{1.962659in}}%
\pgfpathlineto{\pgfqpoint{6.147240in}{1.954091in}}%
\pgfpathlineto{\pgfqpoint{6.149720in}{1.953081in}}%
\pgfpathlineto{\pgfqpoint{6.150960in}{1.953667in}}%
\pgfpathlineto{\pgfqpoint{6.153440in}{1.958518in}}%
\pgfpathlineto{\pgfqpoint{6.154680in}{1.960770in}}%
\pgfpathlineto{\pgfqpoint{6.158400in}{1.958312in}}%
\pgfpathlineto{\pgfqpoint{6.159640in}{1.958106in}}%
\pgfpathlineto{\pgfqpoint{6.163360in}{1.943889in}}%
\pgfpathlineto{\pgfqpoint{6.164600in}{1.945821in}}%
\pgfpathlineto{\pgfqpoint{6.165840in}{1.944444in}}%
\pgfpathlineto{\pgfqpoint{6.168320in}{1.946918in}}%
\pgfpathlineto{\pgfqpoint{6.169560in}{1.949338in}}%
\pgfpathlineto{\pgfqpoint{6.172040in}{1.949314in}}%
\pgfpathlineto{\pgfqpoint{6.174520in}{1.941750in}}%
\pgfpathlineto{\pgfqpoint{6.177000in}{1.945243in}}%
\pgfpathlineto{\pgfqpoint{6.179480in}{1.947881in}}%
\pgfpathlineto{\pgfqpoint{6.184440in}{1.957955in}}%
\pgfpathlineto{\pgfqpoint{6.185680in}{1.955029in}}%
\pgfpathlineto{\pgfqpoint{6.186920in}{1.955420in}}%
\pgfpathlineto{\pgfqpoint{6.188160in}{1.957298in}}%
\pgfpathlineto{\pgfqpoint{6.191880in}{1.954433in}}%
\pgfpathlineto{\pgfqpoint{6.193120in}{1.950366in}}%
\pgfpathlineto{\pgfqpoint{6.194360in}{1.951330in}}%
\pgfpathlineto{\pgfqpoint{6.196840in}{1.948358in}}%
\pgfpathlineto{\pgfqpoint{6.199320in}{1.949500in}}%
\pgfpathlineto{\pgfqpoint{6.200560in}{1.948620in}}%
\pgfpathlineto{\pgfqpoint{6.201800in}{1.950791in}}%
\pgfpathlineto{\pgfqpoint{6.204280in}{1.956824in}}%
\pgfpathlineto{\pgfqpoint{6.205520in}{1.959478in}}%
\pgfpathlineto{\pgfqpoint{6.209240in}{1.947925in}}%
\pgfpathlineto{\pgfqpoint{6.211720in}{1.953041in}}%
\pgfpathlineto{\pgfqpoint{6.215440in}{1.947467in}}%
\pgfpathlineto{\pgfqpoint{6.217920in}{1.950676in}}%
\pgfpathlineto{\pgfqpoint{6.219160in}{1.951994in}}%
\pgfpathlineto{\pgfqpoint{6.221640in}{1.948867in}}%
\pgfpathlineto{\pgfqpoint{6.222880in}{1.952675in}}%
\pgfpathlineto{\pgfqpoint{6.225360in}{1.950815in}}%
\pgfpathlineto{\pgfqpoint{6.227840in}{1.948071in}}%
\pgfpathlineto{\pgfqpoint{6.229080in}{1.947996in}}%
\pgfpathlineto{\pgfqpoint{6.230320in}{1.949481in}}%
\pgfpathlineto{\pgfqpoint{6.234040in}{1.943170in}}%
\pgfpathlineto{\pgfqpoint{6.236520in}{1.944813in}}%
\pgfpathlineto{\pgfqpoint{6.237760in}{1.942929in}}%
\pgfpathlineto{\pgfqpoint{6.240240in}{1.943435in}}%
\pgfpathlineto{\pgfqpoint{6.242720in}{1.948754in}}%
\pgfpathlineto{\pgfqpoint{6.243960in}{1.948070in}}%
\pgfpathlineto{\pgfqpoint{6.246440in}{1.953468in}}%
\pgfpathlineto{\pgfqpoint{6.250160in}{1.948748in}}%
\pgfpathlineto{\pgfqpoint{6.252640in}{1.948683in}}%
\pgfpathlineto{\pgfqpoint{6.253880in}{1.950958in}}%
\pgfpathlineto{\pgfqpoint{6.256360in}{1.944371in}}%
\pgfpathlineto{\pgfqpoint{6.257600in}{1.945828in}}%
\pgfpathlineto{\pgfqpoint{6.258840in}{1.942925in}}%
\pgfpathlineto{\pgfqpoint{6.261320in}{1.949504in}}%
\pgfpathlineto{\pgfqpoint{6.262560in}{1.946769in}}%
\pgfpathlineto{\pgfqpoint{6.266280in}{1.947134in}}%
\pgfpathlineto{\pgfqpoint{6.273720in}{1.939564in}}%
\pgfpathlineto{\pgfqpoint{6.274960in}{1.940307in}}%
\pgfpathlineto{\pgfqpoint{6.277440in}{1.946763in}}%
\pgfpathlineto{\pgfqpoint{6.279920in}{1.949308in}}%
\pgfpathlineto{\pgfqpoint{6.281160in}{1.946855in}}%
\pgfpathlineto{\pgfqpoint{6.283640in}{1.949274in}}%
\pgfpathlineto{\pgfqpoint{6.287360in}{1.933771in}}%
\pgfpathlineto{\pgfqpoint{6.288600in}{1.935660in}}%
\pgfpathlineto{\pgfqpoint{6.289840in}{1.934235in}}%
\pgfpathlineto{\pgfqpoint{6.291080in}{1.934914in}}%
\pgfpathlineto{\pgfqpoint{6.296040in}{1.942249in}}%
\pgfpathlineto{\pgfqpoint{6.298520in}{1.933471in}}%
\pgfpathlineto{\pgfqpoint{6.301000in}{1.935847in}}%
\pgfpathlineto{\pgfqpoint{6.302240in}{1.936129in}}%
\pgfpathlineto{\pgfqpoint{6.308440in}{1.945950in}}%
\pgfpathlineto{\pgfqpoint{6.309680in}{1.942763in}}%
\pgfpathlineto{\pgfqpoint{6.312160in}{1.943863in}}%
\pgfpathlineto{\pgfqpoint{6.314640in}{1.942682in}}%
\pgfpathlineto{\pgfqpoint{6.315880in}{1.942402in}}%
\pgfpathlineto{\pgfqpoint{6.317120in}{1.939283in}}%
\pgfpathlineto{\pgfqpoint{6.318360in}{1.941553in}}%
\pgfpathlineto{\pgfqpoint{6.320840in}{1.938137in}}%
\pgfpathlineto{\pgfqpoint{6.322080in}{1.939288in}}%
\pgfpathlineto{\pgfqpoint{6.324560in}{1.936682in}}%
\pgfpathlineto{\pgfqpoint{6.325800in}{1.939219in}}%
\pgfpathlineto{\pgfqpoint{6.328280in}{1.947268in}}%
\pgfpathlineto{\pgfqpoint{6.329520in}{1.949139in}}%
\pgfpathlineto{\pgfqpoint{6.333240in}{1.932117in}}%
\pgfpathlineto{\pgfqpoint{6.335720in}{1.938072in}}%
\pgfpathlineto{\pgfqpoint{6.338200in}{1.934688in}}%
\pgfpathlineto{\pgfqpoint{6.339440in}{1.933235in}}%
\pgfpathlineto{\pgfqpoint{6.343160in}{1.940865in}}%
\pgfpathlineto{\pgfqpoint{6.345640in}{1.936434in}}%
\pgfpathlineto{\pgfqpoint{6.346880in}{1.940528in}}%
\pgfpathlineto{\pgfqpoint{6.349360in}{1.939067in}}%
\pgfpathlineto{\pgfqpoint{6.351840in}{1.933364in}}%
\pgfpathlineto{\pgfqpoint{6.354320in}{1.936972in}}%
\pgfpathlineto{\pgfqpoint{6.356800in}{1.932490in}}%
\pgfpathlineto{\pgfqpoint{6.358040in}{1.932050in}}%
\pgfpathlineto{\pgfqpoint{6.359280in}{1.933153in}}%
\pgfpathlineto{\pgfqpoint{6.360520in}{1.932344in}}%
\pgfpathlineto{\pgfqpoint{6.361760in}{1.929787in}}%
\pgfpathlineto{\pgfqpoint{6.365480in}{1.931287in}}%
\pgfpathlineto{\pgfqpoint{6.366720in}{1.934956in}}%
\pgfpathlineto{\pgfqpoint{6.367960in}{1.935071in}}%
\pgfpathlineto{\pgfqpoint{6.370440in}{1.940104in}}%
\pgfpathlineto{\pgfqpoint{6.375400in}{1.934284in}}%
\pgfpathlineto{\pgfqpoint{6.377880in}{1.938982in}}%
\pgfpathlineto{\pgfqpoint{6.380360in}{1.934027in}}%
\pgfpathlineto{\pgfqpoint{6.381600in}{1.935965in}}%
\pgfpathlineto{\pgfqpoint{6.382840in}{1.933704in}}%
\pgfpathlineto{\pgfqpoint{6.385320in}{1.942265in}}%
\pgfpathlineto{\pgfqpoint{6.386560in}{1.940250in}}%
\pgfpathlineto{\pgfqpoint{6.387800in}{1.940854in}}%
\pgfpathlineto{\pgfqpoint{6.390280in}{1.938075in}}%
\pgfpathlineto{\pgfqpoint{6.395240in}{1.929439in}}%
\pgfpathlineto{\pgfqpoint{6.398960in}{1.931297in}}%
\pgfpathlineto{\pgfqpoint{6.401440in}{1.939853in}}%
\pgfpathlineto{\pgfqpoint{6.403920in}{1.941793in}}%
\pgfpathlineto{\pgfqpoint{6.406400in}{1.939001in}}%
\pgfpathlineto{\pgfqpoint{6.407640in}{1.939542in}}%
\pgfpathlineto{\pgfqpoint{6.411360in}{1.922854in}}%
\pgfpathlineto{\pgfqpoint{6.412600in}{1.924355in}}%
\pgfpathlineto{\pgfqpoint{6.415080in}{1.923876in}}%
\pgfpathlineto{\pgfqpoint{6.416320in}{1.925519in}}%
\pgfpathlineto{\pgfqpoint{6.418800in}{1.931264in}}%
\pgfpathlineto{\pgfqpoint{6.420040in}{1.933575in}}%
\pgfpathlineto{\pgfqpoint{6.422520in}{1.924372in}}%
\pgfpathlineto{\pgfqpoint{6.425000in}{1.927490in}}%
\pgfpathlineto{\pgfqpoint{6.426240in}{1.927158in}}%
\pgfpathlineto{\pgfqpoint{6.431200in}{1.936327in}}%
\pgfpathlineto{\pgfqpoint{6.432440in}{1.936947in}}%
\pgfpathlineto{\pgfqpoint{6.433680in}{1.931239in}}%
\pgfpathlineto{\pgfqpoint{6.436160in}{1.933942in}}%
\pgfpathlineto{\pgfqpoint{6.438640in}{1.931190in}}%
\pgfpathlineto{\pgfqpoint{6.439880in}{1.930798in}}%
\pgfpathlineto{\pgfqpoint{6.441120in}{1.925970in}}%
\pgfpathlineto{\pgfqpoint{6.442360in}{1.928212in}}%
\pgfpathlineto{\pgfqpoint{6.444840in}{1.925446in}}%
\pgfpathlineto{\pgfqpoint{6.447320in}{1.927822in}}%
\pgfpathlineto{\pgfqpoint{6.448560in}{1.926218in}}%
\pgfpathlineto{\pgfqpoint{6.449800in}{1.927627in}}%
\pgfpathlineto{\pgfqpoint{6.452280in}{1.934650in}}%
\pgfpathlineto{\pgfqpoint{6.453520in}{1.936138in}}%
\pgfpathlineto{\pgfqpoint{6.457240in}{1.924560in}}%
\pgfpathlineto{\pgfqpoint{6.459720in}{1.935955in}}%
\pgfpathlineto{\pgfqpoint{6.464680in}{1.930235in}}%
\pgfpathlineto{\pgfqpoint{6.467160in}{1.934596in}}%
\pgfpathlineto{\pgfqpoint{6.469640in}{1.929280in}}%
\pgfpathlineto{\pgfqpoint{6.472120in}{1.934121in}}%
\pgfpathlineto{\pgfqpoint{6.473360in}{1.933610in}}%
\pgfpathlineto{\pgfqpoint{6.475840in}{1.925604in}}%
\pgfpathlineto{\pgfqpoint{6.478320in}{1.929073in}}%
\pgfpathlineto{\pgfqpoint{6.480800in}{1.924173in}}%
\pgfpathlineto{\pgfqpoint{6.482040in}{1.922961in}}%
\pgfpathlineto{\pgfqpoint{6.484520in}{1.925965in}}%
\pgfpathlineto{\pgfqpoint{6.487000in}{1.926817in}}%
\pgfpathlineto{\pgfqpoint{6.489480in}{1.923633in}}%
\pgfpathlineto{\pgfqpoint{6.491960in}{1.929242in}}%
\pgfpathlineto{\pgfqpoint{6.494440in}{1.935467in}}%
\pgfpathlineto{\pgfqpoint{6.496920in}{1.929315in}}%
\pgfpathlineto{\pgfqpoint{6.500640in}{1.926049in}}%
\pgfpathlineto{\pgfqpoint{6.501880in}{1.927686in}}%
\pgfpathlineto{\pgfqpoint{6.504360in}{1.922528in}}%
\pgfpathlineto{\pgfqpoint{6.509320in}{1.930830in}}%
\pgfpathlineto{\pgfqpoint{6.510560in}{1.929283in}}%
\pgfpathlineto{\pgfqpoint{6.511800in}{1.929856in}}%
\pgfpathlineto{\pgfqpoint{6.515520in}{1.925699in}}%
\pgfpathlineto{\pgfqpoint{6.519240in}{1.914958in}}%
\pgfpathlineto{\pgfqpoint{6.520480in}{1.915653in}}%
\pgfpathlineto{\pgfqpoint{6.522960in}{1.913591in}}%
\pgfpathlineto{\pgfqpoint{6.525440in}{1.922103in}}%
\pgfpathlineto{\pgfqpoint{6.527920in}{1.924126in}}%
\pgfpathlineto{\pgfqpoint{6.530400in}{1.919312in}}%
\pgfpathlineto{\pgfqpoint{6.531640in}{1.920045in}}%
\pgfpathlineto{\pgfqpoint{6.534120in}{1.907489in}}%
\pgfpathlineto{\pgfqpoint{6.535360in}{1.907666in}}%
\pgfpathlineto{\pgfqpoint{6.536600in}{1.910830in}}%
\pgfpathlineto{\pgfqpoint{6.540320in}{1.911280in}}%
\pgfpathlineto{\pgfqpoint{6.542800in}{1.914457in}}%
\pgfpathlineto{\pgfqpoint{6.544040in}{1.916939in}}%
\pgfpathlineto{\pgfqpoint{6.546520in}{1.909479in}}%
\pgfpathlineto{\pgfqpoint{6.550240in}{1.913584in}}%
\pgfpathlineto{\pgfqpoint{6.553960in}{1.919885in}}%
\pgfpathlineto{\pgfqpoint{6.556440in}{1.923011in}}%
\pgfpathlineto{\pgfqpoint{6.557680in}{1.917413in}}%
\pgfpathlineto{\pgfqpoint{6.560160in}{1.922539in}}%
\pgfpathlineto{\pgfqpoint{6.561400in}{1.920077in}}%
\pgfpathlineto{\pgfqpoint{6.562640in}{1.921499in}}%
\pgfpathlineto{\pgfqpoint{6.563880in}{1.920345in}}%
\pgfpathlineto{\pgfqpoint{6.565120in}{1.915666in}}%
\pgfpathlineto{\pgfqpoint{6.566360in}{1.917564in}}%
\pgfpathlineto{\pgfqpoint{6.567600in}{1.915541in}}%
\pgfpathlineto{\pgfqpoint{6.568840in}{1.915953in}}%
\pgfpathlineto{\pgfqpoint{6.570080in}{1.919115in}}%
\pgfpathlineto{\pgfqpoint{6.573800in}{1.918747in}}%
\pgfpathlineto{\pgfqpoint{6.576280in}{1.921822in}}%
\pgfpathlineto{\pgfqpoint{6.577520in}{1.923062in}}%
\pgfpathlineto{\pgfqpoint{6.581240in}{1.910566in}}%
\pgfpathlineto{\pgfqpoint{6.583720in}{1.922819in}}%
\pgfpathlineto{\pgfqpoint{6.586200in}{1.915568in}}%
\pgfpathlineto{\pgfqpoint{6.588680in}{1.911972in}}%
\pgfpathlineto{\pgfqpoint{6.591160in}{1.916106in}}%
\pgfpathlineto{\pgfqpoint{6.593640in}{1.909813in}}%
\pgfpathlineto{\pgfqpoint{6.597360in}{1.912386in}}%
\pgfpathlineto{\pgfqpoint{6.599840in}{1.902559in}}%
\pgfpathlineto{\pgfqpoint{6.602320in}{1.906670in}}%
\pgfpathlineto{\pgfqpoint{6.603560in}{1.905114in}}%
\pgfpathlineto{\pgfqpoint{6.611000in}{1.912395in}}%
\pgfpathlineto{\pgfqpoint{6.613480in}{1.908345in}}%
\pgfpathlineto{\pgfqpoint{6.617200in}{1.919210in}}%
\pgfpathlineto{\pgfqpoint{6.618440in}{1.921697in}}%
\pgfpathlineto{\pgfqpoint{6.623400in}{1.915798in}}%
\pgfpathlineto{\pgfqpoint{6.625880in}{1.916810in}}%
\pgfpathlineto{\pgfqpoint{6.628360in}{1.913199in}}%
\pgfpathlineto{\pgfqpoint{6.629600in}{1.915154in}}%
\pgfpathlineto{\pgfqpoint{6.630840in}{1.912880in}}%
\pgfpathlineto{\pgfqpoint{6.633320in}{1.913716in}}%
\pgfpathlineto{\pgfqpoint{6.634560in}{1.910133in}}%
\pgfpathlineto{\pgfqpoint{6.635800in}{1.911324in}}%
\pgfpathlineto{\pgfqpoint{6.640760in}{1.906859in}}%
\pgfpathlineto{\pgfqpoint{6.643240in}{1.899180in}}%
\pgfpathlineto{\pgfqpoint{6.644480in}{1.899649in}}%
\pgfpathlineto{\pgfqpoint{6.645720in}{1.899668in}}%
\pgfpathlineto{\pgfqpoint{6.646960in}{1.900820in}}%
\pgfpathlineto{\pgfqpoint{6.649440in}{1.911257in}}%
\pgfpathlineto{\pgfqpoint{6.650680in}{1.910631in}}%
\pgfpathlineto{\pgfqpoint{6.651920in}{1.911331in}}%
\pgfpathlineto{\pgfqpoint{6.653160in}{1.910002in}}%
\pgfpathlineto{\pgfqpoint{6.654400in}{1.906901in}}%
\pgfpathlineto{\pgfqpoint{6.655640in}{1.908244in}}%
\pgfpathlineto{\pgfqpoint{6.658120in}{1.898091in}}%
\pgfpathlineto{\pgfqpoint{6.660600in}{1.901538in}}%
\pgfpathlineto{\pgfqpoint{6.661840in}{1.897177in}}%
\pgfpathlineto{\pgfqpoint{6.664320in}{1.898190in}}%
\pgfpathlineto{\pgfqpoint{6.668040in}{1.906631in}}%
\pgfpathlineto{\pgfqpoint{6.670520in}{1.897978in}}%
\pgfpathlineto{\pgfqpoint{6.673000in}{1.896736in}}%
\pgfpathlineto{\pgfqpoint{6.680440in}{1.905569in}}%
\pgfpathlineto{\pgfqpoint{6.681680in}{1.900795in}}%
\pgfpathlineto{\pgfqpoint{6.684160in}{1.907136in}}%
\pgfpathlineto{\pgfqpoint{6.687880in}{1.906298in}}%
\pgfpathlineto{\pgfqpoint{6.691600in}{1.897947in}}%
\pgfpathlineto{\pgfqpoint{6.692840in}{1.899284in}}%
\pgfpathlineto{\pgfqpoint{6.694080in}{1.902904in}}%
\pgfpathlineto{\pgfqpoint{6.695320in}{1.902826in}}%
\pgfpathlineto{\pgfqpoint{6.696560in}{1.901064in}}%
\pgfpathlineto{\pgfqpoint{6.697800in}{1.901586in}}%
\pgfpathlineto{\pgfqpoint{6.699040in}{1.905493in}}%
\pgfpathlineto{\pgfqpoint{6.702760in}{1.905682in}}%
\pgfpathlineto{\pgfqpoint{6.704000in}{1.901101in}}%
\pgfpathlineto{\pgfqpoint{6.705240in}{1.890205in}}%
\pgfpathlineto{\pgfqpoint{6.707720in}{1.901468in}}%
\pgfpathlineto{\pgfqpoint{6.711440in}{1.892049in}}%
\pgfpathlineto{\pgfqpoint{6.712680in}{1.891433in}}%
\pgfpathlineto{\pgfqpoint{6.715160in}{1.899441in}}%
\pgfpathlineto{\pgfqpoint{6.720120in}{1.889094in}}%
\pgfpathlineto{\pgfqpoint{6.721360in}{1.890522in}}%
\pgfpathlineto{\pgfqpoint{6.723840in}{1.876757in}}%
\pgfpathlineto{\pgfqpoint{6.727560in}{1.883123in}}%
\pgfpathlineto{\pgfqpoint{6.728800in}{1.885705in}}%
\pgfpathlineto{\pgfqpoint{6.731280in}{1.884288in}}%
\pgfpathlineto{\pgfqpoint{6.732520in}{1.884003in}}%
\pgfpathlineto{\pgfqpoint{6.735000in}{1.886008in}}%
\pgfpathlineto{\pgfqpoint{6.737480in}{1.884093in}}%
\pgfpathlineto{\pgfqpoint{6.741200in}{1.898387in}}%
\pgfpathlineto{\pgfqpoint{6.742440in}{1.899309in}}%
\pgfpathlineto{\pgfqpoint{6.743680in}{1.897947in}}%
\pgfpathlineto{\pgfqpoint{6.746160in}{1.899731in}}%
\pgfpathlineto{\pgfqpoint{6.748640in}{1.894382in}}%
\pgfpathlineto{\pgfqpoint{6.749880in}{1.896434in}}%
\pgfpathlineto{\pgfqpoint{6.752360in}{1.894482in}}%
\pgfpathlineto{\pgfqpoint{6.753600in}{1.894272in}}%
\pgfpathlineto{\pgfqpoint{6.761040in}{1.884369in}}%
\pgfpathlineto{\pgfqpoint{6.763520in}{1.887244in}}%
\pgfpathlineto{\pgfqpoint{6.767240in}{1.877544in}}%
\pgfpathlineto{\pgfqpoint{6.768480in}{1.878485in}}%
\pgfpathlineto{\pgfqpoint{6.769720in}{1.876732in}}%
\pgfpathlineto{\pgfqpoint{6.770960in}{1.877501in}}%
\pgfpathlineto{\pgfqpoint{6.773440in}{1.889996in}}%
\pgfpathlineto{\pgfqpoint{6.774680in}{1.890189in}}%
\pgfpathlineto{\pgfqpoint{6.775920in}{1.894008in}}%
\pgfpathlineto{\pgfqpoint{6.779640in}{1.889511in}}%
\pgfpathlineto{\pgfqpoint{6.782120in}{1.876763in}}%
\pgfpathlineto{\pgfqpoint{6.783360in}{1.877225in}}%
\pgfpathlineto{\pgfqpoint{6.784600in}{1.880088in}}%
\pgfpathlineto{\pgfqpoint{6.785840in}{1.874911in}}%
\pgfpathlineto{\pgfqpoint{6.787080in}{1.875754in}}%
\pgfpathlineto{\pgfqpoint{6.789560in}{1.883189in}}%
\pgfpathlineto{\pgfqpoint{6.792040in}{1.890580in}}%
\pgfpathlineto{\pgfqpoint{6.797000in}{1.870244in}}%
\pgfpathlineto{\pgfqpoint{6.799480in}{1.870827in}}%
\pgfpathlineto{\pgfqpoint{6.800720in}{1.868629in}}%
\pgfpathlineto{\pgfqpoint{6.804440in}{1.877764in}}%
\pgfpathlineto{\pgfqpoint{6.805680in}{1.872623in}}%
\pgfpathlineto{\pgfqpoint{6.809400in}{1.884025in}}%
\pgfpathlineto{\pgfqpoint{6.810640in}{1.881714in}}%
\pgfpathlineto{\pgfqpoint{6.811880in}{1.882021in}}%
\pgfpathlineto{\pgfqpoint{6.813120in}{1.878783in}}%
\pgfpathlineto{\pgfqpoint{6.814360in}{1.879299in}}%
\pgfpathlineto{\pgfqpoint{6.816840in}{1.875461in}}%
\pgfpathlineto{\pgfqpoint{6.818080in}{1.879881in}}%
\pgfpathlineto{\pgfqpoint{6.819320in}{1.879639in}}%
\pgfpathlineto{\pgfqpoint{6.821800in}{1.875634in}}%
\pgfpathlineto{\pgfqpoint{6.826760in}{1.884148in}}%
\pgfpathlineto{\pgfqpoint{6.829240in}{1.876043in}}%
\pgfpathlineto{\pgfqpoint{6.831720in}{1.885579in}}%
\pgfpathlineto{\pgfqpoint{6.834200in}{1.881599in}}%
\pgfpathlineto{\pgfqpoint{6.835440in}{1.881149in}}%
\pgfpathlineto{\pgfqpoint{6.837920in}{1.883612in}}%
\pgfpathlineto{\pgfqpoint{6.839160in}{1.887674in}}%
\pgfpathlineto{\pgfqpoint{6.841640in}{1.882792in}}%
\pgfpathlineto{\pgfqpoint{6.845360in}{1.879748in}}%
\pgfpathlineto{\pgfqpoint{6.847840in}{1.866450in}}%
\pgfpathlineto{\pgfqpoint{6.852800in}{1.885066in}}%
\pgfpathlineto{\pgfqpoint{6.855280in}{1.877693in}}%
\pgfpathlineto{\pgfqpoint{6.857760in}{1.878323in}}%
\pgfpathlineto{\pgfqpoint{6.859000in}{1.877632in}}%
\pgfpathlineto{\pgfqpoint{6.860240in}{1.875361in}}%
\pgfpathlineto{\pgfqpoint{6.861480in}{1.875517in}}%
\pgfpathlineto{\pgfqpoint{6.862720in}{1.878906in}}%
\pgfpathlineto{\pgfqpoint{6.863960in}{1.877745in}}%
\pgfpathlineto{\pgfqpoint{6.866440in}{1.884443in}}%
\pgfpathlineto{\pgfqpoint{6.868920in}{1.880825in}}%
\pgfpathlineto{\pgfqpoint{6.870160in}{1.880660in}}%
\pgfpathlineto{\pgfqpoint{6.871400in}{1.876384in}}%
\pgfpathlineto{\pgfqpoint{6.872640in}{1.877271in}}%
\pgfpathlineto{\pgfqpoint{6.873880in}{1.881141in}}%
\pgfpathlineto{\pgfqpoint{6.875120in}{1.878050in}}%
\pgfpathlineto{\pgfqpoint{6.876360in}{1.878129in}}%
\pgfpathlineto{\pgfqpoint{6.877600in}{1.880066in}}%
\pgfpathlineto{\pgfqpoint{6.882560in}{1.870751in}}%
\pgfpathlineto{\pgfqpoint{6.883800in}{1.874617in}}%
\pgfpathlineto{\pgfqpoint{6.885040in}{1.872449in}}%
\pgfpathlineto{\pgfqpoint{6.887520in}{1.875756in}}%
\pgfpathlineto{\pgfqpoint{6.890000in}{1.870695in}}%
\pgfpathlineto{\pgfqpoint{6.891240in}{1.866660in}}%
\pgfpathlineto{\pgfqpoint{6.893720in}{1.870825in}}%
\pgfpathlineto{\pgfqpoint{6.894960in}{1.869467in}}%
\pgfpathlineto{\pgfqpoint{6.897440in}{1.885254in}}%
\pgfpathlineto{\pgfqpoint{6.899920in}{1.888585in}}%
\pgfpathlineto{\pgfqpoint{6.904880in}{1.868902in}}%
\pgfpathlineto{\pgfqpoint{6.906120in}{1.859167in}}%
\pgfpathlineto{\pgfqpoint{6.907360in}{1.859642in}}%
\pgfpathlineto{\pgfqpoint{6.908600in}{1.863139in}}%
\pgfpathlineto{\pgfqpoint{6.909840in}{1.858652in}}%
\pgfpathlineto{\pgfqpoint{6.912320in}{1.865157in}}%
\pgfpathlineto{\pgfqpoint{6.916040in}{1.888188in}}%
\pgfpathlineto{\pgfqpoint{6.922240in}{1.863912in}}%
\pgfpathlineto{\pgfqpoint{6.923480in}{1.864478in}}%
\pgfpathlineto{\pgfqpoint{6.924720in}{1.863760in}}%
\pgfpathlineto{\pgfqpoint{6.925960in}{1.866148in}}%
\pgfpathlineto{\pgfqpoint{6.928440in}{1.879623in}}%
\pgfpathlineto{\pgfqpoint{6.929680in}{1.872545in}}%
\pgfpathlineto{\pgfqpoint{6.932160in}{1.887291in}}%
\pgfpathlineto{\pgfqpoint{6.937120in}{1.874492in}}%
\pgfpathlineto{\pgfqpoint{6.939600in}{1.876692in}}%
\pgfpathlineto{\pgfqpoint{6.940840in}{1.873524in}}%
\pgfpathlineto{\pgfqpoint{6.942080in}{1.875533in}}%
\pgfpathlineto{\pgfqpoint{6.944560in}{1.871353in}}%
\pgfpathlineto{\pgfqpoint{6.945800in}{1.872434in}}%
\pgfpathlineto{\pgfqpoint{6.950760in}{1.885145in}}%
\pgfpathlineto{\pgfqpoint{6.952000in}{1.884444in}}%
\pgfpathlineto{\pgfqpoint{6.954480in}{1.895076in}}%
\pgfpathlineto{\pgfqpoint{6.955720in}{1.899001in}}%
\pgfpathlineto{\pgfqpoint{6.956960in}{1.891877in}}%
\pgfpathlineto{\pgfqpoint{6.959440in}{1.896190in}}%
\pgfpathlineto{\pgfqpoint{6.961920in}{1.890091in}}%
\pgfpathlineto{\pgfqpoint{6.963160in}{1.892163in}}%
\pgfpathlineto{\pgfqpoint{6.965640in}{1.885699in}}%
\pgfpathlineto{\pgfqpoint{6.966880in}{1.885425in}}%
\pgfpathlineto{\pgfqpoint{6.969360in}{1.893297in}}%
\pgfpathlineto{\pgfqpoint{6.971840in}{1.882669in}}%
\pgfpathlineto{\pgfqpoint{6.976800in}{1.903061in}}%
\pgfpathlineto{\pgfqpoint{6.980520in}{1.889734in}}%
\pgfpathlineto{\pgfqpoint{6.981760in}{1.887802in}}%
\pgfpathlineto{\pgfqpoint{6.983000in}{1.889775in}}%
\pgfpathlineto{\pgfqpoint{6.985480in}{1.880996in}}%
\pgfpathlineto{\pgfqpoint{6.987960in}{1.891533in}}%
\pgfpathlineto{\pgfqpoint{6.990440in}{1.898050in}}%
\pgfpathlineto{\pgfqpoint{6.991680in}{1.894702in}}%
\pgfpathlineto{\pgfqpoint{6.992920in}{1.894648in}}%
\pgfpathlineto{\pgfqpoint{6.994160in}{1.897223in}}%
\pgfpathlineto{\pgfqpoint{6.995400in}{1.892168in}}%
\pgfpathlineto{\pgfqpoint{6.997880in}{1.903988in}}%
\pgfpathlineto{\pgfqpoint{6.999120in}{1.898071in}}%
\pgfpathlineto{\pgfqpoint{7.000360in}{1.898402in}}%
\pgfpathlineto{\pgfqpoint{7.005320in}{1.883536in}}%
\pgfpathlineto{\pgfqpoint{7.006560in}{1.877007in}}%
\pgfpathlineto{\pgfqpoint{7.007800in}{1.885378in}}%
\pgfpathlineto{\pgfqpoint{7.009040in}{1.883253in}}%
\pgfpathlineto{\pgfqpoint{7.010280in}{1.886631in}}%
\pgfpathlineto{\pgfqpoint{7.012760in}{1.879670in}}%
\pgfpathlineto{\pgfqpoint{7.015240in}{1.872386in}}%
\pgfpathlineto{\pgfqpoint{7.017720in}{1.880200in}}%
\pgfpathlineto{\pgfqpoint{7.018960in}{1.880882in}}%
\pgfpathlineto{\pgfqpoint{7.021440in}{1.901611in}}%
\pgfpathlineto{\pgfqpoint{7.022680in}{1.903166in}}%
\pgfpathlineto{\pgfqpoint{7.025160in}{1.894658in}}%
\pgfpathlineto{\pgfqpoint{7.027640in}{1.882372in}}%
\pgfpathlineto{\pgfqpoint{7.028880in}{1.873399in}}%
\pgfpathlineto{\pgfqpoint{7.030120in}{1.874611in}}%
\pgfpathlineto{\pgfqpoint{7.032600in}{1.883593in}}%
\pgfpathlineto{\pgfqpoint{7.033840in}{1.877367in}}%
\pgfpathlineto{\pgfqpoint{7.036320in}{1.881941in}}%
\pgfpathlineto{\pgfqpoint{7.040040in}{1.912208in}}%
\pgfpathlineto{\pgfqpoint{7.046240in}{1.863012in}}%
\pgfpathlineto{\pgfqpoint{7.047480in}{1.862545in}}%
\pgfpathlineto{\pgfqpoint{7.048720in}{1.863847in}}%
\pgfpathlineto{\pgfqpoint{7.052440in}{1.878019in}}%
\pgfpathlineto{\pgfqpoint{7.053680in}{1.872052in}}%
\pgfpathlineto{\pgfqpoint{7.056160in}{1.890682in}}%
\pgfpathlineto{\pgfqpoint{7.061120in}{1.875032in}}%
\pgfpathlineto{\pgfqpoint{7.062360in}{1.877037in}}%
\pgfpathlineto{\pgfqpoint{7.063600in}{1.876369in}}%
\pgfpathlineto{\pgfqpoint{7.064840in}{1.872648in}}%
\pgfpathlineto{\pgfqpoint{7.066080in}{1.877760in}}%
\pgfpathlineto{\pgfqpoint{7.068560in}{1.874481in}}%
\pgfpathlineto{\pgfqpoint{7.069800in}{1.875975in}}%
\pgfpathlineto{\pgfqpoint{7.071040in}{1.880719in}}%
\pgfpathlineto{\pgfqpoint{7.072280in}{1.879597in}}%
\pgfpathlineto{\pgfqpoint{7.074760in}{1.886355in}}%
\pgfpathlineto{\pgfqpoint{7.076000in}{1.881478in}}%
\pgfpathlineto{\pgfqpoint{7.079720in}{1.892893in}}%
\pgfpathlineto{\pgfqpoint{7.080960in}{1.886623in}}%
\pgfpathlineto{\pgfqpoint{7.082200in}{1.891392in}}%
\pgfpathlineto{\pgfqpoint{7.084680in}{1.888533in}}%
\pgfpathlineto{\pgfqpoint{7.085920in}{1.878086in}}%
\pgfpathlineto{\pgfqpoint{7.087160in}{1.881639in}}%
\pgfpathlineto{\pgfqpoint{7.088400in}{1.877099in}}%
\pgfpathlineto{\pgfqpoint{7.089640in}{1.877447in}}%
\pgfpathlineto{\pgfqpoint{7.093360in}{1.896057in}}%
\pgfpathlineto{\pgfqpoint{7.095840in}{1.886330in}}%
\pgfpathlineto{\pgfqpoint{7.099560in}{1.918613in}}%
\pgfpathlineto{\pgfqpoint{7.102040in}{1.936271in}}%
\pgfpathlineto{\pgfqpoint{7.105760in}{1.915832in}}%
\pgfpathlineto{\pgfqpoint{7.107000in}{1.920503in}}%
\pgfpathlineto{\pgfqpoint{7.109480in}{1.917103in}}%
\pgfpathlineto{\pgfqpoint{7.114440in}{1.950078in}}%
\pgfpathlineto{\pgfqpoint{7.116920in}{1.936417in}}%
\pgfpathlineto{\pgfqpoint{7.118160in}{1.944444in}}%
\pgfpathlineto{\pgfqpoint{7.119400in}{1.935195in}}%
\pgfpathlineto{\pgfqpoint{7.121880in}{1.952574in}}%
\pgfpathlineto{\pgfqpoint{7.126840in}{1.936088in}}%
\pgfpathlineto{\pgfqpoint{7.128080in}{1.929821in}}%
\pgfpathlineto{\pgfqpoint{7.129320in}{1.932891in}}%
\pgfpathlineto{\pgfqpoint{7.130560in}{1.920370in}}%
\pgfpathlineto{\pgfqpoint{7.131800in}{1.929210in}}%
\pgfpathlineto{\pgfqpoint{7.134280in}{1.922676in}}%
\pgfpathlineto{\pgfqpoint{7.136760in}{1.908549in}}%
\pgfpathlineto{\pgfqpoint{7.138000in}{1.907919in}}%
\pgfpathlineto{\pgfqpoint{7.141720in}{1.934080in}}%
\pgfpathlineto{\pgfqpoint{7.142960in}{1.929121in}}%
\pgfpathlineto{\pgfqpoint{7.144200in}{1.943592in}}%
\pgfpathlineto{\pgfqpoint{7.145440in}{1.944391in}}%
\pgfpathlineto{\pgfqpoint{7.146680in}{1.942122in}}%
\pgfpathlineto{\pgfqpoint{7.147920in}{1.943727in}}%
\pgfpathlineto{\pgfqpoint{7.149160in}{1.937122in}}%
\pgfpathlineto{\pgfqpoint{7.151640in}{1.909299in}}%
\pgfpathlineto{\pgfqpoint{7.152880in}{1.910283in}}%
\pgfpathlineto{\pgfqpoint{7.155360in}{1.934202in}}%
\pgfpathlineto{\pgfqpoint{7.156600in}{1.936522in}}%
\pgfpathlineto{\pgfqpoint{7.157840in}{1.930895in}}%
\pgfpathlineto{\pgfqpoint{7.164040in}{1.968951in}}%
\pgfpathlineto{\pgfqpoint{7.165280in}{1.957996in}}%
\pgfpathlineto{\pgfqpoint{7.169000in}{1.905771in}}%
\pgfpathlineto{\pgfqpoint{7.171480in}{1.890106in}}%
\pgfpathlineto{\pgfqpoint{7.175200in}{1.905801in}}%
\pgfpathlineto{\pgfqpoint{7.176440in}{1.915294in}}%
\pgfpathlineto{\pgfqpoint{7.177680in}{1.915594in}}%
\pgfpathlineto{\pgfqpoint{7.178920in}{1.924100in}}%
\pgfpathlineto{\pgfqpoint{7.180160in}{1.924452in}}%
\pgfpathlineto{\pgfqpoint{7.185120in}{1.891500in}}%
\pgfpathlineto{\pgfqpoint{7.186360in}{1.893746in}}%
\pgfpathlineto{\pgfqpoint{7.191320in}{1.925939in}}%
\pgfpathlineto{\pgfqpoint{7.193800in}{1.933224in}}%
\pgfpathlineto{\pgfqpoint{7.195040in}{1.927506in}}%
\pgfpathlineto{\pgfqpoint{7.196280in}{1.930810in}}%
\pgfpathlineto{\pgfqpoint{7.200000in}{1.906098in}}%
\pgfpathlineto{\pgfqpoint{7.200000in}{1.906098in}}%
\pgfusepath{stroke}%
\end{pgfscope}%
\begin{pgfscope}%
\pgfpathrectangle{\pgfqpoint{1.000000in}{0.300000in}}{\pgfqpoint{6.200000in}{2.400000in}} %
\pgfusepath{clip}%
\pgfsetrectcap%
\pgfsetroundjoin%
\pgfsetlinewidth{2.007500pt}%
\definecolor{currentstroke}{rgb}{1.000000,0.000000,0.000000}%
\pgfsetstrokecolor{currentstroke}%
\pgfsetdash{}{0pt}%
\pgfpathmoveto{\pgfqpoint{2.241240in}{1.938241in}}%
\pgfpathlineto{\pgfqpoint{6.578760in}{1.938241in}}%
\pgfusepath{stroke}%
\end{pgfscope}%
\begin{pgfscope}%
\pgfpathrectangle{\pgfqpoint{1.000000in}{0.300000in}}{\pgfqpoint{6.200000in}{2.400000in}} %
\pgfusepath{clip}%
\pgfsetbuttcap%
\pgfsetroundjoin%
\pgfsetlinewidth{0.501875pt}%
\definecolor{currentstroke}{rgb}{0.000000,0.000000,0.000000}%
\pgfsetstrokecolor{currentstroke}%
\pgfsetdash{{1.000000pt}{3.000000pt}}{0.000000pt}%
\pgfpathmoveto{\pgfqpoint{1.000000in}{0.300000in}}%
\pgfpathlineto{\pgfqpoint{1.000000in}{2.700000in}}%
\pgfusepath{stroke}%
\end{pgfscope}%
\begin{pgfscope}%
\pgfsetbuttcap%
\pgfsetroundjoin%
\definecolor{currentfill}{rgb}{0.000000,0.000000,0.000000}%
\pgfsetfillcolor{currentfill}%
\pgfsetlinewidth{0.501875pt}%
\definecolor{currentstroke}{rgb}{0.000000,0.000000,0.000000}%
\pgfsetstrokecolor{currentstroke}%
\pgfsetdash{}{0pt}%
\pgfsys@defobject{currentmarker}{\pgfqpoint{0.000000in}{0.000000in}}{\pgfqpoint{0.000000in}{0.055556in}}{%
\pgfpathmoveto{\pgfqpoint{0.000000in}{0.000000in}}%
\pgfpathlineto{\pgfqpoint{0.000000in}{0.055556in}}%
\pgfusepath{stroke,fill}%
}%
\begin{pgfscope}%
\pgfsys@transformshift{1.000000in}{0.300000in}%
\pgfsys@useobject{currentmarker}{}%
\end{pgfscope}%
\end{pgfscope}%
\begin{pgfscope}%
\pgfsetbuttcap%
\pgfsetroundjoin%
\definecolor{currentfill}{rgb}{0.000000,0.000000,0.000000}%
\pgfsetfillcolor{currentfill}%
\pgfsetlinewidth{0.501875pt}%
\definecolor{currentstroke}{rgb}{0.000000,0.000000,0.000000}%
\pgfsetstrokecolor{currentstroke}%
\pgfsetdash{}{0pt}%
\pgfsys@defobject{currentmarker}{\pgfqpoint{0.000000in}{-0.055556in}}{\pgfqpoint{0.000000in}{0.000000in}}{%
\pgfpathmoveto{\pgfqpoint{0.000000in}{0.000000in}}%
\pgfpathlineto{\pgfqpoint{0.000000in}{-0.055556in}}%
\pgfusepath{stroke,fill}%
}%
\begin{pgfscope}%
\pgfsys@transformshift{1.000000in}{2.700000in}%
\pgfsys@useobject{currentmarker}{}%
\end{pgfscope}%
\end{pgfscope}%
\begin{pgfscope}%
\pgftext[left,bottom,x=0.946981in,y=0.118387in,rotate=0.000000]{{\sffamily\fontsize{12.000000}{14.400000}\selectfont 0}}
%
\end{pgfscope}%
\begin{pgfscope}%
\pgfpathrectangle{\pgfqpoint{1.000000in}{0.300000in}}{\pgfqpoint{6.200000in}{2.400000in}} %
\pgfusepath{clip}%
\pgfsetbuttcap%
\pgfsetroundjoin%
\pgfsetlinewidth{0.501875pt}%
\definecolor{currentstroke}{rgb}{0.000000,0.000000,0.000000}%
\pgfsetstrokecolor{currentstroke}%
\pgfsetdash{{1.000000pt}{3.000000pt}}{0.000000pt}%
\pgfpathmoveto{\pgfqpoint{2.240000in}{0.300000in}}%
\pgfpathlineto{\pgfqpoint{2.240000in}{2.700000in}}%
\pgfusepath{stroke}%
\end{pgfscope}%
\begin{pgfscope}%
\pgfsetbuttcap%
\pgfsetroundjoin%
\definecolor{currentfill}{rgb}{0.000000,0.000000,0.000000}%
\pgfsetfillcolor{currentfill}%
\pgfsetlinewidth{0.501875pt}%
\definecolor{currentstroke}{rgb}{0.000000,0.000000,0.000000}%
\pgfsetstrokecolor{currentstroke}%
\pgfsetdash{}{0pt}%
\pgfsys@defobject{currentmarker}{\pgfqpoint{0.000000in}{0.000000in}}{\pgfqpoint{0.000000in}{0.055556in}}{%
\pgfpathmoveto{\pgfqpoint{0.000000in}{0.000000in}}%
\pgfpathlineto{\pgfqpoint{0.000000in}{0.055556in}}%
\pgfusepath{stroke,fill}%
}%
\begin{pgfscope}%
\pgfsys@transformshift{2.240000in}{0.300000in}%
\pgfsys@useobject{currentmarker}{}%
\end{pgfscope}%
\end{pgfscope}%
\begin{pgfscope}%
\pgfsetbuttcap%
\pgfsetroundjoin%
\definecolor{currentfill}{rgb}{0.000000,0.000000,0.000000}%
\pgfsetfillcolor{currentfill}%
\pgfsetlinewidth{0.501875pt}%
\definecolor{currentstroke}{rgb}{0.000000,0.000000,0.000000}%
\pgfsetstrokecolor{currentstroke}%
\pgfsetdash{}{0pt}%
\pgfsys@defobject{currentmarker}{\pgfqpoint{0.000000in}{-0.055556in}}{\pgfqpoint{0.000000in}{0.000000in}}{%
\pgfpathmoveto{\pgfqpoint{0.000000in}{0.000000in}}%
\pgfpathlineto{\pgfqpoint{0.000000in}{-0.055556in}}%
\pgfusepath{stroke,fill}%
}%
\begin{pgfscope}%
\pgfsys@transformshift{2.240000in}{2.700000in}%
\pgfsys@useobject{currentmarker}{}%
\end{pgfscope}%
\end{pgfscope}%
\begin{pgfscope}%
\pgftext[left,bottom,x=2.080942in,y=0.118387in,rotate=0.000000]{{\sffamily\fontsize{12.000000}{14.400000}\selectfont 100}}
%
\end{pgfscope}%
\begin{pgfscope}%
\pgfpathrectangle{\pgfqpoint{1.000000in}{0.300000in}}{\pgfqpoint{6.200000in}{2.400000in}} %
\pgfusepath{clip}%
\pgfsetbuttcap%
\pgfsetroundjoin%
\pgfsetlinewidth{0.501875pt}%
\definecolor{currentstroke}{rgb}{0.000000,0.000000,0.000000}%
\pgfsetstrokecolor{currentstroke}%
\pgfsetdash{{1.000000pt}{3.000000pt}}{0.000000pt}%
\pgfpathmoveto{\pgfqpoint{3.480000in}{0.300000in}}%
\pgfpathlineto{\pgfqpoint{3.480000in}{2.700000in}}%
\pgfusepath{stroke}%
\end{pgfscope}%
\begin{pgfscope}%
\pgfsetbuttcap%
\pgfsetroundjoin%
\definecolor{currentfill}{rgb}{0.000000,0.000000,0.000000}%
\pgfsetfillcolor{currentfill}%
\pgfsetlinewidth{0.501875pt}%
\definecolor{currentstroke}{rgb}{0.000000,0.000000,0.000000}%
\pgfsetstrokecolor{currentstroke}%
\pgfsetdash{}{0pt}%
\pgfsys@defobject{currentmarker}{\pgfqpoint{0.000000in}{0.000000in}}{\pgfqpoint{0.000000in}{0.055556in}}{%
\pgfpathmoveto{\pgfqpoint{0.000000in}{0.000000in}}%
\pgfpathlineto{\pgfqpoint{0.000000in}{0.055556in}}%
\pgfusepath{stroke,fill}%
}%
\begin{pgfscope}%
\pgfsys@transformshift{3.480000in}{0.300000in}%
\pgfsys@useobject{currentmarker}{}%
\end{pgfscope}%
\end{pgfscope}%
\begin{pgfscope}%
\pgfsetbuttcap%
\pgfsetroundjoin%
\definecolor{currentfill}{rgb}{0.000000,0.000000,0.000000}%
\pgfsetfillcolor{currentfill}%
\pgfsetlinewidth{0.501875pt}%
\definecolor{currentstroke}{rgb}{0.000000,0.000000,0.000000}%
\pgfsetstrokecolor{currentstroke}%
\pgfsetdash{}{0pt}%
\pgfsys@defobject{currentmarker}{\pgfqpoint{0.000000in}{-0.055556in}}{\pgfqpoint{0.000000in}{0.000000in}}{%
\pgfpathmoveto{\pgfqpoint{0.000000in}{0.000000in}}%
\pgfpathlineto{\pgfqpoint{0.000000in}{-0.055556in}}%
\pgfusepath{stroke,fill}%
}%
\begin{pgfscope}%
\pgfsys@transformshift{3.480000in}{2.700000in}%
\pgfsys@useobject{currentmarker}{}%
\end{pgfscope}%
\end{pgfscope}%
\begin{pgfscope}%
\pgftext[left,bottom,x=3.320942in,y=0.118387in,rotate=0.000000]{{\sffamily\fontsize{12.000000}{14.400000}\selectfont 200}}
%
\end{pgfscope}%
\begin{pgfscope}%
\pgfpathrectangle{\pgfqpoint{1.000000in}{0.300000in}}{\pgfqpoint{6.200000in}{2.400000in}} %
\pgfusepath{clip}%
\pgfsetbuttcap%
\pgfsetroundjoin%
\pgfsetlinewidth{0.501875pt}%
\definecolor{currentstroke}{rgb}{0.000000,0.000000,0.000000}%
\pgfsetstrokecolor{currentstroke}%
\pgfsetdash{{1.000000pt}{3.000000pt}}{0.000000pt}%
\pgfpathmoveto{\pgfqpoint{4.720000in}{0.300000in}}%
\pgfpathlineto{\pgfqpoint{4.720000in}{2.700000in}}%
\pgfusepath{stroke}%
\end{pgfscope}%
\begin{pgfscope}%
\pgfsetbuttcap%
\pgfsetroundjoin%
\definecolor{currentfill}{rgb}{0.000000,0.000000,0.000000}%
\pgfsetfillcolor{currentfill}%
\pgfsetlinewidth{0.501875pt}%
\definecolor{currentstroke}{rgb}{0.000000,0.000000,0.000000}%
\pgfsetstrokecolor{currentstroke}%
\pgfsetdash{}{0pt}%
\pgfsys@defobject{currentmarker}{\pgfqpoint{0.000000in}{0.000000in}}{\pgfqpoint{0.000000in}{0.055556in}}{%
\pgfpathmoveto{\pgfqpoint{0.000000in}{0.000000in}}%
\pgfpathlineto{\pgfqpoint{0.000000in}{0.055556in}}%
\pgfusepath{stroke,fill}%
}%
\begin{pgfscope}%
\pgfsys@transformshift{4.720000in}{0.300000in}%
\pgfsys@useobject{currentmarker}{}%
\end{pgfscope}%
\end{pgfscope}%
\begin{pgfscope}%
\pgfsetbuttcap%
\pgfsetroundjoin%
\definecolor{currentfill}{rgb}{0.000000,0.000000,0.000000}%
\pgfsetfillcolor{currentfill}%
\pgfsetlinewidth{0.501875pt}%
\definecolor{currentstroke}{rgb}{0.000000,0.000000,0.000000}%
\pgfsetstrokecolor{currentstroke}%
\pgfsetdash{}{0pt}%
\pgfsys@defobject{currentmarker}{\pgfqpoint{0.000000in}{-0.055556in}}{\pgfqpoint{0.000000in}{0.000000in}}{%
\pgfpathmoveto{\pgfqpoint{0.000000in}{0.000000in}}%
\pgfpathlineto{\pgfqpoint{0.000000in}{-0.055556in}}%
\pgfusepath{stroke,fill}%
}%
\begin{pgfscope}%
\pgfsys@transformshift{4.720000in}{2.700000in}%
\pgfsys@useobject{currentmarker}{}%
\end{pgfscope}%
\end{pgfscope}%
\begin{pgfscope}%
\pgftext[left,bottom,x=4.560942in,y=0.118387in,rotate=0.000000]{{\sffamily\fontsize{12.000000}{14.400000}\selectfont 300}}
%
\end{pgfscope}%
\begin{pgfscope}%
\pgfpathrectangle{\pgfqpoint{1.000000in}{0.300000in}}{\pgfqpoint{6.200000in}{2.400000in}} %
\pgfusepath{clip}%
\pgfsetbuttcap%
\pgfsetroundjoin%
\pgfsetlinewidth{0.501875pt}%
\definecolor{currentstroke}{rgb}{0.000000,0.000000,0.000000}%
\pgfsetstrokecolor{currentstroke}%
\pgfsetdash{{1.000000pt}{3.000000pt}}{0.000000pt}%
\pgfpathmoveto{\pgfqpoint{5.960000in}{0.300000in}}%
\pgfpathlineto{\pgfqpoint{5.960000in}{2.700000in}}%
\pgfusepath{stroke}%
\end{pgfscope}%
\begin{pgfscope}%
\pgfsetbuttcap%
\pgfsetroundjoin%
\definecolor{currentfill}{rgb}{0.000000,0.000000,0.000000}%
\pgfsetfillcolor{currentfill}%
\pgfsetlinewidth{0.501875pt}%
\definecolor{currentstroke}{rgb}{0.000000,0.000000,0.000000}%
\pgfsetstrokecolor{currentstroke}%
\pgfsetdash{}{0pt}%
\pgfsys@defobject{currentmarker}{\pgfqpoint{0.000000in}{0.000000in}}{\pgfqpoint{0.000000in}{0.055556in}}{%
\pgfpathmoveto{\pgfqpoint{0.000000in}{0.000000in}}%
\pgfpathlineto{\pgfqpoint{0.000000in}{0.055556in}}%
\pgfusepath{stroke,fill}%
}%
\begin{pgfscope}%
\pgfsys@transformshift{5.960000in}{0.300000in}%
\pgfsys@useobject{currentmarker}{}%
\end{pgfscope}%
\end{pgfscope}%
\begin{pgfscope}%
\pgfsetbuttcap%
\pgfsetroundjoin%
\definecolor{currentfill}{rgb}{0.000000,0.000000,0.000000}%
\pgfsetfillcolor{currentfill}%
\pgfsetlinewidth{0.501875pt}%
\definecolor{currentstroke}{rgb}{0.000000,0.000000,0.000000}%
\pgfsetstrokecolor{currentstroke}%
\pgfsetdash{}{0pt}%
\pgfsys@defobject{currentmarker}{\pgfqpoint{0.000000in}{-0.055556in}}{\pgfqpoint{0.000000in}{0.000000in}}{%
\pgfpathmoveto{\pgfqpoint{0.000000in}{0.000000in}}%
\pgfpathlineto{\pgfqpoint{0.000000in}{-0.055556in}}%
\pgfusepath{stroke,fill}%
}%
\begin{pgfscope}%
\pgfsys@transformshift{5.960000in}{2.700000in}%
\pgfsys@useobject{currentmarker}{}%
\end{pgfscope}%
\end{pgfscope}%
\begin{pgfscope}%
\pgftext[left,bottom,x=5.800942in,y=0.118387in,rotate=0.000000]{{\sffamily\fontsize{12.000000}{14.400000}\selectfont 400}}
%
\end{pgfscope}%
\begin{pgfscope}%
\pgfpathrectangle{\pgfqpoint{1.000000in}{0.300000in}}{\pgfqpoint{6.200000in}{2.400000in}} %
\pgfusepath{clip}%
\pgfsetbuttcap%
\pgfsetroundjoin%
\pgfsetlinewidth{0.501875pt}%
\definecolor{currentstroke}{rgb}{0.000000,0.000000,0.000000}%
\pgfsetstrokecolor{currentstroke}%
\pgfsetdash{{1.000000pt}{3.000000pt}}{0.000000pt}%
\pgfpathmoveto{\pgfqpoint{7.200000in}{0.300000in}}%
\pgfpathlineto{\pgfqpoint{7.200000in}{2.700000in}}%
\pgfusepath{stroke}%
\end{pgfscope}%
\begin{pgfscope}%
\pgfsetbuttcap%
\pgfsetroundjoin%
\definecolor{currentfill}{rgb}{0.000000,0.000000,0.000000}%
\pgfsetfillcolor{currentfill}%
\pgfsetlinewidth{0.501875pt}%
\definecolor{currentstroke}{rgb}{0.000000,0.000000,0.000000}%
\pgfsetstrokecolor{currentstroke}%
\pgfsetdash{}{0pt}%
\pgfsys@defobject{currentmarker}{\pgfqpoint{0.000000in}{0.000000in}}{\pgfqpoint{0.000000in}{0.055556in}}{%
\pgfpathmoveto{\pgfqpoint{0.000000in}{0.000000in}}%
\pgfpathlineto{\pgfqpoint{0.000000in}{0.055556in}}%
\pgfusepath{stroke,fill}%
}%
\begin{pgfscope}%
\pgfsys@transformshift{7.200000in}{0.300000in}%
\pgfsys@useobject{currentmarker}{}%
\end{pgfscope}%
\end{pgfscope}%
\begin{pgfscope}%
\pgfsetbuttcap%
\pgfsetroundjoin%
\definecolor{currentfill}{rgb}{0.000000,0.000000,0.000000}%
\pgfsetfillcolor{currentfill}%
\pgfsetlinewidth{0.501875pt}%
\definecolor{currentstroke}{rgb}{0.000000,0.000000,0.000000}%
\pgfsetstrokecolor{currentstroke}%
\pgfsetdash{}{0pt}%
\pgfsys@defobject{currentmarker}{\pgfqpoint{0.000000in}{-0.055556in}}{\pgfqpoint{0.000000in}{0.000000in}}{%
\pgfpathmoveto{\pgfqpoint{0.000000in}{0.000000in}}%
\pgfpathlineto{\pgfqpoint{0.000000in}{-0.055556in}}%
\pgfusepath{stroke,fill}%
}%
\begin{pgfscope}%
\pgfsys@transformshift{7.200000in}{2.700000in}%
\pgfsys@useobject{currentmarker}{}%
\end{pgfscope}%
\end{pgfscope}%
\begin{pgfscope}%
\pgftext[left,bottom,x=7.040942in,y=0.118387in,rotate=0.000000]{{\sffamily\fontsize{12.000000}{14.400000}\selectfont 500}}
%
\end{pgfscope}%
\begin{pgfscope}%
\pgftext[left,bottom,x=3.723901in,y=-0.112353in,rotate=0.000000]{{\sffamily\fontsize{12.000000}{14.400000}\selectfont time [ps]}}
%
\end{pgfscope}%
\begin{pgfscope}%
\pgfpathrectangle{\pgfqpoint{1.000000in}{0.300000in}}{\pgfqpoint{6.200000in}{2.400000in}} %
\pgfusepath{clip}%
\pgfsetbuttcap%
\pgfsetroundjoin%
\pgfsetlinewidth{0.501875pt}%
\definecolor{currentstroke}{rgb}{0.000000,0.000000,0.000000}%
\pgfsetstrokecolor{currentstroke}%
\pgfsetdash{{1.000000pt}{3.000000pt}}{0.000000pt}%
\pgfpathmoveto{\pgfqpoint{1.000000in}{0.300000in}}%
\pgfpathlineto{\pgfqpoint{7.200000in}{0.300000in}}%
\pgfusepath{stroke}%
\end{pgfscope}%
\begin{pgfscope}%
\pgfsetbuttcap%
\pgfsetroundjoin%
\definecolor{currentfill}{rgb}{0.000000,0.000000,0.000000}%
\pgfsetfillcolor{currentfill}%
\pgfsetlinewidth{0.501875pt}%
\definecolor{currentstroke}{rgb}{0.000000,0.000000,0.000000}%
\pgfsetstrokecolor{currentstroke}%
\pgfsetdash{}{0pt}%
\pgfsys@defobject{currentmarker}{\pgfqpoint{0.000000in}{0.000000in}}{\pgfqpoint{0.055556in}{0.000000in}}{%
\pgfpathmoveto{\pgfqpoint{0.000000in}{0.000000in}}%
\pgfpathlineto{\pgfqpoint{0.055556in}{0.000000in}}%
\pgfusepath{stroke,fill}%
}%
\begin{pgfscope}%
\pgfsys@transformshift{1.000000in}{0.300000in}%
\pgfsys@useobject{currentmarker}{}%
\end{pgfscope}%
\end{pgfscope}%
\begin{pgfscope}%
\pgfsetbuttcap%
\pgfsetroundjoin%
\definecolor{currentfill}{rgb}{0.000000,0.000000,0.000000}%
\pgfsetfillcolor{currentfill}%
\pgfsetlinewidth{0.501875pt}%
\definecolor{currentstroke}{rgb}{0.000000,0.000000,0.000000}%
\pgfsetstrokecolor{currentstroke}%
\pgfsetdash{}{0pt}%
\pgfsys@defobject{currentmarker}{\pgfqpoint{-0.055556in}{0.000000in}}{\pgfqpoint{0.000000in}{0.000000in}}{%
\pgfpathmoveto{\pgfqpoint{0.000000in}{0.000000in}}%
\pgfpathlineto{\pgfqpoint{-0.055556in}{0.000000in}}%
\pgfusepath{stroke,fill}%
}%
\begin{pgfscope}%
\pgfsys@transformshift{7.200000in}{0.300000in}%
\pgfsys@useobject{currentmarker}{}%
\end{pgfscope}%
\end{pgfscope}%
\begin{pgfscope}%
\pgftext[left,bottom,x=0.679389in,y=0.236971in,rotate=0.000000]{{\sffamily\fontsize{12.000000}{14.400000}\selectfont 2.0}}
%
\end{pgfscope}%
\begin{pgfscope}%
\pgfpathrectangle{\pgfqpoint{1.000000in}{0.300000in}}{\pgfqpoint{6.200000in}{2.400000in}} %
\pgfusepath{clip}%
\pgfsetbuttcap%
\pgfsetroundjoin%
\pgfsetlinewidth{0.501875pt}%
\definecolor{currentstroke}{rgb}{0.000000,0.000000,0.000000}%
\pgfsetstrokecolor{currentstroke}%
\pgfsetdash{{1.000000pt}{3.000000pt}}{0.000000pt}%
\pgfpathmoveto{\pgfqpoint{1.000000in}{0.566667in}}%
\pgfpathlineto{\pgfqpoint{7.200000in}{0.566667in}}%
\pgfusepath{stroke}%
\end{pgfscope}%
\begin{pgfscope}%
\pgfsetbuttcap%
\pgfsetroundjoin%
\definecolor{currentfill}{rgb}{0.000000,0.000000,0.000000}%
\pgfsetfillcolor{currentfill}%
\pgfsetlinewidth{0.501875pt}%
\definecolor{currentstroke}{rgb}{0.000000,0.000000,0.000000}%
\pgfsetstrokecolor{currentstroke}%
\pgfsetdash{}{0pt}%
\pgfsys@defobject{currentmarker}{\pgfqpoint{0.000000in}{0.000000in}}{\pgfqpoint{0.055556in}{0.000000in}}{%
\pgfpathmoveto{\pgfqpoint{0.000000in}{0.000000in}}%
\pgfpathlineto{\pgfqpoint{0.055556in}{0.000000in}}%
\pgfusepath{stroke,fill}%
}%
\begin{pgfscope}%
\pgfsys@transformshift{1.000000in}{0.566667in}%
\pgfsys@useobject{currentmarker}{}%
\end{pgfscope}%
\end{pgfscope}%
\begin{pgfscope}%
\pgfsetbuttcap%
\pgfsetroundjoin%
\definecolor{currentfill}{rgb}{0.000000,0.000000,0.000000}%
\pgfsetfillcolor{currentfill}%
\pgfsetlinewidth{0.501875pt}%
\definecolor{currentstroke}{rgb}{0.000000,0.000000,0.000000}%
\pgfsetstrokecolor{currentstroke}%
\pgfsetdash{}{0pt}%
\pgfsys@defobject{currentmarker}{\pgfqpoint{-0.055556in}{0.000000in}}{\pgfqpoint{0.000000in}{0.000000in}}{%
\pgfpathmoveto{\pgfqpoint{0.000000in}{0.000000in}}%
\pgfpathlineto{\pgfqpoint{-0.055556in}{0.000000in}}%
\pgfusepath{stroke,fill}%
}%
\begin{pgfscope}%
\pgfsys@transformshift{7.200000in}{0.566667in}%
\pgfsys@useobject{currentmarker}{}%
\end{pgfscope}%
\end{pgfscope}%
\begin{pgfscope}%
\pgftext[left,bottom,x=0.679389in,y=0.503638in,rotate=0.000000]{{\sffamily\fontsize{12.000000}{14.400000}\selectfont 2.5}}
%
\end{pgfscope}%
\begin{pgfscope}%
\pgfpathrectangle{\pgfqpoint{1.000000in}{0.300000in}}{\pgfqpoint{6.200000in}{2.400000in}} %
\pgfusepath{clip}%
\pgfsetbuttcap%
\pgfsetroundjoin%
\pgfsetlinewidth{0.501875pt}%
\definecolor{currentstroke}{rgb}{0.000000,0.000000,0.000000}%
\pgfsetstrokecolor{currentstroke}%
\pgfsetdash{{1.000000pt}{3.000000pt}}{0.000000pt}%
\pgfpathmoveto{\pgfqpoint{1.000000in}{0.833333in}}%
\pgfpathlineto{\pgfqpoint{7.200000in}{0.833333in}}%
\pgfusepath{stroke}%
\end{pgfscope}%
\begin{pgfscope}%
\pgfsetbuttcap%
\pgfsetroundjoin%
\definecolor{currentfill}{rgb}{0.000000,0.000000,0.000000}%
\pgfsetfillcolor{currentfill}%
\pgfsetlinewidth{0.501875pt}%
\definecolor{currentstroke}{rgb}{0.000000,0.000000,0.000000}%
\pgfsetstrokecolor{currentstroke}%
\pgfsetdash{}{0pt}%
\pgfsys@defobject{currentmarker}{\pgfqpoint{0.000000in}{0.000000in}}{\pgfqpoint{0.055556in}{0.000000in}}{%
\pgfpathmoveto{\pgfqpoint{0.000000in}{0.000000in}}%
\pgfpathlineto{\pgfqpoint{0.055556in}{0.000000in}}%
\pgfusepath{stroke,fill}%
}%
\begin{pgfscope}%
\pgfsys@transformshift{1.000000in}{0.833333in}%
\pgfsys@useobject{currentmarker}{}%
\end{pgfscope}%
\end{pgfscope}%
\begin{pgfscope}%
\pgfsetbuttcap%
\pgfsetroundjoin%
\definecolor{currentfill}{rgb}{0.000000,0.000000,0.000000}%
\pgfsetfillcolor{currentfill}%
\pgfsetlinewidth{0.501875pt}%
\definecolor{currentstroke}{rgb}{0.000000,0.000000,0.000000}%
\pgfsetstrokecolor{currentstroke}%
\pgfsetdash{}{0pt}%
\pgfsys@defobject{currentmarker}{\pgfqpoint{-0.055556in}{0.000000in}}{\pgfqpoint{0.000000in}{0.000000in}}{%
\pgfpathmoveto{\pgfqpoint{0.000000in}{0.000000in}}%
\pgfpathlineto{\pgfqpoint{-0.055556in}{0.000000in}}%
\pgfusepath{stroke,fill}%
}%
\begin{pgfscope}%
\pgfsys@transformshift{7.200000in}{0.833333in}%
\pgfsys@useobject{currentmarker}{}%
\end{pgfscope}%
\end{pgfscope}%
\begin{pgfscope}%
\pgftext[left,bottom,x=0.679389in,y=0.770305in,rotate=0.000000]{{\sffamily\fontsize{12.000000}{14.400000}\selectfont 3.0}}
%
\end{pgfscope}%
\begin{pgfscope}%
\pgfpathrectangle{\pgfqpoint{1.000000in}{0.300000in}}{\pgfqpoint{6.200000in}{2.400000in}} %
\pgfusepath{clip}%
\pgfsetbuttcap%
\pgfsetroundjoin%
\pgfsetlinewidth{0.501875pt}%
\definecolor{currentstroke}{rgb}{0.000000,0.000000,0.000000}%
\pgfsetstrokecolor{currentstroke}%
\pgfsetdash{{1.000000pt}{3.000000pt}}{0.000000pt}%
\pgfpathmoveto{\pgfqpoint{1.000000in}{1.100000in}}%
\pgfpathlineto{\pgfqpoint{7.200000in}{1.100000in}}%
\pgfusepath{stroke}%
\end{pgfscope}%
\begin{pgfscope}%
\pgfsetbuttcap%
\pgfsetroundjoin%
\definecolor{currentfill}{rgb}{0.000000,0.000000,0.000000}%
\pgfsetfillcolor{currentfill}%
\pgfsetlinewidth{0.501875pt}%
\definecolor{currentstroke}{rgb}{0.000000,0.000000,0.000000}%
\pgfsetstrokecolor{currentstroke}%
\pgfsetdash{}{0pt}%
\pgfsys@defobject{currentmarker}{\pgfqpoint{0.000000in}{0.000000in}}{\pgfqpoint{0.055556in}{0.000000in}}{%
\pgfpathmoveto{\pgfqpoint{0.000000in}{0.000000in}}%
\pgfpathlineto{\pgfqpoint{0.055556in}{0.000000in}}%
\pgfusepath{stroke,fill}%
}%
\begin{pgfscope}%
\pgfsys@transformshift{1.000000in}{1.100000in}%
\pgfsys@useobject{currentmarker}{}%
\end{pgfscope}%
\end{pgfscope}%
\begin{pgfscope}%
\pgfsetbuttcap%
\pgfsetroundjoin%
\definecolor{currentfill}{rgb}{0.000000,0.000000,0.000000}%
\pgfsetfillcolor{currentfill}%
\pgfsetlinewidth{0.501875pt}%
\definecolor{currentstroke}{rgb}{0.000000,0.000000,0.000000}%
\pgfsetstrokecolor{currentstroke}%
\pgfsetdash{}{0pt}%
\pgfsys@defobject{currentmarker}{\pgfqpoint{-0.055556in}{0.000000in}}{\pgfqpoint{0.000000in}{0.000000in}}{%
\pgfpathmoveto{\pgfqpoint{0.000000in}{0.000000in}}%
\pgfpathlineto{\pgfqpoint{-0.055556in}{0.000000in}}%
\pgfusepath{stroke,fill}%
}%
\begin{pgfscope}%
\pgfsys@transformshift{7.200000in}{1.100000in}%
\pgfsys@useobject{currentmarker}{}%
\end{pgfscope}%
\end{pgfscope}%
\begin{pgfscope}%
\pgftext[left,bottom,x=0.679389in,y=1.036971in,rotate=0.000000]{{\sffamily\fontsize{12.000000}{14.400000}\selectfont 3.5}}
%
\end{pgfscope}%
\begin{pgfscope}%
\pgfpathrectangle{\pgfqpoint{1.000000in}{0.300000in}}{\pgfqpoint{6.200000in}{2.400000in}} %
\pgfusepath{clip}%
\pgfsetbuttcap%
\pgfsetroundjoin%
\pgfsetlinewidth{0.501875pt}%
\definecolor{currentstroke}{rgb}{0.000000,0.000000,0.000000}%
\pgfsetstrokecolor{currentstroke}%
\pgfsetdash{{1.000000pt}{3.000000pt}}{0.000000pt}%
\pgfpathmoveto{\pgfqpoint{1.000000in}{1.366667in}}%
\pgfpathlineto{\pgfqpoint{7.200000in}{1.366667in}}%
\pgfusepath{stroke}%
\end{pgfscope}%
\begin{pgfscope}%
\pgfsetbuttcap%
\pgfsetroundjoin%
\definecolor{currentfill}{rgb}{0.000000,0.000000,0.000000}%
\pgfsetfillcolor{currentfill}%
\pgfsetlinewidth{0.501875pt}%
\definecolor{currentstroke}{rgb}{0.000000,0.000000,0.000000}%
\pgfsetstrokecolor{currentstroke}%
\pgfsetdash{}{0pt}%
\pgfsys@defobject{currentmarker}{\pgfqpoint{0.000000in}{0.000000in}}{\pgfqpoint{0.055556in}{0.000000in}}{%
\pgfpathmoveto{\pgfqpoint{0.000000in}{0.000000in}}%
\pgfpathlineto{\pgfqpoint{0.055556in}{0.000000in}}%
\pgfusepath{stroke,fill}%
}%
\begin{pgfscope}%
\pgfsys@transformshift{1.000000in}{1.366667in}%
\pgfsys@useobject{currentmarker}{}%
\end{pgfscope}%
\end{pgfscope}%
\begin{pgfscope}%
\pgfsetbuttcap%
\pgfsetroundjoin%
\definecolor{currentfill}{rgb}{0.000000,0.000000,0.000000}%
\pgfsetfillcolor{currentfill}%
\pgfsetlinewidth{0.501875pt}%
\definecolor{currentstroke}{rgb}{0.000000,0.000000,0.000000}%
\pgfsetstrokecolor{currentstroke}%
\pgfsetdash{}{0pt}%
\pgfsys@defobject{currentmarker}{\pgfqpoint{-0.055556in}{0.000000in}}{\pgfqpoint{0.000000in}{0.000000in}}{%
\pgfpathmoveto{\pgfqpoint{0.000000in}{0.000000in}}%
\pgfpathlineto{\pgfqpoint{-0.055556in}{0.000000in}}%
\pgfusepath{stroke,fill}%
}%
\begin{pgfscope}%
\pgfsys@transformshift{7.200000in}{1.366667in}%
\pgfsys@useobject{currentmarker}{}%
\end{pgfscope}%
\end{pgfscope}%
\begin{pgfscope}%
\pgftext[left,bottom,x=0.679389in,y=1.303638in,rotate=0.000000]{{\sffamily\fontsize{12.000000}{14.400000}\selectfont 4.0}}
%
\end{pgfscope}%
\begin{pgfscope}%
\pgfpathrectangle{\pgfqpoint{1.000000in}{0.300000in}}{\pgfqpoint{6.200000in}{2.400000in}} %
\pgfusepath{clip}%
\pgfsetbuttcap%
\pgfsetroundjoin%
\pgfsetlinewidth{0.501875pt}%
\definecolor{currentstroke}{rgb}{0.000000,0.000000,0.000000}%
\pgfsetstrokecolor{currentstroke}%
\pgfsetdash{{1.000000pt}{3.000000pt}}{0.000000pt}%
\pgfpathmoveto{\pgfqpoint{1.000000in}{1.633333in}}%
\pgfpathlineto{\pgfqpoint{7.200000in}{1.633333in}}%
\pgfusepath{stroke}%
\end{pgfscope}%
\begin{pgfscope}%
\pgfsetbuttcap%
\pgfsetroundjoin%
\definecolor{currentfill}{rgb}{0.000000,0.000000,0.000000}%
\pgfsetfillcolor{currentfill}%
\pgfsetlinewidth{0.501875pt}%
\definecolor{currentstroke}{rgb}{0.000000,0.000000,0.000000}%
\pgfsetstrokecolor{currentstroke}%
\pgfsetdash{}{0pt}%
\pgfsys@defobject{currentmarker}{\pgfqpoint{0.000000in}{0.000000in}}{\pgfqpoint{0.055556in}{0.000000in}}{%
\pgfpathmoveto{\pgfqpoint{0.000000in}{0.000000in}}%
\pgfpathlineto{\pgfqpoint{0.055556in}{0.000000in}}%
\pgfusepath{stroke,fill}%
}%
\begin{pgfscope}%
\pgfsys@transformshift{1.000000in}{1.633333in}%
\pgfsys@useobject{currentmarker}{}%
\end{pgfscope}%
\end{pgfscope}%
\begin{pgfscope}%
\pgfsetbuttcap%
\pgfsetroundjoin%
\definecolor{currentfill}{rgb}{0.000000,0.000000,0.000000}%
\pgfsetfillcolor{currentfill}%
\pgfsetlinewidth{0.501875pt}%
\definecolor{currentstroke}{rgb}{0.000000,0.000000,0.000000}%
\pgfsetstrokecolor{currentstroke}%
\pgfsetdash{}{0pt}%
\pgfsys@defobject{currentmarker}{\pgfqpoint{-0.055556in}{0.000000in}}{\pgfqpoint{0.000000in}{0.000000in}}{%
\pgfpathmoveto{\pgfqpoint{0.000000in}{0.000000in}}%
\pgfpathlineto{\pgfqpoint{-0.055556in}{0.000000in}}%
\pgfusepath{stroke,fill}%
}%
\begin{pgfscope}%
\pgfsys@transformshift{7.200000in}{1.633333in}%
\pgfsys@useobject{currentmarker}{}%
\end{pgfscope}%
\end{pgfscope}%
\begin{pgfscope}%
\pgftext[left,bottom,x=0.679389in,y=1.571403in,rotate=0.000000]{{\sffamily\fontsize{12.000000}{14.400000}\selectfont 4.5}}
%
\end{pgfscope}%
\begin{pgfscope}%
\pgfpathrectangle{\pgfqpoint{1.000000in}{0.300000in}}{\pgfqpoint{6.200000in}{2.400000in}} %
\pgfusepath{clip}%
\pgfsetbuttcap%
\pgfsetroundjoin%
\pgfsetlinewidth{0.501875pt}%
\definecolor{currentstroke}{rgb}{0.000000,0.000000,0.000000}%
\pgfsetstrokecolor{currentstroke}%
\pgfsetdash{{1.000000pt}{3.000000pt}}{0.000000pt}%
\pgfpathmoveto{\pgfqpoint{1.000000in}{1.900000in}}%
\pgfpathlineto{\pgfqpoint{7.200000in}{1.900000in}}%
\pgfusepath{stroke}%
\end{pgfscope}%
\begin{pgfscope}%
\pgfsetbuttcap%
\pgfsetroundjoin%
\definecolor{currentfill}{rgb}{0.000000,0.000000,0.000000}%
\pgfsetfillcolor{currentfill}%
\pgfsetlinewidth{0.501875pt}%
\definecolor{currentstroke}{rgb}{0.000000,0.000000,0.000000}%
\pgfsetstrokecolor{currentstroke}%
\pgfsetdash{}{0pt}%
\pgfsys@defobject{currentmarker}{\pgfqpoint{0.000000in}{0.000000in}}{\pgfqpoint{0.055556in}{0.000000in}}{%
\pgfpathmoveto{\pgfqpoint{0.000000in}{0.000000in}}%
\pgfpathlineto{\pgfqpoint{0.055556in}{0.000000in}}%
\pgfusepath{stroke,fill}%
}%
\begin{pgfscope}%
\pgfsys@transformshift{1.000000in}{1.900000in}%
\pgfsys@useobject{currentmarker}{}%
\end{pgfscope}%
\end{pgfscope}%
\begin{pgfscope}%
\pgfsetbuttcap%
\pgfsetroundjoin%
\definecolor{currentfill}{rgb}{0.000000,0.000000,0.000000}%
\pgfsetfillcolor{currentfill}%
\pgfsetlinewidth{0.501875pt}%
\definecolor{currentstroke}{rgb}{0.000000,0.000000,0.000000}%
\pgfsetstrokecolor{currentstroke}%
\pgfsetdash{}{0pt}%
\pgfsys@defobject{currentmarker}{\pgfqpoint{-0.055556in}{0.000000in}}{\pgfqpoint{0.000000in}{0.000000in}}{%
\pgfpathmoveto{\pgfqpoint{0.000000in}{0.000000in}}%
\pgfpathlineto{\pgfqpoint{-0.055556in}{0.000000in}}%
\pgfusepath{stroke,fill}%
}%
\begin{pgfscope}%
\pgfsys@transformshift{7.200000in}{1.900000in}%
\pgfsys@useobject{currentmarker}{}%
\end{pgfscope}%
\end{pgfscope}%
\begin{pgfscope}%
\pgftext[left,bottom,x=0.679389in,y=1.836971in,rotate=0.000000]{{\sffamily\fontsize{12.000000}{14.400000}\selectfont 5.0}}
%
\end{pgfscope}%
\begin{pgfscope}%
\pgfpathrectangle{\pgfqpoint{1.000000in}{0.300000in}}{\pgfqpoint{6.200000in}{2.400000in}} %
\pgfusepath{clip}%
\pgfsetbuttcap%
\pgfsetroundjoin%
\pgfsetlinewidth{0.501875pt}%
\definecolor{currentstroke}{rgb}{0.000000,0.000000,0.000000}%
\pgfsetstrokecolor{currentstroke}%
\pgfsetdash{{1.000000pt}{3.000000pt}}{0.000000pt}%
\pgfpathmoveto{\pgfqpoint{1.000000in}{2.166667in}}%
\pgfpathlineto{\pgfqpoint{7.200000in}{2.166667in}}%
\pgfusepath{stroke}%
\end{pgfscope}%
\begin{pgfscope}%
\pgfsetbuttcap%
\pgfsetroundjoin%
\definecolor{currentfill}{rgb}{0.000000,0.000000,0.000000}%
\pgfsetfillcolor{currentfill}%
\pgfsetlinewidth{0.501875pt}%
\definecolor{currentstroke}{rgb}{0.000000,0.000000,0.000000}%
\pgfsetstrokecolor{currentstroke}%
\pgfsetdash{}{0pt}%
\pgfsys@defobject{currentmarker}{\pgfqpoint{0.000000in}{0.000000in}}{\pgfqpoint{0.055556in}{0.000000in}}{%
\pgfpathmoveto{\pgfqpoint{0.000000in}{0.000000in}}%
\pgfpathlineto{\pgfqpoint{0.055556in}{0.000000in}}%
\pgfusepath{stroke,fill}%
}%
\begin{pgfscope}%
\pgfsys@transformshift{1.000000in}{2.166667in}%
\pgfsys@useobject{currentmarker}{}%
\end{pgfscope}%
\end{pgfscope}%
\begin{pgfscope}%
\pgfsetbuttcap%
\pgfsetroundjoin%
\definecolor{currentfill}{rgb}{0.000000,0.000000,0.000000}%
\pgfsetfillcolor{currentfill}%
\pgfsetlinewidth{0.501875pt}%
\definecolor{currentstroke}{rgb}{0.000000,0.000000,0.000000}%
\pgfsetstrokecolor{currentstroke}%
\pgfsetdash{}{0pt}%
\pgfsys@defobject{currentmarker}{\pgfqpoint{-0.055556in}{0.000000in}}{\pgfqpoint{0.000000in}{0.000000in}}{%
\pgfpathmoveto{\pgfqpoint{0.000000in}{0.000000in}}%
\pgfpathlineto{\pgfqpoint{-0.055556in}{0.000000in}}%
\pgfusepath{stroke,fill}%
}%
\begin{pgfscope}%
\pgfsys@transformshift{7.200000in}{2.166667in}%
\pgfsys@useobject{currentmarker}{}%
\end{pgfscope}%
\end{pgfscope}%
\begin{pgfscope}%
\pgftext[left,bottom,x=0.679389in,y=2.104736in,rotate=0.000000]{{\sffamily\fontsize{12.000000}{14.400000}\selectfont 5.5}}
%
\end{pgfscope}%
\begin{pgfscope}%
\pgfpathrectangle{\pgfqpoint{1.000000in}{0.300000in}}{\pgfqpoint{6.200000in}{2.400000in}} %
\pgfusepath{clip}%
\pgfsetbuttcap%
\pgfsetroundjoin%
\pgfsetlinewidth{0.501875pt}%
\definecolor{currentstroke}{rgb}{0.000000,0.000000,0.000000}%
\pgfsetstrokecolor{currentstroke}%
\pgfsetdash{{1.000000pt}{3.000000pt}}{0.000000pt}%
\pgfpathmoveto{\pgfqpoint{1.000000in}{2.433333in}}%
\pgfpathlineto{\pgfqpoint{7.200000in}{2.433333in}}%
\pgfusepath{stroke}%
\end{pgfscope}%
\begin{pgfscope}%
\pgfsetbuttcap%
\pgfsetroundjoin%
\definecolor{currentfill}{rgb}{0.000000,0.000000,0.000000}%
\pgfsetfillcolor{currentfill}%
\pgfsetlinewidth{0.501875pt}%
\definecolor{currentstroke}{rgb}{0.000000,0.000000,0.000000}%
\pgfsetstrokecolor{currentstroke}%
\pgfsetdash{}{0pt}%
\pgfsys@defobject{currentmarker}{\pgfqpoint{0.000000in}{0.000000in}}{\pgfqpoint{0.055556in}{0.000000in}}{%
\pgfpathmoveto{\pgfqpoint{0.000000in}{0.000000in}}%
\pgfpathlineto{\pgfqpoint{0.055556in}{0.000000in}}%
\pgfusepath{stroke,fill}%
}%
\begin{pgfscope}%
\pgfsys@transformshift{1.000000in}{2.433333in}%
\pgfsys@useobject{currentmarker}{}%
\end{pgfscope}%
\end{pgfscope}%
\begin{pgfscope}%
\pgfsetbuttcap%
\pgfsetroundjoin%
\definecolor{currentfill}{rgb}{0.000000,0.000000,0.000000}%
\pgfsetfillcolor{currentfill}%
\pgfsetlinewidth{0.501875pt}%
\definecolor{currentstroke}{rgb}{0.000000,0.000000,0.000000}%
\pgfsetstrokecolor{currentstroke}%
\pgfsetdash{}{0pt}%
\pgfsys@defobject{currentmarker}{\pgfqpoint{-0.055556in}{0.000000in}}{\pgfqpoint{0.000000in}{0.000000in}}{%
\pgfpathmoveto{\pgfqpoint{0.000000in}{0.000000in}}%
\pgfpathlineto{\pgfqpoint{-0.055556in}{0.000000in}}%
\pgfusepath{stroke,fill}%
}%
\begin{pgfscope}%
\pgfsys@transformshift{7.200000in}{2.433333in}%
\pgfsys@useobject{currentmarker}{}%
\end{pgfscope}%
\end{pgfscope}%
\begin{pgfscope}%
\pgftext[left,bottom,x=0.679389in,y=2.370305in,rotate=0.000000]{{\sffamily\fontsize{12.000000}{14.400000}\selectfont 6.0}}
%
\end{pgfscope}%
\begin{pgfscope}%
\pgfpathrectangle{\pgfqpoint{1.000000in}{0.300000in}}{\pgfqpoint{6.200000in}{2.400000in}} %
\pgfusepath{clip}%
\pgfsetbuttcap%
\pgfsetroundjoin%
\pgfsetlinewidth{0.501875pt}%
\definecolor{currentstroke}{rgb}{0.000000,0.000000,0.000000}%
\pgfsetstrokecolor{currentstroke}%
\pgfsetdash{{1.000000pt}{3.000000pt}}{0.000000pt}%
\pgfpathmoveto{\pgfqpoint{1.000000in}{2.700000in}}%
\pgfpathlineto{\pgfqpoint{7.200000in}{2.700000in}}%
\pgfusepath{stroke}%
\end{pgfscope}%
\begin{pgfscope}%
\pgfsetbuttcap%
\pgfsetroundjoin%
\definecolor{currentfill}{rgb}{0.000000,0.000000,0.000000}%
\pgfsetfillcolor{currentfill}%
\pgfsetlinewidth{0.501875pt}%
\definecolor{currentstroke}{rgb}{0.000000,0.000000,0.000000}%
\pgfsetstrokecolor{currentstroke}%
\pgfsetdash{}{0pt}%
\pgfsys@defobject{currentmarker}{\pgfqpoint{0.000000in}{0.000000in}}{\pgfqpoint{0.055556in}{0.000000in}}{%
\pgfpathmoveto{\pgfqpoint{0.000000in}{0.000000in}}%
\pgfpathlineto{\pgfqpoint{0.055556in}{0.000000in}}%
\pgfusepath{stroke,fill}%
}%
\begin{pgfscope}%
\pgfsys@transformshift{1.000000in}{2.700000in}%
\pgfsys@useobject{currentmarker}{}%
\end{pgfscope}%
\end{pgfscope}%
\begin{pgfscope}%
\pgfsetbuttcap%
\pgfsetroundjoin%
\definecolor{currentfill}{rgb}{0.000000,0.000000,0.000000}%
\pgfsetfillcolor{currentfill}%
\pgfsetlinewidth{0.501875pt}%
\definecolor{currentstroke}{rgb}{0.000000,0.000000,0.000000}%
\pgfsetstrokecolor{currentstroke}%
\pgfsetdash{}{0pt}%
\pgfsys@defobject{currentmarker}{\pgfqpoint{-0.055556in}{0.000000in}}{\pgfqpoint{0.000000in}{0.000000in}}{%
\pgfpathmoveto{\pgfqpoint{0.000000in}{0.000000in}}%
\pgfpathlineto{\pgfqpoint{-0.055556in}{0.000000in}}%
\pgfusepath{stroke,fill}%
}%
\begin{pgfscope}%
\pgfsys@transformshift{7.200000in}{2.700000in}%
\pgfsys@useobject{currentmarker}{}%
\end{pgfscope}%
\end{pgfscope}%
\begin{pgfscope}%
\pgftext[left,bottom,x=0.679389in,y=2.636971in,rotate=0.000000]{{\sffamily\fontsize{12.000000}{14.400000}\selectfont 6.5}}
%
\end{pgfscope}%
\begin{pgfscope}%
\pgftext[left,bottom,x=0.609945in,y=0.148085in,rotate=90.000000]{{\sffamily\fontsize{12.000000}{14.400000}\selectfont diffusion coefficient [10\(\displaystyle ^{-5}\) cm\(\displaystyle ^2\)/s]}}
%
\end{pgfscope}%
\begin{pgfscope}%
\pgfsetrectcap%
\pgfsetroundjoin%
\pgfsetlinewidth{1.003750pt}%
\definecolor{currentstroke}{rgb}{0.000000,0.000000,0.000000}%
\pgfsetstrokecolor{currentstroke}%
\pgfsetdash{}{0pt}%
\pgfpathmoveto{\pgfqpoint{1.000000in}{2.700000in}}%
\pgfpathlineto{\pgfqpoint{7.200000in}{2.700000in}}%
\pgfusepath{stroke}%
\end{pgfscope}%
\begin{pgfscope}%
\pgfsetrectcap%
\pgfsetroundjoin%
\pgfsetlinewidth{1.003750pt}%
\definecolor{currentstroke}{rgb}{0.000000,0.000000,0.000000}%
\pgfsetstrokecolor{currentstroke}%
\pgfsetdash{}{0pt}%
\pgfpathmoveto{\pgfqpoint{7.200000in}{0.300000in}}%
\pgfpathlineto{\pgfqpoint{7.200000in}{2.700000in}}%
\pgfusepath{stroke}%
\end{pgfscope}%
\begin{pgfscope}%
\pgfsetrectcap%
\pgfsetroundjoin%
\pgfsetlinewidth{1.003750pt}%
\definecolor{currentstroke}{rgb}{0.000000,0.000000,0.000000}%
\pgfsetstrokecolor{currentstroke}%
\pgfsetdash{}{0pt}%
\pgfpathmoveto{\pgfqpoint{1.000000in}{0.300000in}}%
\pgfpathlineto{\pgfqpoint{7.200000in}{0.300000in}}%
\pgfusepath{stroke}%
\end{pgfscope}%
\begin{pgfscope}%
\pgfsetrectcap%
\pgfsetroundjoin%
\pgfsetlinewidth{1.003750pt}%
\definecolor{currentstroke}{rgb}{0.000000,0.000000,0.000000}%
\pgfsetstrokecolor{currentstroke}%
\pgfsetdash{}{0pt}%
\pgfpathmoveto{\pgfqpoint{1.000000in}{0.300000in}}%
\pgfpathlineto{\pgfqpoint{1.000000in}{2.700000in}}%
\pgfusepath{stroke}%
\end{pgfscope}%
\begin{pgfscope}%
\pgfsetrectcap%
\pgfsetroundjoin%
\definecolor{currentfill}{rgb}{1.000000,1.000000,1.000000}%
\pgfsetfillcolor{currentfill}%
\pgfsetlinewidth{1.003750pt}%
\definecolor{currentstroke}{rgb}{0.000000,0.000000,0.000000}%
\pgfsetstrokecolor{currentstroke}%
\pgfsetdash{}{0pt}%
\pgfpathmoveto{\pgfqpoint{1.069417in}{1.977606in}}%
\pgfpathlineto{\pgfqpoint{3.053134in}{1.977606in}}%
\pgfpathlineto{\pgfqpoint{3.053134in}{2.630583in}}%
\pgfpathlineto{\pgfqpoint{1.069417in}{2.630583in}}%
\pgfpathlineto{\pgfqpoint{1.069417in}{1.977606in}}%
\pgfpathclose%
\pgfusepath{stroke,fill}%
\end{pgfscope}%
\begin{pgfscope}%
\pgfsetrectcap%
\pgfsetroundjoin%
\pgfsetlinewidth{1.003750pt}%
\definecolor{currentstroke}{rgb}{0.000000,0.000000,1.000000}%
\pgfsetstrokecolor{currentstroke}%
\pgfsetdash{}{0pt}%
\pgfpathmoveto{\pgfqpoint{1.166600in}{2.518161in}}%
\pgfpathlineto{\pgfqpoint{1.360967in}{2.518161in}}%
\pgfusepath{stroke}%
\end{pgfscope}%
\begin{pgfscope}%
\pgftext[left,bottom,x=1.513683in,y=2.440691in,rotate=0.000000]{{\sffamily\fontsize{9.996000}{11.995200}\selectfont spc, mean: 3.47378}}
%
\end{pgfscope}%
\begin{pgfscope}%
\pgfsetrectcap%
\pgfsetroundjoin%
\pgfsetlinewidth{1.003750pt}%
\definecolor{currentstroke}{rgb}{0.000000,0.500000,0.000000}%
\pgfsetstrokecolor{currentstroke}%
\pgfsetdash{}{0pt}%
\pgfpathmoveto{\pgfqpoint{1.166600in}{2.314385in}}%
\pgfpathlineto{\pgfqpoint{1.360967in}{2.314385in}}%
\pgfusepath{stroke}%
\end{pgfscope}%
\begin{pgfscope}%
\pgftext[left,bottom,x=1.513683in,y=2.236915in,rotate=0.000000]{{\sffamily\fontsize{9.996000}{11.995200}\selectfont spce, mean: 2.48983}}
%
\end{pgfscope}%
\begin{pgfscope}%
\pgfsetrectcap%
\pgfsetroundjoin%
\pgfsetlinewidth{1.003750pt}%
\definecolor{currentstroke}{rgb}{1.000000,0.000000,0.000000}%
\pgfsetstrokecolor{currentstroke}%
\pgfsetdash{}{0pt}%
\pgfpathmoveto{\pgfqpoint{1.166600in}{2.110609in}}%
\pgfpathlineto{\pgfqpoint{1.360967in}{2.110609in}}%
\pgfusepath{stroke}%
\end{pgfscope}%
\begin{pgfscope}%
\pgftext[left,bottom,x=1.513683in,y=2.033139in,rotate=0.000000]{{\sffamily\fontsize{9.996000}{11.995200}\selectfont tip3p, mean: 5.0717}}
%
\end{pgfscope}%
\end{pgfpicture}%
\makeatother%
\endgroup%
}
    \caption{Diffusion coefficient.} \label{fig:diffusion}
\end{figure}

The formula $< \Delta r(t)^2 > = 6 D t$ was used to to calculate the diffusion coefficient $D$. The average values were determined between $\unit[100]{ps}$ and $\unit[400]{ps}$.

\end{document}


% =============== Comments ============
\begin{comment}
\verb{x_init {}}

\begin{figure}[H]
	\resizebox{1\textwidth}{!{\input{../plots/GGA_mesh.pdf}}
	\caption{CAPTION}\label{fig:NAME}
\end{figure}
\end{comment}

\begin{figure}[H]
		\begin{subfigure}[a]{\textwidth}
			\resizebox{\linewidth}{!}{%% Creator: Matplotlib, PGF backend
%%
%% To include the figure in your LaTeX document, write
%%   \input{<filename>.pgf}
%%
%% Make sure the required packages are loaded in your preamble
%%   \usepackage{pgf}
%%
%% Figures using additional raster images can only be included by \input if
%% they are in the same directory as the main LaTeX file. For loading figures
%% from other directories you can use the `import` package
%%   \usepackage{import}
%% and then include the figures with
%%   \import{<path to file>}{<filename>.pgf}
%%
%% Matplotlib used the following preamble
%%   \usepackage{fontspec}
%%   \setmainfont{DejaVu Serif}
%%   \setsansfont{DejaVu Sans}
%%   \setmonofont{DejaVu Sans Mono}
%%
\begingroup%
\makeatletter%
\begin{pgfpicture}%
\pgfpathrectangle{\pgfpointorigin}{\pgfqpoint{8.000000in}{3.000000in}}%
\pgfusepath{use as bounding box}%
\begin{pgfscope}%
\pgfsetrectcap%
\pgfsetroundjoin%
\definecolor{currentfill}{rgb}{1.000000,1.000000,1.000000}%
\pgfsetfillcolor{currentfill}%
\pgfsetlinewidth{0.000000pt}%
\definecolor{currentstroke}{rgb}{1.000000,1.000000,1.000000}%
\pgfsetstrokecolor{currentstroke}%
\pgfsetdash{}{0pt}%
\pgfpathmoveto{\pgfqpoint{0.000000in}{0.000000in}}%
\pgfpathlineto{\pgfqpoint{8.000000in}{0.000000in}}%
\pgfpathlineto{\pgfqpoint{8.000000in}{3.000000in}}%
\pgfpathlineto{\pgfqpoint{0.000000in}{3.000000in}}%
\pgfpathclose%
\pgfusepath{fill}%
\end{pgfscope}%
\begin{pgfscope}%
\pgfsetrectcap%
\pgfsetroundjoin%
\definecolor{currentfill}{rgb}{1.000000,1.000000,1.000000}%
\pgfsetfillcolor{currentfill}%
\pgfsetlinewidth{0.000000pt}%
\definecolor{currentstroke}{rgb}{0.000000,0.000000,0.000000}%
\pgfsetstrokecolor{currentstroke}%
\pgfsetdash{}{0pt}%
\pgfpathmoveto{\pgfqpoint{1.000000in}{0.300000in}}%
\pgfpathlineto{\pgfqpoint{7.200000in}{0.300000in}}%
\pgfpathlineto{\pgfqpoint{7.200000in}{2.700000in}}%
\pgfpathlineto{\pgfqpoint{1.000000in}{2.700000in}}%
\pgfpathclose%
\pgfusepath{fill}%
\end{pgfscope}%
\begin{pgfscope}%
\pgfpathrectangle{\pgfqpoint{1.000000in}{0.300000in}}{\pgfqpoint{6.200000in}{2.400000in}} %
\pgfusepath{clip}%
\pgfsetrectcap%
\pgfsetroundjoin%
\pgfsetlinewidth{1.003750pt}%
\definecolor{currentstroke}{rgb}{0.000000,0.000000,1.000000}%
\pgfsetstrokecolor{currentstroke}%
\pgfsetdash{}{0pt}%
\pgfpathmoveto{\pgfqpoint{1.000000in}{0.490652in}}%
\pgfpathlineto{\pgfqpoint{1.466596in}{0.678595in}}%
\pgfpathlineto{\pgfqpoint{1.739538in}{0.776805in}}%
\pgfpathlineto{\pgfqpoint{1.933193in}{0.837675in}}%
\pgfpathlineto{\pgfqpoint{2.614159in}{1.018632in}}%
\pgfpathlineto{\pgfqpoint{2.822941in}{1.068905in}}%
\pgfpathlineto{\pgfqpoint{2.945672in}{1.103021in}}%
\pgfpathlineto{\pgfqpoint{3.311611in}{1.206745in}}%
\pgfpathlineto{\pgfqpoint{3.448665in}{1.243931in}}%
\pgfpathlineto{\pgfqpoint{3.562479in}{1.278696in}}%
\pgfpathlineto{\pgfqpoint{3.697562in}{1.318435in}}%
\pgfpathlineto{\pgfqpoint{3.810016in}{1.353491in}}%
\pgfpathlineto{\pgfqpoint{3.982633in}{1.403738in}}%
\pgfpathlineto{\pgfqpoint{4.006255in}{1.409618in}}%
\pgfpathlineto{\pgfqpoint{4.036514in}{1.418703in}}%
\pgfpathlineto{\pgfqpoint{4.188203in}{1.463418in}}%
\pgfpathlineto{\pgfqpoint{4.211417in}{1.470487in}}%
\pgfpathlineto{\pgfqpoint{4.244804in}{1.481321in}}%
\pgfpathlineto{\pgfqpoint{4.331290in}{1.507673in}}%
\pgfpathlineto{\pgfqpoint{4.345462in}{1.513228in}}%
\pgfpathlineto{\pgfqpoint{4.395014in}{1.527894in}}%
\pgfpathlineto{\pgfqpoint{4.453224in}{1.547018in}}%
\pgfpathlineto{\pgfqpoint{4.491922in}{1.557454in}}%
\pgfpathlineto{\pgfqpoint{4.950955in}{1.696155in}}%
\pgfpathlineto{\pgfqpoint{4.956636in}{1.697696in}}%
\pgfpathlineto{\pgfqpoint{4.969707in}{1.702707in}}%
\pgfpathlineto{\pgfqpoint{4.998665in}{1.710671in}}%
\pgfpathlineto{\pgfqpoint{5.017874in}{1.716260in}}%
\pgfpathlineto{\pgfqpoint{5.263288in}{1.791778in}}%
\pgfpathlineto{\pgfqpoint{5.270424in}{1.793495in}}%
\pgfpathlineto{\pgfqpoint{5.283313in}{1.797389in}}%
\pgfpathlineto{\pgfqpoint{5.300501in}{1.802135in}}%
\pgfpathlineto{\pgfqpoint{5.309492in}{1.805282in}}%
\pgfpathlineto{\pgfqpoint{5.347472in}{1.816202in}}%
\pgfpathlineto{\pgfqpoint{5.354818in}{1.818604in}}%
\pgfpathlineto{\pgfqpoint{5.380416in}{1.826285in}}%
\pgfpathlineto{\pgfqpoint{5.385421in}{1.827569in}}%
\pgfpathlineto{\pgfqpoint{5.394337in}{1.830562in}}%
\pgfpathlineto{\pgfqpoint{5.406044in}{1.833712in}}%
\pgfpathlineto{\pgfqpoint{5.432595in}{1.842333in}}%
\pgfpathlineto{\pgfqpoint{5.453647in}{1.847937in}}%
\pgfpathlineto{\pgfqpoint{5.461709in}{1.850274in}}%
\pgfpathlineto{\pgfqpoint{5.501470in}{1.862880in}}%
\pgfpathlineto{\pgfqpoint{5.505654in}{1.864186in}}%
\pgfpathlineto{\pgfqpoint{5.547689in}{1.876256in}}%
\pgfpathlineto{\pgfqpoint{5.553930in}{1.878220in}}%
\pgfpathlineto{\pgfqpoint{5.562416in}{1.880542in}}%
\pgfpathlineto{\pgfqpoint{5.574573in}{1.883765in}}%
\pgfpathlineto{\pgfqpoint{5.630190in}{1.900503in}}%
\pgfpathlineto{\pgfqpoint{5.633647in}{1.901690in}}%
\pgfpathlineto{\pgfqpoint{5.639142in}{1.903373in}}%
\pgfpathlineto{\pgfqpoint{5.642554in}{1.904314in}}%
\pgfpathlineto{\pgfqpoint{5.647977in}{1.905600in}}%
\pgfpathlineto{\pgfqpoint{5.651345in}{1.907215in}}%
\pgfpathlineto{\pgfqpoint{5.654696in}{1.908233in}}%
\pgfpathlineto{\pgfqpoint{5.680274in}{1.915580in}}%
\pgfpathlineto{\pgfqpoint{5.694915in}{1.920019in}}%
\pgfpathlineto{\pgfqpoint{5.698057in}{1.920621in}}%
\pgfpathlineto{\pgfqpoint{5.705536in}{1.922882in}}%
\pgfpathlineto{\pgfqpoint{5.714159in}{1.925454in}}%
\pgfpathlineto{\pgfqpoint{5.716602in}{1.926911in}}%
\pgfpathlineto{\pgfqpoint{5.727489in}{1.929980in}}%
\pgfpathlineto{\pgfqpoint{5.735244in}{1.932092in}}%
\pgfpathlineto{\pgfqpoint{5.741149in}{1.933727in}}%
\pgfpathlineto{\pgfqpoint{5.746419in}{1.935760in}}%
\pgfpathlineto{\pgfqpoint{5.749329in}{1.936639in}}%
\pgfpathlineto{\pgfqpoint{5.755112in}{1.938305in}}%
\pgfpathlineto{\pgfqpoint{5.758559in}{1.939157in}}%
\pgfpathlineto{\pgfqpoint{5.769920in}{1.942704in}}%
\pgfpathlineto{\pgfqpoint{5.784409in}{1.947343in}}%
\pgfpathlineto{\pgfqpoint{5.790447in}{1.949074in}}%
\pgfpathlineto{\pgfqpoint{5.796972in}{1.950721in}}%
\pgfpathlineto{\pgfqpoint{5.810366in}{1.954884in}}%
\pgfpathlineto{\pgfqpoint{5.813541in}{1.955664in}}%
\pgfpathlineto{\pgfqpoint{5.822978in}{1.958735in}}%
\pgfpathlineto{\pgfqpoint{5.829712in}{1.961101in}}%
\pgfpathlineto{\pgfqpoint{5.862408in}{1.970415in}}%
\pgfpathlineto{\pgfqpoint{5.866325in}{1.971851in}}%
\pgfpathlineto{\pgfqpoint{5.909800in}{1.984563in}}%
\pgfpathlineto{\pgfqpoint{5.914360in}{1.986203in}}%
\pgfpathlineto{\pgfqpoint{5.929640in}{1.991107in}}%
\pgfpathlineto{\pgfqpoint{5.932743in}{1.991880in}}%
\pgfpathlineto{\pgfqpoint{5.952786in}{1.997284in}}%
\pgfpathlineto{\pgfqpoint{5.957066in}{1.998419in}}%
\pgfpathlineto{\pgfqpoint{5.972250in}{2.003475in}}%
\pgfpathlineto{\pgfqpoint{5.975579in}{2.004147in}}%
\pgfpathlineto{\pgfqpoint{5.994807in}{2.009010in}}%
\pgfpathlineto{\pgfqpoint{5.999229in}{2.010413in}}%
\pgfpathlineto{\pgfqpoint{6.010356in}{2.013824in}}%
\pgfpathlineto{\pgfqpoint{6.025939in}{2.017674in}}%
\pgfpathlineto{\pgfqpoint{6.033217in}{2.019566in}}%
\pgfpathlineto{\pgfqpoint{6.036257in}{2.020037in}}%
\pgfpathlineto{\pgfqpoint{6.040039in}{2.021398in}}%
\pgfpathlineto{\pgfqpoint{6.041922in}{2.022084in}}%
\pgfpathlineto{\pgfqpoint{6.070274in}{2.029160in}}%
\pgfpathlineto{\pgfqpoint{6.072793in}{2.029274in}}%
\pgfpathlineto{\pgfqpoint{6.079228in}{2.031932in}}%
\pgfpathlineto{\pgfqpoint{6.103340in}{2.038358in}}%
\pgfpathlineto{\pgfqpoint{6.105397in}{2.038453in}}%
\pgfpathlineto{\pgfqpoint{6.108129in}{2.038930in}}%
\pgfpathlineto{\pgfqpoint{6.111868in}{2.040383in}}%
\pgfpathlineto{\pgfqpoint{6.113898in}{2.041364in}}%
\pgfpathlineto{\pgfqpoint{6.131245in}{2.045936in}}%
\pgfpathlineto{\pgfqpoint{6.133547in}{2.046757in}}%
\pgfpathlineto{\pgfqpoint{6.137474in}{2.048063in}}%
\pgfpathlineto{\pgfqpoint{6.142351in}{2.048772in}}%
\pgfpathlineto{\pgfqpoint{6.145260in}{2.049656in}}%
\pgfpathlineto{\pgfqpoint{6.159936in}{2.054396in}}%
\pgfpathlineto{\pgfqpoint{6.164340in}{2.055031in}}%
\pgfpathlineto{\pgfqpoint{6.177381in}{2.058923in}}%
\pgfpathlineto{\pgfqpoint{6.180143in}{2.060203in}}%
\pgfpathlineto{\pgfqpoint{6.187756in}{2.062394in}}%
\pgfpathlineto{\pgfqpoint{6.190174in}{2.063065in}}%
\pgfpathlineto{\pgfqpoint{6.192884in}{2.063385in}}%
\pgfpathlineto{\pgfqpoint{6.195284in}{2.063814in}}%
\pgfpathlineto{\pgfqpoint{6.202729in}{2.065390in}}%
\pgfpathlineto{\pgfqpoint{6.204504in}{2.066239in}}%
\pgfpathlineto{\pgfqpoint{6.218245in}{2.070677in}}%
\pgfpathlineto{\pgfqpoint{6.241008in}{2.077864in}}%
\pgfpathlineto{\pgfqpoint{6.244079in}{2.078182in}}%
\pgfpathlineto{\pgfqpoint{6.263568in}{2.082853in}}%
\pgfpathlineto{\pgfqpoint{6.266538in}{2.084775in}}%
\pgfpathlineto{\pgfqpoint{6.268152in}{2.085261in}}%
\pgfpathlineto{\pgfqpoint{6.271369in}{2.085694in}}%
\pgfpathlineto{\pgfqpoint{6.276962in}{2.087250in}}%
\pgfpathlineto{\pgfqpoint{6.278816in}{2.087385in}}%
\pgfpathlineto{\pgfqpoint{6.283560in}{2.088839in}}%
\pgfpathlineto{\pgfqpoint{6.286705in}{2.088934in}}%
\pgfpathlineto{\pgfqpoint{6.288532in}{2.089970in}}%
\pgfpathlineto{\pgfqpoint{6.290355in}{2.090515in}}%
\pgfpathlineto{\pgfqpoint{6.292950in}{2.091989in}}%
\pgfpathlineto{\pgfqpoint{6.294502in}{2.092319in}}%
\pgfpathlineto{\pgfqpoint{6.301954in}{2.094096in}}%
\pgfpathlineto{\pgfqpoint{6.302976in}{2.094429in}}%
\pgfpathlineto{\pgfqpoint{6.304505in}{2.094523in}}%
\pgfpathlineto{\pgfqpoint{6.308567in}{2.095866in}}%
\pgfpathlineto{\pgfqpoint{6.334871in}{2.102772in}}%
\pgfpathlineto{\pgfqpoint{6.337058in}{2.102965in}}%
\pgfpathlineto{\pgfqpoint{6.339237in}{2.104060in}}%
\pgfpathlineto{\pgfqpoint{6.342854in}{2.105757in}}%
\pgfpathlineto{\pgfqpoint{6.344536in}{2.106093in}}%
\pgfpathlineto{\pgfqpoint{6.347170in}{2.106529in}}%
\pgfpathlineto{\pgfqpoint{6.392002in}{2.119209in}}%
\pgfpathlineto{\pgfqpoint{6.393565in}{2.119220in}}%
\pgfpathlineto{\pgfqpoint{6.396902in}{2.120088in}}%
\pgfpathlineto{\pgfqpoint{6.399560in}{2.120643in}}%
\pgfpathlineto{\pgfqpoint{6.401327in}{2.121254in}}%
\pgfpathlineto{\pgfqpoint{6.410959in}{2.124137in}}%
\pgfpathlineto{\pgfqpoint{6.414860in}{2.125281in}}%
\pgfpathlineto{\pgfqpoint{6.416371in}{2.125686in}}%
\pgfpathlineto{\pgfqpoint{6.421954in}{2.126974in}}%
\pgfpathlineto{\pgfqpoint{6.426004in}{2.127602in}}%
\pgfpathlineto{\pgfqpoint{6.427491in}{2.127653in}}%
\pgfpathlineto{\pgfqpoint{6.430242in}{2.128678in}}%
\pgfpathlineto{\pgfqpoint{6.440929in}{2.132036in}}%
\pgfpathlineto{\pgfqpoint{6.444040in}{2.133084in}}%
\pgfpathlineto{\pgfqpoint{6.451244in}{2.134426in}}%
\pgfpathlineto{\pgfqpoint{6.453697in}{2.135367in}}%
\pgfpathlineto{\pgfqpoint{6.455733in}{2.136042in}}%
\pgfpathlineto{\pgfqpoint{6.458372in}{2.137109in}}%
\pgfpathlineto{\pgfqpoint{6.460395in}{2.137370in}}%
\pgfpathlineto{\pgfqpoint{6.465826in}{2.139498in}}%
\pgfpathlineto{\pgfqpoint{6.467227in}{2.139225in}}%
\pgfpathlineto{\pgfqpoint{6.468425in}{2.139067in}}%
\pgfpathlineto{\pgfqpoint{6.474782in}{2.141288in}}%
\pgfpathlineto{\pgfqpoint{6.476164in}{2.142283in}}%
\pgfpathlineto{\pgfqpoint{6.478724in}{2.142863in}}%
\pgfpathlineto{\pgfqpoint{6.480098in}{2.143210in}}%
\pgfpathlineto{\pgfqpoint{6.485179in}{2.145018in}}%
\pgfpathlineto{\pgfqpoint{6.486928in}{2.144984in}}%
\pgfpathlineto{\pgfqpoint{6.487705in}{2.144757in}}%
\pgfpathlineto{\pgfqpoint{6.488092in}{2.144965in}}%
\pgfpathlineto{\pgfqpoint{6.493305in}{2.147049in}}%
\pgfpathlineto{\pgfqpoint{6.494458in}{2.146772in}}%
\pgfpathlineto{\pgfqpoint{6.498288in}{2.148178in}}%
\pgfpathlineto{\pgfqpoint{6.500004in}{2.148699in}}%
\pgfpathlineto{\pgfqpoint{6.502664in}{2.149678in}}%
\pgfpathlineto{\pgfqpoint{6.503612in}{2.150240in}}%
\pgfpathlineto{\pgfqpoint{6.505881in}{2.150714in}}%
\pgfpathlineto{\pgfqpoint{6.507201in}{2.150395in}}%
\pgfpathlineto{\pgfqpoint{6.512269in}{2.152104in}}%
\pgfpathlineto{\pgfqpoint{6.513950in}{2.152363in}}%
\pgfpathlineto{\pgfqpoint{6.516184in}{2.153438in}}%
\pgfpathlineto{\pgfqpoint{6.539748in}{2.160542in}}%
\pgfpathlineto{\pgfqpoint{6.541362in}{2.161838in}}%
\pgfpathlineto{\pgfqpoint{6.542793in}{2.161362in}}%
\pgfpathlineto{\pgfqpoint{6.544043in}{2.161378in}}%
\pgfpathlineto{\pgfqpoint{6.549019in}{2.162800in}}%
\pgfpathlineto{\pgfqpoint{6.552551in}{2.164128in}}%
\pgfpathlineto{\pgfqpoint{6.555538in}{2.165406in}}%
\pgfpathlineto{\pgfqpoint{6.557114in}{2.165562in}}%
\pgfpathlineto{\pgfqpoint{6.560256in}{2.167340in}}%
\pgfpathlineto{\pgfqpoint{6.561126in}{2.166801in}}%
\pgfpathlineto{\pgfqpoint{6.561474in}{2.166970in}}%
\pgfpathlineto{\pgfqpoint{6.564595in}{2.167948in}}%
\pgfpathlineto{\pgfqpoint{6.565805in}{2.168142in}}%
\pgfpathlineto{\pgfqpoint{6.570966in}{2.169870in}}%
\pgfpathlineto{\pgfqpoint{6.572848in}{2.170470in}}%
\pgfpathlineto{\pgfqpoint{6.573702in}{2.170172in}}%
\pgfpathlineto{\pgfqpoint{6.574043in}{2.170764in}}%
\pgfpathlineto{\pgfqpoint{6.575066in}{2.171476in}}%
\pgfpathlineto{\pgfqpoint{6.576937in}{2.173129in}}%
\pgfpathlineto{\pgfqpoint{6.577107in}{2.172978in}}%
\pgfpathlineto{\pgfqpoint{6.578634in}{2.172310in}}%
\pgfpathlineto{\pgfqpoint{6.582856in}{2.173352in}}%
\pgfpathlineto{\pgfqpoint{6.583866in}{2.174314in}}%
\pgfpathlineto{\pgfqpoint{6.585713in}{2.175049in}}%
\pgfpathlineto{\pgfqpoint{6.586048in}{2.174278in}}%
\pgfpathlineto{\pgfqpoint{6.586885in}{2.174965in}}%
\pgfpathlineto{\pgfqpoint{6.588557in}{2.175851in}}%
\pgfpathlineto{\pgfqpoint{6.590058in}{2.175563in}}%
\pgfpathlineto{\pgfqpoint{6.596028in}{2.177899in}}%
\pgfpathlineto{\pgfqpoint{6.597183in}{2.178310in}}%
\pgfpathlineto{\pgfqpoint{6.598171in}{2.178648in}}%
\pgfpathlineto{\pgfqpoint{6.598336in}{2.178407in}}%
\pgfpathlineto{\pgfqpoint{6.599651in}{2.178726in}}%
\pgfpathlineto{\pgfqpoint{6.600471in}{2.180028in}}%
\pgfpathlineto{\pgfqpoint{6.602273in}{2.180906in}}%
\pgfpathlineto{\pgfqpoint{6.602437in}{2.180676in}}%
\pgfpathlineto{\pgfqpoint{6.602600in}{2.180206in}}%
\pgfpathlineto{\pgfqpoint{6.603417in}{2.180624in}}%
\pgfpathlineto{\pgfqpoint{6.605048in}{2.181798in}}%
\pgfpathlineto{\pgfqpoint{6.605211in}{2.181616in}}%
\pgfpathlineto{\pgfqpoint{6.607975in}{2.182094in}}%
\pgfpathlineto{\pgfqpoint{6.609595in}{2.183868in}}%
\pgfpathlineto{\pgfqpoint{6.612984in}{2.183317in}}%
\pgfpathlineto{\pgfqpoint{6.614432in}{2.184199in}}%
\pgfpathlineto{\pgfqpoint{6.615716in}{2.184056in}}%
\pgfpathlineto{\pgfqpoint{6.617797in}{2.185624in}}%
\pgfpathlineto{\pgfqpoint{6.618277in}{2.185843in}}%
\pgfpathlineto{\pgfqpoint{6.618596in}{2.185276in}}%
\pgfpathlineto{\pgfqpoint{6.619553in}{2.185081in}}%
\pgfpathlineto{\pgfqpoint{6.619713in}{2.185391in}}%
\pgfpathlineto{\pgfqpoint{6.620509in}{2.186252in}}%
\pgfpathlineto{\pgfqpoint{6.622258in}{2.187613in}}%
\pgfpathlineto{\pgfqpoint{6.622417in}{2.187389in}}%
\pgfpathlineto{\pgfqpoint{6.623052in}{2.186987in}}%
\pgfpathlineto{\pgfqpoint{6.623528in}{2.187721in}}%
\pgfpathlineto{\pgfqpoint{6.625111in}{2.188748in}}%
\pgfpathlineto{\pgfqpoint{6.626059in}{2.189038in}}%
\pgfpathlineto{\pgfqpoint{6.626217in}{2.188846in}}%
\pgfpathlineto{\pgfqpoint{6.627478in}{2.187994in}}%
\pgfpathlineto{\pgfqpoint{6.627636in}{2.188084in}}%
\pgfpathlineto{\pgfqpoint{6.629367in}{2.188924in}}%
\pgfpathlineto{\pgfqpoint{6.630779in}{2.189491in}}%
\pgfpathlineto{\pgfqpoint{6.632189in}{2.189305in}}%
\pgfpathlineto{\pgfqpoint{6.633908in}{2.190838in}}%
\pgfpathlineto{\pgfqpoint{6.634064in}{2.190760in}}%
\pgfpathlineto{\pgfqpoint{6.635311in}{2.189722in}}%
\pgfpathlineto{\pgfqpoint{6.635467in}{2.190004in}}%
\pgfpathlineto{\pgfqpoint{6.641667in}{2.193905in}}%
\pgfpathlineto{\pgfqpoint{6.644131in}{2.192785in}}%
\pgfpathlineto{\pgfqpoint{6.651622in}{2.195961in}}%
\pgfpathlineto{\pgfqpoint{6.653292in}{2.197653in}}%
\pgfpathlineto{\pgfqpoint{6.653444in}{2.197419in}}%
\pgfpathlineto{\pgfqpoint{6.654201in}{2.196954in}}%
\pgfpathlineto{\pgfqpoint{6.654655in}{2.197303in}}%
\pgfpathlineto{\pgfqpoint{6.657072in}{2.198880in}}%
\pgfpathlineto{\pgfqpoint{6.657223in}{2.198727in}}%
\pgfpathlineto{\pgfqpoint{6.659030in}{2.198228in}}%
\pgfpathlineto{\pgfqpoint{6.659481in}{2.198140in}}%
\pgfpathlineto{\pgfqpoint{6.659932in}{2.198542in}}%
\pgfpathlineto{\pgfqpoint{6.661581in}{2.198542in}}%
\pgfpathlineto{\pgfqpoint{6.662479in}{2.197958in}}%
\pgfpathlineto{\pgfqpoint{6.662779in}{2.198948in}}%
\pgfpathlineto{\pgfqpoint{6.663675in}{2.199790in}}%
\pgfpathlineto{\pgfqpoint{6.664123in}{2.199634in}}%
\pgfpathlineto{\pgfqpoint{6.664421in}{2.200219in}}%
\pgfpathlineto{\pgfqpoint{6.664869in}{2.199462in}}%
\pgfpathlineto{\pgfqpoint{6.664869in}{2.199462in}}%
\pgfpathlineto{\pgfqpoint{6.665018in}{2.199014in}}%
\pgfpathlineto{\pgfqpoint{6.665316in}{2.199485in}}%
\pgfpathlineto{\pgfqpoint{6.665316in}{2.199485in}}%
\pgfpathlineto{\pgfqpoint{6.667546in}{2.202355in}}%
\pgfpathlineto{\pgfqpoint{6.668881in}{2.203152in}}%
\pgfpathlineto{\pgfqpoint{6.669029in}{2.202673in}}%
\pgfpathlineto{\pgfqpoint{6.670213in}{2.203476in}}%
\pgfpathlineto{\pgfqpoint{6.671986in}{2.204608in}}%
\pgfpathlineto{\pgfqpoint{6.677275in}{2.202886in}}%
\pgfpathlineto{\pgfqpoint{6.678444in}{2.204860in}}%
\pgfpathlineto{\pgfqpoint{6.678590in}{2.204448in}}%
\pgfpathlineto{\pgfqpoint{6.679320in}{2.204816in}}%
\pgfpathlineto{\pgfqpoint{6.679612in}{2.204240in}}%
\pgfpathlineto{\pgfqpoint{6.679758in}{2.203927in}}%
\pgfpathlineto{\pgfqpoint{6.679904in}{2.204343in}}%
\pgfpathlineto{\pgfqpoint{6.679904in}{2.204343in}}%
\pgfpathlineto{\pgfqpoint{6.685998in}{2.209760in}}%
\pgfpathlineto{\pgfqpoint{6.686431in}{2.209102in}}%
\pgfpathlineto{\pgfqpoint{6.688881in}{2.207784in}}%
\pgfpathlineto{\pgfqpoint{6.689600in}{2.209137in}}%
\pgfpathlineto{\pgfqpoint{6.690461in}{2.208545in}}%
\pgfpathlineto{\pgfqpoint{6.691752in}{2.207458in}}%
\pgfpathlineto{\pgfqpoint{6.692038in}{2.208328in}}%
\pgfpathlineto{\pgfqpoint{6.693611in}{2.210186in}}%
\pgfpathlineto{\pgfqpoint{6.694895in}{2.209148in}}%
\pgfpathlineto{\pgfqpoint{6.695038in}{2.209693in}}%
\pgfpathlineto{\pgfqpoint{6.695465in}{2.210773in}}%
\pgfpathlineto{\pgfqpoint{6.700291in}{2.216004in}}%
\pgfpathlineto{\pgfqpoint{6.700432in}{2.215673in}}%
\pgfpathlineto{\pgfqpoint{6.701845in}{2.213630in}}%
\pgfpathlineto{\pgfqpoint{6.702268in}{2.214524in}}%
\pgfpathlineto{\pgfqpoint{6.702972in}{2.214140in}}%
\pgfpathlineto{\pgfqpoint{6.703113in}{2.213787in}}%
\pgfpathlineto{\pgfqpoint{6.703536in}{2.214169in}}%
\pgfpathlineto{\pgfqpoint{6.703536in}{2.214169in}}%
\pgfpathlineto{\pgfqpoint{6.703817in}{2.215381in}}%
\pgfpathlineto{\pgfqpoint{6.704661in}{2.214629in}}%
\pgfpathlineto{\pgfqpoint{6.706905in}{2.215050in}}%
\pgfpathlineto{\pgfqpoint{6.707045in}{2.215327in}}%
\pgfpathlineto{\pgfqpoint{6.707045in}{2.215327in}}%
\pgfpathlineto{\pgfqpoint{6.707045in}{2.215327in}}%
\pgfpathlineto{\pgfqpoint{6.707185in}{2.214886in}}%
\pgfpathlineto{\pgfqpoint{6.707605in}{2.215443in}}%
\pgfpathlineto{\pgfqpoint{6.707605in}{2.215443in}}%
\pgfpathlineto{\pgfqpoint{6.707745in}{2.215585in}}%
\pgfpathlineto{\pgfqpoint{6.707884in}{2.215429in}}%
\pgfpathlineto{\pgfqpoint{6.707884in}{2.215429in}}%
\pgfpathlineto{\pgfqpoint{6.708304in}{2.213859in}}%
\pgfpathlineto{\pgfqpoint{6.709002in}{2.214235in}}%
\pgfpathlineto{\pgfqpoint{6.711093in}{2.218737in}}%
\pgfpathlineto{\pgfqpoint{6.711927in}{2.218740in}}%
\pgfpathlineto{\pgfqpoint{6.712066in}{2.217961in}}%
\pgfpathlineto{\pgfqpoint{6.712205in}{2.218065in}}%
\pgfpathlineto{\pgfqpoint{6.712205in}{2.218065in}}%
\pgfpathlineto{\pgfqpoint{6.713455in}{2.219377in}}%
\pgfpathlineto{\pgfqpoint{6.714286in}{2.221357in}}%
\pgfpathlineto{\pgfqpoint{6.714840in}{2.219542in}}%
\pgfpathlineto{\pgfqpoint{6.717464in}{2.217770in}}%
\pgfpathlineto{\pgfqpoint{6.717602in}{2.218182in}}%
\pgfpathlineto{\pgfqpoint{6.718842in}{2.219703in}}%
\pgfpathlineto{\pgfqpoint{6.718979in}{2.219371in}}%
\pgfpathlineto{\pgfqpoint{6.719117in}{2.219549in}}%
\pgfpathlineto{\pgfqpoint{6.719117in}{2.219549in}}%
\pgfpathlineto{\pgfqpoint{6.719254in}{2.219912in}}%
\pgfpathlineto{\pgfqpoint{6.719254in}{2.219912in}}%
\pgfpathlineto{\pgfqpoint{6.719254in}{2.219912in}}%
\pgfpathlineto{\pgfqpoint{6.719529in}{2.219265in}}%
\pgfpathlineto{\pgfqpoint{6.719804in}{2.219508in}}%
\pgfpathlineto{\pgfqpoint{6.719804in}{2.219508in}}%
\pgfpathlineto{\pgfqpoint{6.720765in}{2.222614in}}%
\pgfpathlineto{\pgfqpoint{6.721039in}{2.221647in}}%
\pgfpathlineto{\pgfqpoint{6.722272in}{2.218901in}}%
\pgfpathlineto{\pgfqpoint{6.722819in}{2.219120in}}%
\pgfpathlineto{\pgfqpoint{6.725140in}{2.224938in}}%
\pgfpathlineto{\pgfqpoint{6.725277in}{2.223782in}}%
\pgfpathlineto{\pgfqpoint{6.725413in}{2.223461in}}%
\pgfpathlineto{\pgfqpoint{6.725413in}{2.223461in}}%
\pgfpathlineto{\pgfqpoint{6.725413in}{2.223461in}}%
\pgfpathlineto{\pgfqpoint{6.726774in}{2.225884in}}%
\pgfpathlineto{\pgfqpoint{6.727997in}{2.227680in}}%
\pgfpathlineto{\pgfqpoint{6.728132in}{2.226928in}}%
\pgfpathlineto{\pgfqpoint{6.729488in}{2.224938in}}%
\pgfpathlineto{\pgfqpoint{6.729623in}{2.224753in}}%
\pgfpathlineto{\pgfqpoint{6.729623in}{2.224753in}}%
\pgfpathlineto{\pgfqpoint{6.729623in}{2.224753in}}%
\pgfpathlineto{\pgfqpoint{6.729894in}{2.226158in}}%
\pgfpathlineto{\pgfqpoint{6.730570in}{2.225070in}}%
\pgfpathlineto{\pgfqpoint{6.730976in}{2.223766in}}%
\pgfpathlineto{\pgfqpoint{6.731246in}{2.224455in}}%
\pgfpathlineto{\pgfqpoint{6.731246in}{2.224455in}}%
\pgfpathlineto{\pgfqpoint{6.732460in}{2.226528in}}%
\pgfpathlineto{\pgfqpoint{6.732999in}{2.225536in}}%
\pgfpathlineto{\pgfqpoint{6.733404in}{2.226469in}}%
\pgfusepath{stroke}%
\end{pgfscope}%
\begin{pgfscope}%
\pgfpathrectangle{\pgfqpoint{1.000000in}{0.300000in}}{\pgfqpoint{6.200000in}{2.400000in}} %
\pgfusepath{clip}%
\pgfsetrectcap%
\pgfsetroundjoin%
\pgfsetlinewidth{1.003750pt}%
\definecolor{currentstroke}{rgb}{0.000000,0.500000,0.000000}%
\pgfsetstrokecolor{currentstroke}%
\pgfsetdash{}{0pt}%
\pgfpathmoveto{\pgfqpoint{1.000000in}{0.479596in}}%
\pgfpathlineto{\pgfqpoint{1.466596in}{0.659706in}}%
\pgfpathlineto{\pgfqpoint{1.739538in}{0.747768in}}%
\pgfpathlineto{\pgfqpoint{1.933193in}{0.797601in}}%
\pgfpathlineto{\pgfqpoint{2.550000in}{0.933805in}}%
\pgfpathlineto{\pgfqpoint{2.672731in}{0.960515in}}%
\pgfpathlineto{\pgfqpoint{2.907196in}{1.014980in}}%
\pgfpathlineto{\pgfqpoint{3.110678in}{1.066823in}}%
\pgfpathlineto{\pgfqpoint{3.193209in}{1.089701in}}%
\pgfpathlineto{\pgfqpoint{3.448665in}{1.163300in}}%
\pgfpathlineto{\pgfqpoint{3.516036in}{1.180984in}}%
\pgfpathlineto{\pgfqpoint{3.633404in}{1.213700in}}%
\pgfpathlineto{\pgfqpoint{3.685210in}{1.226580in}}%
\pgfpathlineto{\pgfqpoint{3.767261in}{1.251933in}}%
\pgfpathlineto{\pgfqpoint{3.820293in}{1.268300in}}%
\pgfpathlineto{\pgfqpoint{3.949789in}{1.302819in}}%
\pgfpathlineto{\pgfqpoint{4.199910in}{1.379119in}}%
\pgfpathlineto{\pgfqpoint{4.255574in}{1.393159in}}%
\pgfpathlineto{\pgfqpoint{4.271414in}{1.399346in}}%
\pgfpathlineto{\pgfqpoint{4.302017in}{1.408146in}}%
\pgfpathlineto{\pgfqpoint{4.331290in}{1.416301in}}%
\pgfpathlineto{\pgfqpoint{4.359342in}{1.425235in}}%
\pgfpathlineto{\pgfqpoint{4.372941in}{1.428946in}}%
\pgfpathlineto{\pgfqpoint{4.395014in}{1.434524in}}%
\pgfpathlineto{\pgfqpoint{4.453224in}{1.452799in}}%
\pgfpathlineto{\pgfqpoint{4.517745in}{1.471600in}}%
\pgfpathlineto{\pgfqpoint{4.528516in}{1.475431in}}%
\pgfpathlineto{\pgfqpoint{4.539117in}{1.479185in}}%
\pgfpathlineto{\pgfqpoint{4.573295in}{1.488764in}}%
\pgfpathlineto{\pgfqpoint{4.596227in}{1.496721in}}%
\pgfpathlineto{\pgfqpoint{4.686517in}{1.523112in}}%
\pgfpathlineto{\pgfqpoint{4.703207in}{1.528736in}}%
\pgfpathlineto{\pgfqpoint{4.722171in}{1.534127in}}%
\pgfpathlineto{\pgfqpoint{4.732772in}{1.536441in}}%
\pgfpathlineto{\pgfqpoint{4.761092in}{1.545700in}}%
\pgfpathlineto{\pgfqpoint{4.783409in}{1.552366in}}%
\pgfpathlineto{\pgfqpoint{4.835035in}{1.568195in}}%
\pgfpathlineto{\pgfqpoint{4.844011in}{1.570761in}}%
\pgfpathlineto{\pgfqpoint{4.885083in}{1.583742in}}%
\pgfpathlineto{\pgfqpoint{4.949051in}{1.603028in}}%
\pgfpathlineto{\pgfqpoint{4.956636in}{1.605310in}}%
\pgfpathlineto{\pgfqpoint{4.978891in}{1.612508in}}%
\pgfpathlineto{\pgfqpoint{4.986149in}{1.614374in}}%
\pgfpathlineto{\pgfqpoint{4.993329in}{1.617488in}}%
\pgfpathlineto{\pgfqpoint{5.000434in}{1.619183in}}%
\pgfpathlineto{\pgfqpoint{5.009210in}{1.622085in}}%
\pgfpathlineto{\pgfqpoint{5.043215in}{1.633270in}}%
\pgfpathlineto{\pgfqpoint{5.051455in}{1.635868in}}%
\pgfpathlineto{\pgfqpoint{5.062823in}{1.639198in}}%
\pgfpathlineto{\pgfqpoint{5.074003in}{1.641954in}}%
\pgfpathlineto{\pgfqpoint{5.088109in}{1.647035in}}%
\pgfpathlineto{\pgfqpoint{5.106469in}{1.652707in}}%
\pgfpathlineto{\pgfqpoint{5.147456in}{1.665077in}}%
\pgfpathlineto{\pgfqpoint{5.153113in}{1.667082in}}%
\pgfpathlineto{\pgfqpoint{5.161510in}{1.669145in}}%
\pgfpathlineto{\pgfqpoint{5.252440in}{1.697068in}}%
\pgfpathlineto{\pgfqpoint{5.259691in}{1.699665in}}%
\pgfpathlineto{\pgfqpoint{5.364146in}{1.731708in}}%
\pgfpathlineto{\pgfqpoint{5.369272in}{1.733604in}}%
\pgfpathlineto{\pgfqpoint{5.379410in}{1.736908in}}%
\pgfpathlineto{\pgfqpoint{5.382422in}{1.737538in}}%
\pgfpathlineto{\pgfqpoint{5.392366in}{1.741402in}}%
\pgfpathlineto{\pgfqpoint{5.398262in}{1.742668in}}%
\pgfpathlineto{\pgfqpoint{5.425115in}{1.751773in}}%
\pgfpathlineto{\pgfqpoint{5.430733in}{1.753310in}}%
\pgfpathlineto{\pgfqpoint{5.438150in}{1.755758in}}%
\pgfpathlineto{\pgfqpoint{5.459032in}{1.761687in}}%
\pgfpathlineto{\pgfqpoint{5.465261in}{1.763580in}}%
\pgfpathlineto{\pgfqpoint{5.485331in}{1.770267in}}%
\pgfpathlineto{\pgfqpoint{5.518870in}{1.780770in}}%
\pgfpathlineto{\pgfqpoint{5.522947in}{1.782075in}}%
\pgfpathlineto{\pgfqpoint{5.537424in}{1.786760in}}%
\pgfpathlineto{\pgfqpoint{5.542971in}{1.787668in}}%
\pgfpathlineto{\pgfqpoint{5.549255in}{1.789657in}}%
\pgfpathlineto{\pgfqpoint{5.553930in}{1.791037in}}%
\pgfpathlineto{\pgfqpoint{5.580570in}{1.799782in}}%
\pgfpathlineto{\pgfqpoint{5.587992in}{1.802328in}}%
\pgfpathlineto{\pgfqpoint{5.615472in}{1.810782in}}%
\pgfpathlineto{\pgfqpoint{5.617594in}{1.810897in}}%
\pgfpathlineto{\pgfqpoint{5.623221in}{1.812565in}}%
\pgfpathlineto{\pgfqpoint{5.627411in}{1.813759in}}%
\pgfpathlineto{\pgfqpoint{5.646626in}{1.819923in}}%
\pgfpathlineto{\pgfqpoint{5.652687in}{1.822527in}}%
\pgfpathlineto{\pgfqpoint{5.658695in}{1.824426in}}%
\pgfpathlineto{\pgfqpoint{5.665307in}{1.826128in}}%
\pgfpathlineto{\pgfqpoint{5.669244in}{1.827519in}}%
\pgfpathlineto{\pgfqpoint{5.678986in}{1.830939in}}%
\pgfpathlineto{\pgfqpoint{5.697430in}{1.835807in}}%
\pgfpathlineto{\pgfqpoint{5.701807in}{1.837529in}}%
\pgfpathlineto{\pgfqpoint{5.722068in}{1.844395in}}%
\pgfpathlineto{\pgfqpoint{5.728688in}{1.845831in}}%
\pgfpathlineto{\pgfqpoint{5.732272in}{1.847337in}}%
\pgfpathlineto{\pgfqpoint{5.738203in}{1.849341in}}%
\pgfpathlineto{\pgfqpoint{5.742910in}{1.850683in}}%
\pgfpathlineto{\pgfqpoint{5.745835in}{1.851165in}}%
\pgfpathlineto{\pgfqpoint{5.749329in}{1.852049in}}%
\pgfpathlineto{\pgfqpoint{5.776646in}{1.860705in}}%
\pgfpathlineto{\pgfqpoint{5.780539in}{1.862226in}}%
\pgfpathlineto{\pgfqpoint{5.792084in}{1.865512in}}%
\pgfpathlineto{\pgfqpoint{5.794804in}{1.866629in}}%
\pgfpathlineto{\pgfqpoint{5.806108in}{1.869476in}}%
\pgfpathlineto{\pgfqpoint{5.809835in}{1.870446in}}%
\pgfpathlineto{\pgfqpoint{5.813013in}{1.871819in}}%
\pgfpathlineto{\pgfqpoint{5.817226in}{1.873008in}}%
\pgfpathlineto{\pgfqpoint{5.820892in}{1.873906in}}%
\pgfpathlineto{\pgfqpoint{5.824018in}{1.874763in}}%
\pgfpathlineto{\pgfqpoint{5.829196in}{1.876692in}}%
\pgfpathlineto{\pgfqpoint{5.841970in}{1.880815in}}%
\pgfpathlineto{\pgfqpoint{5.845504in}{1.882184in}}%
\pgfpathlineto{\pgfqpoint{5.851019in}{1.883818in}}%
\pgfpathlineto{\pgfqpoint{5.853512in}{1.884259in}}%
\pgfpathlineto{\pgfqpoint{5.855995in}{1.884634in}}%
\pgfpathlineto{\pgfqpoint{5.870219in}{1.888887in}}%
\pgfpathlineto{\pgfqpoint{5.873124in}{1.889749in}}%
\pgfpathlineto{\pgfqpoint{5.881767in}{1.893095in}}%
\pgfpathlineto{\pgfqpoint{5.898261in}{1.898248in}}%
\pgfpathlineto{\pgfqpoint{5.900584in}{1.898506in}}%
\pgfpathlineto{\pgfqpoint{5.902438in}{1.898538in}}%
\pgfpathlineto{\pgfqpoint{5.932300in}{1.908398in}}%
\pgfpathlineto{\pgfqpoint{5.941967in}{1.911828in}}%
\pgfpathlineto{\pgfqpoint{5.945882in}{1.912883in}}%
\pgfpathlineto{\pgfqpoint{5.949774in}{1.913403in}}%
\pgfpathlineto{\pgfqpoint{5.953644in}{1.914457in}}%
\pgfpathlineto{\pgfqpoint{5.957492in}{1.916318in}}%
\pgfpathlineto{\pgfqpoint{5.963434in}{1.917881in}}%
\pgfpathlineto{\pgfqpoint{5.968067in}{1.919555in}}%
\pgfpathlineto{\pgfqpoint{5.981364in}{1.924081in}}%
\pgfpathlineto{\pgfqpoint{5.983828in}{1.924877in}}%
\pgfpathlineto{\pgfqpoint{5.987916in}{1.926036in}}%
\pgfpathlineto{\pgfqpoint{5.991573in}{1.926438in}}%
\pgfpathlineto{\pgfqpoint{5.997625in}{1.928497in}}%
\pgfpathlineto{\pgfqpoint{6.015069in}{1.934505in}}%
\pgfpathlineto{\pgfqpoint{6.018193in}{1.935223in}}%
\pgfpathlineto{\pgfqpoint{6.035878in}{1.940450in}}%
\pgfpathlineto{\pgfqpoint{6.038529in}{1.941391in}}%
\pgfpathlineto{\pgfqpoint{6.069913in}{1.950775in}}%
\pgfpathlineto{\pgfqpoint{6.072434in}{1.951633in}}%
\pgfpathlineto{\pgfqpoint{6.075660in}{1.953006in}}%
\pgfpathlineto{\pgfqpoint{6.078160in}{1.953508in}}%
\pgfpathlineto{\pgfqpoint{6.081004in}{1.954721in}}%
\pgfpathlineto{\pgfqpoint{6.084544in}{1.956512in}}%
\pgfpathlineto{\pgfqpoint{6.087713in}{1.957554in}}%
\pgfpathlineto{\pgfqpoint{6.089818in}{1.957861in}}%
\pgfpathlineto{\pgfqpoint{6.094356in}{1.959567in}}%
\pgfpathlineto{\pgfqpoint{6.097479in}{1.960809in}}%
\pgfpathlineto{\pgfqpoint{6.105397in}{1.961883in}}%
\pgfpathlineto{\pgfqpoint{6.112207in}{1.964551in}}%
\pgfpathlineto{\pgfqpoint{6.113560in}{1.965059in}}%
\pgfpathlineto{\pgfqpoint{6.119284in}{1.968015in}}%
\pgfpathlineto{\pgfqpoint{6.126950in}{1.970348in}}%
\pgfpathlineto{\pgfqpoint{6.134531in}{1.972318in}}%
\pgfpathlineto{\pgfqpoint{6.136821in}{1.972738in}}%
\pgfpathlineto{\pgfqpoint{6.169649in}{1.983884in}}%
\pgfpathlineto{\pgfqpoint{6.171202in}{1.984158in}}%
\pgfpathlineto{\pgfqpoint{6.173681in}{1.985355in}}%
\pgfpathlineto{\pgfqpoint{6.176766in}{1.986332in}}%
\pgfpathlineto{\pgfqpoint{6.177996in}{1.986749in}}%
\pgfpathlineto{\pgfqpoint{6.182589in}{1.989081in}}%
\pgfpathlineto{\pgfqpoint{6.186847in}{1.990530in}}%
\pgfpathlineto{\pgfqpoint{6.189873in}{1.991447in}}%
\pgfpathlineto{\pgfqpoint{6.194085in}{1.993247in}}%
\pgfpathlineto{\pgfqpoint{6.211263in}{1.998386in}}%
\pgfpathlineto{\pgfqpoint{6.215345in}{2.000194in}}%
\pgfpathlineto{\pgfqpoint{6.217376in}{2.000495in}}%
\pgfpathlineto{\pgfqpoint{6.219113in}{2.001067in}}%
\pgfpathlineto{\pgfqpoint{6.225155in}{2.003314in}}%
\pgfpathlineto{\pgfqpoint{6.227157in}{2.003341in}}%
\pgfpathlineto{\pgfqpoint{6.231144in}{2.003851in}}%
\pgfpathlineto{\pgfqpoint{6.233694in}{2.005412in}}%
\pgfpathlineto{\pgfqpoint{6.236798in}{2.006291in}}%
\pgfpathlineto{\pgfqpoint{6.240728in}{2.008142in}}%
\pgfpathlineto{\pgfqpoint{6.245748in}{2.009671in}}%
\pgfpathlineto{\pgfqpoint{6.247413in}{2.010169in}}%
\pgfpathlineto{\pgfqpoint{6.252383in}{2.012048in}}%
\pgfpathlineto{\pgfqpoint{6.260585in}{2.014068in}}%
\pgfpathlineto{\pgfqpoint{6.263027in}{2.014847in}}%
\pgfpathlineto{\pgfqpoint{6.265729in}{2.015551in}}%
\pgfpathlineto{\pgfqpoint{6.269226in}{2.017471in}}%
\pgfpathlineto{\pgfqpoint{6.272972in}{2.018469in}}%
\pgfpathlineto{\pgfqpoint{6.274571in}{2.018901in}}%
\pgfpathlineto{\pgfqpoint{6.280929in}{2.020776in}}%
\pgfpathlineto{\pgfqpoint{6.282772in}{2.020628in}}%
\pgfpathlineto{\pgfqpoint{6.307046in}{2.028435in}}%
\pgfpathlineto{\pgfqpoint{6.310588in}{2.028899in}}%
\pgfpathlineto{\pgfqpoint{6.314111in}{2.030325in}}%
\pgfpathlineto{\pgfqpoint{6.315615in}{2.030801in}}%
\pgfpathlineto{\pgfqpoint{6.318364in}{2.032047in}}%
\pgfpathlineto{\pgfqpoint{6.319610in}{2.032656in}}%
\pgfpathlineto{\pgfqpoint{6.322839in}{2.033888in}}%
\pgfpathlineto{\pgfqpoint{6.337785in}{2.037744in}}%
\pgfpathlineto{\pgfqpoint{6.351932in}{2.042716in}}%
\pgfpathlineto{\pgfqpoint{6.353117in}{2.043096in}}%
\pgfpathlineto{\pgfqpoint{6.355482in}{2.044065in}}%
\pgfpathlineto{\pgfqpoint{6.357132in}{2.043649in}}%
\pgfpathlineto{\pgfqpoint{6.368803in}{2.047886in}}%
\pgfpathlineto{\pgfqpoint{6.372035in}{2.049052in}}%
\pgfpathlineto{\pgfqpoint{6.373645in}{2.049603in}}%
\pgfpathlineto{\pgfqpoint{6.379136in}{2.050703in}}%
\pgfpathlineto{\pgfqpoint{6.381184in}{2.051064in}}%
\pgfpathlineto{\pgfqpoint{6.385035in}{2.052453in}}%
\pgfpathlineto{\pgfqpoint{6.388640in}{2.053733in}}%
\pgfpathlineto{\pgfqpoint{6.389987in}{2.053872in}}%
\pgfpathlineto{\pgfqpoint{6.394234in}{2.055730in}}%
\pgfpathlineto{\pgfqpoint{6.396236in}{2.056381in}}%
\pgfpathlineto{\pgfqpoint{6.404187in}{2.057541in}}%
\pgfpathlineto{\pgfqpoint{6.405722in}{2.058381in}}%
\pgfpathlineto{\pgfqpoint{6.408782in}{2.059130in}}%
\pgfpathlineto{\pgfqpoint{6.410741in}{2.060205in}}%
\pgfpathlineto{\pgfqpoint{6.412262in}{2.060617in}}%
\pgfpathlineto{\pgfqpoint{6.416802in}{2.062484in}}%
\pgfpathlineto{\pgfqpoint{6.418524in}{2.062961in}}%
\pgfpathlineto{\pgfqpoint{6.475572in}{2.079222in}}%
\pgfpathlineto{\pgfqpoint{6.477150in}{2.080135in}}%
\pgfpathlineto{\pgfqpoint{6.479510in}{2.080467in}}%
\pgfpathlineto{\pgfqpoint{6.514136in}{2.092148in}}%
\pgfpathlineto{\pgfqpoint{6.515440in}{2.092580in}}%
\pgfpathlineto{\pgfqpoint{6.517299in}{2.092479in}}%
\pgfpathlineto{\pgfqpoint{6.519152in}{2.092870in}}%
\pgfpathlineto{\pgfqpoint{6.520077in}{2.092963in}}%
\pgfpathlineto{\pgfqpoint{6.520262in}{2.093248in}}%
\pgfpathlineto{\pgfqpoint{6.521923in}{2.094245in}}%
\pgfpathlineto{\pgfqpoint{6.522291in}{2.093806in}}%
\pgfpathlineto{\pgfqpoint{6.525049in}{2.094114in}}%
\pgfpathlineto{\pgfqpoint{6.527064in}{2.095345in}}%
\pgfpathlineto{\pgfqpoint{6.530349in}{2.096927in}}%
\pgfpathlineto{\pgfqpoint{6.530713in}{2.096668in}}%
\pgfpathlineto{\pgfqpoint{6.531440in}{2.097211in}}%
\pgfpathlineto{\pgfqpoint{6.535788in}{2.098513in}}%
\pgfpathlineto{\pgfqpoint{6.537050in}{2.098731in}}%
\pgfpathlineto{\pgfqpoint{6.538671in}{2.099513in}}%
\pgfpathlineto{\pgfqpoint{6.539928in}{2.100138in}}%
\pgfpathlineto{\pgfqpoint{6.541362in}{2.100056in}}%
\pgfpathlineto{\pgfqpoint{6.550434in}{2.103519in}}%
\pgfpathlineto{\pgfqpoint{6.550964in}{2.104543in}}%
\pgfpathlineto{\pgfqpoint{6.551669in}{2.104323in}}%
\pgfpathlineto{\pgfqpoint{6.553431in}{2.104227in}}%
\pgfpathlineto{\pgfqpoint{6.558163in}{2.105831in}}%
\pgfpathlineto{\pgfqpoint{6.559559in}{2.105789in}}%
\pgfpathlineto{\pgfqpoint{6.561474in}{2.106198in}}%
\pgfpathlineto{\pgfqpoint{6.563036in}{2.106924in}}%
\pgfpathlineto{\pgfqpoint{6.564422in}{2.107370in}}%
\pgfpathlineto{\pgfqpoint{6.567874in}{2.109399in}}%
\pgfpathlineto{\pgfqpoint{6.568390in}{2.110566in}}%
\pgfpathlineto{\pgfqpoint{6.569078in}{2.110338in}}%
\pgfpathlineto{\pgfqpoint{6.570623in}{2.110155in}}%
\pgfpathlineto{\pgfqpoint{6.572506in}{2.110885in}}%
\pgfpathlineto{\pgfqpoint{6.573873in}{2.111213in}}%
\pgfpathlineto{\pgfqpoint{6.578634in}{2.111872in}}%
\pgfpathlineto{\pgfqpoint{6.580157in}{2.112620in}}%
\pgfpathlineto{\pgfqpoint{6.583361in}{2.113910in}}%
\pgfpathlineto{\pgfqpoint{6.587387in}{2.116166in}}%
\pgfpathlineto{\pgfqpoint{6.588557in}{2.116625in}}%
\pgfpathlineto{\pgfqpoint{6.590890in}{2.117380in}}%
\pgfpathlineto{\pgfqpoint{6.591056in}{2.117061in}}%
\pgfpathlineto{\pgfqpoint{6.591555in}{2.117607in}}%
\pgfpathlineto{\pgfqpoint{6.591555in}{2.117607in}}%
\pgfpathlineto{\pgfqpoint{6.592718in}{2.117558in}}%
\pgfpathlineto{\pgfqpoint{6.594375in}{2.117055in}}%
\pgfpathlineto{\pgfqpoint{6.596523in}{2.118472in}}%
\pgfpathlineto{\pgfqpoint{6.598994in}{2.119406in}}%
\pgfpathlineto{\pgfqpoint{6.601455in}{2.121031in}}%
\pgfpathlineto{\pgfqpoint{6.601946in}{2.122079in}}%
\pgfpathlineto{\pgfqpoint{6.602600in}{2.121474in}}%
\pgfpathlineto{\pgfqpoint{6.604233in}{2.121620in}}%
\pgfpathlineto{\pgfqpoint{6.607001in}{2.122978in}}%
\pgfpathlineto{\pgfqpoint{6.607650in}{2.122634in}}%
\pgfpathlineto{\pgfqpoint{6.608137in}{2.123397in}}%
\pgfpathlineto{\pgfqpoint{6.608299in}{2.123608in}}%
\pgfpathlineto{\pgfqpoint{6.608623in}{2.123156in}}%
\pgfpathlineto{\pgfqpoint{6.608623in}{2.123156in}}%
\pgfpathlineto{\pgfqpoint{6.609756in}{2.122990in}}%
\pgfpathlineto{\pgfqpoint{6.625111in}{2.128013in}}%
\pgfpathlineto{\pgfqpoint{6.627006in}{2.127666in}}%
\pgfpathlineto{\pgfqpoint{6.629681in}{2.129842in}}%
\pgfpathlineto{\pgfqpoint{6.629838in}{2.129599in}}%
\pgfpathlineto{\pgfqpoint{6.630779in}{2.129583in}}%
\pgfpathlineto{\pgfqpoint{6.630936in}{2.129777in}}%
\pgfpathlineto{\pgfqpoint{6.632658in}{2.131126in}}%
\pgfpathlineto{\pgfqpoint{6.633908in}{2.132723in}}%
\pgfpathlineto{\pgfqpoint{6.634376in}{2.131801in}}%
\pgfpathlineto{\pgfqpoint{6.640277in}{2.133117in}}%
\pgfpathlineto{\pgfqpoint{6.641050in}{2.132653in}}%
\pgfpathlineto{\pgfqpoint{6.641358in}{2.133099in}}%
\pgfpathlineto{\pgfqpoint{6.642900in}{2.132895in}}%
\pgfpathlineto{\pgfqpoint{6.643054in}{2.133103in}}%
\pgfpathlineto{\pgfqpoint{6.645206in}{2.134478in}}%
\pgfpathlineto{\pgfqpoint{6.645360in}{2.134181in}}%
\pgfpathlineto{\pgfqpoint{6.645667in}{2.133964in}}%
\pgfpathlineto{\pgfqpoint{6.646433in}{2.134449in}}%
\pgfpathlineto{\pgfqpoint{6.649185in}{2.136144in}}%
\pgfpathlineto{\pgfqpoint{6.650405in}{2.136674in}}%
\pgfpathlineto{\pgfqpoint{6.653292in}{2.136863in}}%
\pgfpathlineto{\pgfqpoint{6.654958in}{2.137517in}}%
\pgfpathlineto{\pgfqpoint{6.655260in}{2.138043in}}%
\pgfpathlineto{\pgfqpoint{6.655412in}{2.137449in}}%
\pgfpathlineto{\pgfqpoint{6.655412in}{2.137449in}}%
\pgfpathlineto{\pgfqpoint{6.656922in}{2.136413in}}%
\pgfpathlineto{\pgfqpoint{6.659781in}{2.138675in}}%
\pgfpathlineto{\pgfqpoint{6.664272in}{2.140814in}}%
\pgfpathlineto{\pgfqpoint{6.664421in}{2.141100in}}%
\pgfpathlineto{\pgfqpoint{6.664421in}{2.141100in}}%
\pgfpathlineto{\pgfqpoint{6.664421in}{2.141100in}}%
\pgfpathlineto{\pgfqpoint{6.664720in}{2.140339in}}%
\pgfpathlineto{\pgfqpoint{6.665465in}{2.140970in}}%
\pgfpathlineto{\pgfqpoint{6.670213in}{2.141699in}}%
\pgfpathlineto{\pgfqpoint{6.670361in}{2.140978in}}%
\pgfpathlineto{\pgfqpoint{6.671248in}{2.140211in}}%
\pgfpathlineto{\pgfqpoint{6.671543in}{2.140615in}}%
\pgfpathlineto{\pgfqpoint{6.673312in}{2.140832in}}%
\pgfpathlineto{\pgfqpoint{6.674782in}{2.142776in}}%
\pgfpathlineto{\pgfqpoint{6.677275in}{2.142176in}}%
\pgfpathlineto{\pgfqpoint{6.677421in}{2.142559in}}%
\pgfpathlineto{\pgfqpoint{6.682958in}{2.147154in}}%
\pgfpathlineto{\pgfqpoint{6.683248in}{2.146174in}}%
\pgfpathlineto{\pgfqpoint{6.686864in}{2.144579in}}%
\pgfpathlineto{\pgfqpoint{6.687009in}{2.144204in}}%
\pgfpathlineto{\pgfqpoint{6.687297in}{2.144712in}}%
\pgfpathlineto{\pgfqpoint{6.687297in}{2.144712in}}%
\pgfpathlineto{\pgfqpoint{6.689312in}{2.147332in}}%
\pgfpathlineto{\pgfqpoint{6.689744in}{2.146980in}}%
\pgfpathlineto{\pgfqpoint{6.690174in}{2.146246in}}%
\pgfpathlineto{\pgfqpoint{6.690892in}{2.146993in}}%
\pgfpathlineto{\pgfqpoint{6.692754in}{2.147797in}}%
\pgfpathlineto{\pgfqpoint{6.696035in}{2.151148in}}%
\pgfpathlineto{\pgfqpoint{6.696177in}{2.150874in}}%
\pgfpathlineto{\pgfqpoint{6.696319in}{2.150398in}}%
\pgfpathlineto{\pgfqpoint{6.696462in}{2.150930in}}%
\pgfpathlineto{\pgfqpoint{6.696462in}{2.150930in}}%
\pgfpathlineto{\pgfqpoint{6.697030in}{2.152528in}}%
\pgfpathlineto{\pgfqpoint{6.697598in}{2.150817in}}%
\pgfpathlineto{\pgfqpoint{6.702127in}{2.150412in}}%
\pgfpathlineto{\pgfqpoint{6.703395in}{2.151676in}}%
\pgfpathlineto{\pgfqpoint{6.703536in}{2.151358in}}%
\pgfpathlineto{\pgfqpoint{6.704942in}{2.150389in}}%
\pgfpathlineto{\pgfqpoint{6.707045in}{2.152276in}}%
\pgfpathlineto{\pgfqpoint{6.708444in}{2.153634in}}%
\pgfpathlineto{\pgfqpoint{6.708723in}{2.152846in}}%
\pgfpathlineto{\pgfqpoint{6.709002in}{2.153261in}}%
\pgfpathlineto{\pgfqpoint{6.709002in}{2.153261in}}%
\pgfpathlineto{\pgfqpoint{6.709281in}{2.154393in}}%
\pgfpathlineto{\pgfqpoint{6.709421in}{2.153753in}}%
\pgfpathlineto{\pgfqpoint{6.709421in}{2.153753in}}%
\pgfpathlineto{\pgfqpoint{6.709700in}{2.152917in}}%
\pgfpathlineto{\pgfqpoint{6.709979in}{2.153173in}}%
\pgfpathlineto{\pgfqpoint{6.709979in}{2.153173in}}%
\pgfpathlineto{\pgfqpoint{6.711371in}{2.155587in}}%
\pgfpathlineto{\pgfqpoint{6.712483in}{2.153423in}}%
\pgfpathlineto{\pgfqpoint{6.712761in}{2.153998in}}%
\pgfpathlineto{\pgfqpoint{6.713038in}{2.155324in}}%
\pgfpathlineto{\pgfqpoint{6.714009in}{2.154639in}}%
\pgfpathlineto{\pgfqpoint{6.714286in}{2.155186in}}%
\pgfpathlineto{\pgfqpoint{6.714563in}{2.154536in}}%
\pgfpathlineto{\pgfqpoint{6.714563in}{2.154536in}}%
\pgfpathlineto{\pgfqpoint{6.715532in}{2.154008in}}%
\pgfpathlineto{\pgfqpoint{6.715670in}{2.154439in}}%
\pgfpathlineto{\pgfqpoint{6.716775in}{2.157093in}}%
\pgfpathlineto{\pgfqpoint{6.717189in}{2.156769in}}%
\pgfpathlineto{\pgfqpoint{6.717327in}{2.157056in}}%
\pgfpathlineto{\pgfqpoint{6.717464in}{2.156639in}}%
\pgfpathlineto{\pgfqpoint{6.717464in}{2.156639in}}%
\pgfpathlineto{\pgfqpoint{6.718291in}{2.154857in}}%
\pgfpathlineto{\pgfqpoint{6.718979in}{2.155261in}}%
\pgfpathlineto{\pgfqpoint{6.719117in}{2.154968in}}%
\pgfpathlineto{\pgfqpoint{6.719254in}{2.155293in}}%
\pgfpathlineto{\pgfqpoint{6.719254in}{2.155293in}}%
\pgfpathlineto{\pgfqpoint{6.719941in}{2.157962in}}%
\pgfpathlineto{\pgfqpoint{6.720628in}{2.156795in}}%
\pgfpathlineto{\pgfqpoint{6.722546in}{2.160032in}}%
\pgfpathlineto{\pgfqpoint{6.722819in}{2.160107in}}%
\pgfpathlineto{\pgfqpoint{6.722956in}{2.161124in}}%
\pgfpathlineto{\pgfqpoint{6.723230in}{2.160185in}}%
\pgfpathlineto{\pgfqpoint{6.723230in}{2.160185in}}%
\pgfpathlineto{\pgfqpoint{6.723503in}{2.158690in}}%
\pgfpathlineto{\pgfqpoint{6.723776in}{2.159416in}}%
\pgfpathlineto{\pgfqpoint{6.723776in}{2.159416in}}%
\pgfpathlineto{\pgfqpoint{6.723913in}{2.160772in}}%
\pgfpathlineto{\pgfqpoint{6.724186in}{2.159644in}}%
\pgfpathlineto{\pgfqpoint{6.724186in}{2.159644in}}%
\pgfpathlineto{\pgfqpoint{6.724595in}{2.159205in}}%
\pgfpathlineto{\pgfqpoint{6.725004in}{2.159743in}}%
\pgfpathlineto{\pgfqpoint{6.725004in}{2.159743in}}%
\pgfpathlineto{\pgfqpoint{6.725140in}{2.160135in}}%
\pgfpathlineto{\pgfqpoint{6.725140in}{2.160135in}}%
\pgfpathlineto{\pgfqpoint{6.725140in}{2.160135in}}%
\pgfpathlineto{\pgfqpoint{6.725277in}{2.159035in}}%
\pgfpathlineto{\pgfqpoint{6.725413in}{2.159853in}}%
\pgfpathlineto{\pgfqpoint{6.725413in}{2.159853in}}%
\pgfpathlineto{\pgfqpoint{6.725822in}{2.160487in}}%
\pgfpathlineto{\pgfqpoint{6.726094in}{2.159883in}}%
\pgfpathlineto{\pgfqpoint{6.726094in}{2.159883in}}%
\pgfpathlineto{\pgfqpoint{6.726230in}{2.158965in}}%
\pgfpathlineto{\pgfqpoint{6.726366in}{2.159126in}}%
\pgfpathlineto{\pgfqpoint{6.726366in}{2.159126in}}%
\pgfpathlineto{\pgfqpoint{6.726774in}{2.162268in}}%
\pgfpathlineto{\pgfqpoint{6.727454in}{2.159415in}}%
\pgfpathlineto{\pgfqpoint{6.728404in}{2.157636in}}%
\pgfpathlineto{\pgfqpoint{6.728539in}{2.158889in}}%
\pgfpathlineto{\pgfqpoint{6.730705in}{2.164150in}}%
\pgfpathlineto{\pgfqpoint{6.731516in}{2.160899in}}%
\pgfpathlineto{\pgfqpoint{6.732056in}{2.162660in}}%
\pgfpathlineto{\pgfqpoint{6.732191in}{2.163222in}}%
\pgfpathlineto{\pgfqpoint{6.732595in}{2.162848in}}%
\pgfpathlineto{\pgfqpoint{6.732595in}{2.162848in}}%
\pgfpathlineto{\pgfqpoint{6.732730in}{2.162368in}}%
\pgfpathlineto{\pgfqpoint{6.732865in}{2.162859in}}%
\pgfpathlineto{\pgfqpoint{6.732865in}{2.162859in}}%
\pgfpathlineto{\pgfqpoint{6.733269in}{2.163804in}}%
\pgfpathlineto{\pgfqpoint{6.733404in}{2.163490in}}%
\pgfusepath{stroke}%
\end{pgfscope}%
\begin{pgfscope}%
\pgfpathrectangle{\pgfqpoint{1.000000in}{0.300000in}}{\pgfqpoint{6.200000in}{2.400000in}} %
\pgfusepath{clip}%
\pgfsetrectcap%
\pgfsetroundjoin%
\pgfsetlinewidth{1.003750pt}%
\definecolor{currentstroke}{rgb}{1.000000,0.000000,0.000000}%
\pgfsetstrokecolor{currentstroke}%
\pgfsetdash{}{0pt}%
\pgfpathmoveto{\pgfqpoint{1.000000in}{0.497039in}}%
\pgfpathlineto{\pgfqpoint{1.466596in}{0.692308in}}%
\pgfpathlineto{\pgfqpoint{1.739538in}{0.795276in}}%
\pgfpathlineto{\pgfqpoint{1.933193in}{0.858682in}}%
\pgfpathlineto{\pgfqpoint{2.309902in}{0.971959in}}%
\pgfpathlineto{\pgfqpoint{2.866386in}{1.132900in}}%
\pgfpathlineto{\pgfqpoint{2.945672in}{1.155466in}}%
\pgfpathlineto{\pgfqpoint{3.049440in}{1.187056in}}%
\pgfpathlineto{\pgfqpoint{3.218614in}{1.238008in}}%
\pgfpathlineto{\pgfqpoint{3.332982in}{1.269423in}}%
\pgfpathlineto{\pgfqpoint{3.393305in}{1.287605in}}%
\pgfpathlineto{\pgfqpoint{3.448665in}{1.304469in}}%
\pgfpathlineto{\pgfqpoint{3.799579in}{1.406377in}}%
\pgfpathlineto{\pgfqpoint{3.897309in}{1.435879in}}%
\pgfpathlineto{\pgfqpoint{3.941322in}{1.448895in}}%
\pgfpathlineto{\pgfqpoint{4.043871in}{1.479495in}}%
\pgfpathlineto{\pgfqpoint{4.086400in}{1.493359in}}%
\pgfpathlineto{\pgfqpoint{4.119898in}{1.502007in}}%
\pgfpathlineto{\pgfqpoint{4.524945in}{1.625851in}}%
\pgfpathlineto{\pgfqpoint{4.535602in}{1.628961in}}%
\pgfpathlineto{\pgfqpoint{4.552997in}{1.634833in}}%
\pgfpathlineto{\pgfqpoint{4.563222in}{1.637750in}}%
\pgfpathlineto{\pgfqpoint{4.579927in}{1.641933in}}%
\pgfpathlineto{\pgfqpoint{4.998665in}{1.769457in}}%
\pgfpathlineto{\pgfqpoint{5.007464in}{1.771740in}}%
\pgfpathlineto{\pgfqpoint{5.021308in}{1.775907in}}%
\pgfpathlineto{\pgfqpoint{5.049815in}{1.784149in}}%
\pgfpathlineto{\pgfqpoint{5.081876in}{1.794044in}}%
\pgfpathlineto{\pgfqpoint{5.091204in}{1.796726in}}%
\pgfpathlineto{\pgfqpoint{5.130195in}{1.809168in}}%
\pgfpathlineto{\pgfqpoint{5.241415in}{1.842110in}}%
\pgfpathlineto{\pgfqpoint{5.264483in}{1.849405in}}%
\pgfpathlineto{\pgfqpoint{5.301632in}{1.860818in}}%
\pgfpathlineto{\pgfqpoint{5.305012in}{1.861593in}}%
\pgfpathlineto{\pgfqpoint{5.312833in}{1.863141in}}%
\pgfpathlineto{\pgfqpoint{5.334690in}{1.869678in}}%
\pgfpathlineto{\pgfqpoint{5.358979in}{1.877564in}}%
\pgfpathlineto{\pgfqpoint{5.365174in}{1.879731in}}%
\pgfpathlineto{\pgfqpoint{5.373346in}{1.882373in}}%
\pgfpathlineto{\pgfqpoint{5.380416in}{1.884268in}}%
\pgfpathlineto{\pgfqpoint{5.391378in}{1.887810in}}%
\pgfpathlineto{\pgfqpoint{5.400216in}{1.890299in}}%
\pgfpathlineto{\pgfqpoint{5.407976in}{1.892807in}}%
\pgfpathlineto{\pgfqpoint{5.411823in}{1.893105in}}%
\pgfpathlineto{\pgfqpoint{5.456345in}{1.906656in}}%
\pgfpathlineto{\pgfqpoint{5.460818in}{1.908371in}}%
\pgfpathlineto{\pgfqpoint{5.467030in}{1.910269in}}%
\pgfpathlineto{\pgfqpoint{5.619005in}{1.957172in}}%
\pgfpathlineto{\pgfqpoint{5.624621in}{1.958803in}}%
\pgfpathlineto{\pgfqpoint{5.629496in}{1.960127in}}%
\pgfpathlineto{\pgfqpoint{5.649327in}{1.966088in}}%
\pgfpathlineto{\pgfqpoint{5.653357in}{1.966965in}}%
\pgfpathlineto{\pgfqpoint{5.673808in}{1.973736in}}%
\pgfpathlineto{\pgfqpoint{5.679630in}{1.976203in}}%
\pgfpathlineto{\pgfqpoint{5.712934in}{1.986445in}}%
\pgfpathlineto{\pgfqpoint{5.715992in}{1.986608in}}%
\pgfpathlineto{\pgfqpoint{5.735836in}{1.993214in}}%
\pgfpathlineto{\pgfqpoint{5.741149in}{1.995296in}}%
\pgfpathlineto{\pgfqpoint{5.769357in}{2.003805in}}%
\pgfpathlineto{\pgfqpoint{5.785511in}{2.008552in}}%
\pgfpathlineto{\pgfqpoint{5.793717in}{2.011597in}}%
\pgfpathlineto{\pgfqpoint{5.815123in}{2.018426in}}%
\pgfpathlineto{\pgfqpoint{5.818276in}{2.019400in}}%
\pgfpathlineto{\pgfqpoint{5.830227in}{2.022609in}}%
\pgfpathlineto{\pgfqpoint{5.841970in}{2.027038in}}%
\pgfpathlineto{\pgfqpoint{5.846007in}{2.028141in}}%
\pgfpathlineto{\pgfqpoint{5.876017in}{2.037078in}}%
\pgfpathlineto{\pgfqpoint{5.879377in}{2.037980in}}%
\pgfpathlineto{\pgfqpoint{5.888885in}{2.041443in}}%
\pgfpathlineto{\pgfqpoint{5.906129in}{2.047084in}}%
\pgfpathlineto{\pgfqpoint{5.911628in}{2.048694in}}%
\pgfpathlineto{\pgfqpoint{5.917082in}{2.050500in}}%
\pgfpathlineto{\pgfqpoint{5.924734in}{2.052293in}}%
\pgfpathlineto{\pgfqpoint{5.945448in}{2.059634in}}%
\pgfpathlineto{\pgfqpoint{6.043049in}{2.090130in}}%
\pgfpathlineto{\pgfqpoint{6.047540in}{2.091141in}}%
\pgfpathlineto{\pgfqpoint{6.049774in}{2.092381in}}%
\pgfpathlineto{\pgfqpoint{6.052371in}{2.093284in}}%
\pgfpathlineto{\pgfqpoint{6.056063in}{2.094414in}}%
\pgfpathlineto{\pgfqpoint{6.064116in}{2.096893in}}%
\pgfpathlineto{\pgfqpoint{6.066296in}{2.097719in}}%
\pgfpathlineto{\pgfqpoint{6.069191in}{2.098393in}}%
\pgfpathlineto{\pgfqpoint{6.072434in}{2.099375in}}%
\pgfpathlineto{\pgfqpoint{6.085602in}{2.103313in}}%
\pgfpathlineto{\pgfqpoint{6.089117in}{2.104423in}}%
\pgfpathlineto{\pgfqpoint{6.094008in}{2.106086in}}%
\pgfpathlineto{\pgfqpoint{6.097133in}{2.107807in}}%
\pgfpathlineto{\pgfqpoint{6.174299in}{2.132652in}}%
\pgfpathlineto{\pgfqpoint{6.177073in}{2.133243in}}%
\pgfpathlineto{\pgfqpoint{6.179530in}{2.134044in}}%
\pgfpathlineto{\pgfqpoint{6.181978in}{2.134561in}}%
\pgfpathlineto{\pgfqpoint{6.191681in}{2.138286in}}%
\pgfpathlineto{\pgfqpoint{6.194685in}{2.139681in}}%
\pgfpathlineto{\pgfqpoint{6.197377in}{2.139982in}}%
\pgfpathlineto{\pgfqpoint{6.218245in}{2.146838in}}%
\pgfpathlineto{\pgfqpoint{6.318364in}{2.180251in}}%
\pgfpathlineto{\pgfqpoint{6.320357in}{2.180704in}}%
\pgfpathlineto{\pgfqpoint{6.322839in}{2.181591in}}%
\pgfpathlineto{\pgfqpoint{6.365090in}{2.195834in}}%
\pgfpathlineto{\pgfqpoint{6.367645in}{2.196672in}}%
\pgfpathlineto{\pgfqpoint{6.372955in}{2.198341in}}%
\pgfpathlineto{\pgfqpoint{6.377082in}{2.200538in}}%
\pgfpathlineto{\pgfqpoint{6.383452in}{2.202164in}}%
\pgfpathlineto{\pgfqpoint{6.385938in}{2.203395in}}%
\pgfpathlineto{\pgfqpoint{6.388190in}{2.203301in}}%
\pgfpathlineto{\pgfqpoint{6.396680in}{2.206281in}}%
\pgfpathlineto{\pgfqpoint{6.398454in}{2.207027in}}%
\pgfpathlineto{\pgfqpoint{6.400444in}{2.207682in}}%
\pgfpathlineto{\pgfqpoint{6.401767in}{2.208038in}}%
\pgfpathlineto{\pgfqpoint{6.409871in}{2.210441in}}%
\pgfpathlineto{\pgfqpoint{6.413562in}{2.211562in}}%
\pgfpathlineto{\pgfqpoint{6.430242in}{2.217153in}}%
\pgfpathlineto{\pgfqpoint{6.434663in}{2.217938in}}%
\pgfpathlineto{\pgfqpoint{6.436968in}{2.219095in}}%
\pgfpathlineto{\pgfqpoint{6.438847in}{2.219358in}}%
\pgfpathlineto{\pgfqpoint{6.449399in}{2.223862in}}%
\pgfpathlineto{\pgfqpoint{6.451040in}{2.223801in}}%
\pgfpathlineto{\pgfqpoint{6.455123in}{2.224347in}}%
\pgfpathlineto{\pgfqpoint{6.458372in}{2.225655in}}%
\pgfpathlineto{\pgfqpoint{6.468824in}{2.229473in}}%
\pgfpathlineto{\pgfqpoint{6.472405in}{2.230411in}}%
\pgfpathlineto{\pgfqpoint{6.473792in}{2.230417in}}%
\pgfpathlineto{\pgfqpoint{6.475374in}{2.231303in}}%
\pgfpathlineto{\pgfqpoint{6.477740in}{2.232369in}}%
\pgfpathlineto{\pgfqpoint{6.481861in}{2.234155in}}%
\pgfpathlineto{\pgfqpoint{6.483229in}{2.234577in}}%
\pgfpathlineto{\pgfqpoint{6.484400in}{2.234315in}}%
\pgfpathlineto{\pgfqpoint{6.484594in}{2.234523in}}%
\pgfpathlineto{\pgfqpoint{6.487899in}{2.235700in}}%
\pgfpathlineto{\pgfqpoint{6.488867in}{2.236107in}}%
\pgfpathlineto{\pgfqpoint{6.490607in}{2.236870in}}%
\pgfpathlineto{\pgfqpoint{6.495801in}{2.238429in}}%
\pgfpathlineto{\pgfqpoint{6.497715in}{2.238562in}}%
\pgfpathlineto{\pgfqpoint{6.500194in}{2.240128in}}%
\pgfpathlineto{\pgfqpoint{6.502285in}{2.240955in}}%
\pgfpathlineto{\pgfqpoint{6.506259in}{2.241595in}}%
\pgfpathlineto{\pgfqpoint{6.514323in}{2.244063in}}%
\pgfpathlineto{\pgfqpoint{6.515626in}{2.244475in}}%
\pgfpathlineto{\pgfqpoint{6.538131in}{2.251191in}}%
\pgfpathlineto{\pgfqpoint{6.539389in}{2.251686in}}%
\pgfpathlineto{\pgfqpoint{6.540466in}{2.251502in}}%
\pgfpathlineto{\pgfqpoint{6.540645in}{2.251699in}}%
\pgfpathlineto{\pgfqpoint{6.543151in}{2.252497in}}%
\pgfpathlineto{\pgfqpoint{6.544578in}{2.252838in}}%
\pgfpathlineto{\pgfqpoint{6.545824in}{2.253592in}}%
\pgfpathlineto{\pgfqpoint{6.546180in}{2.253302in}}%
\pgfpathlineto{\pgfqpoint{6.550610in}{2.254579in}}%
\pgfpathlineto{\pgfqpoint{6.551846in}{2.254795in}}%
\pgfpathlineto{\pgfqpoint{6.554134in}{2.255789in}}%
\pgfpathlineto{\pgfqpoint{6.568046in}{2.259637in}}%
\pgfpathlineto{\pgfqpoint{6.569250in}{2.259864in}}%
\pgfpathlineto{\pgfqpoint{6.572164in}{2.260993in}}%
\pgfpathlineto{\pgfqpoint{6.574385in}{2.261659in}}%
\pgfpathlineto{\pgfqpoint{6.574555in}{2.261326in}}%
\pgfpathlineto{\pgfqpoint{6.575747in}{2.261692in}}%
\pgfpathlineto{\pgfqpoint{6.576598in}{2.262053in}}%
\pgfpathlineto{\pgfqpoint{6.576937in}{2.261506in}}%
\pgfpathlineto{\pgfqpoint{6.579480in}{2.262356in}}%
\pgfpathlineto{\pgfqpoint{6.580495in}{2.262987in}}%
\pgfpathlineto{\pgfqpoint{6.580833in}{2.262700in}}%
\pgfpathlineto{\pgfqpoint{6.587053in}{2.264223in}}%
\pgfpathlineto{\pgfqpoint{6.588056in}{2.264886in}}%
\pgfpathlineto{\pgfqpoint{6.589391in}{2.265758in}}%
\pgfpathlineto{\pgfqpoint{6.589724in}{2.265532in}}%
\pgfpathlineto{\pgfqpoint{6.591555in}{2.265608in}}%
\pgfpathlineto{\pgfqpoint{6.592552in}{2.266069in}}%
\pgfpathlineto{\pgfqpoint{6.593381in}{2.266255in}}%
\pgfpathlineto{\pgfqpoint{6.593547in}{2.265880in}}%
\pgfpathlineto{\pgfqpoint{6.594209in}{2.265656in}}%
\pgfpathlineto{\pgfqpoint{6.594540in}{2.265951in}}%
\pgfpathlineto{\pgfqpoint{6.597348in}{2.267362in}}%
\pgfpathlineto{\pgfqpoint{6.598994in}{2.267561in}}%
\pgfpathlineto{\pgfqpoint{6.600307in}{2.268166in}}%
\pgfpathlineto{\pgfqpoint{6.602437in}{2.268418in}}%
\pgfpathlineto{\pgfqpoint{6.609756in}{2.270727in}}%
\pgfpathlineto{\pgfqpoint{6.609918in}{2.270273in}}%
\pgfpathlineto{\pgfqpoint{6.610242in}{2.269786in}}%
\pgfpathlineto{\pgfqpoint{6.610888in}{2.270219in}}%
\pgfpathlineto{\pgfqpoint{6.612984in}{2.271715in}}%
\pgfpathlineto{\pgfqpoint{6.614592in}{2.271477in}}%
\pgfpathlineto{\pgfqpoint{6.616837in}{2.272204in}}%
\pgfpathlineto{\pgfqpoint{6.618756in}{2.272739in}}%
\pgfpathlineto{\pgfqpoint{6.621146in}{2.273749in}}%
\pgfpathlineto{\pgfqpoint{6.622576in}{2.273532in}}%
\pgfpathlineto{\pgfqpoint{6.623686in}{2.274094in}}%
\pgfpathlineto{\pgfqpoint{6.623844in}{2.273952in}}%
\pgfpathlineto{\pgfqpoint{6.624953in}{2.274563in}}%
\pgfpathlineto{\pgfqpoint{6.625743in}{2.275000in}}%
\pgfpathlineto{\pgfqpoint{6.625901in}{2.274494in}}%
\pgfpathlineto{\pgfqpoint{6.626217in}{2.273954in}}%
\pgfpathlineto{\pgfqpoint{6.626848in}{2.274423in}}%
\pgfpathlineto{\pgfqpoint{6.628108in}{2.274722in}}%
\pgfpathlineto{\pgfqpoint{6.630152in}{2.275772in}}%
\pgfpathlineto{\pgfqpoint{6.630936in}{2.275640in}}%
\pgfpathlineto{\pgfqpoint{6.631093in}{2.275884in}}%
\pgfpathlineto{\pgfqpoint{6.631563in}{2.276792in}}%
\pgfpathlineto{\pgfqpoint{6.631876in}{2.276250in}}%
\pgfpathlineto{\pgfqpoint{6.631876in}{2.276250in}}%
\pgfpathlineto{\pgfqpoint{6.632033in}{2.275627in}}%
\pgfpathlineto{\pgfqpoint{6.632971in}{2.276252in}}%
\pgfpathlineto{\pgfqpoint{6.634064in}{2.276792in}}%
\pgfpathlineto{\pgfqpoint{6.634220in}{2.276519in}}%
\pgfpathlineto{\pgfqpoint{6.636090in}{2.276819in}}%
\pgfpathlineto{\pgfqpoint{6.637644in}{2.277894in}}%
\pgfpathlineto{\pgfqpoint{6.638264in}{2.277680in}}%
\pgfpathlineto{\pgfqpoint{6.638729in}{2.278328in}}%
\pgfpathlineto{\pgfqpoint{6.641358in}{2.279088in}}%
\pgfpathlineto{\pgfqpoint{6.641821in}{2.277942in}}%
\pgfpathlineto{\pgfqpoint{6.642592in}{2.278738in}}%
\pgfpathlineto{\pgfqpoint{6.645206in}{2.279635in}}%
\pgfpathlineto{\pgfqpoint{6.651622in}{2.281431in}}%
\pgfpathlineto{\pgfqpoint{6.654655in}{2.282137in}}%
\pgfpathlineto{\pgfqpoint{6.655563in}{2.281478in}}%
\pgfpathlineto{\pgfqpoint{6.655865in}{2.282233in}}%
\pgfpathlineto{\pgfqpoint{6.656620in}{2.282307in}}%
\pgfpathlineto{\pgfqpoint{6.656771in}{2.281878in}}%
\pgfpathlineto{\pgfqpoint{6.656922in}{2.281424in}}%
\pgfpathlineto{\pgfqpoint{6.657826in}{2.282217in}}%
\pgfpathlineto{\pgfqpoint{6.663376in}{2.283856in}}%
\pgfpathlineto{\pgfqpoint{6.665762in}{2.284350in}}%
\pgfpathlineto{\pgfqpoint{6.669178in}{2.285518in}}%
\pgfpathlineto{\pgfqpoint{6.672870in}{2.286023in}}%
\pgfpathlineto{\pgfqpoint{6.673017in}{2.286341in}}%
\pgfpathlineto{\pgfqpoint{6.673165in}{2.285980in}}%
\pgfpathlineto{\pgfqpoint{6.673165in}{2.285980in}}%
\pgfpathlineto{\pgfqpoint{6.673606in}{2.285751in}}%
\pgfpathlineto{\pgfqpoint{6.674194in}{2.286356in}}%
\pgfpathlineto{\pgfqpoint{6.676249in}{2.287045in}}%
\pgfpathlineto{\pgfqpoint{6.678298in}{2.286788in}}%
\pgfpathlineto{\pgfqpoint{6.678590in}{2.287506in}}%
\pgfpathlineto{\pgfqpoint{6.679320in}{2.287033in}}%
\pgfpathlineto{\pgfqpoint{6.679612in}{2.286041in}}%
\pgfpathlineto{\pgfqpoint{6.680486in}{2.286905in}}%
\pgfpathlineto{\pgfqpoint{6.681505in}{2.287784in}}%
\pgfpathlineto{\pgfqpoint{6.682232in}{2.288656in}}%
\pgfpathlineto{\pgfqpoint{6.682667in}{2.288533in}}%
\pgfpathlineto{\pgfqpoint{6.685131in}{2.287809in}}%
\pgfpathlineto{\pgfqpoint{6.685709in}{2.289285in}}%
\pgfpathlineto{\pgfqpoint{6.686287in}{2.288631in}}%
\pgfpathlineto{\pgfqpoint{6.686864in}{2.288141in}}%
\pgfpathlineto{\pgfqpoint{6.687297in}{2.288927in}}%
\pgfpathlineto{\pgfqpoint{6.687585in}{2.289597in}}%
\pgfpathlineto{\pgfqpoint{6.687873in}{2.288742in}}%
\pgfpathlineto{\pgfqpoint{6.687873in}{2.288742in}}%
\pgfpathlineto{\pgfqpoint{6.688593in}{2.288180in}}%
\pgfpathlineto{\pgfqpoint{6.688881in}{2.288912in}}%
\pgfpathlineto{\pgfqpoint{6.691608in}{2.290338in}}%
\pgfpathlineto{\pgfqpoint{6.691752in}{2.290147in}}%
\pgfpathlineto{\pgfqpoint{6.691895in}{2.289788in}}%
\pgfpathlineto{\pgfqpoint{6.692038in}{2.290214in}}%
\pgfpathlineto{\pgfqpoint{6.692038in}{2.290214in}}%
\pgfpathlineto{\pgfqpoint{6.693039in}{2.291058in}}%
\pgfpathlineto{\pgfqpoint{6.693325in}{2.290763in}}%
\pgfpathlineto{\pgfqpoint{6.694039in}{2.289694in}}%
\pgfpathlineto{\pgfqpoint{6.694467in}{2.291079in}}%
\pgfpathlineto{\pgfqpoint{6.695750in}{2.291208in}}%
\pgfpathlineto{\pgfqpoint{6.697314in}{2.291631in}}%
\pgfpathlineto{\pgfqpoint{6.698592in}{2.291839in}}%
\pgfpathlineto{\pgfqpoint{6.698733in}{2.291732in}}%
\pgfpathlineto{\pgfqpoint{6.699017in}{2.291252in}}%
\pgfpathlineto{\pgfqpoint{6.699442in}{2.291606in}}%
\pgfpathlineto{\pgfqpoint{6.700008in}{2.293288in}}%
\pgfpathlineto{\pgfqpoint{6.700574in}{2.291912in}}%
\pgfpathlineto{\pgfqpoint{6.700715in}{2.291184in}}%
\pgfpathlineto{\pgfqpoint{6.701421in}{2.291878in}}%
\pgfpathlineto{\pgfqpoint{6.701421in}{2.291878in}}%
\pgfpathlineto{\pgfqpoint{6.701845in}{2.293825in}}%
\pgfpathlineto{\pgfqpoint{6.702550in}{2.292129in}}%
\pgfpathlineto{\pgfqpoint{6.703536in}{2.293723in}}%
\pgfpathlineto{\pgfqpoint{6.703676in}{2.294322in}}%
\pgfpathlineto{\pgfqpoint{6.704098in}{2.293846in}}%
\pgfpathlineto{\pgfqpoint{6.704098in}{2.293846in}}%
\pgfpathlineto{\pgfqpoint{6.705082in}{2.293502in}}%
\pgfpathlineto{\pgfqpoint{6.705222in}{2.293631in}}%
\pgfpathlineto{\pgfqpoint{6.707185in}{2.295791in}}%
\pgfpathlineto{\pgfqpoint{6.707605in}{2.295392in}}%
\pgfpathlineto{\pgfqpoint{6.707745in}{2.295855in}}%
\pgfpathlineto{\pgfqpoint{6.707745in}{2.295855in}}%
\pgfpathlineto{\pgfqpoint{6.707884in}{2.296096in}}%
\pgfpathlineto{\pgfqpoint{6.708024in}{2.295766in}}%
\pgfpathlineto{\pgfqpoint{6.708024in}{2.295766in}}%
\pgfpathlineto{\pgfqpoint{6.708164in}{2.295348in}}%
\pgfpathlineto{\pgfqpoint{6.708164in}{2.295348in}}%
\pgfpathlineto{\pgfqpoint{6.708164in}{2.295348in}}%
\pgfpathlineto{\pgfqpoint{6.708723in}{2.297120in}}%
\pgfpathlineto{\pgfqpoint{6.709421in}{2.296296in}}%
\pgfpathlineto{\pgfqpoint{6.709700in}{2.295692in}}%
\pgfpathlineto{\pgfqpoint{6.709979in}{2.296607in}}%
\pgfpathlineto{\pgfqpoint{6.709979in}{2.296607in}}%
\pgfpathlineto{\pgfqpoint{6.711093in}{2.297926in}}%
\pgfpathlineto{\pgfqpoint{6.711371in}{2.297576in}}%
\pgfpathlineto{\pgfqpoint{6.713038in}{2.296047in}}%
\pgfpathlineto{\pgfqpoint{6.713177in}{2.296442in}}%
\pgfpathlineto{\pgfqpoint{6.713871in}{2.298722in}}%
\pgfpathlineto{\pgfqpoint{6.714425in}{2.297264in}}%
\pgfpathlineto{\pgfqpoint{6.714563in}{2.296599in}}%
\pgfpathlineto{\pgfqpoint{6.714840in}{2.297346in}}%
\pgfpathlineto{\pgfqpoint{6.714840in}{2.297346in}}%
\pgfpathlineto{\pgfqpoint{6.714978in}{2.297531in}}%
\pgfpathlineto{\pgfqpoint{6.714978in}{2.297531in}}%
\pgfpathlineto{\pgfqpoint{6.714978in}{2.297531in}}%
\pgfpathlineto{\pgfqpoint{6.715117in}{2.297084in}}%
\pgfpathlineto{\pgfqpoint{6.715393in}{2.297530in}}%
\pgfpathlineto{\pgfqpoint{6.715393in}{2.297530in}}%
\pgfpathlineto{\pgfqpoint{6.715808in}{2.300027in}}%
\pgfpathlineto{\pgfqpoint{6.716223in}{2.297609in}}%
\pgfpathlineto{\pgfqpoint{6.716637in}{2.296383in}}%
\pgfpathlineto{\pgfqpoint{6.717051in}{2.297414in}}%
\pgfpathlineto{\pgfqpoint{6.717051in}{2.297414in}}%
\pgfpathlineto{\pgfqpoint{6.717602in}{2.298901in}}%
\pgfpathlineto{\pgfqpoint{6.718429in}{2.298031in}}%
\pgfpathlineto{\pgfqpoint{6.718566in}{2.297780in}}%
\pgfpathlineto{\pgfqpoint{6.718566in}{2.297780in}}%
\pgfpathlineto{\pgfqpoint{6.718566in}{2.297780in}}%
\pgfpathlineto{\pgfqpoint{6.720216in}{2.299884in}}%
\pgfpathlineto{\pgfqpoint{6.721176in}{2.298939in}}%
\pgfpathlineto{\pgfqpoint{6.721313in}{2.299009in}}%
\pgfpathlineto{\pgfqpoint{6.722683in}{2.304020in}}%
\pgfpathlineto{\pgfqpoint{6.723230in}{2.302970in}}%
\pgfpathlineto{\pgfqpoint{6.723503in}{2.302791in}}%
\pgfpathlineto{\pgfqpoint{6.723503in}{2.302791in}}%
\pgfpathlineto{\pgfqpoint{6.723503in}{2.302791in}}%
\pgfpathlineto{\pgfqpoint{6.724868in}{2.305950in}}%
\pgfpathlineto{\pgfqpoint{6.725140in}{2.305648in}}%
\pgfpathlineto{\pgfqpoint{6.726638in}{2.303047in}}%
\pgfpathlineto{\pgfqpoint{6.727725in}{2.306156in}}%
\pgfpathlineto{\pgfqpoint{6.727861in}{2.305689in}}%
\pgfpathlineto{\pgfqpoint{6.728132in}{2.303617in}}%
\pgfpathlineto{\pgfqpoint{6.728539in}{2.305674in}}%
\pgfpathlineto{\pgfqpoint{6.728539in}{2.305674in}}%
\pgfpathlineto{\pgfqpoint{6.728675in}{2.305895in}}%
\pgfpathlineto{\pgfqpoint{6.728675in}{2.305895in}}%
\pgfpathlineto{\pgfqpoint{6.728675in}{2.305895in}}%
\pgfpathlineto{\pgfqpoint{6.728810in}{2.305503in}}%
\pgfpathlineto{\pgfqpoint{6.728810in}{2.305503in}}%
\pgfpathlineto{\pgfqpoint{6.728810in}{2.305503in}}%
\pgfpathlineto{\pgfqpoint{6.729488in}{2.308633in}}%
\pgfpathlineto{\pgfqpoint{6.729894in}{2.305845in}}%
\pgfpathlineto{\pgfqpoint{6.730300in}{2.302788in}}%
\pgfpathlineto{\pgfqpoint{6.730976in}{2.304987in}}%
\pgfpathlineto{\pgfqpoint{6.731246in}{2.305758in}}%
\pgfpathlineto{\pgfqpoint{6.731381in}{2.305057in}}%
\pgfpathlineto{\pgfqpoint{6.731381in}{2.305057in}}%
\pgfpathlineto{\pgfqpoint{6.731786in}{2.303357in}}%
\pgfpathlineto{\pgfqpoint{6.732191in}{2.304877in}}%
\pgfpathlineto{\pgfqpoint{6.732191in}{2.304877in}}%
\pgfpathlineto{\pgfqpoint{6.732730in}{2.306896in}}%
\pgfpathlineto{\pgfqpoint{6.733269in}{2.305633in}}%
\pgfpathlineto{\pgfqpoint{6.733404in}{2.305000in}}%
\pgfpathlineto{\pgfqpoint{6.733404in}{2.305000in}}%
\pgfusepath{stroke}%
\end{pgfscope}%
\begin{pgfscope}%
\pgfpathrectangle{\pgfqpoint{1.000000in}{0.300000in}}{\pgfqpoint{6.200000in}{2.400000in}} %
\pgfusepath{clip}%
\pgfsetbuttcap%
\pgfsetroundjoin%
\pgfsetlinewidth{0.501875pt}%
\definecolor{currentstroke}{rgb}{0.000000,0.000000,0.000000}%
\pgfsetstrokecolor{currentstroke}%
\pgfsetdash{{1.000000pt}{3.000000pt}}{0.000000pt}%
\pgfpathmoveto{\pgfqpoint{1.000000in}{0.300000in}}%
\pgfpathlineto{\pgfqpoint{1.000000in}{2.700000in}}%
\pgfusepath{stroke}%
\end{pgfscope}%
\begin{pgfscope}%
\pgfsetbuttcap%
\pgfsetroundjoin%
\definecolor{currentfill}{rgb}{0.000000,0.000000,0.000000}%
\pgfsetfillcolor{currentfill}%
\pgfsetlinewidth{0.501875pt}%
\definecolor{currentstroke}{rgb}{0.000000,0.000000,0.000000}%
\pgfsetstrokecolor{currentstroke}%
\pgfsetdash{}{0pt}%
\pgfsys@defobject{currentmarker}{\pgfqpoint{0.000000in}{0.000000in}}{\pgfqpoint{0.000000in}{0.055556in}}{%
\pgfpathmoveto{\pgfqpoint{0.000000in}{0.000000in}}%
\pgfpathlineto{\pgfqpoint{0.000000in}{0.055556in}}%
\pgfusepath{stroke,fill}%
}%
\begin{pgfscope}%
\pgfsys@transformshift{1.000000in}{0.300000in}%
\pgfsys@useobject{currentmarker}{}%
\end{pgfscope}%
\end{pgfscope}%
\begin{pgfscope}%
\pgfsetbuttcap%
\pgfsetroundjoin%
\definecolor{currentfill}{rgb}{0.000000,0.000000,0.000000}%
\pgfsetfillcolor{currentfill}%
\pgfsetlinewidth{0.501875pt}%
\definecolor{currentstroke}{rgb}{0.000000,0.000000,0.000000}%
\pgfsetstrokecolor{currentstroke}%
\pgfsetdash{}{0pt}%
\pgfsys@defobject{currentmarker}{\pgfqpoint{0.000000in}{-0.055556in}}{\pgfqpoint{0.000000in}{0.000000in}}{%
\pgfpathmoveto{\pgfqpoint{0.000000in}{0.000000in}}%
\pgfpathlineto{\pgfqpoint{0.000000in}{-0.055556in}}%
\pgfusepath{stroke,fill}%
}%
\begin{pgfscope}%
\pgfsys@transformshift{1.000000in}{2.700000in}%
\pgfsys@useobject{currentmarker}{}%
\end{pgfscope}%
\end{pgfscope}%
\begin{pgfscope}%
\pgftext[left,bottom,x=0.839506in,y=0.104024in,rotate=0.000000]{{\sffamily\fontsize{12.000000}{14.400000}\selectfont \(\displaystyle {10^{-1}}\)}}
%
\end{pgfscope}%
\begin{pgfscope}%
\pgfpathrectangle{\pgfqpoint{1.000000in}{0.300000in}}{\pgfqpoint{6.200000in}{2.400000in}} %
\pgfusepath{clip}%
\pgfsetbuttcap%
\pgfsetroundjoin%
\pgfsetlinewidth{0.501875pt}%
\definecolor{currentstroke}{rgb}{0.000000,0.000000,0.000000}%
\pgfsetstrokecolor{currentstroke}%
\pgfsetdash{{1.000000pt}{3.000000pt}}{0.000000pt}%
\pgfpathmoveto{\pgfqpoint{2.550000in}{0.300000in}}%
\pgfpathlineto{\pgfqpoint{2.550000in}{2.700000in}}%
\pgfusepath{stroke}%
\end{pgfscope}%
\begin{pgfscope}%
\pgfsetbuttcap%
\pgfsetroundjoin%
\definecolor{currentfill}{rgb}{0.000000,0.000000,0.000000}%
\pgfsetfillcolor{currentfill}%
\pgfsetlinewidth{0.501875pt}%
\definecolor{currentstroke}{rgb}{0.000000,0.000000,0.000000}%
\pgfsetstrokecolor{currentstroke}%
\pgfsetdash{}{0pt}%
\pgfsys@defobject{currentmarker}{\pgfqpoint{0.000000in}{0.000000in}}{\pgfqpoint{0.000000in}{0.055556in}}{%
\pgfpathmoveto{\pgfqpoint{0.000000in}{0.000000in}}%
\pgfpathlineto{\pgfqpoint{0.000000in}{0.055556in}}%
\pgfusepath{stroke,fill}%
}%
\begin{pgfscope}%
\pgfsys@transformshift{2.550000in}{0.300000in}%
\pgfsys@useobject{currentmarker}{}%
\end{pgfscope}%
\end{pgfscope}%
\begin{pgfscope}%
\pgfsetbuttcap%
\pgfsetroundjoin%
\definecolor{currentfill}{rgb}{0.000000,0.000000,0.000000}%
\pgfsetfillcolor{currentfill}%
\pgfsetlinewidth{0.501875pt}%
\definecolor{currentstroke}{rgb}{0.000000,0.000000,0.000000}%
\pgfsetstrokecolor{currentstroke}%
\pgfsetdash{}{0pt}%
\pgfsys@defobject{currentmarker}{\pgfqpoint{0.000000in}{-0.055556in}}{\pgfqpoint{0.000000in}{0.000000in}}{%
\pgfpathmoveto{\pgfqpoint{0.000000in}{0.000000in}}%
\pgfpathlineto{\pgfqpoint{0.000000in}{-0.055556in}}%
\pgfusepath{stroke,fill}%
}%
\begin{pgfscope}%
\pgfsys@transformshift{2.550000in}{2.700000in}%
\pgfsys@useobject{currentmarker}{}%
\end{pgfscope}%
\end{pgfscope}%
\begin{pgfscope}%
\pgftext[left,bottom,x=2.435417in,y=0.104024in,rotate=0.000000]{{\sffamily\fontsize{12.000000}{14.400000}\selectfont \(\displaystyle {10^{0}}\)}}
%
\end{pgfscope}%
\begin{pgfscope}%
\pgfpathrectangle{\pgfqpoint{1.000000in}{0.300000in}}{\pgfqpoint{6.200000in}{2.400000in}} %
\pgfusepath{clip}%
\pgfsetbuttcap%
\pgfsetroundjoin%
\pgfsetlinewidth{0.501875pt}%
\definecolor{currentstroke}{rgb}{0.000000,0.000000,0.000000}%
\pgfsetstrokecolor{currentstroke}%
\pgfsetdash{{1.000000pt}{3.000000pt}}{0.000000pt}%
\pgfpathmoveto{\pgfqpoint{4.100000in}{0.300000in}}%
\pgfpathlineto{\pgfqpoint{4.100000in}{2.700000in}}%
\pgfusepath{stroke}%
\end{pgfscope}%
\begin{pgfscope}%
\pgfsetbuttcap%
\pgfsetroundjoin%
\definecolor{currentfill}{rgb}{0.000000,0.000000,0.000000}%
\pgfsetfillcolor{currentfill}%
\pgfsetlinewidth{0.501875pt}%
\definecolor{currentstroke}{rgb}{0.000000,0.000000,0.000000}%
\pgfsetstrokecolor{currentstroke}%
\pgfsetdash{}{0pt}%
\pgfsys@defobject{currentmarker}{\pgfqpoint{0.000000in}{0.000000in}}{\pgfqpoint{0.000000in}{0.055556in}}{%
\pgfpathmoveto{\pgfqpoint{0.000000in}{0.000000in}}%
\pgfpathlineto{\pgfqpoint{0.000000in}{0.055556in}}%
\pgfusepath{stroke,fill}%
}%
\begin{pgfscope}%
\pgfsys@transformshift{4.100000in}{0.300000in}%
\pgfsys@useobject{currentmarker}{}%
\end{pgfscope}%
\end{pgfscope}%
\begin{pgfscope}%
\pgfsetbuttcap%
\pgfsetroundjoin%
\definecolor{currentfill}{rgb}{0.000000,0.000000,0.000000}%
\pgfsetfillcolor{currentfill}%
\pgfsetlinewidth{0.501875pt}%
\definecolor{currentstroke}{rgb}{0.000000,0.000000,0.000000}%
\pgfsetstrokecolor{currentstroke}%
\pgfsetdash{}{0pt}%
\pgfsys@defobject{currentmarker}{\pgfqpoint{0.000000in}{-0.055556in}}{\pgfqpoint{0.000000in}{0.000000in}}{%
\pgfpathmoveto{\pgfqpoint{0.000000in}{0.000000in}}%
\pgfpathlineto{\pgfqpoint{0.000000in}{-0.055556in}}%
\pgfusepath{stroke,fill}%
}%
\begin{pgfscope}%
\pgfsys@transformshift{4.100000in}{2.700000in}%
\pgfsys@useobject{currentmarker}{}%
\end{pgfscope}%
\end{pgfscope}%
\begin{pgfscope}%
\pgftext[left,bottom,x=3.985417in,y=0.104024in,rotate=0.000000]{{\sffamily\fontsize{12.000000}{14.400000}\selectfont \(\displaystyle {10^{1}}\)}}
%
\end{pgfscope}%
\begin{pgfscope}%
\pgfpathrectangle{\pgfqpoint{1.000000in}{0.300000in}}{\pgfqpoint{6.200000in}{2.400000in}} %
\pgfusepath{clip}%
\pgfsetbuttcap%
\pgfsetroundjoin%
\pgfsetlinewidth{0.501875pt}%
\definecolor{currentstroke}{rgb}{0.000000,0.000000,0.000000}%
\pgfsetstrokecolor{currentstroke}%
\pgfsetdash{{1.000000pt}{3.000000pt}}{0.000000pt}%
\pgfpathmoveto{\pgfqpoint{5.650000in}{0.300000in}}%
\pgfpathlineto{\pgfqpoint{5.650000in}{2.700000in}}%
\pgfusepath{stroke}%
\end{pgfscope}%
\begin{pgfscope}%
\pgfsetbuttcap%
\pgfsetroundjoin%
\definecolor{currentfill}{rgb}{0.000000,0.000000,0.000000}%
\pgfsetfillcolor{currentfill}%
\pgfsetlinewidth{0.501875pt}%
\definecolor{currentstroke}{rgb}{0.000000,0.000000,0.000000}%
\pgfsetstrokecolor{currentstroke}%
\pgfsetdash{}{0pt}%
\pgfsys@defobject{currentmarker}{\pgfqpoint{0.000000in}{0.000000in}}{\pgfqpoint{0.000000in}{0.055556in}}{%
\pgfpathmoveto{\pgfqpoint{0.000000in}{0.000000in}}%
\pgfpathlineto{\pgfqpoint{0.000000in}{0.055556in}}%
\pgfusepath{stroke,fill}%
}%
\begin{pgfscope}%
\pgfsys@transformshift{5.650000in}{0.300000in}%
\pgfsys@useobject{currentmarker}{}%
\end{pgfscope}%
\end{pgfscope}%
\begin{pgfscope}%
\pgfsetbuttcap%
\pgfsetroundjoin%
\definecolor{currentfill}{rgb}{0.000000,0.000000,0.000000}%
\pgfsetfillcolor{currentfill}%
\pgfsetlinewidth{0.501875pt}%
\definecolor{currentstroke}{rgb}{0.000000,0.000000,0.000000}%
\pgfsetstrokecolor{currentstroke}%
\pgfsetdash{}{0pt}%
\pgfsys@defobject{currentmarker}{\pgfqpoint{0.000000in}{-0.055556in}}{\pgfqpoint{0.000000in}{0.000000in}}{%
\pgfpathmoveto{\pgfqpoint{0.000000in}{0.000000in}}%
\pgfpathlineto{\pgfqpoint{0.000000in}{-0.055556in}}%
\pgfusepath{stroke,fill}%
}%
\begin{pgfscope}%
\pgfsys@transformshift{5.650000in}{2.700000in}%
\pgfsys@useobject{currentmarker}{}%
\end{pgfscope}%
\end{pgfscope}%
\begin{pgfscope}%
\pgftext[left,bottom,x=5.535417in,y=0.104024in,rotate=0.000000]{{\sffamily\fontsize{12.000000}{14.400000}\selectfont \(\displaystyle {10^{2}}\)}}
%
\end{pgfscope}%
\begin{pgfscope}%
\pgfpathrectangle{\pgfqpoint{1.000000in}{0.300000in}}{\pgfqpoint{6.200000in}{2.400000in}} %
\pgfusepath{clip}%
\pgfsetbuttcap%
\pgfsetroundjoin%
\pgfsetlinewidth{0.501875pt}%
\definecolor{currentstroke}{rgb}{0.000000,0.000000,0.000000}%
\pgfsetstrokecolor{currentstroke}%
\pgfsetdash{{1.000000pt}{3.000000pt}}{0.000000pt}%
\pgfpathmoveto{\pgfqpoint{7.200000in}{0.300000in}}%
\pgfpathlineto{\pgfqpoint{7.200000in}{2.700000in}}%
\pgfusepath{stroke}%
\end{pgfscope}%
\begin{pgfscope}%
\pgfsetbuttcap%
\pgfsetroundjoin%
\definecolor{currentfill}{rgb}{0.000000,0.000000,0.000000}%
\pgfsetfillcolor{currentfill}%
\pgfsetlinewidth{0.501875pt}%
\definecolor{currentstroke}{rgb}{0.000000,0.000000,0.000000}%
\pgfsetstrokecolor{currentstroke}%
\pgfsetdash{}{0pt}%
\pgfsys@defobject{currentmarker}{\pgfqpoint{0.000000in}{0.000000in}}{\pgfqpoint{0.000000in}{0.055556in}}{%
\pgfpathmoveto{\pgfqpoint{0.000000in}{0.000000in}}%
\pgfpathlineto{\pgfqpoint{0.000000in}{0.055556in}}%
\pgfusepath{stroke,fill}%
}%
\begin{pgfscope}%
\pgfsys@transformshift{7.200000in}{0.300000in}%
\pgfsys@useobject{currentmarker}{}%
\end{pgfscope}%
\end{pgfscope}%
\begin{pgfscope}%
\pgfsetbuttcap%
\pgfsetroundjoin%
\definecolor{currentfill}{rgb}{0.000000,0.000000,0.000000}%
\pgfsetfillcolor{currentfill}%
\pgfsetlinewidth{0.501875pt}%
\definecolor{currentstroke}{rgb}{0.000000,0.000000,0.000000}%
\pgfsetstrokecolor{currentstroke}%
\pgfsetdash{}{0pt}%
\pgfsys@defobject{currentmarker}{\pgfqpoint{0.000000in}{-0.055556in}}{\pgfqpoint{0.000000in}{0.000000in}}{%
\pgfpathmoveto{\pgfqpoint{0.000000in}{0.000000in}}%
\pgfpathlineto{\pgfqpoint{0.000000in}{-0.055556in}}%
\pgfusepath{stroke,fill}%
}%
\begin{pgfscope}%
\pgfsys@transformshift{7.200000in}{2.700000in}%
\pgfsys@useobject{currentmarker}{}%
\end{pgfscope}%
\end{pgfscope}%
\begin{pgfscope}%
\pgftext[left,bottom,x=7.085417in,y=0.104024in,rotate=0.000000]{{\sffamily\fontsize{12.000000}{14.400000}\selectfont \(\displaystyle {10^{3}}\)}}
%
\end{pgfscope}%
\begin{pgfscope}%
\pgfsetbuttcap%
\pgfsetroundjoin%
\definecolor{currentfill}{rgb}{0.000000,0.000000,0.000000}%
\pgfsetfillcolor{currentfill}%
\pgfsetlinewidth{0.501875pt}%
\definecolor{currentstroke}{rgb}{0.000000,0.000000,0.000000}%
\pgfsetstrokecolor{currentstroke}%
\pgfsetdash{}{0pt}%
\pgfsys@defobject{currentmarker}{\pgfqpoint{0.000000in}{0.000000in}}{\pgfqpoint{0.000000in}{0.027778in}}{%
\pgfpathmoveto{\pgfqpoint{0.000000in}{0.000000in}}%
\pgfpathlineto{\pgfqpoint{0.000000in}{0.027778in}}%
\pgfusepath{stroke,fill}%
}%
\begin{pgfscope}%
\pgfsys@transformshift{1.466596in}{0.300000in}%
\pgfsys@useobject{currentmarker}{}%
\end{pgfscope}%
\end{pgfscope}%
\begin{pgfscope}%
\pgfsetbuttcap%
\pgfsetroundjoin%
\definecolor{currentfill}{rgb}{0.000000,0.000000,0.000000}%
\pgfsetfillcolor{currentfill}%
\pgfsetlinewidth{0.501875pt}%
\definecolor{currentstroke}{rgb}{0.000000,0.000000,0.000000}%
\pgfsetstrokecolor{currentstroke}%
\pgfsetdash{}{0pt}%
\pgfsys@defobject{currentmarker}{\pgfqpoint{0.000000in}{-0.027778in}}{\pgfqpoint{0.000000in}{0.000000in}}{%
\pgfpathmoveto{\pgfqpoint{0.000000in}{0.000000in}}%
\pgfpathlineto{\pgfqpoint{0.000000in}{-0.027778in}}%
\pgfusepath{stroke,fill}%
}%
\begin{pgfscope}%
\pgfsys@transformshift{1.466596in}{2.700000in}%
\pgfsys@useobject{currentmarker}{}%
\end{pgfscope}%
\end{pgfscope}%
\begin{pgfscope}%
\pgfsetbuttcap%
\pgfsetroundjoin%
\definecolor{currentfill}{rgb}{0.000000,0.000000,0.000000}%
\pgfsetfillcolor{currentfill}%
\pgfsetlinewidth{0.501875pt}%
\definecolor{currentstroke}{rgb}{0.000000,0.000000,0.000000}%
\pgfsetstrokecolor{currentstroke}%
\pgfsetdash{}{0pt}%
\pgfsys@defobject{currentmarker}{\pgfqpoint{0.000000in}{0.000000in}}{\pgfqpoint{0.000000in}{0.027778in}}{%
\pgfpathmoveto{\pgfqpoint{0.000000in}{0.000000in}}%
\pgfpathlineto{\pgfqpoint{0.000000in}{0.027778in}}%
\pgfusepath{stroke,fill}%
}%
\begin{pgfscope}%
\pgfsys@transformshift{1.739538in}{0.300000in}%
\pgfsys@useobject{currentmarker}{}%
\end{pgfscope}%
\end{pgfscope}%
\begin{pgfscope}%
\pgfsetbuttcap%
\pgfsetroundjoin%
\definecolor{currentfill}{rgb}{0.000000,0.000000,0.000000}%
\pgfsetfillcolor{currentfill}%
\pgfsetlinewidth{0.501875pt}%
\definecolor{currentstroke}{rgb}{0.000000,0.000000,0.000000}%
\pgfsetstrokecolor{currentstroke}%
\pgfsetdash{}{0pt}%
\pgfsys@defobject{currentmarker}{\pgfqpoint{0.000000in}{-0.027778in}}{\pgfqpoint{0.000000in}{0.000000in}}{%
\pgfpathmoveto{\pgfqpoint{0.000000in}{0.000000in}}%
\pgfpathlineto{\pgfqpoint{0.000000in}{-0.027778in}}%
\pgfusepath{stroke,fill}%
}%
\begin{pgfscope}%
\pgfsys@transformshift{1.739538in}{2.700000in}%
\pgfsys@useobject{currentmarker}{}%
\end{pgfscope}%
\end{pgfscope}%
\begin{pgfscope}%
\pgfsetbuttcap%
\pgfsetroundjoin%
\definecolor{currentfill}{rgb}{0.000000,0.000000,0.000000}%
\pgfsetfillcolor{currentfill}%
\pgfsetlinewidth{0.501875pt}%
\definecolor{currentstroke}{rgb}{0.000000,0.000000,0.000000}%
\pgfsetstrokecolor{currentstroke}%
\pgfsetdash{}{0pt}%
\pgfsys@defobject{currentmarker}{\pgfqpoint{0.000000in}{0.000000in}}{\pgfqpoint{0.000000in}{0.027778in}}{%
\pgfpathmoveto{\pgfqpoint{0.000000in}{0.000000in}}%
\pgfpathlineto{\pgfqpoint{0.000000in}{0.027778in}}%
\pgfusepath{stroke,fill}%
}%
\begin{pgfscope}%
\pgfsys@transformshift{1.933193in}{0.300000in}%
\pgfsys@useobject{currentmarker}{}%
\end{pgfscope}%
\end{pgfscope}%
\begin{pgfscope}%
\pgfsetbuttcap%
\pgfsetroundjoin%
\definecolor{currentfill}{rgb}{0.000000,0.000000,0.000000}%
\pgfsetfillcolor{currentfill}%
\pgfsetlinewidth{0.501875pt}%
\definecolor{currentstroke}{rgb}{0.000000,0.000000,0.000000}%
\pgfsetstrokecolor{currentstroke}%
\pgfsetdash{}{0pt}%
\pgfsys@defobject{currentmarker}{\pgfqpoint{0.000000in}{-0.027778in}}{\pgfqpoint{0.000000in}{0.000000in}}{%
\pgfpathmoveto{\pgfqpoint{0.000000in}{0.000000in}}%
\pgfpathlineto{\pgfqpoint{0.000000in}{-0.027778in}}%
\pgfusepath{stroke,fill}%
}%
\begin{pgfscope}%
\pgfsys@transformshift{1.933193in}{2.700000in}%
\pgfsys@useobject{currentmarker}{}%
\end{pgfscope}%
\end{pgfscope}%
\begin{pgfscope}%
\pgfsetbuttcap%
\pgfsetroundjoin%
\definecolor{currentfill}{rgb}{0.000000,0.000000,0.000000}%
\pgfsetfillcolor{currentfill}%
\pgfsetlinewidth{0.501875pt}%
\definecolor{currentstroke}{rgb}{0.000000,0.000000,0.000000}%
\pgfsetstrokecolor{currentstroke}%
\pgfsetdash{}{0pt}%
\pgfsys@defobject{currentmarker}{\pgfqpoint{0.000000in}{0.000000in}}{\pgfqpoint{0.000000in}{0.027778in}}{%
\pgfpathmoveto{\pgfqpoint{0.000000in}{0.000000in}}%
\pgfpathlineto{\pgfqpoint{0.000000in}{0.027778in}}%
\pgfusepath{stroke,fill}%
}%
\begin{pgfscope}%
\pgfsys@transformshift{2.083404in}{0.300000in}%
\pgfsys@useobject{currentmarker}{}%
\end{pgfscope}%
\end{pgfscope}%
\begin{pgfscope}%
\pgfsetbuttcap%
\pgfsetroundjoin%
\definecolor{currentfill}{rgb}{0.000000,0.000000,0.000000}%
\pgfsetfillcolor{currentfill}%
\pgfsetlinewidth{0.501875pt}%
\definecolor{currentstroke}{rgb}{0.000000,0.000000,0.000000}%
\pgfsetstrokecolor{currentstroke}%
\pgfsetdash{}{0pt}%
\pgfsys@defobject{currentmarker}{\pgfqpoint{0.000000in}{-0.027778in}}{\pgfqpoint{0.000000in}{0.000000in}}{%
\pgfpathmoveto{\pgfqpoint{0.000000in}{0.000000in}}%
\pgfpathlineto{\pgfqpoint{0.000000in}{-0.027778in}}%
\pgfusepath{stroke,fill}%
}%
\begin{pgfscope}%
\pgfsys@transformshift{2.083404in}{2.700000in}%
\pgfsys@useobject{currentmarker}{}%
\end{pgfscope}%
\end{pgfscope}%
\begin{pgfscope}%
\pgfsetbuttcap%
\pgfsetroundjoin%
\definecolor{currentfill}{rgb}{0.000000,0.000000,0.000000}%
\pgfsetfillcolor{currentfill}%
\pgfsetlinewidth{0.501875pt}%
\definecolor{currentstroke}{rgb}{0.000000,0.000000,0.000000}%
\pgfsetstrokecolor{currentstroke}%
\pgfsetdash{}{0pt}%
\pgfsys@defobject{currentmarker}{\pgfqpoint{0.000000in}{0.000000in}}{\pgfqpoint{0.000000in}{0.027778in}}{%
\pgfpathmoveto{\pgfqpoint{0.000000in}{0.000000in}}%
\pgfpathlineto{\pgfqpoint{0.000000in}{0.027778in}}%
\pgfusepath{stroke,fill}%
}%
\begin{pgfscope}%
\pgfsys@transformshift{2.206134in}{0.300000in}%
\pgfsys@useobject{currentmarker}{}%
\end{pgfscope}%
\end{pgfscope}%
\begin{pgfscope}%
\pgfsetbuttcap%
\pgfsetroundjoin%
\definecolor{currentfill}{rgb}{0.000000,0.000000,0.000000}%
\pgfsetfillcolor{currentfill}%
\pgfsetlinewidth{0.501875pt}%
\definecolor{currentstroke}{rgb}{0.000000,0.000000,0.000000}%
\pgfsetstrokecolor{currentstroke}%
\pgfsetdash{}{0pt}%
\pgfsys@defobject{currentmarker}{\pgfqpoint{0.000000in}{-0.027778in}}{\pgfqpoint{0.000000in}{0.000000in}}{%
\pgfpathmoveto{\pgfqpoint{0.000000in}{0.000000in}}%
\pgfpathlineto{\pgfqpoint{0.000000in}{-0.027778in}}%
\pgfusepath{stroke,fill}%
}%
\begin{pgfscope}%
\pgfsys@transformshift{2.206134in}{2.700000in}%
\pgfsys@useobject{currentmarker}{}%
\end{pgfscope}%
\end{pgfscope}%
\begin{pgfscope}%
\pgfsetbuttcap%
\pgfsetroundjoin%
\definecolor{currentfill}{rgb}{0.000000,0.000000,0.000000}%
\pgfsetfillcolor{currentfill}%
\pgfsetlinewidth{0.501875pt}%
\definecolor{currentstroke}{rgb}{0.000000,0.000000,0.000000}%
\pgfsetstrokecolor{currentstroke}%
\pgfsetdash{}{0pt}%
\pgfsys@defobject{currentmarker}{\pgfqpoint{0.000000in}{0.000000in}}{\pgfqpoint{0.000000in}{0.027778in}}{%
\pgfpathmoveto{\pgfqpoint{0.000000in}{0.000000in}}%
\pgfpathlineto{\pgfqpoint{0.000000in}{0.027778in}}%
\pgfusepath{stroke,fill}%
}%
\begin{pgfscope}%
\pgfsys@transformshift{2.309902in}{0.300000in}%
\pgfsys@useobject{currentmarker}{}%
\end{pgfscope}%
\end{pgfscope}%
\begin{pgfscope}%
\pgfsetbuttcap%
\pgfsetroundjoin%
\definecolor{currentfill}{rgb}{0.000000,0.000000,0.000000}%
\pgfsetfillcolor{currentfill}%
\pgfsetlinewidth{0.501875pt}%
\definecolor{currentstroke}{rgb}{0.000000,0.000000,0.000000}%
\pgfsetstrokecolor{currentstroke}%
\pgfsetdash{}{0pt}%
\pgfsys@defobject{currentmarker}{\pgfqpoint{0.000000in}{-0.027778in}}{\pgfqpoint{0.000000in}{0.000000in}}{%
\pgfpathmoveto{\pgfqpoint{0.000000in}{0.000000in}}%
\pgfpathlineto{\pgfqpoint{0.000000in}{-0.027778in}}%
\pgfusepath{stroke,fill}%
}%
\begin{pgfscope}%
\pgfsys@transformshift{2.309902in}{2.700000in}%
\pgfsys@useobject{currentmarker}{}%
\end{pgfscope}%
\end{pgfscope}%
\begin{pgfscope}%
\pgfsetbuttcap%
\pgfsetroundjoin%
\definecolor{currentfill}{rgb}{0.000000,0.000000,0.000000}%
\pgfsetfillcolor{currentfill}%
\pgfsetlinewidth{0.501875pt}%
\definecolor{currentstroke}{rgb}{0.000000,0.000000,0.000000}%
\pgfsetstrokecolor{currentstroke}%
\pgfsetdash{}{0pt}%
\pgfsys@defobject{currentmarker}{\pgfqpoint{0.000000in}{0.000000in}}{\pgfqpoint{0.000000in}{0.027778in}}{%
\pgfpathmoveto{\pgfqpoint{0.000000in}{0.000000in}}%
\pgfpathlineto{\pgfqpoint{0.000000in}{0.027778in}}%
\pgfusepath{stroke,fill}%
}%
\begin{pgfscope}%
\pgfsys@transformshift{2.399789in}{0.300000in}%
\pgfsys@useobject{currentmarker}{}%
\end{pgfscope}%
\end{pgfscope}%
\begin{pgfscope}%
\pgfsetbuttcap%
\pgfsetroundjoin%
\definecolor{currentfill}{rgb}{0.000000,0.000000,0.000000}%
\pgfsetfillcolor{currentfill}%
\pgfsetlinewidth{0.501875pt}%
\definecolor{currentstroke}{rgb}{0.000000,0.000000,0.000000}%
\pgfsetstrokecolor{currentstroke}%
\pgfsetdash{}{0pt}%
\pgfsys@defobject{currentmarker}{\pgfqpoint{0.000000in}{-0.027778in}}{\pgfqpoint{0.000000in}{0.000000in}}{%
\pgfpathmoveto{\pgfqpoint{0.000000in}{0.000000in}}%
\pgfpathlineto{\pgfqpoint{0.000000in}{-0.027778in}}%
\pgfusepath{stroke,fill}%
}%
\begin{pgfscope}%
\pgfsys@transformshift{2.399789in}{2.700000in}%
\pgfsys@useobject{currentmarker}{}%
\end{pgfscope}%
\end{pgfscope}%
\begin{pgfscope}%
\pgfsetbuttcap%
\pgfsetroundjoin%
\definecolor{currentfill}{rgb}{0.000000,0.000000,0.000000}%
\pgfsetfillcolor{currentfill}%
\pgfsetlinewidth{0.501875pt}%
\definecolor{currentstroke}{rgb}{0.000000,0.000000,0.000000}%
\pgfsetstrokecolor{currentstroke}%
\pgfsetdash{}{0pt}%
\pgfsys@defobject{currentmarker}{\pgfqpoint{0.000000in}{0.000000in}}{\pgfqpoint{0.000000in}{0.027778in}}{%
\pgfpathmoveto{\pgfqpoint{0.000000in}{0.000000in}}%
\pgfpathlineto{\pgfqpoint{0.000000in}{0.027778in}}%
\pgfusepath{stroke,fill}%
}%
\begin{pgfscope}%
\pgfsys@transformshift{2.479076in}{0.300000in}%
\pgfsys@useobject{currentmarker}{}%
\end{pgfscope}%
\end{pgfscope}%
\begin{pgfscope}%
\pgfsetbuttcap%
\pgfsetroundjoin%
\definecolor{currentfill}{rgb}{0.000000,0.000000,0.000000}%
\pgfsetfillcolor{currentfill}%
\pgfsetlinewidth{0.501875pt}%
\definecolor{currentstroke}{rgb}{0.000000,0.000000,0.000000}%
\pgfsetstrokecolor{currentstroke}%
\pgfsetdash{}{0pt}%
\pgfsys@defobject{currentmarker}{\pgfqpoint{0.000000in}{-0.027778in}}{\pgfqpoint{0.000000in}{0.000000in}}{%
\pgfpathmoveto{\pgfqpoint{0.000000in}{0.000000in}}%
\pgfpathlineto{\pgfqpoint{0.000000in}{-0.027778in}}%
\pgfusepath{stroke,fill}%
}%
\begin{pgfscope}%
\pgfsys@transformshift{2.479076in}{2.700000in}%
\pgfsys@useobject{currentmarker}{}%
\end{pgfscope}%
\end{pgfscope}%
\begin{pgfscope}%
\pgfsetbuttcap%
\pgfsetroundjoin%
\definecolor{currentfill}{rgb}{0.000000,0.000000,0.000000}%
\pgfsetfillcolor{currentfill}%
\pgfsetlinewidth{0.501875pt}%
\definecolor{currentstroke}{rgb}{0.000000,0.000000,0.000000}%
\pgfsetstrokecolor{currentstroke}%
\pgfsetdash{}{0pt}%
\pgfsys@defobject{currentmarker}{\pgfqpoint{0.000000in}{0.000000in}}{\pgfqpoint{0.000000in}{0.027778in}}{%
\pgfpathmoveto{\pgfqpoint{0.000000in}{0.000000in}}%
\pgfpathlineto{\pgfqpoint{0.000000in}{0.027778in}}%
\pgfusepath{stroke,fill}%
}%
\begin{pgfscope}%
\pgfsys@transformshift{3.016596in}{0.300000in}%
\pgfsys@useobject{currentmarker}{}%
\end{pgfscope}%
\end{pgfscope}%
\begin{pgfscope}%
\pgfsetbuttcap%
\pgfsetroundjoin%
\definecolor{currentfill}{rgb}{0.000000,0.000000,0.000000}%
\pgfsetfillcolor{currentfill}%
\pgfsetlinewidth{0.501875pt}%
\definecolor{currentstroke}{rgb}{0.000000,0.000000,0.000000}%
\pgfsetstrokecolor{currentstroke}%
\pgfsetdash{}{0pt}%
\pgfsys@defobject{currentmarker}{\pgfqpoint{0.000000in}{-0.027778in}}{\pgfqpoint{0.000000in}{0.000000in}}{%
\pgfpathmoveto{\pgfqpoint{0.000000in}{0.000000in}}%
\pgfpathlineto{\pgfqpoint{0.000000in}{-0.027778in}}%
\pgfusepath{stroke,fill}%
}%
\begin{pgfscope}%
\pgfsys@transformshift{3.016596in}{2.700000in}%
\pgfsys@useobject{currentmarker}{}%
\end{pgfscope}%
\end{pgfscope}%
\begin{pgfscope}%
\pgfsetbuttcap%
\pgfsetroundjoin%
\definecolor{currentfill}{rgb}{0.000000,0.000000,0.000000}%
\pgfsetfillcolor{currentfill}%
\pgfsetlinewidth{0.501875pt}%
\definecolor{currentstroke}{rgb}{0.000000,0.000000,0.000000}%
\pgfsetstrokecolor{currentstroke}%
\pgfsetdash{}{0pt}%
\pgfsys@defobject{currentmarker}{\pgfqpoint{0.000000in}{0.000000in}}{\pgfqpoint{0.000000in}{0.027778in}}{%
\pgfpathmoveto{\pgfqpoint{0.000000in}{0.000000in}}%
\pgfpathlineto{\pgfqpoint{0.000000in}{0.027778in}}%
\pgfusepath{stroke,fill}%
}%
\begin{pgfscope}%
\pgfsys@transformshift{3.289538in}{0.300000in}%
\pgfsys@useobject{currentmarker}{}%
\end{pgfscope}%
\end{pgfscope}%
\begin{pgfscope}%
\pgfsetbuttcap%
\pgfsetroundjoin%
\definecolor{currentfill}{rgb}{0.000000,0.000000,0.000000}%
\pgfsetfillcolor{currentfill}%
\pgfsetlinewidth{0.501875pt}%
\definecolor{currentstroke}{rgb}{0.000000,0.000000,0.000000}%
\pgfsetstrokecolor{currentstroke}%
\pgfsetdash{}{0pt}%
\pgfsys@defobject{currentmarker}{\pgfqpoint{0.000000in}{-0.027778in}}{\pgfqpoint{0.000000in}{0.000000in}}{%
\pgfpathmoveto{\pgfqpoint{0.000000in}{0.000000in}}%
\pgfpathlineto{\pgfqpoint{0.000000in}{-0.027778in}}%
\pgfusepath{stroke,fill}%
}%
\begin{pgfscope}%
\pgfsys@transformshift{3.289538in}{2.700000in}%
\pgfsys@useobject{currentmarker}{}%
\end{pgfscope}%
\end{pgfscope}%
\begin{pgfscope}%
\pgfsetbuttcap%
\pgfsetroundjoin%
\definecolor{currentfill}{rgb}{0.000000,0.000000,0.000000}%
\pgfsetfillcolor{currentfill}%
\pgfsetlinewidth{0.501875pt}%
\definecolor{currentstroke}{rgb}{0.000000,0.000000,0.000000}%
\pgfsetstrokecolor{currentstroke}%
\pgfsetdash{}{0pt}%
\pgfsys@defobject{currentmarker}{\pgfqpoint{0.000000in}{0.000000in}}{\pgfqpoint{0.000000in}{0.027778in}}{%
\pgfpathmoveto{\pgfqpoint{0.000000in}{0.000000in}}%
\pgfpathlineto{\pgfqpoint{0.000000in}{0.027778in}}%
\pgfusepath{stroke,fill}%
}%
\begin{pgfscope}%
\pgfsys@transformshift{3.483193in}{0.300000in}%
\pgfsys@useobject{currentmarker}{}%
\end{pgfscope}%
\end{pgfscope}%
\begin{pgfscope}%
\pgfsetbuttcap%
\pgfsetroundjoin%
\definecolor{currentfill}{rgb}{0.000000,0.000000,0.000000}%
\pgfsetfillcolor{currentfill}%
\pgfsetlinewidth{0.501875pt}%
\definecolor{currentstroke}{rgb}{0.000000,0.000000,0.000000}%
\pgfsetstrokecolor{currentstroke}%
\pgfsetdash{}{0pt}%
\pgfsys@defobject{currentmarker}{\pgfqpoint{0.000000in}{-0.027778in}}{\pgfqpoint{0.000000in}{0.000000in}}{%
\pgfpathmoveto{\pgfqpoint{0.000000in}{0.000000in}}%
\pgfpathlineto{\pgfqpoint{0.000000in}{-0.027778in}}%
\pgfusepath{stroke,fill}%
}%
\begin{pgfscope}%
\pgfsys@transformshift{3.483193in}{2.700000in}%
\pgfsys@useobject{currentmarker}{}%
\end{pgfscope}%
\end{pgfscope}%
\begin{pgfscope}%
\pgfsetbuttcap%
\pgfsetroundjoin%
\definecolor{currentfill}{rgb}{0.000000,0.000000,0.000000}%
\pgfsetfillcolor{currentfill}%
\pgfsetlinewidth{0.501875pt}%
\definecolor{currentstroke}{rgb}{0.000000,0.000000,0.000000}%
\pgfsetstrokecolor{currentstroke}%
\pgfsetdash{}{0pt}%
\pgfsys@defobject{currentmarker}{\pgfqpoint{0.000000in}{0.000000in}}{\pgfqpoint{0.000000in}{0.027778in}}{%
\pgfpathmoveto{\pgfqpoint{0.000000in}{0.000000in}}%
\pgfpathlineto{\pgfqpoint{0.000000in}{0.027778in}}%
\pgfusepath{stroke,fill}%
}%
\begin{pgfscope}%
\pgfsys@transformshift{3.633404in}{0.300000in}%
\pgfsys@useobject{currentmarker}{}%
\end{pgfscope}%
\end{pgfscope}%
\begin{pgfscope}%
\pgfsetbuttcap%
\pgfsetroundjoin%
\definecolor{currentfill}{rgb}{0.000000,0.000000,0.000000}%
\pgfsetfillcolor{currentfill}%
\pgfsetlinewidth{0.501875pt}%
\definecolor{currentstroke}{rgb}{0.000000,0.000000,0.000000}%
\pgfsetstrokecolor{currentstroke}%
\pgfsetdash{}{0pt}%
\pgfsys@defobject{currentmarker}{\pgfqpoint{0.000000in}{-0.027778in}}{\pgfqpoint{0.000000in}{0.000000in}}{%
\pgfpathmoveto{\pgfqpoint{0.000000in}{0.000000in}}%
\pgfpathlineto{\pgfqpoint{0.000000in}{-0.027778in}}%
\pgfusepath{stroke,fill}%
}%
\begin{pgfscope}%
\pgfsys@transformshift{3.633404in}{2.700000in}%
\pgfsys@useobject{currentmarker}{}%
\end{pgfscope}%
\end{pgfscope}%
\begin{pgfscope}%
\pgfsetbuttcap%
\pgfsetroundjoin%
\definecolor{currentfill}{rgb}{0.000000,0.000000,0.000000}%
\pgfsetfillcolor{currentfill}%
\pgfsetlinewidth{0.501875pt}%
\definecolor{currentstroke}{rgb}{0.000000,0.000000,0.000000}%
\pgfsetstrokecolor{currentstroke}%
\pgfsetdash{}{0pt}%
\pgfsys@defobject{currentmarker}{\pgfqpoint{0.000000in}{0.000000in}}{\pgfqpoint{0.000000in}{0.027778in}}{%
\pgfpathmoveto{\pgfqpoint{0.000000in}{0.000000in}}%
\pgfpathlineto{\pgfqpoint{0.000000in}{0.027778in}}%
\pgfusepath{stroke,fill}%
}%
\begin{pgfscope}%
\pgfsys@transformshift{3.756134in}{0.300000in}%
\pgfsys@useobject{currentmarker}{}%
\end{pgfscope}%
\end{pgfscope}%
\begin{pgfscope}%
\pgfsetbuttcap%
\pgfsetroundjoin%
\definecolor{currentfill}{rgb}{0.000000,0.000000,0.000000}%
\pgfsetfillcolor{currentfill}%
\pgfsetlinewidth{0.501875pt}%
\definecolor{currentstroke}{rgb}{0.000000,0.000000,0.000000}%
\pgfsetstrokecolor{currentstroke}%
\pgfsetdash{}{0pt}%
\pgfsys@defobject{currentmarker}{\pgfqpoint{0.000000in}{-0.027778in}}{\pgfqpoint{0.000000in}{0.000000in}}{%
\pgfpathmoveto{\pgfqpoint{0.000000in}{0.000000in}}%
\pgfpathlineto{\pgfqpoint{0.000000in}{-0.027778in}}%
\pgfusepath{stroke,fill}%
}%
\begin{pgfscope}%
\pgfsys@transformshift{3.756134in}{2.700000in}%
\pgfsys@useobject{currentmarker}{}%
\end{pgfscope}%
\end{pgfscope}%
\begin{pgfscope}%
\pgfsetbuttcap%
\pgfsetroundjoin%
\definecolor{currentfill}{rgb}{0.000000,0.000000,0.000000}%
\pgfsetfillcolor{currentfill}%
\pgfsetlinewidth{0.501875pt}%
\definecolor{currentstroke}{rgb}{0.000000,0.000000,0.000000}%
\pgfsetstrokecolor{currentstroke}%
\pgfsetdash{}{0pt}%
\pgfsys@defobject{currentmarker}{\pgfqpoint{0.000000in}{0.000000in}}{\pgfqpoint{0.000000in}{0.027778in}}{%
\pgfpathmoveto{\pgfqpoint{0.000000in}{0.000000in}}%
\pgfpathlineto{\pgfqpoint{0.000000in}{0.027778in}}%
\pgfusepath{stroke,fill}%
}%
\begin{pgfscope}%
\pgfsys@transformshift{3.859902in}{0.300000in}%
\pgfsys@useobject{currentmarker}{}%
\end{pgfscope}%
\end{pgfscope}%
\begin{pgfscope}%
\pgfsetbuttcap%
\pgfsetroundjoin%
\definecolor{currentfill}{rgb}{0.000000,0.000000,0.000000}%
\pgfsetfillcolor{currentfill}%
\pgfsetlinewidth{0.501875pt}%
\definecolor{currentstroke}{rgb}{0.000000,0.000000,0.000000}%
\pgfsetstrokecolor{currentstroke}%
\pgfsetdash{}{0pt}%
\pgfsys@defobject{currentmarker}{\pgfqpoint{0.000000in}{-0.027778in}}{\pgfqpoint{0.000000in}{0.000000in}}{%
\pgfpathmoveto{\pgfqpoint{0.000000in}{0.000000in}}%
\pgfpathlineto{\pgfqpoint{0.000000in}{-0.027778in}}%
\pgfusepath{stroke,fill}%
}%
\begin{pgfscope}%
\pgfsys@transformshift{3.859902in}{2.700000in}%
\pgfsys@useobject{currentmarker}{}%
\end{pgfscope}%
\end{pgfscope}%
\begin{pgfscope}%
\pgfsetbuttcap%
\pgfsetroundjoin%
\definecolor{currentfill}{rgb}{0.000000,0.000000,0.000000}%
\pgfsetfillcolor{currentfill}%
\pgfsetlinewidth{0.501875pt}%
\definecolor{currentstroke}{rgb}{0.000000,0.000000,0.000000}%
\pgfsetstrokecolor{currentstroke}%
\pgfsetdash{}{0pt}%
\pgfsys@defobject{currentmarker}{\pgfqpoint{0.000000in}{0.000000in}}{\pgfqpoint{0.000000in}{0.027778in}}{%
\pgfpathmoveto{\pgfqpoint{0.000000in}{0.000000in}}%
\pgfpathlineto{\pgfqpoint{0.000000in}{0.027778in}}%
\pgfusepath{stroke,fill}%
}%
\begin{pgfscope}%
\pgfsys@transformshift{3.949789in}{0.300000in}%
\pgfsys@useobject{currentmarker}{}%
\end{pgfscope}%
\end{pgfscope}%
\begin{pgfscope}%
\pgfsetbuttcap%
\pgfsetroundjoin%
\definecolor{currentfill}{rgb}{0.000000,0.000000,0.000000}%
\pgfsetfillcolor{currentfill}%
\pgfsetlinewidth{0.501875pt}%
\definecolor{currentstroke}{rgb}{0.000000,0.000000,0.000000}%
\pgfsetstrokecolor{currentstroke}%
\pgfsetdash{}{0pt}%
\pgfsys@defobject{currentmarker}{\pgfqpoint{0.000000in}{-0.027778in}}{\pgfqpoint{0.000000in}{0.000000in}}{%
\pgfpathmoveto{\pgfqpoint{0.000000in}{0.000000in}}%
\pgfpathlineto{\pgfqpoint{0.000000in}{-0.027778in}}%
\pgfusepath{stroke,fill}%
}%
\begin{pgfscope}%
\pgfsys@transformshift{3.949789in}{2.700000in}%
\pgfsys@useobject{currentmarker}{}%
\end{pgfscope}%
\end{pgfscope}%
\begin{pgfscope}%
\pgfsetbuttcap%
\pgfsetroundjoin%
\definecolor{currentfill}{rgb}{0.000000,0.000000,0.000000}%
\pgfsetfillcolor{currentfill}%
\pgfsetlinewidth{0.501875pt}%
\definecolor{currentstroke}{rgb}{0.000000,0.000000,0.000000}%
\pgfsetstrokecolor{currentstroke}%
\pgfsetdash{}{0pt}%
\pgfsys@defobject{currentmarker}{\pgfqpoint{0.000000in}{0.000000in}}{\pgfqpoint{0.000000in}{0.027778in}}{%
\pgfpathmoveto{\pgfqpoint{0.000000in}{0.000000in}}%
\pgfpathlineto{\pgfqpoint{0.000000in}{0.027778in}}%
\pgfusepath{stroke,fill}%
}%
\begin{pgfscope}%
\pgfsys@transformshift{4.029076in}{0.300000in}%
\pgfsys@useobject{currentmarker}{}%
\end{pgfscope}%
\end{pgfscope}%
\begin{pgfscope}%
\pgfsetbuttcap%
\pgfsetroundjoin%
\definecolor{currentfill}{rgb}{0.000000,0.000000,0.000000}%
\pgfsetfillcolor{currentfill}%
\pgfsetlinewidth{0.501875pt}%
\definecolor{currentstroke}{rgb}{0.000000,0.000000,0.000000}%
\pgfsetstrokecolor{currentstroke}%
\pgfsetdash{}{0pt}%
\pgfsys@defobject{currentmarker}{\pgfqpoint{0.000000in}{-0.027778in}}{\pgfqpoint{0.000000in}{0.000000in}}{%
\pgfpathmoveto{\pgfqpoint{0.000000in}{0.000000in}}%
\pgfpathlineto{\pgfqpoint{0.000000in}{-0.027778in}}%
\pgfusepath{stroke,fill}%
}%
\begin{pgfscope}%
\pgfsys@transformshift{4.029076in}{2.700000in}%
\pgfsys@useobject{currentmarker}{}%
\end{pgfscope}%
\end{pgfscope}%
\begin{pgfscope}%
\pgfsetbuttcap%
\pgfsetroundjoin%
\definecolor{currentfill}{rgb}{0.000000,0.000000,0.000000}%
\pgfsetfillcolor{currentfill}%
\pgfsetlinewidth{0.501875pt}%
\definecolor{currentstroke}{rgb}{0.000000,0.000000,0.000000}%
\pgfsetstrokecolor{currentstroke}%
\pgfsetdash{}{0pt}%
\pgfsys@defobject{currentmarker}{\pgfqpoint{0.000000in}{0.000000in}}{\pgfqpoint{0.000000in}{0.027778in}}{%
\pgfpathmoveto{\pgfqpoint{0.000000in}{0.000000in}}%
\pgfpathlineto{\pgfqpoint{0.000000in}{0.027778in}}%
\pgfusepath{stroke,fill}%
}%
\begin{pgfscope}%
\pgfsys@transformshift{4.566596in}{0.300000in}%
\pgfsys@useobject{currentmarker}{}%
\end{pgfscope}%
\end{pgfscope}%
\begin{pgfscope}%
\pgfsetbuttcap%
\pgfsetroundjoin%
\definecolor{currentfill}{rgb}{0.000000,0.000000,0.000000}%
\pgfsetfillcolor{currentfill}%
\pgfsetlinewidth{0.501875pt}%
\definecolor{currentstroke}{rgb}{0.000000,0.000000,0.000000}%
\pgfsetstrokecolor{currentstroke}%
\pgfsetdash{}{0pt}%
\pgfsys@defobject{currentmarker}{\pgfqpoint{0.000000in}{-0.027778in}}{\pgfqpoint{0.000000in}{0.000000in}}{%
\pgfpathmoveto{\pgfqpoint{0.000000in}{0.000000in}}%
\pgfpathlineto{\pgfqpoint{0.000000in}{-0.027778in}}%
\pgfusepath{stroke,fill}%
}%
\begin{pgfscope}%
\pgfsys@transformshift{4.566596in}{2.700000in}%
\pgfsys@useobject{currentmarker}{}%
\end{pgfscope}%
\end{pgfscope}%
\begin{pgfscope}%
\pgfsetbuttcap%
\pgfsetroundjoin%
\definecolor{currentfill}{rgb}{0.000000,0.000000,0.000000}%
\pgfsetfillcolor{currentfill}%
\pgfsetlinewidth{0.501875pt}%
\definecolor{currentstroke}{rgb}{0.000000,0.000000,0.000000}%
\pgfsetstrokecolor{currentstroke}%
\pgfsetdash{}{0pt}%
\pgfsys@defobject{currentmarker}{\pgfqpoint{0.000000in}{0.000000in}}{\pgfqpoint{0.000000in}{0.027778in}}{%
\pgfpathmoveto{\pgfqpoint{0.000000in}{0.000000in}}%
\pgfpathlineto{\pgfqpoint{0.000000in}{0.027778in}}%
\pgfusepath{stroke,fill}%
}%
\begin{pgfscope}%
\pgfsys@transformshift{4.839538in}{0.300000in}%
\pgfsys@useobject{currentmarker}{}%
\end{pgfscope}%
\end{pgfscope}%
\begin{pgfscope}%
\pgfsetbuttcap%
\pgfsetroundjoin%
\definecolor{currentfill}{rgb}{0.000000,0.000000,0.000000}%
\pgfsetfillcolor{currentfill}%
\pgfsetlinewidth{0.501875pt}%
\definecolor{currentstroke}{rgb}{0.000000,0.000000,0.000000}%
\pgfsetstrokecolor{currentstroke}%
\pgfsetdash{}{0pt}%
\pgfsys@defobject{currentmarker}{\pgfqpoint{0.000000in}{-0.027778in}}{\pgfqpoint{0.000000in}{0.000000in}}{%
\pgfpathmoveto{\pgfqpoint{0.000000in}{0.000000in}}%
\pgfpathlineto{\pgfqpoint{0.000000in}{-0.027778in}}%
\pgfusepath{stroke,fill}%
}%
\begin{pgfscope}%
\pgfsys@transformshift{4.839538in}{2.700000in}%
\pgfsys@useobject{currentmarker}{}%
\end{pgfscope}%
\end{pgfscope}%
\begin{pgfscope}%
\pgfsetbuttcap%
\pgfsetroundjoin%
\definecolor{currentfill}{rgb}{0.000000,0.000000,0.000000}%
\pgfsetfillcolor{currentfill}%
\pgfsetlinewidth{0.501875pt}%
\definecolor{currentstroke}{rgb}{0.000000,0.000000,0.000000}%
\pgfsetstrokecolor{currentstroke}%
\pgfsetdash{}{0pt}%
\pgfsys@defobject{currentmarker}{\pgfqpoint{0.000000in}{0.000000in}}{\pgfqpoint{0.000000in}{0.027778in}}{%
\pgfpathmoveto{\pgfqpoint{0.000000in}{0.000000in}}%
\pgfpathlineto{\pgfqpoint{0.000000in}{0.027778in}}%
\pgfusepath{stroke,fill}%
}%
\begin{pgfscope}%
\pgfsys@transformshift{5.033193in}{0.300000in}%
\pgfsys@useobject{currentmarker}{}%
\end{pgfscope}%
\end{pgfscope}%
\begin{pgfscope}%
\pgfsetbuttcap%
\pgfsetroundjoin%
\definecolor{currentfill}{rgb}{0.000000,0.000000,0.000000}%
\pgfsetfillcolor{currentfill}%
\pgfsetlinewidth{0.501875pt}%
\definecolor{currentstroke}{rgb}{0.000000,0.000000,0.000000}%
\pgfsetstrokecolor{currentstroke}%
\pgfsetdash{}{0pt}%
\pgfsys@defobject{currentmarker}{\pgfqpoint{0.000000in}{-0.027778in}}{\pgfqpoint{0.000000in}{0.000000in}}{%
\pgfpathmoveto{\pgfqpoint{0.000000in}{0.000000in}}%
\pgfpathlineto{\pgfqpoint{0.000000in}{-0.027778in}}%
\pgfusepath{stroke,fill}%
}%
\begin{pgfscope}%
\pgfsys@transformshift{5.033193in}{2.700000in}%
\pgfsys@useobject{currentmarker}{}%
\end{pgfscope}%
\end{pgfscope}%
\begin{pgfscope}%
\pgfsetbuttcap%
\pgfsetroundjoin%
\definecolor{currentfill}{rgb}{0.000000,0.000000,0.000000}%
\pgfsetfillcolor{currentfill}%
\pgfsetlinewidth{0.501875pt}%
\definecolor{currentstroke}{rgb}{0.000000,0.000000,0.000000}%
\pgfsetstrokecolor{currentstroke}%
\pgfsetdash{}{0pt}%
\pgfsys@defobject{currentmarker}{\pgfqpoint{0.000000in}{0.000000in}}{\pgfqpoint{0.000000in}{0.027778in}}{%
\pgfpathmoveto{\pgfqpoint{0.000000in}{0.000000in}}%
\pgfpathlineto{\pgfqpoint{0.000000in}{0.027778in}}%
\pgfusepath{stroke,fill}%
}%
\begin{pgfscope}%
\pgfsys@transformshift{5.183404in}{0.300000in}%
\pgfsys@useobject{currentmarker}{}%
\end{pgfscope}%
\end{pgfscope}%
\begin{pgfscope}%
\pgfsetbuttcap%
\pgfsetroundjoin%
\definecolor{currentfill}{rgb}{0.000000,0.000000,0.000000}%
\pgfsetfillcolor{currentfill}%
\pgfsetlinewidth{0.501875pt}%
\definecolor{currentstroke}{rgb}{0.000000,0.000000,0.000000}%
\pgfsetstrokecolor{currentstroke}%
\pgfsetdash{}{0pt}%
\pgfsys@defobject{currentmarker}{\pgfqpoint{0.000000in}{-0.027778in}}{\pgfqpoint{0.000000in}{0.000000in}}{%
\pgfpathmoveto{\pgfqpoint{0.000000in}{0.000000in}}%
\pgfpathlineto{\pgfqpoint{0.000000in}{-0.027778in}}%
\pgfusepath{stroke,fill}%
}%
\begin{pgfscope}%
\pgfsys@transformshift{5.183404in}{2.700000in}%
\pgfsys@useobject{currentmarker}{}%
\end{pgfscope}%
\end{pgfscope}%
\begin{pgfscope}%
\pgfsetbuttcap%
\pgfsetroundjoin%
\definecolor{currentfill}{rgb}{0.000000,0.000000,0.000000}%
\pgfsetfillcolor{currentfill}%
\pgfsetlinewidth{0.501875pt}%
\definecolor{currentstroke}{rgb}{0.000000,0.000000,0.000000}%
\pgfsetstrokecolor{currentstroke}%
\pgfsetdash{}{0pt}%
\pgfsys@defobject{currentmarker}{\pgfqpoint{0.000000in}{0.000000in}}{\pgfqpoint{0.000000in}{0.027778in}}{%
\pgfpathmoveto{\pgfqpoint{0.000000in}{0.000000in}}%
\pgfpathlineto{\pgfqpoint{0.000000in}{0.027778in}}%
\pgfusepath{stroke,fill}%
}%
\begin{pgfscope}%
\pgfsys@transformshift{5.306134in}{0.300000in}%
\pgfsys@useobject{currentmarker}{}%
\end{pgfscope}%
\end{pgfscope}%
\begin{pgfscope}%
\pgfsetbuttcap%
\pgfsetroundjoin%
\definecolor{currentfill}{rgb}{0.000000,0.000000,0.000000}%
\pgfsetfillcolor{currentfill}%
\pgfsetlinewidth{0.501875pt}%
\definecolor{currentstroke}{rgb}{0.000000,0.000000,0.000000}%
\pgfsetstrokecolor{currentstroke}%
\pgfsetdash{}{0pt}%
\pgfsys@defobject{currentmarker}{\pgfqpoint{0.000000in}{-0.027778in}}{\pgfqpoint{0.000000in}{0.000000in}}{%
\pgfpathmoveto{\pgfqpoint{0.000000in}{0.000000in}}%
\pgfpathlineto{\pgfqpoint{0.000000in}{-0.027778in}}%
\pgfusepath{stroke,fill}%
}%
\begin{pgfscope}%
\pgfsys@transformshift{5.306134in}{2.700000in}%
\pgfsys@useobject{currentmarker}{}%
\end{pgfscope}%
\end{pgfscope}%
\begin{pgfscope}%
\pgfsetbuttcap%
\pgfsetroundjoin%
\definecolor{currentfill}{rgb}{0.000000,0.000000,0.000000}%
\pgfsetfillcolor{currentfill}%
\pgfsetlinewidth{0.501875pt}%
\definecolor{currentstroke}{rgb}{0.000000,0.000000,0.000000}%
\pgfsetstrokecolor{currentstroke}%
\pgfsetdash{}{0pt}%
\pgfsys@defobject{currentmarker}{\pgfqpoint{0.000000in}{0.000000in}}{\pgfqpoint{0.000000in}{0.027778in}}{%
\pgfpathmoveto{\pgfqpoint{0.000000in}{0.000000in}}%
\pgfpathlineto{\pgfqpoint{0.000000in}{0.027778in}}%
\pgfusepath{stroke,fill}%
}%
\begin{pgfscope}%
\pgfsys@transformshift{5.409902in}{0.300000in}%
\pgfsys@useobject{currentmarker}{}%
\end{pgfscope}%
\end{pgfscope}%
\begin{pgfscope}%
\pgfsetbuttcap%
\pgfsetroundjoin%
\definecolor{currentfill}{rgb}{0.000000,0.000000,0.000000}%
\pgfsetfillcolor{currentfill}%
\pgfsetlinewidth{0.501875pt}%
\definecolor{currentstroke}{rgb}{0.000000,0.000000,0.000000}%
\pgfsetstrokecolor{currentstroke}%
\pgfsetdash{}{0pt}%
\pgfsys@defobject{currentmarker}{\pgfqpoint{0.000000in}{-0.027778in}}{\pgfqpoint{0.000000in}{0.000000in}}{%
\pgfpathmoveto{\pgfqpoint{0.000000in}{0.000000in}}%
\pgfpathlineto{\pgfqpoint{0.000000in}{-0.027778in}}%
\pgfusepath{stroke,fill}%
}%
\begin{pgfscope}%
\pgfsys@transformshift{5.409902in}{2.700000in}%
\pgfsys@useobject{currentmarker}{}%
\end{pgfscope}%
\end{pgfscope}%
\begin{pgfscope}%
\pgfsetbuttcap%
\pgfsetroundjoin%
\definecolor{currentfill}{rgb}{0.000000,0.000000,0.000000}%
\pgfsetfillcolor{currentfill}%
\pgfsetlinewidth{0.501875pt}%
\definecolor{currentstroke}{rgb}{0.000000,0.000000,0.000000}%
\pgfsetstrokecolor{currentstroke}%
\pgfsetdash{}{0pt}%
\pgfsys@defobject{currentmarker}{\pgfqpoint{0.000000in}{0.000000in}}{\pgfqpoint{0.000000in}{0.027778in}}{%
\pgfpathmoveto{\pgfqpoint{0.000000in}{0.000000in}}%
\pgfpathlineto{\pgfqpoint{0.000000in}{0.027778in}}%
\pgfusepath{stroke,fill}%
}%
\begin{pgfscope}%
\pgfsys@transformshift{5.499789in}{0.300000in}%
\pgfsys@useobject{currentmarker}{}%
\end{pgfscope}%
\end{pgfscope}%
\begin{pgfscope}%
\pgfsetbuttcap%
\pgfsetroundjoin%
\definecolor{currentfill}{rgb}{0.000000,0.000000,0.000000}%
\pgfsetfillcolor{currentfill}%
\pgfsetlinewidth{0.501875pt}%
\definecolor{currentstroke}{rgb}{0.000000,0.000000,0.000000}%
\pgfsetstrokecolor{currentstroke}%
\pgfsetdash{}{0pt}%
\pgfsys@defobject{currentmarker}{\pgfqpoint{0.000000in}{-0.027778in}}{\pgfqpoint{0.000000in}{0.000000in}}{%
\pgfpathmoveto{\pgfqpoint{0.000000in}{0.000000in}}%
\pgfpathlineto{\pgfqpoint{0.000000in}{-0.027778in}}%
\pgfusepath{stroke,fill}%
}%
\begin{pgfscope}%
\pgfsys@transformshift{5.499789in}{2.700000in}%
\pgfsys@useobject{currentmarker}{}%
\end{pgfscope}%
\end{pgfscope}%
\begin{pgfscope}%
\pgfsetbuttcap%
\pgfsetroundjoin%
\definecolor{currentfill}{rgb}{0.000000,0.000000,0.000000}%
\pgfsetfillcolor{currentfill}%
\pgfsetlinewidth{0.501875pt}%
\definecolor{currentstroke}{rgb}{0.000000,0.000000,0.000000}%
\pgfsetstrokecolor{currentstroke}%
\pgfsetdash{}{0pt}%
\pgfsys@defobject{currentmarker}{\pgfqpoint{0.000000in}{0.000000in}}{\pgfqpoint{0.000000in}{0.027778in}}{%
\pgfpathmoveto{\pgfqpoint{0.000000in}{0.000000in}}%
\pgfpathlineto{\pgfqpoint{0.000000in}{0.027778in}}%
\pgfusepath{stroke,fill}%
}%
\begin{pgfscope}%
\pgfsys@transformshift{5.579076in}{0.300000in}%
\pgfsys@useobject{currentmarker}{}%
\end{pgfscope}%
\end{pgfscope}%
\begin{pgfscope}%
\pgfsetbuttcap%
\pgfsetroundjoin%
\definecolor{currentfill}{rgb}{0.000000,0.000000,0.000000}%
\pgfsetfillcolor{currentfill}%
\pgfsetlinewidth{0.501875pt}%
\definecolor{currentstroke}{rgb}{0.000000,0.000000,0.000000}%
\pgfsetstrokecolor{currentstroke}%
\pgfsetdash{}{0pt}%
\pgfsys@defobject{currentmarker}{\pgfqpoint{0.000000in}{-0.027778in}}{\pgfqpoint{0.000000in}{0.000000in}}{%
\pgfpathmoveto{\pgfqpoint{0.000000in}{0.000000in}}%
\pgfpathlineto{\pgfqpoint{0.000000in}{-0.027778in}}%
\pgfusepath{stroke,fill}%
}%
\begin{pgfscope}%
\pgfsys@transformshift{5.579076in}{2.700000in}%
\pgfsys@useobject{currentmarker}{}%
\end{pgfscope}%
\end{pgfscope}%
\begin{pgfscope}%
\pgfsetbuttcap%
\pgfsetroundjoin%
\definecolor{currentfill}{rgb}{0.000000,0.000000,0.000000}%
\pgfsetfillcolor{currentfill}%
\pgfsetlinewidth{0.501875pt}%
\definecolor{currentstroke}{rgb}{0.000000,0.000000,0.000000}%
\pgfsetstrokecolor{currentstroke}%
\pgfsetdash{}{0pt}%
\pgfsys@defobject{currentmarker}{\pgfqpoint{0.000000in}{0.000000in}}{\pgfqpoint{0.000000in}{0.027778in}}{%
\pgfpathmoveto{\pgfqpoint{0.000000in}{0.000000in}}%
\pgfpathlineto{\pgfqpoint{0.000000in}{0.027778in}}%
\pgfusepath{stroke,fill}%
}%
\begin{pgfscope}%
\pgfsys@transformshift{6.116596in}{0.300000in}%
\pgfsys@useobject{currentmarker}{}%
\end{pgfscope}%
\end{pgfscope}%
\begin{pgfscope}%
\pgfsetbuttcap%
\pgfsetroundjoin%
\definecolor{currentfill}{rgb}{0.000000,0.000000,0.000000}%
\pgfsetfillcolor{currentfill}%
\pgfsetlinewidth{0.501875pt}%
\definecolor{currentstroke}{rgb}{0.000000,0.000000,0.000000}%
\pgfsetstrokecolor{currentstroke}%
\pgfsetdash{}{0pt}%
\pgfsys@defobject{currentmarker}{\pgfqpoint{0.000000in}{-0.027778in}}{\pgfqpoint{0.000000in}{0.000000in}}{%
\pgfpathmoveto{\pgfqpoint{0.000000in}{0.000000in}}%
\pgfpathlineto{\pgfqpoint{0.000000in}{-0.027778in}}%
\pgfusepath{stroke,fill}%
}%
\begin{pgfscope}%
\pgfsys@transformshift{6.116596in}{2.700000in}%
\pgfsys@useobject{currentmarker}{}%
\end{pgfscope}%
\end{pgfscope}%
\begin{pgfscope}%
\pgfsetbuttcap%
\pgfsetroundjoin%
\definecolor{currentfill}{rgb}{0.000000,0.000000,0.000000}%
\pgfsetfillcolor{currentfill}%
\pgfsetlinewidth{0.501875pt}%
\definecolor{currentstroke}{rgb}{0.000000,0.000000,0.000000}%
\pgfsetstrokecolor{currentstroke}%
\pgfsetdash{}{0pt}%
\pgfsys@defobject{currentmarker}{\pgfqpoint{0.000000in}{0.000000in}}{\pgfqpoint{0.000000in}{0.027778in}}{%
\pgfpathmoveto{\pgfqpoint{0.000000in}{0.000000in}}%
\pgfpathlineto{\pgfqpoint{0.000000in}{0.027778in}}%
\pgfusepath{stroke,fill}%
}%
\begin{pgfscope}%
\pgfsys@transformshift{6.389538in}{0.300000in}%
\pgfsys@useobject{currentmarker}{}%
\end{pgfscope}%
\end{pgfscope}%
\begin{pgfscope}%
\pgfsetbuttcap%
\pgfsetroundjoin%
\definecolor{currentfill}{rgb}{0.000000,0.000000,0.000000}%
\pgfsetfillcolor{currentfill}%
\pgfsetlinewidth{0.501875pt}%
\definecolor{currentstroke}{rgb}{0.000000,0.000000,0.000000}%
\pgfsetstrokecolor{currentstroke}%
\pgfsetdash{}{0pt}%
\pgfsys@defobject{currentmarker}{\pgfqpoint{0.000000in}{-0.027778in}}{\pgfqpoint{0.000000in}{0.000000in}}{%
\pgfpathmoveto{\pgfqpoint{0.000000in}{0.000000in}}%
\pgfpathlineto{\pgfqpoint{0.000000in}{-0.027778in}}%
\pgfusepath{stroke,fill}%
}%
\begin{pgfscope}%
\pgfsys@transformshift{6.389538in}{2.700000in}%
\pgfsys@useobject{currentmarker}{}%
\end{pgfscope}%
\end{pgfscope}%
\begin{pgfscope}%
\pgfsetbuttcap%
\pgfsetroundjoin%
\definecolor{currentfill}{rgb}{0.000000,0.000000,0.000000}%
\pgfsetfillcolor{currentfill}%
\pgfsetlinewidth{0.501875pt}%
\definecolor{currentstroke}{rgb}{0.000000,0.000000,0.000000}%
\pgfsetstrokecolor{currentstroke}%
\pgfsetdash{}{0pt}%
\pgfsys@defobject{currentmarker}{\pgfqpoint{0.000000in}{0.000000in}}{\pgfqpoint{0.000000in}{0.027778in}}{%
\pgfpathmoveto{\pgfqpoint{0.000000in}{0.000000in}}%
\pgfpathlineto{\pgfqpoint{0.000000in}{0.027778in}}%
\pgfusepath{stroke,fill}%
}%
\begin{pgfscope}%
\pgfsys@transformshift{6.583193in}{0.300000in}%
\pgfsys@useobject{currentmarker}{}%
\end{pgfscope}%
\end{pgfscope}%
\begin{pgfscope}%
\pgfsetbuttcap%
\pgfsetroundjoin%
\definecolor{currentfill}{rgb}{0.000000,0.000000,0.000000}%
\pgfsetfillcolor{currentfill}%
\pgfsetlinewidth{0.501875pt}%
\definecolor{currentstroke}{rgb}{0.000000,0.000000,0.000000}%
\pgfsetstrokecolor{currentstroke}%
\pgfsetdash{}{0pt}%
\pgfsys@defobject{currentmarker}{\pgfqpoint{0.000000in}{-0.027778in}}{\pgfqpoint{0.000000in}{0.000000in}}{%
\pgfpathmoveto{\pgfqpoint{0.000000in}{0.000000in}}%
\pgfpathlineto{\pgfqpoint{0.000000in}{-0.027778in}}%
\pgfusepath{stroke,fill}%
}%
\begin{pgfscope}%
\pgfsys@transformshift{6.583193in}{2.700000in}%
\pgfsys@useobject{currentmarker}{}%
\end{pgfscope}%
\end{pgfscope}%
\begin{pgfscope}%
\pgfsetbuttcap%
\pgfsetroundjoin%
\definecolor{currentfill}{rgb}{0.000000,0.000000,0.000000}%
\pgfsetfillcolor{currentfill}%
\pgfsetlinewidth{0.501875pt}%
\definecolor{currentstroke}{rgb}{0.000000,0.000000,0.000000}%
\pgfsetstrokecolor{currentstroke}%
\pgfsetdash{}{0pt}%
\pgfsys@defobject{currentmarker}{\pgfqpoint{0.000000in}{0.000000in}}{\pgfqpoint{0.000000in}{0.027778in}}{%
\pgfpathmoveto{\pgfqpoint{0.000000in}{0.000000in}}%
\pgfpathlineto{\pgfqpoint{0.000000in}{0.027778in}}%
\pgfusepath{stroke,fill}%
}%
\begin{pgfscope}%
\pgfsys@transformshift{6.733404in}{0.300000in}%
\pgfsys@useobject{currentmarker}{}%
\end{pgfscope}%
\end{pgfscope}%
\begin{pgfscope}%
\pgfsetbuttcap%
\pgfsetroundjoin%
\definecolor{currentfill}{rgb}{0.000000,0.000000,0.000000}%
\pgfsetfillcolor{currentfill}%
\pgfsetlinewidth{0.501875pt}%
\definecolor{currentstroke}{rgb}{0.000000,0.000000,0.000000}%
\pgfsetstrokecolor{currentstroke}%
\pgfsetdash{}{0pt}%
\pgfsys@defobject{currentmarker}{\pgfqpoint{0.000000in}{-0.027778in}}{\pgfqpoint{0.000000in}{0.000000in}}{%
\pgfpathmoveto{\pgfqpoint{0.000000in}{0.000000in}}%
\pgfpathlineto{\pgfqpoint{0.000000in}{-0.027778in}}%
\pgfusepath{stroke,fill}%
}%
\begin{pgfscope}%
\pgfsys@transformshift{6.733404in}{2.700000in}%
\pgfsys@useobject{currentmarker}{}%
\end{pgfscope}%
\end{pgfscope}%
\begin{pgfscope}%
\pgfsetbuttcap%
\pgfsetroundjoin%
\definecolor{currentfill}{rgb}{0.000000,0.000000,0.000000}%
\pgfsetfillcolor{currentfill}%
\pgfsetlinewidth{0.501875pt}%
\definecolor{currentstroke}{rgb}{0.000000,0.000000,0.000000}%
\pgfsetstrokecolor{currentstroke}%
\pgfsetdash{}{0pt}%
\pgfsys@defobject{currentmarker}{\pgfqpoint{0.000000in}{0.000000in}}{\pgfqpoint{0.000000in}{0.027778in}}{%
\pgfpathmoveto{\pgfqpoint{0.000000in}{0.000000in}}%
\pgfpathlineto{\pgfqpoint{0.000000in}{0.027778in}}%
\pgfusepath{stroke,fill}%
}%
\begin{pgfscope}%
\pgfsys@transformshift{6.856134in}{0.300000in}%
\pgfsys@useobject{currentmarker}{}%
\end{pgfscope}%
\end{pgfscope}%
\begin{pgfscope}%
\pgfsetbuttcap%
\pgfsetroundjoin%
\definecolor{currentfill}{rgb}{0.000000,0.000000,0.000000}%
\pgfsetfillcolor{currentfill}%
\pgfsetlinewidth{0.501875pt}%
\definecolor{currentstroke}{rgb}{0.000000,0.000000,0.000000}%
\pgfsetstrokecolor{currentstroke}%
\pgfsetdash{}{0pt}%
\pgfsys@defobject{currentmarker}{\pgfqpoint{0.000000in}{-0.027778in}}{\pgfqpoint{0.000000in}{0.000000in}}{%
\pgfpathmoveto{\pgfqpoint{0.000000in}{0.000000in}}%
\pgfpathlineto{\pgfqpoint{0.000000in}{-0.027778in}}%
\pgfusepath{stroke,fill}%
}%
\begin{pgfscope}%
\pgfsys@transformshift{6.856134in}{2.700000in}%
\pgfsys@useobject{currentmarker}{}%
\end{pgfscope}%
\end{pgfscope}%
\begin{pgfscope}%
\pgfsetbuttcap%
\pgfsetroundjoin%
\definecolor{currentfill}{rgb}{0.000000,0.000000,0.000000}%
\pgfsetfillcolor{currentfill}%
\pgfsetlinewidth{0.501875pt}%
\definecolor{currentstroke}{rgb}{0.000000,0.000000,0.000000}%
\pgfsetstrokecolor{currentstroke}%
\pgfsetdash{}{0pt}%
\pgfsys@defobject{currentmarker}{\pgfqpoint{0.000000in}{0.000000in}}{\pgfqpoint{0.000000in}{0.027778in}}{%
\pgfpathmoveto{\pgfqpoint{0.000000in}{0.000000in}}%
\pgfpathlineto{\pgfqpoint{0.000000in}{0.027778in}}%
\pgfusepath{stroke,fill}%
}%
\begin{pgfscope}%
\pgfsys@transformshift{6.959902in}{0.300000in}%
\pgfsys@useobject{currentmarker}{}%
\end{pgfscope}%
\end{pgfscope}%
\begin{pgfscope}%
\pgfsetbuttcap%
\pgfsetroundjoin%
\definecolor{currentfill}{rgb}{0.000000,0.000000,0.000000}%
\pgfsetfillcolor{currentfill}%
\pgfsetlinewidth{0.501875pt}%
\definecolor{currentstroke}{rgb}{0.000000,0.000000,0.000000}%
\pgfsetstrokecolor{currentstroke}%
\pgfsetdash{}{0pt}%
\pgfsys@defobject{currentmarker}{\pgfqpoint{0.000000in}{-0.027778in}}{\pgfqpoint{0.000000in}{0.000000in}}{%
\pgfpathmoveto{\pgfqpoint{0.000000in}{0.000000in}}%
\pgfpathlineto{\pgfqpoint{0.000000in}{-0.027778in}}%
\pgfusepath{stroke,fill}%
}%
\begin{pgfscope}%
\pgfsys@transformshift{6.959902in}{2.700000in}%
\pgfsys@useobject{currentmarker}{}%
\end{pgfscope}%
\end{pgfscope}%
\begin{pgfscope}%
\pgfsetbuttcap%
\pgfsetroundjoin%
\definecolor{currentfill}{rgb}{0.000000,0.000000,0.000000}%
\pgfsetfillcolor{currentfill}%
\pgfsetlinewidth{0.501875pt}%
\definecolor{currentstroke}{rgb}{0.000000,0.000000,0.000000}%
\pgfsetstrokecolor{currentstroke}%
\pgfsetdash{}{0pt}%
\pgfsys@defobject{currentmarker}{\pgfqpoint{0.000000in}{0.000000in}}{\pgfqpoint{0.000000in}{0.027778in}}{%
\pgfpathmoveto{\pgfqpoint{0.000000in}{0.000000in}}%
\pgfpathlineto{\pgfqpoint{0.000000in}{0.027778in}}%
\pgfusepath{stroke,fill}%
}%
\begin{pgfscope}%
\pgfsys@transformshift{7.049789in}{0.300000in}%
\pgfsys@useobject{currentmarker}{}%
\end{pgfscope}%
\end{pgfscope}%
\begin{pgfscope}%
\pgfsetbuttcap%
\pgfsetroundjoin%
\definecolor{currentfill}{rgb}{0.000000,0.000000,0.000000}%
\pgfsetfillcolor{currentfill}%
\pgfsetlinewidth{0.501875pt}%
\definecolor{currentstroke}{rgb}{0.000000,0.000000,0.000000}%
\pgfsetstrokecolor{currentstroke}%
\pgfsetdash{}{0pt}%
\pgfsys@defobject{currentmarker}{\pgfqpoint{0.000000in}{-0.027778in}}{\pgfqpoint{0.000000in}{0.000000in}}{%
\pgfpathmoveto{\pgfqpoint{0.000000in}{0.000000in}}%
\pgfpathlineto{\pgfqpoint{0.000000in}{-0.027778in}}%
\pgfusepath{stroke,fill}%
}%
\begin{pgfscope}%
\pgfsys@transformshift{7.049789in}{2.700000in}%
\pgfsys@useobject{currentmarker}{}%
\end{pgfscope}%
\end{pgfscope}%
\begin{pgfscope}%
\pgfsetbuttcap%
\pgfsetroundjoin%
\definecolor{currentfill}{rgb}{0.000000,0.000000,0.000000}%
\pgfsetfillcolor{currentfill}%
\pgfsetlinewidth{0.501875pt}%
\definecolor{currentstroke}{rgb}{0.000000,0.000000,0.000000}%
\pgfsetstrokecolor{currentstroke}%
\pgfsetdash{}{0pt}%
\pgfsys@defobject{currentmarker}{\pgfqpoint{0.000000in}{0.000000in}}{\pgfqpoint{0.000000in}{0.027778in}}{%
\pgfpathmoveto{\pgfqpoint{0.000000in}{0.000000in}}%
\pgfpathlineto{\pgfqpoint{0.000000in}{0.027778in}}%
\pgfusepath{stroke,fill}%
}%
\begin{pgfscope}%
\pgfsys@transformshift{7.129076in}{0.300000in}%
\pgfsys@useobject{currentmarker}{}%
\end{pgfscope}%
\end{pgfscope}%
\begin{pgfscope}%
\pgfsetbuttcap%
\pgfsetroundjoin%
\definecolor{currentfill}{rgb}{0.000000,0.000000,0.000000}%
\pgfsetfillcolor{currentfill}%
\pgfsetlinewidth{0.501875pt}%
\definecolor{currentstroke}{rgb}{0.000000,0.000000,0.000000}%
\pgfsetstrokecolor{currentstroke}%
\pgfsetdash{}{0pt}%
\pgfsys@defobject{currentmarker}{\pgfqpoint{0.000000in}{-0.027778in}}{\pgfqpoint{0.000000in}{0.000000in}}{%
\pgfpathmoveto{\pgfqpoint{0.000000in}{0.000000in}}%
\pgfpathlineto{\pgfqpoint{0.000000in}{-0.027778in}}%
\pgfusepath{stroke,fill}%
}%
\begin{pgfscope}%
\pgfsys@transformshift{7.129076in}{2.700000in}%
\pgfsys@useobject{currentmarker}{}%
\end{pgfscope}%
\end{pgfscope}%
\begin{pgfscope}%
\pgftext[left,bottom,x=3.723901in,y=-0.126716in,rotate=0.000000]{{\sffamily\fontsize{12.000000}{14.400000}\selectfont time [ps]}}
%
\end{pgfscope}%
\begin{pgfscope}%
\pgfpathrectangle{\pgfqpoint{1.000000in}{0.300000in}}{\pgfqpoint{6.200000in}{2.400000in}} %
\pgfusepath{clip}%
\pgfsetbuttcap%
\pgfsetroundjoin%
\pgfsetlinewidth{0.501875pt}%
\definecolor{currentstroke}{rgb}{0.000000,0.000000,0.000000}%
\pgfsetstrokecolor{currentstroke}%
\pgfsetdash{{1.000000pt}{3.000000pt}}{0.000000pt}%
\pgfpathmoveto{\pgfqpoint{1.000000in}{0.300000in}}%
\pgfpathlineto{\pgfqpoint{7.200000in}{0.300000in}}%
\pgfusepath{stroke}%
\end{pgfscope}%
\begin{pgfscope}%
\pgfsetbuttcap%
\pgfsetroundjoin%
\definecolor{currentfill}{rgb}{0.000000,0.000000,0.000000}%
\pgfsetfillcolor{currentfill}%
\pgfsetlinewidth{0.501875pt}%
\definecolor{currentstroke}{rgb}{0.000000,0.000000,0.000000}%
\pgfsetstrokecolor{currentstroke}%
\pgfsetdash{}{0pt}%
\pgfsys@defobject{currentmarker}{\pgfqpoint{0.000000in}{0.000000in}}{\pgfqpoint{0.055556in}{0.000000in}}{%
\pgfpathmoveto{\pgfqpoint{0.000000in}{0.000000in}}%
\pgfpathlineto{\pgfqpoint{0.055556in}{0.000000in}}%
\pgfusepath{stroke,fill}%
}%
\begin{pgfscope}%
\pgfsys@transformshift{1.000000in}{0.300000in}%
\pgfsys@useobject{currentmarker}{}%
\end{pgfscope}%
\end{pgfscope}%
\begin{pgfscope}%
\pgfsetbuttcap%
\pgfsetroundjoin%
\definecolor{currentfill}{rgb}{0.000000,0.000000,0.000000}%
\pgfsetfillcolor{currentfill}%
\pgfsetlinewidth{0.501875pt}%
\definecolor{currentstroke}{rgb}{0.000000,0.000000,0.000000}%
\pgfsetstrokecolor{currentstroke}%
\pgfsetdash{}{0pt}%
\pgfsys@defobject{currentmarker}{\pgfqpoint{-0.055556in}{0.000000in}}{\pgfqpoint{0.000000in}{0.000000in}}{%
\pgfpathmoveto{\pgfqpoint{0.000000in}{0.000000in}}%
\pgfpathlineto{\pgfqpoint{-0.055556in}{0.000000in}}%
\pgfusepath{stroke,fill}%
}%
\begin{pgfscope}%
\pgfsys@transformshift{7.200000in}{0.300000in}%
\pgfsys@useobject{currentmarker}{}%
\end{pgfscope}%
\end{pgfscope}%
\begin{pgfscope}%
\pgftext[left,bottom,x=0.623456in,y=0.229790in,rotate=0.000000]{{\sffamily\fontsize{12.000000}{14.400000}\selectfont \(\displaystyle {10^{-3}}\)}}
%
\end{pgfscope}%
\begin{pgfscope}%
\pgfpathrectangle{\pgfqpoint{1.000000in}{0.300000in}}{\pgfqpoint{6.200000in}{2.400000in}} %
\pgfusepath{clip}%
\pgfsetbuttcap%
\pgfsetroundjoin%
\pgfsetlinewidth{0.501875pt}%
\definecolor{currentstroke}{rgb}{0.000000,0.000000,0.000000}%
\pgfsetstrokecolor{currentstroke}%
\pgfsetdash{{1.000000pt}{3.000000pt}}{0.000000pt}%
\pgfpathmoveto{\pgfqpoint{1.000000in}{0.780000in}}%
\pgfpathlineto{\pgfqpoint{7.200000in}{0.780000in}}%
\pgfusepath{stroke}%
\end{pgfscope}%
\begin{pgfscope}%
\pgfsetbuttcap%
\pgfsetroundjoin%
\definecolor{currentfill}{rgb}{0.000000,0.000000,0.000000}%
\pgfsetfillcolor{currentfill}%
\pgfsetlinewidth{0.501875pt}%
\definecolor{currentstroke}{rgb}{0.000000,0.000000,0.000000}%
\pgfsetstrokecolor{currentstroke}%
\pgfsetdash{}{0pt}%
\pgfsys@defobject{currentmarker}{\pgfqpoint{0.000000in}{0.000000in}}{\pgfqpoint{0.055556in}{0.000000in}}{%
\pgfpathmoveto{\pgfqpoint{0.000000in}{0.000000in}}%
\pgfpathlineto{\pgfqpoint{0.055556in}{0.000000in}}%
\pgfusepath{stroke,fill}%
}%
\begin{pgfscope}%
\pgfsys@transformshift{1.000000in}{0.780000in}%
\pgfsys@useobject{currentmarker}{}%
\end{pgfscope}%
\end{pgfscope}%
\begin{pgfscope}%
\pgfsetbuttcap%
\pgfsetroundjoin%
\definecolor{currentfill}{rgb}{0.000000,0.000000,0.000000}%
\pgfsetfillcolor{currentfill}%
\pgfsetlinewidth{0.501875pt}%
\definecolor{currentstroke}{rgb}{0.000000,0.000000,0.000000}%
\pgfsetstrokecolor{currentstroke}%
\pgfsetdash{}{0pt}%
\pgfsys@defobject{currentmarker}{\pgfqpoint{-0.055556in}{0.000000in}}{\pgfqpoint{0.000000in}{0.000000in}}{%
\pgfpathmoveto{\pgfqpoint{0.000000in}{0.000000in}}%
\pgfpathlineto{\pgfqpoint{-0.055556in}{0.000000in}}%
\pgfusepath{stroke,fill}%
}%
\begin{pgfscope}%
\pgfsys@transformshift{7.200000in}{0.780000in}%
\pgfsys@useobject{currentmarker}{}%
\end{pgfscope}%
\end{pgfscope}%
\begin{pgfscope}%
\pgftext[left,bottom,x=0.623456in,y=0.709790in,rotate=0.000000]{{\sffamily\fontsize{12.000000}{14.400000}\selectfont \(\displaystyle {10^{-2}}\)}}
%
\end{pgfscope}%
\begin{pgfscope}%
\pgfpathrectangle{\pgfqpoint{1.000000in}{0.300000in}}{\pgfqpoint{6.200000in}{2.400000in}} %
\pgfusepath{clip}%
\pgfsetbuttcap%
\pgfsetroundjoin%
\pgfsetlinewidth{0.501875pt}%
\definecolor{currentstroke}{rgb}{0.000000,0.000000,0.000000}%
\pgfsetstrokecolor{currentstroke}%
\pgfsetdash{{1.000000pt}{3.000000pt}}{0.000000pt}%
\pgfpathmoveto{\pgfqpoint{1.000000in}{1.260000in}}%
\pgfpathlineto{\pgfqpoint{7.200000in}{1.260000in}}%
\pgfusepath{stroke}%
\end{pgfscope}%
\begin{pgfscope}%
\pgfsetbuttcap%
\pgfsetroundjoin%
\definecolor{currentfill}{rgb}{0.000000,0.000000,0.000000}%
\pgfsetfillcolor{currentfill}%
\pgfsetlinewidth{0.501875pt}%
\definecolor{currentstroke}{rgb}{0.000000,0.000000,0.000000}%
\pgfsetstrokecolor{currentstroke}%
\pgfsetdash{}{0pt}%
\pgfsys@defobject{currentmarker}{\pgfqpoint{0.000000in}{0.000000in}}{\pgfqpoint{0.055556in}{0.000000in}}{%
\pgfpathmoveto{\pgfqpoint{0.000000in}{0.000000in}}%
\pgfpathlineto{\pgfqpoint{0.055556in}{0.000000in}}%
\pgfusepath{stroke,fill}%
}%
\begin{pgfscope}%
\pgfsys@transformshift{1.000000in}{1.260000in}%
\pgfsys@useobject{currentmarker}{}%
\end{pgfscope}%
\end{pgfscope}%
\begin{pgfscope}%
\pgfsetbuttcap%
\pgfsetroundjoin%
\definecolor{currentfill}{rgb}{0.000000,0.000000,0.000000}%
\pgfsetfillcolor{currentfill}%
\pgfsetlinewidth{0.501875pt}%
\definecolor{currentstroke}{rgb}{0.000000,0.000000,0.000000}%
\pgfsetstrokecolor{currentstroke}%
\pgfsetdash{}{0pt}%
\pgfsys@defobject{currentmarker}{\pgfqpoint{-0.055556in}{0.000000in}}{\pgfqpoint{0.000000in}{0.000000in}}{%
\pgfpathmoveto{\pgfqpoint{0.000000in}{0.000000in}}%
\pgfpathlineto{\pgfqpoint{-0.055556in}{0.000000in}}%
\pgfusepath{stroke,fill}%
}%
\begin{pgfscope}%
\pgfsys@transformshift{7.200000in}{1.260000in}%
\pgfsys@useobject{currentmarker}{}%
\end{pgfscope}%
\end{pgfscope}%
\begin{pgfscope}%
\pgftext[left,bottom,x=0.623456in,y=1.189790in,rotate=0.000000]{{\sffamily\fontsize{12.000000}{14.400000}\selectfont \(\displaystyle {10^{-1}}\)}}
%
\end{pgfscope}%
\begin{pgfscope}%
\pgfpathrectangle{\pgfqpoint{1.000000in}{0.300000in}}{\pgfqpoint{6.200000in}{2.400000in}} %
\pgfusepath{clip}%
\pgfsetbuttcap%
\pgfsetroundjoin%
\pgfsetlinewidth{0.501875pt}%
\definecolor{currentstroke}{rgb}{0.000000,0.000000,0.000000}%
\pgfsetstrokecolor{currentstroke}%
\pgfsetdash{{1.000000pt}{3.000000pt}}{0.000000pt}%
\pgfpathmoveto{\pgfqpoint{1.000000in}{1.740000in}}%
\pgfpathlineto{\pgfqpoint{7.200000in}{1.740000in}}%
\pgfusepath{stroke}%
\end{pgfscope}%
\begin{pgfscope}%
\pgfsetbuttcap%
\pgfsetroundjoin%
\definecolor{currentfill}{rgb}{0.000000,0.000000,0.000000}%
\pgfsetfillcolor{currentfill}%
\pgfsetlinewidth{0.501875pt}%
\definecolor{currentstroke}{rgb}{0.000000,0.000000,0.000000}%
\pgfsetstrokecolor{currentstroke}%
\pgfsetdash{}{0pt}%
\pgfsys@defobject{currentmarker}{\pgfqpoint{0.000000in}{0.000000in}}{\pgfqpoint{0.055556in}{0.000000in}}{%
\pgfpathmoveto{\pgfqpoint{0.000000in}{0.000000in}}%
\pgfpathlineto{\pgfqpoint{0.055556in}{0.000000in}}%
\pgfusepath{stroke,fill}%
}%
\begin{pgfscope}%
\pgfsys@transformshift{1.000000in}{1.740000in}%
\pgfsys@useobject{currentmarker}{}%
\end{pgfscope}%
\end{pgfscope}%
\begin{pgfscope}%
\pgfsetbuttcap%
\pgfsetroundjoin%
\definecolor{currentfill}{rgb}{0.000000,0.000000,0.000000}%
\pgfsetfillcolor{currentfill}%
\pgfsetlinewidth{0.501875pt}%
\definecolor{currentstroke}{rgb}{0.000000,0.000000,0.000000}%
\pgfsetstrokecolor{currentstroke}%
\pgfsetdash{}{0pt}%
\pgfsys@defobject{currentmarker}{\pgfqpoint{-0.055556in}{0.000000in}}{\pgfqpoint{0.000000in}{0.000000in}}{%
\pgfpathmoveto{\pgfqpoint{0.000000in}{0.000000in}}%
\pgfpathlineto{\pgfqpoint{-0.055556in}{0.000000in}}%
\pgfusepath{stroke,fill}%
}%
\begin{pgfscope}%
\pgfsys@transformshift{7.200000in}{1.740000in}%
\pgfsys@useobject{currentmarker}{}%
\end{pgfscope}%
\end{pgfscope}%
\begin{pgfscope}%
\pgftext[left,bottom,x=0.715279in,y=1.669790in,rotate=0.000000]{{\sffamily\fontsize{12.000000}{14.400000}\selectfont \(\displaystyle {10^{0}}\)}}
%
\end{pgfscope}%
\begin{pgfscope}%
\pgfpathrectangle{\pgfqpoint{1.000000in}{0.300000in}}{\pgfqpoint{6.200000in}{2.400000in}} %
\pgfusepath{clip}%
\pgfsetbuttcap%
\pgfsetroundjoin%
\pgfsetlinewidth{0.501875pt}%
\definecolor{currentstroke}{rgb}{0.000000,0.000000,0.000000}%
\pgfsetstrokecolor{currentstroke}%
\pgfsetdash{{1.000000pt}{3.000000pt}}{0.000000pt}%
\pgfpathmoveto{\pgfqpoint{1.000000in}{2.220000in}}%
\pgfpathlineto{\pgfqpoint{7.200000in}{2.220000in}}%
\pgfusepath{stroke}%
\end{pgfscope}%
\begin{pgfscope}%
\pgfsetbuttcap%
\pgfsetroundjoin%
\definecolor{currentfill}{rgb}{0.000000,0.000000,0.000000}%
\pgfsetfillcolor{currentfill}%
\pgfsetlinewidth{0.501875pt}%
\definecolor{currentstroke}{rgb}{0.000000,0.000000,0.000000}%
\pgfsetstrokecolor{currentstroke}%
\pgfsetdash{}{0pt}%
\pgfsys@defobject{currentmarker}{\pgfqpoint{0.000000in}{0.000000in}}{\pgfqpoint{0.055556in}{0.000000in}}{%
\pgfpathmoveto{\pgfqpoint{0.000000in}{0.000000in}}%
\pgfpathlineto{\pgfqpoint{0.055556in}{0.000000in}}%
\pgfusepath{stroke,fill}%
}%
\begin{pgfscope}%
\pgfsys@transformshift{1.000000in}{2.220000in}%
\pgfsys@useobject{currentmarker}{}%
\end{pgfscope}%
\end{pgfscope}%
\begin{pgfscope}%
\pgfsetbuttcap%
\pgfsetroundjoin%
\definecolor{currentfill}{rgb}{0.000000,0.000000,0.000000}%
\pgfsetfillcolor{currentfill}%
\pgfsetlinewidth{0.501875pt}%
\definecolor{currentstroke}{rgb}{0.000000,0.000000,0.000000}%
\pgfsetstrokecolor{currentstroke}%
\pgfsetdash{}{0pt}%
\pgfsys@defobject{currentmarker}{\pgfqpoint{-0.055556in}{0.000000in}}{\pgfqpoint{0.000000in}{0.000000in}}{%
\pgfpathmoveto{\pgfqpoint{0.000000in}{0.000000in}}%
\pgfpathlineto{\pgfqpoint{-0.055556in}{0.000000in}}%
\pgfusepath{stroke,fill}%
}%
\begin{pgfscope}%
\pgfsys@transformshift{7.200000in}{2.220000in}%
\pgfsys@useobject{currentmarker}{}%
\end{pgfscope}%
\end{pgfscope}%
\begin{pgfscope}%
\pgftext[left,bottom,x=0.715279in,y=2.149790in,rotate=0.000000]{{\sffamily\fontsize{12.000000}{14.400000}\selectfont \(\displaystyle {10^{1}}\)}}
%
\end{pgfscope}%
\begin{pgfscope}%
\pgfpathrectangle{\pgfqpoint{1.000000in}{0.300000in}}{\pgfqpoint{6.200000in}{2.400000in}} %
\pgfusepath{clip}%
\pgfsetbuttcap%
\pgfsetroundjoin%
\pgfsetlinewidth{0.501875pt}%
\definecolor{currentstroke}{rgb}{0.000000,0.000000,0.000000}%
\pgfsetstrokecolor{currentstroke}%
\pgfsetdash{{1.000000pt}{3.000000pt}}{0.000000pt}%
\pgfpathmoveto{\pgfqpoint{1.000000in}{2.700000in}}%
\pgfpathlineto{\pgfqpoint{7.200000in}{2.700000in}}%
\pgfusepath{stroke}%
\end{pgfscope}%
\begin{pgfscope}%
\pgfsetbuttcap%
\pgfsetroundjoin%
\definecolor{currentfill}{rgb}{0.000000,0.000000,0.000000}%
\pgfsetfillcolor{currentfill}%
\pgfsetlinewidth{0.501875pt}%
\definecolor{currentstroke}{rgb}{0.000000,0.000000,0.000000}%
\pgfsetstrokecolor{currentstroke}%
\pgfsetdash{}{0pt}%
\pgfsys@defobject{currentmarker}{\pgfqpoint{0.000000in}{0.000000in}}{\pgfqpoint{0.055556in}{0.000000in}}{%
\pgfpathmoveto{\pgfqpoint{0.000000in}{0.000000in}}%
\pgfpathlineto{\pgfqpoint{0.055556in}{0.000000in}}%
\pgfusepath{stroke,fill}%
}%
\begin{pgfscope}%
\pgfsys@transformshift{1.000000in}{2.700000in}%
\pgfsys@useobject{currentmarker}{}%
\end{pgfscope}%
\end{pgfscope}%
\begin{pgfscope}%
\pgfsetbuttcap%
\pgfsetroundjoin%
\definecolor{currentfill}{rgb}{0.000000,0.000000,0.000000}%
\pgfsetfillcolor{currentfill}%
\pgfsetlinewidth{0.501875pt}%
\definecolor{currentstroke}{rgb}{0.000000,0.000000,0.000000}%
\pgfsetstrokecolor{currentstroke}%
\pgfsetdash{}{0pt}%
\pgfsys@defobject{currentmarker}{\pgfqpoint{-0.055556in}{0.000000in}}{\pgfqpoint{0.000000in}{0.000000in}}{%
\pgfpathmoveto{\pgfqpoint{0.000000in}{0.000000in}}%
\pgfpathlineto{\pgfqpoint{-0.055556in}{0.000000in}}%
\pgfusepath{stroke,fill}%
}%
\begin{pgfscope}%
\pgfsys@transformshift{7.200000in}{2.700000in}%
\pgfsys@useobject{currentmarker}{}%
\end{pgfscope}%
\end{pgfscope}%
\begin{pgfscope}%
\pgftext[left,bottom,x=0.715279in,y=2.629790in,rotate=0.000000]{{\sffamily\fontsize{12.000000}{14.400000}\selectfont \(\displaystyle {10^{2}}\)}}
%
\end{pgfscope}%
\begin{pgfscope}%
\pgfsetbuttcap%
\pgfsetroundjoin%
\definecolor{currentfill}{rgb}{0.000000,0.000000,0.000000}%
\pgfsetfillcolor{currentfill}%
\pgfsetlinewidth{0.501875pt}%
\definecolor{currentstroke}{rgb}{0.000000,0.000000,0.000000}%
\pgfsetstrokecolor{currentstroke}%
\pgfsetdash{}{0pt}%
\pgfsys@defobject{currentmarker}{\pgfqpoint{0.000000in}{0.000000in}}{\pgfqpoint{0.027778in}{0.000000in}}{%
\pgfpathmoveto{\pgfqpoint{0.000000in}{0.000000in}}%
\pgfpathlineto{\pgfqpoint{0.027778in}{0.000000in}}%
\pgfusepath{stroke,fill}%
}%
\begin{pgfscope}%
\pgfsys@transformshift{1.000000in}{0.444494in}%
\pgfsys@useobject{currentmarker}{}%
\end{pgfscope}%
\end{pgfscope}%
\begin{pgfscope}%
\pgfsetbuttcap%
\pgfsetroundjoin%
\definecolor{currentfill}{rgb}{0.000000,0.000000,0.000000}%
\pgfsetfillcolor{currentfill}%
\pgfsetlinewidth{0.501875pt}%
\definecolor{currentstroke}{rgb}{0.000000,0.000000,0.000000}%
\pgfsetstrokecolor{currentstroke}%
\pgfsetdash{}{0pt}%
\pgfsys@defobject{currentmarker}{\pgfqpoint{-0.027778in}{0.000000in}}{\pgfqpoint{0.000000in}{0.000000in}}{%
\pgfpathmoveto{\pgfqpoint{0.000000in}{0.000000in}}%
\pgfpathlineto{\pgfqpoint{-0.027778in}{0.000000in}}%
\pgfusepath{stroke,fill}%
}%
\begin{pgfscope}%
\pgfsys@transformshift{7.200000in}{0.444494in}%
\pgfsys@useobject{currentmarker}{}%
\end{pgfscope}%
\end{pgfscope}%
\begin{pgfscope}%
\pgfsetbuttcap%
\pgfsetroundjoin%
\definecolor{currentfill}{rgb}{0.000000,0.000000,0.000000}%
\pgfsetfillcolor{currentfill}%
\pgfsetlinewidth{0.501875pt}%
\definecolor{currentstroke}{rgb}{0.000000,0.000000,0.000000}%
\pgfsetstrokecolor{currentstroke}%
\pgfsetdash{}{0pt}%
\pgfsys@defobject{currentmarker}{\pgfqpoint{0.000000in}{0.000000in}}{\pgfqpoint{0.027778in}{0.000000in}}{%
\pgfpathmoveto{\pgfqpoint{0.000000in}{0.000000in}}%
\pgfpathlineto{\pgfqpoint{0.027778in}{0.000000in}}%
\pgfusepath{stroke,fill}%
}%
\begin{pgfscope}%
\pgfsys@transformshift{1.000000in}{0.529018in}%
\pgfsys@useobject{currentmarker}{}%
\end{pgfscope}%
\end{pgfscope}%
\begin{pgfscope}%
\pgfsetbuttcap%
\pgfsetroundjoin%
\definecolor{currentfill}{rgb}{0.000000,0.000000,0.000000}%
\pgfsetfillcolor{currentfill}%
\pgfsetlinewidth{0.501875pt}%
\definecolor{currentstroke}{rgb}{0.000000,0.000000,0.000000}%
\pgfsetstrokecolor{currentstroke}%
\pgfsetdash{}{0pt}%
\pgfsys@defobject{currentmarker}{\pgfqpoint{-0.027778in}{0.000000in}}{\pgfqpoint{0.000000in}{0.000000in}}{%
\pgfpathmoveto{\pgfqpoint{0.000000in}{0.000000in}}%
\pgfpathlineto{\pgfqpoint{-0.027778in}{0.000000in}}%
\pgfusepath{stroke,fill}%
}%
\begin{pgfscope}%
\pgfsys@transformshift{7.200000in}{0.529018in}%
\pgfsys@useobject{currentmarker}{}%
\end{pgfscope}%
\end{pgfscope}%
\begin{pgfscope}%
\pgfsetbuttcap%
\pgfsetroundjoin%
\definecolor{currentfill}{rgb}{0.000000,0.000000,0.000000}%
\pgfsetfillcolor{currentfill}%
\pgfsetlinewidth{0.501875pt}%
\definecolor{currentstroke}{rgb}{0.000000,0.000000,0.000000}%
\pgfsetstrokecolor{currentstroke}%
\pgfsetdash{}{0pt}%
\pgfsys@defobject{currentmarker}{\pgfqpoint{0.000000in}{0.000000in}}{\pgfqpoint{0.027778in}{0.000000in}}{%
\pgfpathmoveto{\pgfqpoint{0.000000in}{0.000000in}}%
\pgfpathlineto{\pgfqpoint{0.027778in}{0.000000in}}%
\pgfusepath{stroke,fill}%
}%
\begin{pgfscope}%
\pgfsys@transformshift{1.000000in}{0.588989in}%
\pgfsys@useobject{currentmarker}{}%
\end{pgfscope}%
\end{pgfscope}%
\begin{pgfscope}%
\pgfsetbuttcap%
\pgfsetroundjoin%
\definecolor{currentfill}{rgb}{0.000000,0.000000,0.000000}%
\pgfsetfillcolor{currentfill}%
\pgfsetlinewidth{0.501875pt}%
\definecolor{currentstroke}{rgb}{0.000000,0.000000,0.000000}%
\pgfsetstrokecolor{currentstroke}%
\pgfsetdash{}{0pt}%
\pgfsys@defobject{currentmarker}{\pgfqpoint{-0.027778in}{0.000000in}}{\pgfqpoint{0.000000in}{0.000000in}}{%
\pgfpathmoveto{\pgfqpoint{0.000000in}{0.000000in}}%
\pgfpathlineto{\pgfqpoint{-0.027778in}{0.000000in}}%
\pgfusepath{stroke,fill}%
}%
\begin{pgfscope}%
\pgfsys@transformshift{7.200000in}{0.588989in}%
\pgfsys@useobject{currentmarker}{}%
\end{pgfscope}%
\end{pgfscope}%
\begin{pgfscope}%
\pgfsetbuttcap%
\pgfsetroundjoin%
\definecolor{currentfill}{rgb}{0.000000,0.000000,0.000000}%
\pgfsetfillcolor{currentfill}%
\pgfsetlinewidth{0.501875pt}%
\definecolor{currentstroke}{rgb}{0.000000,0.000000,0.000000}%
\pgfsetstrokecolor{currentstroke}%
\pgfsetdash{}{0pt}%
\pgfsys@defobject{currentmarker}{\pgfqpoint{0.000000in}{0.000000in}}{\pgfqpoint{0.027778in}{0.000000in}}{%
\pgfpathmoveto{\pgfqpoint{0.000000in}{0.000000in}}%
\pgfpathlineto{\pgfqpoint{0.027778in}{0.000000in}}%
\pgfusepath{stroke,fill}%
}%
\begin{pgfscope}%
\pgfsys@transformshift{1.000000in}{0.635506in}%
\pgfsys@useobject{currentmarker}{}%
\end{pgfscope}%
\end{pgfscope}%
\begin{pgfscope}%
\pgfsetbuttcap%
\pgfsetroundjoin%
\definecolor{currentfill}{rgb}{0.000000,0.000000,0.000000}%
\pgfsetfillcolor{currentfill}%
\pgfsetlinewidth{0.501875pt}%
\definecolor{currentstroke}{rgb}{0.000000,0.000000,0.000000}%
\pgfsetstrokecolor{currentstroke}%
\pgfsetdash{}{0pt}%
\pgfsys@defobject{currentmarker}{\pgfqpoint{-0.027778in}{0.000000in}}{\pgfqpoint{0.000000in}{0.000000in}}{%
\pgfpathmoveto{\pgfqpoint{0.000000in}{0.000000in}}%
\pgfpathlineto{\pgfqpoint{-0.027778in}{0.000000in}}%
\pgfusepath{stroke,fill}%
}%
\begin{pgfscope}%
\pgfsys@transformshift{7.200000in}{0.635506in}%
\pgfsys@useobject{currentmarker}{}%
\end{pgfscope}%
\end{pgfscope}%
\begin{pgfscope}%
\pgfsetbuttcap%
\pgfsetroundjoin%
\definecolor{currentfill}{rgb}{0.000000,0.000000,0.000000}%
\pgfsetfillcolor{currentfill}%
\pgfsetlinewidth{0.501875pt}%
\definecolor{currentstroke}{rgb}{0.000000,0.000000,0.000000}%
\pgfsetstrokecolor{currentstroke}%
\pgfsetdash{}{0pt}%
\pgfsys@defobject{currentmarker}{\pgfqpoint{0.000000in}{0.000000in}}{\pgfqpoint{0.027778in}{0.000000in}}{%
\pgfpathmoveto{\pgfqpoint{0.000000in}{0.000000in}}%
\pgfpathlineto{\pgfqpoint{0.027778in}{0.000000in}}%
\pgfusepath{stroke,fill}%
}%
\begin{pgfscope}%
\pgfsys@transformshift{1.000000in}{0.673513in}%
\pgfsys@useobject{currentmarker}{}%
\end{pgfscope}%
\end{pgfscope}%
\begin{pgfscope}%
\pgfsetbuttcap%
\pgfsetroundjoin%
\definecolor{currentfill}{rgb}{0.000000,0.000000,0.000000}%
\pgfsetfillcolor{currentfill}%
\pgfsetlinewidth{0.501875pt}%
\definecolor{currentstroke}{rgb}{0.000000,0.000000,0.000000}%
\pgfsetstrokecolor{currentstroke}%
\pgfsetdash{}{0pt}%
\pgfsys@defobject{currentmarker}{\pgfqpoint{-0.027778in}{0.000000in}}{\pgfqpoint{0.000000in}{0.000000in}}{%
\pgfpathmoveto{\pgfqpoint{0.000000in}{0.000000in}}%
\pgfpathlineto{\pgfqpoint{-0.027778in}{0.000000in}}%
\pgfusepath{stroke,fill}%
}%
\begin{pgfscope}%
\pgfsys@transformshift{7.200000in}{0.673513in}%
\pgfsys@useobject{currentmarker}{}%
\end{pgfscope}%
\end{pgfscope}%
\begin{pgfscope}%
\pgfsetbuttcap%
\pgfsetroundjoin%
\definecolor{currentfill}{rgb}{0.000000,0.000000,0.000000}%
\pgfsetfillcolor{currentfill}%
\pgfsetlinewidth{0.501875pt}%
\definecolor{currentstroke}{rgb}{0.000000,0.000000,0.000000}%
\pgfsetstrokecolor{currentstroke}%
\pgfsetdash{}{0pt}%
\pgfsys@defobject{currentmarker}{\pgfqpoint{0.000000in}{0.000000in}}{\pgfqpoint{0.027778in}{0.000000in}}{%
\pgfpathmoveto{\pgfqpoint{0.000000in}{0.000000in}}%
\pgfpathlineto{\pgfqpoint{0.027778in}{0.000000in}}%
\pgfusepath{stroke,fill}%
}%
\begin{pgfscope}%
\pgfsys@transformshift{1.000000in}{0.705647in}%
\pgfsys@useobject{currentmarker}{}%
\end{pgfscope}%
\end{pgfscope}%
\begin{pgfscope}%
\pgfsetbuttcap%
\pgfsetroundjoin%
\definecolor{currentfill}{rgb}{0.000000,0.000000,0.000000}%
\pgfsetfillcolor{currentfill}%
\pgfsetlinewidth{0.501875pt}%
\definecolor{currentstroke}{rgb}{0.000000,0.000000,0.000000}%
\pgfsetstrokecolor{currentstroke}%
\pgfsetdash{}{0pt}%
\pgfsys@defobject{currentmarker}{\pgfqpoint{-0.027778in}{0.000000in}}{\pgfqpoint{0.000000in}{0.000000in}}{%
\pgfpathmoveto{\pgfqpoint{0.000000in}{0.000000in}}%
\pgfpathlineto{\pgfqpoint{-0.027778in}{0.000000in}}%
\pgfusepath{stroke,fill}%
}%
\begin{pgfscope}%
\pgfsys@transformshift{7.200000in}{0.705647in}%
\pgfsys@useobject{currentmarker}{}%
\end{pgfscope}%
\end{pgfscope}%
\begin{pgfscope}%
\pgfsetbuttcap%
\pgfsetroundjoin%
\definecolor{currentfill}{rgb}{0.000000,0.000000,0.000000}%
\pgfsetfillcolor{currentfill}%
\pgfsetlinewidth{0.501875pt}%
\definecolor{currentstroke}{rgb}{0.000000,0.000000,0.000000}%
\pgfsetstrokecolor{currentstroke}%
\pgfsetdash{}{0pt}%
\pgfsys@defobject{currentmarker}{\pgfqpoint{0.000000in}{0.000000in}}{\pgfqpoint{0.027778in}{0.000000in}}{%
\pgfpathmoveto{\pgfqpoint{0.000000in}{0.000000in}}%
\pgfpathlineto{\pgfqpoint{0.027778in}{0.000000in}}%
\pgfusepath{stroke,fill}%
}%
\begin{pgfscope}%
\pgfsys@transformshift{1.000000in}{0.733483in}%
\pgfsys@useobject{currentmarker}{}%
\end{pgfscope}%
\end{pgfscope}%
\begin{pgfscope}%
\pgfsetbuttcap%
\pgfsetroundjoin%
\definecolor{currentfill}{rgb}{0.000000,0.000000,0.000000}%
\pgfsetfillcolor{currentfill}%
\pgfsetlinewidth{0.501875pt}%
\definecolor{currentstroke}{rgb}{0.000000,0.000000,0.000000}%
\pgfsetstrokecolor{currentstroke}%
\pgfsetdash{}{0pt}%
\pgfsys@defobject{currentmarker}{\pgfqpoint{-0.027778in}{0.000000in}}{\pgfqpoint{0.000000in}{0.000000in}}{%
\pgfpathmoveto{\pgfqpoint{0.000000in}{0.000000in}}%
\pgfpathlineto{\pgfqpoint{-0.027778in}{0.000000in}}%
\pgfusepath{stroke,fill}%
}%
\begin{pgfscope}%
\pgfsys@transformshift{7.200000in}{0.733483in}%
\pgfsys@useobject{currentmarker}{}%
\end{pgfscope}%
\end{pgfscope}%
\begin{pgfscope}%
\pgfsetbuttcap%
\pgfsetroundjoin%
\definecolor{currentfill}{rgb}{0.000000,0.000000,0.000000}%
\pgfsetfillcolor{currentfill}%
\pgfsetlinewidth{0.501875pt}%
\definecolor{currentstroke}{rgb}{0.000000,0.000000,0.000000}%
\pgfsetstrokecolor{currentstroke}%
\pgfsetdash{}{0pt}%
\pgfsys@defobject{currentmarker}{\pgfqpoint{0.000000in}{0.000000in}}{\pgfqpoint{0.027778in}{0.000000in}}{%
\pgfpathmoveto{\pgfqpoint{0.000000in}{0.000000in}}%
\pgfpathlineto{\pgfqpoint{0.027778in}{0.000000in}}%
\pgfusepath{stroke,fill}%
}%
\begin{pgfscope}%
\pgfsys@transformshift{1.000000in}{0.758036in}%
\pgfsys@useobject{currentmarker}{}%
\end{pgfscope}%
\end{pgfscope}%
\begin{pgfscope}%
\pgfsetbuttcap%
\pgfsetroundjoin%
\definecolor{currentfill}{rgb}{0.000000,0.000000,0.000000}%
\pgfsetfillcolor{currentfill}%
\pgfsetlinewidth{0.501875pt}%
\definecolor{currentstroke}{rgb}{0.000000,0.000000,0.000000}%
\pgfsetstrokecolor{currentstroke}%
\pgfsetdash{}{0pt}%
\pgfsys@defobject{currentmarker}{\pgfqpoint{-0.027778in}{0.000000in}}{\pgfqpoint{0.000000in}{0.000000in}}{%
\pgfpathmoveto{\pgfqpoint{0.000000in}{0.000000in}}%
\pgfpathlineto{\pgfqpoint{-0.027778in}{0.000000in}}%
\pgfusepath{stroke,fill}%
}%
\begin{pgfscope}%
\pgfsys@transformshift{7.200000in}{0.758036in}%
\pgfsys@useobject{currentmarker}{}%
\end{pgfscope}%
\end{pgfscope}%
\begin{pgfscope}%
\pgfsetbuttcap%
\pgfsetroundjoin%
\definecolor{currentfill}{rgb}{0.000000,0.000000,0.000000}%
\pgfsetfillcolor{currentfill}%
\pgfsetlinewidth{0.501875pt}%
\definecolor{currentstroke}{rgb}{0.000000,0.000000,0.000000}%
\pgfsetstrokecolor{currentstroke}%
\pgfsetdash{}{0pt}%
\pgfsys@defobject{currentmarker}{\pgfqpoint{0.000000in}{0.000000in}}{\pgfqpoint{0.027778in}{0.000000in}}{%
\pgfpathmoveto{\pgfqpoint{0.000000in}{0.000000in}}%
\pgfpathlineto{\pgfqpoint{0.027778in}{0.000000in}}%
\pgfusepath{stroke,fill}%
}%
\begin{pgfscope}%
\pgfsys@transformshift{1.000000in}{0.924494in}%
\pgfsys@useobject{currentmarker}{}%
\end{pgfscope}%
\end{pgfscope}%
\begin{pgfscope}%
\pgfsetbuttcap%
\pgfsetroundjoin%
\definecolor{currentfill}{rgb}{0.000000,0.000000,0.000000}%
\pgfsetfillcolor{currentfill}%
\pgfsetlinewidth{0.501875pt}%
\definecolor{currentstroke}{rgb}{0.000000,0.000000,0.000000}%
\pgfsetstrokecolor{currentstroke}%
\pgfsetdash{}{0pt}%
\pgfsys@defobject{currentmarker}{\pgfqpoint{-0.027778in}{0.000000in}}{\pgfqpoint{0.000000in}{0.000000in}}{%
\pgfpathmoveto{\pgfqpoint{0.000000in}{0.000000in}}%
\pgfpathlineto{\pgfqpoint{-0.027778in}{0.000000in}}%
\pgfusepath{stroke,fill}%
}%
\begin{pgfscope}%
\pgfsys@transformshift{7.200000in}{0.924494in}%
\pgfsys@useobject{currentmarker}{}%
\end{pgfscope}%
\end{pgfscope}%
\begin{pgfscope}%
\pgfsetbuttcap%
\pgfsetroundjoin%
\definecolor{currentfill}{rgb}{0.000000,0.000000,0.000000}%
\pgfsetfillcolor{currentfill}%
\pgfsetlinewidth{0.501875pt}%
\definecolor{currentstroke}{rgb}{0.000000,0.000000,0.000000}%
\pgfsetstrokecolor{currentstroke}%
\pgfsetdash{}{0pt}%
\pgfsys@defobject{currentmarker}{\pgfqpoint{0.000000in}{0.000000in}}{\pgfqpoint{0.027778in}{0.000000in}}{%
\pgfpathmoveto{\pgfqpoint{0.000000in}{0.000000in}}%
\pgfpathlineto{\pgfqpoint{0.027778in}{0.000000in}}%
\pgfusepath{stroke,fill}%
}%
\begin{pgfscope}%
\pgfsys@transformshift{1.000000in}{1.009018in}%
\pgfsys@useobject{currentmarker}{}%
\end{pgfscope}%
\end{pgfscope}%
\begin{pgfscope}%
\pgfsetbuttcap%
\pgfsetroundjoin%
\definecolor{currentfill}{rgb}{0.000000,0.000000,0.000000}%
\pgfsetfillcolor{currentfill}%
\pgfsetlinewidth{0.501875pt}%
\definecolor{currentstroke}{rgb}{0.000000,0.000000,0.000000}%
\pgfsetstrokecolor{currentstroke}%
\pgfsetdash{}{0pt}%
\pgfsys@defobject{currentmarker}{\pgfqpoint{-0.027778in}{0.000000in}}{\pgfqpoint{0.000000in}{0.000000in}}{%
\pgfpathmoveto{\pgfqpoint{0.000000in}{0.000000in}}%
\pgfpathlineto{\pgfqpoint{-0.027778in}{0.000000in}}%
\pgfusepath{stroke,fill}%
}%
\begin{pgfscope}%
\pgfsys@transformshift{7.200000in}{1.009018in}%
\pgfsys@useobject{currentmarker}{}%
\end{pgfscope}%
\end{pgfscope}%
\begin{pgfscope}%
\pgfsetbuttcap%
\pgfsetroundjoin%
\definecolor{currentfill}{rgb}{0.000000,0.000000,0.000000}%
\pgfsetfillcolor{currentfill}%
\pgfsetlinewidth{0.501875pt}%
\definecolor{currentstroke}{rgb}{0.000000,0.000000,0.000000}%
\pgfsetstrokecolor{currentstroke}%
\pgfsetdash{}{0pt}%
\pgfsys@defobject{currentmarker}{\pgfqpoint{0.000000in}{0.000000in}}{\pgfqpoint{0.027778in}{0.000000in}}{%
\pgfpathmoveto{\pgfqpoint{0.000000in}{0.000000in}}%
\pgfpathlineto{\pgfqpoint{0.027778in}{0.000000in}}%
\pgfusepath{stroke,fill}%
}%
\begin{pgfscope}%
\pgfsys@transformshift{1.000000in}{1.068989in}%
\pgfsys@useobject{currentmarker}{}%
\end{pgfscope}%
\end{pgfscope}%
\begin{pgfscope}%
\pgfsetbuttcap%
\pgfsetroundjoin%
\definecolor{currentfill}{rgb}{0.000000,0.000000,0.000000}%
\pgfsetfillcolor{currentfill}%
\pgfsetlinewidth{0.501875pt}%
\definecolor{currentstroke}{rgb}{0.000000,0.000000,0.000000}%
\pgfsetstrokecolor{currentstroke}%
\pgfsetdash{}{0pt}%
\pgfsys@defobject{currentmarker}{\pgfqpoint{-0.027778in}{0.000000in}}{\pgfqpoint{0.000000in}{0.000000in}}{%
\pgfpathmoveto{\pgfqpoint{0.000000in}{0.000000in}}%
\pgfpathlineto{\pgfqpoint{-0.027778in}{0.000000in}}%
\pgfusepath{stroke,fill}%
}%
\begin{pgfscope}%
\pgfsys@transformshift{7.200000in}{1.068989in}%
\pgfsys@useobject{currentmarker}{}%
\end{pgfscope}%
\end{pgfscope}%
\begin{pgfscope}%
\pgfsetbuttcap%
\pgfsetroundjoin%
\definecolor{currentfill}{rgb}{0.000000,0.000000,0.000000}%
\pgfsetfillcolor{currentfill}%
\pgfsetlinewidth{0.501875pt}%
\definecolor{currentstroke}{rgb}{0.000000,0.000000,0.000000}%
\pgfsetstrokecolor{currentstroke}%
\pgfsetdash{}{0pt}%
\pgfsys@defobject{currentmarker}{\pgfqpoint{0.000000in}{0.000000in}}{\pgfqpoint{0.027778in}{0.000000in}}{%
\pgfpathmoveto{\pgfqpoint{0.000000in}{0.000000in}}%
\pgfpathlineto{\pgfqpoint{0.027778in}{0.000000in}}%
\pgfusepath{stroke,fill}%
}%
\begin{pgfscope}%
\pgfsys@transformshift{1.000000in}{1.115506in}%
\pgfsys@useobject{currentmarker}{}%
\end{pgfscope}%
\end{pgfscope}%
\begin{pgfscope}%
\pgfsetbuttcap%
\pgfsetroundjoin%
\definecolor{currentfill}{rgb}{0.000000,0.000000,0.000000}%
\pgfsetfillcolor{currentfill}%
\pgfsetlinewidth{0.501875pt}%
\definecolor{currentstroke}{rgb}{0.000000,0.000000,0.000000}%
\pgfsetstrokecolor{currentstroke}%
\pgfsetdash{}{0pt}%
\pgfsys@defobject{currentmarker}{\pgfqpoint{-0.027778in}{0.000000in}}{\pgfqpoint{0.000000in}{0.000000in}}{%
\pgfpathmoveto{\pgfqpoint{0.000000in}{0.000000in}}%
\pgfpathlineto{\pgfqpoint{-0.027778in}{0.000000in}}%
\pgfusepath{stroke,fill}%
}%
\begin{pgfscope}%
\pgfsys@transformshift{7.200000in}{1.115506in}%
\pgfsys@useobject{currentmarker}{}%
\end{pgfscope}%
\end{pgfscope}%
\begin{pgfscope}%
\pgfsetbuttcap%
\pgfsetroundjoin%
\definecolor{currentfill}{rgb}{0.000000,0.000000,0.000000}%
\pgfsetfillcolor{currentfill}%
\pgfsetlinewidth{0.501875pt}%
\definecolor{currentstroke}{rgb}{0.000000,0.000000,0.000000}%
\pgfsetstrokecolor{currentstroke}%
\pgfsetdash{}{0pt}%
\pgfsys@defobject{currentmarker}{\pgfqpoint{0.000000in}{0.000000in}}{\pgfqpoint{0.027778in}{0.000000in}}{%
\pgfpathmoveto{\pgfqpoint{0.000000in}{0.000000in}}%
\pgfpathlineto{\pgfqpoint{0.027778in}{0.000000in}}%
\pgfusepath{stroke,fill}%
}%
\begin{pgfscope}%
\pgfsys@transformshift{1.000000in}{1.153513in}%
\pgfsys@useobject{currentmarker}{}%
\end{pgfscope}%
\end{pgfscope}%
\begin{pgfscope}%
\pgfsetbuttcap%
\pgfsetroundjoin%
\definecolor{currentfill}{rgb}{0.000000,0.000000,0.000000}%
\pgfsetfillcolor{currentfill}%
\pgfsetlinewidth{0.501875pt}%
\definecolor{currentstroke}{rgb}{0.000000,0.000000,0.000000}%
\pgfsetstrokecolor{currentstroke}%
\pgfsetdash{}{0pt}%
\pgfsys@defobject{currentmarker}{\pgfqpoint{-0.027778in}{0.000000in}}{\pgfqpoint{0.000000in}{0.000000in}}{%
\pgfpathmoveto{\pgfqpoint{0.000000in}{0.000000in}}%
\pgfpathlineto{\pgfqpoint{-0.027778in}{0.000000in}}%
\pgfusepath{stroke,fill}%
}%
\begin{pgfscope}%
\pgfsys@transformshift{7.200000in}{1.153513in}%
\pgfsys@useobject{currentmarker}{}%
\end{pgfscope}%
\end{pgfscope}%
\begin{pgfscope}%
\pgfsetbuttcap%
\pgfsetroundjoin%
\definecolor{currentfill}{rgb}{0.000000,0.000000,0.000000}%
\pgfsetfillcolor{currentfill}%
\pgfsetlinewidth{0.501875pt}%
\definecolor{currentstroke}{rgb}{0.000000,0.000000,0.000000}%
\pgfsetstrokecolor{currentstroke}%
\pgfsetdash{}{0pt}%
\pgfsys@defobject{currentmarker}{\pgfqpoint{0.000000in}{0.000000in}}{\pgfqpoint{0.027778in}{0.000000in}}{%
\pgfpathmoveto{\pgfqpoint{0.000000in}{0.000000in}}%
\pgfpathlineto{\pgfqpoint{0.027778in}{0.000000in}}%
\pgfusepath{stroke,fill}%
}%
\begin{pgfscope}%
\pgfsys@transformshift{1.000000in}{1.185647in}%
\pgfsys@useobject{currentmarker}{}%
\end{pgfscope}%
\end{pgfscope}%
\begin{pgfscope}%
\pgfsetbuttcap%
\pgfsetroundjoin%
\definecolor{currentfill}{rgb}{0.000000,0.000000,0.000000}%
\pgfsetfillcolor{currentfill}%
\pgfsetlinewidth{0.501875pt}%
\definecolor{currentstroke}{rgb}{0.000000,0.000000,0.000000}%
\pgfsetstrokecolor{currentstroke}%
\pgfsetdash{}{0pt}%
\pgfsys@defobject{currentmarker}{\pgfqpoint{-0.027778in}{0.000000in}}{\pgfqpoint{0.000000in}{0.000000in}}{%
\pgfpathmoveto{\pgfqpoint{0.000000in}{0.000000in}}%
\pgfpathlineto{\pgfqpoint{-0.027778in}{0.000000in}}%
\pgfusepath{stroke,fill}%
}%
\begin{pgfscope}%
\pgfsys@transformshift{7.200000in}{1.185647in}%
\pgfsys@useobject{currentmarker}{}%
\end{pgfscope}%
\end{pgfscope}%
\begin{pgfscope}%
\pgfsetbuttcap%
\pgfsetroundjoin%
\definecolor{currentfill}{rgb}{0.000000,0.000000,0.000000}%
\pgfsetfillcolor{currentfill}%
\pgfsetlinewidth{0.501875pt}%
\definecolor{currentstroke}{rgb}{0.000000,0.000000,0.000000}%
\pgfsetstrokecolor{currentstroke}%
\pgfsetdash{}{0pt}%
\pgfsys@defobject{currentmarker}{\pgfqpoint{0.000000in}{0.000000in}}{\pgfqpoint{0.027778in}{0.000000in}}{%
\pgfpathmoveto{\pgfqpoint{0.000000in}{0.000000in}}%
\pgfpathlineto{\pgfqpoint{0.027778in}{0.000000in}}%
\pgfusepath{stroke,fill}%
}%
\begin{pgfscope}%
\pgfsys@transformshift{1.000000in}{1.213483in}%
\pgfsys@useobject{currentmarker}{}%
\end{pgfscope}%
\end{pgfscope}%
\begin{pgfscope}%
\pgfsetbuttcap%
\pgfsetroundjoin%
\definecolor{currentfill}{rgb}{0.000000,0.000000,0.000000}%
\pgfsetfillcolor{currentfill}%
\pgfsetlinewidth{0.501875pt}%
\definecolor{currentstroke}{rgb}{0.000000,0.000000,0.000000}%
\pgfsetstrokecolor{currentstroke}%
\pgfsetdash{}{0pt}%
\pgfsys@defobject{currentmarker}{\pgfqpoint{-0.027778in}{0.000000in}}{\pgfqpoint{0.000000in}{0.000000in}}{%
\pgfpathmoveto{\pgfqpoint{0.000000in}{0.000000in}}%
\pgfpathlineto{\pgfqpoint{-0.027778in}{0.000000in}}%
\pgfusepath{stroke,fill}%
}%
\begin{pgfscope}%
\pgfsys@transformshift{7.200000in}{1.213483in}%
\pgfsys@useobject{currentmarker}{}%
\end{pgfscope}%
\end{pgfscope}%
\begin{pgfscope}%
\pgfsetbuttcap%
\pgfsetroundjoin%
\definecolor{currentfill}{rgb}{0.000000,0.000000,0.000000}%
\pgfsetfillcolor{currentfill}%
\pgfsetlinewidth{0.501875pt}%
\definecolor{currentstroke}{rgb}{0.000000,0.000000,0.000000}%
\pgfsetstrokecolor{currentstroke}%
\pgfsetdash{}{0pt}%
\pgfsys@defobject{currentmarker}{\pgfqpoint{0.000000in}{0.000000in}}{\pgfqpoint{0.027778in}{0.000000in}}{%
\pgfpathmoveto{\pgfqpoint{0.000000in}{0.000000in}}%
\pgfpathlineto{\pgfqpoint{0.027778in}{0.000000in}}%
\pgfusepath{stroke,fill}%
}%
\begin{pgfscope}%
\pgfsys@transformshift{1.000000in}{1.238036in}%
\pgfsys@useobject{currentmarker}{}%
\end{pgfscope}%
\end{pgfscope}%
\begin{pgfscope}%
\pgfsetbuttcap%
\pgfsetroundjoin%
\definecolor{currentfill}{rgb}{0.000000,0.000000,0.000000}%
\pgfsetfillcolor{currentfill}%
\pgfsetlinewidth{0.501875pt}%
\definecolor{currentstroke}{rgb}{0.000000,0.000000,0.000000}%
\pgfsetstrokecolor{currentstroke}%
\pgfsetdash{}{0pt}%
\pgfsys@defobject{currentmarker}{\pgfqpoint{-0.027778in}{0.000000in}}{\pgfqpoint{0.000000in}{0.000000in}}{%
\pgfpathmoveto{\pgfqpoint{0.000000in}{0.000000in}}%
\pgfpathlineto{\pgfqpoint{-0.027778in}{0.000000in}}%
\pgfusepath{stroke,fill}%
}%
\begin{pgfscope}%
\pgfsys@transformshift{7.200000in}{1.238036in}%
\pgfsys@useobject{currentmarker}{}%
\end{pgfscope}%
\end{pgfscope}%
\begin{pgfscope}%
\pgfsetbuttcap%
\pgfsetroundjoin%
\definecolor{currentfill}{rgb}{0.000000,0.000000,0.000000}%
\pgfsetfillcolor{currentfill}%
\pgfsetlinewidth{0.501875pt}%
\definecolor{currentstroke}{rgb}{0.000000,0.000000,0.000000}%
\pgfsetstrokecolor{currentstroke}%
\pgfsetdash{}{0pt}%
\pgfsys@defobject{currentmarker}{\pgfqpoint{0.000000in}{0.000000in}}{\pgfqpoint{0.027778in}{0.000000in}}{%
\pgfpathmoveto{\pgfqpoint{0.000000in}{0.000000in}}%
\pgfpathlineto{\pgfqpoint{0.027778in}{0.000000in}}%
\pgfusepath{stroke,fill}%
}%
\begin{pgfscope}%
\pgfsys@transformshift{1.000000in}{1.404494in}%
\pgfsys@useobject{currentmarker}{}%
\end{pgfscope}%
\end{pgfscope}%
\begin{pgfscope}%
\pgfsetbuttcap%
\pgfsetroundjoin%
\definecolor{currentfill}{rgb}{0.000000,0.000000,0.000000}%
\pgfsetfillcolor{currentfill}%
\pgfsetlinewidth{0.501875pt}%
\definecolor{currentstroke}{rgb}{0.000000,0.000000,0.000000}%
\pgfsetstrokecolor{currentstroke}%
\pgfsetdash{}{0pt}%
\pgfsys@defobject{currentmarker}{\pgfqpoint{-0.027778in}{0.000000in}}{\pgfqpoint{0.000000in}{0.000000in}}{%
\pgfpathmoveto{\pgfqpoint{0.000000in}{0.000000in}}%
\pgfpathlineto{\pgfqpoint{-0.027778in}{0.000000in}}%
\pgfusepath{stroke,fill}%
}%
\begin{pgfscope}%
\pgfsys@transformshift{7.200000in}{1.404494in}%
\pgfsys@useobject{currentmarker}{}%
\end{pgfscope}%
\end{pgfscope}%
\begin{pgfscope}%
\pgfsetbuttcap%
\pgfsetroundjoin%
\definecolor{currentfill}{rgb}{0.000000,0.000000,0.000000}%
\pgfsetfillcolor{currentfill}%
\pgfsetlinewidth{0.501875pt}%
\definecolor{currentstroke}{rgb}{0.000000,0.000000,0.000000}%
\pgfsetstrokecolor{currentstroke}%
\pgfsetdash{}{0pt}%
\pgfsys@defobject{currentmarker}{\pgfqpoint{0.000000in}{0.000000in}}{\pgfqpoint{0.027778in}{0.000000in}}{%
\pgfpathmoveto{\pgfqpoint{0.000000in}{0.000000in}}%
\pgfpathlineto{\pgfqpoint{0.027778in}{0.000000in}}%
\pgfusepath{stroke,fill}%
}%
\begin{pgfscope}%
\pgfsys@transformshift{1.000000in}{1.489018in}%
\pgfsys@useobject{currentmarker}{}%
\end{pgfscope}%
\end{pgfscope}%
\begin{pgfscope}%
\pgfsetbuttcap%
\pgfsetroundjoin%
\definecolor{currentfill}{rgb}{0.000000,0.000000,0.000000}%
\pgfsetfillcolor{currentfill}%
\pgfsetlinewidth{0.501875pt}%
\definecolor{currentstroke}{rgb}{0.000000,0.000000,0.000000}%
\pgfsetstrokecolor{currentstroke}%
\pgfsetdash{}{0pt}%
\pgfsys@defobject{currentmarker}{\pgfqpoint{-0.027778in}{0.000000in}}{\pgfqpoint{0.000000in}{0.000000in}}{%
\pgfpathmoveto{\pgfqpoint{0.000000in}{0.000000in}}%
\pgfpathlineto{\pgfqpoint{-0.027778in}{0.000000in}}%
\pgfusepath{stroke,fill}%
}%
\begin{pgfscope}%
\pgfsys@transformshift{7.200000in}{1.489018in}%
\pgfsys@useobject{currentmarker}{}%
\end{pgfscope}%
\end{pgfscope}%
\begin{pgfscope}%
\pgfsetbuttcap%
\pgfsetroundjoin%
\definecolor{currentfill}{rgb}{0.000000,0.000000,0.000000}%
\pgfsetfillcolor{currentfill}%
\pgfsetlinewidth{0.501875pt}%
\definecolor{currentstroke}{rgb}{0.000000,0.000000,0.000000}%
\pgfsetstrokecolor{currentstroke}%
\pgfsetdash{}{0pt}%
\pgfsys@defobject{currentmarker}{\pgfqpoint{0.000000in}{0.000000in}}{\pgfqpoint{0.027778in}{0.000000in}}{%
\pgfpathmoveto{\pgfqpoint{0.000000in}{0.000000in}}%
\pgfpathlineto{\pgfqpoint{0.027778in}{0.000000in}}%
\pgfusepath{stroke,fill}%
}%
\begin{pgfscope}%
\pgfsys@transformshift{1.000000in}{1.548989in}%
\pgfsys@useobject{currentmarker}{}%
\end{pgfscope}%
\end{pgfscope}%
\begin{pgfscope}%
\pgfsetbuttcap%
\pgfsetroundjoin%
\definecolor{currentfill}{rgb}{0.000000,0.000000,0.000000}%
\pgfsetfillcolor{currentfill}%
\pgfsetlinewidth{0.501875pt}%
\definecolor{currentstroke}{rgb}{0.000000,0.000000,0.000000}%
\pgfsetstrokecolor{currentstroke}%
\pgfsetdash{}{0pt}%
\pgfsys@defobject{currentmarker}{\pgfqpoint{-0.027778in}{0.000000in}}{\pgfqpoint{0.000000in}{0.000000in}}{%
\pgfpathmoveto{\pgfqpoint{0.000000in}{0.000000in}}%
\pgfpathlineto{\pgfqpoint{-0.027778in}{0.000000in}}%
\pgfusepath{stroke,fill}%
}%
\begin{pgfscope}%
\pgfsys@transformshift{7.200000in}{1.548989in}%
\pgfsys@useobject{currentmarker}{}%
\end{pgfscope}%
\end{pgfscope}%
\begin{pgfscope}%
\pgfsetbuttcap%
\pgfsetroundjoin%
\definecolor{currentfill}{rgb}{0.000000,0.000000,0.000000}%
\pgfsetfillcolor{currentfill}%
\pgfsetlinewidth{0.501875pt}%
\definecolor{currentstroke}{rgb}{0.000000,0.000000,0.000000}%
\pgfsetstrokecolor{currentstroke}%
\pgfsetdash{}{0pt}%
\pgfsys@defobject{currentmarker}{\pgfqpoint{0.000000in}{0.000000in}}{\pgfqpoint{0.027778in}{0.000000in}}{%
\pgfpathmoveto{\pgfqpoint{0.000000in}{0.000000in}}%
\pgfpathlineto{\pgfqpoint{0.027778in}{0.000000in}}%
\pgfusepath{stroke,fill}%
}%
\begin{pgfscope}%
\pgfsys@transformshift{1.000000in}{1.595506in}%
\pgfsys@useobject{currentmarker}{}%
\end{pgfscope}%
\end{pgfscope}%
\begin{pgfscope}%
\pgfsetbuttcap%
\pgfsetroundjoin%
\definecolor{currentfill}{rgb}{0.000000,0.000000,0.000000}%
\pgfsetfillcolor{currentfill}%
\pgfsetlinewidth{0.501875pt}%
\definecolor{currentstroke}{rgb}{0.000000,0.000000,0.000000}%
\pgfsetstrokecolor{currentstroke}%
\pgfsetdash{}{0pt}%
\pgfsys@defobject{currentmarker}{\pgfqpoint{-0.027778in}{0.000000in}}{\pgfqpoint{0.000000in}{0.000000in}}{%
\pgfpathmoveto{\pgfqpoint{0.000000in}{0.000000in}}%
\pgfpathlineto{\pgfqpoint{-0.027778in}{0.000000in}}%
\pgfusepath{stroke,fill}%
}%
\begin{pgfscope}%
\pgfsys@transformshift{7.200000in}{1.595506in}%
\pgfsys@useobject{currentmarker}{}%
\end{pgfscope}%
\end{pgfscope}%
\begin{pgfscope}%
\pgfsetbuttcap%
\pgfsetroundjoin%
\definecolor{currentfill}{rgb}{0.000000,0.000000,0.000000}%
\pgfsetfillcolor{currentfill}%
\pgfsetlinewidth{0.501875pt}%
\definecolor{currentstroke}{rgb}{0.000000,0.000000,0.000000}%
\pgfsetstrokecolor{currentstroke}%
\pgfsetdash{}{0pt}%
\pgfsys@defobject{currentmarker}{\pgfqpoint{0.000000in}{0.000000in}}{\pgfqpoint{0.027778in}{0.000000in}}{%
\pgfpathmoveto{\pgfqpoint{0.000000in}{0.000000in}}%
\pgfpathlineto{\pgfqpoint{0.027778in}{0.000000in}}%
\pgfusepath{stroke,fill}%
}%
\begin{pgfscope}%
\pgfsys@transformshift{1.000000in}{1.633513in}%
\pgfsys@useobject{currentmarker}{}%
\end{pgfscope}%
\end{pgfscope}%
\begin{pgfscope}%
\pgfsetbuttcap%
\pgfsetroundjoin%
\definecolor{currentfill}{rgb}{0.000000,0.000000,0.000000}%
\pgfsetfillcolor{currentfill}%
\pgfsetlinewidth{0.501875pt}%
\definecolor{currentstroke}{rgb}{0.000000,0.000000,0.000000}%
\pgfsetstrokecolor{currentstroke}%
\pgfsetdash{}{0pt}%
\pgfsys@defobject{currentmarker}{\pgfqpoint{-0.027778in}{0.000000in}}{\pgfqpoint{0.000000in}{0.000000in}}{%
\pgfpathmoveto{\pgfqpoint{0.000000in}{0.000000in}}%
\pgfpathlineto{\pgfqpoint{-0.027778in}{0.000000in}}%
\pgfusepath{stroke,fill}%
}%
\begin{pgfscope}%
\pgfsys@transformshift{7.200000in}{1.633513in}%
\pgfsys@useobject{currentmarker}{}%
\end{pgfscope}%
\end{pgfscope}%
\begin{pgfscope}%
\pgfsetbuttcap%
\pgfsetroundjoin%
\definecolor{currentfill}{rgb}{0.000000,0.000000,0.000000}%
\pgfsetfillcolor{currentfill}%
\pgfsetlinewidth{0.501875pt}%
\definecolor{currentstroke}{rgb}{0.000000,0.000000,0.000000}%
\pgfsetstrokecolor{currentstroke}%
\pgfsetdash{}{0pt}%
\pgfsys@defobject{currentmarker}{\pgfqpoint{0.000000in}{0.000000in}}{\pgfqpoint{0.027778in}{0.000000in}}{%
\pgfpathmoveto{\pgfqpoint{0.000000in}{0.000000in}}%
\pgfpathlineto{\pgfqpoint{0.027778in}{0.000000in}}%
\pgfusepath{stroke,fill}%
}%
\begin{pgfscope}%
\pgfsys@transformshift{1.000000in}{1.665647in}%
\pgfsys@useobject{currentmarker}{}%
\end{pgfscope}%
\end{pgfscope}%
\begin{pgfscope}%
\pgfsetbuttcap%
\pgfsetroundjoin%
\definecolor{currentfill}{rgb}{0.000000,0.000000,0.000000}%
\pgfsetfillcolor{currentfill}%
\pgfsetlinewidth{0.501875pt}%
\definecolor{currentstroke}{rgb}{0.000000,0.000000,0.000000}%
\pgfsetstrokecolor{currentstroke}%
\pgfsetdash{}{0pt}%
\pgfsys@defobject{currentmarker}{\pgfqpoint{-0.027778in}{0.000000in}}{\pgfqpoint{0.000000in}{0.000000in}}{%
\pgfpathmoveto{\pgfqpoint{0.000000in}{0.000000in}}%
\pgfpathlineto{\pgfqpoint{-0.027778in}{0.000000in}}%
\pgfusepath{stroke,fill}%
}%
\begin{pgfscope}%
\pgfsys@transformshift{7.200000in}{1.665647in}%
\pgfsys@useobject{currentmarker}{}%
\end{pgfscope}%
\end{pgfscope}%
\begin{pgfscope}%
\pgfsetbuttcap%
\pgfsetroundjoin%
\definecolor{currentfill}{rgb}{0.000000,0.000000,0.000000}%
\pgfsetfillcolor{currentfill}%
\pgfsetlinewidth{0.501875pt}%
\definecolor{currentstroke}{rgb}{0.000000,0.000000,0.000000}%
\pgfsetstrokecolor{currentstroke}%
\pgfsetdash{}{0pt}%
\pgfsys@defobject{currentmarker}{\pgfqpoint{0.000000in}{0.000000in}}{\pgfqpoint{0.027778in}{0.000000in}}{%
\pgfpathmoveto{\pgfqpoint{0.000000in}{0.000000in}}%
\pgfpathlineto{\pgfqpoint{0.027778in}{0.000000in}}%
\pgfusepath{stroke,fill}%
}%
\begin{pgfscope}%
\pgfsys@transformshift{1.000000in}{1.693483in}%
\pgfsys@useobject{currentmarker}{}%
\end{pgfscope}%
\end{pgfscope}%
\begin{pgfscope}%
\pgfsetbuttcap%
\pgfsetroundjoin%
\definecolor{currentfill}{rgb}{0.000000,0.000000,0.000000}%
\pgfsetfillcolor{currentfill}%
\pgfsetlinewidth{0.501875pt}%
\definecolor{currentstroke}{rgb}{0.000000,0.000000,0.000000}%
\pgfsetstrokecolor{currentstroke}%
\pgfsetdash{}{0pt}%
\pgfsys@defobject{currentmarker}{\pgfqpoint{-0.027778in}{0.000000in}}{\pgfqpoint{0.000000in}{0.000000in}}{%
\pgfpathmoveto{\pgfqpoint{0.000000in}{0.000000in}}%
\pgfpathlineto{\pgfqpoint{-0.027778in}{0.000000in}}%
\pgfusepath{stroke,fill}%
}%
\begin{pgfscope}%
\pgfsys@transformshift{7.200000in}{1.693483in}%
\pgfsys@useobject{currentmarker}{}%
\end{pgfscope}%
\end{pgfscope}%
\begin{pgfscope}%
\pgfsetbuttcap%
\pgfsetroundjoin%
\definecolor{currentfill}{rgb}{0.000000,0.000000,0.000000}%
\pgfsetfillcolor{currentfill}%
\pgfsetlinewidth{0.501875pt}%
\definecolor{currentstroke}{rgb}{0.000000,0.000000,0.000000}%
\pgfsetstrokecolor{currentstroke}%
\pgfsetdash{}{0pt}%
\pgfsys@defobject{currentmarker}{\pgfqpoint{0.000000in}{0.000000in}}{\pgfqpoint{0.027778in}{0.000000in}}{%
\pgfpathmoveto{\pgfqpoint{0.000000in}{0.000000in}}%
\pgfpathlineto{\pgfqpoint{0.027778in}{0.000000in}}%
\pgfusepath{stroke,fill}%
}%
\begin{pgfscope}%
\pgfsys@transformshift{1.000000in}{1.718036in}%
\pgfsys@useobject{currentmarker}{}%
\end{pgfscope}%
\end{pgfscope}%
\begin{pgfscope}%
\pgfsetbuttcap%
\pgfsetroundjoin%
\definecolor{currentfill}{rgb}{0.000000,0.000000,0.000000}%
\pgfsetfillcolor{currentfill}%
\pgfsetlinewidth{0.501875pt}%
\definecolor{currentstroke}{rgb}{0.000000,0.000000,0.000000}%
\pgfsetstrokecolor{currentstroke}%
\pgfsetdash{}{0pt}%
\pgfsys@defobject{currentmarker}{\pgfqpoint{-0.027778in}{0.000000in}}{\pgfqpoint{0.000000in}{0.000000in}}{%
\pgfpathmoveto{\pgfqpoint{0.000000in}{0.000000in}}%
\pgfpathlineto{\pgfqpoint{-0.027778in}{0.000000in}}%
\pgfusepath{stroke,fill}%
}%
\begin{pgfscope}%
\pgfsys@transformshift{7.200000in}{1.718036in}%
\pgfsys@useobject{currentmarker}{}%
\end{pgfscope}%
\end{pgfscope}%
\begin{pgfscope}%
\pgfsetbuttcap%
\pgfsetroundjoin%
\definecolor{currentfill}{rgb}{0.000000,0.000000,0.000000}%
\pgfsetfillcolor{currentfill}%
\pgfsetlinewidth{0.501875pt}%
\definecolor{currentstroke}{rgb}{0.000000,0.000000,0.000000}%
\pgfsetstrokecolor{currentstroke}%
\pgfsetdash{}{0pt}%
\pgfsys@defobject{currentmarker}{\pgfqpoint{0.000000in}{0.000000in}}{\pgfqpoint{0.027778in}{0.000000in}}{%
\pgfpathmoveto{\pgfqpoint{0.000000in}{0.000000in}}%
\pgfpathlineto{\pgfqpoint{0.027778in}{0.000000in}}%
\pgfusepath{stroke,fill}%
}%
\begin{pgfscope}%
\pgfsys@transformshift{1.000000in}{1.884494in}%
\pgfsys@useobject{currentmarker}{}%
\end{pgfscope}%
\end{pgfscope}%
\begin{pgfscope}%
\pgfsetbuttcap%
\pgfsetroundjoin%
\definecolor{currentfill}{rgb}{0.000000,0.000000,0.000000}%
\pgfsetfillcolor{currentfill}%
\pgfsetlinewidth{0.501875pt}%
\definecolor{currentstroke}{rgb}{0.000000,0.000000,0.000000}%
\pgfsetstrokecolor{currentstroke}%
\pgfsetdash{}{0pt}%
\pgfsys@defobject{currentmarker}{\pgfqpoint{-0.027778in}{0.000000in}}{\pgfqpoint{0.000000in}{0.000000in}}{%
\pgfpathmoveto{\pgfqpoint{0.000000in}{0.000000in}}%
\pgfpathlineto{\pgfqpoint{-0.027778in}{0.000000in}}%
\pgfusepath{stroke,fill}%
}%
\begin{pgfscope}%
\pgfsys@transformshift{7.200000in}{1.884494in}%
\pgfsys@useobject{currentmarker}{}%
\end{pgfscope}%
\end{pgfscope}%
\begin{pgfscope}%
\pgfsetbuttcap%
\pgfsetroundjoin%
\definecolor{currentfill}{rgb}{0.000000,0.000000,0.000000}%
\pgfsetfillcolor{currentfill}%
\pgfsetlinewidth{0.501875pt}%
\definecolor{currentstroke}{rgb}{0.000000,0.000000,0.000000}%
\pgfsetstrokecolor{currentstroke}%
\pgfsetdash{}{0pt}%
\pgfsys@defobject{currentmarker}{\pgfqpoint{0.000000in}{0.000000in}}{\pgfqpoint{0.027778in}{0.000000in}}{%
\pgfpathmoveto{\pgfqpoint{0.000000in}{0.000000in}}%
\pgfpathlineto{\pgfqpoint{0.027778in}{0.000000in}}%
\pgfusepath{stroke,fill}%
}%
\begin{pgfscope}%
\pgfsys@transformshift{1.000000in}{1.969018in}%
\pgfsys@useobject{currentmarker}{}%
\end{pgfscope}%
\end{pgfscope}%
\begin{pgfscope}%
\pgfsetbuttcap%
\pgfsetroundjoin%
\definecolor{currentfill}{rgb}{0.000000,0.000000,0.000000}%
\pgfsetfillcolor{currentfill}%
\pgfsetlinewidth{0.501875pt}%
\definecolor{currentstroke}{rgb}{0.000000,0.000000,0.000000}%
\pgfsetstrokecolor{currentstroke}%
\pgfsetdash{}{0pt}%
\pgfsys@defobject{currentmarker}{\pgfqpoint{-0.027778in}{0.000000in}}{\pgfqpoint{0.000000in}{0.000000in}}{%
\pgfpathmoveto{\pgfqpoint{0.000000in}{0.000000in}}%
\pgfpathlineto{\pgfqpoint{-0.027778in}{0.000000in}}%
\pgfusepath{stroke,fill}%
}%
\begin{pgfscope}%
\pgfsys@transformshift{7.200000in}{1.969018in}%
\pgfsys@useobject{currentmarker}{}%
\end{pgfscope}%
\end{pgfscope}%
\begin{pgfscope}%
\pgfsetbuttcap%
\pgfsetroundjoin%
\definecolor{currentfill}{rgb}{0.000000,0.000000,0.000000}%
\pgfsetfillcolor{currentfill}%
\pgfsetlinewidth{0.501875pt}%
\definecolor{currentstroke}{rgb}{0.000000,0.000000,0.000000}%
\pgfsetstrokecolor{currentstroke}%
\pgfsetdash{}{0pt}%
\pgfsys@defobject{currentmarker}{\pgfqpoint{0.000000in}{0.000000in}}{\pgfqpoint{0.027778in}{0.000000in}}{%
\pgfpathmoveto{\pgfqpoint{0.000000in}{0.000000in}}%
\pgfpathlineto{\pgfqpoint{0.027778in}{0.000000in}}%
\pgfusepath{stroke,fill}%
}%
\begin{pgfscope}%
\pgfsys@transformshift{1.000000in}{2.028989in}%
\pgfsys@useobject{currentmarker}{}%
\end{pgfscope}%
\end{pgfscope}%
\begin{pgfscope}%
\pgfsetbuttcap%
\pgfsetroundjoin%
\definecolor{currentfill}{rgb}{0.000000,0.000000,0.000000}%
\pgfsetfillcolor{currentfill}%
\pgfsetlinewidth{0.501875pt}%
\definecolor{currentstroke}{rgb}{0.000000,0.000000,0.000000}%
\pgfsetstrokecolor{currentstroke}%
\pgfsetdash{}{0pt}%
\pgfsys@defobject{currentmarker}{\pgfqpoint{-0.027778in}{0.000000in}}{\pgfqpoint{0.000000in}{0.000000in}}{%
\pgfpathmoveto{\pgfqpoint{0.000000in}{0.000000in}}%
\pgfpathlineto{\pgfqpoint{-0.027778in}{0.000000in}}%
\pgfusepath{stroke,fill}%
}%
\begin{pgfscope}%
\pgfsys@transformshift{7.200000in}{2.028989in}%
\pgfsys@useobject{currentmarker}{}%
\end{pgfscope}%
\end{pgfscope}%
\begin{pgfscope}%
\pgfsetbuttcap%
\pgfsetroundjoin%
\definecolor{currentfill}{rgb}{0.000000,0.000000,0.000000}%
\pgfsetfillcolor{currentfill}%
\pgfsetlinewidth{0.501875pt}%
\definecolor{currentstroke}{rgb}{0.000000,0.000000,0.000000}%
\pgfsetstrokecolor{currentstroke}%
\pgfsetdash{}{0pt}%
\pgfsys@defobject{currentmarker}{\pgfqpoint{0.000000in}{0.000000in}}{\pgfqpoint{0.027778in}{0.000000in}}{%
\pgfpathmoveto{\pgfqpoint{0.000000in}{0.000000in}}%
\pgfpathlineto{\pgfqpoint{0.027778in}{0.000000in}}%
\pgfusepath{stroke,fill}%
}%
\begin{pgfscope}%
\pgfsys@transformshift{1.000000in}{2.075506in}%
\pgfsys@useobject{currentmarker}{}%
\end{pgfscope}%
\end{pgfscope}%
\begin{pgfscope}%
\pgfsetbuttcap%
\pgfsetroundjoin%
\definecolor{currentfill}{rgb}{0.000000,0.000000,0.000000}%
\pgfsetfillcolor{currentfill}%
\pgfsetlinewidth{0.501875pt}%
\definecolor{currentstroke}{rgb}{0.000000,0.000000,0.000000}%
\pgfsetstrokecolor{currentstroke}%
\pgfsetdash{}{0pt}%
\pgfsys@defobject{currentmarker}{\pgfqpoint{-0.027778in}{0.000000in}}{\pgfqpoint{0.000000in}{0.000000in}}{%
\pgfpathmoveto{\pgfqpoint{0.000000in}{0.000000in}}%
\pgfpathlineto{\pgfqpoint{-0.027778in}{0.000000in}}%
\pgfusepath{stroke,fill}%
}%
\begin{pgfscope}%
\pgfsys@transformshift{7.200000in}{2.075506in}%
\pgfsys@useobject{currentmarker}{}%
\end{pgfscope}%
\end{pgfscope}%
\begin{pgfscope}%
\pgfsetbuttcap%
\pgfsetroundjoin%
\definecolor{currentfill}{rgb}{0.000000,0.000000,0.000000}%
\pgfsetfillcolor{currentfill}%
\pgfsetlinewidth{0.501875pt}%
\definecolor{currentstroke}{rgb}{0.000000,0.000000,0.000000}%
\pgfsetstrokecolor{currentstroke}%
\pgfsetdash{}{0pt}%
\pgfsys@defobject{currentmarker}{\pgfqpoint{0.000000in}{0.000000in}}{\pgfqpoint{0.027778in}{0.000000in}}{%
\pgfpathmoveto{\pgfqpoint{0.000000in}{0.000000in}}%
\pgfpathlineto{\pgfqpoint{0.027778in}{0.000000in}}%
\pgfusepath{stroke,fill}%
}%
\begin{pgfscope}%
\pgfsys@transformshift{1.000000in}{2.113513in}%
\pgfsys@useobject{currentmarker}{}%
\end{pgfscope}%
\end{pgfscope}%
\begin{pgfscope}%
\pgfsetbuttcap%
\pgfsetroundjoin%
\definecolor{currentfill}{rgb}{0.000000,0.000000,0.000000}%
\pgfsetfillcolor{currentfill}%
\pgfsetlinewidth{0.501875pt}%
\definecolor{currentstroke}{rgb}{0.000000,0.000000,0.000000}%
\pgfsetstrokecolor{currentstroke}%
\pgfsetdash{}{0pt}%
\pgfsys@defobject{currentmarker}{\pgfqpoint{-0.027778in}{0.000000in}}{\pgfqpoint{0.000000in}{0.000000in}}{%
\pgfpathmoveto{\pgfqpoint{0.000000in}{0.000000in}}%
\pgfpathlineto{\pgfqpoint{-0.027778in}{0.000000in}}%
\pgfusepath{stroke,fill}%
}%
\begin{pgfscope}%
\pgfsys@transformshift{7.200000in}{2.113513in}%
\pgfsys@useobject{currentmarker}{}%
\end{pgfscope}%
\end{pgfscope}%
\begin{pgfscope}%
\pgfsetbuttcap%
\pgfsetroundjoin%
\definecolor{currentfill}{rgb}{0.000000,0.000000,0.000000}%
\pgfsetfillcolor{currentfill}%
\pgfsetlinewidth{0.501875pt}%
\definecolor{currentstroke}{rgb}{0.000000,0.000000,0.000000}%
\pgfsetstrokecolor{currentstroke}%
\pgfsetdash{}{0pt}%
\pgfsys@defobject{currentmarker}{\pgfqpoint{0.000000in}{0.000000in}}{\pgfqpoint{0.027778in}{0.000000in}}{%
\pgfpathmoveto{\pgfqpoint{0.000000in}{0.000000in}}%
\pgfpathlineto{\pgfqpoint{0.027778in}{0.000000in}}%
\pgfusepath{stroke,fill}%
}%
\begin{pgfscope}%
\pgfsys@transformshift{1.000000in}{2.145647in}%
\pgfsys@useobject{currentmarker}{}%
\end{pgfscope}%
\end{pgfscope}%
\begin{pgfscope}%
\pgfsetbuttcap%
\pgfsetroundjoin%
\definecolor{currentfill}{rgb}{0.000000,0.000000,0.000000}%
\pgfsetfillcolor{currentfill}%
\pgfsetlinewidth{0.501875pt}%
\definecolor{currentstroke}{rgb}{0.000000,0.000000,0.000000}%
\pgfsetstrokecolor{currentstroke}%
\pgfsetdash{}{0pt}%
\pgfsys@defobject{currentmarker}{\pgfqpoint{-0.027778in}{0.000000in}}{\pgfqpoint{0.000000in}{0.000000in}}{%
\pgfpathmoveto{\pgfqpoint{0.000000in}{0.000000in}}%
\pgfpathlineto{\pgfqpoint{-0.027778in}{0.000000in}}%
\pgfusepath{stroke,fill}%
}%
\begin{pgfscope}%
\pgfsys@transformshift{7.200000in}{2.145647in}%
\pgfsys@useobject{currentmarker}{}%
\end{pgfscope}%
\end{pgfscope}%
\begin{pgfscope}%
\pgfsetbuttcap%
\pgfsetroundjoin%
\definecolor{currentfill}{rgb}{0.000000,0.000000,0.000000}%
\pgfsetfillcolor{currentfill}%
\pgfsetlinewidth{0.501875pt}%
\definecolor{currentstroke}{rgb}{0.000000,0.000000,0.000000}%
\pgfsetstrokecolor{currentstroke}%
\pgfsetdash{}{0pt}%
\pgfsys@defobject{currentmarker}{\pgfqpoint{0.000000in}{0.000000in}}{\pgfqpoint{0.027778in}{0.000000in}}{%
\pgfpathmoveto{\pgfqpoint{0.000000in}{0.000000in}}%
\pgfpathlineto{\pgfqpoint{0.027778in}{0.000000in}}%
\pgfusepath{stroke,fill}%
}%
\begin{pgfscope}%
\pgfsys@transformshift{1.000000in}{2.173483in}%
\pgfsys@useobject{currentmarker}{}%
\end{pgfscope}%
\end{pgfscope}%
\begin{pgfscope}%
\pgfsetbuttcap%
\pgfsetroundjoin%
\definecolor{currentfill}{rgb}{0.000000,0.000000,0.000000}%
\pgfsetfillcolor{currentfill}%
\pgfsetlinewidth{0.501875pt}%
\definecolor{currentstroke}{rgb}{0.000000,0.000000,0.000000}%
\pgfsetstrokecolor{currentstroke}%
\pgfsetdash{}{0pt}%
\pgfsys@defobject{currentmarker}{\pgfqpoint{-0.027778in}{0.000000in}}{\pgfqpoint{0.000000in}{0.000000in}}{%
\pgfpathmoveto{\pgfqpoint{0.000000in}{0.000000in}}%
\pgfpathlineto{\pgfqpoint{-0.027778in}{0.000000in}}%
\pgfusepath{stroke,fill}%
}%
\begin{pgfscope}%
\pgfsys@transformshift{7.200000in}{2.173483in}%
\pgfsys@useobject{currentmarker}{}%
\end{pgfscope}%
\end{pgfscope}%
\begin{pgfscope}%
\pgfsetbuttcap%
\pgfsetroundjoin%
\definecolor{currentfill}{rgb}{0.000000,0.000000,0.000000}%
\pgfsetfillcolor{currentfill}%
\pgfsetlinewidth{0.501875pt}%
\definecolor{currentstroke}{rgb}{0.000000,0.000000,0.000000}%
\pgfsetstrokecolor{currentstroke}%
\pgfsetdash{}{0pt}%
\pgfsys@defobject{currentmarker}{\pgfqpoint{0.000000in}{0.000000in}}{\pgfqpoint{0.027778in}{0.000000in}}{%
\pgfpathmoveto{\pgfqpoint{0.000000in}{0.000000in}}%
\pgfpathlineto{\pgfqpoint{0.027778in}{0.000000in}}%
\pgfusepath{stroke,fill}%
}%
\begin{pgfscope}%
\pgfsys@transformshift{1.000000in}{2.198036in}%
\pgfsys@useobject{currentmarker}{}%
\end{pgfscope}%
\end{pgfscope}%
\begin{pgfscope}%
\pgfsetbuttcap%
\pgfsetroundjoin%
\definecolor{currentfill}{rgb}{0.000000,0.000000,0.000000}%
\pgfsetfillcolor{currentfill}%
\pgfsetlinewidth{0.501875pt}%
\definecolor{currentstroke}{rgb}{0.000000,0.000000,0.000000}%
\pgfsetstrokecolor{currentstroke}%
\pgfsetdash{}{0pt}%
\pgfsys@defobject{currentmarker}{\pgfqpoint{-0.027778in}{0.000000in}}{\pgfqpoint{0.000000in}{0.000000in}}{%
\pgfpathmoveto{\pgfqpoint{0.000000in}{0.000000in}}%
\pgfpathlineto{\pgfqpoint{-0.027778in}{0.000000in}}%
\pgfusepath{stroke,fill}%
}%
\begin{pgfscope}%
\pgfsys@transformshift{7.200000in}{2.198036in}%
\pgfsys@useobject{currentmarker}{}%
\end{pgfscope}%
\end{pgfscope}%
\begin{pgfscope}%
\pgfsetbuttcap%
\pgfsetroundjoin%
\definecolor{currentfill}{rgb}{0.000000,0.000000,0.000000}%
\pgfsetfillcolor{currentfill}%
\pgfsetlinewidth{0.501875pt}%
\definecolor{currentstroke}{rgb}{0.000000,0.000000,0.000000}%
\pgfsetstrokecolor{currentstroke}%
\pgfsetdash{}{0pt}%
\pgfsys@defobject{currentmarker}{\pgfqpoint{0.000000in}{0.000000in}}{\pgfqpoint{0.027778in}{0.000000in}}{%
\pgfpathmoveto{\pgfqpoint{0.000000in}{0.000000in}}%
\pgfpathlineto{\pgfqpoint{0.027778in}{0.000000in}}%
\pgfusepath{stroke,fill}%
}%
\begin{pgfscope}%
\pgfsys@transformshift{1.000000in}{2.364494in}%
\pgfsys@useobject{currentmarker}{}%
\end{pgfscope}%
\end{pgfscope}%
\begin{pgfscope}%
\pgfsetbuttcap%
\pgfsetroundjoin%
\definecolor{currentfill}{rgb}{0.000000,0.000000,0.000000}%
\pgfsetfillcolor{currentfill}%
\pgfsetlinewidth{0.501875pt}%
\definecolor{currentstroke}{rgb}{0.000000,0.000000,0.000000}%
\pgfsetstrokecolor{currentstroke}%
\pgfsetdash{}{0pt}%
\pgfsys@defobject{currentmarker}{\pgfqpoint{-0.027778in}{0.000000in}}{\pgfqpoint{0.000000in}{0.000000in}}{%
\pgfpathmoveto{\pgfqpoint{0.000000in}{0.000000in}}%
\pgfpathlineto{\pgfqpoint{-0.027778in}{0.000000in}}%
\pgfusepath{stroke,fill}%
}%
\begin{pgfscope}%
\pgfsys@transformshift{7.200000in}{2.364494in}%
\pgfsys@useobject{currentmarker}{}%
\end{pgfscope}%
\end{pgfscope}%
\begin{pgfscope}%
\pgfsetbuttcap%
\pgfsetroundjoin%
\definecolor{currentfill}{rgb}{0.000000,0.000000,0.000000}%
\pgfsetfillcolor{currentfill}%
\pgfsetlinewidth{0.501875pt}%
\definecolor{currentstroke}{rgb}{0.000000,0.000000,0.000000}%
\pgfsetstrokecolor{currentstroke}%
\pgfsetdash{}{0pt}%
\pgfsys@defobject{currentmarker}{\pgfqpoint{0.000000in}{0.000000in}}{\pgfqpoint{0.027778in}{0.000000in}}{%
\pgfpathmoveto{\pgfqpoint{0.000000in}{0.000000in}}%
\pgfpathlineto{\pgfqpoint{0.027778in}{0.000000in}}%
\pgfusepath{stroke,fill}%
}%
\begin{pgfscope}%
\pgfsys@transformshift{1.000000in}{2.449018in}%
\pgfsys@useobject{currentmarker}{}%
\end{pgfscope}%
\end{pgfscope}%
\begin{pgfscope}%
\pgfsetbuttcap%
\pgfsetroundjoin%
\definecolor{currentfill}{rgb}{0.000000,0.000000,0.000000}%
\pgfsetfillcolor{currentfill}%
\pgfsetlinewidth{0.501875pt}%
\definecolor{currentstroke}{rgb}{0.000000,0.000000,0.000000}%
\pgfsetstrokecolor{currentstroke}%
\pgfsetdash{}{0pt}%
\pgfsys@defobject{currentmarker}{\pgfqpoint{-0.027778in}{0.000000in}}{\pgfqpoint{0.000000in}{0.000000in}}{%
\pgfpathmoveto{\pgfqpoint{0.000000in}{0.000000in}}%
\pgfpathlineto{\pgfqpoint{-0.027778in}{0.000000in}}%
\pgfusepath{stroke,fill}%
}%
\begin{pgfscope}%
\pgfsys@transformshift{7.200000in}{2.449018in}%
\pgfsys@useobject{currentmarker}{}%
\end{pgfscope}%
\end{pgfscope}%
\begin{pgfscope}%
\pgfsetbuttcap%
\pgfsetroundjoin%
\definecolor{currentfill}{rgb}{0.000000,0.000000,0.000000}%
\pgfsetfillcolor{currentfill}%
\pgfsetlinewidth{0.501875pt}%
\definecolor{currentstroke}{rgb}{0.000000,0.000000,0.000000}%
\pgfsetstrokecolor{currentstroke}%
\pgfsetdash{}{0pt}%
\pgfsys@defobject{currentmarker}{\pgfqpoint{0.000000in}{0.000000in}}{\pgfqpoint{0.027778in}{0.000000in}}{%
\pgfpathmoveto{\pgfqpoint{0.000000in}{0.000000in}}%
\pgfpathlineto{\pgfqpoint{0.027778in}{0.000000in}}%
\pgfusepath{stroke,fill}%
}%
\begin{pgfscope}%
\pgfsys@transformshift{1.000000in}{2.508989in}%
\pgfsys@useobject{currentmarker}{}%
\end{pgfscope}%
\end{pgfscope}%
\begin{pgfscope}%
\pgfsetbuttcap%
\pgfsetroundjoin%
\definecolor{currentfill}{rgb}{0.000000,0.000000,0.000000}%
\pgfsetfillcolor{currentfill}%
\pgfsetlinewidth{0.501875pt}%
\definecolor{currentstroke}{rgb}{0.000000,0.000000,0.000000}%
\pgfsetstrokecolor{currentstroke}%
\pgfsetdash{}{0pt}%
\pgfsys@defobject{currentmarker}{\pgfqpoint{-0.027778in}{0.000000in}}{\pgfqpoint{0.000000in}{0.000000in}}{%
\pgfpathmoveto{\pgfqpoint{0.000000in}{0.000000in}}%
\pgfpathlineto{\pgfqpoint{-0.027778in}{0.000000in}}%
\pgfusepath{stroke,fill}%
}%
\begin{pgfscope}%
\pgfsys@transformshift{7.200000in}{2.508989in}%
\pgfsys@useobject{currentmarker}{}%
\end{pgfscope}%
\end{pgfscope}%
\begin{pgfscope}%
\pgfsetbuttcap%
\pgfsetroundjoin%
\definecolor{currentfill}{rgb}{0.000000,0.000000,0.000000}%
\pgfsetfillcolor{currentfill}%
\pgfsetlinewidth{0.501875pt}%
\definecolor{currentstroke}{rgb}{0.000000,0.000000,0.000000}%
\pgfsetstrokecolor{currentstroke}%
\pgfsetdash{}{0pt}%
\pgfsys@defobject{currentmarker}{\pgfqpoint{0.000000in}{0.000000in}}{\pgfqpoint{0.027778in}{0.000000in}}{%
\pgfpathmoveto{\pgfqpoint{0.000000in}{0.000000in}}%
\pgfpathlineto{\pgfqpoint{0.027778in}{0.000000in}}%
\pgfusepath{stroke,fill}%
}%
\begin{pgfscope}%
\pgfsys@transformshift{1.000000in}{2.555506in}%
\pgfsys@useobject{currentmarker}{}%
\end{pgfscope}%
\end{pgfscope}%
\begin{pgfscope}%
\pgfsetbuttcap%
\pgfsetroundjoin%
\definecolor{currentfill}{rgb}{0.000000,0.000000,0.000000}%
\pgfsetfillcolor{currentfill}%
\pgfsetlinewidth{0.501875pt}%
\definecolor{currentstroke}{rgb}{0.000000,0.000000,0.000000}%
\pgfsetstrokecolor{currentstroke}%
\pgfsetdash{}{0pt}%
\pgfsys@defobject{currentmarker}{\pgfqpoint{-0.027778in}{0.000000in}}{\pgfqpoint{0.000000in}{0.000000in}}{%
\pgfpathmoveto{\pgfqpoint{0.000000in}{0.000000in}}%
\pgfpathlineto{\pgfqpoint{-0.027778in}{0.000000in}}%
\pgfusepath{stroke,fill}%
}%
\begin{pgfscope}%
\pgfsys@transformshift{7.200000in}{2.555506in}%
\pgfsys@useobject{currentmarker}{}%
\end{pgfscope}%
\end{pgfscope}%
\begin{pgfscope}%
\pgfsetbuttcap%
\pgfsetroundjoin%
\definecolor{currentfill}{rgb}{0.000000,0.000000,0.000000}%
\pgfsetfillcolor{currentfill}%
\pgfsetlinewidth{0.501875pt}%
\definecolor{currentstroke}{rgb}{0.000000,0.000000,0.000000}%
\pgfsetstrokecolor{currentstroke}%
\pgfsetdash{}{0pt}%
\pgfsys@defobject{currentmarker}{\pgfqpoint{0.000000in}{0.000000in}}{\pgfqpoint{0.027778in}{0.000000in}}{%
\pgfpathmoveto{\pgfqpoint{0.000000in}{0.000000in}}%
\pgfpathlineto{\pgfqpoint{0.027778in}{0.000000in}}%
\pgfusepath{stroke,fill}%
}%
\begin{pgfscope}%
\pgfsys@transformshift{1.000000in}{2.593513in}%
\pgfsys@useobject{currentmarker}{}%
\end{pgfscope}%
\end{pgfscope}%
\begin{pgfscope}%
\pgfsetbuttcap%
\pgfsetroundjoin%
\definecolor{currentfill}{rgb}{0.000000,0.000000,0.000000}%
\pgfsetfillcolor{currentfill}%
\pgfsetlinewidth{0.501875pt}%
\definecolor{currentstroke}{rgb}{0.000000,0.000000,0.000000}%
\pgfsetstrokecolor{currentstroke}%
\pgfsetdash{}{0pt}%
\pgfsys@defobject{currentmarker}{\pgfqpoint{-0.027778in}{0.000000in}}{\pgfqpoint{0.000000in}{0.000000in}}{%
\pgfpathmoveto{\pgfqpoint{0.000000in}{0.000000in}}%
\pgfpathlineto{\pgfqpoint{-0.027778in}{0.000000in}}%
\pgfusepath{stroke,fill}%
}%
\begin{pgfscope}%
\pgfsys@transformshift{7.200000in}{2.593513in}%
\pgfsys@useobject{currentmarker}{}%
\end{pgfscope}%
\end{pgfscope}%
\begin{pgfscope}%
\pgfsetbuttcap%
\pgfsetroundjoin%
\definecolor{currentfill}{rgb}{0.000000,0.000000,0.000000}%
\pgfsetfillcolor{currentfill}%
\pgfsetlinewidth{0.501875pt}%
\definecolor{currentstroke}{rgb}{0.000000,0.000000,0.000000}%
\pgfsetstrokecolor{currentstroke}%
\pgfsetdash{}{0pt}%
\pgfsys@defobject{currentmarker}{\pgfqpoint{0.000000in}{0.000000in}}{\pgfqpoint{0.027778in}{0.000000in}}{%
\pgfpathmoveto{\pgfqpoint{0.000000in}{0.000000in}}%
\pgfpathlineto{\pgfqpoint{0.027778in}{0.000000in}}%
\pgfusepath{stroke,fill}%
}%
\begin{pgfscope}%
\pgfsys@transformshift{1.000000in}{2.625647in}%
\pgfsys@useobject{currentmarker}{}%
\end{pgfscope}%
\end{pgfscope}%
\begin{pgfscope}%
\pgfsetbuttcap%
\pgfsetroundjoin%
\definecolor{currentfill}{rgb}{0.000000,0.000000,0.000000}%
\pgfsetfillcolor{currentfill}%
\pgfsetlinewidth{0.501875pt}%
\definecolor{currentstroke}{rgb}{0.000000,0.000000,0.000000}%
\pgfsetstrokecolor{currentstroke}%
\pgfsetdash{}{0pt}%
\pgfsys@defobject{currentmarker}{\pgfqpoint{-0.027778in}{0.000000in}}{\pgfqpoint{0.000000in}{0.000000in}}{%
\pgfpathmoveto{\pgfqpoint{0.000000in}{0.000000in}}%
\pgfpathlineto{\pgfqpoint{-0.027778in}{0.000000in}}%
\pgfusepath{stroke,fill}%
}%
\begin{pgfscope}%
\pgfsys@transformshift{7.200000in}{2.625647in}%
\pgfsys@useobject{currentmarker}{}%
\end{pgfscope}%
\end{pgfscope}%
\begin{pgfscope}%
\pgfsetbuttcap%
\pgfsetroundjoin%
\definecolor{currentfill}{rgb}{0.000000,0.000000,0.000000}%
\pgfsetfillcolor{currentfill}%
\pgfsetlinewidth{0.501875pt}%
\definecolor{currentstroke}{rgb}{0.000000,0.000000,0.000000}%
\pgfsetstrokecolor{currentstroke}%
\pgfsetdash{}{0pt}%
\pgfsys@defobject{currentmarker}{\pgfqpoint{0.000000in}{0.000000in}}{\pgfqpoint{0.027778in}{0.000000in}}{%
\pgfpathmoveto{\pgfqpoint{0.000000in}{0.000000in}}%
\pgfpathlineto{\pgfqpoint{0.027778in}{0.000000in}}%
\pgfusepath{stroke,fill}%
}%
\begin{pgfscope}%
\pgfsys@transformshift{1.000000in}{2.653483in}%
\pgfsys@useobject{currentmarker}{}%
\end{pgfscope}%
\end{pgfscope}%
\begin{pgfscope}%
\pgfsetbuttcap%
\pgfsetroundjoin%
\definecolor{currentfill}{rgb}{0.000000,0.000000,0.000000}%
\pgfsetfillcolor{currentfill}%
\pgfsetlinewidth{0.501875pt}%
\definecolor{currentstroke}{rgb}{0.000000,0.000000,0.000000}%
\pgfsetstrokecolor{currentstroke}%
\pgfsetdash{}{0pt}%
\pgfsys@defobject{currentmarker}{\pgfqpoint{-0.027778in}{0.000000in}}{\pgfqpoint{0.000000in}{0.000000in}}{%
\pgfpathmoveto{\pgfqpoint{0.000000in}{0.000000in}}%
\pgfpathlineto{\pgfqpoint{-0.027778in}{0.000000in}}%
\pgfusepath{stroke,fill}%
}%
\begin{pgfscope}%
\pgfsys@transformshift{7.200000in}{2.653483in}%
\pgfsys@useobject{currentmarker}{}%
\end{pgfscope}%
\end{pgfscope}%
\begin{pgfscope}%
\pgfsetbuttcap%
\pgfsetroundjoin%
\definecolor{currentfill}{rgb}{0.000000,0.000000,0.000000}%
\pgfsetfillcolor{currentfill}%
\pgfsetlinewidth{0.501875pt}%
\definecolor{currentstroke}{rgb}{0.000000,0.000000,0.000000}%
\pgfsetstrokecolor{currentstroke}%
\pgfsetdash{}{0pt}%
\pgfsys@defobject{currentmarker}{\pgfqpoint{0.000000in}{0.000000in}}{\pgfqpoint{0.027778in}{0.000000in}}{%
\pgfpathmoveto{\pgfqpoint{0.000000in}{0.000000in}}%
\pgfpathlineto{\pgfqpoint{0.027778in}{0.000000in}}%
\pgfusepath{stroke,fill}%
}%
\begin{pgfscope}%
\pgfsys@transformshift{1.000000in}{2.678036in}%
\pgfsys@useobject{currentmarker}{}%
\end{pgfscope}%
\end{pgfscope}%
\begin{pgfscope}%
\pgfsetbuttcap%
\pgfsetroundjoin%
\definecolor{currentfill}{rgb}{0.000000,0.000000,0.000000}%
\pgfsetfillcolor{currentfill}%
\pgfsetlinewidth{0.501875pt}%
\definecolor{currentstroke}{rgb}{0.000000,0.000000,0.000000}%
\pgfsetstrokecolor{currentstroke}%
\pgfsetdash{}{0pt}%
\pgfsys@defobject{currentmarker}{\pgfqpoint{-0.027778in}{0.000000in}}{\pgfqpoint{0.000000in}{0.000000in}}{%
\pgfpathmoveto{\pgfqpoint{0.000000in}{0.000000in}}%
\pgfpathlineto{\pgfqpoint{-0.027778in}{0.000000in}}%
\pgfusepath{stroke,fill}%
}%
\begin{pgfscope}%
\pgfsys@transformshift{7.200000in}{2.678036in}%
\pgfsys@useobject{currentmarker}{}%
\end{pgfscope}%
\end{pgfscope}%
\begin{pgfscope}%
\pgftext[left,bottom,x=0.554012in,y=0.106703in,rotate=90.000000]{{\sffamily\fontsize{12.000000}{14.400000}\selectfont mean square displacement [nm\(\displaystyle ^2\)]}}
%
\end{pgfscope}%
\begin{pgfscope}%
\pgfsetrectcap%
\pgfsetroundjoin%
\pgfsetlinewidth{1.003750pt}%
\definecolor{currentstroke}{rgb}{0.000000,0.000000,0.000000}%
\pgfsetstrokecolor{currentstroke}%
\pgfsetdash{}{0pt}%
\pgfpathmoveto{\pgfqpoint{1.000000in}{2.700000in}}%
\pgfpathlineto{\pgfqpoint{7.200000in}{2.700000in}}%
\pgfusepath{stroke}%
\end{pgfscope}%
\begin{pgfscope}%
\pgfsetrectcap%
\pgfsetroundjoin%
\pgfsetlinewidth{1.003750pt}%
\definecolor{currentstroke}{rgb}{0.000000,0.000000,0.000000}%
\pgfsetstrokecolor{currentstroke}%
\pgfsetdash{}{0pt}%
\pgfpathmoveto{\pgfqpoint{7.200000in}{0.300000in}}%
\pgfpathlineto{\pgfqpoint{7.200000in}{2.700000in}}%
\pgfusepath{stroke}%
\end{pgfscope}%
\begin{pgfscope}%
\pgfsetrectcap%
\pgfsetroundjoin%
\pgfsetlinewidth{1.003750pt}%
\definecolor{currentstroke}{rgb}{0.000000,0.000000,0.000000}%
\pgfsetstrokecolor{currentstroke}%
\pgfsetdash{}{0pt}%
\pgfpathmoveto{\pgfqpoint{1.000000in}{0.300000in}}%
\pgfpathlineto{\pgfqpoint{7.200000in}{0.300000in}}%
\pgfusepath{stroke}%
\end{pgfscope}%
\begin{pgfscope}%
\pgfsetrectcap%
\pgfsetroundjoin%
\pgfsetlinewidth{1.003750pt}%
\definecolor{currentstroke}{rgb}{0.000000,0.000000,0.000000}%
\pgfsetstrokecolor{currentstroke}%
\pgfsetdash{}{0pt}%
\pgfpathmoveto{\pgfqpoint{1.000000in}{0.300000in}}%
\pgfpathlineto{\pgfqpoint{1.000000in}{2.700000in}}%
\pgfusepath{stroke}%
\end{pgfscope}%
\begin{pgfscope}%
\pgfsetrectcap%
\pgfsetroundjoin%
\definecolor{currentfill}{rgb}{1.000000,1.000000,1.000000}%
\pgfsetfillcolor{currentfill}%
\pgfsetlinewidth{1.003750pt}%
\definecolor{currentstroke}{rgb}{0.000000,0.000000,0.000000}%
\pgfsetstrokecolor{currentstroke}%
\pgfsetdash{}{0pt}%
\pgfpathmoveto{\pgfqpoint{1.069417in}{1.977606in}}%
\pgfpathlineto{\pgfqpoint{1.926808in}{1.977606in}}%
\pgfpathlineto{\pgfqpoint{1.926808in}{2.630583in}}%
\pgfpathlineto{\pgfqpoint{1.069417in}{2.630583in}}%
\pgfpathlineto{\pgfqpoint{1.069417in}{1.977606in}}%
\pgfpathclose%
\pgfusepath{stroke,fill}%
\end{pgfscope}%
\begin{pgfscope}%
\pgfsetrectcap%
\pgfsetroundjoin%
\pgfsetlinewidth{1.003750pt}%
\definecolor{currentstroke}{rgb}{0.000000,0.000000,1.000000}%
\pgfsetstrokecolor{currentstroke}%
\pgfsetdash{}{0pt}%
\pgfpathmoveto{\pgfqpoint{1.166600in}{2.518161in}}%
\pgfpathlineto{\pgfqpoint{1.360967in}{2.518161in}}%
\pgfusepath{stroke}%
\end{pgfscope}%
\begin{pgfscope}%
\pgftext[left,bottom,x=1.513683in,y=2.440691in,rotate=0.000000]{{\sffamily\fontsize{9.996000}{11.995200}\selectfont spc}}
%
\end{pgfscope}%
\begin{pgfscope}%
\pgfsetrectcap%
\pgfsetroundjoin%
\pgfsetlinewidth{1.003750pt}%
\definecolor{currentstroke}{rgb}{0.000000,0.500000,0.000000}%
\pgfsetstrokecolor{currentstroke}%
\pgfsetdash{}{0pt}%
\pgfpathmoveto{\pgfqpoint{1.166600in}{2.314385in}}%
\pgfpathlineto{\pgfqpoint{1.360967in}{2.314385in}}%
\pgfusepath{stroke}%
\end{pgfscope}%
\begin{pgfscope}%
\pgftext[left,bottom,x=1.513683in,y=2.236915in,rotate=0.000000]{{\sffamily\fontsize{9.996000}{11.995200}\selectfont spce}}
%
\end{pgfscope}%
\begin{pgfscope}%
\pgfsetrectcap%
\pgfsetroundjoin%
\pgfsetlinewidth{1.003750pt}%
\definecolor{currentstroke}{rgb}{1.000000,0.000000,0.000000}%
\pgfsetstrokecolor{currentstroke}%
\pgfsetdash{}{0pt}%
\pgfpathmoveto{\pgfqpoint{1.166600in}{2.110609in}}%
\pgfpathlineto{\pgfqpoint{1.360967in}{2.110609in}}%
\pgfusepath{stroke}%
\end{pgfscope}%
\begin{pgfscope}%
\pgftext[left,bottom,x=1.513683in,y=2.033139in,rotate=0.000000]{{\sffamily\fontsize{9.996000}{11.995200}\selectfont tip3p}}
%
\end{pgfscope}%
\end{pgfpicture}%
\makeatother%
\endgroup%
}
    		\caption{LONG}
		\end{subfigure}
		\begin{subfigure}[a]{\textwidth}
			\resizebox{\linewidth}{!}{%% Creator: Matplotlib, PGF backend
%%
%% To include the figure in your LaTeX document, write
%%   \input{<filename>.pgf}
%%
%% Make sure the required packages are loaded in your preamble
%%   \usepackage{pgf}
%%
%% Figures using additional raster images can only be included by \input if
%% they are in the same directory as the main LaTeX file. For loading figures
%% from other directories you can use the `import` package
%%   \usepackage{import}
%% and then include the figures with
%%   \import{<path to file>}{<filename>.pgf}
%%
%% Matplotlib used the following preamble
%%   \usepackage{fontspec}
%%   \setmainfont{DejaVu Serif}
%%   \setsansfont{DejaVu Sans}
%%   \setmonofont{DejaVu Sans Mono}
%%
\begingroup%
\makeatletter%
\begin{pgfpicture}%
\pgfpathrectangle{\pgfpointorigin}{\pgfqpoint{8.000000in}{3.000000in}}%
\pgfusepath{use as bounding box}%
\begin{pgfscope}%
\pgfsetrectcap%
\pgfsetroundjoin%
\definecolor{currentfill}{rgb}{1.000000,1.000000,1.000000}%
\pgfsetfillcolor{currentfill}%
\pgfsetlinewidth{0.000000pt}%
\definecolor{currentstroke}{rgb}{1.000000,1.000000,1.000000}%
\pgfsetstrokecolor{currentstroke}%
\pgfsetdash{}{0pt}%
\pgfpathmoveto{\pgfqpoint{0.000000in}{0.000000in}}%
\pgfpathlineto{\pgfqpoint{8.000000in}{0.000000in}}%
\pgfpathlineto{\pgfqpoint{8.000000in}{3.000000in}}%
\pgfpathlineto{\pgfqpoint{0.000000in}{3.000000in}}%
\pgfpathclose%
\pgfusepath{fill}%
\end{pgfscope}%
\begin{pgfscope}%
\pgfsetrectcap%
\pgfsetroundjoin%
\definecolor{currentfill}{rgb}{1.000000,1.000000,1.000000}%
\pgfsetfillcolor{currentfill}%
\pgfsetlinewidth{0.000000pt}%
\definecolor{currentstroke}{rgb}{0.000000,0.000000,0.000000}%
\pgfsetstrokecolor{currentstroke}%
\pgfsetdash{}{0pt}%
\pgfpathmoveto{\pgfqpoint{1.000000in}{0.300000in}}%
\pgfpathlineto{\pgfqpoint{7.200000in}{0.300000in}}%
\pgfpathlineto{\pgfqpoint{7.200000in}{2.700000in}}%
\pgfpathlineto{\pgfqpoint{1.000000in}{2.700000in}}%
\pgfpathclose%
\pgfusepath{fill}%
\end{pgfscope}%
\begin{pgfscope}%
\pgfpathrectangle{\pgfqpoint{1.000000in}{0.300000in}}{\pgfqpoint{6.200000in}{2.400000in}} %
\pgfusepath{clip}%
\pgfsetrectcap%
\pgfsetroundjoin%
\pgfsetlinewidth{1.003750pt}%
\definecolor{currentstroke}{rgb}{0.000000,0.000000,1.000000}%
\pgfsetstrokecolor{currentstroke}%
\pgfsetdash{}{0pt}%
\pgfpathmoveto{\pgfqpoint{1.001240in}{1.451724in}}%
\pgfpathlineto{\pgfqpoint{1.002480in}{1.965809in}}%
\pgfpathlineto{\pgfqpoint{1.003720in}{2.151236in}}%
\pgfpathlineto{\pgfqpoint{1.004960in}{2.163844in}}%
\pgfpathlineto{\pgfqpoint{1.009920in}{1.875944in}}%
\pgfpathlineto{\pgfqpoint{1.019840in}{1.583228in}}%
\pgfpathlineto{\pgfqpoint{1.021080in}{1.577501in}}%
\pgfpathlineto{\pgfqpoint{1.023560in}{1.549291in}}%
\pgfpathlineto{\pgfqpoint{1.026040in}{1.526794in}}%
\pgfpathlineto{\pgfqpoint{1.027280in}{1.513883in}}%
\pgfpathlineto{\pgfqpoint{1.028520in}{1.513028in}}%
\pgfpathlineto{\pgfqpoint{1.029760in}{1.508574in}}%
\pgfpathlineto{\pgfqpoint{1.043400in}{1.412726in}}%
\pgfpathlineto{\pgfqpoint{1.044640in}{1.413151in}}%
\pgfpathlineto{\pgfqpoint{1.047120in}{1.398978in}}%
\pgfpathlineto{\pgfqpoint{1.049600in}{1.399489in}}%
\pgfpathlineto{\pgfqpoint{1.050840in}{1.391081in}}%
\pgfpathlineto{\pgfqpoint{1.054560in}{1.396424in}}%
\pgfpathlineto{\pgfqpoint{1.058280in}{1.386865in}}%
\pgfpathlineto{\pgfqpoint{1.059520in}{1.388074in}}%
\pgfpathlineto{\pgfqpoint{1.060760in}{1.381215in}}%
\pgfpathlineto{\pgfqpoint{1.063240in}{1.381987in}}%
\pgfpathlineto{\pgfqpoint{1.064480in}{1.382838in}}%
\pgfpathlineto{\pgfqpoint{1.068200in}{1.372404in}}%
\pgfpathlineto{\pgfqpoint{1.069440in}{1.373683in}}%
\pgfpathlineto{\pgfqpoint{1.070680in}{1.372967in}}%
\pgfpathlineto{\pgfqpoint{1.073160in}{1.379849in}}%
\pgfpathlineto{\pgfqpoint{1.075640in}{1.378965in}}%
\pgfpathlineto{\pgfqpoint{1.076880in}{1.379470in}}%
\pgfpathlineto{\pgfqpoint{1.081840in}{1.373293in}}%
\pgfpathlineto{\pgfqpoint{1.084320in}{1.370405in}}%
\pgfpathlineto{\pgfqpoint{1.085560in}{1.368715in}}%
\pgfpathlineto{\pgfqpoint{1.089280in}{1.358049in}}%
\pgfpathlineto{\pgfqpoint{1.091760in}{1.356565in}}%
\pgfpathlineto{\pgfqpoint{1.093000in}{1.357730in}}%
\pgfpathlineto{\pgfqpoint{1.096720in}{1.347590in}}%
\pgfpathlineto{\pgfqpoint{1.100440in}{1.348658in}}%
\pgfpathlineto{\pgfqpoint{1.101680in}{1.349225in}}%
\pgfpathlineto{\pgfqpoint{1.107880in}{1.327607in}}%
\pgfpathlineto{\pgfqpoint{1.110360in}{1.329243in}}%
\pgfpathlineto{\pgfqpoint{1.114080in}{1.324193in}}%
\pgfpathlineto{\pgfqpoint{1.116560in}{1.325551in}}%
\pgfpathlineto{\pgfqpoint{1.117800in}{1.325104in}}%
\pgfpathlineto{\pgfqpoint{1.120280in}{1.317979in}}%
\pgfpathlineto{\pgfqpoint{1.122760in}{1.318747in}}%
\pgfpathlineto{\pgfqpoint{1.125240in}{1.315314in}}%
\pgfpathlineto{\pgfqpoint{1.126480in}{1.315739in}}%
\pgfpathlineto{\pgfqpoint{1.128960in}{1.319538in}}%
\pgfpathlineto{\pgfqpoint{1.131440in}{1.316352in}}%
\pgfpathlineto{\pgfqpoint{1.133920in}{1.313259in}}%
\pgfpathlineto{\pgfqpoint{1.135160in}{1.311392in}}%
\pgfpathlineto{\pgfqpoint{1.137640in}{1.315183in}}%
\pgfpathlineto{\pgfqpoint{1.143840in}{1.299326in}}%
\pgfpathlineto{\pgfqpoint{1.145080in}{1.301550in}}%
\pgfpathlineto{\pgfqpoint{1.146320in}{1.301002in}}%
\pgfpathlineto{\pgfqpoint{1.148800in}{1.306652in}}%
\pgfpathlineto{\pgfqpoint{1.150040in}{1.305399in}}%
\pgfpathlineto{\pgfqpoint{1.152520in}{1.307379in}}%
\pgfpathlineto{\pgfqpoint{1.157480in}{1.301862in}}%
\pgfpathlineto{\pgfqpoint{1.158720in}{1.303194in}}%
\pgfpathlineto{\pgfqpoint{1.161200in}{1.302147in}}%
\pgfpathlineto{\pgfqpoint{1.163680in}{1.301401in}}%
\pgfpathlineto{\pgfqpoint{1.164920in}{1.297714in}}%
\pgfpathlineto{\pgfqpoint{1.168640in}{1.299379in}}%
\pgfpathlineto{\pgfqpoint{1.172360in}{1.303824in}}%
\pgfpathlineto{\pgfqpoint{1.174840in}{1.301633in}}%
\pgfpathlineto{\pgfqpoint{1.177320in}{1.312003in}}%
\pgfpathlineto{\pgfqpoint{1.178560in}{1.313241in}}%
\pgfpathlineto{\pgfqpoint{1.181040in}{1.310435in}}%
\pgfpathlineto{\pgfqpoint{1.183520in}{1.311718in}}%
\pgfpathlineto{\pgfqpoint{1.186000in}{1.308877in}}%
\pgfpathlineto{\pgfqpoint{1.188480in}{1.311474in}}%
\pgfpathlineto{\pgfqpoint{1.190960in}{1.309587in}}%
\pgfpathlineto{\pgfqpoint{1.192200in}{1.306469in}}%
\pgfpathlineto{\pgfqpoint{1.194680in}{1.308164in}}%
\pgfpathlineto{\pgfqpoint{1.197160in}{1.316760in}}%
\pgfpathlineto{\pgfqpoint{1.199640in}{1.321063in}}%
\pgfpathlineto{\pgfqpoint{1.200880in}{1.322573in}}%
\pgfpathlineto{\pgfqpoint{1.202120in}{1.320581in}}%
\pgfpathlineto{\pgfqpoint{1.204600in}{1.321075in}}%
\pgfpathlineto{\pgfqpoint{1.205840in}{1.317237in}}%
\pgfpathlineto{\pgfqpoint{1.207080in}{1.318613in}}%
\pgfpathlineto{\pgfqpoint{1.209560in}{1.317420in}}%
\pgfpathlineto{\pgfqpoint{1.214520in}{1.307491in}}%
\pgfpathlineto{\pgfqpoint{1.217000in}{1.304937in}}%
\pgfpathlineto{\pgfqpoint{1.219480in}{1.300723in}}%
\pgfpathlineto{\pgfqpoint{1.220720in}{1.300419in}}%
\pgfpathlineto{\pgfqpoint{1.223200in}{1.301793in}}%
\pgfpathlineto{\pgfqpoint{1.224440in}{1.302239in}}%
\pgfpathlineto{\pgfqpoint{1.225680in}{1.304591in}}%
\pgfpathlineto{\pgfqpoint{1.230640in}{1.301620in}}%
\pgfpathlineto{\pgfqpoint{1.231880in}{1.298631in}}%
\pgfpathlineto{\pgfqpoint{1.234360in}{1.299224in}}%
\pgfpathlineto{\pgfqpoint{1.235600in}{1.300449in}}%
\pgfpathlineto{\pgfqpoint{1.238080in}{1.299111in}}%
\pgfpathlineto{\pgfqpoint{1.239320in}{1.298457in}}%
\pgfpathlineto{\pgfqpoint{1.241800in}{1.294056in}}%
\pgfpathlineto{\pgfqpoint{1.244280in}{1.291544in}}%
\pgfpathlineto{\pgfqpoint{1.246760in}{1.294966in}}%
\pgfpathlineto{\pgfqpoint{1.248000in}{1.293742in}}%
\pgfpathlineto{\pgfqpoint{1.250480in}{1.298209in}}%
\pgfpathlineto{\pgfqpoint{1.251720in}{1.299794in}}%
\pgfpathlineto{\pgfqpoint{1.252960in}{1.299081in}}%
\pgfpathlineto{\pgfqpoint{1.255440in}{1.296686in}}%
\pgfpathlineto{\pgfqpoint{1.256680in}{1.297028in}}%
\pgfpathlineto{\pgfqpoint{1.259160in}{1.294097in}}%
\pgfpathlineto{\pgfqpoint{1.261640in}{1.296398in}}%
\pgfpathlineto{\pgfqpoint{1.265360in}{1.290280in}}%
\pgfpathlineto{\pgfqpoint{1.267840in}{1.289774in}}%
\pgfpathlineto{\pgfqpoint{1.272800in}{1.294166in}}%
\pgfpathlineto{\pgfqpoint{1.274040in}{1.293319in}}%
\pgfpathlineto{\pgfqpoint{1.277760in}{1.298119in}}%
\pgfpathlineto{\pgfqpoint{1.279000in}{1.297365in}}%
\pgfpathlineto{\pgfqpoint{1.281480in}{1.292533in}}%
\pgfpathlineto{\pgfqpoint{1.283960in}{1.297640in}}%
\pgfpathlineto{\pgfqpoint{1.285200in}{1.297909in}}%
\pgfpathlineto{\pgfqpoint{1.287680in}{1.294280in}}%
\pgfpathlineto{\pgfqpoint{1.290160in}{1.291028in}}%
\pgfpathlineto{\pgfqpoint{1.293880in}{1.288051in}}%
\pgfpathlineto{\pgfqpoint{1.295120in}{1.289244in}}%
\pgfpathlineto{\pgfqpoint{1.296360in}{1.292588in}}%
\pgfpathlineto{\pgfqpoint{1.298840in}{1.288013in}}%
\pgfpathlineto{\pgfqpoint{1.300080in}{1.291504in}}%
\pgfpathlineto{\pgfqpoint{1.302560in}{1.292022in}}%
\pgfpathlineto{\pgfqpoint{1.305040in}{1.289384in}}%
\pgfpathlineto{\pgfqpoint{1.306280in}{1.288963in}}%
\pgfpathlineto{\pgfqpoint{1.307520in}{1.290444in}}%
\pgfpathlineto{\pgfqpoint{1.310000in}{1.287264in}}%
\pgfpathlineto{\pgfqpoint{1.312480in}{1.289894in}}%
\pgfpathlineto{\pgfqpoint{1.317440in}{1.284427in}}%
\pgfpathlineto{\pgfqpoint{1.318680in}{1.284934in}}%
\pgfpathlineto{\pgfqpoint{1.323640in}{1.292821in}}%
\pgfpathlineto{\pgfqpoint{1.324880in}{1.293510in}}%
\pgfpathlineto{\pgfqpoint{1.326120in}{1.292443in}}%
\pgfpathlineto{\pgfqpoint{1.327360in}{1.293872in}}%
\pgfpathlineto{\pgfqpoint{1.328600in}{1.292711in}}%
\pgfpathlineto{\pgfqpoint{1.329840in}{1.288732in}}%
\pgfpathlineto{\pgfqpoint{1.331080in}{1.288950in}}%
\pgfpathlineto{\pgfqpoint{1.333560in}{1.285841in}}%
\pgfpathlineto{\pgfqpoint{1.337280in}{1.276837in}}%
\pgfpathlineto{\pgfqpoint{1.339760in}{1.278498in}}%
\pgfpathlineto{\pgfqpoint{1.341000in}{1.278314in}}%
\pgfpathlineto{\pgfqpoint{1.343480in}{1.276324in}}%
\pgfpathlineto{\pgfqpoint{1.350920in}{1.278701in}}%
\pgfpathlineto{\pgfqpoint{1.352160in}{1.278457in}}%
\pgfpathlineto{\pgfqpoint{1.355880in}{1.274194in}}%
\pgfpathlineto{\pgfqpoint{1.359600in}{1.276680in}}%
\pgfpathlineto{\pgfqpoint{1.363320in}{1.276965in}}%
\pgfpathlineto{\pgfqpoint{1.365800in}{1.275340in}}%
\pgfpathlineto{\pgfqpoint{1.368280in}{1.274938in}}%
\pgfpathlineto{\pgfqpoint{1.370760in}{1.278316in}}%
\pgfpathlineto{\pgfqpoint{1.372000in}{1.277938in}}%
\pgfpathlineto{\pgfqpoint{1.373240in}{1.281629in}}%
\pgfpathlineto{\pgfqpoint{1.374480in}{1.281475in}}%
\pgfpathlineto{\pgfqpoint{1.375720in}{1.282853in}}%
\pgfpathlineto{\pgfqpoint{1.379440in}{1.279407in}}%
\pgfpathlineto{\pgfqpoint{1.381920in}{1.280234in}}%
\pgfpathlineto{\pgfqpoint{1.383160in}{1.280548in}}%
\pgfpathlineto{\pgfqpoint{1.385640in}{1.282977in}}%
\pgfpathlineto{\pgfqpoint{1.386880in}{1.280781in}}%
\pgfpathlineto{\pgfqpoint{1.388120in}{1.280987in}}%
\pgfpathlineto{\pgfqpoint{1.393080in}{1.278540in}}%
\pgfpathlineto{\pgfqpoint{1.394320in}{1.278563in}}%
\pgfpathlineto{\pgfqpoint{1.396800in}{1.282653in}}%
\pgfpathlineto{\pgfqpoint{1.399280in}{1.283166in}}%
\pgfpathlineto{\pgfqpoint{1.400520in}{1.285208in}}%
\pgfpathlineto{\pgfqpoint{1.403000in}{1.283809in}}%
\pgfpathlineto{\pgfqpoint{1.405480in}{1.279382in}}%
\pgfpathlineto{\pgfqpoint{1.407960in}{1.282906in}}%
\pgfpathlineto{\pgfqpoint{1.409200in}{1.282069in}}%
\pgfpathlineto{\pgfqpoint{1.410440in}{1.283122in}}%
\pgfpathlineto{\pgfqpoint{1.419120in}{1.273370in}}%
\pgfpathlineto{\pgfqpoint{1.420360in}{1.275782in}}%
\pgfpathlineto{\pgfqpoint{1.422840in}{1.274081in}}%
\pgfpathlineto{\pgfqpoint{1.425320in}{1.277275in}}%
\pgfpathlineto{\pgfqpoint{1.427800in}{1.275080in}}%
\pgfpathlineto{\pgfqpoint{1.430280in}{1.273420in}}%
\pgfpathlineto{\pgfqpoint{1.431520in}{1.274715in}}%
\pgfpathlineto{\pgfqpoint{1.434000in}{1.269928in}}%
\pgfpathlineto{\pgfqpoint{1.436480in}{1.271452in}}%
\pgfpathlineto{\pgfqpoint{1.438960in}{1.268033in}}%
\pgfpathlineto{\pgfqpoint{1.440200in}{1.264547in}}%
\pgfpathlineto{\pgfqpoint{1.442680in}{1.265908in}}%
\pgfpathlineto{\pgfqpoint{1.445160in}{1.269525in}}%
\pgfpathlineto{\pgfqpoint{1.451360in}{1.275314in}}%
\pgfpathlineto{\pgfqpoint{1.452600in}{1.274037in}}%
\pgfpathlineto{\pgfqpoint{1.453840in}{1.270028in}}%
\pgfpathlineto{\pgfqpoint{1.456320in}{1.269563in}}%
\pgfpathlineto{\pgfqpoint{1.457560in}{1.269498in}}%
\pgfpathlineto{\pgfqpoint{1.461280in}{1.261508in}}%
\pgfpathlineto{\pgfqpoint{1.463760in}{1.261861in}}%
\pgfpathlineto{\pgfqpoint{1.466240in}{1.262021in}}%
\pgfpathlineto{\pgfqpoint{1.469960in}{1.264604in}}%
\pgfpathlineto{\pgfqpoint{1.473680in}{1.267374in}}%
\pgfpathlineto{\pgfqpoint{1.477400in}{1.262614in}}%
\pgfpathlineto{\pgfqpoint{1.479880in}{1.259862in}}%
\pgfpathlineto{\pgfqpoint{1.483600in}{1.262541in}}%
\pgfpathlineto{\pgfqpoint{1.491040in}{1.259912in}}%
\pgfpathlineto{\pgfqpoint{1.493520in}{1.261184in}}%
\pgfpathlineto{\pgfqpoint{1.494760in}{1.263021in}}%
\pgfpathlineto{\pgfqpoint{1.496000in}{1.262138in}}%
\pgfpathlineto{\pgfqpoint{1.497240in}{1.264540in}}%
\pgfpathlineto{\pgfqpoint{1.498480in}{1.263976in}}%
\pgfpathlineto{\pgfqpoint{1.499720in}{1.265669in}}%
\pgfpathlineto{\pgfqpoint{1.503440in}{1.263396in}}%
\pgfpathlineto{\pgfqpoint{1.504680in}{1.263560in}}%
\pgfpathlineto{\pgfqpoint{1.507160in}{1.261897in}}%
\pgfpathlineto{\pgfqpoint{1.509640in}{1.264540in}}%
\pgfpathlineto{\pgfqpoint{1.510880in}{1.262831in}}%
\pgfpathlineto{\pgfqpoint{1.513360in}{1.262849in}}%
\pgfpathlineto{\pgfqpoint{1.514600in}{1.261189in}}%
\pgfpathlineto{\pgfqpoint{1.519560in}{1.263029in}}%
\pgfpathlineto{\pgfqpoint{1.524520in}{1.267710in}}%
\pgfpathlineto{\pgfqpoint{1.527000in}{1.265986in}}%
\pgfpathlineto{\pgfqpoint{1.528240in}{1.262308in}}%
\pgfpathlineto{\pgfqpoint{1.530720in}{1.263202in}}%
\pgfpathlineto{\pgfqpoint{1.531960in}{1.263941in}}%
\pgfpathlineto{\pgfqpoint{1.539400in}{1.256816in}}%
\pgfpathlineto{\pgfqpoint{1.541880in}{1.254760in}}%
\pgfpathlineto{\pgfqpoint{1.545600in}{1.257879in}}%
\pgfpathlineto{\pgfqpoint{1.546840in}{1.257057in}}%
\pgfpathlineto{\pgfqpoint{1.549320in}{1.261324in}}%
\pgfpathlineto{\pgfqpoint{1.550560in}{1.260921in}}%
\pgfpathlineto{\pgfqpoint{1.554280in}{1.257117in}}%
\pgfpathlineto{\pgfqpoint{1.555520in}{1.259226in}}%
\pgfpathlineto{\pgfqpoint{1.558000in}{1.256109in}}%
\pgfpathlineto{\pgfqpoint{1.561720in}{1.259097in}}%
\pgfpathlineto{\pgfqpoint{1.565440in}{1.255848in}}%
\pgfpathlineto{\pgfqpoint{1.569160in}{1.260484in}}%
\pgfpathlineto{\pgfqpoint{1.572880in}{1.264560in}}%
\pgfpathlineto{\pgfqpoint{1.574120in}{1.264032in}}%
\pgfpathlineto{\pgfqpoint{1.575360in}{1.265153in}}%
\pgfpathlineto{\pgfqpoint{1.576600in}{1.264091in}}%
\pgfpathlineto{\pgfqpoint{1.579080in}{1.259962in}}%
\pgfpathlineto{\pgfqpoint{1.587760in}{1.253361in}}%
\pgfpathlineto{\pgfqpoint{1.593960in}{1.257031in}}%
\pgfpathlineto{\pgfqpoint{1.600160in}{1.256933in}}%
\pgfpathlineto{\pgfqpoint{1.603880in}{1.252991in}}%
\pgfpathlineto{\pgfqpoint{1.606360in}{1.254456in}}%
\pgfpathlineto{\pgfqpoint{1.607600in}{1.254721in}}%
\pgfpathlineto{\pgfqpoint{1.610080in}{1.253478in}}%
\pgfpathlineto{\pgfqpoint{1.611320in}{1.253527in}}%
\pgfpathlineto{\pgfqpoint{1.615040in}{1.250358in}}%
\pgfpathlineto{\pgfqpoint{1.617520in}{1.250647in}}%
\pgfpathlineto{\pgfqpoint{1.618760in}{1.253088in}}%
\pgfpathlineto{\pgfqpoint{1.620000in}{1.253013in}}%
\pgfpathlineto{\pgfqpoint{1.621240in}{1.254358in}}%
\pgfpathlineto{\pgfqpoint{1.622480in}{1.253891in}}%
\pgfpathlineto{\pgfqpoint{1.623720in}{1.255935in}}%
\pgfpathlineto{\pgfqpoint{1.626200in}{1.255776in}}%
\pgfpathlineto{\pgfqpoint{1.628680in}{1.255897in}}%
\pgfpathlineto{\pgfqpoint{1.631160in}{1.254709in}}%
\pgfpathlineto{\pgfqpoint{1.632400in}{1.256828in}}%
\pgfpathlineto{\pgfqpoint{1.634880in}{1.256076in}}%
\pgfpathlineto{\pgfqpoint{1.636120in}{1.256881in}}%
\pgfpathlineto{\pgfqpoint{1.637360in}{1.255625in}}%
\pgfpathlineto{\pgfqpoint{1.638600in}{1.252475in}}%
\pgfpathlineto{\pgfqpoint{1.642320in}{1.252879in}}%
\pgfpathlineto{\pgfqpoint{1.649760in}{1.257540in}}%
\pgfpathlineto{\pgfqpoint{1.651000in}{1.256698in}}%
\pgfpathlineto{\pgfqpoint{1.652240in}{1.253392in}}%
\pgfpathlineto{\pgfqpoint{1.654720in}{1.254226in}}%
\pgfpathlineto{\pgfqpoint{1.655960in}{1.254774in}}%
\pgfpathlineto{\pgfqpoint{1.665880in}{1.247916in}}%
\pgfpathlineto{\pgfqpoint{1.668360in}{1.251342in}}%
\pgfpathlineto{\pgfqpoint{1.670840in}{1.248351in}}%
\pgfpathlineto{\pgfqpoint{1.673320in}{1.252748in}}%
\pgfpathlineto{\pgfqpoint{1.677040in}{1.251665in}}%
\pgfpathlineto{\pgfqpoint{1.678280in}{1.250705in}}%
\pgfpathlineto{\pgfqpoint{1.679520in}{1.252052in}}%
\pgfpathlineto{\pgfqpoint{1.682000in}{1.251095in}}%
\pgfpathlineto{\pgfqpoint{1.684480in}{1.254702in}}%
\pgfpathlineto{\pgfqpoint{1.689440in}{1.251079in}}%
\pgfpathlineto{\pgfqpoint{1.699360in}{1.258440in}}%
\pgfpathlineto{\pgfqpoint{1.705560in}{1.252548in}}%
\pgfpathlineto{\pgfqpoint{1.708040in}{1.249103in}}%
\pgfpathlineto{\pgfqpoint{1.711760in}{1.246822in}}%
\pgfpathlineto{\pgfqpoint{1.716720in}{1.249573in}}%
\pgfpathlineto{\pgfqpoint{1.721680in}{1.252638in}}%
\pgfpathlineto{\pgfqpoint{1.724160in}{1.250776in}}%
\pgfpathlineto{\pgfqpoint{1.730360in}{1.247368in}}%
\pgfpathlineto{\pgfqpoint{1.735320in}{1.247828in}}%
\pgfpathlineto{\pgfqpoint{1.736560in}{1.245889in}}%
\pgfpathlineto{\pgfqpoint{1.741520in}{1.246800in}}%
\pgfpathlineto{\pgfqpoint{1.742760in}{1.249226in}}%
\pgfpathlineto{\pgfqpoint{1.745240in}{1.248079in}}%
\pgfpathlineto{\pgfqpoint{1.746480in}{1.247759in}}%
\pgfpathlineto{\pgfqpoint{1.747720in}{1.249534in}}%
\pgfpathlineto{\pgfqpoint{1.755160in}{1.247360in}}%
\pgfpathlineto{\pgfqpoint{1.756400in}{1.248823in}}%
\pgfpathlineto{\pgfqpoint{1.758880in}{1.247872in}}%
\pgfpathlineto{\pgfqpoint{1.760120in}{1.247993in}}%
\pgfpathlineto{\pgfqpoint{1.761360in}{1.246840in}}%
\pgfpathlineto{\pgfqpoint{1.763840in}{1.243160in}}%
\pgfpathlineto{\pgfqpoint{1.767560in}{1.244735in}}%
\pgfpathlineto{\pgfqpoint{1.772520in}{1.247145in}}%
\pgfpathlineto{\pgfqpoint{1.777480in}{1.242010in}}%
\pgfpathlineto{\pgfqpoint{1.781200in}{1.243944in}}%
\pgfpathlineto{\pgfqpoint{1.784920in}{1.241141in}}%
\pgfpathlineto{\pgfqpoint{1.786160in}{1.240007in}}%
\pgfpathlineto{\pgfqpoint{1.787400in}{1.240458in}}%
\pgfpathlineto{\pgfqpoint{1.789880in}{1.239599in}}%
\pgfpathlineto{\pgfqpoint{1.792360in}{1.243196in}}%
\pgfpathlineto{\pgfqpoint{1.794840in}{1.239851in}}%
\pgfpathlineto{\pgfqpoint{1.797320in}{1.243107in}}%
\pgfpathlineto{\pgfqpoint{1.802280in}{1.242428in}}%
\pgfpathlineto{\pgfqpoint{1.803520in}{1.243923in}}%
\pgfpathlineto{\pgfqpoint{1.806000in}{1.242728in}}%
\pgfpathlineto{\pgfqpoint{1.808480in}{1.245426in}}%
\pgfpathlineto{\pgfqpoint{1.814680in}{1.241180in}}%
\pgfpathlineto{\pgfqpoint{1.820880in}{1.246425in}}%
\pgfpathlineto{\pgfqpoint{1.824600in}{1.246165in}}%
\pgfpathlineto{\pgfqpoint{1.827080in}{1.242889in}}%
\pgfpathlineto{\pgfqpoint{1.830800in}{1.240272in}}%
\pgfpathlineto{\pgfqpoint{1.834520in}{1.235909in}}%
\pgfpathlineto{\pgfqpoint{1.838240in}{1.239264in}}%
\pgfpathlineto{\pgfqpoint{1.839480in}{1.240463in}}%
\pgfpathlineto{\pgfqpoint{1.841960in}{1.239853in}}%
\pgfpathlineto{\pgfqpoint{1.845680in}{1.241896in}}%
\pgfpathlineto{\pgfqpoint{1.850640in}{1.236754in}}%
\pgfpathlineto{\pgfqpoint{1.851880in}{1.236012in}}%
\pgfpathlineto{\pgfqpoint{1.854360in}{1.237810in}}%
\pgfpathlineto{\pgfqpoint{1.859320in}{1.238502in}}%
\pgfpathlineto{\pgfqpoint{1.861800in}{1.236403in}}%
\pgfpathlineto{\pgfqpoint{1.863040in}{1.235364in}}%
\pgfpathlineto{\pgfqpoint{1.865520in}{1.236746in}}%
\pgfpathlineto{\pgfqpoint{1.866760in}{1.238865in}}%
\pgfpathlineto{\pgfqpoint{1.868000in}{1.238146in}}%
\pgfpathlineto{\pgfqpoint{1.869240in}{1.240865in}}%
\pgfpathlineto{\pgfqpoint{1.870480in}{1.240487in}}%
\pgfpathlineto{\pgfqpoint{1.871720in}{1.241969in}}%
\pgfpathlineto{\pgfqpoint{1.879160in}{1.240467in}}%
\pgfpathlineto{\pgfqpoint{1.881640in}{1.241797in}}%
\pgfpathlineto{\pgfqpoint{1.885360in}{1.239732in}}%
\pgfpathlineto{\pgfqpoint{1.887840in}{1.236561in}}%
\pgfpathlineto{\pgfqpoint{1.895280in}{1.238935in}}%
\pgfpathlineto{\pgfqpoint{1.896520in}{1.240255in}}%
\pgfpathlineto{\pgfqpoint{1.899000in}{1.238385in}}%
\pgfpathlineto{\pgfqpoint{1.901480in}{1.235192in}}%
\pgfpathlineto{\pgfqpoint{1.905200in}{1.235671in}}%
\pgfpathlineto{\pgfqpoint{1.907680in}{1.234013in}}%
\pgfpathlineto{\pgfqpoint{1.910160in}{1.231238in}}%
\pgfpathlineto{\pgfqpoint{1.911400in}{1.231386in}}%
\pgfpathlineto{\pgfqpoint{1.913880in}{1.229884in}}%
\pgfpathlineto{\pgfqpoint{1.916360in}{1.231505in}}%
\pgfpathlineto{\pgfqpoint{1.918840in}{1.227671in}}%
\pgfpathlineto{\pgfqpoint{1.921320in}{1.231276in}}%
\pgfpathlineto{\pgfqpoint{1.926280in}{1.230418in}}%
\pgfpathlineto{\pgfqpoint{1.928760in}{1.231351in}}%
\pgfpathlineto{\pgfqpoint{1.930000in}{1.230441in}}%
\pgfpathlineto{\pgfqpoint{1.932480in}{1.233806in}}%
\pgfpathlineto{\pgfqpoint{1.938680in}{1.228742in}}%
\pgfpathlineto{\pgfqpoint{1.947360in}{1.235300in}}%
\pgfpathlineto{\pgfqpoint{1.953560in}{1.230998in}}%
\pgfpathlineto{\pgfqpoint{1.957280in}{1.226621in}}%
\pgfpathlineto{\pgfqpoint{1.958520in}{1.226192in}}%
\pgfpathlineto{\pgfqpoint{1.969680in}{1.233629in}}%
\pgfpathlineto{\pgfqpoint{1.974640in}{1.226480in}}%
\pgfpathlineto{\pgfqpoint{1.975880in}{1.225766in}}%
\pgfpathlineto{\pgfqpoint{1.978360in}{1.228568in}}%
\pgfpathlineto{\pgfqpoint{1.979600in}{1.229193in}}%
\pgfpathlineto{\pgfqpoint{1.982080in}{1.228249in}}%
\pgfpathlineto{\pgfqpoint{1.983320in}{1.228412in}}%
\pgfpathlineto{\pgfqpoint{1.987040in}{1.225706in}}%
\pgfpathlineto{\pgfqpoint{1.989520in}{1.226416in}}%
\pgfpathlineto{\pgfqpoint{1.990760in}{1.228283in}}%
\pgfpathlineto{\pgfqpoint{1.992000in}{1.226900in}}%
\pgfpathlineto{\pgfqpoint{1.993240in}{1.231891in}}%
\pgfpathlineto{\pgfqpoint{1.994480in}{1.231693in}}%
\pgfpathlineto{\pgfqpoint{1.995720in}{1.233710in}}%
\pgfpathlineto{\pgfqpoint{2.000680in}{1.231791in}}%
\pgfpathlineto{\pgfqpoint{2.003160in}{1.231729in}}%
\pgfpathlineto{\pgfqpoint{2.005640in}{1.232862in}}%
\pgfpathlineto{\pgfqpoint{2.011840in}{1.229031in}}%
\pgfpathlineto{\pgfqpoint{2.019280in}{1.230479in}}%
\pgfpathlineto{\pgfqpoint{2.020520in}{1.232080in}}%
\pgfpathlineto{\pgfqpoint{2.024240in}{1.226930in}}%
\pgfpathlineto{\pgfqpoint{2.029200in}{1.228171in}}%
\pgfpathlineto{\pgfqpoint{2.031680in}{1.226357in}}%
\pgfpathlineto{\pgfqpoint{2.034160in}{1.223773in}}%
\pgfpathlineto{\pgfqpoint{2.035400in}{1.224498in}}%
\pgfpathlineto{\pgfqpoint{2.037880in}{1.223117in}}%
\pgfpathlineto{\pgfqpoint{2.040360in}{1.224402in}}%
\pgfpathlineto{\pgfqpoint{2.042840in}{1.220692in}}%
\pgfpathlineto{\pgfqpoint{2.045320in}{1.223717in}}%
\pgfpathlineto{\pgfqpoint{2.047800in}{1.223193in}}%
\pgfpathlineto{\pgfqpoint{2.052760in}{1.224476in}}%
\pgfpathlineto{\pgfqpoint{2.054000in}{1.222876in}}%
\pgfpathlineto{\pgfqpoint{2.056480in}{1.225164in}}%
\pgfpathlineto{\pgfqpoint{2.062680in}{1.221634in}}%
\pgfpathlineto{\pgfqpoint{2.071360in}{1.226574in}}%
\pgfpathlineto{\pgfqpoint{2.072600in}{1.226366in}}%
\pgfpathlineto{\pgfqpoint{2.078800in}{1.219162in}}%
\pgfpathlineto{\pgfqpoint{2.083760in}{1.217223in}}%
\pgfpathlineto{\pgfqpoint{2.093680in}{1.223205in}}%
\pgfpathlineto{\pgfqpoint{2.098640in}{1.215285in}}%
\pgfpathlineto{\pgfqpoint{2.099880in}{1.214684in}}%
\pgfpathlineto{\pgfqpoint{2.103600in}{1.217873in}}%
\pgfpathlineto{\pgfqpoint{2.106080in}{1.216502in}}%
\pgfpathlineto{\pgfqpoint{2.107320in}{1.216362in}}%
\pgfpathlineto{\pgfqpoint{2.109800in}{1.215019in}}%
\pgfpathlineto{\pgfqpoint{2.112280in}{1.214693in}}%
\pgfpathlineto{\pgfqpoint{2.113520in}{1.214496in}}%
\pgfpathlineto{\pgfqpoint{2.114760in}{1.216149in}}%
\pgfpathlineto{\pgfqpoint{2.116000in}{1.214726in}}%
\pgfpathlineto{\pgfqpoint{2.117240in}{1.219650in}}%
\pgfpathlineto{\pgfqpoint{2.118480in}{1.219734in}}%
\pgfpathlineto{\pgfqpoint{2.119720in}{1.222033in}}%
\pgfpathlineto{\pgfqpoint{2.127160in}{1.220298in}}%
\pgfpathlineto{\pgfqpoint{2.129640in}{1.221517in}}%
\pgfpathlineto{\pgfqpoint{2.132120in}{1.220209in}}%
\pgfpathlineto{\pgfqpoint{2.133360in}{1.219971in}}%
\pgfpathlineto{\pgfqpoint{2.135840in}{1.217681in}}%
\pgfpathlineto{\pgfqpoint{2.138320in}{1.218528in}}%
\pgfpathlineto{\pgfqpoint{2.139560in}{1.219802in}}%
\pgfpathlineto{\pgfqpoint{2.142040in}{1.218673in}}%
\pgfpathlineto{\pgfqpoint{2.144520in}{1.221093in}}%
\pgfpathlineto{\pgfqpoint{2.148240in}{1.216304in}}%
\pgfpathlineto{\pgfqpoint{2.151960in}{1.218637in}}%
\pgfpathlineto{\pgfqpoint{2.155680in}{1.216805in}}%
\pgfpathlineto{\pgfqpoint{2.158160in}{1.214366in}}%
\pgfpathlineto{\pgfqpoint{2.159400in}{1.214852in}}%
\pgfpathlineto{\pgfqpoint{2.161880in}{1.214009in}}%
\pgfpathlineto{\pgfqpoint{2.164360in}{1.215395in}}%
\pgfpathlineto{\pgfqpoint{2.166840in}{1.211749in}}%
\pgfpathlineto{\pgfqpoint{2.169320in}{1.215084in}}%
\pgfpathlineto{\pgfqpoint{2.171800in}{1.214822in}}%
\pgfpathlineto{\pgfqpoint{2.175520in}{1.217412in}}%
\pgfpathlineto{\pgfqpoint{2.178000in}{1.215209in}}%
\pgfpathlineto{\pgfqpoint{2.180480in}{1.218077in}}%
\pgfpathlineto{\pgfqpoint{2.185440in}{1.214830in}}%
\pgfpathlineto{\pgfqpoint{2.189160in}{1.215648in}}%
\pgfpathlineto{\pgfqpoint{2.194120in}{1.218491in}}%
\pgfpathlineto{\pgfqpoint{2.196600in}{1.218973in}}%
\pgfpathlineto{\pgfqpoint{2.200320in}{1.214380in}}%
\pgfpathlineto{\pgfqpoint{2.201560in}{1.213629in}}%
\pgfpathlineto{\pgfqpoint{2.204040in}{1.210347in}}%
\pgfpathlineto{\pgfqpoint{2.206520in}{1.208201in}}%
\pgfpathlineto{\pgfqpoint{2.211480in}{1.212717in}}%
\pgfpathlineto{\pgfqpoint{2.213960in}{1.212232in}}%
\pgfpathlineto{\pgfqpoint{2.217680in}{1.214578in}}%
\pgfpathlineto{\pgfqpoint{2.222640in}{1.207108in}}%
\pgfpathlineto{\pgfqpoint{2.223880in}{1.206982in}}%
\pgfpathlineto{\pgfqpoint{2.226360in}{1.210172in}}%
\pgfpathlineto{\pgfqpoint{2.227600in}{1.210967in}}%
\pgfpathlineto{\pgfqpoint{2.230080in}{1.209355in}}%
\pgfpathlineto{\pgfqpoint{2.232560in}{1.207928in}}%
\pgfpathlineto{\pgfqpoint{2.237520in}{1.205669in}}%
\pgfpathlineto{\pgfqpoint{2.238760in}{1.207361in}}%
\pgfpathlineto{\pgfqpoint{2.240000in}{1.206151in}}%
\pgfpathlineto{\pgfqpoint{2.241240in}{1.211586in}}%
\pgfpathlineto{\pgfqpoint{2.242480in}{1.211874in}}%
\pgfpathlineto{\pgfqpoint{2.244960in}{1.214467in}}%
\pgfpathlineto{\pgfqpoint{2.249920in}{1.211270in}}%
\pgfpathlineto{\pgfqpoint{2.251160in}{1.211600in}}%
\pgfpathlineto{\pgfqpoint{2.253640in}{1.212962in}}%
\pgfpathlineto{\pgfqpoint{2.254880in}{1.211664in}}%
\pgfpathlineto{\pgfqpoint{2.257360in}{1.211926in}}%
\pgfpathlineto{\pgfqpoint{2.259840in}{1.209536in}}%
\pgfpathlineto{\pgfqpoint{2.262320in}{1.209513in}}%
\pgfpathlineto{\pgfqpoint{2.263560in}{1.210505in}}%
\pgfpathlineto{\pgfqpoint{2.266040in}{1.209609in}}%
\pgfpathlineto{\pgfqpoint{2.268520in}{1.211819in}}%
\pgfpathlineto{\pgfqpoint{2.272240in}{1.207316in}}%
\pgfpathlineto{\pgfqpoint{2.277200in}{1.209385in}}%
\pgfpathlineto{\pgfqpoint{2.279680in}{1.208682in}}%
\pgfpathlineto{\pgfqpoint{2.282160in}{1.207019in}}%
\pgfpathlineto{\pgfqpoint{2.283400in}{1.207036in}}%
\pgfpathlineto{\pgfqpoint{2.285880in}{1.206118in}}%
\pgfpathlineto{\pgfqpoint{2.288360in}{1.206872in}}%
\pgfpathlineto{\pgfqpoint{2.290840in}{1.202713in}}%
\pgfpathlineto{\pgfqpoint{2.293320in}{1.204962in}}%
\pgfpathlineto{\pgfqpoint{2.294560in}{1.204257in}}%
\pgfpathlineto{\pgfqpoint{2.300760in}{1.207056in}}%
\pgfpathlineto{\pgfqpoint{2.302000in}{1.205143in}}%
\pgfpathlineto{\pgfqpoint{2.304480in}{1.208416in}}%
\pgfpathlineto{\pgfqpoint{2.309440in}{1.205530in}}%
\pgfpathlineto{\pgfqpoint{2.313160in}{1.206507in}}%
\pgfpathlineto{\pgfqpoint{2.319360in}{1.210911in}}%
\pgfpathlineto{\pgfqpoint{2.320600in}{1.210481in}}%
\pgfpathlineto{\pgfqpoint{2.324320in}{1.205801in}}%
\pgfpathlineto{\pgfqpoint{2.325560in}{1.205345in}}%
\pgfpathlineto{\pgfqpoint{2.328040in}{1.202276in}}%
\pgfpathlineto{\pgfqpoint{2.330520in}{1.200603in}}%
\pgfpathlineto{\pgfqpoint{2.335480in}{1.204694in}}%
\pgfpathlineto{\pgfqpoint{2.336720in}{1.203731in}}%
\pgfpathlineto{\pgfqpoint{2.340440in}{1.206206in}}%
\pgfpathlineto{\pgfqpoint{2.341680in}{1.206962in}}%
\pgfpathlineto{\pgfqpoint{2.346640in}{1.201322in}}%
\pgfpathlineto{\pgfqpoint{2.347880in}{1.200844in}}%
\pgfpathlineto{\pgfqpoint{2.351600in}{1.205182in}}%
\pgfpathlineto{\pgfqpoint{2.359040in}{1.200746in}}%
\pgfpathlineto{\pgfqpoint{2.364000in}{1.200388in}}%
\pgfpathlineto{\pgfqpoint{2.365240in}{1.204971in}}%
\pgfpathlineto{\pgfqpoint{2.366480in}{1.205110in}}%
\pgfpathlineto{\pgfqpoint{2.367720in}{1.207199in}}%
\pgfpathlineto{\pgfqpoint{2.375160in}{1.205857in}}%
\pgfpathlineto{\pgfqpoint{2.377640in}{1.207187in}}%
\pgfpathlineto{\pgfqpoint{2.380120in}{1.206124in}}%
\pgfpathlineto{\pgfqpoint{2.381360in}{1.205972in}}%
\pgfpathlineto{\pgfqpoint{2.383840in}{1.203305in}}%
\pgfpathlineto{\pgfqpoint{2.386320in}{1.202978in}}%
\pgfpathlineto{\pgfqpoint{2.388800in}{1.203556in}}%
\pgfpathlineto{\pgfqpoint{2.390040in}{1.203202in}}%
\pgfpathlineto{\pgfqpoint{2.392520in}{1.204696in}}%
\pgfpathlineto{\pgfqpoint{2.397480in}{1.200735in}}%
\pgfpathlineto{\pgfqpoint{2.399960in}{1.203399in}}%
\pgfpathlineto{\pgfqpoint{2.403680in}{1.202513in}}%
\pgfpathlineto{\pgfqpoint{2.406160in}{1.201227in}}%
\pgfpathlineto{\pgfqpoint{2.407400in}{1.201419in}}%
\pgfpathlineto{\pgfqpoint{2.408640in}{1.200423in}}%
\pgfpathlineto{\pgfqpoint{2.412360in}{1.202062in}}%
\pgfpathlineto{\pgfqpoint{2.414840in}{1.197645in}}%
\pgfpathlineto{\pgfqpoint{2.424760in}{1.202110in}}%
\pgfpathlineto{\pgfqpoint{2.426000in}{1.200460in}}%
\pgfpathlineto{\pgfqpoint{2.428480in}{1.203387in}}%
\pgfpathlineto{\pgfqpoint{2.434680in}{1.201834in}}%
\pgfpathlineto{\pgfqpoint{2.437160in}{1.203162in}}%
\pgfpathlineto{\pgfqpoint{2.443360in}{1.207026in}}%
\pgfpathlineto{\pgfqpoint{2.445840in}{1.204646in}}%
\pgfpathlineto{\pgfqpoint{2.448320in}{1.202260in}}%
\pgfpathlineto{\pgfqpoint{2.449560in}{1.201983in}}%
\pgfpathlineto{\pgfqpoint{2.452040in}{1.199486in}}%
\pgfpathlineto{\pgfqpoint{2.454520in}{1.198255in}}%
\pgfpathlineto{\pgfqpoint{2.459480in}{1.201599in}}%
\pgfpathlineto{\pgfqpoint{2.460720in}{1.200562in}}%
\pgfpathlineto{\pgfqpoint{2.465680in}{1.203959in}}%
\pgfpathlineto{\pgfqpoint{2.470640in}{1.199089in}}%
\pgfpathlineto{\pgfqpoint{2.471880in}{1.198309in}}%
\pgfpathlineto{\pgfqpoint{2.475600in}{1.202476in}}%
\pgfpathlineto{\pgfqpoint{2.481800in}{1.200225in}}%
\pgfpathlineto{\pgfqpoint{2.485520in}{1.199314in}}%
\pgfpathlineto{\pgfqpoint{2.486760in}{1.200691in}}%
\pgfpathlineto{\pgfqpoint{2.488000in}{1.199911in}}%
\pgfpathlineto{\pgfqpoint{2.489240in}{1.203610in}}%
\pgfpathlineto{\pgfqpoint{2.490480in}{1.203568in}}%
\pgfpathlineto{\pgfqpoint{2.491720in}{1.205093in}}%
\pgfpathlineto{\pgfqpoint{2.499160in}{1.203608in}}%
\pgfpathlineto{\pgfqpoint{2.501640in}{1.204619in}}%
\pgfpathlineto{\pgfqpoint{2.504120in}{1.204490in}}%
\pgfpathlineto{\pgfqpoint{2.505360in}{1.204624in}}%
\pgfpathlineto{\pgfqpoint{2.509080in}{1.200802in}}%
\pgfpathlineto{\pgfqpoint{2.516520in}{1.202945in}}%
\pgfpathlineto{\pgfqpoint{2.521480in}{1.198611in}}%
\pgfpathlineto{\pgfqpoint{2.523960in}{1.201119in}}%
\pgfpathlineto{\pgfqpoint{2.527680in}{1.200346in}}%
\pgfpathlineto{\pgfqpoint{2.530160in}{1.198801in}}%
\pgfpathlineto{\pgfqpoint{2.533880in}{1.199157in}}%
\pgfpathlineto{\pgfqpoint{2.536360in}{1.200698in}}%
\pgfpathlineto{\pgfqpoint{2.538840in}{1.196238in}}%
\pgfpathlineto{\pgfqpoint{2.545040in}{1.198213in}}%
\pgfpathlineto{\pgfqpoint{2.546280in}{1.197749in}}%
\pgfpathlineto{\pgfqpoint{2.547520in}{1.199117in}}%
\pgfpathlineto{\pgfqpoint{2.548760in}{1.198625in}}%
\pgfpathlineto{\pgfqpoint{2.550000in}{1.196832in}}%
\pgfpathlineto{\pgfqpoint{2.552480in}{1.199865in}}%
\pgfpathlineto{\pgfqpoint{2.558680in}{1.198266in}}%
\pgfpathlineto{\pgfqpoint{2.561160in}{1.199974in}}%
\pgfpathlineto{\pgfqpoint{2.566120in}{1.203077in}}%
\pgfpathlineto{\pgfqpoint{2.568600in}{1.201994in}}%
\pgfpathlineto{\pgfqpoint{2.576040in}{1.194666in}}%
\pgfpathlineto{\pgfqpoint{2.578520in}{1.193511in}}%
\pgfpathlineto{\pgfqpoint{2.584720in}{1.196185in}}%
\pgfpathlineto{\pgfqpoint{2.588440in}{1.197583in}}%
\pgfpathlineto{\pgfqpoint{2.589680in}{1.198416in}}%
\pgfpathlineto{\pgfqpoint{2.595880in}{1.192812in}}%
\pgfpathlineto{\pgfqpoint{2.599600in}{1.197282in}}%
\pgfpathlineto{\pgfqpoint{2.603320in}{1.196458in}}%
\pgfpathlineto{\pgfqpoint{2.607040in}{1.195213in}}%
\pgfpathlineto{\pgfqpoint{2.610760in}{1.196327in}}%
\pgfpathlineto{\pgfqpoint{2.612000in}{1.195125in}}%
\pgfpathlineto{\pgfqpoint{2.613240in}{1.197286in}}%
\pgfpathlineto{\pgfqpoint{2.614480in}{1.197095in}}%
\pgfpathlineto{\pgfqpoint{2.616960in}{1.199066in}}%
\pgfpathlineto{\pgfqpoint{2.620680in}{1.199138in}}%
\pgfpathlineto{\pgfqpoint{2.623160in}{1.198919in}}%
\pgfpathlineto{\pgfqpoint{2.625640in}{1.199812in}}%
\pgfpathlineto{\pgfqpoint{2.630600in}{1.198690in}}%
\pgfpathlineto{\pgfqpoint{2.633080in}{1.195726in}}%
\pgfpathlineto{\pgfqpoint{2.636800in}{1.196404in}}%
\pgfpathlineto{\pgfqpoint{2.641760in}{1.194931in}}%
\pgfpathlineto{\pgfqpoint{2.644240in}{1.192147in}}%
\pgfpathlineto{\pgfqpoint{2.650440in}{1.196001in}}%
\pgfpathlineto{\pgfqpoint{2.652920in}{1.194697in}}%
\pgfpathlineto{\pgfqpoint{2.656640in}{1.192808in}}%
\pgfpathlineto{\pgfqpoint{2.660360in}{1.194731in}}%
\pgfpathlineto{\pgfqpoint{2.662840in}{1.190360in}}%
\pgfpathlineto{\pgfqpoint{2.666560in}{1.191038in}}%
\pgfpathlineto{\pgfqpoint{2.670280in}{1.191938in}}%
\pgfpathlineto{\pgfqpoint{2.672760in}{1.193567in}}%
\pgfpathlineto{\pgfqpoint{2.674000in}{1.192174in}}%
\pgfpathlineto{\pgfqpoint{2.676480in}{1.195588in}}%
\pgfpathlineto{\pgfqpoint{2.682680in}{1.194096in}}%
\pgfpathlineto{\pgfqpoint{2.685160in}{1.196018in}}%
\pgfpathlineto{\pgfqpoint{2.690120in}{1.199317in}}%
\pgfpathlineto{\pgfqpoint{2.692600in}{1.198676in}}%
\pgfpathlineto{\pgfqpoint{2.700040in}{1.191463in}}%
\pgfpathlineto{\pgfqpoint{2.702520in}{1.190009in}}%
\pgfpathlineto{\pgfqpoint{2.705000in}{1.191255in}}%
\pgfpathlineto{\pgfqpoint{2.707480in}{1.194240in}}%
\pgfpathlineto{\pgfqpoint{2.709960in}{1.193562in}}%
\pgfpathlineto{\pgfqpoint{2.712440in}{1.194090in}}%
\pgfpathlineto{\pgfqpoint{2.713680in}{1.194877in}}%
\pgfpathlineto{\pgfqpoint{2.719880in}{1.190004in}}%
\pgfpathlineto{\pgfqpoint{2.724840in}{1.193592in}}%
\pgfpathlineto{\pgfqpoint{2.727320in}{1.193930in}}%
\pgfpathlineto{\pgfqpoint{2.732280in}{1.192681in}}%
\pgfpathlineto{\pgfqpoint{2.733520in}{1.192723in}}%
\pgfpathlineto{\pgfqpoint{2.734760in}{1.193953in}}%
\pgfpathlineto{\pgfqpoint{2.737240in}{1.192493in}}%
\pgfpathlineto{\pgfqpoint{2.738480in}{1.192091in}}%
\pgfpathlineto{\pgfqpoint{2.740960in}{1.194226in}}%
\pgfpathlineto{\pgfqpoint{2.743440in}{1.194889in}}%
\pgfpathlineto{\pgfqpoint{2.750880in}{1.196267in}}%
\pgfpathlineto{\pgfqpoint{2.753360in}{1.197080in}}%
\pgfpathlineto{\pgfqpoint{2.757080in}{1.192151in}}%
\pgfpathlineto{\pgfqpoint{2.764520in}{1.193717in}}%
\pgfpathlineto{\pgfqpoint{2.768240in}{1.189444in}}%
\pgfpathlineto{\pgfqpoint{2.769480in}{1.189979in}}%
\pgfpathlineto{\pgfqpoint{2.771960in}{1.192777in}}%
\pgfpathlineto{\pgfqpoint{2.775680in}{1.192514in}}%
\pgfpathlineto{\pgfqpoint{2.778160in}{1.190817in}}%
\pgfpathlineto{\pgfqpoint{2.779400in}{1.190834in}}%
\pgfpathlineto{\pgfqpoint{2.780640in}{1.189669in}}%
\pgfpathlineto{\pgfqpoint{2.784360in}{1.191712in}}%
\pgfpathlineto{\pgfqpoint{2.786840in}{1.187378in}}%
\pgfpathlineto{\pgfqpoint{2.789320in}{1.189197in}}%
\pgfpathlineto{\pgfqpoint{2.791800in}{1.188557in}}%
\pgfpathlineto{\pgfqpoint{2.793040in}{1.189368in}}%
\pgfpathlineto{\pgfqpoint{2.794280in}{1.188268in}}%
\pgfpathlineto{\pgfqpoint{2.796760in}{1.189907in}}%
\pgfpathlineto{\pgfqpoint{2.798000in}{1.188950in}}%
\pgfpathlineto{\pgfqpoint{2.800480in}{1.192372in}}%
\pgfpathlineto{\pgfqpoint{2.802960in}{1.191518in}}%
\pgfpathlineto{\pgfqpoint{2.805440in}{1.189915in}}%
\pgfpathlineto{\pgfqpoint{2.806680in}{1.190237in}}%
\pgfpathlineto{\pgfqpoint{2.807920in}{1.192114in}}%
\pgfpathlineto{\pgfqpoint{2.809160in}{1.191886in}}%
\pgfpathlineto{\pgfqpoint{2.814120in}{1.195402in}}%
\pgfpathlineto{\pgfqpoint{2.816600in}{1.193901in}}%
\pgfpathlineto{\pgfqpoint{2.824040in}{1.186508in}}%
\pgfpathlineto{\pgfqpoint{2.827760in}{1.184903in}}%
\pgfpathlineto{\pgfqpoint{2.833960in}{1.188823in}}%
\pgfpathlineto{\pgfqpoint{2.836440in}{1.188643in}}%
\pgfpathlineto{\pgfqpoint{2.837680in}{1.188997in}}%
\pgfpathlineto{\pgfqpoint{2.840160in}{1.187757in}}%
\pgfpathlineto{\pgfqpoint{2.841400in}{1.187446in}}%
\pgfpathlineto{\pgfqpoint{2.843880in}{1.184053in}}%
\pgfpathlineto{\pgfqpoint{2.848840in}{1.187542in}}%
\pgfpathlineto{\pgfqpoint{2.862480in}{1.187463in}}%
\pgfpathlineto{\pgfqpoint{2.864960in}{1.189935in}}%
\pgfpathlineto{\pgfqpoint{2.868680in}{1.190871in}}%
\pgfpathlineto{\pgfqpoint{2.871160in}{1.190197in}}%
\pgfpathlineto{\pgfqpoint{2.876120in}{1.191680in}}%
\pgfpathlineto{\pgfqpoint{2.877360in}{1.191772in}}%
\pgfpathlineto{\pgfqpoint{2.882320in}{1.187173in}}%
\pgfpathlineto{\pgfqpoint{2.884800in}{1.188404in}}%
\pgfpathlineto{\pgfqpoint{2.889760in}{1.186533in}}%
\pgfpathlineto{\pgfqpoint{2.892240in}{1.184817in}}%
\pgfpathlineto{\pgfqpoint{2.897200in}{1.188349in}}%
\pgfpathlineto{\pgfqpoint{2.900920in}{1.186784in}}%
\pgfpathlineto{\pgfqpoint{2.905880in}{1.185026in}}%
\pgfpathlineto{\pgfqpoint{2.908360in}{1.186688in}}%
\pgfpathlineto{\pgfqpoint{2.910840in}{1.182751in}}%
\pgfpathlineto{\pgfqpoint{2.914560in}{1.183316in}}%
\pgfpathlineto{\pgfqpoint{2.919520in}{1.183939in}}%
\pgfpathlineto{\pgfqpoint{2.923240in}{1.185926in}}%
\pgfpathlineto{\pgfqpoint{2.925720in}{1.186451in}}%
\pgfpathlineto{\pgfqpoint{2.930680in}{1.184601in}}%
\pgfpathlineto{\pgfqpoint{2.931920in}{1.186344in}}%
\pgfpathlineto{\pgfqpoint{2.933160in}{1.185901in}}%
\pgfpathlineto{\pgfqpoint{2.938120in}{1.188895in}}%
\pgfpathlineto{\pgfqpoint{2.940600in}{1.186821in}}%
\pgfpathlineto{\pgfqpoint{2.946800in}{1.178562in}}%
\pgfpathlineto{\pgfqpoint{2.948040in}{1.178676in}}%
\pgfpathlineto{\pgfqpoint{2.951760in}{1.176250in}}%
\pgfpathlineto{\pgfqpoint{2.956720in}{1.179206in}}%
\pgfpathlineto{\pgfqpoint{2.960440in}{1.179348in}}%
\pgfpathlineto{\pgfqpoint{2.965400in}{1.179366in}}%
\pgfpathlineto{\pgfqpoint{2.967880in}{1.176012in}}%
\pgfpathlineto{\pgfqpoint{2.974080in}{1.180095in}}%
\pgfpathlineto{\pgfqpoint{2.977800in}{1.179487in}}%
\pgfpathlineto{\pgfqpoint{2.981520in}{1.179800in}}%
\pgfpathlineto{\pgfqpoint{2.982760in}{1.180417in}}%
\pgfpathlineto{\pgfqpoint{2.986480in}{1.177947in}}%
\pgfpathlineto{\pgfqpoint{2.988960in}{1.180260in}}%
\pgfpathlineto{\pgfqpoint{2.993920in}{1.180702in}}%
\pgfpathlineto{\pgfqpoint{2.996400in}{1.181424in}}%
\pgfpathlineto{\pgfqpoint{3.000120in}{1.182607in}}%
\pgfpathlineto{\pgfqpoint{3.001360in}{1.182506in}}%
\pgfpathlineto{\pgfqpoint{3.006320in}{1.177407in}}%
\pgfpathlineto{\pgfqpoint{3.008800in}{1.178370in}}%
\pgfpathlineto{\pgfqpoint{3.012520in}{1.179168in}}%
\pgfpathlineto{\pgfqpoint{3.015000in}{1.175919in}}%
\pgfpathlineto{\pgfqpoint{3.016240in}{1.175353in}}%
\pgfpathlineto{\pgfqpoint{3.021200in}{1.178080in}}%
\pgfpathlineto{\pgfqpoint{3.026160in}{1.175507in}}%
\pgfpathlineto{\pgfqpoint{3.029880in}{1.173489in}}%
\pgfpathlineto{\pgfqpoint{3.032360in}{1.174636in}}%
\pgfpathlineto{\pgfqpoint{3.034840in}{1.170640in}}%
\pgfpathlineto{\pgfqpoint{3.037320in}{1.172269in}}%
\pgfpathlineto{\pgfqpoint{3.039800in}{1.170771in}}%
\pgfpathlineto{\pgfqpoint{3.041040in}{1.171867in}}%
\pgfpathlineto{\pgfqpoint{3.042280in}{1.171424in}}%
\pgfpathlineto{\pgfqpoint{3.044760in}{1.172917in}}%
\pgfpathlineto{\pgfqpoint{3.046000in}{1.172409in}}%
\pgfpathlineto{\pgfqpoint{3.048480in}{1.175394in}}%
\pgfpathlineto{\pgfqpoint{3.050960in}{1.174798in}}%
\pgfpathlineto{\pgfqpoint{3.053440in}{1.173795in}}%
\pgfpathlineto{\pgfqpoint{3.054680in}{1.173944in}}%
\pgfpathlineto{\pgfqpoint{3.055920in}{1.175802in}}%
\pgfpathlineto{\pgfqpoint{3.057160in}{1.174985in}}%
\pgfpathlineto{\pgfqpoint{3.063360in}{1.177751in}}%
\pgfpathlineto{\pgfqpoint{3.065840in}{1.174323in}}%
\pgfpathlineto{\pgfqpoint{3.069560in}{1.168991in}}%
\pgfpathlineto{\pgfqpoint{3.070800in}{1.168157in}}%
\pgfpathlineto{\pgfqpoint{3.072040in}{1.168675in}}%
\pgfpathlineto{\pgfqpoint{3.075760in}{1.166141in}}%
\pgfpathlineto{\pgfqpoint{3.078240in}{1.168125in}}%
\pgfpathlineto{\pgfqpoint{3.079480in}{1.169945in}}%
\pgfpathlineto{\pgfqpoint{3.083200in}{1.169302in}}%
\pgfpathlineto{\pgfqpoint{3.085680in}{1.169732in}}%
\pgfpathlineto{\pgfqpoint{3.088160in}{1.168994in}}%
\pgfpathlineto{\pgfqpoint{3.089400in}{1.169291in}}%
\pgfpathlineto{\pgfqpoint{3.091880in}{1.166190in}}%
\pgfpathlineto{\pgfqpoint{3.093120in}{1.166761in}}%
\pgfpathlineto{\pgfqpoint{3.095600in}{1.170391in}}%
\pgfpathlineto{\pgfqpoint{3.104280in}{1.170179in}}%
\pgfpathlineto{\pgfqpoint{3.108000in}{1.169485in}}%
\pgfpathlineto{\pgfqpoint{3.110480in}{1.167267in}}%
\pgfpathlineto{\pgfqpoint{3.114200in}{1.169412in}}%
\pgfpathlineto{\pgfqpoint{3.121640in}{1.170290in}}%
\pgfpathlineto{\pgfqpoint{3.125360in}{1.170624in}}%
\pgfpathlineto{\pgfqpoint{3.130320in}{1.166387in}}%
\pgfpathlineto{\pgfqpoint{3.132800in}{1.167690in}}%
\pgfpathlineto{\pgfqpoint{3.136520in}{1.167866in}}%
\pgfpathlineto{\pgfqpoint{3.139000in}{1.164515in}}%
\pgfpathlineto{\pgfqpoint{3.140240in}{1.163963in}}%
\pgfpathlineto{\pgfqpoint{3.143960in}{1.165692in}}%
\pgfpathlineto{\pgfqpoint{3.150160in}{1.163314in}}%
\pgfpathlineto{\pgfqpoint{3.153880in}{1.161373in}}%
\pgfpathlineto{\pgfqpoint{3.156360in}{1.162698in}}%
\pgfpathlineto{\pgfqpoint{3.158840in}{1.159389in}}%
\pgfpathlineto{\pgfqpoint{3.161320in}{1.160529in}}%
\pgfpathlineto{\pgfqpoint{3.163800in}{1.158809in}}%
\pgfpathlineto{\pgfqpoint{3.165040in}{1.159885in}}%
\pgfpathlineto{\pgfqpoint{3.166280in}{1.159251in}}%
\pgfpathlineto{\pgfqpoint{3.168760in}{1.160281in}}%
\pgfpathlineto{\pgfqpoint{3.170000in}{1.159845in}}%
\pgfpathlineto{\pgfqpoint{3.172480in}{1.162760in}}%
\pgfpathlineto{\pgfqpoint{3.174960in}{1.161984in}}%
\pgfpathlineto{\pgfqpoint{3.177440in}{1.160562in}}%
\pgfpathlineto{\pgfqpoint{3.178680in}{1.160472in}}%
\pgfpathlineto{\pgfqpoint{3.179920in}{1.162394in}}%
\pgfpathlineto{\pgfqpoint{3.181160in}{1.161621in}}%
\pgfpathlineto{\pgfqpoint{3.186120in}{1.164178in}}%
\pgfpathlineto{\pgfqpoint{3.188600in}{1.161442in}}%
\pgfpathlineto{\pgfqpoint{3.192320in}{1.155023in}}%
\pgfpathlineto{\pgfqpoint{3.199760in}{1.151489in}}%
\pgfpathlineto{\pgfqpoint{3.205960in}{1.154258in}}%
\pgfpathlineto{\pgfqpoint{3.209680in}{1.154426in}}%
\pgfpathlineto{\pgfqpoint{3.212160in}{1.153528in}}%
\pgfpathlineto{\pgfqpoint{3.213400in}{1.154011in}}%
\pgfpathlineto{\pgfqpoint{3.215880in}{1.150956in}}%
\pgfpathlineto{\pgfqpoint{3.217120in}{1.151623in}}%
\pgfpathlineto{\pgfqpoint{3.219600in}{1.154962in}}%
\pgfpathlineto{\pgfqpoint{3.229520in}{1.154767in}}%
\pgfpathlineto{\pgfqpoint{3.230760in}{1.155913in}}%
\pgfpathlineto{\pgfqpoint{3.234480in}{1.151661in}}%
\pgfpathlineto{\pgfqpoint{3.238200in}{1.154860in}}%
\pgfpathlineto{\pgfqpoint{3.240680in}{1.155055in}}%
\pgfpathlineto{\pgfqpoint{3.243160in}{1.153427in}}%
\pgfpathlineto{\pgfqpoint{3.248120in}{1.155427in}}%
\pgfpathlineto{\pgfqpoint{3.254320in}{1.150189in}}%
\pgfpathlineto{\pgfqpoint{3.256800in}{1.151253in}}%
\pgfpathlineto{\pgfqpoint{3.260520in}{1.152183in}}%
\pgfpathlineto{\pgfqpoint{3.263000in}{1.148692in}}%
\pgfpathlineto{\pgfqpoint{3.265480in}{1.148171in}}%
\pgfpathlineto{\pgfqpoint{3.267960in}{1.149300in}}%
\pgfpathlineto{\pgfqpoint{3.275400in}{1.146464in}}%
\pgfpathlineto{\pgfqpoint{3.277880in}{1.145509in}}%
\pgfpathlineto{\pgfqpoint{3.279120in}{1.146746in}}%
\pgfpathlineto{\pgfqpoint{3.280360in}{1.146281in}}%
\pgfpathlineto{\pgfqpoint{3.282840in}{1.142248in}}%
\pgfpathlineto{\pgfqpoint{3.285320in}{1.142949in}}%
\pgfpathlineto{\pgfqpoint{3.287800in}{1.141096in}}%
\pgfpathlineto{\pgfqpoint{3.289040in}{1.142359in}}%
\pgfpathlineto{\pgfqpoint{3.291520in}{1.141982in}}%
\pgfpathlineto{\pgfqpoint{3.295240in}{1.143969in}}%
\pgfpathlineto{\pgfqpoint{3.297720in}{1.144823in}}%
\pgfpathlineto{\pgfqpoint{3.302680in}{1.142338in}}%
\pgfpathlineto{\pgfqpoint{3.303920in}{1.144143in}}%
\pgfpathlineto{\pgfqpoint{3.306400in}{1.144454in}}%
\pgfpathlineto{\pgfqpoint{3.307640in}{1.144177in}}%
\pgfpathlineto{\pgfqpoint{3.310120in}{1.145418in}}%
\pgfpathlineto{\pgfqpoint{3.312600in}{1.143382in}}%
\pgfpathlineto{\pgfqpoint{3.318800in}{1.135204in}}%
\pgfpathlineto{\pgfqpoint{3.321280in}{1.134910in}}%
\pgfpathlineto{\pgfqpoint{3.323760in}{1.133236in}}%
\pgfpathlineto{\pgfqpoint{3.329960in}{1.137240in}}%
\pgfpathlineto{\pgfqpoint{3.338640in}{1.136192in}}%
\pgfpathlineto{\pgfqpoint{3.339880in}{1.135019in}}%
\pgfpathlineto{\pgfqpoint{3.341120in}{1.135683in}}%
\pgfpathlineto{\pgfqpoint{3.343600in}{1.139073in}}%
\pgfpathlineto{\pgfqpoint{3.347320in}{1.139026in}}%
\pgfpathlineto{\pgfqpoint{3.352280in}{1.139346in}}%
\pgfpathlineto{\pgfqpoint{3.354760in}{1.140812in}}%
\pgfpathlineto{\pgfqpoint{3.358480in}{1.136084in}}%
\pgfpathlineto{\pgfqpoint{3.363440in}{1.140079in}}%
\pgfpathlineto{\pgfqpoint{3.365920in}{1.139106in}}%
\pgfpathlineto{\pgfqpoint{3.368400in}{1.138813in}}%
\pgfpathlineto{\pgfqpoint{3.372120in}{1.139849in}}%
\pgfpathlineto{\pgfqpoint{3.378320in}{1.135148in}}%
\pgfpathlineto{\pgfqpoint{3.380800in}{1.136509in}}%
\pgfpathlineto{\pgfqpoint{3.383280in}{1.136490in}}%
\pgfpathlineto{\pgfqpoint{3.384520in}{1.136720in}}%
\pgfpathlineto{\pgfqpoint{3.387000in}{1.134175in}}%
\pgfpathlineto{\pgfqpoint{3.388240in}{1.133211in}}%
\pgfpathlineto{\pgfqpoint{3.391960in}{1.134851in}}%
\pgfpathlineto{\pgfqpoint{3.394440in}{1.134469in}}%
\pgfpathlineto{\pgfqpoint{3.396920in}{1.134686in}}%
\pgfpathlineto{\pgfqpoint{3.399400in}{1.133364in}}%
\pgfpathlineto{\pgfqpoint{3.401880in}{1.131861in}}%
\pgfpathlineto{\pgfqpoint{3.404360in}{1.132709in}}%
\pgfpathlineto{\pgfqpoint{3.406840in}{1.129002in}}%
\pgfpathlineto{\pgfqpoint{3.409320in}{1.130519in}}%
\pgfpathlineto{\pgfqpoint{3.411800in}{1.128358in}}%
\pgfpathlineto{\pgfqpoint{3.413040in}{1.129289in}}%
\pgfpathlineto{\pgfqpoint{3.414280in}{1.128374in}}%
\pgfpathlineto{\pgfqpoint{3.419240in}{1.131182in}}%
\pgfpathlineto{\pgfqpoint{3.421720in}{1.132165in}}%
\pgfpathlineto{\pgfqpoint{3.426680in}{1.128711in}}%
\pgfpathlineto{\pgfqpoint{3.427920in}{1.130389in}}%
\pgfpathlineto{\pgfqpoint{3.431640in}{1.129812in}}%
\pgfpathlineto{\pgfqpoint{3.434120in}{1.130747in}}%
\pgfpathlineto{\pgfqpoint{3.437840in}{1.126762in}}%
\pgfpathlineto{\pgfqpoint{3.441560in}{1.122875in}}%
\pgfpathlineto{\pgfqpoint{3.444040in}{1.121751in}}%
\pgfpathlineto{\pgfqpoint{3.445280in}{1.121812in}}%
\pgfpathlineto{\pgfqpoint{3.447760in}{1.120381in}}%
\pgfpathlineto{\pgfqpoint{3.453960in}{1.125256in}}%
\pgfpathlineto{\pgfqpoint{3.461400in}{1.126125in}}%
\pgfpathlineto{\pgfqpoint{3.463880in}{1.123119in}}%
\pgfpathlineto{\pgfqpoint{3.465120in}{1.123760in}}%
\pgfpathlineto{\pgfqpoint{3.467600in}{1.127337in}}%
\pgfpathlineto{\pgfqpoint{3.477520in}{1.128611in}}%
\pgfpathlineto{\pgfqpoint{3.478760in}{1.129584in}}%
\pgfpathlineto{\pgfqpoint{3.480000in}{1.127867in}}%
\pgfpathlineto{\pgfqpoint{3.482480in}{1.128771in}}%
\pgfpathlineto{\pgfqpoint{3.488680in}{1.132212in}}%
\pgfpathlineto{\pgfqpoint{3.492400in}{1.130536in}}%
\pgfpathlineto{\pgfqpoint{3.496120in}{1.131709in}}%
\pgfpathlineto{\pgfqpoint{3.502320in}{1.126142in}}%
\pgfpathlineto{\pgfqpoint{3.506040in}{1.126456in}}%
\pgfpathlineto{\pgfqpoint{3.508520in}{1.126697in}}%
\pgfpathlineto{\pgfqpoint{3.511000in}{1.124217in}}%
\pgfpathlineto{\pgfqpoint{3.512240in}{1.123407in}}%
\pgfpathlineto{\pgfqpoint{3.517200in}{1.124990in}}%
\pgfpathlineto{\pgfqpoint{3.520920in}{1.125145in}}%
\pgfpathlineto{\pgfqpoint{3.525880in}{1.121814in}}%
\pgfpathlineto{\pgfqpoint{3.528360in}{1.123345in}}%
\pgfpathlineto{\pgfqpoint{3.530840in}{1.119807in}}%
\pgfpathlineto{\pgfqpoint{3.534560in}{1.120170in}}%
\pgfpathlineto{\pgfqpoint{3.535800in}{1.118921in}}%
\pgfpathlineto{\pgfqpoint{3.537040in}{1.119746in}}%
\pgfpathlineto{\pgfqpoint{3.539520in}{1.119175in}}%
\pgfpathlineto{\pgfqpoint{3.540760in}{1.119843in}}%
\pgfpathlineto{\pgfqpoint{3.542000in}{1.118896in}}%
\pgfpathlineto{\pgfqpoint{3.544480in}{1.122214in}}%
\pgfpathlineto{\pgfqpoint{3.546960in}{1.121885in}}%
\pgfpathlineto{\pgfqpoint{3.549440in}{1.119550in}}%
\pgfpathlineto{\pgfqpoint{3.550680in}{1.119182in}}%
\pgfpathlineto{\pgfqpoint{3.553160in}{1.121357in}}%
\pgfpathlineto{\pgfqpoint{3.556880in}{1.121869in}}%
\pgfpathlineto{\pgfqpoint{3.558120in}{1.121962in}}%
\pgfpathlineto{\pgfqpoint{3.561840in}{1.117722in}}%
\pgfpathlineto{\pgfqpoint{3.566800in}{1.112749in}}%
\pgfpathlineto{\pgfqpoint{3.569280in}{1.113157in}}%
\pgfpathlineto{\pgfqpoint{3.571760in}{1.111623in}}%
\pgfpathlineto{\pgfqpoint{3.574240in}{1.113329in}}%
\pgfpathlineto{\pgfqpoint{3.575480in}{1.114966in}}%
\pgfpathlineto{\pgfqpoint{3.576720in}{1.114719in}}%
\pgfpathlineto{\pgfqpoint{3.579200in}{1.116406in}}%
\pgfpathlineto{\pgfqpoint{3.581680in}{1.117043in}}%
\pgfpathlineto{\pgfqpoint{3.582920in}{1.115892in}}%
\pgfpathlineto{\pgfqpoint{3.585400in}{1.117322in}}%
\pgfpathlineto{\pgfqpoint{3.587880in}{1.114562in}}%
\pgfpathlineto{\pgfqpoint{3.589120in}{1.114823in}}%
\pgfpathlineto{\pgfqpoint{3.591600in}{1.118752in}}%
\pgfpathlineto{\pgfqpoint{3.601520in}{1.120172in}}%
\pgfpathlineto{\pgfqpoint{3.602760in}{1.120907in}}%
\pgfpathlineto{\pgfqpoint{3.604000in}{1.119310in}}%
\pgfpathlineto{\pgfqpoint{3.608960in}{1.121969in}}%
\pgfpathlineto{\pgfqpoint{3.611440in}{1.123185in}}%
\pgfpathlineto{\pgfqpoint{3.613920in}{1.122387in}}%
\pgfpathlineto{\pgfqpoint{3.616400in}{1.122117in}}%
\pgfpathlineto{\pgfqpoint{3.620120in}{1.123491in}}%
\pgfpathlineto{\pgfqpoint{3.626320in}{1.117404in}}%
\pgfpathlineto{\pgfqpoint{3.632520in}{1.118305in}}%
\pgfpathlineto{\pgfqpoint{3.635000in}{1.116218in}}%
\pgfpathlineto{\pgfqpoint{3.637480in}{1.115312in}}%
\pgfpathlineto{\pgfqpoint{3.641200in}{1.116113in}}%
\pgfpathlineto{\pgfqpoint{3.644920in}{1.116328in}}%
\pgfpathlineto{\pgfqpoint{3.649880in}{1.112607in}}%
\pgfpathlineto{\pgfqpoint{3.652360in}{1.113963in}}%
\pgfpathlineto{\pgfqpoint{3.654840in}{1.110786in}}%
\pgfpathlineto{\pgfqpoint{3.657320in}{1.111585in}}%
\pgfpathlineto{\pgfqpoint{3.659800in}{1.109535in}}%
\pgfpathlineto{\pgfqpoint{3.661040in}{1.110488in}}%
\pgfpathlineto{\pgfqpoint{3.663520in}{1.109477in}}%
\pgfpathlineto{\pgfqpoint{3.664760in}{1.109749in}}%
\pgfpathlineto{\pgfqpoint{3.666000in}{1.108670in}}%
\pgfpathlineto{\pgfqpoint{3.668480in}{1.111516in}}%
\pgfpathlineto{\pgfqpoint{3.670960in}{1.111695in}}%
\pgfpathlineto{\pgfqpoint{3.673440in}{1.109425in}}%
\pgfpathlineto{\pgfqpoint{3.674680in}{1.109317in}}%
\pgfpathlineto{\pgfqpoint{3.677160in}{1.111919in}}%
\pgfpathlineto{\pgfqpoint{3.682120in}{1.113767in}}%
\pgfpathlineto{\pgfqpoint{3.685840in}{1.110362in}}%
\pgfpathlineto{\pgfqpoint{3.690800in}{1.105563in}}%
\pgfpathlineto{\pgfqpoint{3.693280in}{1.106061in}}%
\pgfpathlineto{\pgfqpoint{3.695760in}{1.104289in}}%
\pgfpathlineto{\pgfqpoint{3.698240in}{1.106270in}}%
\pgfpathlineto{\pgfqpoint{3.700720in}{1.107697in}}%
\pgfpathlineto{\pgfqpoint{3.704440in}{1.108498in}}%
\pgfpathlineto{\pgfqpoint{3.710640in}{1.107083in}}%
\pgfpathlineto{\pgfqpoint{3.713120in}{1.106618in}}%
\pgfpathlineto{\pgfqpoint{3.716840in}{1.110426in}}%
\pgfpathlineto{\pgfqpoint{3.721800in}{1.111687in}}%
\pgfpathlineto{\pgfqpoint{3.726760in}{1.112583in}}%
\pgfpathlineto{\pgfqpoint{3.728000in}{1.111083in}}%
\pgfpathlineto{\pgfqpoint{3.729240in}{1.112423in}}%
\pgfpathlineto{\pgfqpoint{3.731720in}{1.112831in}}%
\pgfpathlineto{\pgfqpoint{3.735440in}{1.115058in}}%
\pgfpathlineto{\pgfqpoint{3.736680in}{1.115675in}}%
\pgfpathlineto{\pgfqpoint{3.740400in}{1.113904in}}%
\pgfpathlineto{\pgfqpoint{3.745360in}{1.113827in}}%
\pgfpathlineto{\pgfqpoint{3.750320in}{1.108945in}}%
\pgfpathlineto{\pgfqpoint{3.755280in}{1.110545in}}%
\pgfpathlineto{\pgfqpoint{3.756520in}{1.110752in}}%
\pgfpathlineto{\pgfqpoint{3.760240in}{1.108286in}}%
\pgfpathlineto{\pgfqpoint{3.768920in}{1.109864in}}%
\pgfpathlineto{\pgfqpoint{3.773880in}{1.106095in}}%
\pgfpathlineto{\pgfqpoint{3.776360in}{1.107460in}}%
\pgfpathlineto{\pgfqpoint{3.778840in}{1.103947in}}%
\pgfpathlineto{\pgfqpoint{3.781320in}{1.104118in}}%
\pgfpathlineto{\pgfqpoint{3.783800in}{1.102197in}}%
\pgfpathlineto{\pgfqpoint{3.785040in}{1.103510in}}%
\pgfpathlineto{\pgfqpoint{3.787520in}{1.102570in}}%
\pgfpathlineto{\pgfqpoint{3.788760in}{1.102731in}}%
\pgfpathlineto{\pgfqpoint{3.790000in}{1.101402in}}%
\pgfpathlineto{\pgfqpoint{3.792480in}{1.104046in}}%
\pgfpathlineto{\pgfqpoint{3.794960in}{1.103462in}}%
\pgfpathlineto{\pgfqpoint{3.798680in}{1.100965in}}%
\pgfpathlineto{\pgfqpoint{3.802400in}{1.104248in}}%
\pgfpathlineto{\pgfqpoint{3.804880in}{1.105305in}}%
\pgfpathlineto{\pgfqpoint{3.807360in}{1.104739in}}%
\pgfpathlineto{\pgfqpoint{3.816040in}{1.097632in}}%
\pgfpathlineto{\pgfqpoint{3.818520in}{1.096035in}}%
\pgfpathlineto{\pgfqpoint{3.819760in}{1.094848in}}%
\pgfpathlineto{\pgfqpoint{3.822240in}{1.096782in}}%
\pgfpathlineto{\pgfqpoint{3.824720in}{1.098158in}}%
\pgfpathlineto{\pgfqpoint{3.828440in}{1.099252in}}%
\pgfpathlineto{\pgfqpoint{3.832160in}{1.098965in}}%
\pgfpathlineto{\pgfqpoint{3.833400in}{1.099611in}}%
\pgfpathlineto{\pgfqpoint{3.835880in}{1.096739in}}%
\pgfpathlineto{\pgfqpoint{3.837120in}{1.097284in}}%
\pgfpathlineto{\pgfqpoint{3.840840in}{1.101699in}}%
\pgfpathlineto{\pgfqpoint{3.844560in}{1.102325in}}%
\pgfpathlineto{\pgfqpoint{3.849520in}{1.104375in}}%
\pgfpathlineto{\pgfqpoint{3.850760in}{1.104493in}}%
\pgfpathlineto{\pgfqpoint{3.852000in}{1.103022in}}%
\pgfpathlineto{\pgfqpoint{3.853240in}{1.104860in}}%
\pgfpathlineto{\pgfqpoint{3.854480in}{1.104286in}}%
\pgfpathlineto{\pgfqpoint{3.860680in}{1.108342in}}%
\pgfpathlineto{\pgfqpoint{3.864400in}{1.107188in}}%
\pgfpathlineto{\pgfqpoint{3.869360in}{1.106861in}}%
\pgfpathlineto{\pgfqpoint{3.873080in}{1.101983in}}%
\pgfpathlineto{\pgfqpoint{3.875560in}{1.101878in}}%
\pgfpathlineto{\pgfqpoint{3.878040in}{1.102853in}}%
\pgfpathlineto{\pgfqpoint{3.886720in}{1.100764in}}%
\pgfpathlineto{\pgfqpoint{3.892920in}{1.102446in}}%
\pgfpathlineto{\pgfqpoint{3.897880in}{1.099365in}}%
\pgfpathlineto{\pgfqpoint{3.900360in}{1.100247in}}%
\pgfpathlineto{\pgfqpoint{3.902840in}{1.096211in}}%
\pgfpathlineto{\pgfqpoint{3.905320in}{1.096680in}}%
\pgfpathlineto{\pgfqpoint{3.907800in}{1.094489in}}%
\pgfpathlineto{\pgfqpoint{3.909040in}{1.095593in}}%
\pgfpathlineto{\pgfqpoint{3.911520in}{1.094643in}}%
\pgfpathlineto{\pgfqpoint{3.912760in}{1.094581in}}%
\pgfpathlineto{\pgfqpoint{3.914000in}{1.092900in}}%
\pgfpathlineto{\pgfqpoint{3.916480in}{1.095340in}}%
\pgfpathlineto{\pgfqpoint{3.918960in}{1.094449in}}%
\pgfpathlineto{\pgfqpoint{3.922680in}{1.092197in}}%
\pgfpathlineto{\pgfqpoint{3.925160in}{1.095015in}}%
\pgfpathlineto{\pgfqpoint{3.928880in}{1.096869in}}%
\pgfpathlineto{\pgfqpoint{3.931360in}{1.096172in}}%
\pgfpathlineto{\pgfqpoint{3.940040in}{1.090076in}}%
\pgfpathlineto{\pgfqpoint{3.941280in}{1.090242in}}%
\pgfpathlineto{\pgfqpoint{3.943760in}{1.087955in}}%
\pgfpathlineto{\pgfqpoint{3.945000in}{1.088577in}}%
\pgfpathlineto{\pgfqpoint{3.947480in}{1.092188in}}%
\pgfpathlineto{\pgfqpoint{3.951200in}{1.092654in}}%
\pgfpathlineto{\pgfqpoint{3.953680in}{1.092761in}}%
\pgfpathlineto{\pgfqpoint{3.954920in}{1.091645in}}%
\pgfpathlineto{\pgfqpoint{3.957400in}{1.092993in}}%
\pgfpathlineto{\pgfqpoint{3.959880in}{1.090508in}}%
\pgfpathlineto{\pgfqpoint{3.961120in}{1.091189in}}%
\pgfpathlineto{\pgfqpoint{3.964840in}{1.095844in}}%
\pgfpathlineto{\pgfqpoint{3.968560in}{1.096216in}}%
\pgfpathlineto{\pgfqpoint{3.974760in}{1.099118in}}%
\pgfpathlineto{\pgfqpoint{3.976000in}{1.097307in}}%
\pgfpathlineto{\pgfqpoint{3.977240in}{1.098204in}}%
\pgfpathlineto{\pgfqpoint{3.978480in}{1.097502in}}%
\pgfpathlineto{\pgfqpoint{3.984680in}{1.102046in}}%
\pgfpathlineto{\pgfqpoint{3.988400in}{1.100601in}}%
\pgfpathlineto{\pgfqpoint{3.990880in}{1.102112in}}%
\pgfpathlineto{\pgfqpoint{3.992120in}{1.102700in}}%
\pgfpathlineto{\pgfqpoint{3.998320in}{1.095625in}}%
\pgfpathlineto{\pgfqpoint{4.007000in}{1.095645in}}%
\pgfpathlineto{\pgfqpoint{4.009480in}{1.094136in}}%
\pgfpathlineto{\pgfqpoint{4.016920in}{1.095714in}}%
\pgfpathlineto{\pgfqpoint{4.020640in}{1.093023in}}%
\pgfpathlineto{\pgfqpoint{4.024360in}{1.095226in}}%
\pgfpathlineto{\pgfqpoint{4.026840in}{1.092167in}}%
\pgfpathlineto{\pgfqpoint{4.029320in}{1.091737in}}%
\pgfpathlineto{\pgfqpoint{4.031800in}{1.089751in}}%
\pgfpathlineto{\pgfqpoint{4.033040in}{1.090355in}}%
\pgfpathlineto{\pgfqpoint{4.035520in}{1.089020in}}%
\pgfpathlineto{\pgfqpoint{4.036760in}{1.088832in}}%
\pgfpathlineto{\pgfqpoint{4.038000in}{1.087207in}}%
\pgfpathlineto{\pgfqpoint{4.040480in}{1.089168in}}%
\pgfpathlineto{\pgfqpoint{4.044200in}{1.088207in}}%
\pgfpathlineto{\pgfqpoint{4.046680in}{1.086079in}}%
\pgfpathlineto{\pgfqpoint{4.049160in}{1.088465in}}%
\pgfpathlineto{\pgfqpoint{4.051640in}{1.089153in}}%
\pgfpathlineto{\pgfqpoint{4.054120in}{1.089390in}}%
\pgfpathlineto{\pgfqpoint{4.057840in}{1.085726in}}%
\pgfpathlineto{\pgfqpoint{4.059080in}{1.085523in}}%
\pgfpathlineto{\pgfqpoint{4.061560in}{1.082971in}}%
\pgfpathlineto{\pgfqpoint{4.064040in}{1.082816in}}%
\pgfpathlineto{\pgfqpoint{4.065280in}{1.082909in}}%
\pgfpathlineto{\pgfqpoint{4.067760in}{1.080548in}}%
\pgfpathlineto{\pgfqpoint{4.069000in}{1.081303in}}%
\pgfpathlineto{\pgfqpoint{4.071480in}{1.085707in}}%
\pgfpathlineto{\pgfqpoint{4.080160in}{1.085371in}}%
\pgfpathlineto{\pgfqpoint{4.081400in}{1.085749in}}%
\pgfpathlineto{\pgfqpoint{4.083880in}{1.083448in}}%
\pgfpathlineto{\pgfqpoint{4.085120in}{1.083766in}}%
\pgfpathlineto{\pgfqpoint{4.088840in}{1.088224in}}%
\pgfpathlineto{\pgfqpoint{4.090080in}{1.088083in}}%
\pgfpathlineto{\pgfqpoint{4.092560in}{1.089115in}}%
\pgfpathlineto{\pgfqpoint{4.098760in}{1.092690in}}%
\pgfpathlineto{\pgfqpoint{4.100000in}{1.090884in}}%
\pgfpathlineto{\pgfqpoint{4.101240in}{1.091420in}}%
\pgfpathlineto{\pgfqpoint{4.103720in}{1.090664in}}%
\pgfpathlineto{\pgfqpoint{4.107440in}{1.092995in}}%
\pgfpathlineto{\pgfqpoint{4.108680in}{1.093444in}}%
\pgfpathlineto{\pgfqpoint{4.111160in}{1.090973in}}%
\pgfpathlineto{\pgfqpoint{4.113640in}{1.092113in}}%
\pgfpathlineto{\pgfqpoint{4.116120in}{1.094029in}}%
\pgfpathlineto{\pgfqpoint{4.121080in}{1.087551in}}%
\pgfpathlineto{\pgfqpoint{4.123560in}{1.086905in}}%
\pgfpathlineto{\pgfqpoint{4.126040in}{1.088167in}}%
\pgfpathlineto{\pgfqpoint{4.135960in}{1.086883in}}%
\pgfpathlineto{\pgfqpoint{4.139680in}{1.088292in}}%
\pgfpathlineto{\pgfqpoint{4.144640in}{1.083743in}}%
\pgfpathlineto{\pgfqpoint{4.148360in}{1.087236in}}%
\pgfpathlineto{\pgfqpoint{4.150840in}{1.084010in}}%
\pgfpathlineto{\pgfqpoint{4.153320in}{1.084082in}}%
\pgfpathlineto{\pgfqpoint{4.155800in}{1.082086in}}%
\pgfpathlineto{\pgfqpoint{4.157040in}{1.082641in}}%
\pgfpathlineto{\pgfqpoint{4.159520in}{1.081211in}}%
\pgfpathlineto{\pgfqpoint{4.160760in}{1.081071in}}%
\pgfpathlineto{\pgfqpoint{4.162000in}{1.079733in}}%
\pgfpathlineto{\pgfqpoint{4.164480in}{1.082250in}}%
\pgfpathlineto{\pgfqpoint{4.166960in}{1.081617in}}%
\pgfpathlineto{\pgfqpoint{4.169440in}{1.080431in}}%
\pgfpathlineto{\pgfqpoint{4.170680in}{1.079170in}}%
\pgfpathlineto{\pgfqpoint{4.173160in}{1.081392in}}%
\pgfpathlineto{\pgfqpoint{4.178120in}{1.081948in}}%
\pgfpathlineto{\pgfqpoint{4.181840in}{1.078623in}}%
\pgfpathlineto{\pgfqpoint{4.183080in}{1.078188in}}%
\pgfpathlineto{\pgfqpoint{4.185560in}{1.075209in}}%
\pgfpathlineto{\pgfqpoint{4.188040in}{1.075193in}}%
\pgfpathlineto{\pgfqpoint{4.189280in}{1.075708in}}%
\pgfpathlineto{\pgfqpoint{4.191760in}{1.073447in}}%
\pgfpathlineto{\pgfqpoint{4.193000in}{1.074172in}}%
\pgfpathlineto{\pgfqpoint{4.195480in}{1.078835in}}%
\pgfpathlineto{\pgfqpoint{4.204160in}{1.078005in}}%
\pgfpathlineto{\pgfqpoint{4.205400in}{1.078625in}}%
\pgfpathlineto{\pgfqpoint{4.207880in}{1.076357in}}%
\pgfpathlineto{\pgfqpoint{4.209120in}{1.076407in}}%
\pgfpathlineto{\pgfqpoint{4.211600in}{1.079940in}}%
\pgfpathlineto{\pgfqpoint{4.214080in}{1.079129in}}%
\pgfpathlineto{\pgfqpoint{4.215320in}{1.080529in}}%
\pgfpathlineto{\pgfqpoint{4.216560in}{1.080015in}}%
\pgfpathlineto{\pgfqpoint{4.221520in}{1.083646in}}%
\pgfpathlineto{\pgfqpoint{4.222760in}{1.084100in}}%
\pgfpathlineto{\pgfqpoint{4.225240in}{1.082277in}}%
\pgfpathlineto{\pgfqpoint{4.226480in}{1.080818in}}%
\pgfpathlineto{\pgfqpoint{4.233920in}{1.082161in}}%
\pgfpathlineto{\pgfqpoint{4.235160in}{1.081820in}}%
\pgfpathlineto{\pgfqpoint{4.240120in}{1.085573in}}%
\pgfpathlineto{\pgfqpoint{4.245080in}{1.079495in}}%
\pgfpathlineto{\pgfqpoint{4.247560in}{1.079208in}}%
\pgfpathlineto{\pgfqpoint{4.250040in}{1.080947in}}%
\pgfpathlineto{\pgfqpoint{4.258720in}{1.078502in}}%
\pgfpathlineto{\pgfqpoint{4.262440in}{1.079057in}}%
\pgfpathlineto{\pgfqpoint{4.264920in}{1.078667in}}%
\pgfpathlineto{\pgfqpoint{4.266160in}{1.078055in}}%
\pgfpathlineto{\pgfqpoint{4.268640in}{1.075586in}}%
\pgfpathlineto{\pgfqpoint{4.272360in}{1.079716in}}%
\pgfpathlineto{\pgfqpoint{4.274840in}{1.076268in}}%
\pgfpathlineto{\pgfqpoint{4.276080in}{1.076960in}}%
\pgfpathlineto{\pgfqpoint{4.282280in}{1.074975in}}%
\pgfpathlineto{\pgfqpoint{4.286000in}{1.073293in}}%
\pgfpathlineto{\pgfqpoint{4.287240in}{1.075489in}}%
\pgfpathlineto{\pgfqpoint{4.293440in}{1.074264in}}%
\pgfpathlineto{\pgfqpoint{4.294680in}{1.073059in}}%
\pgfpathlineto{\pgfqpoint{4.297160in}{1.075062in}}%
\pgfpathlineto{\pgfqpoint{4.300880in}{1.076055in}}%
\pgfpathlineto{\pgfqpoint{4.302120in}{1.075920in}}%
\pgfpathlineto{\pgfqpoint{4.308320in}{1.068629in}}%
\pgfpathlineto{\pgfqpoint{4.310800in}{1.067401in}}%
\pgfpathlineto{\pgfqpoint{4.313280in}{1.068140in}}%
\pgfpathlineto{\pgfqpoint{4.317000in}{1.066847in}}%
\pgfpathlineto{\pgfqpoint{4.320720in}{1.070588in}}%
\pgfpathlineto{\pgfqpoint{4.326920in}{1.068735in}}%
\pgfpathlineto{\pgfqpoint{4.329400in}{1.070347in}}%
\pgfpathlineto{\pgfqpoint{4.331880in}{1.068305in}}%
\pgfpathlineto{\pgfqpoint{4.333120in}{1.068502in}}%
\pgfpathlineto{\pgfqpoint{4.335600in}{1.071701in}}%
\pgfpathlineto{\pgfqpoint{4.338080in}{1.070642in}}%
\pgfpathlineto{\pgfqpoint{4.339320in}{1.072469in}}%
\pgfpathlineto{\pgfqpoint{4.340560in}{1.071895in}}%
\pgfpathlineto{\pgfqpoint{4.345520in}{1.075336in}}%
\pgfpathlineto{\pgfqpoint{4.346760in}{1.075751in}}%
\pgfpathlineto{\pgfqpoint{4.348000in}{1.073580in}}%
\pgfpathlineto{\pgfqpoint{4.349240in}{1.075598in}}%
\pgfpathlineto{\pgfqpoint{4.350480in}{1.073521in}}%
\pgfpathlineto{\pgfqpoint{4.351720in}{1.073830in}}%
\pgfpathlineto{\pgfqpoint{4.354200in}{1.076127in}}%
\pgfpathlineto{\pgfqpoint{4.361640in}{1.076907in}}%
\pgfpathlineto{\pgfqpoint{4.364120in}{1.079054in}}%
\pgfpathlineto{\pgfqpoint{4.369080in}{1.072470in}}%
\pgfpathlineto{\pgfqpoint{4.371560in}{1.073069in}}%
\pgfpathlineto{\pgfqpoint{4.374040in}{1.074372in}}%
\pgfpathlineto{\pgfqpoint{4.376520in}{1.074349in}}%
\pgfpathlineto{\pgfqpoint{4.379000in}{1.072952in}}%
\pgfpathlineto{\pgfqpoint{4.381480in}{1.071127in}}%
\pgfpathlineto{\pgfqpoint{4.383960in}{1.071894in}}%
\pgfpathlineto{\pgfqpoint{4.386440in}{1.070443in}}%
\pgfpathlineto{\pgfqpoint{4.388920in}{1.070081in}}%
\pgfpathlineto{\pgfqpoint{4.390160in}{1.069669in}}%
\pgfpathlineto{\pgfqpoint{4.391400in}{1.067886in}}%
\pgfpathlineto{\pgfqpoint{4.392640in}{1.068343in}}%
\pgfpathlineto{\pgfqpoint{4.396360in}{1.072756in}}%
\pgfpathlineto{\pgfqpoint{4.398840in}{1.069432in}}%
\pgfpathlineto{\pgfqpoint{4.401320in}{1.069908in}}%
\pgfpathlineto{\pgfqpoint{4.403800in}{1.067514in}}%
\pgfpathlineto{\pgfqpoint{4.405040in}{1.069290in}}%
\pgfpathlineto{\pgfqpoint{4.410000in}{1.065844in}}%
\pgfpathlineto{\pgfqpoint{4.411240in}{1.068002in}}%
\pgfpathlineto{\pgfqpoint{4.414960in}{1.067278in}}%
\pgfpathlineto{\pgfqpoint{4.417440in}{1.066654in}}%
\pgfpathlineto{\pgfqpoint{4.418680in}{1.065815in}}%
\pgfpathlineto{\pgfqpoint{4.421160in}{1.068063in}}%
\pgfpathlineto{\pgfqpoint{4.426120in}{1.068154in}}%
\pgfpathlineto{\pgfqpoint{4.433560in}{1.060169in}}%
\pgfpathlineto{\pgfqpoint{4.436040in}{1.060576in}}%
\pgfpathlineto{\pgfqpoint{4.437280in}{1.060940in}}%
\pgfpathlineto{\pgfqpoint{4.439760in}{1.059772in}}%
\pgfpathlineto{\pgfqpoint{4.441000in}{1.060431in}}%
\pgfpathlineto{\pgfqpoint{4.444720in}{1.064643in}}%
\pgfpathlineto{\pgfqpoint{4.452160in}{1.064112in}}%
\pgfpathlineto{\pgfqpoint{4.453400in}{1.065166in}}%
\pgfpathlineto{\pgfqpoint{4.455880in}{1.062623in}}%
\pgfpathlineto{\pgfqpoint{4.457120in}{1.062866in}}%
\pgfpathlineto{\pgfqpoint{4.459600in}{1.065805in}}%
\pgfpathlineto{\pgfqpoint{4.462080in}{1.065154in}}%
\pgfpathlineto{\pgfqpoint{4.463320in}{1.067271in}}%
\pgfpathlineto{\pgfqpoint{4.464560in}{1.066411in}}%
\pgfpathlineto{\pgfqpoint{4.469520in}{1.068971in}}%
\pgfpathlineto{\pgfqpoint{4.470760in}{1.069186in}}%
\pgfpathlineto{\pgfqpoint{4.472000in}{1.067422in}}%
\pgfpathlineto{\pgfqpoint{4.473240in}{1.068259in}}%
\pgfpathlineto{\pgfqpoint{4.475720in}{1.065992in}}%
\pgfpathlineto{\pgfqpoint{4.478200in}{1.067550in}}%
\pgfpathlineto{\pgfqpoint{4.485640in}{1.068536in}}%
\pgfpathlineto{\pgfqpoint{4.488120in}{1.070044in}}%
\pgfpathlineto{\pgfqpoint{4.493080in}{1.064224in}}%
\pgfpathlineto{\pgfqpoint{4.495560in}{1.065059in}}%
\pgfpathlineto{\pgfqpoint{4.499280in}{1.066977in}}%
\pgfpathlineto{\pgfqpoint{4.500520in}{1.066960in}}%
\pgfpathlineto{\pgfqpoint{4.501760in}{1.065360in}}%
\pgfpathlineto{\pgfqpoint{4.503000in}{1.065901in}}%
\pgfpathlineto{\pgfqpoint{4.505480in}{1.063863in}}%
\pgfpathlineto{\pgfqpoint{4.507960in}{1.065057in}}%
\pgfpathlineto{\pgfqpoint{4.510440in}{1.063490in}}%
\pgfpathlineto{\pgfqpoint{4.514160in}{1.062779in}}%
\pgfpathlineto{\pgfqpoint{4.515400in}{1.061090in}}%
\pgfpathlineto{\pgfqpoint{4.516640in}{1.061508in}}%
\pgfpathlineto{\pgfqpoint{4.519120in}{1.064589in}}%
\pgfpathlineto{\pgfqpoint{4.520360in}{1.065456in}}%
\pgfpathlineto{\pgfqpoint{4.522840in}{1.062110in}}%
\pgfpathlineto{\pgfqpoint{4.525320in}{1.062397in}}%
\pgfpathlineto{\pgfqpoint{4.527800in}{1.059545in}}%
\pgfpathlineto{\pgfqpoint{4.529040in}{1.061340in}}%
\pgfpathlineto{\pgfqpoint{4.534000in}{1.058032in}}%
\pgfpathlineto{\pgfqpoint{4.535240in}{1.060634in}}%
\pgfpathlineto{\pgfqpoint{4.538960in}{1.060134in}}%
\pgfpathlineto{\pgfqpoint{4.541440in}{1.059885in}}%
\pgfpathlineto{\pgfqpoint{4.542680in}{1.059040in}}%
\pgfpathlineto{\pgfqpoint{4.545160in}{1.061080in}}%
\pgfpathlineto{\pgfqpoint{4.547640in}{1.061393in}}%
\pgfpathlineto{\pgfqpoint{4.552600in}{1.058608in}}%
\pgfpathlineto{\pgfqpoint{4.557560in}{1.054146in}}%
\pgfpathlineto{\pgfqpoint{4.565000in}{1.053963in}}%
\pgfpathlineto{\pgfqpoint{4.568720in}{1.057548in}}%
\pgfpathlineto{\pgfqpoint{4.578640in}{1.056095in}}%
\pgfpathlineto{\pgfqpoint{4.581120in}{1.054829in}}%
\pgfpathlineto{\pgfqpoint{4.583600in}{1.057404in}}%
\pgfpathlineto{\pgfqpoint{4.586080in}{1.056739in}}%
\pgfpathlineto{\pgfqpoint{4.587320in}{1.058963in}}%
\pgfpathlineto{\pgfqpoint{4.588560in}{1.058004in}}%
\pgfpathlineto{\pgfqpoint{4.594760in}{1.060998in}}%
\pgfpathlineto{\pgfqpoint{4.596000in}{1.059157in}}%
\pgfpathlineto{\pgfqpoint{4.597240in}{1.060158in}}%
\pgfpathlineto{\pgfqpoint{4.599720in}{1.058296in}}%
\pgfpathlineto{\pgfqpoint{4.602200in}{1.059809in}}%
\pgfpathlineto{\pgfqpoint{4.607160in}{1.059021in}}%
\pgfpathlineto{\pgfqpoint{4.610880in}{1.061688in}}%
\pgfpathlineto{\pgfqpoint{4.612120in}{1.062409in}}%
\pgfpathlineto{\pgfqpoint{4.618320in}{1.055766in}}%
\pgfpathlineto{\pgfqpoint{4.623280in}{1.058771in}}%
\pgfpathlineto{\pgfqpoint{4.624520in}{1.058633in}}%
\pgfpathlineto{\pgfqpoint{4.625760in}{1.056528in}}%
\pgfpathlineto{\pgfqpoint{4.627000in}{1.057063in}}%
\pgfpathlineto{\pgfqpoint{4.629480in}{1.055407in}}%
\pgfpathlineto{\pgfqpoint{4.631960in}{1.056773in}}%
\pgfpathlineto{\pgfqpoint{4.635680in}{1.054946in}}%
\pgfpathlineto{\pgfqpoint{4.638160in}{1.054538in}}%
\pgfpathlineto{\pgfqpoint{4.639400in}{1.052948in}}%
\pgfpathlineto{\pgfqpoint{4.641880in}{1.055392in}}%
\pgfpathlineto{\pgfqpoint{4.644360in}{1.056929in}}%
\pgfpathlineto{\pgfqpoint{4.646840in}{1.053570in}}%
\pgfpathlineto{\pgfqpoint{4.649320in}{1.054281in}}%
\pgfpathlineto{\pgfqpoint{4.651800in}{1.051931in}}%
\pgfpathlineto{\pgfqpoint{4.653040in}{1.053618in}}%
\pgfpathlineto{\pgfqpoint{4.655520in}{1.051771in}}%
\pgfpathlineto{\pgfqpoint{4.656760in}{1.051643in}}%
\pgfpathlineto{\pgfqpoint{4.658000in}{1.050255in}}%
\pgfpathlineto{\pgfqpoint{4.659240in}{1.052857in}}%
\pgfpathlineto{\pgfqpoint{4.667920in}{1.053302in}}%
\pgfpathlineto{\pgfqpoint{4.671640in}{1.053463in}}%
\pgfpathlineto{\pgfqpoint{4.675360in}{1.051267in}}%
\pgfpathlineto{\pgfqpoint{4.681560in}{1.045131in}}%
\pgfpathlineto{\pgfqpoint{4.685280in}{1.045416in}}%
\pgfpathlineto{\pgfqpoint{4.687760in}{1.044742in}}%
\pgfpathlineto{\pgfqpoint{4.690240in}{1.047843in}}%
\pgfpathlineto{\pgfqpoint{4.691480in}{1.049757in}}%
\pgfpathlineto{\pgfqpoint{4.693960in}{1.049170in}}%
\pgfpathlineto{\pgfqpoint{4.695200in}{1.049098in}}%
\pgfpathlineto{\pgfqpoint{4.697680in}{1.050935in}}%
\pgfpathlineto{\pgfqpoint{4.698920in}{1.049381in}}%
\pgfpathlineto{\pgfqpoint{4.701400in}{1.050999in}}%
\pgfpathlineto{\pgfqpoint{4.703880in}{1.048360in}}%
\pgfpathlineto{\pgfqpoint{4.705120in}{1.048344in}}%
\pgfpathlineto{\pgfqpoint{4.708840in}{1.050601in}}%
\pgfpathlineto{\pgfqpoint{4.710080in}{1.049961in}}%
\pgfpathlineto{\pgfqpoint{4.711320in}{1.051953in}}%
\pgfpathlineto{\pgfqpoint{4.712560in}{1.051114in}}%
\pgfpathlineto{\pgfqpoint{4.718760in}{1.054503in}}%
\pgfpathlineto{\pgfqpoint{4.722480in}{1.050270in}}%
\pgfpathlineto{\pgfqpoint{4.723720in}{1.050162in}}%
\pgfpathlineto{\pgfqpoint{4.726200in}{1.051766in}}%
\pgfpathlineto{\pgfqpoint{4.732400in}{1.052599in}}%
\pgfpathlineto{\pgfqpoint{4.736120in}{1.055854in}}%
\pgfpathlineto{\pgfqpoint{4.742320in}{1.049556in}}%
\pgfpathlineto{\pgfqpoint{4.748520in}{1.051642in}}%
\pgfpathlineto{\pgfqpoint{4.749760in}{1.049671in}}%
\pgfpathlineto{\pgfqpoint{4.751000in}{1.050011in}}%
\pgfpathlineto{\pgfqpoint{4.753480in}{1.048455in}}%
\pgfpathlineto{\pgfqpoint{4.755960in}{1.050399in}}%
\pgfpathlineto{\pgfqpoint{4.759680in}{1.048068in}}%
\pgfpathlineto{\pgfqpoint{4.762160in}{1.047320in}}%
\pgfpathlineto{\pgfqpoint{4.763400in}{1.045311in}}%
\pgfpathlineto{\pgfqpoint{4.764640in}{1.045616in}}%
\pgfpathlineto{\pgfqpoint{4.767120in}{1.048150in}}%
\pgfpathlineto{\pgfqpoint{4.768360in}{1.048632in}}%
\pgfpathlineto{\pgfqpoint{4.770840in}{1.045936in}}%
\pgfpathlineto{\pgfqpoint{4.773320in}{1.047905in}}%
\pgfpathlineto{\pgfqpoint{4.775800in}{1.045782in}}%
\pgfpathlineto{\pgfqpoint{4.777040in}{1.047863in}}%
\pgfpathlineto{\pgfqpoint{4.782000in}{1.044163in}}%
\pgfpathlineto{\pgfqpoint{4.783240in}{1.046378in}}%
\pgfpathlineto{\pgfqpoint{4.790680in}{1.045629in}}%
\pgfpathlineto{\pgfqpoint{4.794400in}{1.048517in}}%
\pgfpathlineto{\pgfqpoint{4.798120in}{1.047116in}}%
\pgfpathlineto{\pgfqpoint{4.808040in}{1.038122in}}%
\pgfpathlineto{\pgfqpoint{4.813000in}{1.038616in}}%
\pgfpathlineto{\pgfqpoint{4.816720in}{1.041881in}}%
\pgfpathlineto{\pgfqpoint{4.825400in}{1.042947in}}%
\pgfpathlineto{\pgfqpoint{4.829120in}{1.039801in}}%
\pgfpathlineto{\pgfqpoint{4.832840in}{1.042175in}}%
\pgfpathlineto{\pgfqpoint{4.834080in}{1.041314in}}%
\pgfpathlineto{\pgfqpoint{4.835320in}{1.043227in}}%
\pgfpathlineto{\pgfqpoint{4.836560in}{1.042412in}}%
\pgfpathlineto{\pgfqpoint{4.842760in}{1.046079in}}%
\pgfpathlineto{\pgfqpoint{4.844000in}{1.043877in}}%
\pgfpathlineto{\pgfqpoint{4.845240in}{1.043946in}}%
\pgfpathlineto{\pgfqpoint{4.846480in}{1.041798in}}%
\pgfpathlineto{\pgfqpoint{4.855160in}{1.042470in}}%
\pgfpathlineto{\pgfqpoint{4.858880in}{1.046758in}}%
\pgfpathlineto{\pgfqpoint{4.860120in}{1.047609in}}%
\pgfpathlineto{\pgfqpoint{4.866320in}{1.042131in}}%
\pgfpathlineto{\pgfqpoint{4.868800in}{1.043835in}}%
\pgfpathlineto{\pgfqpoint{4.871280in}{1.044369in}}%
\pgfpathlineto{\pgfqpoint{4.872520in}{1.043849in}}%
\pgfpathlineto{\pgfqpoint{4.873760in}{1.041972in}}%
\pgfpathlineto{\pgfqpoint{4.876240in}{1.041619in}}%
\pgfpathlineto{\pgfqpoint{4.877480in}{1.041118in}}%
\pgfpathlineto{\pgfqpoint{4.879960in}{1.043312in}}%
\pgfpathlineto{\pgfqpoint{4.886160in}{1.039709in}}%
\pgfpathlineto{\pgfqpoint{4.887400in}{1.037503in}}%
\pgfpathlineto{\pgfqpoint{4.888640in}{1.037973in}}%
\pgfpathlineto{\pgfqpoint{4.892360in}{1.042113in}}%
\pgfpathlineto{\pgfqpoint{4.894840in}{1.039863in}}%
\pgfpathlineto{\pgfqpoint{4.897320in}{1.042136in}}%
\pgfpathlineto{\pgfqpoint{4.899800in}{1.039151in}}%
\pgfpathlineto{\pgfqpoint{4.901040in}{1.041168in}}%
\pgfpathlineto{\pgfqpoint{4.907240in}{1.040105in}}%
\pgfpathlineto{\pgfqpoint{4.913440in}{1.039180in}}%
\pgfpathlineto{\pgfqpoint{4.914680in}{1.038580in}}%
\pgfpathlineto{\pgfqpoint{4.918400in}{1.041865in}}%
\pgfpathlineto{\pgfqpoint{4.920880in}{1.041587in}}%
\pgfpathlineto{\pgfqpoint{4.922120in}{1.041184in}}%
\pgfpathlineto{\pgfqpoint{4.925840in}{1.036147in}}%
\pgfpathlineto{\pgfqpoint{4.932040in}{1.031440in}}%
\pgfpathlineto{\pgfqpoint{4.937000in}{1.032666in}}%
\pgfpathlineto{\pgfqpoint{4.940720in}{1.035398in}}%
\pgfpathlineto{\pgfqpoint{4.944440in}{1.034805in}}%
\pgfpathlineto{\pgfqpoint{4.945680in}{1.035306in}}%
\pgfpathlineto{\pgfqpoint{4.946920in}{1.034055in}}%
\pgfpathlineto{\pgfqpoint{4.949400in}{1.035732in}}%
\pgfpathlineto{\pgfqpoint{4.953120in}{1.032904in}}%
\pgfpathlineto{\pgfqpoint{4.955600in}{1.035462in}}%
\pgfpathlineto{\pgfqpoint{4.958080in}{1.032671in}}%
\pgfpathlineto{\pgfqpoint{4.959320in}{1.034471in}}%
\pgfpathlineto{\pgfqpoint{4.961800in}{1.034194in}}%
\pgfpathlineto{\pgfqpoint{4.964280in}{1.036621in}}%
\pgfpathlineto{\pgfqpoint{4.966760in}{1.037308in}}%
\pgfpathlineto{\pgfqpoint{4.970480in}{1.032026in}}%
\pgfpathlineto{\pgfqpoint{4.979160in}{1.032950in}}%
\pgfpathlineto{\pgfqpoint{4.982880in}{1.037006in}}%
\pgfpathlineto{\pgfqpoint{4.984120in}{1.038005in}}%
\pgfpathlineto{\pgfqpoint{4.990320in}{1.032430in}}%
\pgfpathlineto{\pgfqpoint{4.996520in}{1.035224in}}%
\pgfpathlineto{\pgfqpoint{4.997760in}{1.033683in}}%
\pgfpathlineto{\pgfqpoint{4.999000in}{1.034195in}}%
\pgfpathlineto{\pgfqpoint{5.001480in}{1.032720in}}%
\pgfpathlineto{\pgfqpoint{5.003960in}{1.035490in}}%
\pgfpathlineto{\pgfqpoint{5.010160in}{1.032239in}}%
\pgfpathlineto{\pgfqpoint{5.011400in}{1.029947in}}%
\pgfpathlineto{\pgfqpoint{5.016360in}{1.034284in}}%
\pgfpathlineto{\pgfqpoint{5.018840in}{1.031023in}}%
\pgfpathlineto{\pgfqpoint{5.021320in}{1.034042in}}%
\pgfpathlineto{\pgfqpoint{5.023800in}{1.031362in}}%
\pgfpathlineto{\pgfqpoint{5.025040in}{1.033561in}}%
\pgfpathlineto{\pgfqpoint{5.030000in}{1.032037in}}%
\pgfpathlineto{\pgfqpoint{5.032480in}{1.034082in}}%
\pgfpathlineto{\pgfqpoint{5.036200in}{1.033488in}}%
\pgfpathlineto{\pgfqpoint{5.038680in}{1.032480in}}%
\pgfpathlineto{\pgfqpoint{5.043640in}{1.036393in}}%
\pgfpathlineto{\pgfqpoint{5.046120in}{1.035895in}}%
\pgfpathlineto{\pgfqpoint{5.049840in}{1.030231in}}%
\pgfpathlineto{\pgfqpoint{5.053560in}{1.025732in}}%
\pgfpathlineto{\pgfqpoint{5.054800in}{1.026592in}}%
\pgfpathlineto{\pgfqpoint{5.056040in}{1.025136in}}%
\pgfpathlineto{\pgfqpoint{5.062240in}{1.027475in}}%
\pgfpathlineto{\pgfqpoint{5.064720in}{1.029279in}}%
\pgfpathlineto{\pgfqpoint{5.072160in}{1.028605in}}%
\pgfpathlineto{\pgfqpoint{5.073400in}{1.029049in}}%
\pgfpathlineto{\pgfqpoint{5.077120in}{1.026564in}}%
\pgfpathlineto{\pgfqpoint{5.079600in}{1.028883in}}%
\pgfpathlineto{\pgfqpoint{5.082080in}{1.026178in}}%
\pgfpathlineto{\pgfqpoint{5.083320in}{1.027315in}}%
\pgfpathlineto{\pgfqpoint{5.085800in}{1.026760in}}%
\pgfpathlineto{\pgfqpoint{5.089520in}{1.030007in}}%
\pgfpathlineto{\pgfqpoint{5.090760in}{1.029891in}}%
\pgfpathlineto{\pgfqpoint{5.094480in}{1.025720in}}%
\pgfpathlineto{\pgfqpoint{5.098200in}{1.026049in}}%
\pgfpathlineto{\pgfqpoint{5.103160in}{1.026426in}}%
\pgfpathlineto{\pgfqpoint{5.106880in}{1.031009in}}%
\pgfpathlineto{\pgfqpoint{5.108120in}{1.032009in}}%
\pgfpathlineto{\pgfqpoint{5.110600in}{1.030023in}}%
\pgfpathlineto{\pgfqpoint{5.113080in}{1.025762in}}%
\pgfpathlineto{\pgfqpoint{5.116800in}{1.029548in}}%
\pgfpathlineto{\pgfqpoint{5.120520in}{1.030662in}}%
\pgfpathlineto{\pgfqpoint{5.121760in}{1.029354in}}%
\pgfpathlineto{\pgfqpoint{5.123000in}{1.029675in}}%
\pgfpathlineto{\pgfqpoint{5.125480in}{1.027184in}}%
\pgfpathlineto{\pgfqpoint{5.127960in}{1.029060in}}%
\pgfpathlineto{\pgfqpoint{5.134160in}{1.027297in}}%
\pgfpathlineto{\pgfqpoint{5.135400in}{1.024971in}}%
\pgfpathlineto{\pgfqpoint{5.140360in}{1.029155in}}%
\pgfpathlineto{\pgfqpoint{5.142840in}{1.026045in}}%
\pgfpathlineto{\pgfqpoint{5.145320in}{1.028817in}}%
\pgfpathlineto{\pgfqpoint{5.147800in}{1.026277in}}%
\pgfpathlineto{\pgfqpoint{5.149040in}{1.028562in}}%
\pgfpathlineto{\pgfqpoint{5.151520in}{1.027745in}}%
\pgfpathlineto{\pgfqpoint{5.152760in}{1.027989in}}%
\pgfpathlineto{\pgfqpoint{5.154000in}{1.026387in}}%
\pgfpathlineto{\pgfqpoint{5.155240in}{1.029109in}}%
\pgfpathlineto{\pgfqpoint{5.158960in}{1.029766in}}%
\pgfpathlineto{\pgfqpoint{5.161440in}{1.029233in}}%
\pgfpathlineto{\pgfqpoint{5.162680in}{1.028865in}}%
\pgfpathlineto{\pgfqpoint{5.167640in}{1.031809in}}%
\pgfpathlineto{\pgfqpoint{5.170120in}{1.032235in}}%
\pgfpathlineto{\pgfqpoint{5.176320in}{1.021713in}}%
\pgfpathlineto{\pgfqpoint{5.177560in}{1.021208in}}%
\pgfpathlineto{\pgfqpoint{5.178800in}{1.021928in}}%
\pgfpathlineto{\pgfqpoint{5.180040in}{1.020446in}}%
\pgfpathlineto{\pgfqpoint{5.182520in}{1.020733in}}%
\pgfpathlineto{\pgfqpoint{5.186240in}{1.022164in}}%
\pgfpathlineto{\pgfqpoint{5.187480in}{1.024498in}}%
\pgfpathlineto{\pgfqpoint{5.192440in}{1.023181in}}%
\pgfpathlineto{\pgfqpoint{5.193680in}{1.023940in}}%
\pgfpathlineto{\pgfqpoint{5.194920in}{1.022977in}}%
\pgfpathlineto{\pgfqpoint{5.197400in}{1.023923in}}%
\pgfpathlineto{\pgfqpoint{5.199880in}{1.021845in}}%
\pgfpathlineto{\pgfqpoint{5.202360in}{1.023069in}}%
\pgfpathlineto{\pgfqpoint{5.203600in}{1.024308in}}%
\pgfpathlineto{\pgfqpoint{5.208560in}{1.021891in}}%
\pgfpathlineto{\pgfqpoint{5.211040in}{1.024295in}}%
\pgfpathlineto{\pgfqpoint{5.214760in}{1.026174in}}%
\pgfpathlineto{\pgfqpoint{5.217240in}{1.023837in}}%
\pgfpathlineto{\pgfqpoint{5.218480in}{1.022007in}}%
\pgfpathlineto{\pgfqpoint{5.222200in}{1.022898in}}%
\pgfpathlineto{\pgfqpoint{5.225920in}{1.022011in}}%
\pgfpathlineto{\pgfqpoint{5.227160in}{1.022824in}}%
\pgfpathlineto{\pgfqpoint{5.230880in}{1.027769in}}%
\pgfpathlineto{\pgfqpoint{5.232120in}{1.028535in}}%
\pgfpathlineto{\pgfqpoint{5.234600in}{1.026075in}}%
\pgfpathlineto{\pgfqpoint{5.237080in}{1.020879in}}%
\pgfpathlineto{\pgfqpoint{5.242040in}{1.026262in}}%
\pgfpathlineto{\pgfqpoint{5.243280in}{1.025582in}}%
\pgfpathlineto{\pgfqpoint{5.244520in}{1.026292in}}%
\pgfpathlineto{\pgfqpoint{5.245760in}{1.025415in}}%
\pgfpathlineto{\pgfqpoint{5.247000in}{1.025941in}}%
\pgfpathlineto{\pgfqpoint{5.249480in}{1.023439in}}%
\pgfpathlineto{\pgfqpoint{5.251960in}{1.026439in}}%
\pgfpathlineto{\pgfqpoint{5.256920in}{1.024969in}}%
\pgfpathlineto{\pgfqpoint{5.258160in}{1.024271in}}%
\pgfpathlineto{\pgfqpoint{5.259400in}{1.021654in}}%
\pgfpathlineto{\pgfqpoint{5.264360in}{1.026012in}}%
\pgfpathlineto{\pgfqpoint{5.266840in}{1.022030in}}%
\pgfpathlineto{\pgfqpoint{5.269320in}{1.024131in}}%
\pgfpathlineto{\pgfqpoint{5.271800in}{1.022253in}}%
\pgfpathlineto{\pgfqpoint{5.273040in}{1.024669in}}%
\pgfpathlineto{\pgfqpoint{5.275520in}{1.024019in}}%
\pgfpathlineto{\pgfqpoint{5.276760in}{1.024361in}}%
\pgfpathlineto{\pgfqpoint{5.278000in}{1.022566in}}%
\pgfpathlineto{\pgfqpoint{5.280480in}{1.025969in}}%
\pgfpathlineto{\pgfqpoint{5.284200in}{1.027430in}}%
\pgfpathlineto{\pgfqpoint{5.286680in}{1.026775in}}%
\pgfpathlineto{\pgfqpoint{5.294120in}{1.029058in}}%
\pgfpathlineto{\pgfqpoint{5.299080in}{1.022333in}}%
\pgfpathlineto{\pgfqpoint{5.300320in}{1.018411in}}%
\pgfpathlineto{\pgfqpoint{5.310240in}{1.020724in}}%
\pgfpathlineto{\pgfqpoint{5.312720in}{1.022719in}}%
\pgfpathlineto{\pgfqpoint{5.321400in}{1.023701in}}%
\pgfpathlineto{\pgfqpoint{5.323880in}{1.021976in}}%
\pgfpathlineto{\pgfqpoint{5.328840in}{1.023089in}}%
\pgfpathlineto{\pgfqpoint{5.330080in}{1.020677in}}%
\pgfpathlineto{\pgfqpoint{5.332560in}{1.021534in}}%
\pgfpathlineto{\pgfqpoint{5.337520in}{1.024312in}}%
\pgfpathlineto{\pgfqpoint{5.340000in}{1.023109in}}%
\pgfpathlineto{\pgfqpoint{5.342480in}{1.017414in}}%
\pgfpathlineto{\pgfqpoint{5.351160in}{1.019306in}}%
\pgfpathlineto{\pgfqpoint{5.354880in}{1.023842in}}%
\pgfpathlineto{\pgfqpoint{5.357360in}{1.023931in}}%
\pgfpathlineto{\pgfqpoint{5.359840in}{1.019874in}}%
\pgfpathlineto{\pgfqpoint{5.361080in}{1.016970in}}%
\pgfpathlineto{\pgfqpoint{5.364800in}{1.021321in}}%
\pgfpathlineto{\pgfqpoint{5.371000in}{1.021687in}}%
\pgfpathlineto{\pgfqpoint{5.373480in}{1.018596in}}%
\pgfpathlineto{\pgfqpoint{5.375960in}{1.022091in}}%
\pgfpathlineto{\pgfqpoint{5.382160in}{1.021534in}}%
\pgfpathlineto{\pgfqpoint{5.383400in}{1.019467in}}%
\pgfpathlineto{\pgfqpoint{5.388360in}{1.023170in}}%
\pgfpathlineto{\pgfqpoint{5.390840in}{1.019202in}}%
\pgfpathlineto{\pgfqpoint{5.393320in}{1.021280in}}%
\pgfpathlineto{\pgfqpoint{5.395800in}{1.019112in}}%
\pgfpathlineto{\pgfqpoint{5.397040in}{1.021276in}}%
\pgfpathlineto{\pgfqpoint{5.398280in}{1.019997in}}%
\pgfpathlineto{\pgfqpoint{5.400760in}{1.020801in}}%
\pgfpathlineto{\pgfqpoint{5.402000in}{1.019279in}}%
\pgfpathlineto{\pgfqpoint{5.403240in}{1.022901in}}%
\pgfpathlineto{\pgfqpoint{5.406960in}{1.023101in}}%
\pgfpathlineto{\pgfqpoint{5.409440in}{1.023770in}}%
\pgfpathlineto{\pgfqpoint{5.414400in}{1.025021in}}%
\pgfpathlineto{\pgfqpoint{5.419360in}{1.024120in}}%
\pgfpathlineto{\pgfqpoint{5.421840in}{1.021139in}}%
\pgfpathlineto{\pgfqpoint{5.425560in}{1.014725in}}%
\pgfpathlineto{\pgfqpoint{5.426800in}{1.015346in}}%
\pgfpathlineto{\pgfqpoint{5.428040in}{1.014205in}}%
\pgfpathlineto{\pgfqpoint{5.431760in}{1.015344in}}%
\pgfpathlineto{\pgfqpoint{5.441680in}{1.018727in}}%
\pgfpathlineto{\pgfqpoint{5.442920in}{1.017678in}}%
\pgfpathlineto{\pgfqpoint{5.445400in}{1.018530in}}%
\pgfpathlineto{\pgfqpoint{5.447880in}{1.016052in}}%
\pgfpathlineto{\pgfqpoint{5.451600in}{1.017435in}}%
\pgfpathlineto{\pgfqpoint{5.452840in}{1.016173in}}%
\pgfpathlineto{\pgfqpoint{5.454080in}{1.013220in}}%
\pgfpathlineto{\pgfqpoint{5.456560in}{1.014606in}}%
\pgfpathlineto{\pgfqpoint{5.461520in}{1.017280in}}%
\pgfpathlineto{\pgfqpoint{5.464000in}{1.016099in}}%
\pgfpathlineto{\pgfqpoint{5.467720in}{1.013161in}}%
\pgfpathlineto{\pgfqpoint{5.468960in}{1.013827in}}%
\pgfpathlineto{\pgfqpoint{5.471440in}{1.013238in}}%
\pgfpathlineto{\pgfqpoint{5.475160in}{1.013870in}}%
\pgfpathlineto{\pgfqpoint{5.480120in}{1.019934in}}%
\pgfpathlineto{\pgfqpoint{5.482600in}{1.018144in}}%
\pgfpathlineto{\pgfqpoint{5.485080in}{1.012885in}}%
\pgfpathlineto{\pgfqpoint{5.487560in}{1.015974in}}%
\pgfpathlineto{\pgfqpoint{5.488800in}{1.017446in}}%
\pgfpathlineto{\pgfqpoint{5.491280in}{1.017077in}}%
\pgfpathlineto{\pgfqpoint{5.492520in}{1.018597in}}%
\pgfpathlineto{\pgfqpoint{5.493760in}{1.017415in}}%
\pgfpathlineto{\pgfqpoint{5.495000in}{1.017977in}}%
\pgfpathlineto{\pgfqpoint{5.497480in}{1.015520in}}%
\pgfpathlineto{\pgfqpoint{5.499960in}{1.019515in}}%
\pgfpathlineto{\pgfqpoint{5.501200in}{1.019868in}}%
\pgfpathlineto{\pgfqpoint{5.502440in}{1.018942in}}%
\pgfpathlineto{\pgfqpoint{5.503680in}{1.019408in}}%
\pgfpathlineto{\pgfqpoint{5.504920in}{1.018422in}}%
\pgfpathlineto{\pgfqpoint{5.506160in}{1.019071in}}%
\pgfpathlineto{\pgfqpoint{5.507400in}{1.017335in}}%
\pgfpathlineto{\pgfqpoint{5.513600in}{1.019023in}}%
\pgfpathlineto{\pgfqpoint{5.514840in}{1.016367in}}%
\pgfpathlineto{\pgfqpoint{5.517320in}{1.018729in}}%
\pgfpathlineto{\pgfqpoint{5.519800in}{1.016013in}}%
\pgfpathlineto{\pgfqpoint{5.521040in}{1.017818in}}%
\pgfpathlineto{\pgfqpoint{5.523520in}{1.017359in}}%
\pgfpathlineto{\pgfqpoint{5.524760in}{1.018270in}}%
\pgfpathlineto{\pgfqpoint{5.526000in}{1.016432in}}%
\pgfpathlineto{\pgfqpoint{5.527240in}{1.019749in}}%
\pgfpathlineto{\pgfqpoint{5.530960in}{1.019175in}}%
\pgfpathlineto{\pgfqpoint{5.535920in}{1.021553in}}%
\pgfpathlineto{\pgfqpoint{5.542120in}{1.022995in}}%
\pgfpathlineto{\pgfqpoint{5.545840in}{1.018121in}}%
\pgfpathlineto{\pgfqpoint{5.548320in}{1.011559in}}%
\pgfpathlineto{\pgfqpoint{5.550800in}{1.011949in}}%
\pgfpathlineto{\pgfqpoint{5.552040in}{1.010344in}}%
\pgfpathlineto{\pgfqpoint{5.554520in}{1.011300in}}%
\pgfpathlineto{\pgfqpoint{5.558240in}{1.011629in}}%
\pgfpathlineto{\pgfqpoint{5.559480in}{1.013195in}}%
\pgfpathlineto{\pgfqpoint{5.561960in}{1.012926in}}%
\pgfpathlineto{\pgfqpoint{5.564440in}{1.014422in}}%
\pgfpathlineto{\pgfqpoint{5.565680in}{1.015203in}}%
\pgfpathlineto{\pgfqpoint{5.566920in}{1.013585in}}%
\pgfpathlineto{\pgfqpoint{5.569400in}{1.015258in}}%
\pgfpathlineto{\pgfqpoint{5.570640in}{1.013670in}}%
\pgfpathlineto{\pgfqpoint{5.575600in}{1.016524in}}%
\pgfpathlineto{\pgfqpoint{5.578080in}{1.013023in}}%
\pgfpathlineto{\pgfqpoint{5.579320in}{1.014315in}}%
\pgfpathlineto{\pgfqpoint{5.580560in}{1.013806in}}%
\pgfpathlineto{\pgfqpoint{5.585520in}{1.015933in}}%
\pgfpathlineto{\pgfqpoint{5.588000in}{1.014532in}}%
\pgfpathlineto{\pgfqpoint{5.589240in}{1.017363in}}%
\pgfpathlineto{\pgfqpoint{5.590480in}{1.015051in}}%
\pgfpathlineto{\pgfqpoint{5.595440in}{1.015800in}}%
\pgfpathlineto{\pgfqpoint{5.599160in}{1.016784in}}%
\pgfpathlineto{\pgfqpoint{5.604120in}{1.020359in}}%
\pgfpathlineto{\pgfqpoint{5.606600in}{1.018686in}}%
\pgfpathlineto{\pgfqpoint{5.609080in}{1.012950in}}%
\pgfpathlineto{\pgfqpoint{5.610320in}{1.015579in}}%
\pgfpathlineto{\pgfqpoint{5.611560in}{1.015394in}}%
\pgfpathlineto{\pgfqpoint{5.612800in}{1.016796in}}%
\pgfpathlineto{\pgfqpoint{5.615280in}{1.016375in}}%
\pgfpathlineto{\pgfqpoint{5.616520in}{1.017147in}}%
\pgfpathlineto{\pgfqpoint{5.617760in}{1.015694in}}%
\pgfpathlineto{\pgfqpoint{5.619000in}{1.016731in}}%
\pgfpathlineto{\pgfqpoint{5.621480in}{1.014739in}}%
\pgfpathlineto{\pgfqpoint{5.623960in}{1.018443in}}%
\pgfpathlineto{\pgfqpoint{5.625200in}{1.018818in}}%
\pgfpathlineto{\pgfqpoint{5.626440in}{1.017367in}}%
\pgfpathlineto{\pgfqpoint{5.627680in}{1.018795in}}%
\pgfpathlineto{\pgfqpoint{5.632640in}{1.018151in}}%
\pgfpathlineto{\pgfqpoint{5.636360in}{1.019009in}}%
\pgfpathlineto{\pgfqpoint{5.640080in}{1.015976in}}%
\pgfpathlineto{\pgfqpoint{5.641320in}{1.017483in}}%
\pgfpathlineto{\pgfqpoint{5.642560in}{1.014639in}}%
\pgfpathlineto{\pgfqpoint{5.643800in}{1.015215in}}%
\pgfpathlineto{\pgfqpoint{5.645040in}{1.017428in}}%
\pgfpathlineto{\pgfqpoint{5.648760in}{1.017311in}}%
\pgfpathlineto{\pgfqpoint{5.650000in}{1.015484in}}%
\pgfpathlineto{\pgfqpoint{5.651240in}{1.019310in}}%
\pgfpathlineto{\pgfqpoint{5.652480in}{1.018879in}}%
\pgfpathlineto{\pgfqpoint{5.656200in}{1.020672in}}%
\pgfpathlineto{\pgfqpoint{5.658680in}{1.021138in}}%
\pgfpathlineto{\pgfqpoint{5.661160in}{1.022306in}}%
\pgfpathlineto{\pgfqpoint{5.667360in}{1.022118in}}%
\pgfpathlineto{\pgfqpoint{5.671080in}{1.014438in}}%
\pgfpathlineto{\pgfqpoint{5.672320in}{1.011772in}}%
\pgfpathlineto{\pgfqpoint{5.674800in}{1.011734in}}%
\pgfpathlineto{\pgfqpoint{5.676040in}{1.010053in}}%
\pgfpathlineto{\pgfqpoint{5.679760in}{1.011267in}}%
\pgfpathlineto{\pgfqpoint{5.689680in}{1.013790in}}%
\pgfpathlineto{\pgfqpoint{5.692160in}{1.011715in}}%
\pgfpathlineto{\pgfqpoint{5.693400in}{1.013504in}}%
\pgfpathlineto{\pgfqpoint{5.694640in}{1.011769in}}%
\pgfpathlineto{\pgfqpoint{5.697120in}{1.013397in}}%
\pgfpathlineto{\pgfqpoint{5.698360in}{1.014144in}}%
\pgfpathlineto{\pgfqpoint{5.700840in}{1.011845in}}%
\pgfpathlineto{\pgfqpoint{5.702080in}{1.009449in}}%
\pgfpathlineto{\pgfqpoint{5.703320in}{1.010460in}}%
\pgfpathlineto{\pgfqpoint{5.704560in}{1.009206in}}%
\pgfpathlineto{\pgfqpoint{5.708280in}{1.010760in}}%
\pgfpathlineto{\pgfqpoint{5.713240in}{1.012482in}}%
\pgfpathlineto{\pgfqpoint{5.715720in}{1.009905in}}%
\pgfpathlineto{\pgfqpoint{5.718200in}{1.011104in}}%
\pgfpathlineto{\pgfqpoint{5.730600in}{1.013767in}}%
\pgfpathlineto{\pgfqpoint{5.733080in}{1.007655in}}%
\pgfpathlineto{\pgfqpoint{5.735560in}{1.011207in}}%
\pgfpathlineto{\pgfqpoint{5.736800in}{1.012819in}}%
\pgfpathlineto{\pgfqpoint{5.739280in}{1.012258in}}%
\pgfpathlineto{\pgfqpoint{5.740520in}{1.013071in}}%
\pgfpathlineto{\pgfqpoint{5.741760in}{1.011569in}}%
\pgfpathlineto{\pgfqpoint{5.743000in}{1.012552in}}%
\pgfpathlineto{\pgfqpoint{5.745480in}{1.010974in}}%
\pgfpathlineto{\pgfqpoint{5.747960in}{1.013939in}}%
\pgfpathlineto{\pgfqpoint{5.749200in}{1.014258in}}%
\pgfpathlineto{\pgfqpoint{5.750440in}{1.012700in}}%
\pgfpathlineto{\pgfqpoint{5.751680in}{1.013816in}}%
\pgfpathlineto{\pgfqpoint{5.752920in}{1.013272in}}%
\pgfpathlineto{\pgfqpoint{5.755400in}{1.014291in}}%
\pgfpathlineto{\pgfqpoint{5.760360in}{1.015267in}}%
\pgfpathlineto{\pgfqpoint{5.764080in}{1.011525in}}%
\pgfpathlineto{\pgfqpoint{5.765320in}{1.012638in}}%
\pgfpathlineto{\pgfqpoint{5.766560in}{1.009990in}}%
\pgfpathlineto{\pgfqpoint{5.770280in}{1.013246in}}%
\pgfpathlineto{\pgfqpoint{5.771520in}{1.012432in}}%
\pgfpathlineto{\pgfqpoint{5.772760in}{1.013298in}}%
\pgfpathlineto{\pgfqpoint{5.774000in}{1.011494in}}%
\pgfpathlineto{\pgfqpoint{5.776480in}{1.015521in}}%
\pgfpathlineto{\pgfqpoint{5.780200in}{1.018370in}}%
\pgfpathlineto{\pgfqpoint{5.782680in}{1.020373in}}%
\pgfpathlineto{\pgfqpoint{5.785160in}{1.022082in}}%
\pgfpathlineto{\pgfqpoint{5.790120in}{1.024317in}}%
\pgfpathlineto{\pgfqpoint{5.792600in}{1.021276in}}%
\pgfpathlineto{\pgfqpoint{5.795080in}{1.015506in}}%
\pgfpathlineto{\pgfqpoint{5.797560in}{1.013209in}}%
\pgfpathlineto{\pgfqpoint{5.802520in}{1.012927in}}%
\pgfpathlineto{\pgfqpoint{5.805000in}{1.013231in}}%
\pgfpathlineto{\pgfqpoint{5.809960in}{1.012937in}}%
\pgfpathlineto{\pgfqpoint{5.813680in}{1.014472in}}%
\pgfpathlineto{\pgfqpoint{5.816160in}{1.012704in}}%
\pgfpathlineto{\pgfqpoint{5.817400in}{1.014600in}}%
\pgfpathlineto{\pgfqpoint{5.818640in}{1.012477in}}%
\pgfpathlineto{\pgfqpoint{5.819880in}{1.012561in}}%
\pgfpathlineto{\pgfqpoint{5.822360in}{1.014873in}}%
\pgfpathlineto{\pgfqpoint{5.823600in}{1.014943in}}%
\pgfpathlineto{\pgfqpoint{5.826080in}{1.010853in}}%
\pgfpathlineto{\pgfqpoint{5.827320in}{1.012155in}}%
\pgfpathlineto{\pgfqpoint{5.828560in}{1.011088in}}%
\pgfpathlineto{\pgfqpoint{5.831040in}{1.012126in}}%
\pgfpathlineto{\pgfqpoint{5.836000in}{1.010819in}}%
\pgfpathlineto{\pgfqpoint{5.837240in}{1.013326in}}%
\pgfpathlineto{\pgfqpoint{5.838480in}{1.011359in}}%
\pgfpathlineto{\pgfqpoint{5.839720in}{1.011690in}}%
\pgfpathlineto{\pgfqpoint{5.842200in}{1.013688in}}%
\pgfpathlineto{\pgfqpoint{5.844680in}{1.013923in}}%
\pgfpathlineto{\pgfqpoint{5.848400in}{1.014221in}}%
\pgfpathlineto{\pgfqpoint{5.853360in}{1.016087in}}%
\pgfpathlineto{\pgfqpoint{5.855840in}{1.012296in}}%
\pgfpathlineto{\pgfqpoint{5.857080in}{1.008454in}}%
\pgfpathlineto{\pgfqpoint{5.859560in}{1.010932in}}%
\pgfpathlineto{\pgfqpoint{5.860800in}{1.012819in}}%
\pgfpathlineto{\pgfqpoint{5.863280in}{1.012634in}}%
\pgfpathlineto{\pgfqpoint{5.864520in}{1.014066in}}%
\pgfpathlineto{\pgfqpoint{5.865760in}{1.012473in}}%
\pgfpathlineto{\pgfqpoint{5.867000in}{1.013068in}}%
\pgfpathlineto{\pgfqpoint{5.869480in}{1.011306in}}%
\pgfpathlineto{\pgfqpoint{5.873200in}{1.014493in}}%
\pgfpathlineto{\pgfqpoint{5.874440in}{1.013571in}}%
\pgfpathlineto{\pgfqpoint{5.875680in}{1.014222in}}%
\pgfpathlineto{\pgfqpoint{5.876920in}{1.012781in}}%
\pgfpathlineto{\pgfqpoint{5.879400in}{1.013558in}}%
\pgfpathlineto{\pgfqpoint{5.883120in}{1.012244in}}%
\pgfpathlineto{\pgfqpoint{5.884360in}{1.012736in}}%
\pgfpathlineto{\pgfqpoint{5.886840in}{1.008251in}}%
\pgfpathlineto{\pgfqpoint{5.889320in}{1.010347in}}%
\pgfpathlineto{\pgfqpoint{5.890560in}{1.007940in}}%
\pgfpathlineto{\pgfqpoint{5.893040in}{1.012084in}}%
\pgfpathlineto{\pgfqpoint{5.895520in}{1.010830in}}%
\pgfpathlineto{\pgfqpoint{5.896760in}{1.012205in}}%
\pgfpathlineto{\pgfqpoint{5.898000in}{1.010220in}}%
\pgfpathlineto{\pgfqpoint{5.900480in}{1.015461in}}%
\pgfpathlineto{\pgfqpoint{5.904200in}{1.018741in}}%
\pgfpathlineto{\pgfqpoint{5.905440in}{1.018573in}}%
\pgfpathlineto{\pgfqpoint{5.910400in}{1.023414in}}%
\pgfpathlineto{\pgfqpoint{5.912880in}{1.023325in}}%
\pgfpathlineto{\pgfqpoint{5.914120in}{1.024665in}}%
\pgfpathlineto{\pgfqpoint{5.917840in}{1.018214in}}%
\pgfpathlineto{\pgfqpoint{5.920320in}{1.012997in}}%
\pgfpathlineto{\pgfqpoint{5.924040in}{1.012893in}}%
\pgfpathlineto{\pgfqpoint{5.931480in}{1.013120in}}%
\pgfpathlineto{\pgfqpoint{5.933960in}{1.011723in}}%
\pgfpathlineto{\pgfqpoint{5.941400in}{1.015202in}}%
\pgfpathlineto{\pgfqpoint{5.943880in}{1.012821in}}%
\pgfpathlineto{\pgfqpoint{5.947600in}{1.014896in}}%
\pgfpathlineto{\pgfqpoint{5.950080in}{1.011051in}}%
\pgfpathlineto{\pgfqpoint{5.951320in}{1.012834in}}%
\pgfpathlineto{\pgfqpoint{5.952560in}{1.011403in}}%
\pgfpathlineto{\pgfqpoint{5.955040in}{1.011970in}}%
\pgfpathlineto{\pgfqpoint{5.960000in}{1.010480in}}%
\pgfpathlineto{\pgfqpoint{5.961240in}{1.015450in}}%
\pgfpathlineto{\pgfqpoint{5.962480in}{1.013999in}}%
\pgfpathlineto{\pgfqpoint{5.963720in}{1.014524in}}%
\pgfpathlineto{\pgfqpoint{5.967440in}{1.018362in}}%
\pgfpathlineto{\pgfqpoint{5.973640in}{1.017930in}}%
\pgfpathlineto{\pgfqpoint{5.976120in}{1.019112in}}%
\pgfpathlineto{\pgfqpoint{5.977360in}{1.019074in}}%
\pgfpathlineto{\pgfqpoint{5.978600in}{1.017826in}}%
\pgfpathlineto{\pgfqpoint{5.981080in}{1.010350in}}%
\pgfpathlineto{\pgfqpoint{5.984800in}{1.014656in}}%
\pgfpathlineto{\pgfqpoint{5.987280in}{1.013997in}}%
\pgfpathlineto{\pgfqpoint{5.988520in}{1.015811in}}%
\pgfpathlineto{\pgfqpoint{5.989760in}{1.014621in}}%
\pgfpathlineto{\pgfqpoint{5.991000in}{1.015983in}}%
\pgfpathlineto{\pgfqpoint{5.993480in}{1.014424in}}%
\pgfpathlineto{\pgfqpoint{5.997200in}{1.017812in}}%
\pgfpathlineto{\pgfqpoint{5.998440in}{1.016611in}}%
\pgfpathlineto{\pgfqpoint{5.999680in}{1.017152in}}%
\pgfpathlineto{\pgfqpoint{6.000920in}{1.014997in}}%
\pgfpathlineto{\pgfqpoint{6.003400in}{1.015689in}}%
\pgfpathlineto{\pgfqpoint{6.005880in}{1.014332in}}%
\pgfpathlineto{\pgfqpoint{6.008360in}{1.015000in}}%
\pgfpathlineto{\pgfqpoint{6.010840in}{1.010720in}}%
\pgfpathlineto{\pgfqpoint{6.012080in}{1.011091in}}%
\pgfpathlineto{\pgfqpoint{6.013320in}{1.013356in}}%
\pgfpathlineto{\pgfqpoint{6.014560in}{1.011170in}}%
\pgfpathlineto{\pgfqpoint{6.020760in}{1.016426in}}%
\pgfpathlineto{\pgfqpoint{6.022000in}{1.013920in}}%
\pgfpathlineto{\pgfqpoint{6.024480in}{1.018298in}}%
\pgfpathlineto{\pgfqpoint{6.030680in}{1.023646in}}%
\pgfpathlineto{\pgfqpoint{6.031920in}{1.025991in}}%
\pgfpathlineto{\pgfqpoint{6.034400in}{1.024930in}}%
\pgfpathlineto{\pgfqpoint{6.036880in}{1.024519in}}%
\pgfpathlineto{\pgfqpoint{6.038120in}{1.026003in}}%
\pgfpathlineto{\pgfqpoint{6.043080in}{1.016527in}}%
\pgfpathlineto{\pgfqpoint{6.044320in}{1.013683in}}%
\pgfpathlineto{\pgfqpoint{6.055480in}{1.014879in}}%
\pgfpathlineto{\pgfqpoint{6.057960in}{1.012506in}}%
\pgfpathlineto{\pgfqpoint{6.060440in}{1.013923in}}%
\pgfpathlineto{\pgfqpoint{6.061680in}{1.015917in}}%
\pgfpathlineto{\pgfqpoint{6.064160in}{1.015329in}}%
\pgfpathlineto{\pgfqpoint{6.065400in}{1.017186in}}%
\pgfpathlineto{\pgfqpoint{6.066640in}{1.015010in}}%
\pgfpathlineto{\pgfqpoint{6.071600in}{1.015603in}}%
\pgfpathlineto{\pgfqpoint{6.074080in}{1.011578in}}%
\pgfpathlineto{\pgfqpoint{6.075320in}{1.014136in}}%
\pgfpathlineto{\pgfqpoint{6.076560in}{1.013204in}}%
\pgfpathlineto{\pgfqpoint{6.079040in}{1.014036in}}%
\pgfpathlineto{\pgfqpoint{6.082760in}{1.012360in}}%
\pgfpathlineto{\pgfqpoint{6.084000in}{1.012225in}}%
\pgfpathlineto{\pgfqpoint{6.085240in}{1.019209in}}%
\pgfpathlineto{\pgfqpoint{6.086480in}{1.017853in}}%
\pgfpathlineto{\pgfqpoint{6.092680in}{1.023675in}}%
\pgfpathlineto{\pgfqpoint{6.097640in}{1.024363in}}%
\pgfpathlineto{\pgfqpoint{6.100120in}{1.026369in}}%
\pgfpathlineto{\pgfqpoint{6.102600in}{1.024074in}}%
\pgfpathlineto{\pgfqpoint{6.105080in}{1.017198in}}%
\pgfpathlineto{\pgfqpoint{6.106320in}{1.018707in}}%
\pgfpathlineto{\pgfqpoint{6.107560in}{1.018388in}}%
\pgfpathlineto{\pgfqpoint{6.108800in}{1.019808in}}%
\pgfpathlineto{\pgfqpoint{6.111280in}{1.018611in}}%
\pgfpathlineto{\pgfqpoint{6.112520in}{1.020567in}}%
\pgfpathlineto{\pgfqpoint{6.113760in}{1.019024in}}%
\pgfpathlineto{\pgfqpoint{6.115000in}{1.020791in}}%
\pgfpathlineto{\pgfqpoint{6.117480in}{1.018581in}}%
\pgfpathlineto{\pgfqpoint{6.121200in}{1.023964in}}%
\pgfpathlineto{\pgfqpoint{6.122440in}{1.023256in}}%
\pgfpathlineto{\pgfqpoint{6.123680in}{1.024348in}}%
\pgfpathlineto{\pgfqpoint{6.126160in}{1.021810in}}%
\pgfpathlineto{\pgfqpoint{6.132360in}{1.021213in}}%
\pgfpathlineto{\pgfqpoint{6.134840in}{1.016716in}}%
\pgfpathlineto{\pgfqpoint{6.136080in}{1.017689in}}%
\pgfpathlineto{\pgfqpoint{6.137320in}{1.020418in}}%
\pgfpathlineto{\pgfqpoint{6.138560in}{1.016965in}}%
\pgfpathlineto{\pgfqpoint{6.142280in}{1.020532in}}%
\pgfpathlineto{\pgfqpoint{6.143520in}{1.020036in}}%
\pgfpathlineto{\pgfqpoint{6.144760in}{1.021178in}}%
\pgfpathlineto{\pgfqpoint{6.146000in}{1.019115in}}%
\pgfpathlineto{\pgfqpoint{6.148480in}{1.023314in}}%
\pgfpathlineto{\pgfqpoint{6.155920in}{1.031739in}}%
\pgfpathlineto{\pgfqpoint{6.158400in}{1.030049in}}%
\pgfpathlineto{\pgfqpoint{6.159640in}{1.029056in}}%
\pgfpathlineto{\pgfqpoint{6.162120in}{1.031334in}}%
\pgfpathlineto{\pgfqpoint{6.170800in}{1.018550in}}%
\pgfpathlineto{\pgfqpoint{6.175760in}{1.017381in}}%
\pgfpathlineto{\pgfqpoint{6.177000in}{1.015755in}}%
\pgfpathlineto{\pgfqpoint{6.179480in}{1.015914in}}%
\pgfpathlineto{\pgfqpoint{6.181960in}{1.014000in}}%
\pgfpathlineto{\pgfqpoint{6.184440in}{1.016305in}}%
\pgfpathlineto{\pgfqpoint{6.185680in}{1.019401in}}%
\pgfpathlineto{\pgfqpoint{6.188160in}{1.019052in}}%
\pgfpathlineto{\pgfqpoint{6.189400in}{1.020996in}}%
\pgfpathlineto{\pgfqpoint{6.190640in}{1.018664in}}%
\pgfpathlineto{\pgfqpoint{6.191880in}{1.019168in}}%
\pgfpathlineto{\pgfqpoint{6.193120in}{1.020919in}}%
\pgfpathlineto{\pgfqpoint{6.195600in}{1.020019in}}%
\pgfpathlineto{\pgfqpoint{6.198080in}{1.015144in}}%
\pgfpathlineto{\pgfqpoint{6.199320in}{1.017583in}}%
\pgfpathlineto{\pgfqpoint{6.201800in}{1.015747in}}%
\pgfpathlineto{\pgfqpoint{6.206760in}{1.015345in}}%
\pgfpathlineto{\pgfqpoint{6.208000in}{1.014025in}}%
\pgfpathlineto{\pgfqpoint{6.209240in}{1.018481in}}%
\pgfpathlineto{\pgfqpoint{6.210480in}{1.017383in}}%
\pgfpathlineto{\pgfqpoint{6.215440in}{1.023164in}}%
\pgfpathlineto{\pgfqpoint{6.221640in}{1.023306in}}%
\pgfpathlineto{\pgfqpoint{6.224120in}{1.024492in}}%
\pgfpathlineto{\pgfqpoint{6.225360in}{1.023907in}}%
\pgfpathlineto{\pgfqpoint{6.227840in}{1.018202in}}%
\pgfpathlineto{\pgfqpoint{6.229080in}{1.013828in}}%
\pgfpathlineto{\pgfqpoint{6.232800in}{1.016811in}}%
\pgfpathlineto{\pgfqpoint{6.235280in}{1.013993in}}%
\pgfpathlineto{\pgfqpoint{6.236520in}{1.016222in}}%
\pgfpathlineto{\pgfqpoint{6.237760in}{1.015800in}}%
\pgfpathlineto{\pgfqpoint{6.239000in}{1.019066in}}%
\pgfpathlineto{\pgfqpoint{6.241480in}{1.019630in}}%
\pgfpathlineto{\pgfqpoint{6.245200in}{1.026058in}}%
\pgfpathlineto{\pgfqpoint{6.246440in}{1.024834in}}%
\pgfpathlineto{\pgfqpoint{6.247680in}{1.025564in}}%
\pgfpathlineto{\pgfqpoint{6.248920in}{1.022591in}}%
\pgfpathlineto{\pgfqpoint{6.256360in}{1.028530in}}%
\pgfpathlineto{\pgfqpoint{6.258840in}{1.023680in}}%
\pgfpathlineto{\pgfqpoint{6.260080in}{1.023904in}}%
\pgfpathlineto{\pgfqpoint{6.261320in}{1.025334in}}%
\pgfpathlineto{\pgfqpoint{6.262560in}{1.021033in}}%
\pgfpathlineto{\pgfqpoint{6.266280in}{1.026072in}}%
\pgfpathlineto{\pgfqpoint{6.267520in}{1.025673in}}%
\pgfpathlineto{\pgfqpoint{6.268760in}{1.026621in}}%
\pgfpathlineto{\pgfqpoint{6.270000in}{1.023903in}}%
\pgfpathlineto{\pgfqpoint{6.272480in}{1.026966in}}%
\pgfpathlineto{\pgfqpoint{6.276200in}{1.030251in}}%
\pgfpathlineto{\pgfqpoint{6.277440in}{1.029492in}}%
\pgfpathlineto{\pgfqpoint{6.279920in}{1.033008in}}%
\pgfpathlineto{\pgfqpoint{6.283640in}{1.029306in}}%
\pgfpathlineto{\pgfqpoint{6.284880in}{1.029285in}}%
\pgfpathlineto{\pgfqpoint{6.286120in}{1.030661in}}%
\pgfpathlineto{\pgfqpoint{6.287360in}{1.028591in}}%
\pgfpathlineto{\pgfqpoint{6.289840in}{1.022778in}}%
\pgfpathlineto{\pgfqpoint{6.292320in}{1.018336in}}%
\pgfpathlineto{\pgfqpoint{6.296040in}{1.018799in}}%
\pgfpathlineto{\pgfqpoint{6.298520in}{1.018267in}}%
\pgfpathlineto{\pgfqpoint{6.307200in}{1.015433in}}%
\pgfpathlineto{\pgfqpoint{6.308440in}{1.016742in}}%
\pgfpathlineto{\pgfqpoint{6.309680in}{1.020367in}}%
\pgfpathlineto{\pgfqpoint{6.312160in}{1.020880in}}%
\pgfpathlineto{\pgfqpoint{6.313400in}{1.022935in}}%
\pgfpathlineto{\pgfqpoint{6.314640in}{1.021120in}}%
\pgfpathlineto{\pgfqpoint{6.319600in}{1.023116in}}%
\pgfpathlineto{\pgfqpoint{6.322080in}{1.020116in}}%
\pgfpathlineto{\pgfqpoint{6.323320in}{1.021986in}}%
\pgfpathlineto{\pgfqpoint{6.325800in}{1.019901in}}%
\pgfpathlineto{\pgfqpoint{6.329520in}{1.019320in}}%
\pgfpathlineto{\pgfqpoint{6.333240in}{1.017028in}}%
\pgfpathlineto{\pgfqpoint{6.334480in}{1.016651in}}%
\pgfpathlineto{\pgfqpoint{6.336960in}{1.020599in}}%
\pgfpathlineto{\pgfqpoint{6.341920in}{1.023198in}}%
\pgfpathlineto{\pgfqpoint{6.343160in}{1.021155in}}%
\pgfpathlineto{\pgfqpoint{6.348120in}{1.025235in}}%
\pgfpathlineto{\pgfqpoint{6.349360in}{1.024148in}}%
\pgfpathlineto{\pgfqpoint{6.353080in}{1.013345in}}%
\pgfpathlineto{\pgfqpoint{6.354320in}{1.015315in}}%
\pgfpathlineto{\pgfqpoint{6.355560in}{1.014244in}}%
\pgfpathlineto{\pgfqpoint{6.356800in}{1.015381in}}%
\pgfpathlineto{\pgfqpoint{6.359280in}{1.011954in}}%
\pgfpathlineto{\pgfqpoint{6.364240in}{1.019275in}}%
\pgfpathlineto{\pgfqpoint{6.366720in}{1.020842in}}%
\pgfpathlineto{\pgfqpoint{6.367960in}{1.022302in}}%
\pgfpathlineto{\pgfqpoint{6.369200in}{1.025518in}}%
\pgfpathlineto{\pgfqpoint{6.370440in}{1.024710in}}%
\pgfpathlineto{\pgfqpoint{6.371680in}{1.027180in}}%
\pgfpathlineto{\pgfqpoint{6.372920in}{1.024265in}}%
\pgfpathlineto{\pgfqpoint{6.380360in}{1.030067in}}%
\pgfpathlineto{\pgfqpoint{6.382840in}{1.026260in}}%
\pgfpathlineto{\pgfqpoint{6.385320in}{1.027364in}}%
\pgfpathlineto{\pgfqpoint{6.386560in}{1.023571in}}%
\pgfpathlineto{\pgfqpoint{6.387800in}{1.024198in}}%
\pgfpathlineto{\pgfqpoint{6.389040in}{1.027322in}}%
\pgfpathlineto{\pgfqpoint{6.390280in}{1.027249in}}%
\pgfpathlineto{\pgfqpoint{6.391520in}{1.025761in}}%
\pgfpathlineto{\pgfqpoint{6.392760in}{1.026401in}}%
\pgfpathlineto{\pgfqpoint{6.394000in}{1.023697in}}%
\pgfpathlineto{\pgfqpoint{6.396480in}{1.026973in}}%
\pgfpathlineto{\pgfqpoint{6.397720in}{1.026679in}}%
\pgfpathlineto{\pgfqpoint{6.400200in}{1.029656in}}%
\pgfpathlineto{\pgfqpoint{6.401440in}{1.028597in}}%
\pgfpathlineto{\pgfqpoint{6.403920in}{1.031940in}}%
\pgfpathlineto{\pgfqpoint{6.406400in}{1.030302in}}%
\pgfpathlineto{\pgfqpoint{6.410120in}{1.030955in}}%
\pgfpathlineto{\pgfqpoint{6.417560in}{1.021221in}}%
\pgfpathlineto{\pgfqpoint{6.425000in}{1.016763in}}%
\pgfpathlineto{\pgfqpoint{6.426240in}{1.018154in}}%
\pgfpathlineto{\pgfqpoint{6.427480in}{1.017862in}}%
\pgfpathlineto{\pgfqpoint{6.429960in}{1.014785in}}%
\pgfpathlineto{\pgfqpoint{6.432440in}{1.015053in}}%
\pgfpathlineto{\pgfqpoint{6.433680in}{1.017898in}}%
\pgfpathlineto{\pgfqpoint{6.434920in}{1.017072in}}%
\pgfpathlineto{\pgfqpoint{6.437400in}{1.019130in}}%
\pgfpathlineto{\pgfqpoint{6.439880in}{1.016681in}}%
\pgfpathlineto{\pgfqpoint{6.442360in}{1.016490in}}%
\pgfpathlineto{\pgfqpoint{6.444840in}{1.015463in}}%
\pgfpathlineto{\pgfqpoint{6.446080in}{1.014398in}}%
\pgfpathlineto{\pgfqpoint{6.447320in}{1.015081in}}%
\pgfpathlineto{\pgfqpoint{6.449800in}{1.011853in}}%
\pgfpathlineto{\pgfqpoint{6.452280in}{1.013509in}}%
\pgfpathlineto{\pgfqpoint{6.454760in}{1.012463in}}%
\pgfpathlineto{\pgfqpoint{6.456000in}{1.010176in}}%
\pgfpathlineto{\pgfqpoint{6.457240in}{1.015316in}}%
\pgfpathlineto{\pgfqpoint{6.458480in}{1.014732in}}%
\pgfpathlineto{\pgfqpoint{6.460960in}{1.018422in}}%
\pgfpathlineto{\pgfqpoint{6.464680in}{1.018719in}}%
\pgfpathlineto{\pgfqpoint{6.465920in}{1.020040in}}%
\pgfpathlineto{\pgfqpoint{6.467160in}{1.018061in}}%
\pgfpathlineto{\pgfqpoint{6.468400in}{1.019841in}}%
\pgfpathlineto{\pgfqpoint{6.469640in}{1.018948in}}%
\pgfpathlineto{\pgfqpoint{6.472120in}{1.020407in}}%
\pgfpathlineto{\pgfqpoint{6.474600in}{1.017793in}}%
\pgfpathlineto{\pgfqpoint{6.477080in}{1.011842in}}%
\pgfpathlineto{\pgfqpoint{6.480800in}{1.017193in}}%
\pgfpathlineto{\pgfqpoint{6.483280in}{1.012341in}}%
\pgfpathlineto{\pgfqpoint{6.488240in}{1.020645in}}%
\pgfpathlineto{\pgfqpoint{6.489480in}{1.020645in}}%
\pgfpathlineto{\pgfqpoint{6.491960in}{1.024299in}}%
\pgfpathlineto{\pgfqpoint{6.494440in}{1.028191in}}%
\pgfpathlineto{\pgfqpoint{6.495680in}{1.030620in}}%
\pgfpathlineto{\pgfqpoint{6.496920in}{1.028538in}}%
\pgfpathlineto{\pgfqpoint{6.499400in}{1.030324in}}%
\pgfpathlineto{\pgfqpoint{6.501880in}{1.030946in}}%
\pgfpathlineto{\pgfqpoint{6.504360in}{1.032229in}}%
\pgfpathlineto{\pgfqpoint{6.506840in}{1.027346in}}%
\pgfpathlineto{\pgfqpoint{6.509320in}{1.028407in}}%
\pgfpathlineto{\pgfqpoint{6.510560in}{1.023362in}}%
\pgfpathlineto{\pgfqpoint{6.511800in}{1.023784in}}%
\pgfpathlineto{\pgfqpoint{6.513040in}{1.027259in}}%
\pgfpathlineto{\pgfqpoint{6.515520in}{1.025582in}}%
\pgfpathlineto{\pgfqpoint{6.516760in}{1.026767in}}%
\pgfpathlineto{\pgfqpoint{6.518000in}{1.024504in}}%
\pgfpathlineto{\pgfqpoint{6.520480in}{1.028919in}}%
\pgfpathlineto{\pgfqpoint{6.522960in}{1.030773in}}%
\pgfpathlineto{\pgfqpoint{6.524200in}{1.033177in}}%
\pgfpathlineto{\pgfqpoint{6.525440in}{1.031410in}}%
\pgfpathlineto{\pgfqpoint{6.527920in}{1.034769in}}%
\pgfpathlineto{\pgfqpoint{6.529160in}{1.034261in}}%
\pgfpathlineto{\pgfqpoint{6.530400in}{1.035771in}}%
\pgfpathlineto{\pgfqpoint{6.532880in}{1.033106in}}%
\pgfpathlineto{\pgfqpoint{6.534120in}{1.033990in}}%
\pgfpathlineto{\pgfqpoint{6.545280in}{1.023616in}}%
\pgfpathlineto{\pgfqpoint{6.546520in}{1.024791in}}%
\pgfpathlineto{\pgfqpoint{6.549000in}{1.021200in}}%
\pgfpathlineto{\pgfqpoint{6.550240in}{1.022635in}}%
\pgfpathlineto{\pgfqpoint{6.551480in}{1.021888in}}%
\pgfpathlineto{\pgfqpoint{6.552720in}{1.019454in}}%
\pgfpathlineto{\pgfqpoint{6.556440in}{1.020413in}}%
\pgfpathlineto{\pgfqpoint{6.557680in}{1.022806in}}%
\pgfpathlineto{\pgfqpoint{6.558920in}{1.022189in}}%
\pgfpathlineto{\pgfqpoint{6.560160in}{1.022948in}}%
\pgfpathlineto{\pgfqpoint{6.561400in}{1.025064in}}%
\pgfpathlineto{\pgfqpoint{6.562640in}{1.022269in}}%
\pgfpathlineto{\pgfqpoint{6.565120in}{1.022767in}}%
\pgfpathlineto{\pgfqpoint{6.566360in}{1.022679in}}%
\pgfpathlineto{\pgfqpoint{6.570080in}{1.019347in}}%
\pgfpathlineto{\pgfqpoint{6.571320in}{1.019543in}}%
\pgfpathlineto{\pgfqpoint{6.573800in}{1.016878in}}%
\pgfpathlineto{\pgfqpoint{6.576280in}{1.017450in}}%
\pgfpathlineto{\pgfqpoint{6.578760in}{1.014183in}}%
\pgfpathlineto{\pgfqpoint{6.580000in}{1.010443in}}%
\pgfpathlineto{\pgfqpoint{6.581240in}{1.018497in}}%
\pgfpathlineto{\pgfqpoint{6.582480in}{1.018110in}}%
\pgfpathlineto{\pgfqpoint{6.584960in}{1.023432in}}%
\pgfpathlineto{\pgfqpoint{6.588680in}{1.021915in}}%
\pgfpathlineto{\pgfqpoint{6.589920in}{1.022950in}}%
\pgfpathlineto{\pgfqpoint{6.591160in}{1.018673in}}%
\pgfpathlineto{\pgfqpoint{6.594880in}{1.022055in}}%
\pgfpathlineto{\pgfqpoint{6.596120in}{1.024648in}}%
\pgfpathlineto{\pgfqpoint{6.598600in}{1.020890in}}%
\pgfpathlineto{\pgfqpoint{6.601080in}{1.012746in}}%
\pgfpathlineto{\pgfqpoint{6.604800in}{1.018970in}}%
\pgfpathlineto{\pgfqpoint{6.607280in}{1.014657in}}%
\pgfpathlineto{\pgfqpoint{6.613480in}{1.026546in}}%
\pgfpathlineto{\pgfqpoint{6.615960in}{1.029342in}}%
\pgfpathlineto{\pgfqpoint{6.617200in}{1.032629in}}%
\pgfpathlineto{\pgfqpoint{6.618440in}{1.032534in}}%
\pgfpathlineto{\pgfqpoint{6.619680in}{1.036030in}}%
\pgfpathlineto{\pgfqpoint{6.620920in}{1.032421in}}%
\pgfpathlineto{\pgfqpoint{6.623400in}{1.036245in}}%
\pgfpathlineto{\pgfqpoint{6.627120in}{1.036744in}}%
\pgfpathlineto{\pgfqpoint{6.628360in}{1.038780in}}%
\pgfpathlineto{\pgfqpoint{6.630840in}{1.036126in}}%
\pgfpathlineto{\pgfqpoint{6.633320in}{1.038034in}}%
\pgfpathlineto{\pgfqpoint{6.635800in}{1.032828in}}%
\pgfpathlineto{\pgfqpoint{6.637040in}{1.037172in}}%
\pgfpathlineto{\pgfqpoint{6.639520in}{1.034612in}}%
\pgfpathlineto{\pgfqpoint{6.640760in}{1.035802in}}%
\pgfpathlineto{\pgfqpoint{6.642000in}{1.032564in}}%
\pgfpathlineto{\pgfqpoint{6.645720in}{1.039330in}}%
\pgfpathlineto{\pgfqpoint{6.648200in}{1.042494in}}%
\pgfpathlineto{\pgfqpoint{6.649440in}{1.039980in}}%
\pgfpathlineto{\pgfqpoint{6.651920in}{1.044874in}}%
\pgfpathlineto{\pgfqpoint{6.653160in}{1.043974in}}%
\pgfpathlineto{\pgfqpoint{6.654400in}{1.046211in}}%
\pgfpathlineto{\pgfqpoint{6.660600in}{1.041448in}}%
\pgfpathlineto{\pgfqpoint{6.661840in}{1.043131in}}%
\pgfpathlineto{\pgfqpoint{6.665560in}{1.035496in}}%
\pgfpathlineto{\pgfqpoint{6.671760in}{1.029839in}}%
\pgfpathlineto{\pgfqpoint{6.673000in}{1.027230in}}%
\pgfpathlineto{\pgfqpoint{6.675480in}{1.028409in}}%
\pgfpathlineto{\pgfqpoint{6.677960in}{1.025061in}}%
\pgfpathlineto{\pgfqpoint{6.679200in}{1.023707in}}%
\pgfpathlineto{\pgfqpoint{6.682920in}{1.025469in}}%
\pgfpathlineto{\pgfqpoint{6.685400in}{1.031793in}}%
\pgfpathlineto{\pgfqpoint{6.686640in}{1.028804in}}%
\pgfpathlineto{\pgfqpoint{6.687880in}{1.028786in}}%
\pgfpathlineto{\pgfqpoint{6.691600in}{1.022401in}}%
\pgfpathlineto{\pgfqpoint{6.695320in}{1.020205in}}%
\pgfpathlineto{\pgfqpoint{6.697800in}{1.017021in}}%
\pgfpathlineto{\pgfqpoint{6.700280in}{1.017720in}}%
\pgfpathlineto{\pgfqpoint{6.701520in}{1.015757in}}%
\pgfpathlineto{\pgfqpoint{6.702760in}{1.016127in}}%
\pgfpathlineto{\pgfqpoint{6.704000in}{1.013401in}}%
\pgfpathlineto{\pgfqpoint{6.706480in}{1.022853in}}%
\pgfpathlineto{\pgfqpoint{6.708960in}{1.027985in}}%
\pgfpathlineto{\pgfqpoint{6.710200in}{1.025648in}}%
\pgfpathlineto{\pgfqpoint{6.711440in}{1.027731in}}%
\pgfpathlineto{\pgfqpoint{6.713920in}{1.027215in}}%
\pgfpathlineto{\pgfqpoint{6.715160in}{1.023281in}}%
\pgfpathlineto{\pgfqpoint{6.716400in}{1.024018in}}%
\pgfpathlineto{\pgfqpoint{6.718880in}{1.023273in}}%
\pgfpathlineto{\pgfqpoint{6.720120in}{1.025819in}}%
\pgfpathlineto{\pgfqpoint{6.721360in}{1.024502in}}%
\pgfpathlineto{\pgfqpoint{6.725080in}{1.015722in}}%
\pgfpathlineto{\pgfqpoint{6.727560in}{1.019677in}}%
\pgfpathlineto{\pgfqpoint{6.728800in}{1.021720in}}%
\pgfpathlineto{\pgfqpoint{6.731280in}{1.016665in}}%
\pgfpathlineto{\pgfqpoint{6.736240in}{1.026629in}}%
\pgfpathlineto{\pgfqpoint{6.737480in}{1.027592in}}%
\pgfpathlineto{\pgfqpoint{6.741200in}{1.036726in}}%
\pgfpathlineto{\pgfqpoint{6.742440in}{1.035231in}}%
\pgfpathlineto{\pgfqpoint{6.743680in}{1.038284in}}%
\pgfpathlineto{\pgfqpoint{6.744920in}{1.032505in}}%
\pgfpathlineto{\pgfqpoint{6.749880in}{1.037910in}}%
\pgfpathlineto{\pgfqpoint{6.754840in}{1.034614in}}%
\pgfpathlineto{\pgfqpoint{6.757320in}{1.037973in}}%
\pgfpathlineto{\pgfqpoint{6.759800in}{1.031796in}}%
\pgfpathlineto{\pgfqpoint{6.761040in}{1.036770in}}%
\pgfpathlineto{\pgfqpoint{6.763520in}{1.033710in}}%
\pgfpathlineto{\pgfqpoint{6.764760in}{1.034512in}}%
\pgfpathlineto{\pgfqpoint{6.766000in}{1.032423in}}%
\pgfpathlineto{\pgfqpoint{6.769720in}{1.040410in}}%
\pgfpathlineto{\pgfqpoint{6.772200in}{1.042544in}}%
\pgfpathlineto{\pgfqpoint{6.773440in}{1.039047in}}%
\pgfpathlineto{\pgfqpoint{6.778400in}{1.049385in}}%
\pgfpathlineto{\pgfqpoint{6.784600in}{1.039136in}}%
\pgfpathlineto{\pgfqpoint{6.785840in}{1.040799in}}%
\pgfpathlineto{\pgfqpoint{6.788320in}{1.034310in}}%
\pgfpathlineto{\pgfqpoint{6.790800in}{1.032418in}}%
\pgfpathlineto{\pgfqpoint{6.792040in}{1.033013in}}%
\pgfpathlineto{\pgfqpoint{6.794520in}{1.035617in}}%
\pgfpathlineto{\pgfqpoint{6.798240in}{1.030142in}}%
\pgfpathlineto{\pgfqpoint{6.799480in}{1.029626in}}%
\pgfpathlineto{\pgfqpoint{6.800720in}{1.026288in}}%
\pgfpathlineto{\pgfqpoint{6.801960in}{1.026879in}}%
\pgfpathlineto{\pgfqpoint{6.803200in}{1.024566in}}%
\pgfpathlineto{\pgfqpoint{6.804440in}{1.026561in}}%
\pgfpathlineto{\pgfqpoint{6.805680in}{1.025642in}}%
\pgfpathlineto{\pgfqpoint{6.806920in}{1.026474in}}%
\pgfpathlineto{\pgfqpoint{6.809400in}{1.034308in}}%
\pgfpathlineto{\pgfqpoint{6.811880in}{1.031504in}}%
\pgfpathlineto{\pgfqpoint{6.814360in}{1.029884in}}%
\pgfpathlineto{\pgfqpoint{6.816840in}{1.026903in}}%
\pgfpathlineto{\pgfqpoint{6.818080in}{1.025208in}}%
\pgfpathlineto{\pgfqpoint{6.819320in}{1.025485in}}%
\pgfpathlineto{\pgfqpoint{6.821800in}{1.019529in}}%
\pgfpathlineto{\pgfqpoint{6.824280in}{1.016724in}}%
\pgfpathlineto{\pgfqpoint{6.825520in}{1.014600in}}%
\pgfpathlineto{\pgfqpoint{6.826760in}{1.016599in}}%
\pgfpathlineto{\pgfqpoint{6.828000in}{1.014156in}}%
\pgfpathlineto{\pgfqpoint{6.832960in}{1.026355in}}%
\pgfpathlineto{\pgfqpoint{6.834200in}{1.024413in}}%
\pgfpathlineto{\pgfqpoint{6.836680in}{1.030851in}}%
\pgfpathlineto{\pgfqpoint{6.837920in}{1.032012in}}%
\pgfpathlineto{\pgfqpoint{6.839160in}{1.029008in}}%
\pgfpathlineto{\pgfqpoint{6.844120in}{1.032638in}}%
\pgfpathlineto{\pgfqpoint{6.847840in}{1.023374in}}%
\pgfpathlineto{\pgfqpoint{6.849080in}{1.019405in}}%
\pgfpathlineto{\pgfqpoint{6.851560in}{1.023428in}}%
\pgfpathlineto{\pgfqpoint{6.852800in}{1.024814in}}%
\pgfpathlineto{\pgfqpoint{6.855280in}{1.020288in}}%
\pgfpathlineto{\pgfqpoint{6.856520in}{1.024588in}}%
\pgfpathlineto{\pgfqpoint{6.857760in}{1.024141in}}%
\pgfpathlineto{\pgfqpoint{6.862720in}{1.037823in}}%
\pgfpathlineto{\pgfqpoint{6.865200in}{1.042681in}}%
\pgfpathlineto{\pgfqpoint{6.866440in}{1.039441in}}%
\pgfpathlineto{\pgfqpoint{6.867680in}{1.044409in}}%
\pgfpathlineto{\pgfqpoint{6.868920in}{1.039062in}}%
\pgfpathlineto{\pgfqpoint{6.870160in}{1.040604in}}%
\pgfpathlineto{\pgfqpoint{6.873880in}{1.052482in}}%
\pgfpathlineto{\pgfqpoint{6.875120in}{1.051563in}}%
\pgfpathlineto{\pgfqpoint{6.876360in}{1.053061in}}%
\pgfpathlineto{\pgfqpoint{6.878840in}{1.043584in}}%
\pgfpathlineto{\pgfqpoint{6.881320in}{1.049227in}}%
\pgfpathlineto{\pgfqpoint{6.883800in}{1.044428in}}%
\pgfpathlineto{\pgfqpoint{6.885040in}{1.048556in}}%
\pgfpathlineto{\pgfqpoint{6.887520in}{1.045115in}}%
\pgfpathlineto{\pgfqpoint{6.890000in}{1.049986in}}%
\pgfpathlineto{\pgfqpoint{6.893720in}{1.055011in}}%
\pgfpathlineto{\pgfqpoint{6.894960in}{1.054651in}}%
\pgfpathlineto{\pgfqpoint{6.896200in}{1.056204in}}%
\pgfpathlineto{\pgfqpoint{6.897440in}{1.053431in}}%
\pgfpathlineto{\pgfqpoint{6.901160in}{1.065935in}}%
\pgfpathlineto{\pgfqpoint{6.902400in}{1.065290in}}%
\pgfpathlineto{\pgfqpoint{6.907360in}{1.054318in}}%
\pgfpathlineto{\pgfqpoint{6.909840in}{1.053739in}}%
\pgfpathlineto{\pgfqpoint{6.913560in}{1.042794in}}%
\pgfpathlineto{\pgfqpoint{6.916040in}{1.040375in}}%
\pgfpathlineto{\pgfqpoint{6.918520in}{1.047153in}}%
\pgfpathlineto{\pgfqpoint{6.919760in}{1.046996in}}%
\pgfpathlineto{\pgfqpoint{6.921000in}{1.044222in}}%
\pgfpathlineto{\pgfqpoint{6.922240in}{1.045273in}}%
\pgfpathlineto{\pgfqpoint{6.923480in}{1.044787in}}%
\pgfpathlineto{\pgfqpoint{6.927200in}{1.038330in}}%
\pgfpathlineto{\pgfqpoint{6.929680in}{1.040144in}}%
\pgfpathlineto{\pgfqpoint{6.930920in}{1.040508in}}%
\pgfpathlineto{\pgfqpoint{6.933400in}{1.050282in}}%
\pgfpathlineto{\pgfqpoint{6.939600in}{1.040521in}}%
\pgfpathlineto{\pgfqpoint{6.940840in}{1.041468in}}%
\pgfpathlineto{\pgfqpoint{6.942080in}{1.040707in}}%
\pgfpathlineto{\pgfqpoint{6.943320in}{1.041272in}}%
\pgfpathlineto{\pgfqpoint{6.945800in}{1.035362in}}%
\pgfpathlineto{\pgfqpoint{6.948280in}{1.032245in}}%
\pgfpathlineto{\pgfqpoint{6.949520in}{1.029498in}}%
\pgfpathlineto{\pgfqpoint{6.950760in}{1.032532in}}%
\pgfpathlineto{\pgfqpoint{6.952000in}{1.029907in}}%
\pgfpathlineto{\pgfqpoint{6.954480in}{1.033220in}}%
\pgfpathlineto{\pgfqpoint{6.955720in}{1.038357in}}%
\pgfpathlineto{\pgfqpoint{6.958200in}{1.032276in}}%
\pgfpathlineto{\pgfqpoint{6.961920in}{1.041120in}}%
\pgfpathlineto{\pgfqpoint{6.963160in}{1.036929in}}%
\pgfpathlineto{\pgfqpoint{6.965640in}{1.039394in}}%
\pgfpathlineto{\pgfqpoint{6.968120in}{1.041477in}}%
\pgfpathlineto{\pgfqpoint{6.969360in}{1.039750in}}%
\pgfpathlineto{\pgfqpoint{6.973080in}{1.025080in}}%
\pgfpathlineto{\pgfqpoint{6.974320in}{1.027035in}}%
\pgfpathlineto{\pgfqpoint{6.976800in}{1.032658in}}%
\pgfpathlineto{\pgfqpoint{6.978040in}{1.026486in}}%
\pgfpathlineto{\pgfqpoint{6.979280in}{1.026454in}}%
\pgfpathlineto{\pgfqpoint{6.986720in}{1.055189in}}%
\pgfpathlineto{\pgfqpoint{6.987960in}{1.053207in}}%
\pgfpathlineto{\pgfqpoint{6.989200in}{1.055843in}}%
\pgfpathlineto{\pgfqpoint{6.990440in}{1.054766in}}%
\pgfpathlineto{\pgfqpoint{6.991680in}{1.061591in}}%
\pgfpathlineto{\pgfqpoint{6.994160in}{1.050513in}}%
\pgfpathlineto{\pgfqpoint{6.997880in}{1.059920in}}%
\pgfpathlineto{\pgfqpoint{6.999120in}{1.060294in}}%
\pgfpathlineto{\pgfqpoint{7.000360in}{1.065853in}}%
\pgfpathlineto{\pgfqpoint{7.002840in}{1.051668in}}%
\pgfpathlineto{\pgfqpoint{7.005320in}{1.057685in}}%
\pgfpathlineto{\pgfqpoint{7.006560in}{1.050506in}}%
\pgfpathlineto{\pgfqpoint{7.007800in}{1.051035in}}%
\pgfpathlineto{\pgfqpoint{7.009040in}{1.057859in}}%
\pgfpathlineto{\pgfqpoint{7.011520in}{1.049549in}}%
\pgfpathlineto{\pgfqpoint{7.012760in}{1.051360in}}%
\pgfpathlineto{\pgfqpoint{7.014000in}{1.048771in}}%
\pgfpathlineto{\pgfqpoint{7.016480in}{1.052430in}}%
\pgfpathlineto{\pgfqpoint{7.017720in}{1.058765in}}%
\pgfpathlineto{\pgfqpoint{7.020200in}{1.060332in}}%
\pgfpathlineto{\pgfqpoint{7.021440in}{1.057241in}}%
\pgfpathlineto{\pgfqpoint{7.026400in}{1.074266in}}%
\pgfpathlineto{\pgfqpoint{7.027640in}{1.069407in}}%
\pgfpathlineto{\pgfqpoint{7.028880in}{1.057526in}}%
\pgfpathlineto{\pgfqpoint{7.031360in}{1.056809in}}%
\pgfpathlineto{\pgfqpoint{7.032600in}{1.055408in}}%
\pgfpathlineto{\pgfqpoint{7.033840in}{1.057404in}}%
\pgfpathlineto{\pgfqpoint{7.036320in}{1.048505in}}%
\pgfpathlineto{\pgfqpoint{7.037560in}{1.048722in}}%
\pgfpathlineto{\pgfqpoint{7.041280in}{1.052817in}}%
\pgfpathlineto{\pgfqpoint{7.042520in}{1.052745in}}%
\pgfpathlineto{\pgfqpoint{7.043760in}{1.053900in}}%
\pgfpathlineto{\pgfqpoint{7.045000in}{1.053289in}}%
\pgfpathlineto{\pgfqpoint{7.051200in}{1.035148in}}%
\pgfpathlineto{\pgfqpoint{7.052440in}{1.035738in}}%
\pgfpathlineto{\pgfqpoint{7.054920in}{1.034339in}}%
\pgfpathlineto{\pgfqpoint{7.056160in}{1.037534in}}%
\pgfpathlineto{\pgfqpoint{7.057400in}{1.044474in}}%
\pgfpathlineto{\pgfqpoint{7.058640in}{1.043712in}}%
\pgfpathlineto{\pgfqpoint{7.059880in}{1.045313in}}%
\pgfpathlineto{\pgfqpoint{7.061120in}{1.044526in}}%
\pgfpathlineto{\pgfqpoint{7.062360in}{1.042269in}}%
\pgfpathlineto{\pgfqpoint{7.064840in}{1.043517in}}%
\pgfpathlineto{\pgfqpoint{7.066080in}{1.042856in}}%
\pgfpathlineto{\pgfqpoint{7.067320in}{1.047402in}}%
\pgfpathlineto{\pgfqpoint{7.068560in}{1.044146in}}%
\pgfpathlineto{\pgfqpoint{7.069800in}{1.045319in}}%
\pgfpathlineto{\pgfqpoint{7.071040in}{1.048110in}}%
\pgfpathlineto{\pgfqpoint{7.073520in}{1.041743in}}%
\pgfpathlineto{\pgfqpoint{7.074760in}{1.044578in}}%
\pgfpathlineto{\pgfqpoint{7.076000in}{1.043115in}}%
\pgfpathlineto{\pgfqpoint{7.077240in}{1.057124in}}%
\pgfpathlineto{\pgfqpoint{7.078480in}{1.058239in}}%
\pgfpathlineto{\pgfqpoint{7.079720in}{1.062943in}}%
\pgfpathlineto{\pgfqpoint{7.080960in}{1.061591in}}%
\pgfpathlineto{\pgfqpoint{7.082200in}{1.057177in}}%
\pgfpathlineto{\pgfqpoint{7.083440in}{1.059342in}}%
\pgfpathlineto{\pgfqpoint{7.084680in}{1.067665in}}%
\pgfpathlineto{\pgfqpoint{7.087160in}{1.058426in}}%
\pgfpathlineto{\pgfqpoint{7.089640in}{1.053936in}}%
\pgfpathlineto{\pgfqpoint{7.090880in}{1.051122in}}%
\pgfpathlineto{\pgfqpoint{7.093360in}{1.043653in}}%
\pgfpathlineto{\pgfqpoint{7.097080in}{1.031883in}}%
\pgfpathlineto{\pgfqpoint{7.098320in}{1.031248in}}%
\pgfpathlineto{\pgfqpoint{7.100800in}{1.040939in}}%
\pgfpathlineto{\pgfqpoint{7.103280in}{1.031674in}}%
\pgfpathlineto{\pgfqpoint{7.107000in}{1.044429in}}%
\pgfpathlineto{\pgfqpoint{7.110720in}{1.074224in}}%
\pgfpathlineto{\pgfqpoint{7.113200in}{1.065184in}}%
\pgfpathlineto{\pgfqpoint{7.114440in}{1.065912in}}%
\pgfpathlineto{\pgfqpoint{7.115680in}{1.071902in}}%
\pgfpathlineto{\pgfqpoint{7.118160in}{1.060960in}}%
\pgfpathlineto{\pgfqpoint{7.119400in}{1.066750in}}%
\pgfpathlineto{\pgfqpoint{7.120640in}{1.067153in}}%
\pgfpathlineto{\pgfqpoint{7.121880in}{1.070977in}}%
\pgfpathlineto{\pgfqpoint{7.123120in}{1.071108in}}%
\pgfpathlineto{\pgfqpoint{7.124360in}{1.076208in}}%
\pgfpathlineto{\pgfqpoint{7.126840in}{1.062455in}}%
\pgfpathlineto{\pgfqpoint{7.129320in}{1.076694in}}%
\pgfpathlineto{\pgfqpoint{7.130560in}{1.072150in}}%
\pgfpathlineto{\pgfqpoint{7.131800in}{1.074511in}}%
\pgfpathlineto{\pgfqpoint{7.133040in}{1.082765in}}%
\pgfpathlineto{\pgfqpoint{7.134280in}{1.081367in}}%
\pgfpathlineto{\pgfqpoint{7.136760in}{1.075483in}}%
\pgfpathlineto{\pgfqpoint{7.139240in}{1.080107in}}%
\pgfpathlineto{\pgfqpoint{7.140480in}{1.081099in}}%
\pgfpathlineto{\pgfqpoint{7.142960in}{1.092913in}}%
\pgfpathlineto{\pgfqpoint{7.144200in}{1.089595in}}%
\pgfpathlineto{\pgfqpoint{7.146680in}{1.078876in}}%
\pgfpathlineto{\pgfqpoint{7.150400in}{1.092706in}}%
\pgfpathlineto{\pgfqpoint{7.152880in}{1.075207in}}%
\pgfpathlineto{\pgfqpoint{7.154120in}{1.082717in}}%
\pgfpathlineto{\pgfqpoint{7.156600in}{1.078158in}}%
\pgfpathlineto{\pgfqpoint{7.157840in}{1.082637in}}%
\pgfpathlineto{\pgfqpoint{7.165280in}{1.062349in}}%
\pgfpathlineto{\pgfqpoint{7.167760in}{1.073980in}}%
\pgfpathlineto{\pgfqpoint{7.170240in}{1.071990in}}%
\pgfpathlineto{\pgfqpoint{7.172720in}{1.059627in}}%
\pgfpathlineto{\pgfqpoint{7.173960in}{1.062563in}}%
\pgfpathlineto{\pgfqpoint{7.176440in}{1.051674in}}%
\pgfpathlineto{\pgfqpoint{7.177680in}{1.050060in}}%
\pgfpathlineto{\pgfqpoint{7.180160in}{1.055342in}}%
\pgfpathlineto{\pgfqpoint{7.181400in}{1.065996in}}%
\pgfpathlineto{\pgfqpoint{7.182640in}{1.064541in}}%
\pgfpathlineto{\pgfqpoint{7.185120in}{1.068369in}}%
\pgfpathlineto{\pgfqpoint{7.186360in}{1.064758in}}%
\pgfpathlineto{\pgfqpoint{7.187600in}{1.065228in}}%
\pgfpathlineto{\pgfqpoint{7.190080in}{1.068180in}}%
\pgfpathlineto{\pgfqpoint{7.191320in}{1.070234in}}%
\pgfpathlineto{\pgfqpoint{7.193800in}{1.063590in}}%
\pgfpathlineto{\pgfqpoint{7.195040in}{1.063829in}}%
\pgfpathlineto{\pgfqpoint{7.196280in}{1.060047in}}%
\pgfpathlineto{\pgfqpoint{7.197520in}{1.061122in}}%
\pgfpathlineto{\pgfqpoint{7.200000in}{1.067147in}}%
\pgfpathlineto{\pgfqpoint{7.200000in}{1.067147in}}%
\pgfusepath{stroke}%
\end{pgfscope}%
\begin{pgfscope}%
\pgfpathrectangle{\pgfqpoint{1.000000in}{0.300000in}}{\pgfqpoint{6.200000in}{2.400000in}} %
\pgfusepath{clip}%
\pgfsetrectcap%
\pgfsetroundjoin%
\pgfsetlinewidth{2.007500pt}%
\definecolor{currentstroke}{rgb}{0.000000,0.000000,1.000000}%
\pgfsetstrokecolor{currentstroke}%
\pgfsetdash{}{0pt}%
\pgfpathmoveto{\pgfqpoint{2.241240in}{1.086016in}}%
\pgfpathlineto{\pgfqpoint{6.578760in}{1.086016in}}%
\pgfusepath{stroke}%
\end{pgfscope}%
\begin{pgfscope}%
\pgfpathrectangle{\pgfqpoint{1.000000in}{0.300000in}}{\pgfqpoint{6.200000in}{2.400000in}} %
\pgfusepath{clip}%
\pgfsetrectcap%
\pgfsetroundjoin%
\pgfsetlinewidth{1.003750pt}%
\definecolor{currentstroke}{rgb}{0.000000,0.500000,0.000000}%
\pgfsetstrokecolor{currentstroke}%
\pgfsetdash{}{0pt}%
\pgfpathmoveto{\pgfqpoint{1.001240in}{1.337138in}}%
\pgfpathlineto{\pgfqpoint{1.002480in}{1.729107in}}%
\pgfpathlineto{\pgfqpoint{1.003720in}{1.771831in}}%
\pgfpathlineto{\pgfqpoint{1.011160in}{1.152010in}}%
\pgfpathlineto{\pgfqpoint{1.016120in}{0.958250in}}%
\pgfpathlineto{\pgfqpoint{1.021080in}{0.847472in}}%
\pgfpathlineto{\pgfqpoint{1.027280in}{0.773010in}}%
\pgfpathlineto{\pgfqpoint{1.031000in}{0.743431in}}%
\pgfpathlineto{\pgfqpoint{1.032240in}{0.743706in}}%
\pgfpathlineto{\pgfqpoint{1.035960in}{0.734112in}}%
\pgfpathlineto{\pgfqpoint{1.037200in}{0.733846in}}%
\pgfpathlineto{\pgfqpoint{1.038440in}{0.731513in}}%
\pgfpathlineto{\pgfqpoint{1.042160in}{0.717786in}}%
\pgfpathlineto{\pgfqpoint{1.044640in}{0.720069in}}%
\pgfpathlineto{\pgfqpoint{1.047120in}{0.704306in}}%
\pgfpathlineto{\pgfqpoint{1.050840in}{0.684013in}}%
\pgfpathlineto{\pgfqpoint{1.053320in}{0.679370in}}%
\pgfpathlineto{\pgfqpoint{1.055800in}{0.670825in}}%
\pgfpathlineto{\pgfqpoint{1.059520in}{0.666126in}}%
\pgfpathlineto{\pgfqpoint{1.062000in}{0.657035in}}%
\pgfpathlineto{\pgfqpoint{1.066960in}{0.635592in}}%
\pgfpathlineto{\pgfqpoint{1.068200in}{0.634537in}}%
\pgfpathlineto{\pgfqpoint{1.073160in}{0.638264in}}%
\pgfpathlineto{\pgfqpoint{1.080600in}{0.632786in}}%
\pgfpathlineto{\pgfqpoint{1.081840in}{0.634842in}}%
\pgfpathlineto{\pgfqpoint{1.086800in}{0.620343in}}%
\pgfpathlineto{\pgfqpoint{1.089280in}{0.611148in}}%
\pgfpathlineto{\pgfqpoint{1.090520in}{0.610174in}}%
\pgfpathlineto{\pgfqpoint{1.091760in}{0.612192in}}%
\pgfpathlineto{\pgfqpoint{1.097960in}{0.598172in}}%
\pgfpathlineto{\pgfqpoint{1.100440in}{0.597745in}}%
\pgfpathlineto{\pgfqpoint{1.101680in}{0.602005in}}%
\pgfpathlineto{\pgfqpoint{1.109120in}{0.597646in}}%
\pgfpathlineto{\pgfqpoint{1.111600in}{0.600810in}}%
\pgfpathlineto{\pgfqpoint{1.112840in}{0.599438in}}%
\pgfpathlineto{\pgfqpoint{1.114080in}{0.600155in}}%
\pgfpathlineto{\pgfqpoint{1.115320in}{0.602385in}}%
\pgfpathlineto{\pgfqpoint{1.116560in}{0.602317in}}%
\pgfpathlineto{\pgfqpoint{1.117800in}{0.600753in}}%
\pgfpathlineto{\pgfqpoint{1.120280in}{0.602910in}}%
\pgfpathlineto{\pgfqpoint{1.122760in}{0.599798in}}%
\pgfpathlineto{\pgfqpoint{1.124000in}{0.600516in}}%
\pgfpathlineto{\pgfqpoint{1.126480in}{0.594510in}}%
\pgfpathlineto{\pgfqpoint{1.127720in}{0.595085in}}%
\pgfpathlineto{\pgfqpoint{1.131440in}{0.601090in}}%
\pgfpathlineto{\pgfqpoint{1.132680in}{0.600893in}}%
\pgfpathlineto{\pgfqpoint{1.133920in}{0.602305in}}%
\pgfpathlineto{\pgfqpoint{1.142600in}{0.590458in}}%
\pgfpathlineto{\pgfqpoint{1.143840in}{0.590253in}}%
\pgfpathlineto{\pgfqpoint{1.155000in}{0.567627in}}%
\pgfpathlineto{\pgfqpoint{1.157480in}{0.573421in}}%
\pgfpathlineto{\pgfqpoint{1.159960in}{0.577836in}}%
\pgfpathlineto{\pgfqpoint{1.164920in}{0.571425in}}%
\pgfpathlineto{\pgfqpoint{1.166160in}{0.570992in}}%
\pgfpathlineto{\pgfqpoint{1.168640in}{0.573712in}}%
\pgfpathlineto{\pgfqpoint{1.171120in}{0.569037in}}%
\pgfpathlineto{\pgfqpoint{1.172360in}{0.567767in}}%
\pgfpathlineto{\pgfqpoint{1.173600in}{0.569016in}}%
\pgfpathlineto{\pgfqpoint{1.176080in}{0.566710in}}%
\pgfpathlineto{\pgfqpoint{1.181040in}{0.567196in}}%
\pgfpathlineto{\pgfqpoint{1.183520in}{0.570553in}}%
\pgfpathlineto{\pgfqpoint{1.184760in}{0.570682in}}%
\pgfpathlineto{\pgfqpoint{1.187240in}{0.562987in}}%
\pgfpathlineto{\pgfqpoint{1.188480in}{0.563901in}}%
\pgfpathlineto{\pgfqpoint{1.193440in}{0.558621in}}%
\pgfpathlineto{\pgfqpoint{1.195920in}{0.559482in}}%
\pgfpathlineto{\pgfqpoint{1.197160in}{0.561904in}}%
\pgfpathlineto{\pgfqpoint{1.199640in}{0.561543in}}%
\pgfpathlineto{\pgfqpoint{1.200880in}{0.561849in}}%
\pgfpathlineto{\pgfqpoint{1.202120in}{0.564093in}}%
\pgfpathlineto{\pgfqpoint{1.207080in}{0.561738in}}%
\pgfpathlineto{\pgfqpoint{1.213280in}{0.555504in}}%
\pgfpathlineto{\pgfqpoint{1.214520in}{0.555188in}}%
\pgfpathlineto{\pgfqpoint{1.217000in}{0.557346in}}%
\pgfpathlineto{\pgfqpoint{1.221960in}{0.552509in}}%
\pgfpathlineto{\pgfqpoint{1.224440in}{0.554674in}}%
\pgfpathlineto{\pgfqpoint{1.225680in}{0.555297in}}%
\pgfpathlineto{\pgfqpoint{1.228160in}{0.550850in}}%
\pgfpathlineto{\pgfqpoint{1.229400in}{0.550249in}}%
\pgfpathlineto{\pgfqpoint{1.230640in}{0.552100in}}%
\pgfpathlineto{\pgfqpoint{1.231880in}{0.551608in}}%
\pgfpathlineto{\pgfqpoint{1.233120in}{0.552444in}}%
\pgfpathlineto{\pgfqpoint{1.235600in}{0.558143in}}%
\pgfpathlineto{\pgfqpoint{1.241800in}{0.556789in}}%
\pgfpathlineto{\pgfqpoint{1.244280in}{0.558707in}}%
\pgfpathlineto{\pgfqpoint{1.249240in}{0.550724in}}%
\pgfpathlineto{\pgfqpoint{1.251720in}{0.553495in}}%
\pgfpathlineto{\pgfqpoint{1.254200in}{0.555965in}}%
\pgfpathlineto{\pgfqpoint{1.256680in}{0.556090in}}%
\pgfpathlineto{\pgfqpoint{1.257920in}{0.558479in}}%
\pgfpathlineto{\pgfqpoint{1.262880in}{0.554243in}}%
\pgfpathlineto{\pgfqpoint{1.265360in}{0.555219in}}%
\pgfpathlineto{\pgfqpoint{1.267840in}{0.558465in}}%
\pgfpathlineto{\pgfqpoint{1.270320in}{0.555959in}}%
\pgfpathlineto{\pgfqpoint{1.275280in}{0.551944in}}%
\pgfpathlineto{\pgfqpoint{1.279000in}{0.545100in}}%
\pgfpathlineto{\pgfqpoint{1.283960in}{0.549681in}}%
\pgfpathlineto{\pgfqpoint{1.286440in}{0.546961in}}%
\pgfpathlineto{\pgfqpoint{1.288920in}{0.547073in}}%
\pgfpathlineto{\pgfqpoint{1.290160in}{0.546746in}}%
\pgfpathlineto{\pgfqpoint{1.292640in}{0.549307in}}%
\pgfpathlineto{\pgfqpoint{1.295120in}{0.547328in}}%
\pgfpathlineto{\pgfqpoint{1.296360in}{0.547349in}}%
\pgfpathlineto{\pgfqpoint{1.297600in}{0.549048in}}%
\pgfpathlineto{\pgfqpoint{1.300080in}{0.547365in}}%
\pgfpathlineto{\pgfqpoint{1.302560in}{0.549279in}}%
\pgfpathlineto{\pgfqpoint{1.307520in}{0.552262in}}%
\pgfpathlineto{\pgfqpoint{1.308760in}{0.552027in}}%
\pgfpathlineto{\pgfqpoint{1.311240in}{0.546807in}}%
\pgfpathlineto{\pgfqpoint{1.312480in}{0.547178in}}%
\pgfpathlineto{\pgfqpoint{1.316200in}{0.540028in}}%
\pgfpathlineto{\pgfqpoint{1.321160in}{0.543525in}}%
\pgfpathlineto{\pgfqpoint{1.323640in}{0.541720in}}%
\pgfpathlineto{\pgfqpoint{1.327360in}{0.545485in}}%
\pgfpathlineto{\pgfqpoint{1.328600in}{0.544086in}}%
\pgfpathlineto{\pgfqpoint{1.329840in}{0.545551in}}%
\pgfpathlineto{\pgfqpoint{1.336040in}{0.539554in}}%
\pgfpathlineto{\pgfqpoint{1.338520in}{0.539479in}}%
\pgfpathlineto{\pgfqpoint{1.341000in}{0.544005in}}%
\pgfpathlineto{\pgfqpoint{1.343480in}{0.542891in}}%
\pgfpathlineto{\pgfqpoint{1.344720in}{0.541797in}}%
\pgfpathlineto{\pgfqpoint{1.345960in}{0.538636in}}%
\pgfpathlineto{\pgfqpoint{1.349680in}{0.541220in}}%
\pgfpathlineto{\pgfqpoint{1.352160in}{0.538629in}}%
\pgfpathlineto{\pgfqpoint{1.357120in}{0.539315in}}%
\pgfpathlineto{\pgfqpoint{1.362080in}{0.544551in}}%
\pgfpathlineto{\pgfqpoint{1.363320in}{0.543516in}}%
\pgfpathlineto{\pgfqpoint{1.364560in}{0.543991in}}%
\pgfpathlineto{\pgfqpoint{1.365800in}{0.543215in}}%
\pgfpathlineto{\pgfqpoint{1.368280in}{0.544220in}}%
\pgfpathlineto{\pgfqpoint{1.372000in}{0.538948in}}%
\pgfpathlineto{\pgfqpoint{1.373240in}{0.538432in}}%
\pgfpathlineto{\pgfqpoint{1.376960in}{0.542480in}}%
\pgfpathlineto{\pgfqpoint{1.379440in}{0.542106in}}%
\pgfpathlineto{\pgfqpoint{1.381920in}{0.545573in}}%
\pgfpathlineto{\pgfqpoint{1.386880in}{0.540974in}}%
\pgfpathlineto{\pgfqpoint{1.391840in}{0.545395in}}%
\pgfpathlineto{\pgfqpoint{1.395560in}{0.543388in}}%
\pgfpathlineto{\pgfqpoint{1.396800in}{0.543714in}}%
\pgfpathlineto{\pgfqpoint{1.399280in}{0.542021in}}%
\pgfpathlineto{\pgfqpoint{1.404240in}{0.538615in}}%
\pgfpathlineto{\pgfqpoint{1.407960in}{0.542572in}}%
\pgfpathlineto{\pgfqpoint{1.412920in}{0.540075in}}%
\pgfpathlineto{\pgfqpoint{1.415400in}{0.540393in}}%
\pgfpathlineto{\pgfqpoint{1.416640in}{0.543193in}}%
\pgfpathlineto{\pgfqpoint{1.420360in}{0.541631in}}%
\pgfpathlineto{\pgfqpoint{1.421600in}{0.542473in}}%
\pgfpathlineto{\pgfqpoint{1.424080in}{0.541713in}}%
\pgfpathlineto{\pgfqpoint{1.425320in}{0.542693in}}%
\pgfpathlineto{\pgfqpoint{1.426560in}{0.541979in}}%
\pgfpathlineto{\pgfqpoint{1.432760in}{0.545122in}}%
\pgfpathlineto{\pgfqpoint{1.435240in}{0.539719in}}%
\pgfpathlineto{\pgfqpoint{1.436480in}{0.540611in}}%
\pgfpathlineto{\pgfqpoint{1.440200in}{0.535310in}}%
\pgfpathlineto{\pgfqpoint{1.442680in}{0.538225in}}%
\pgfpathlineto{\pgfqpoint{1.446400in}{0.539551in}}%
\pgfpathlineto{\pgfqpoint{1.447640in}{0.539138in}}%
\pgfpathlineto{\pgfqpoint{1.451360in}{0.541810in}}%
\pgfpathlineto{\pgfqpoint{1.452600in}{0.541062in}}%
\pgfpathlineto{\pgfqpoint{1.453840in}{0.542115in}}%
\pgfpathlineto{\pgfqpoint{1.456320in}{0.540287in}}%
\pgfpathlineto{\pgfqpoint{1.457560in}{0.540145in}}%
\pgfpathlineto{\pgfqpoint{1.460040in}{0.536222in}}%
\pgfpathlineto{\pgfqpoint{1.462520in}{0.537755in}}%
\pgfpathlineto{\pgfqpoint{1.465000in}{0.543172in}}%
\pgfpathlineto{\pgfqpoint{1.466240in}{0.542598in}}%
\pgfpathlineto{\pgfqpoint{1.467480in}{0.543339in}}%
\pgfpathlineto{\pgfqpoint{1.469960in}{0.539648in}}%
\pgfpathlineto{\pgfqpoint{1.473680in}{0.541787in}}%
\pgfpathlineto{\pgfqpoint{1.474920in}{0.540072in}}%
\pgfpathlineto{\pgfqpoint{1.482360in}{0.542388in}}%
\pgfpathlineto{\pgfqpoint{1.484840in}{0.545145in}}%
\pgfpathlineto{\pgfqpoint{1.486080in}{0.545270in}}%
\pgfpathlineto{\pgfqpoint{1.489800in}{0.543040in}}%
\pgfpathlineto{\pgfqpoint{1.492280in}{0.544939in}}%
\pgfpathlineto{\pgfqpoint{1.496000in}{0.539158in}}%
\pgfpathlineto{\pgfqpoint{1.497240in}{0.539577in}}%
\pgfpathlineto{\pgfqpoint{1.500960in}{0.544315in}}%
\pgfpathlineto{\pgfqpoint{1.503440in}{0.545434in}}%
\pgfpathlineto{\pgfqpoint{1.507160in}{0.547594in}}%
\pgfpathlineto{\pgfqpoint{1.510880in}{0.544649in}}%
\pgfpathlineto{\pgfqpoint{1.515840in}{0.546162in}}%
\pgfpathlineto{\pgfqpoint{1.519560in}{0.544896in}}%
\pgfpathlineto{\pgfqpoint{1.520800in}{0.545475in}}%
\pgfpathlineto{\pgfqpoint{1.523280in}{0.543212in}}%
\pgfpathlineto{\pgfqpoint{1.528240in}{0.539871in}}%
\pgfpathlineto{\pgfqpoint{1.531960in}{0.543601in}}%
\pgfpathlineto{\pgfqpoint{1.535680in}{0.543998in}}%
\pgfpathlineto{\pgfqpoint{1.538160in}{0.544571in}}%
\pgfpathlineto{\pgfqpoint{1.541880in}{0.545366in}}%
\pgfpathlineto{\pgfqpoint{1.544360in}{0.543434in}}%
\pgfpathlineto{\pgfqpoint{1.546840in}{0.543404in}}%
\pgfpathlineto{\pgfqpoint{1.548080in}{0.542912in}}%
\pgfpathlineto{\pgfqpoint{1.549320in}{0.543877in}}%
\pgfpathlineto{\pgfqpoint{1.550560in}{0.543359in}}%
\pgfpathlineto{\pgfqpoint{1.553040in}{0.544480in}}%
\pgfpathlineto{\pgfqpoint{1.555520in}{0.544538in}}%
\pgfpathlineto{\pgfqpoint{1.556760in}{0.545050in}}%
\pgfpathlineto{\pgfqpoint{1.559240in}{0.539799in}}%
\pgfpathlineto{\pgfqpoint{1.560480in}{0.540735in}}%
\pgfpathlineto{\pgfqpoint{1.564200in}{0.537189in}}%
\pgfpathlineto{\pgfqpoint{1.566680in}{0.539934in}}%
\pgfpathlineto{\pgfqpoint{1.571640in}{0.540094in}}%
\pgfpathlineto{\pgfqpoint{1.575360in}{0.542979in}}%
\pgfpathlineto{\pgfqpoint{1.576600in}{0.541476in}}%
\pgfpathlineto{\pgfqpoint{1.577840in}{0.542693in}}%
\pgfpathlineto{\pgfqpoint{1.580320in}{0.540851in}}%
\pgfpathlineto{\pgfqpoint{1.581560in}{0.540970in}}%
\pgfpathlineto{\pgfqpoint{1.584040in}{0.537754in}}%
\pgfpathlineto{\pgfqpoint{1.586520in}{0.539064in}}%
\pgfpathlineto{\pgfqpoint{1.589000in}{0.544562in}}%
\pgfpathlineto{\pgfqpoint{1.591480in}{0.544485in}}%
\pgfpathlineto{\pgfqpoint{1.592720in}{0.544043in}}%
\pgfpathlineto{\pgfqpoint{1.593960in}{0.542327in}}%
\pgfpathlineto{\pgfqpoint{1.595200in}{0.543033in}}%
\pgfpathlineto{\pgfqpoint{1.600160in}{0.540669in}}%
\pgfpathlineto{\pgfqpoint{1.606360in}{0.541136in}}%
\pgfpathlineto{\pgfqpoint{1.608840in}{0.544059in}}%
\pgfpathlineto{\pgfqpoint{1.610080in}{0.545180in}}%
\pgfpathlineto{\pgfqpoint{1.613800in}{0.543606in}}%
\pgfpathlineto{\pgfqpoint{1.616280in}{0.545113in}}%
\pgfpathlineto{\pgfqpoint{1.621240in}{0.540797in}}%
\pgfpathlineto{\pgfqpoint{1.624960in}{0.544808in}}%
\pgfpathlineto{\pgfqpoint{1.627440in}{0.544871in}}%
\pgfpathlineto{\pgfqpoint{1.629920in}{0.548954in}}%
\pgfpathlineto{\pgfqpoint{1.634880in}{0.545411in}}%
\pgfpathlineto{\pgfqpoint{1.638600in}{0.545679in}}%
\pgfpathlineto{\pgfqpoint{1.641080in}{0.544591in}}%
\pgfpathlineto{\pgfqpoint{1.643560in}{0.543787in}}%
\pgfpathlineto{\pgfqpoint{1.644800in}{0.544656in}}%
\pgfpathlineto{\pgfqpoint{1.647280in}{0.543392in}}%
\pgfpathlineto{\pgfqpoint{1.652240in}{0.540269in}}%
\pgfpathlineto{\pgfqpoint{1.655960in}{0.543127in}}%
\pgfpathlineto{\pgfqpoint{1.663400in}{0.544536in}}%
\pgfpathlineto{\pgfqpoint{1.664640in}{0.545831in}}%
\pgfpathlineto{\pgfqpoint{1.668360in}{0.543392in}}%
\pgfpathlineto{\pgfqpoint{1.669600in}{0.543702in}}%
\pgfpathlineto{\pgfqpoint{1.670840in}{0.542780in}}%
\pgfpathlineto{\pgfqpoint{1.674560in}{0.543938in}}%
\pgfpathlineto{\pgfqpoint{1.680760in}{0.544415in}}%
\pgfpathlineto{\pgfqpoint{1.683240in}{0.539303in}}%
\pgfpathlineto{\pgfqpoint{1.684480in}{0.540382in}}%
\pgfpathlineto{\pgfqpoint{1.688200in}{0.538442in}}%
\pgfpathlineto{\pgfqpoint{1.690680in}{0.540895in}}%
\pgfpathlineto{\pgfqpoint{1.701840in}{0.544225in}}%
\pgfpathlineto{\pgfqpoint{1.704320in}{0.542410in}}%
\pgfpathlineto{\pgfqpoint{1.705560in}{0.542076in}}%
\pgfpathlineto{\pgfqpoint{1.708040in}{0.538596in}}%
\pgfpathlineto{\pgfqpoint{1.710520in}{0.540226in}}%
\pgfpathlineto{\pgfqpoint{1.713000in}{0.544913in}}%
\pgfpathlineto{\pgfqpoint{1.714240in}{0.543866in}}%
\pgfpathlineto{\pgfqpoint{1.716720in}{0.544874in}}%
\pgfpathlineto{\pgfqpoint{1.717960in}{0.543420in}}%
\pgfpathlineto{\pgfqpoint{1.719200in}{0.544634in}}%
\pgfpathlineto{\pgfqpoint{1.725400in}{0.542699in}}%
\pgfpathlineto{\pgfqpoint{1.726640in}{0.543765in}}%
\pgfpathlineto{\pgfqpoint{1.729120in}{0.541970in}}%
\pgfpathlineto{\pgfqpoint{1.732840in}{0.545270in}}%
\pgfpathlineto{\pgfqpoint{1.734080in}{0.546474in}}%
\pgfpathlineto{\pgfqpoint{1.737800in}{0.544250in}}%
\pgfpathlineto{\pgfqpoint{1.740280in}{0.545381in}}%
\pgfpathlineto{\pgfqpoint{1.745240in}{0.542295in}}%
\pgfpathlineto{\pgfqpoint{1.748960in}{0.545794in}}%
\pgfpathlineto{\pgfqpoint{1.751440in}{0.545659in}}%
\pgfpathlineto{\pgfqpoint{1.753920in}{0.548341in}}%
\pgfpathlineto{\pgfqpoint{1.757640in}{0.545445in}}%
\pgfpathlineto{\pgfqpoint{1.760120in}{0.544645in}}%
\pgfpathlineto{\pgfqpoint{1.763840in}{0.544991in}}%
\pgfpathlineto{\pgfqpoint{1.766320in}{0.543141in}}%
\pgfpathlineto{\pgfqpoint{1.770040in}{0.544413in}}%
\pgfpathlineto{\pgfqpoint{1.772520in}{0.543721in}}%
\pgfpathlineto{\pgfqpoint{1.776240in}{0.541626in}}%
\pgfpathlineto{\pgfqpoint{1.778720in}{0.543731in}}%
\pgfpathlineto{\pgfqpoint{1.787400in}{0.544628in}}%
\pgfpathlineto{\pgfqpoint{1.788640in}{0.546696in}}%
\pgfpathlineto{\pgfqpoint{1.797320in}{0.544602in}}%
\pgfpathlineto{\pgfqpoint{1.801040in}{0.545285in}}%
\pgfpathlineto{\pgfqpoint{1.803520in}{0.544514in}}%
\pgfpathlineto{\pgfqpoint{1.804760in}{0.545281in}}%
\pgfpathlineto{\pgfqpoint{1.808480in}{0.540890in}}%
\pgfpathlineto{\pgfqpoint{1.812200in}{0.538696in}}%
\pgfpathlineto{\pgfqpoint{1.814680in}{0.541572in}}%
\pgfpathlineto{\pgfqpoint{1.819640in}{0.542322in}}%
\pgfpathlineto{\pgfqpoint{1.823360in}{0.545171in}}%
\pgfpathlineto{\pgfqpoint{1.824600in}{0.543747in}}%
\pgfpathlineto{\pgfqpoint{1.825840in}{0.545008in}}%
\pgfpathlineto{\pgfqpoint{1.828320in}{0.543070in}}%
\pgfpathlineto{\pgfqpoint{1.829560in}{0.542455in}}%
\pgfpathlineto{\pgfqpoint{1.832040in}{0.539361in}}%
\pgfpathlineto{\pgfqpoint{1.834520in}{0.541819in}}%
\pgfpathlineto{\pgfqpoint{1.837000in}{0.546432in}}%
\pgfpathlineto{\pgfqpoint{1.838240in}{0.545283in}}%
\pgfpathlineto{\pgfqpoint{1.840720in}{0.546450in}}%
\pgfpathlineto{\pgfqpoint{1.841960in}{0.545577in}}%
\pgfpathlineto{\pgfqpoint{1.843200in}{0.546418in}}%
\pgfpathlineto{\pgfqpoint{1.854360in}{0.542400in}}%
\pgfpathlineto{\pgfqpoint{1.856840in}{0.545197in}}%
\pgfpathlineto{\pgfqpoint{1.858080in}{0.546204in}}%
\pgfpathlineto{\pgfqpoint{1.861800in}{0.544412in}}%
\pgfpathlineto{\pgfqpoint{1.865520in}{0.544865in}}%
\pgfpathlineto{\pgfqpoint{1.868000in}{0.542362in}}%
\pgfpathlineto{\pgfqpoint{1.871720in}{0.546878in}}%
\pgfpathlineto{\pgfqpoint{1.874200in}{0.547401in}}%
\pgfpathlineto{\pgfqpoint{1.875440in}{0.546925in}}%
\pgfpathlineto{\pgfqpoint{1.877920in}{0.549944in}}%
\pgfpathlineto{\pgfqpoint{1.882880in}{0.547179in}}%
\pgfpathlineto{\pgfqpoint{1.887840in}{0.546927in}}%
\pgfpathlineto{\pgfqpoint{1.890320in}{0.544333in}}%
\pgfpathlineto{\pgfqpoint{1.891560in}{0.544624in}}%
\pgfpathlineto{\pgfqpoint{1.894040in}{0.546251in}}%
\pgfpathlineto{\pgfqpoint{1.900240in}{0.543881in}}%
\pgfpathlineto{\pgfqpoint{1.902720in}{0.546020in}}%
\pgfpathlineto{\pgfqpoint{1.907680in}{0.546485in}}%
\pgfpathlineto{\pgfqpoint{1.910160in}{0.546370in}}%
\pgfpathlineto{\pgfqpoint{1.911400in}{0.546579in}}%
\pgfpathlineto{\pgfqpoint{1.912640in}{0.548514in}}%
\pgfpathlineto{\pgfqpoint{1.916360in}{0.547445in}}%
\pgfpathlineto{\pgfqpoint{1.917600in}{0.548492in}}%
\pgfpathlineto{\pgfqpoint{1.920080in}{0.546960in}}%
\pgfpathlineto{\pgfqpoint{1.922560in}{0.546953in}}%
\pgfpathlineto{\pgfqpoint{1.925040in}{0.547971in}}%
\pgfpathlineto{\pgfqpoint{1.926280in}{0.546450in}}%
\pgfpathlineto{\pgfqpoint{1.928760in}{0.548082in}}%
\pgfpathlineto{\pgfqpoint{1.930000in}{0.546542in}}%
\pgfpathlineto{\pgfqpoint{1.931240in}{0.543030in}}%
\pgfpathlineto{\pgfqpoint{1.933720in}{0.543223in}}%
\pgfpathlineto{\pgfqpoint{1.936200in}{0.540718in}}%
\pgfpathlineto{\pgfqpoint{1.938680in}{0.542466in}}%
\pgfpathlineto{\pgfqpoint{1.944880in}{0.544619in}}%
\pgfpathlineto{\pgfqpoint{1.947360in}{0.546562in}}%
\pgfpathlineto{\pgfqpoint{1.948600in}{0.545508in}}%
\pgfpathlineto{\pgfqpoint{1.949840in}{0.546475in}}%
\pgfpathlineto{\pgfqpoint{1.952320in}{0.544873in}}%
\pgfpathlineto{\pgfqpoint{1.953560in}{0.544369in}}%
\pgfpathlineto{\pgfqpoint{1.956040in}{0.540957in}}%
\pgfpathlineto{\pgfqpoint{1.961000in}{0.547868in}}%
\pgfpathlineto{\pgfqpoint{1.962240in}{0.546403in}}%
\pgfpathlineto{\pgfqpoint{1.967200in}{0.547350in}}%
\pgfpathlineto{\pgfqpoint{1.972160in}{0.546168in}}%
\pgfpathlineto{\pgfqpoint{1.977120in}{0.543565in}}%
\pgfpathlineto{\pgfqpoint{1.978360in}{0.544292in}}%
\pgfpathlineto{\pgfqpoint{1.980840in}{0.547406in}}%
\pgfpathlineto{\pgfqpoint{1.982080in}{0.548036in}}%
\pgfpathlineto{\pgfqpoint{1.984560in}{0.546460in}}%
\pgfpathlineto{\pgfqpoint{1.985800in}{0.545993in}}%
\pgfpathlineto{\pgfqpoint{1.988280in}{0.548365in}}%
\pgfpathlineto{\pgfqpoint{1.992000in}{0.544900in}}%
\pgfpathlineto{\pgfqpoint{1.994480in}{0.547359in}}%
\pgfpathlineto{\pgfqpoint{1.998200in}{0.548900in}}%
\pgfpathlineto{\pgfqpoint{1.999440in}{0.548834in}}%
\pgfpathlineto{\pgfqpoint{2.001920in}{0.551914in}}%
\pgfpathlineto{\pgfqpoint{2.006880in}{0.548900in}}%
\pgfpathlineto{\pgfqpoint{2.011840in}{0.548377in}}%
\pgfpathlineto{\pgfqpoint{2.014320in}{0.545814in}}%
\pgfpathlineto{\pgfqpoint{2.020520in}{0.546696in}}%
\pgfpathlineto{\pgfqpoint{2.023000in}{0.544773in}}%
\pgfpathlineto{\pgfqpoint{2.030440in}{0.548022in}}%
\pgfpathlineto{\pgfqpoint{2.035400in}{0.547356in}}%
\pgfpathlineto{\pgfqpoint{2.036640in}{0.548963in}}%
\pgfpathlineto{\pgfqpoint{2.045320in}{0.546928in}}%
\pgfpathlineto{\pgfqpoint{2.049040in}{0.548237in}}%
\pgfpathlineto{\pgfqpoint{2.050280in}{0.546570in}}%
\pgfpathlineto{\pgfqpoint{2.052760in}{0.548062in}}%
\pgfpathlineto{\pgfqpoint{2.054000in}{0.546515in}}%
\pgfpathlineto{\pgfqpoint{2.055240in}{0.543060in}}%
\pgfpathlineto{\pgfqpoint{2.057720in}{0.543137in}}%
\pgfpathlineto{\pgfqpoint{2.060200in}{0.540998in}}%
\pgfpathlineto{\pgfqpoint{2.063920in}{0.543077in}}%
\pgfpathlineto{\pgfqpoint{2.066400in}{0.542987in}}%
\pgfpathlineto{\pgfqpoint{2.073840in}{0.544496in}}%
\pgfpathlineto{\pgfqpoint{2.076320in}{0.542888in}}%
\pgfpathlineto{\pgfqpoint{2.077560in}{0.542680in}}%
\pgfpathlineto{\pgfqpoint{2.080040in}{0.539469in}}%
\pgfpathlineto{\pgfqpoint{2.085000in}{0.546248in}}%
\pgfpathlineto{\pgfqpoint{2.086240in}{0.544668in}}%
\pgfpathlineto{\pgfqpoint{2.091200in}{0.545040in}}%
\pgfpathlineto{\pgfqpoint{2.096160in}{0.543972in}}%
\pgfpathlineto{\pgfqpoint{2.101120in}{0.541602in}}%
\pgfpathlineto{\pgfqpoint{2.106080in}{0.545830in}}%
\pgfpathlineto{\pgfqpoint{2.108560in}{0.544673in}}%
\pgfpathlineto{\pgfqpoint{2.109800in}{0.544330in}}%
\pgfpathlineto{\pgfqpoint{2.112280in}{0.546986in}}%
\pgfpathlineto{\pgfqpoint{2.116000in}{0.543852in}}%
\pgfpathlineto{\pgfqpoint{2.120960in}{0.548024in}}%
\pgfpathlineto{\pgfqpoint{2.123440in}{0.547250in}}%
\pgfpathlineto{\pgfqpoint{2.125920in}{0.550103in}}%
\pgfpathlineto{\pgfqpoint{2.130880in}{0.547661in}}%
\pgfpathlineto{\pgfqpoint{2.135840in}{0.547986in}}%
\pgfpathlineto{\pgfqpoint{2.139560in}{0.545397in}}%
\pgfpathlineto{\pgfqpoint{2.142040in}{0.545868in}}%
\pgfpathlineto{\pgfqpoint{2.148240in}{0.544896in}}%
\pgfpathlineto{\pgfqpoint{2.150720in}{0.546236in}}%
\pgfpathlineto{\pgfqpoint{2.154440in}{0.547829in}}%
\pgfpathlineto{\pgfqpoint{2.158160in}{0.547100in}}%
\pgfpathlineto{\pgfqpoint{2.159400in}{0.547349in}}%
\pgfpathlineto{\pgfqpoint{2.160640in}{0.549183in}}%
\pgfpathlineto{\pgfqpoint{2.175520in}{0.547773in}}%
\pgfpathlineto{\pgfqpoint{2.176760in}{0.548458in}}%
\pgfpathlineto{\pgfqpoint{2.178000in}{0.547308in}}%
\pgfpathlineto{\pgfqpoint{2.179240in}{0.543907in}}%
\pgfpathlineto{\pgfqpoint{2.181720in}{0.543889in}}%
\pgfpathlineto{\pgfqpoint{2.184200in}{0.542373in}}%
\pgfpathlineto{\pgfqpoint{2.187920in}{0.543452in}}%
\pgfpathlineto{\pgfqpoint{2.190400in}{0.543139in}}%
\pgfpathlineto{\pgfqpoint{2.197840in}{0.544076in}}%
\pgfpathlineto{\pgfqpoint{2.200320in}{0.542599in}}%
\pgfpathlineto{\pgfqpoint{2.202800in}{0.541026in}}%
\pgfpathlineto{\pgfqpoint{2.204040in}{0.540119in}}%
\pgfpathlineto{\pgfqpoint{2.209000in}{0.546582in}}%
\pgfpathlineto{\pgfqpoint{2.210240in}{0.545419in}}%
\pgfpathlineto{\pgfqpoint{2.213960in}{0.545974in}}%
\pgfpathlineto{\pgfqpoint{2.222640in}{0.544453in}}%
\pgfpathlineto{\pgfqpoint{2.225120in}{0.542465in}}%
\pgfpathlineto{\pgfqpoint{2.230080in}{0.545591in}}%
\pgfpathlineto{\pgfqpoint{2.232560in}{0.544096in}}%
\pgfpathlineto{\pgfqpoint{2.233800in}{0.544118in}}%
\pgfpathlineto{\pgfqpoint{2.236280in}{0.546348in}}%
\pgfpathlineto{\pgfqpoint{2.240000in}{0.543689in}}%
\pgfpathlineto{\pgfqpoint{2.243720in}{0.547560in}}%
\pgfpathlineto{\pgfqpoint{2.244960in}{0.548694in}}%
\pgfpathlineto{\pgfqpoint{2.247440in}{0.547758in}}%
\pgfpathlineto{\pgfqpoint{2.249920in}{0.550432in}}%
\pgfpathlineto{\pgfqpoint{2.254880in}{0.548950in}}%
\pgfpathlineto{\pgfqpoint{2.259840in}{0.548959in}}%
\pgfpathlineto{\pgfqpoint{2.262320in}{0.546933in}}%
\pgfpathlineto{\pgfqpoint{2.268520in}{0.546760in}}%
\pgfpathlineto{\pgfqpoint{2.271000in}{0.545377in}}%
\pgfpathlineto{\pgfqpoint{2.282160in}{0.548769in}}%
\pgfpathlineto{\pgfqpoint{2.283400in}{0.548442in}}%
\pgfpathlineto{\pgfqpoint{2.284640in}{0.550339in}}%
\pgfpathlineto{\pgfqpoint{2.292080in}{0.549170in}}%
\pgfpathlineto{\pgfqpoint{2.297040in}{0.550155in}}%
\pgfpathlineto{\pgfqpoint{2.298280in}{0.548388in}}%
\pgfpathlineto{\pgfqpoint{2.300760in}{0.550007in}}%
\pgfpathlineto{\pgfqpoint{2.302000in}{0.548796in}}%
\pgfpathlineto{\pgfqpoint{2.303240in}{0.545836in}}%
\pgfpathlineto{\pgfqpoint{2.304480in}{0.546498in}}%
\pgfpathlineto{\pgfqpoint{2.308200in}{0.544002in}}%
\pgfpathlineto{\pgfqpoint{2.311920in}{0.545469in}}%
\pgfpathlineto{\pgfqpoint{2.315640in}{0.544859in}}%
\pgfpathlineto{\pgfqpoint{2.319360in}{0.547343in}}%
\pgfpathlineto{\pgfqpoint{2.321840in}{0.545595in}}%
\pgfpathlineto{\pgfqpoint{2.326800in}{0.542683in}}%
\pgfpathlineto{\pgfqpoint{2.328040in}{0.542158in}}%
\pgfpathlineto{\pgfqpoint{2.333000in}{0.547417in}}%
\pgfpathlineto{\pgfqpoint{2.334240in}{0.546477in}}%
\pgfpathlineto{\pgfqpoint{2.337960in}{0.547563in}}%
\pgfpathlineto{\pgfqpoint{2.346640in}{0.545828in}}%
\pgfpathlineto{\pgfqpoint{2.349120in}{0.543783in}}%
\pgfpathlineto{\pgfqpoint{2.354080in}{0.547074in}}%
\pgfpathlineto{\pgfqpoint{2.356560in}{0.545899in}}%
\pgfpathlineto{\pgfqpoint{2.357800in}{0.545812in}}%
\pgfpathlineto{\pgfqpoint{2.361520in}{0.547634in}}%
\pgfpathlineto{\pgfqpoint{2.364000in}{0.545931in}}%
\pgfpathlineto{\pgfqpoint{2.368960in}{0.549718in}}%
\pgfpathlineto{\pgfqpoint{2.371440in}{0.548559in}}%
\pgfpathlineto{\pgfqpoint{2.375160in}{0.550941in}}%
\pgfpathlineto{\pgfqpoint{2.378880in}{0.551239in}}%
\pgfpathlineto{\pgfqpoint{2.383840in}{0.550569in}}%
\pgfpathlineto{\pgfqpoint{2.386320in}{0.548627in}}%
\pgfpathlineto{\pgfqpoint{2.390040in}{0.548873in}}%
\pgfpathlineto{\pgfqpoint{2.393760in}{0.547236in}}%
\pgfpathlineto{\pgfqpoint{2.396240in}{0.546907in}}%
\pgfpathlineto{\pgfqpoint{2.398720in}{0.548794in}}%
\pgfpathlineto{\pgfqpoint{2.404920in}{0.550348in}}%
\pgfpathlineto{\pgfqpoint{2.407400in}{0.549774in}}%
\pgfpathlineto{\pgfqpoint{2.408640in}{0.551643in}}%
\pgfpathlineto{\pgfqpoint{2.414840in}{0.550003in}}%
\pgfpathlineto{\pgfqpoint{2.421040in}{0.550944in}}%
\pgfpathlineto{\pgfqpoint{2.422280in}{0.549106in}}%
\pgfpathlineto{\pgfqpoint{2.424760in}{0.550862in}}%
\pgfpathlineto{\pgfqpoint{2.426000in}{0.550056in}}%
\pgfpathlineto{\pgfqpoint{2.427240in}{0.547344in}}%
\pgfpathlineto{\pgfqpoint{2.428480in}{0.548218in}}%
\pgfpathlineto{\pgfqpoint{2.433440in}{0.545790in}}%
\pgfpathlineto{\pgfqpoint{2.437160in}{0.546136in}}%
\pgfpathlineto{\pgfqpoint{2.440880in}{0.546846in}}%
\pgfpathlineto{\pgfqpoint{2.443360in}{0.547583in}}%
\pgfpathlineto{\pgfqpoint{2.444600in}{0.546180in}}%
\pgfpathlineto{\pgfqpoint{2.445840in}{0.546564in}}%
\pgfpathlineto{\pgfqpoint{2.449560in}{0.544996in}}%
\pgfpathlineto{\pgfqpoint{2.452040in}{0.543636in}}%
\pgfpathlineto{\pgfqpoint{2.457000in}{0.548132in}}%
\pgfpathlineto{\pgfqpoint{2.458240in}{0.546735in}}%
\pgfpathlineto{\pgfqpoint{2.464440in}{0.547165in}}%
\pgfpathlineto{\pgfqpoint{2.465680in}{0.547302in}}%
\pgfpathlineto{\pgfqpoint{2.466920in}{0.546026in}}%
\pgfpathlineto{\pgfqpoint{2.470640in}{0.546168in}}%
\pgfpathlineto{\pgfqpoint{2.473120in}{0.544444in}}%
\pgfpathlineto{\pgfqpoint{2.476840in}{0.546090in}}%
\pgfpathlineto{\pgfqpoint{2.478080in}{0.546861in}}%
\pgfpathlineto{\pgfqpoint{2.480560in}{0.545315in}}%
\pgfpathlineto{\pgfqpoint{2.483040in}{0.545983in}}%
\pgfpathlineto{\pgfqpoint{2.485520in}{0.547071in}}%
\pgfpathlineto{\pgfqpoint{2.490480in}{0.545051in}}%
\pgfpathlineto{\pgfqpoint{2.494200in}{0.546683in}}%
\pgfpathlineto{\pgfqpoint{2.495440in}{0.546383in}}%
\pgfpathlineto{\pgfqpoint{2.499160in}{0.548227in}}%
\pgfpathlineto{\pgfqpoint{2.500400in}{0.547780in}}%
\pgfpathlineto{\pgfqpoint{2.504120in}{0.548860in}}%
\pgfpathlineto{\pgfqpoint{2.506600in}{0.549006in}}%
\pgfpathlineto{\pgfqpoint{2.512800in}{0.546798in}}%
\pgfpathlineto{\pgfqpoint{2.520240in}{0.545177in}}%
\pgfpathlineto{\pgfqpoint{2.522720in}{0.546384in}}%
\pgfpathlineto{\pgfqpoint{2.530160in}{0.547880in}}%
\pgfpathlineto{\pgfqpoint{2.531400in}{0.547543in}}%
\pgfpathlineto{\pgfqpoint{2.532640in}{0.549601in}}%
\pgfpathlineto{\pgfqpoint{2.536360in}{0.548681in}}%
\pgfpathlineto{\pgfqpoint{2.541320in}{0.549840in}}%
\pgfpathlineto{\pgfqpoint{2.542560in}{0.550418in}}%
\pgfpathlineto{\pgfqpoint{2.546280in}{0.547741in}}%
\pgfpathlineto{\pgfqpoint{2.548760in}{0.549465in}}%
\pgfpathlineto{\pgfqpoint{2.550000in}{0.548825in}}%
\pgfpathlineto{\pgfqpoint{2.551240in}{0.546317in}}%
\pgfpathlineto{\pgfqpoint{2.552480in}{0.547519in}}%
\pgfpathlineto{\pgfqpoint{2.554960in}{0.545281in}}%
\pgfpathlineto{\pgfqpoint{2.556200in}{0.543981in}}%
\pgfpathlineto{\pgfqpoint{2.562400in}{0.545093in}}%
\pgfpathlineto{\pgfqpoint{2.569840in}{0.544838in}}%
\pgfpathlineto{\pgfqpoint{2.574800in}{0.542835in}}%
\pgfpathlineto{\pgfqpoint{2.576040in}{0.542238in}}%
\pgfpathlineto{\pgfqpoint{2.581000in}{0.547111in}}%
\pgfpathlineto{\pgfqpoint{2.583480in}{0.545617in}}%
\pgfpathlineto{\pgfqpoint{2.585960in}{0.546005in}}%
\pgfpathlineto{\pgfqpoint{2.589680in}{0.545713in}}%
\pgfpathlineto{\pgfqpoint{2.590920in}{0.544552in}}%
\pgfpathlineto{\pgfqpoint{2.593400in}{0.544773in}}%
\pgfpathlineto{\pgfqpoint{2.594640in}{0.545004in}}%
\pgfpathlineto{\pgfqpoint{2.597120in}{0.543112in}}%
\pgfpathlineto{\pgfqpoint{2.599600in}{0.545533in}}%
\pgfpathlineto{\pgfqpoint{2.600840in}{0.544558in}}%
\pgfpathlineto{\pgfqpoint{2.602080in}{0.545146in}}%
\pgfpathlineto{\pgfqpoint{2.604560in}{0.543771in}}%
\pgfpathlineto{\pgfqpoint{2.605800in}{0.543521in}}%
\pgfpathlineto{\pgfqpoint{2.609520in}{0.545746in}}%
\pgfpathlineto{\pgfqpoint{2.614480in}{0.544144in}}%
\pgfpathlineto{\pgfqpoint{2.616960in}{0.546128in}}%
\pgfpathlineto{\pgfqpoint{2.619440in}{0.545336in}}%
\pgfpathlineto{\pgfqpoint{2.623160in}{0.547144in}}%
\pgfpathlineto{\pgfqpoint{2.624400in}{0.546711in}}%
\pgfpathlineto{\pgfqpoint{2.628120in}{0.547907in}}%
\pgfpathlineto{\pgfqpoint{2.630600in}{0.548037in}}%
\pgfpathlineto{\pgfqpoint{2.636800in}{0.545542in}}%
\pgfpathlineto{\pgfqpoint{2.644240in}{0.544471in}}%
\pgfpathlineto{\pgfqpoint{2.646720in}{0.545535in}}%
\pgfpathlineto{\pgfqpoint{2.649200in}{0.546623in}}%
\pgfpathlineto{\pgfqpoint{2.650440in}{0.547847in}}%
\pgfpathlineto{\pgfqpoint{2.652920in}{0.548015in}}%
\pgfpathlineto{\pgfqpoint{2.655400in}{0.547551in}}%
\pgfpathlineto{\pgfqpoint{2.656640in}{0.549222in}}%
\pgfpathlineto{\pgfqpoint{2.660360in}{0.547774in}}%
\pgfpathlineto{\pgfqpoint{2.666560in}{0.549921in}}%
\pgfpathlineto{\pgfqpoint{2.671520in}{0.547887in}}%
\pgfpathlineto{\pgfqpoint{2.672760in}{0.548652in}}%
\pgfpathlineto{\pgfqpoint{2.674000in}{0.547987in}}%
\pgfpathlineto{\pgfqpoint{2.675240in}{0.545777in}}%
\pgfpathlineto{\pgfqpoint{2.676480in}{0.546864in}}%
\pgfpathlineto{\pgfqpoint{2.681440in}{0.543114in}}%
\pgfpathlineto{\pgfqpoint{2.686400in}{0.543830in}}%
\pgfpathlineto{\pgfqpoint{2.693840in}{0.542544in}}%
\pgfpathlineto{\pgfqpoint{2.696320in}{0.541650in}}%
\pgfpathlineto{\pgfqpoint{2.700040in}{0.540233in}}%
\pgfpathlineto{\pgfqpoint{2.705000in}{0.544823in}}%
\pgfpathlineto{\pgfqpoint{2.706240in}{0.543611in}}%
\pgfpathlineto{\pgfqpoint{2.711200in}{0.544399in}}%
\pgfpathlineto{\pgfqpoint{2.714920in}{0.543059in}}%
\pgfpathlineto{\pgfqpoint{2.716160in}{0.543725in}}%
\pgfpathlineto{\pgfqpoint{2.721120in}{0.541223in}}%
\pgfpathlineto{\pgfqpoint{2.724840in}{0.542729in}}%
\pgfpathlineto{\pgfqpoint{2.726080in}{0.543206in}}%
\pgfpathlineto{\pgfqpoint{2.728560in}{0.541970in}}%
\pgfpathlineto{\pgfqpoint{2.731040in}{0.542426in}}%
\pgfpathlineto{\pgfqpoint{2.733520in}{0.543853in}}%
\pgfpathlineto{\pgfqpoint{2.736000in}{0.542463in}}%
\pgfpathlineto{\pgfqpoint{2.737240in}{0.541485in}}%
\pgfpathlineto{\pgfqpoint{2.744680in}{0.545035in}}%
\pgfpathlineto{\pgfqpoint{2.745920in}{0.546572in}}%
\pgfpathlineto{\pgfqpoint{2.749640in}{0.546344in}}%
\pgfpathlineto{\pgfqpoint{2.755840in}{0.546102in}}%
\pgfpathlineto{\pgfqpoint{2.758320in}{0.543467in}}%
\pgfpathlineto{\pgfqpoint{2.765760in}{0.543502in}}%
\pgfpathlineto{\pgfqpoint{2.768240in}{0.542917in}}%
\pgfpathlineto{\pgfqpoint{2.770720in}{0.543971in}}%
\pgfpathlineto{\pgfqpoint{2.783120in}{0.545471in}}%
\pgfpathlineto{\pgfqpoint{2.789320in}{0.546604in}}%
\pgfpathlineto{\pgfqpoint{2.791800in}{0.546404in}}%
\pgfpathlineto{\pgfqpoint{2.793040in}{0.546627in}}%
\pgfpathlineto{\pgfqpoint{2.794280in}{0.544835in}}%
\pgfpathlineto{\pgfqpoint{2.798000in}{0.545750in}}%
\pgfpathlineto{\pgfqpoint{2.799240in}{0.543725in}}%
\pgfpathlineto{\pgfqpoint{2.800480in}{0.544934in}}%
\pgfpathlineto{\pgfqpoint{2.805440in}{0.540495in}}%
\pgfpathlineto{\pgfqpoint{2.807920in}{0.540695in}}%
\pgfpathlineto{\pgfqpoint{2.811640in}{0.540925in}}%
\pgfpathlineto{\pgfqpoint{2.814120in}{0.541896in}}%
\pgfpathlineto{\pgfqpoint{2.815360in}{0.542125in}}%
\pgfpathlineto{\pgfqpoint{2.819080in}{0.539703in}}%
\pgfpathlineto{\pgfqpoint{2.822800in}{0.539867in}}%
\pgfpathlineto{\pgfqpoint{2.824040in}{0.539347in}}%
\pgfpathlineto{\pgfqpoint{2.829000in}{0.543779in}}%
\pgfpathlineto{\pgfqpoint{2.830240in}{0.542758in}}%
\pgfpathlineto{\pgfqpoint{2.835200in}{0.544414in}}%
\pgfpathlineto{\pgfqpoint{2.838920in}{0.542895in}}%
\pgfpathlineto{\pgfqpoint{2.840160in}{0.543941in}}%
\pgfpathlineto{\pgfqpoint{2.845120in}{0.541410in}}%
\pgfpathlineto{\pgfqpoint{2.847600in}{0.543538in}}%
\pgfpathlineto{\pgfqpoint{2.851320in}{0.542185in}}%
\pgfpathlineto{\pgfqpoint{2.853800in}{0.541528in}}%
\pgfpathlineto{\pgfqpoint{2.858760in}{0.543410in}}%
\pgfpathlineto{\pgfqpoint{2.862480in}{0.541988in}}%
\pgfpathlineto{\pgfqpoint{2.866200in}{0.543827in}}%
\pgfpathlineto{\pgfqpoint{2.868680in}{0.544813in}}%
\pgfpathlineto{\pgfqpoint{2.869920in}{0.546183in}}%
\pgfpathlineto{\pgfqpoint{2.872400in}{0.545251in}}%
\pgfpathlineto{\pgfqpoint{2.874880in}{0.546196in}}%
\pgfpathlineto{\pgfqpoint{2.879840in}{0.546373in}}%
\pgfpathlineto{\pgfqpoint{2.882320in}{0.543783in}}%
\pgfpathlineto{\pgfqpoint{2.888520in}{0.544156in}}%
\pgfpathlineto{\pgfqpoint{2.892240in}{0.543145in}}%
\pgfpathlineto{\pgfqpoint{2.894720in}{0.544212in}}%
\pgfpathlineto{\pgfqpoint{2.897200in}{0.544770in}}%
\pgfpathlineto{\pgfqpoint{2.899680in}{0.545292in}}%
\pgfpathlineto{\pgfqpoint{2.902160in}{0.545667in}}%
\pgfpathlineto{\pgfqpoint{2.903400in}{0.545397in}}%
\pgfpathlineto{\pgfqpoint{2.904640in}{0.547014in}}%
\pgfpathlineto{\pgfqpoint{2.907120in}{0.544936in}}%
\pgfpathlineto{\pgfqpoint{2.913320in}{0.546913in}}%
\pgfpathlineto{\pgfqpoint{2.917040in}{0.547595in}}%
\pgfpathlineto{\pgfqpoint{2.918280in}{0.545792in}}%
\pgfpathlineto{\pgfqpoint{2.922000in}{0.546406in}}%
\pgfpathlineto{\pgfqpoint{2.923240in}{0.544591in}}%
\pgfpathlineto{\pgfqpoint{2.924480in}{0.545922in}}%
\pgfpathlineto{\pgfqpoint{2.929440in}{0.541937in}}%
\pgfpathlineto{\pgfqpoint{2.931920in}{0.541957in}}%
\pgfpathlineto{\pgfqpoint{2.935640in}{0.541612in}}%
\pgfpathlineto{\pgfqpoint{2.939360in}{0.542677in}}%
\pgfpathlineto{\pgfqpoint{2.943080in}{0.540414in}}%
\pgfpathlineto{\pgfqpoint{2.949280in}{0.541736in}}%
\pgfpathlineto{\pgfqpoint{2.953000in}{0.544492in}}%
\pgfpathlineto{\pgfqpoint{2.954240in}{0.543283in}}%
\pgfpathlineto{\pgfqpoint{2.959200in}{0.545300in}}%
\pgfpathlineto{\pgfqpoint{2.961680in}{0.544798in}}%
\pgfpathlineto{\pgfqpoint{2.962920in}{0.543953in}}%
\pgfpathlineto{\pgfqpoint{2.964160in}{0.545000in}}%
\pgfpathlineto{\pgfqpoint{2.969120in}{0.542340in}}%
\pgfpathlineto{\pgfqpoint{2.971600in}{0.544389in}}%
\pgfpathlineto{\pgfqpoint{2.972840in}{0.543598in}}%
\pgfpathlineto{\pgfqpoint{2.974080in}{0.544193in}}%
\pgfpathlineto{\pgfqpoint{2.977800in}{0.542408in}}%
\pgfpathlineto{\pgfqpoint{2.984000in}{0.543572in}}%
\pgfpathlineto{\pgfqpoint{2.985240in}{0.542376in}}%
\pgfpathlineto{\pgfqpoint{3.000120in}{0.547891in}}%
\pgfpathlineto{\pgfqpoint{3.003840in}{0.547712in}}%
\pgfpathlineto{\pgfqpoint{3.006320in}{0.545373in}}%
\pgfpathlineto{\pgfqpoint{3.008800in}{0.546354in}}%
\pgfpathlineto{\pgfqpoint{3.015000in}{0.543463in}}%
\pgfpathlineto{\pgfqpoint{3.024920in}{0.546652in}}%
\pgfpathlineto{\pgfqpoint{3.027400in}{0.545969in}}%
\pgfpathlineto{\pgfqpoint{3.028640in}{0.547243in}}%
\pgfpathlineto{\pgfqpoint{3.031120in}{0.545432in}}%
\pgfpathlineto{\pgfqpoint{3.041040in}{0.548063in}}%
\pgfpathlineto{\pgfqpoint{3.043520in}{0.546429in}}%
\pgfpathlineto{\pgfqpoint{3.046000in}{0.546976in}}%
\pgfpathlineto{\pgfqpoint{3.047240in}{0.545379in}}%
\pgfpathlineto{\pgfqpoint{3.048480in}{0.546780in}}%
\pgfpathlineto{\pgfqpoint{3.053440in}{0.542678in}}%
\pgfpathlineto{\pgfqpoint{3.057160in}{0.542416in}}%
\pgfpathlineto{\pgfqpoint{3.060880in}{0.542776in}}%
\pgfpathlineto{\pgfqpoint{3.063360in}{0.543275in}}%
\pgfpathlineto{\pgfqpoint{3.067080in}{0.541008in}}%
\pgfpathlineto{\pgfqpoint{3.070800in}{0.541352in}}%
\pgfpathlineto{\pgfqpoint{3.072040in}{0.540670in}}%
\pgfpathlineto{\pgfqpoint{3.077000in}{0.544580in}}%
\pgfpathlineto{\pgfqpoint{3.079480in}{0.543583in}}%
\pgfpathlineto{\pgfqpoint{3.081960in}{0.544172in}}%
\pgfpathlineto{\pgfqpoint{3.083200in}{0.544593in}}%
\pgfpathlineto{\pgfqpoint{3.085680in}{0.543884in}}%
\pgfpathlineto{\pgfqpoint{3.086920in}{0.543148in}}%
\pgfpathlineto{\pgfqpoint{3.088160in}{0.544170in}}%
\pgfpathlineto{\pgfqpoint{3.093120in}{0.541359in}}%
\pgfpathlineto{\pgfqpoint{3.098080in}{0.544124in}}%
\pgfpathlineto{\pgfqpoint{3.101800in}{0.542638in}}%
\pgfpathlineto{\pgfqpoint{3.106760in}{0.544175in}}%
\pgfpathlineto{\pgfqpoint{3.110480in}{0.543609in}}%
\pgfpathlineto{\pgfqpoint{3.117920in}{0.547963in}}%
\pgfpathlineto{\pgfqpoint{3.120400in}{0.547127in}}%
\pgfpathlineto{\pgfqpoint{3.122880in}{0.548302in}}%
\pgfpathlineto{\pgfqpoint{3.127840in}{0.548438in}}%
\pgfpathlineto{\pgfqpoint{3.130320in}{0.546007in}}%
\pgfpathlineto{\pgfqpoint{3.132800in}{0.547158in}}%
\pgfpathlineto{\pgfqpoint{3.139000in}{0.543422in}}%
\pgfpathlineto{\pgfqpoint{3.143960in}{0.545414in}}%
\pgfpathlineto{\pgfqpoint{3.152640in}{0.547614in}}%
\pgfpathlineto{\pgfqpoint{3.156360in}{0.545791in}}%
\pgfpathlineto{\pgfqpoint{3.160080in}{0.547843in}}%
\pgfpathlineto{\pgfqpoint{3.162560in}{0.548598in}}%
\pgfpathlineto{\pgfqpoint{3.167520in}{0.546550in}}%
\pgfpathlineto{\pgfqpoint{3.170000in}{0.547187in}}%
\pgfpathlineto{\pgfqpoint{3.171240in}{0.545706in}}%
\pgfpathlineto{\pgfqpoint{3.172480in}{0.547225in}}%
\pgfpathlineto{\pgfqpoint{3.177440in}{0.543554in}}%
\pgfpathlineto{\pgfqpoint{3.179920in}{0.543125in}}%
\pgfpathlineto{\pgfqpoint{3.183640in}{0.542160in}}%
\pgfpathlineto{\pgfqpoint{3.187360in}{0.542973in}}%
\pgfpathlineto{\pgfqpoint{3.191080in}{0.541212in}}%
\pgfpathlineto{\pgfqpoint{3.193560in}{0.541462in}}%
\pgfpathlineto{\pgfqpoint{3.197280in}{0.541821in}}%
\pgfpathlineto{\pgfqpoint{3.201000in}{0.545348in}}%
\pgfpathlineto{\pgfqpoint{3.202240in}{0.544214in}}%
\pgfpathlineto{\pgfqpoint{3.207200in}{0.545913in}}%
\pgfpathlineto{\pgfqpoint{3.210920in}{0.543899in}}%
\pgfpathlineto{\pgfqpoint{3.212160in}{0.544873in}}%
\pgfpathlineto{\pgfqpoint{3.213400in}{0.544059in}}%
\pgfpathlineto{\pgfqpoint{3.214640in}{0.544624in}}%
\pgfpathlineto{\pgfqpoint{3.217120in}{0.542257in}}%
\pgfpathlineto{\pgfqpoint{3.222080in}{0.544722in}}%
\pgfpathlineto{\pgfqpoint{3.224560in}{0.543771in}}%
\pgfpathlineto{\pgfqpoint{3.227040in}{0.543831in}}%
\pgfpathlineto{\pgfqpoint{3.229520in}{0.545433in}}%
\pgfpathlineto{\pgfqpoint{3.233240in}{0.543889in}}%
\pgfpathlineto{\pgfqpoint{3.235720in}{0.545433in}}%
\pgfpathlineto{\pgfqpoint{3.238200in}{0.547091in}}%
\pgfpathlineto{\pgfqpoint{3.248120in}{0.549311in}}%
\pgfpathlineto{\pgfqpoint{3.251840in}{0.548771in}}%
\pgfpathlineto{\pgfqpoint{3.254320in}{0.546498in}}%
\pgfpathlineto{\pgfqpoint{3.256800in}{0.547775in}}%
\pgfpathlineto{\pgfqpoint{3.263000in}{0.544228in}}%
\pgfpathlineto{\pgfqpoint{3.272920in}{0.548161in}}%
\pgfpathlineto{\pgfqpoint{3.277880in}{0.547863in}}%
\pgfpathlineto{\pgfqpoint{3.280360in}{0.547100in}}%
\pgfpathlineto{\pgfqpoint{3.284080in}{0.549299in}}%
\pgfpathlineto{\pgfqpoint{3.286560in}{0.550007in}}%
\pgfpathlineto{\pgfqpoint{3.291520in}{0.547489in}}%
\pgfpathlineto{\pgfqpoint{3.294000in}{0.548144in}}%
\pgfpathlineto{\pgfqpoint{3.295240in}{0.546521in}}%
\pgfpathlineto{\pgfqpoint{3.296480in}{0.547972in}}%
\pgfpathlineto{\pgfqpoint{3.301440in}{0.544871in}}%
\pgfpathlineto{\pgfqpoint{3.303920in}{0.544406in}}%
\pgfpathlineto{\pgfqpoint{3.307640in}{0.543415in}}%
\pgfpathlineto{\pgfqpoint{3.311360in}{0.544316in}}%
\pgfpathlineto{\pgfqpoint{3.315080in}{0.541985in}}%
\pgfpathlineto{\pgfqpoint{3.318800in}{0.542324in}}%
\pgfpathlineto{\pgfqpoint{3.320040in}{0.541093in}}%
\pgfpathlineto{\pgfqpoint{3.325000in}{0.545641in}}%
\pgfpathlineto{\pgfqpoint{3.326240in}{0.544539in}}%
\pgfpathlineto{\pgfqpoint{3.331200in}{0.546392in}}%
\pgfpathlineto{\pgfqpoint{3.334920in}{0.544532in}}%
\pgfpathlineto{\pgfqpoint{3.336160in}{0.545407in}}%
\pgfpathlineto{\pgfqpoint{3.341120in}{0.542811in}}%
\pgfpathlineto{\pgfqpoint{3.346080in}{0.545431in}}%
\pgfpathlineto{\pgfqpoint{3.349800in}{0.543818in}}%
\pgfpathlineto{\pgfqpoint{3.351040in}{0.544149in}}%
\pgfpathlineto{\pgfqpoint{3.352280in}{0.545881in}}%
\pgfpathlineto{\pgfqpoint{3.358480in}{0.546001in}}%
\pgfpathlineto{\pgfqpoint{3.362200in}{0.548819in}}%
\pgfpathlineto{\pgfqpoint{3.370880in}{0.551511in}}%
\pgfpathlineto{\pgfqpoint{3.375840in}{0.550647in}}%
\pgfpathlineto{\pgfqpoint{3.378320in}{0.548476in}}%
\pgfpathlineto{\pgfqpoint{3.380800in}{0.549889in}}%
\pgfpathlineto{\pgfqpoint{3.387000in}{0.546954in}}%
\pgfpathlineto{\pgfqpoint{3.391960in}{0.548912in}}%
\pgfpathlineto{\pgfqpoint{3.395680in}{0.549919in}}%
\pgfpathlineto{\pgfqpoint{3.398160in}{0.550589in}}%
\pgfpathlineto{\pgfqpoint{3.399400in}{0.550373in}}%
\pgfpathlineto{\pgfqpoint{3.400640in}{0.551469in}}%
\pgfpathlineto{\pgfqpoint{3.404360in}{0.549301in}}%
\pgfpathlineto{\pgfqpoint{3.408080in}{0.551448in}}%
\pgfpathlineto{\pgfqpoint{3.410560in}{0.552103in}}%
\pgfpathlineto{\pgfqpoint{3.415520in}{0.549596in}}%
\pgfpathlineto{\pgfqpoint{3.418000in}{0.550234in}}%
\pgfpathlineto{\pgfqpoint{3.419240in}{0.548661in}}%
\pgfpathlineto{\pgfqpoint{3.420480in}{0.550332in}}%
\pgfpathlineto{\pgfqpoint{3.425440in}{0.546426in}}%
\pgfpathlineto{\pgfqpoint{3.435360in}{0.545752in}}%
\pgfpathlineto{\pgfqpoint{3.437840in}{0.544164in}}%
\pgfpathlineto{\pgfqpoint{3.439080in}{0.543421in}}%
\pgfpathlineto{\pgfqpoint{3.441560in}{0.543995in}}%
\pgfpathlineto{\pgfqpoint{3.445280in}{0.544300in}}%
\pgfpathlineto{\pgfqpoint{3.447760in}{0.546115in}}%
\pgfpathlineto{\pgfqpoint{3.449000in}{0.547620in}}%
\pgfpathlineto{\pgfqpoint{3.451480in}{0.547239in}}%
\pgfpathlineto{\pgfqpoint{3.452720in}{0.548475in}}%
\pgfpathlineto{\pgfqpoint{3.453960in}{0.547811in}}%
\pgfpathlineto{\pgfqpoint{3.456440in}{0.548149in}}%
\pgfpathlineto{\pgfqpoint{3.463880in}{0.546934in}}%
\pgfpathlineto{\pgfqpoint{3.465120in}{0.545687in}}%
\pgfpathlineto{\pgfqpoint{3.470080in}{0.548443in}}%
\pgfpathlineto{\pgfqpoint{3.473800in}{0.546946in}}%
\pgfpathlineto{\pgfqpoint{3.475040in}{0.547214in}}%
\pgfpathlineto{\pgfqpoint{3.476280in}{0.548746in}}%
\pgfpathlineto{\pgfqpoint{3.480000in}{0.547427in}}%
\pgfpathlineto{\pgfqpoint{3.486200in}{0.552963in}}%
\pgfpathlineto{\pgfqpoint{3.488680in}{0.554066in}}%
\pgfpathlineto{\pgfqpoint{3.491160in}{0.554663in}}%
\pgfpathlineto{\pgfqpoint{3.498600in}{0.555652in}}%
\pgfpathlineto{\pgfqpoint{3.511000in}{0.551215in}}%
\pgfpathlineto{\pgfqpoint{3.522160in}{0.555763in}}%
\pgfpathlineto{\pgfqpoint{3.528360in}{0.554120in}}%
\pgfpathlineto{\pgfqpoint{3.532080in}{0.556041in}}%
\pgfpathlineto{\pgfqpoint{3.534560in}{0.557056in}}%
\pgfpathlineto{\pgfqpoint{3.543240in}{0.553951in}}%
\pgfpathlineto{\pgfqpoint{3.544480in}{0.555378in}}%
\pgfpathlineto{\pgfqpoint{3.549440in}{0.551232in}}%
\pgfpathlineto{\pgfqpoint{3.561840in}{0.549453in}}%
\pgfpathlineto{\pgfqpoint{3.563080in}{0.548700in}}%
\pgfpathlineto{\pgfqpoint{3.565560in}{0.549851in}}%
\pgfpathlineto{\pgfqpoint{3.568040in}{0.548425in}}%
\pgfpathlineto{\pgfqpoint{3.573000in}{0.553010in}}%
\pgfpathlineto{\pgfqpoint{3.575480in}{0.552587in}}%
\pgfpathlineto{\pgfqpoint{3.576720in}{0.553723in}}%
\pgfpathlineto{\pgfqpoint{3.577960in}{0.553254in}}%
\pgfpathlineto{\pgfqpoint{3.579200in}{0.554107in}}%
\pgfpathlineto{\pgfqpoint{3.582920in}{0.552316in}}%
\pgfpathlineto{\pgfqpoint{3.584160in}{0.553367in}}%
\pgfpathlineto{\pgfqpoint{3.589120in}{0.551341in}}%
\pgfpathlineto{\pgfqpoint{3.591600in}{0.553542in}}%
\pgfpathlineto{\pgfqpoint{3.592840in}{0.552949in}}%
\pgfpathlineto{\pgfqpoint{3.594080in}{0.553869in}}%
\pgfpathlineto{\pgfqpoint{3.597800in}{0.552470in}}%
\pgfpathlineto{\pgfqpoint{3.599040in}{0.552668in}}%
\pgfpathlineto{\pgfqpoint{3.601520in}{0.554354in}}%
\pgfpathlineto{\pgfqpoint{3.604000in}{0.553418in}}%
\pgfpathlineto{\pgfqpoint{3.607720in}{0.556806in}}%
\pgfpathlineto{\pgfqpoint{3.611440in}{0.558618in}}%
\pgfpathlineto{\pgfqpoint{3.620120in}{0.561581in}}%
\pgfpathlineto{\pgfqpoint{3.622600in}{0.561253in}}%
\pgfpathlineto{\pgfqpoint{3.631280in}{0.558109in}}%
\pgfpathlineto{\pgfqpoint{3.633760in}{0.557380in}}%
\pgfpathlineto{\pgfqpoint{3.635000in}{0.556389in}}%
\pgfpathlineto{\pgfqpoint{3.646160in}{0.561073in}}%
\pgfpathlineto{\pgfqpoint{3.649880in}{0.561035in}}%
\pgfpathlineto{\pgfqpoint{3.652360in}{0.559482in}}%
\pgfpathlineto{\pgfqpoint{3.656080in}{0.561641in}}%
\pgfpathlineto{\pgfqpoint{3.661040in}{0.562620in}}%
\pgfpathlineto{\pgfqpoint{3.663520in}{0.560315in}}%
\pgfpathlineto{\pgfqpoint{3.666000in}{0.561284in}}%
\pgfpathlineto{\pgfqpoint{3.667240in}{0.559654in}}%
\pgfpathlineto{\pgfqpoint{3.668480in}{0.560987in}}%
\pgfpathlineto{\pgfqpoint{3.673440in}{0.557196in}}%
\pgfpathlineto{\pgfqpoint{3.679640in}{0.556363in}}%
\pgfpathlineto{\pgfqpoint{3.692040in}{0.553892in}}%
\pgfpathlineto{\pgfqpoint{3.695760in}{0.557070in}}%
\pgfpathlineto{\pgfqpoint{3.697000in}{0.558468in}}%
\pgfpathlineto{\pgfqpoint{3.699480in}{0.558117in}}%
\pgfpathlineto{\pgfqpoint{3.703200in}{0.559413in}}%
\pgfpathlineto{\pgfqpoint{3.706920in}{0.557599in}}%
\pgfpathlineto{\pgfqpoint{3.708160in}{0.558812in}}%
\pgfpathlineto{\pgfqpoint{3.713120in}{0.556949in}}%
\pgfpathlineto{\pgfqpoint{3.715600in}{0.558918in}}%
\pgfpathlineto{\pgfqpoint{3.716840in}{0.558488in}}%
\pgfpathlineto{\pgfqpoint{3.718080in}{0.559485in}}%
\pgfpathlineto{\pgfqpoint{3.719320in}{0.558455in}}%
\pgfpathlineto{\pgfqpoint{3.720560in}{0.558936in}}%
\pgfpathlineto{\pgfqpoint{3.723040in}{0.558199in}}%
\pgfpathlineto{\pgfqpoint{3.725520in}{0.560047in}}%
\pgfpathlineto{\pgfqpoint{3.728000in}{0.558804in}}%
\pgfpathlineto{\pgfqpoint{3.731720in}{0.562079in}}%
\pgfpathlineto{\pgfqpoint{3.737920in}{0.565560in}}%
\pgfpathlineto{\pgfqpoint{3.740400in}{0.565919in}}%
\pgfpathlineto{\pgfqpoint{3.742880in}{0.567056in}}%
\pgfpathlineto{\pgfqpoint{3.746600in}{0.566944in}}%
\pgfpathlineto{\pgfqpoint{3.751560in}{0.564796in}}%
\pgfpathlineto{\pgfqpoint{3.754040in}{0.564517in}}%
\pgfpathlineto{\pgfqpoint{3.760240in}{0.562626in}}%
\pgfpathlineto{\pgfqpoint{3.762720in}{0.564724in}}%
\pgfpathlineto{\pgfqpoint{3.765200in}{0.564985in}}%
\pgfpathlineto{\pgfqpoint{3.768920in}{0.566396in}}%
\pgfpathlineto{\pgfqpoint{3.771400in}{0.566233in}}%
\pgfpathlineto{\pgfqpoint{3.772640in}{0.567287in}}%
\pgfpathlineto{\pgfqpoint{3.776360in}{0.564118in}}%
\pgfpathlineto{\pgfqpoint{3.781320in}{0.567840in}}%
\pgfpathlineto{\pgfqpoint{3.785040in}{0.568388in}}%
\pgfpathlineto{\pgfqpoint{3.787520in}{0.565856in}}%
\pgfpathlineto{\pgfqpoint{3.790000in}{0.567275in}}%
\pgfpathlineto{\pgfqpoint{3.791240in}{0.565581in}}%
\pgfpathlineto{\pgfqpoint{3.792480in}{0.566998in}}%
\pgfpathlineto{\pgfqpoint{3.796200in}{0.562882in}}%
\pgfpathlineto{\pgfqpoint{3.798680in}{0.562913in}}%
\pgfpathlineto{\pgfqpoint{3.802400in}{0.561223in}}%
\pgfpathlineto{\pgfqpoint{3.807360in}{0.562002in}}%
\pgfpathlineto{\pgfqpoint{3.809840in}{0.559841in}}%
\pgfpathlineto{\pgfqpoint{3.811080in}{0.559052in}}%
\pgfpathlineto{\pgfqpoint{3.813560in}{0.560587in}}%
\pgfpathlineto{\pgfqpoint{3.816040in}{0.559214in}}%
\pgfpathlineto{\pgfqpoint{3.818520in}{0.561497in}}%
\pgfpathlineto{\pgfqpoint{3.824720in}{0.564255in}}%
\pgfpathlineto{\pgfqpoint{3.825960in}{0.563812in}}%
\pgfpathlineto{\pgfqpoint{3.828440in}{0.564527in}}%
\pgfpathlineto{\pgfqpoint{3.833400in}{0.564239in}}%
\pgfpathlineto{\pgfqpoint{3.834640in}{0.565049in}}%
\pgfpathlineto{\pgfqpoint{3.837120in}{0.562929in}}%
\pgfpathlineto{\pgfqpoint{3.839600in}{0.564978in}}%
\pgfpathlineto{\pgfqpoint{3.843320in}{0.564790in}}%
\pgfpathlineto{\pgfqpoint{3.845800in}{0.565024in}}%
\pgfpathlineto{\pgfqpoint{3.847040in}{0.565126in}}%
\pgfpathlineto{\pgfqpoint{3.849520in}{0.566674in}}%
\pgfpathlineto{\pgfqpoint{3.852000in}{0.565322in}}%
\pgfpathlineto{\pgfqpoint{3.860680in}{0.569996in}}%
\pgfpathlineto{\pgfqpoint{3.866880in}{0.572145in}}%
\pgfpathlineto{\pgfqpoint{3.870600in}{0.571608in}}%
\pgfpathlineto{\pgfqpoint{3.874320in}{0.568319in}}%
\pgfpathlineto{\pgfqpoint{3.876800in}{0.569655in}}%
\pgfpathlineto{\pgfqpoint{3.883000in}{0.566441in}}%
\pgfpathlineto{\pgfqpoint{3.892920in}{0.570458in}}%
\pgfpathlineto{\pgfqpoint{3.895400in}{0.570123in}}%
\pgfpathlineto{\pgfqpoint{3.896640in}{0.571229in}}%
\pgfpathlineto{\pgfqpoint{3.900360in}{0.568267in}}%
\pgfpathlineto{\pgfqpoint{3.904080in}{0.570785in}}%
\pgfpathlineto{\pgfqpoint{3.906560in}{0.571923in}}%
\pgfpathlineto{\pgfqpoint{3.909040in}{0.572475in}}%
\pgfpathlineto{\pgfqpoint{3.911520in}{0.569956in}}%
\pgfpathlineto{\pgfqpoint{3.914000in}{0.571009in}}%
\pgfpathlineto{\pgfqpoint{3.915240in}{0.569362in}}%
\pgfpathlineto{\pgfqpoint{3.916480in}{0.570775in}}%
\pgfpathlineto{\pgfqpoint{3.920200in}{0.566976in}}%
\pgfpathlineto{\pgfqpoint{3.922680in}{0.567206in}}%
\pgfpathlineto{\pgfqpoint{3.926400in}{0.565571in}}%
\pgfpathlineto{\pgfqpoint{3.931360in}{0.565516in}}%
\pgfpathlineto{\pgfqpoint{3.933840in}{0.563496in}}%
\pgfpathlineto{\pgfqpoint{3.935080in}{0.562600in}}%
\pgfpathlineto{\pgfqpoint{3.938800in}{0.563572in}}%
\pgfpathlineto{\pgfqpoint{3.940040in}{0.562580in}}%
\pgfpathlineto{\pgfqpoint{3.942520in}{0.564970in}}%
\pgfpathlineto{\pgfqpoint{3.948720in}{0.566891in}}%
\pgfpathlineto{\pgfqpoint{3.957400in}{0.566644in}}%
\pgfpathlineto{\pgfqpoint{3.958640in}{0.566980in}}%
\pgfpathlineto{\pgfqpoint{3.961120in}{0.565088in}}%
\pgfpathlineto{\pgfqpoint{3.963600in}{0.567180in}}%
\pgfpathlineto{\pgfqpoint{3.967320in}{0.566942in}}%
\pgfpathlineto{\pgfqpoint{3.968560in}{0.567769in}}%
\pgfpathlineto{\pgfqpoint{3.971040in}{0.566834in}}%
\pgfpathlineto{\pgfqpoint{3.973520in}{0.568027in}}%
\pgfpathlineto{\pgfqpoint{3.977240in}{0.568166in}}%
\pgfpathlineto{\pgfqpoint{3.984680in}{0.572095in}}%
\pgfpathlineto{\pgfqpoint{3.992120in}{0.574240in}}%
\pgfpathlineto{\pgfqpoint{3.994600in}{0.573612in}}%
\pgfpathlineto{\pgfqpoint{3.998320in}{0.570957in}}%
\pgfpathlineto{\pgfqpoint{4.000800in}{0.572173in}}%
\pgfpathlineto{\pgfqpoint{4.008240in}{0.569180in}}%
\pgfpathlineto{\pgfqpoint{4.010720in}{0.570899in}}%
\pgfpathlineto{\pgfqpoint{4.013200in}{0.571312in}}%
\pgfpathlineto{\pgfqpoint{4.015680in}{0.571725in}}%
\pgfpathlineto{\pgfqpoint{4.020640in}{0.573461in}}%
\pgfpathlineto{\pgfqpoint{4.024360in}{0.570803in}}%
\pgfpathlineto{\pgfqpoint{4.029320in}{0.573915in}}%
\pgfpathlineto{\pgfqpoint{4.033040in}{0.574523in}}%
\pgfpathlineto{\pgfqpoint{4.035520in}{0.572622in}}%
\pgfpathlineto{\pgfqpoint{4.038000in}{0.574097in}}%
\pgfpathlineto{\pgfqpoint{4.039240in}{0.572491in}}%
\pgfpathlineto{\pgfqpoint{4.040480in}{0.573768in}}%
\pgfpathlineto{\pgfqpoint{4.044200in}{0.569263in}}%
\pgfpathlineto{\pgfqpoint{4.046680in}{0.569677in}}%
\pgfpathlineto{\pgfqpoint{4.050400in}{0.567646in}}%
\pgfpathlineto{\pgfqpoint{4.055360in}{0.568196in}}%
\pgfpathlineto{\pgfqpoint{4.057840in}{0.566252in}}%
\pgfpathlineto{\pgfqpoint{4.059080in}{0.565186in}}%
\pgfpathlineto{\pgfqpoint{4.062800in}{0.566613in}}%
\pgfpathlineto{\pgfqpoint{4.064040in}{0.565631in}}%
\pgfpathlineto{\pgfqpoint{4.067760in}{0.568589in}}%
\pgfpathlineto{\pgfqpoint{4.069000in}{0.569910in}}%
\pgfpathlineto{\pgfqpoint{4.071480in}{0.569613in}}%
\pgfpathlineto{\pgfqpoint{4.073960in}{0.570120in}}%
\pgfpathlineto{\pgfqpoint{4.076440in}{0.569465in}}%
\pgfpathlineto{\pgfqpoint{4.078920in}{0.567691in}}%
\pgfpathlineto{\pgfqpoint{4.081400in}{0.569167in}}%
\pgfpathlineto{\pgfqpoint{4.082640in}{0.569760in}}%
\pgfpathlineto{\pgfqpoint{4.086360in}{0.568456in}}%
\pgfpathlineto{\pgfqpoint{4.090080in}{0.570433in}}%
\pgfpathlineto{\pgfqpoint{4.091320in}{0.569252in}}%
\pgfpathlineto{\pgfqpoint{4.092560in}{0.570045in}}%
\pgfpathlineto{\pgfqpoint{4.095040in}{0.568907in}}%
\pgfpathlineto{\pgfqpoint{4.097520in}{0.569851in}}%
\pgfpathlineto{\pgfqpoint{4.101240in}{0.570086in}}%
\pgfpathlineto{\pgfqpoint{4.107440in}{0.573495in}}%
\pgfpathlineto{\pgfqpoint{4.118600in}{0.576202in}}%
\pgfpathlineto{\pgfqpoint{4.122320in}{0.573745in}}%
\pgfpathlineto{\pgfqpoint{4.126040in}{0.573969in}}%
\pgfpathlineto{\pgfqpoint{4.129760in}{0.572143in}}%
\pgfpathlineto{\pgfqpoint{4.132240in}{0.571245in}}%
\pgfpathlineto{\pgfqpoint{4.134720in}{0.573024in}}%
\pgfpathlineto{\pgfqpoint{4.137200in}{0.573103in}}%
\pgfpathlineto{\pgfqpoint{4.138440in}{0.573814in}}%
\pgfpathlineto{\pgfqpoint{4.142160in}{0.572826in}}%
\pgfpathlineto{\pgfqpoint{4.143400in}{0.572470in}}%
\pgfpathlineto{\pgfqpoint{4.144640in}{0.573435in}}%
\pgfpathlineto{\pgfqpoint{4.147120in}{0.570516in}}%
\pgfpathlineto{\pgfqpoint{4.150840in}{0.571694in}}%
\pgfpathlineto{\pgfqpoint{4.154560in}{0.573508in}}%
\pgfpathlineto{\pgfqpoint{4.159520in}{0.572088in}}%
\pgfpathlineto{\pgfqpoint{4.160760in}{0.574248in}}%
\pgfpathlineto{\pgfqpoint{4.162000in}{0.574172in}}%
\pgfpathlineto{\pgfqpoint{4.163240in}{0.572656in}}%
\pgfpathlineto{\pgfqpoint{4.164480in}{0.573915in}}%
\pgfpathlineto{\pgfqpoint{4.168200in}{0.569398in}}%
\pgfpathlineto{\pgfqpoint{4.170680in}{0.569649in}}%
\pgfpathlineto{\pgfqpoint{4.174400in}{0.567618in}}%
\pgfpathlineto{\pgfqpoint{4.179360in}{0.567870in}}%
\pgfpathlineto{\pgfqpoint{4.181840in}{0.565853in}}%
\pgfpathlineto{\pgfqpoint{4.183080in}{0.565039in}}%
\pgfpathlineto{\pgfqpoint{4.186800in}{0.566445in}}%
\pgfpathlineto{\pgfqpoint{4.188040in}{0.565623in}}%
\pgfpathlineto{\pgfqpoint{4.191760in}{0.568414in}}%
\pgfpathlineto{\pgfqpoint{4.193000in}{0.569487in}}%
\pgfpathlineto{\pgfqpoint{4.195480in}{0.569009in}}%
\pgfpathlineto{\pgfqpoint{4.197960in}{0.569920in}}%
\pgfpathlineto{\pgfqpoint{4.200440in}{0.570000in}}%
\pgfpathlineto{\pgfqpoint{4.202920in}{0.567833in}}%
\pgfpathlineto{\pgfqpoint{4.204160in}{0.569714in}}%
\pgfpathlineto{\pgfqpoint{4.210360in}{0.569290in}}%
\pgfpathlineto{\pgfqpoint{4.214080in}{0.571505in}}%
\pgfpathlineto{\pgfqpoint{4.215320in}{0.570270in}}%
\pgfpathlineto{\pgfqpoint{4.216560in}{0.571276in}}%
\pgfpathlineto{\pgfqpoint{4.219040in}{0.569916in}}%
\pgfpathlineto{\pgfqpoint{4.221520in}{0.570529in}}%
\pgfpathlineto{\pgfqpoint{4.225240in}{0.569650in}}%
\pgfpathlineto{\pgfqpoint{4.233920in}{0.573640in}}%
\pgfpathlineto{\pgfqpoint{4.236400in}{0.573407in}}%
\pgfpathlineto{\pgfqpoint{4.240120in}{0.575382in}}%
\pgfpathlineto{\pgfqpoint{4.242600in}{0.575056in}}%
\pgfpathlineto{\pgfqpoint{4.246320in}{0.573104in}}%
\pgfpathlineto{\pgfqpoint{4.250040in}{0.573263in}}%
\pgfpathlineto{\pgfqpoint{4.256240in}{0.570356in}}%
\pgfpathlineto{\pgfqpoint{4.261200in}{0.572679in}}%
\pgfpathlineto{\pgfqpoint{4.266160in}{0.571999in}}%
\pgfpathlineto{\pgfqpoint{4.267400in}{0.571629in}}%
\pgfpathlineto{\pgfqpoint{4.268640in}{0.572450in}}%
\pgfpathlineto{\pgfqpoint{4.271120in}{0.569649in}}%
\pgfpathlineto{\pgfqpoint{4.281040in}{0.572972in}}%
\pgfpathlineto{\pgfqpoint{4.283520in}{0.571927in}}%
\pgfpathlineto{\pgfqpoint{4.286000in}{0.574113in}}%
\pgfpathlineto{\pgfqpoint{4.287240in}{0.572632in}}%
\pgfpathlineto{\pgfqpoint{4.288480in}{0.573729in}}%
\pgfpathlineto{\pgfqpoint{4.292200in}{0.568672in}}%
\pgfpathlineto{\pgfqpoint{4.294680in}{0.568768in}}%
\pgfpathlineto{\pgfqpoint{4.298400in}{0.566931in}}%
\pgfpathlineto{\pgfqpoint{4.303360in}{0.568322in}}%
\pgfpathlineto{\pgfqpoint{4.305840in}{0.566010in}}%
\pgfpathlineto{\pgfqpoint{4.308320in}{0.565427in}}%
\pgfpathlineto{\pgfqpoint{4.310800in}{0.566310in}}%
\pgfpathlineto{\pgfqpoint{4.312040in}{0.565415in}}%
\pgfpathlineto{\pgfqpoint{4.314520in}{0.567122in}}%
\pgfpathlineto{\pgfqpoint{4.323200in}{0.569907in}}%
\pgfpathlineto{\pgfqpoint{4.326920in}{0.566620in}}%
\pgfpathlineto{\pgfqpoint{4.329400in}{0.568451in}}%
\pgfpathlineto{\pgfqpoint{4.331880in}{0.568294in}}%
\pgfpathlineto{\pgfqpoint{4.333120in}{0.567632in}}%
\pgfpathlineto{\pgfqpoint{4.338080in}{0.571045in}}%
\pgfpathlineto{\pgfqpoint{4.339320in}{0.569865in}}%
\pgfpathlineto{\pgfqpoint{4.340560in}{0.570897in}}%
\pgfpathlineto{\pgfqpoint{4.343040in}{0.570132in}}%
\pgfpathlineto{\pgfqpoint{4.344280in}{0.571238in}}%
\pgfpathlineto{\pgfqpoint{4.349240in}{0.569484in}}%
\pgfpathlineto{\pgfqpoint{4.355440in}{0.572274in}}%
\pgfpathlineto{\pgfqpoint{4.357920in}{0.572542in}}%
\pgfpathlineto{\pgfqpoint{4.360400in}{0.572489in}}%
\pgfpathlineto{\pgfqpoint{4.362880in}{0.574733in}}%
\pgfpathlineto{\pgfqpoint{4.366600in}{0.574760in}}%
\pgfpathlineto{\pgfqpoint{4.371560in}{0.572989in}}%
\pgfpathlineto{\pgfqpoint{4.374040in}{0.573096in}}%
\pgfpathlineto{\pgfqpoint{4.377760in}{0.571793in}}%
\pgfpathlineto{\pgfqpoint{4.380240in}{0.570381in}}%
\pgfpathlineto{\pgfqpoint{4.386440in}{0.572841in}}%
\pgfpathlineto{\pgfqpoint{4.388920in}{0.572232in}}%
\pgfpathlineto{\pgfqpoint{4.393880in}{0.570427in}}%
\pgfpathlineto{\pgfqpoint{4.396360in}{0.569720in}}%
\pgfpathlineto{\pgfqpoint{4.402560in}{0.571668in}}%
\pgfpathlineto{\pgfqpoint{4.406280in}{0.570459in}}%
\pgfpathlineto{\pgfqpoint{4.407520in}{0.570869in}}%
\pgfpathlineto{\pgfqpoint{4.410000in}{0.573183in}}%
\pgfpathlineto{\pgfqpoint{4.411240in}{0.571914in}}%
\pgfpathlineto{\pgfqpoint{4.412480in}{0.572671in}}%
\pgfpathlineto{\pgfqpoint{4.416200in}{0.568238in}}%
\pgfpathlineto{\pgfqpoint{4.418680in}{0.568375in}}%
\pgfpathlineto{\pgfqpoint{4.421160in}{0.566551in}}%
\pgfpathlineto{\pgfqpoint{4.423640in}{0.567001in}}%
\pgfpathlineto{\pgfqpoint{4.427360in}{0.567686in}}%
\pgfpathlineto{\pgfqpoint{4.429840in}{0.565131in}}%
\pgfpathlineto{\pgfqpoint{4.431080in}{0.564061in}}%
\pgfpathlineto{\pgfqpoint{4.434800in}{0.566118in}}%
\pgfpathlineto{\pgfqpoint{4.436040in}{0.565162in}}%
\pgfpathlineto{\pgfqpoint{4.439760in}{0.567211in}}%
\pgfpathlineto{\pgfqpoint{4.441000in}{0.568095in}}%
\pgfpathlineto{\pgfqpoint{4.443480in}{0.567112in}}%
\pgfpathlineto{\pgfqpoint{4.445960in}{0.568311in}}%
\pgfpathlineto{\pgfqpoint{4.448440in}{0.568000in}}%
\pgfpathlineto{\pgfqpoint{4.450920in}{0.565677in}}%
\pgfpathlineto{\pgfqpoint{4.453400in}{0.567398in}}%
\pgfpathlineto{\pgfqpoint{4.455880in}{0.567336in}}%
\pgfpathlineto{\pgfqpoint{4.457120in}{0.566976in}}%
\pgfpathlineto{\pgfqpoint{4.462080in}{0.569717in}}%
\pgfpathlineto{\pgfqpoint{4.463320in}{0.568602in}}%
\pgfpathlineto{\pgfqpoint{4.465800in}{0.569154in}}%
\pgfpathlineto{\pgfqpoint{4.470760in}{0.569328in}}%
\pgfpathlineto{\pgfqpoint{4.472000in}{0.568121in}}%
\pgfpathlineto{\pgfqpoint{4.479440in}{0.570715in}}%
\pgfpathlineto{\pgfqpoint{4.481920in}{0.571538in}}%
\pgfpathlineto{\pgfqpoint{4.484400in}{0.572304in}}%
\pgfpathlineto{\pgfqpoint{4.486880in}{0.574209in}}%
\pgfpathlineto{\pgfqpoint{4.490600in}{0.574864in}}%
\pgfpathlineto{\pgfqpoint{4.494320in}{0.573098in}}%
\pgfpathlineto{\pgfqpoint{4.499280in}{0.573196in}}%
\pgfpathlineto{\pgfqpoint{4.501760in}{0.572150in}}%
\pgfpathlineto{\pgfqpoint{4.504240in}{0.571492in}}%
\pgfpathlineto{\pgfqpoint{4.510440in}{0.573477in}}%
\pgfpathlineto{\pgfqpoint{4.512920in}{0.573315in}}%
\pgfpathlineto{\pgfqpoint{4.516640in}{0.573343in}}%
\pgfpathlineto{\pgfqpoint{4.519120in}{0.570666in}}%
\pgfpathlineto{\pgfqpoint{4.529040in}{0.573226in}}%
\pgfpathlineto{\pgfqpoint{4.530280in}{0.571828in}}%
\pgfpathlineto{\pgfqpoint{4.531520in}{0.572378in}}%
\pgfpathlineto{\pgfqpoint{4.532760in}{0.574582in}}%
\pgfpathlineto{\pgfqpoint{4.536480in}{0.573523in}}%
\pgfpathlineto{\pgfqpoint{4.538960in}{0.570223in}}%
\pgfpathlineto{\pgfqpoint{4.541440in}{0.569950in}}%
\pgfpathlineto{\pgfqpoint{4.542680in}{0.569834in}}%
\pgfpathlineto{\pgfqpoint{4.545160in}{0.567581in}}%
\pgfpathlineto{\pgfqpoint{4.547640in}{0.568055in}}%
\pgfpathlineto{\pgfqpoint{4.551360in}{0.569091in}}%
\pgfpathlineto{\pgfqpoint{4.553840in}{0.566174in}}%
\pgfpathlineto{\pgfqpoint{4.555080in}{0.565271in}}%
\pgfpathlineto{\pgfqpoint{4.558800in}{0.566979in}}%
\pgfpathlineto{\pgfqpoint{4.560040in}{0.566255in}}%
\pgfpathlineto{\pgfqpoint{4.565000in}{0.569313in}}%
\pgfpathlineto{\pgfqpoint{4.567480in}{0.568508in}}%
\pgfpathlineto{\pgfqpoint{4.569960in}{0.569470in}}%
\pgfpathlineto{\pgfqpoint{4.572440in}{0.569221in}}%
\pgfpathlineto{\pgfqpoint{4.574920in}{0.566759in}}%
\pgfpathlineto{\pgfqpoint{4.577400in}{0.568349in}}%
\pgfpathlineto{\pgfqpoint{4.582360in}{0.569060in}}%
\pgfpathlineto{\pgfqpoint{4.586080in}{0.570582in}}%
\pgfpathlineto{\pgfqpoint{4.587320in}{0.569982in}}%
\pgfpathlineto{\pgfqpoint{4.589800in}{0.570339in}}%
\pgfpathlineto{\pgfqpoint{4.591040in}{0.570497in}}%
\pgfpathlineto{\pgfqpoint{4.593520in}{0.571629in}}%
\pgfpathlineto{\pgfqpoint{4.598480in}{0.569025in}}%
\pgfpathlineto{\pgfqpoint{4.603440in}{0.570726in}}%
\pgfpathlineto{\pgfqpoint{4.607160in}{0.571568in}}%
\pgfpathlineto{\pgfqpoint{4.614600in}{0.574714in}}%
\pgfpathlineto{\pgfqpoint{4.619560in}{0.573226in}}%
\pgfpathlineto{\pgfqpoint{4.622040in}{0.573578in}}%
\pgfpathlineto{\pgfqpoint{4.627000in}{0.570866in}}%
\pgfpathlineto{\pgfqpoint{4.629480in}{0.571064in}}%
\pgfpathlineto{\pgfqpoint{4.631960in}{0.572627in}}%
\pgfpathlineto{\pgfqpoint{4.635680in}{0.572563in}}%
\pgfpathlineto{\pgfqpoint{4.638160in}{0.572156in}}%
\pgfpathlineto{\pgfqpoint{4.639400in}{0.571291in}}%
\pgfpathlineto{\pgfqpoint{4.640640in}{0.572180in}}%
\pgfpathlineto{\pgfqpoint{4.643120in}{0.569250in}}%
\pgfpathlineto{\pgfqpoint{4.653040in}{0.571570in}}%
\pgfpathlineto{\pgfqpoint{4.654280in}{0.569749in}}%
\pgfpathlineto{\pgfqpoint{4.655520in}{0.570237in}}%
\pgfpathlineto{\pgfqpoint{4.658000in}{0.572530in}}%
\pgfpathlineto{\pgfqpoint{4.659240in}{0.571486in}}%
\pgfpathlineto{\pgfqpoint{4.660480in}{0.572162in}}%
\pgfpathlineto{\pgfqpoint{4.662960in}{0.569116in}}%
\pgfpathlineto{\pgfqpoint{4.665440in}{0.568435in}}%
\pgfpathlineto{\pgfqpoint{4.666680in}{0.568101in}}%
\pgfpathlineto{\pgfqpoint{4.669160in}{0.566553in}}%
\pgfpathlineto{\pgfqpoint{4.675360in}{0.567905in}}%
\pgfpathlineto{\pgfqpoint{4.677840in}{0.565441in}}%
\pgfpathlineto{\pgfqpoint{4.679080in}{0.564384in}}%
\pgfpathlineto{\pgfqpoint{4.682800in}{0.566622in}}%
\pgfpathlineto{\pgfqpoint{4.684040in}{0.565934in}}%
\pgfpathlineto{\pgfqpoint{4.689000in}{0.568576in}}%
\pgfpathlineto{\pgfqpoint{4.690240in}{0.567348in}}%
\pgfpathlineto{\pgfqpoint{4.695200in}{0.568623in}}%
\pgfpathlineto{\pgfqpoint{4.697680in}{0.566846in}}%
\pgfpathlineto{\pgfqpoint{4.698920in}{0.566035in}}%
\pgfpathlineto{\pgfqpoint{4.700160in}{0.567542in}}%
\pgfpathlineto{\pgfqpoint{4.705120in}{0.566432in}}%
\pgfpathlineto{\pgfqpoint{4.710080in}{0.569721in}}%
\pgfpathlineto{\pgfqpoint{4.712560in}{0.569650in}}%
\pgfpathlineto{\pgfqpoint{4.715040in}{0.569672in}}%
\pgfpathlineto{\pgfqpoint{4.717520in}{0.571167in}}%
\pgfpathlineto{\pgfqpoint{4.722480in}{0.567890in}}%
\pgfpathlineto{\pgfqpoint{4.727440in}{0.569532in}}%
\pgfpathlineto{\pgfqpoint{4.729920in}{0.569882in}}%
\pgfpathlineto{\pgfqpoint{4.732400in}{0.571197in}}%
\pgfpathlineto{\pgfqpoint{4.734880in}{0.573351in}}%
\pgfpathlineto{\pgfqpoint{4.738600in}{0.573766in}}%
\pgfpathlineto{\pgfqpoint{4.743560in}{0.571534in}}%
\pgfpathlineto{\pgfqpoint{4.746040in}{0.571366in}}%
\pgfpathlineto{\pgfqpoint{4.748520in}{0.571086in}}%
\pgfpathlineto{\pgfqpoint{4.753480in}{0.570073in}}%
\pgfpathlineto{\pgfqpoint{4.754720in}{0.571593in}}%
\pgfpathlineto{\pgfqpoint{4.765880in}{0.568953in}}%
\pgfpathlineto{\pgfqpoint{4.768360in}{0.567155in}}%
\pgfpathlineto{\pgfqpoint{4.777040in}{0.570005in}}%
\pgfpathlineto{\pgfqpoint{4.778280in}{0.568376in}}%
\pgfpathlineto{\pgfqpoint{4.779520in}{0.569125in}}%
\pgfpathlineto{\pgfqpoint{4.780760in}{0.571200in}}%
\pgfpathlineto{\pgfqpoint{4.784480in}{0.570523in}}%
\pgfpathlineto{\pgfqpoint{4.788200in}{0.565945in}}%
\pgfpathlineto{\pgfqpoint{4.790680in}{0.565579in}}%
\pgfpathlineto{\pgfqpoint{4.793160in}{0.563924in}}%
\pgfpathlineto{\pgfqpoint{4.795640in}{0.564550in}}%
\pgfpathlineto{\pgfqpoint{4.799360in}{0.565109in}}%
\pgfpathlineto{\pgfqpoint{4.801840in}{0.563234in}}%
\pgfpathlineto{\pgfqpoint{4.803080in}{0.562606in}}%
\pgfpathlineto{\pgfqpoint{4.806800in}{0.564817in}}%
\pgfpathlineto{\pgfqpoint{4.808040in}{0.564001in}}%
\pgfpathlineto{\pgfqpoint{4.813000in}{0.566788in}}%
\pgfpathlineto{\pgfqpoint{4.814240in}{0.565967in}}%
\pgfpathlineto{\pgfqpoint{4.817960in}{0.567642in}}%
\pgfpathlineto{\pgfqpoint{4.824160in}{0.565384in}}%
\pgfpathlineto{\pgfqpoint{4.829120in}{0.563941in}}%
\pgfpathlineto{\pgfqpoint{4.832840in}{0.566644in}}%
\pgfpathlineto{\pgfqpoint{4.835320in}{0.566876in}}%
\pgfpathlineto{\pgfqpoint{4.837800in}{0.567258in}}%
\pgfpathlineto{\pgfqpoint{4.839040in}{0.567284in}}%
\pgfpathlineto{\pgfqpoint{4.841520in}{0.568939in}}%
\pgfpathlineto{\pgfqpoint{4.845240in}{0.566297in}}%
\pgfpathlineto{\pgfqpoint{4.847720in}{0.566910in}}%
\pgfpathlineto{\pgfqpoint{4.851440in}{0.567605in}}%
\pgfpathlineto{\pgfqpoint{4.853920in}{0.568168in}}%
\pgfpathlineto{\pgfqpoint{4.856400in}{0.569451in}}%
\pgfpathlineto{\pgfqpoint{4.858880in}{0.572108in}}%
\pgfpathlineto{\pgfqpoint{4.863840in}{0.571402in}}%
\pgfpathlineto{\pgfqpoint{4.867560in}{0.570044in}}%
\pgfpathlineto{\pgfqpoint{4.870040in}{0.570460in}}%
\pgfpathlineto{\pgfqpoint{4.872520in}{0.570162in}}%
\pgfpathlineto{\pgfqpoint{4.877480in}{0.568821in}}%
\pgfpathlineto{\pgfqpoint{4.879960in}{0.569843in}}%
\pgfpathlineto{\pgfqpoint{4.882440in}{0.570105in}}%
\pgfpathlineto{\pgfqpoint{4.887400in}{0.568206in}}%
\pgfpathlineto{\pgfqpoint{4.888640in}{0.569274in}}%
\pgfpathlineto{\pgfqpoint{4.892360in}{0.566100in}}%
\pgfpathlineto{\pgfqpoint{4.894840in}{0.566350in}}%
\pgfpathlineto{\pgfqpoint{4.896080in}{0.566056in}}%
\pgfpathlineto{\pgfqpoint{4.898560in}{0.567905in}}%
\pgfpathlineto{\pgfqpoint{4.903520in}{0.567251in}}%
\pgfpathlineto{\pgfqpoint{4.906000in}{0.569373in}}%
\pgfpathlineto{\pgfqpoint{4.907240in}{0.568503in}}%
\pgfpathlineto{\pgfqpoint{4.908480in}{0.569385in}}%
\pgfpathlineto{\pgfqpoint{4.912200in}{0.565030in}}%
\pgfpathlineto{\pgfqpoint{4.914680in}{0.564470in}}%
\pgfpathlineto{\pgfqpoint{4.917160in}{0.562204in}}%
\pgfpathlineto{\pgfqpoint{4.925840in}{0.561015in}}%
\pgfpathlineto{\pgfqpoint{4.927080in}{0.560335in}}%
\pgfpathlineto{\pgfqpoint{4.930800in}{0.562775in}}%
\pgfpathlineto{\pgfqpoint{4.932040in}{0.562154in}}%
\pgfpathlineto{\pgfqpoint{4.937000in}{0.564211in}}%
\pgfpathlineto{\pgfqpoint{4.939480in}{0.563410in}}%
\pgfpathlineto{\pgfqpoint{4.940720in}{0.564714in}}%
\pgfpathlineto{\pgfqpoint{4.945680in}{0.562381in}}%
\pgfpathlineto{\pgfqpoint{4.946920in}{0.561126in}}%
\pgfpathlineto{\pgfqpoint{4.948160in}{0.562613in}}%
\pgfpathlineto{\pgfqpoint{4.953120in}{0.561355in}}%
\pgfpathlineto{\pgfqpoint{4.955600in}{0.563900in}}%
\pgfpathlineto{\pgfqpoint{4.956840in}{0.563433in}}%
\pgfpathlineto{\pgfqpoint{4.958080in}{0.564177in}}%
\pgfpathlineto{\pgfqpoint{4.960560in}{0.564123in}}%
\pgfpathlineto{\pgfqpoint{4.963040in}{0.564208in}}%
\pgfpathlineto{\pgfqpoint{4.965520in}{0.565883in}}%
\pgfpathlineto{\pgfqpoint{4.970480in}{0.561678in}}%
\pgfpathlineto{\pgfqpoint{4.975440in}{0.562921in}}%
\pgfpathlineto{\pgfqpoint{4.977920in}{0.562494in}}%
\pgfpathlineto{\pgfqpoint{4.979160in}{0.562575in}}%
\pgfpathlineto{\pgfqpoint{4.984120in}{0.567245in}}%
\pgfpathlineto{\pgfqpoint{5.001480in}{0.564353in}}%
\pgfpathlineto{\pgfqpoint{5.003960in}{0.565040in}}%
\pgfpathlineto{\pgfqpoint{5.006440in}{0.564933in}}%
\pgfpathlineto{\pgfqpoint{5.007680in}{0.563776in}}%
\pgfpathlineto{\pgfqpoint{5.010160in}{0.564506in}}%
\pgfpathlineto{\pgfqpoint{5.011400in}{0.563617in}}%
\pgfpathlineto{\pgfqpoint{5.012640in}{0.564813in}}%
\pgfpathlineto{\pgfqpoint{5.016360in}{0.561055in}}%
\pgfpathlineto{\pgfqpoint{5.018840in}{0.561744in}}%
\pgfpathlineto{\pgfqpoint{5.020080in}{0.561413in}}%
\pgfpathlineto{\pgfqpoint{5.022560in}{0.562762in}}%
\pgfpathlineto{\pgfqpoint{5.027520in}{0.561954in}}%
\pgfpathlineto{\pgfqpoint{5.030000in}{0.564681in}}%
\pgfpathlineto{\pgfqpoint{5.031240in}{0.564159in}}%
\pgfpathlineto{\pgfqpoint{5.032480in}{0.564972in}}%
\pgfpathlineto{\pgfqpoint{5.036200in}{0.560394in}}%
\pgfpathlineto{\pgfqpoint{5.038680in}{0.560679in}}%
\pgfpathlineto{\pgfqpoint{5.041160in}{0.558787in}}%
\pgfpathlineto{\pgfqpoint{5.048600in}{0.558162in}}%
\pgfpathlineto{\pgfqpoint{5.051080in}{0.556265in}}%
\pgfpathlineto{\pgfqpoint{5.059760in}{0.559809in}}%
\pgfpathlineto{\pgfqpoint{5.062240in}{0.558500in}}%
\pgfpathlineto{\pgfqpoint{5.063480in}{0.558350in}}%
\pgfpathlineto{\pgfqpoint{5.065960in}{0.559106in}}%
\pgfpathlineto{\pgfqpoint{5.069680in}{0.558089in}}%
\pgfpathlineto{\pgfqpoint{5.070920in}{0.556703in}}%
\pgfpathlineto{\pgfqpoint{5.073400in}{0.558016in}}%
\pgfpathlineto{\pgfqpoint{5.075880in}{0.557862in}}%
\pgfpathlineto{\pgfqpoint{5.077120in}{0.556502in}}%
\pgfpathlineto{\pgfqpoint{5.080840in}{0.558656in}}%
\pgfpathlineto{\pgfqpoint{5.082080in}{0.559803in}}%
\pgfpathlineto{\pgfqpoint{5.084560in}{0.558987in}}%
\pgfpathlineto{\pgfqpoint{5.087040in}{0.559426in}}%
\pgfpathlineto{\pgfqpoint{5.089520in}{0.561637in}}%
\pgfpathlineto{\pgfqpoint{5.092000in}{0.559628in}}%
\pgfpathlineto{\pgfqpoint{5.098200in}{0.561839in}}%
\pgfpathlineto{\pgfqpoint{5.101920in}{0.561373in}}%
\pgfpathlineto{\pgfqpoint{5.103160in}{0.561179in}}%
\pgfpathlineto{\pgfqpoint{5.106880in}{0.565958in}}%
\pgfpathlineto{\pgfqpoint{5.110600in}{0.565728in}}%
\pgfpathlineto{\pgfqpoint{5.119280in}{0.563025in}}%
\pgfpathlineto{\pgfqpoint{5.121760in}{0.562950in}}%
\pgfpathlineto{\pgfqpoint{5.124240in}{0.561311in}}%
\pgfpathlineto{\pgfqpoint{5.129200in}{0.563079in}}%
\pgfpathlineto{\pgfqpoint{5.130440in}{0.563424in}}%
\pgfpathlineto{\pgfqpoint{5.131680in}{0.562468in}}%
\pgfpathlineto{\pgfqpoint{5.134160in}{0.562897in}}%
\pgfpathlineto{\pgfqpoint{5.135400in}{0.561816in}}%
\pgfpathlineto{\pgfqpoint{5.136640in}{0.563008in}}%
\pgfpathlineto{\pgfqpoint{5.140360in}{0.559428in}}%
\pgfpathlineto{\pgfqpoint{5.146560in}{0.561483in}}%
\pgfpathlineto{\pgfqpoint{5.151520in}{0.560244in}}%
\pgfpathlineto{\pgfqpoint{5.154000in}{0.562681in}}%
\pgfpathlineto{\pgfqpoint{5.156480in}{0.563052in}}%
\pgfpathlineto{\pgfqpoint{5.158960in}{0.559728in}}%
\pgfpathlineto{\pgfqpoint{5.161440in}{0.559452in}}%
\pgfpathlineto{\pgfqpoint{5.162680in}{0.559594in}}%
\pgfpathlineto{\pgfqpoint{5.163920in}{0.557982in}}%
\pgfpathlineto{\pgfqpoint{5.168880in}{0.558267in}}%
\pgfpathlineto{\pgfqpoint{5.171360in}{0.557939in}}%
\pgfpathlineto{\pgfqpoint{5.177560in}{0.555548in}}%
\pgfpathlineto{\pgfqpoint{5.178800in}{0.556865in}}%
\pgfpathlineto{\pgfqpoint{5.180040in}{0.556533in}}%
\pgfpathlineto{\pgfqpoint{5.183760in}{0.558874in}}%
\pgfpathlineto{\pgfqpoint{5.185000in}{0.559105in}}%
\pgfpathlineto{\pgfqpoint{5.187480in}{0.557775in}}%
\pgfpathlineto{\pgfqpoint{5.189960in}{0.558483in}}%
\pgfpathlineto{\pgfqpoint{5.193680in}{0.558493in}}%
\pgfpathlineto{\pgfqpoint{5.194920in}{0.556984in}}%
\pgfpathlineto{\pgfqpoint{5.196160in}{0.558371in}}%
\pgfpathlineto{\pgfqpoint{5.198640in}{0.558040in}}%
\pgfpathlineto{\pgfqpoint{5.199880in}{0.558494in}}%
\pgfpathlineto{\pgfqpoint{5.201120in}{0.557455in}}%
\pgfpathlineto{\pgfqpoint{5.206080in}{0.560448in}}%
\pgfpathlineto{\pgfqpoint{5.208560in}{0.559402in}}%
\pgfpathlineto{\pgfqpoint{5.211040in}{0.559299in}}%
\pgfpathlineto{\pgfqpoint{5.213520in}{0.560956in}}%
\pgfpathlineto{\pgfqpoint{5.217240in}{0.559113in}}%
\pgfpathlineto{\pgfqpoint{5.223440in}{0.560468in}}%
\pgfpathlineto{\pgfqpoint{5.227160in}{0.560526in}}%
\pgfpathlineto{\pgfqpoint{5.230880in}{0.564770in}}%
\pgfpathlineto{\pgfqpoint{5.233360in}{0.564433in}}%
\pgfpathlineto{\pgfqpoint{5.248240in}{0.559755in}}%
\pgfpathlineto{\pgfqpoint{5.251960in}{0.560686in}}%
\pgfpathlineto{\pgfqpoint{5.254440in}{0.560428in}}%
\pgfpathlineto{\pgfqpoint{5.256920in}{0.559901in}}%
\pgfpathlineto{\pgfqpoint{5.261880in}{0.559278in}}%
\pgfpathlineto{\pgfqpoint{5.263120in}{0.557729in}}%
\pgfpathlineto{\pgfqpoint{5.273040in}{0.560032in}}%
\pgfpathlineto{\pgfqpoint{5.274280in}{0.558304in}}%
\pgfpathlineto{\pgfqpoint{5.275520in}{0.558546in}}%
\pgfpathlineto{\pgfqpoint{5.278000in}{0.561104in}}%
\pgfpathlineto{\pgfqpoint{5.280480in}{0.561287in}}%
\pgfpathlineto{\pgfqpoint{5.282960in}{0.557533in}}%
\pgfpathlineto{\pgfqpoint{5.285440in}{0.557384in}}%
\pgfpathlineto{\pgfqpoint{5.286680in}{0.557631in}}%
\pgfpathlineto{\pgfqpoint{5.287920in}{0.555883in}}%
\pgfpathlineto{\pgfqpoint{5.296600in}{0.556182in}}%
\pgfpathlineto{\pgfqpoint{5.300320in}{0.554031in}}%
\pgfpathlineto{\pgfqpoint{5.301560in}{0.553962in}}%
\pgfpathlineto{\pgfqpoint{5.306520in}{0.557327in}}%
\pgfpathlineto{\pgfqpoint{5.309000in}{0.557806in}}%
\pgfpathlineto{\pgfqpoint{5.311480in}{0.556801in}}%
\pgfpathlineto{\pgfqpoint{5.312720in}{0.557323in}}%
\pgfpathlineto{\pgfqpoint{5.315200in}{0.557119in}}%
\pgfpathlineto{\pgfqpoint{5.316440in}{0.557765in}}%
\pgfpathlineto{\pgfqpoint{5.317680in}{0.557119in}}%
\pgfpathlineto{\pgfqpoint{5.318920in}{0.555093in}}%
\pgfpathlineto{\pgfqpoint{5.323880in}{0.556363in}}%
\pgfpathlineto{\pgfqpoint{5.325120in}{0.555525in}}%
\pgfpathlineto{\pgfqpoint{5.330080in}{0.559315in}}%
\pgfpathlineto{\pgfqpoint{5.333800in}{0.558058in}}%
\pgfpathlineto{\pgfqpoint{5.337520in}{0.559485in}}%
\pgfpathlineto{\pgfqpoint{5.340000in}{0.558047in}}%
\pgfpathlineto{\pgfqpoint{5.347440in}{0.560306in}}%
\pgfpathlineto{\pgfqpoint{5.351160in}{0.560653in}}%
\pgfpathlineto{\pgfqpoint{5.356120in}{0.565710in}}%
\pgfpathlineto{\pgfqpoint{5.363560in}{0.561569in}}%
\pgfpathlineto{\pgfqpoint{5.366040in}{0.561509in}}%
\pgfpathlineto{\pgfqpoint{5.368520in}{0.560412in}}%
\pgfpathlineto{\pgfqpoint{5.379680in}{0.558513in}}%
\pgfpathlineto{\pgfqpoint{5.385880in}{0.560012in}}%
\pgfpathlineto{\pgfqpoint{5.388360in}{0.558697in}}%
\pgfpathlineto{\pgfqpoint{5.392080in}{0.558972in}}%
\pgfpathlineto{\pgfqpoint{5.394560in}{0.561402in}}%
\pgfpathlineto{\pgfqpoint{5.399520in}{0.560100in}}%
\pgfpathlineto{\pgfqpoint{5.402000in}{0.562387in}}%
\pgfpathlineto{\pgfqpoint{5.404480in}{0.563233in}}%
\pgfpathlineto{\pgfqpoint{5.406960in}{0.560044in}}%
\pgfpathlineto{\pgfqpoint{5.409440in}{0.559818in}}%
\pgfpathlineto{\pgfqpoint{5.410680in}{0.560379in}}%
\pgfpathlineto{\pgfqpoint{5.413160in}{0.558714in}}%
\pgfpathlineto{\pgfqpoint{5.418120in}{0.558142in}}%
\pgfpathlineto{\pgfqpoint{5.419360in}{0.559319in}}%
\pgfpathlineto{\pgfqpoint{5.421840in}{0.557476in}}%
\pgfpathlineto{\pgfqpoint{5.424320in}{0.556061in}}%
\pgfpathlineto{\pgfqpoint{5.425560in}{0.555611in}}%
\pgfpathlineto{\pgfqpoint{5.430520in}{0.559295in}}%
\pgfpathlineto{\pgfqpoint{5.433000in}{0.560371in}}%
\pgfpathlineto{\pgfqpoint{5.434240in}{0.559615in}}%
\pgfpathlineto{\pgfqpoint{5.436720in}{0.560501in}}%
\pgfpathlineto{\pgfqpoint{5.437960in}{0.559042in}}%
\pgfpathlineto{\pgfqpoint{5.441680in}{0.560014in}}%
\pgfpathlineto{\pgfqpoint{5.442920in}{0.558142in}}%
\pgfpathlineto{\pgfqpoint{5.447880in}{0.559391in}}%
\pgfpathlineto{\pgfqpoint{5.449120in}{0.558927in}}%
\pgfpathlineto{\pgfqpoint{5.454080in}{0.562707in}}%
\pgfpathlineto{\pgfqpoint{5.456560in}{0.562131in}}%
\pgfpathlineto{\pgfqpoint{5.461520in}{0.563015in}}%
\pgfpathlineto{\pgfqpoint{5.464000in}{0.561546in}}%
\pgfpathlineto{\pgfqpoint{5.473920in}{0.565728in}}%
\pgfpathlineto{\pgfqpoint{5.475160in}{0.565327in}}%
\pgfpathlineto{\pgfqpoint{5.480120in}{0.570313in}}%
\pgfpathlineto{\pgfqpoint{5.483840in}{0.567429in}}%
\pgfpathlineto{\pgfqpoint{5.486320in}{0.565780in}}%
\pgfpathlineto{\pgfqpoint{5.488800in}{0.565260in}}%
\pgfpathlineto{\pgfqpoint{5.490040in}{0.565169in}}%
\pgfpathlineto{\pgfqpoint{5.492520in}{0.564272in}}%
\pgfpathlineto{\pgfqpoint{5.503680in}{0.561564in}}%
\pgfpathlineto{\pgfqpoint{5.509880in}{0.561493in}}%
\pgfpathlineto{\pgfqpoint{5.511120in}{0.560119in}}%
\pgfpathlineto{\pgfqpoint{5.514840in}{0.561030in}}%
\pgfpathlineto{\pgfqpoint{5.516080in}{0.560260in}}%
\pgfpathlineto{\pgfqpoint{5.518560in}{0.562547in}}%
\pgfpathlineto{\pgfqpoint{5.523520in}{0.560899in}}%
\pgfpathlineto{\pgfqpoint{5.526000in}{0.563715in}}%
\pgfpathlineto{\pgfqpoint{5.528480in}{0.564790in}}%
\pgfpathlineto{\pgfqpoint{5.530960in}{0.561261in}}%
\pgfpathlineto{\pgfqpoint{5.533440in}{0.561196in}}%
\pgfpathlineto{\pgfqpoint{5.534680in}{0.562255in}}%
\pgfpathlineto{\pgfqpoint{5.537160in}{0.559940in}}%
\pgfpathlineto{\pgfqpoint{5.542120in}{0.560285in}}%
\pgfpathlineto{\pgfqpoint{5.543360in}{0.561553in}}%
\pgfpathlineto{\pgfqpoint{5.549560in}{0.557790in}}%
\pgfpathlineto{\pgfqpoint{5.554520in}{0.562325in}}%
\pgfpathlineto{\pgfqpoint{5.557000in}{0.563435in}}%
\pgfpathlineto{\pgfqpoint{5.559480in}{0.563299in}}%
\pgfpathlineto{\pgfqpoint{5.560720in}{0.563363in}}%
\pgfpathlineto{\pgfqpoint{5.563200in}{0.561652in}}%
\pgfpathlineto{\pgfqpoint{5.565680in}{0.562135in}}%
\pgfpathlineto{\pgfqpoint{5.566920in}{0.560978in}}%
\pgfpathlineto{\pgfqpoint{5.571880in}{0.561982in}}%
\pgfpathlineto{\pgfqpoint{5.573120in}{0.561347in}}%
\pgfpathlineto{\pgfqpoint{5.576840in}{0.563430in}}%
\pgfpathlineto{\pgfqpoint{5.578080in}{0.563980in}}%
\pgfpathlineto{\pgfqpoint{5.579320in}{0.563244in}}%
\pgfpathlineto{\pgfqpoint{5.585520in}{0.565253in}}%
\pgfpathlineto{\pgfqpoint{5.588000in}{0.562880in}}%
\pgfpathlineto{\pgfqpoint{5.595440in}{0.566964in}}%
\pgfpathlineto{\pgfqpoint{5.597920in}{0.568378in}}%
\pgfpathlineto{\pgfqpoint{5.599160in}{0.567390in}}%
\pgfpathlineto{\pgfqpoint{5.600400in}{0.568574in}}%
\pgfpathlineto{\pgfqpoint{5.602880in}{0.572340in}}%
\pgfpathlineto{\pgfqpoint{5.604120in}{0.572790in}}%
\pgfpathlineto{\pgfqpoint{5.607840in}{0.569566in}}%
\pgfpathlineto{\pgfqpoint{5.620240in}{0.564591in}}%
\pgfpathlineto{\pgfqpoint{5.622720in}{0.564626in}}%
\pgfpathlineto{\pgfqpoint{5.625200in}{0.563218in}}%
\pgfpathlineto{\pgfqpoint{5.627680in}{0.563223in}}%
\pgfpathlineto{\pgfqpoint{5.633880in}{0.563301in}}%
\pgfpathlineto{\pgfqpoint{5.635120in}{0.562137in}}%
\pgfpathlineto{\pgfqpoint{5.638840in}{0.563687in}}%
\pgfpathlineto{\pgfqpoint{5.640080in}{0.563032in}}%
\pgfpathlineto{\pgfqpoint{5.642560in}{0.565311in}}%
\pgfpathlineto{\pgfqpoint{5.647520in}{0.563055in}}%
\pgfpathlineto{\pgfqpoint{5.650000in}{0.565965in}}%
\pgfpathlineto{\pgfqpoint{5.652480in}{0.567598in}}%
\pgfpathlineto{\pgfqpoint{5.654960in}{0.564237in}}%
\pgfpathlineto{\pgfqpoint{5.656200in}{0.563450in}}%
\pgfpathlineto{\pgfqpoint{5.658680in}{0.565583in}}%
\pgfpathlineto{\pgfqpoint{5.661160in}{0.563458in}}%
\pgfpathlineto{\pgfqpoint{5.667360in}{0.563460in}}%
\pgfpathlineto{\pgfqpoint{5.673560in}{0.557603in}}%
\pgfpathlineto{\pgfqpoint{5.681000in}{0.562586in}}%
\pgfpathlineto{\pgfqpoint{5.683480in}{0.562360in}}%
\pgfpathlineto{\pgfqpoint{5.684720in}{0.562902in}}%
\pgfpathlineto{\pgfqpoint{5.687200in}{0.561253in}}%
\pgfpathlineto{\pgfqpoint{5.689680in}{0.562260in}}%
\pgfpathlineto{\pgfqpoint{5.690920in}{0.561152in}}%
\pgfpathlineto{\pgfqpoint{5.694640in}{0.561844in}}%
\pgfpathlineto{\pgfqpoint{5.698360in}{0.561341in}}%
\pgfpathlineto{\pgfqpoint{5.699600in}{0.562982in}}%
\pgfpathlineto{\pgfqpoint{5.703320in}{0.562741in}}%
\pgfpathlineto{\pgfqpoint{5.709520in}{0.565061in}}%
\pgfpathlineto{\pgfqpoint{5.712000in}{0.563242in}}%
\pgfpathlineto{\pgfqpoint{5.716960in}{0.564564in}}%
\pgfpathlineto{\pgfqpoint{5.718200in}{0.564548in}}%
\pgfpathlineto{\pgfqpoint{5.721920in}{0.567381in}}%
\pgfpathlineto{\pgfqpoint{5.723160in}{0.566039in}}%
\pgfpathlineto{\pgfqpoint{5.724400in}{0.567637in}}%
\pgfpathlineto{\pgfqpoint{5.726880in}{0.572760in}}%
\pgfpathlineto{\pgfqpoint{5.728120in}{0.573147in}}%
\pgfpathlineto{\pgfqpoint{5.730600in}{0.571408in}}%
\pgfpathlineto{\pgfqpoint{5.735560in}{0.568517in}}%
\pgfpathlineto{\pgfqpoint{5.738040in}{0.568044in}}%
\pgfpathlineto{\pgfqpoint{5.740520in}{0.567606in}}%
\pgfpathlineto{\pgfqpoint{5.741760in}{0.567980in}}%
\pgfpathlineto{\pgfqpoint{5.745480in}{0.566228in}}%
\pgfpathlineto{\pgfqpoint{5.747960in}{0.565761in}}%
\pgfpathlineto{\pgfqpoint{5.752920in}{0.566047in}}%
\pgfpathlineto{\pgfqpoint{5.754160in}{0.567353in}}%
\pgfpathlineto{\pgfqpoint{5.759120in}{0.564853in}}%
\pgfpathlineto{\pgfqpoint{5.762840in}{0.567032in}}%
\pgfpathlineto{\pgfqpoint{5.764080in}{0.566394in}}%
\pgfpathlineto{\pgfqpoint{5.766560in}{0.568299in}}%
\pgfpathlineto{\pgfqpoint{5.769040in}{0.567094in}}%
\pgfpathlineto{\pgfqpoint{5.770280in}{0.564987in}}%
\pgfpathlineto{\pgfqpoint{5.771520in}{0.565521in}}%
\pgfpathlineto{\pgfqpoint{5.774000in}{0.568428in}}%
\pgfpathlineto{\pgfqpoint{5.776480in}{0.570753in}}%
\pgfpathlineto{\pgfqpoint{5.780200in}{0.565433in}}%
\pgfpathlineto{\pgfqpoint{5.782680in}{0.566770in}}%
\pgfpathlineto{\pgfqpoint{5.785160in}{0.564119in}}%
\pgfpathlineto{\pgfqpoint{5.788880in}{0.564080in}}%
\pgfpathlineto{\pgfqpoint{5.793840in}{0.560817in}}%
\pgfpathlineto{\pgfqpoint{5.795080in}{0.558791in}}%
\pgfpathlineto{\pgfqpoint{5.798800in}{0.560548in}}%
\pgfpathlineto{\pgfqpoint{5.800040in}{0.560003in}}%
\pgfpathlineto{\pgfqpoint{5.803760in}{0.564462in}}%
\pgfpathlineto{\pgfqpoint{5.805000in}{0.565017in}}%
\pgfpathlineto{\pgfqpoint{5.807480in}{0.564539in}}%
\pgfpathlineto{\pgfqpoint{5.809960in}{0.564293in}}%
\pgfpathlineto{\pgfqpoint{5.811200in}{0.563450in}}%
\pgfpathlineto{\pgfqpoint{5.813680in}{0.564452in}}%
\pgfpathlineto{\pgfqpoint{5.814920in}{0.563533in}}%
\pgfpathlineto{\pgfqpoint{5.816160in}{0.564353in}}%
\pgfpathlineto{\pgfqpoint{5.818640in}{0.564263in}}%
\pgfpathlineto{\pgfqpoint{5.819880in}{0.564302in}}%
\pgfpathlineto{\pgfqpoint{5.822360in}{0.563423in}}%
\pgfpathlineto{\pgfqpoint{5.824840in}{0.564586in}}%
\pgfpathlineto{\pgfqpoint{5.833520in}{0.567374in}}%
\pgfpathlineto{\pgfqpoint{5.836000in}{0.564928in}}%
\pgfpathlineto{\pgfqpoint{5.840960in}{0.567703in}}%
\pgfpathlineto{\pgfqpoint{5.842200in}{0.567900in}}%
\pgfpathlineto{\pgfqpoint{5.845920in}{0.570724in}}%
\pgfpathlineto{\pgfqpoint{5.847160in}{0.569684in}}%
\pgfpathlineto{\pgfqpoint{5.848400in}{0.570702in}}%
\pgfpathlineto{\pgfqpoint{5.850880in}{0.576419in}}%
\pgfpathlineto{\pgfqpoint{5.852120in}{0.577180in}}%
\pgfpathlineto{\pgfqpoint{5.855840in}{0.574112in}}%
\pgfpathlineto{\pgfqpoint{5.857080in}{0.574339in}}%
\pgfpathlineto{\pgfqpoint{5.859560in}{0.572205in}}%
\pgfpathlineto{\pgfqpoint{5.860800in}{0.572512in}}%
\pgfpathlineto{\pgfqpoint{5.863280in}{0.570690in}}%
\pgfpathlineto{\pgfqpoint{5.865760in}{0.572493in}}%
\pgfpathlineto{\pgfqpoint{5.868240in}{0.570092in}}%
\pgfpathlineto{\pgfqpoint{5.870720in}{0.571632in}}%
\pgfpathlineto{\pgfqpoint{5.873200in}{0.571012in}}%
\pgfpathlineto{\pgfqpoint{5.878160in}{0.572015in}}%
\pgfpathlineto{\pgfqpoint{5.879400in}{0.571214in}}%
\pgfpathlineto{\pgfqpoint{5.880640in}{0.572012in}}%
\pgfpathlineto{\pgfqpoint{5.883120in}{0.569951in}}%
\pgfpathlineto{\pgfqpoint{5.884360in}{0.570812in}}%
\pgfpathlineto{\pgfqpoint{5.888080in}{0.569296in}}%
\pgfpathlineto{\pgfqpoint{5.890560in}{0.570816in}}%
\pgfpathlineto{\pgfqpoint{5.893040in}{0.570001in}}%
\pgfpathlineto{\pgfqpoint{5.894280in}{0.567847in}}%
\pgfpathlineto{\pgfqpoint{5.895520in}{0.568432in}}%
\pgfpathlineto{\pgfqpoint{5.898000in}{0.570609in}}%
\pgfpathlineto{\pgfqpoint{5.900480in}{0.572560in}}%
\pgfpathlineto{\pgfqpoint{5.904200in}{0.567381in}}%
\pgfpathlineto{\pgfqpoint{5.906680in}{0.568630in}}%
\pgfpathlineto{\pgfqpoint{5.909160in}{0.565796in}}%
\pgfpathlineto{\pgfqpoint{5.911640in}{0.565643in}}%
\pgfpathlineto{\pgfqpoint{5.916600in}{0.564270in}}%
\pgfpathlineto{\pgfqpoint{5.919080in}{0.560460in}}%
\pgfpathlineto{\pgfqpoint{5.922800in}{0.562970in}}%
\pgfpathlineto{\pgfqpoint{5.924040in}{0.562604in}}%
\pgfpathlineto{\pgfqpoint{5.929000in}{0.567767in}}%
\pgfpathlineto{\pgfqpoint{5.931480in}{0.566910in}}%
\pgfpathlineto{\pgfqpoint{5.936440in}{0.566057in}}%
\pgfpathlineto{\pgfqpoint{5.940160in}{0.567423in}}%
\pgfpathlineto{\pgfqpoint{5.941400in}{0.567010in}}%
\pgfpathlineto{\pgfqpoint{5.943880in}{0.567373in}}%
\pgfpathlineto{\pgfqpoint{5.946360in}{0.566907in}}%
\pgfpathlineto{\pgfqpoint{5.948840in}{0.567885in}}%
\pgfpathlineto{\pgfqpoint{5.955040in}{0.567608in}}%
\pgfpathlineto{\pgfqpoint{5.956280in}{0.569749in}}%
\pgfpathlineto{\pgfqpoint{5.957520in}{0.569613in}}%
\pgfpathlineto{\pgfqpoint{5.960000in}{0.567509in}}%
\pgfpathlineto{\pgfqpoint{5.964960in}{0.571704in}}%
\pgfpathlineto{\pgfqpoint{5.967440in}{0.572997in}}%
\pgfpathlineto{\pgfqpoint{5.972400in}{0.574019in}}%
\pgfpathlineto{\pgfqpoint{5.974880in}{0.579962in}}%
\pgfpathlineto{\pgfqpoint{5.976120in}{0.581228in}}%
\pgfpathlineto{\pgfqpoint{5.979840in}{0.576726in}}%
\pgfpathlineto{\pgfqpoint{5.982320in}{0.575832in}}%
\pgfpathlineto{\pgfqpoint{5.983560in}{0.574912in}}%
\pgfpathlineto{\pgfqpoint{5.984800in}{0.575613in}}%
\pgfpathlineto{\pgfqpoint{5.987280in}{0.573361in}}%
\pgfpathlineto{\pgfqpoint{5.989760in}{0.576059in}}%
\pgfpathlineto{\pgfqpoint{5.992240in}{0.574518in}}%
\pgfpathlineto{\pgfqpoint{5.997200in}{0.576824in}}%
\pgfpathlineto{\pgfqpoint{6.000920in}{0.576653in}}%
\pgfpathlineto{\pgfqpoint{6.002160in}{0.577733in}}%
\pgfpathlineto{\pgfqpoint{6.003400in}{0.576767in}}%
\pgfpathlineto{\pgfqpoint{6.004640in}{0.577694in}}%
\pgfpathlineto{\pgfqpoint{6.008360in}{0.576801in}}%
\pgfpathlineto{\pgfqpoint{6.012080in}{0.576495in}}%
\pgfpathlineto{\pgfqpoint{6.014560in}{0.577899in}}%
\pgfpathlineto{\pgfqpoint{6.018280in}{0.573807in}}%
\pgfpathlineto{\pgfqpoint{6.024480in}{0.578695in}}%
\pgfpathlineto{\pgfqpoint{6.028200in}{0.572701in}}%
\pgfpathlineto{\pgfqpoint{6.030680in}{0.573693in}}%
\pgfpathlineto{\pgfqpoint{6.033160in}{0.571539in}}%
\pgfpathlineto{\pgfqpoint{6.034400in}{0.572190in}}%
\pgfpathlineto{\pgfqpoint{6.038120in}{0.570837in}}%
\pgfpathlineto{\pgfqpoint{6.039360in}{0.571879in}}%
\pgfpathlineto{\pgfqpoint{6.043080in}{0.567178in}}%
\pgfpathlineto{\pgfqpoint{6.045560in}{0.568462in}}%
\pgfpathlineto{\pgfqpoint{6.046800in}{0.569305in}}%
\pgfpathlineto{\pgfqpoint{6.048040in}{0.568398in}}%
\pgfpathlineto{\pgfqpoint{6.055480in}{0.572449in}}%
\pgfpathlineto{\pgfqpoint{6.056720in}{0.572872in}}%
\pgfpathlineto{\pgfqpoint{6.059200in}{0.571996in}}%
\pgfpathlineto{\pgfqpoint{6.064160in}{0.574611in}}%
\pgfpathlineto{\pgfqpoint{6.069120in}{0.572429in}}%
\pgfpathlineto{\pgfqpoint{6.075320in}{0.573718in}}%
\pgfpathlineto{\pgfqpoint{6.079040in}{0.573108in}}%
\pgfpathlineto{\pgfqpoint{6.080280in}{0.575716in}}%
\pgfpathlineto{\pgfqpoint{6.082760in}{0.573990in}}%
\pgfpathlineto{\pgfqpoint{6.085240in}{0.573813in}}%
\pgfpathlineto{\pgfqpoint{6.086480in}{0.574117in}}%
\pgfpathlineto{\pgfqpoint{6.088960in}{0.576177in}}%
\pgfpathlineto{\pgfqpoint{6.093920in}{0.578001in}}%
\pgfpathlineto{\pgfqpoint{6.095160in}{0.577704in}}%
\pgfpathlineto{\pgfqpoint{6.096400in}{0.578637in}}%
\pgfpathlineto{\pgfqpoint{6.098880in}{0.583348in}}%
\pgfpathlineto{\pgfqpoint{6.100120in}{0.584429in}}%
\pgfpathlineto{\pgfqpoint{6.102600in}{0.580837in}}%
\pgfpathlineto{\pgfqpoint{6.106320in}{0.578284in}}%
\pgfpathlineto{\pgfqpoint{6.111280in}{0.576681in}}%
\pgfpathlineto{\pgfqpoint{6.113760in}{0.578493in}}%
\pgfpathlineto{\pgfqpoint{6.117480in}{0.576887in}}%
\pgfpathlineto{\pgfqpoint{6.122440in}{0.578902in}}%
\pgfpathlineto{\pgfqpoint{6.124920in}{0.578343in}}%
\pgfpathlineto{\pgfqpoint{6.126160in}{0.579751in}}%
\pgfpathlineto{\pgfqpoint{6.127400in}{0.578561in}}%
\pgfpathlineto{\pgfqpoint{6.128640in}{0.579884in}}%
\pgfpathlineto{\pgfqpoint{6.131120in}{0.577330in}}%
\pgfpathlineto{\pgfqpoint{6.132360in}{0.579260in}}%
\pgfpathlineto{\pgfqpoint{6.136080in}{0.578234in}}%
\pgfpathlineto{\pgfqpoint{6.138560in}{0.580120in}}%
\pgfpathlineto{\pgfqpoint{6.143520in}{0.576599in}}%
\pgfpathlineto{\pgfqpoint{6.147240in}{0.580556in}}%
\pgfpathlineto{\pgfqpoint{6.148480in}{0.581591in}}%
\pgfpathlineto{\pgfqpoint{6.152200in}{0.575111in}}%
\pgfpathlineto{\pgfqpoint{6.154680in}{0.577131in}}%
\pgfpathlineto{\pgfqpoint{6.157160in}{0.574840in}}%
\pgfpathlineto{\pgfqpoint{6.159640in}{0.574693in}}%
\pgfpathlineto{\pgfqpoint{6.163360in}{0.574379in}}%
\pgfpathlineto{\pgfqpoint{6.165840in}{0.572014in}}%
\pgfpathlineto{\pgfqpoint{6.167080in}{0.569271in}}%
\pgfpathlineto{\pgfqpoint{6.174520in}{0.571800in}}%
\pgfpathlineto{\pgfqpoint{6.177000in}{0.575051in}}%
\pgfpathlineto{\pgfqpoint{6.178240in}{0.574240in}}%
\pgfpathlineto{\pgfqpoint{6.180720in}{0.574964in}}%
\pgfpathlineto{\pgfqpoint{6.183200in}{0.575256in}}%
\pgfpathlineto{\pgfqpoint{6.190640in}{0.576711in}}%
\pgfpathlineto{\pgfqpoint{6.193120in}{0.573722in}}%
\pgfpathlineto{\pgfqpoint{6.200560in}{0.574846in}}%
\pgfpathlineto{\pgfqpoint{6.203040in}{0.573513in}}%
\pgfpathlineto{\pgfqpoint{6.204280in}{0.575714in}}%
\pgfpathlineto{\pgfqpoint{6.206760in}{0.574275in}}%
\pgfpathlineto{\pgfqpoint{6.208000in}{0.572637in}}%
\pgfpathlineto{\pgfqpoint{6.211720in}{0.577042in}}%
\pgfpathlineto{\pgfqpoint{6.214200in}{0.577348in}}%
\pgfpathlineto{\pgfqpoint{6.220400in}{0.578516in}}%
\pgfpathlineto{\pgfqpoint{6.222880in}{0.583807in}}%
\pgfpathlineto{\pgfqpoint{6.224120in}{0.585320in}}%
\pgfpathlineto{\pgfqpoint{6.226600in}{0.580781in}}%
\pgfpathlineto{\pgfqpoint{6.227840in}{0.579766in}}%
\pgfpathlineto{\pgfqpoint{6.230320in}{0.580135in}}%
\pgfpathlineto{\pgfqpoint{6.234040in}{0.579862in}}%
\pgfpathlineto{\pgfqpoint{6.235280in}{0.577137in}}%
\pgfpathlineto{\pgfqpoint{6.236520in}{0.577132in}}%
\pgfpathlineto{\pgfqpoint{6.237760in}{0.578598in}}%
\pgfpathlineto{\pgfqpoint{6.241480in}{0.576283in}}%
\pgfpathlineto{\pgfqpoint{6.246440in}{0.578621in}}%
\pgfpathlineto{\pgfqpoint{6.248920in}{0.577246in}}%
\pgfpathlineto{\pgfqpoint{6.250160in}{0.579347in}}%
\pgfpathlineto{\pgfqpoint{6.251400in}{0.578628in}}%
\pgfpathlineto{\pgfqpoint{6.252640in}{0.580755in}}%
\pgfpathlineto{\pgfqpoint{6.255120in}{0.577439in}}%
\pgfpathlineto{\pgfqpoint{6.256360in}{0.580022in}}%
\pgfpathlineto{\pgfqpoint{6.260080in}{0.578510in}}%
\pgfpathlineto{\pgfqpoint{6.262560in}{0.580545in}}%
\pgfpathlineto{\pgfqpoint{6.267520in}{0.577922in}}%
\pgfpathlineto{\pgfqpoint{6.272480in}{0.582611in}}%
\pgfpathlineto{\pgfqpoint{6.276200in}{0.575714in}}%
\pgfpathlineto{\pgfqpoint{6.278680in}{0.576422in}}%
\pgfpathlineto{\pgfqpoint{6.281160in}{0.574909in}}%
\pgfpathlineto{\pgfqpoint{6.282400in}{0.576227in}}%
\pgfpathlineto{\pgfqpoint{6.284880in}{0.575597in}}%
\pgfpathlineto{\pgfqpoint{6.286120in}{0.573737in}}%
\pgfpathlineto{\pgfqpoint{6.287360in}{0.574290in}}%
\pgfpathlineto{\pgfqpoint{6.292320in}{0.569920in}}%
\pgfpathlineto{\pgfqpoint{6.296040in}{0.570325in}}%
\pgfpathlineto{\pgfqpoint{6.298520in}{0.571081in}}%
\pgfpathlineto{\pgfqpoint{6.301000in}{0.575000in}}%
\pgfpathlineto{\pgfqpoint{6.302240in}{0.574793in}}%
\pgfpathlineto{\pgfqpoint{6.304720in}{0.576189in}}%
\pgfpathlineto{\pgfqpoint{6.308440in}{0.576228in}}%
\pgfpathlineto{\pgfqpoint{6.309680in}{0.577827in}}%
\pgfpathlineto{\pgfqpoint{6.310920in}{0.577235in}}%
\pgfpathlineto{\pgfqpoint{6.314640in}{0.579087in}}%
\pgfpathlineto{\pgfqpoint{6.318360in}{0.574876in}}%
\pgfpathlineto{\pgfqpoint{6.322080in}{0.576599in}}%
\pgfpathlineto{\pgfqpoint{6.323320in}{0.575226in}}%
\pgfpathlineto{\pgfqpoint{6.325800in}{0.576029in}}%
\pgfpathlineto{\pgfqpoint{6.327040in}{0.575843in}}%
\pgfpathlineto{\pgfqpoint{6.328280in}{0.578385in}}%
\pgfpathlineto{\pgfqpoint{6.332000in}{0.575616in}}%
\pgfpathlineto{\pgfqpoint{6.335720in}{0.580441in}}%
\pgfpathlineto{\pgfqpoint{6.338200in}{0.581426in}}%
\pgfpathlineto{\pgfqpoint{6.341920in}{0.583642in}}%
\pgfpathlineto{\pgfqpoint{6.344400in}{0.583378in}}%
\pgfpathlineto{\pgfqpoint{6.346880in}{0.587227in}}%
\pgfpathlineto{\pgfqpoint{6.348120in}{0.589275in}}%
\pgfpathlineto{\pgfqpoint{6.350600in}{0.584026in}}%
\pgfpathlineto{\pgfqpoint{6.353080in}{0.582381in}}%
\pgfpathlineto{\pgfqpoint{6.358040in}{0.581877in}}%
\pgfpathlineto{\pgfqpoint{6.359280in}{0.579155in}}%
\pgfpathlineto{\pgfqpoint{6.360520in}{0.579948in}}%
\pgfpathlineto{\pgfqpoint{6.361760in}{0.582130in}}%
\pgfpathlineto{\pgfqpoint{6.367960in}{0.579545in}}%
\pgfpathlineto{\pgfqpoint{6.370440in}{0.580606in}}%
\pgfpathlineto{\pgfqpoint{6.372920in}{0.577804in}}%
\pgfpathlineto{\pgfqpoint{6.374160in}{0.578956in}}%
\pgfpathlineto{\pgfqpoint{6.375400in}{0.578172in}}%
\pgfpathlineto{\pgfqpoint{6.376640in}{0.580496in}}%
\pgfpathlineto{\pgfqpoint{6.379120in}{0.577627in}}%
\pgfpathlineto{\pgfqpoint{6.380360in}{0.580132in}}%
\pgfpathlineto{\pgfqpoint{6.382840in}{0.578803in}}%
\pgfpathlineto{\pgfqpoint{6.389040in}{0.580708in}}%
\pgfpathlineto{\pgfqpoint{6.390280in}{0.578529in}}%
\pgfpathlineto{\pgfqpoint{6.394000in}{0.581504in}}%
\pgfpathlineto{\pgfqpoint{6.396480in}{0.584485in}}%
\pgfpathlineto{\pgfqpoint{6.400200in}{0.575823in}}%
\pgfpathlineto{\pgfqpoint{6.405160in}{0.574518in}}%
\pgfpathlineto{\pgfqpoint{6.406400in}{0.576850in}}%
\pgfpathlineto{\pgfqpoint{6.408880in}{0.575678in}}%
\pgfpathlineto{\pgfqpoint{6.410120in}{0.574032in}}%
\pgfpathlineto{\pgfqpoint{6.411360in}{0.574741in}}%
\pgfpathlineto{\pgfqpoint{6.415080in}{0.569329in}}%
\pgfpathlineto{\pgfqpoint{6.427480in}{0.578137in}}%
\pgfpathlineto{\pgfqpoint{6.434920in}{0.578581in}}%
\pgfpathlineto{\pgfqpoint{6.438640in}{0.577979in}}%
\pgfpathlineto{\pgfqpoint{6.441120in}{0.573558in}}%
\pgfpathlineto{\pgfqpoint{6.442360in}{0.573749in}}%
\pgfpathlineto{\pgfqpoint{6.444840in}{0.575614in}}%
\pgfpathlineto{\pgfqpoint{6.446080in}{0.576005in}}%
\pgfpathlineto{\pgfqpoint{6.447320in}{0.574598in}}%
\pgfpathlineto{\pgfqpoint{6.449800in}{0.575337in}}%
\pgfpathlineto{\pgfqpoint{6.451040in}{0.575272in}}%
\pgfpathlineto{\pgfqpoint{6.452280in}{0.577971in}}%
\pgfpathlineto{\pgfqpoint{6.456000in}{0.574291in}}%
\pgfpathlineto{\pgfqpoint{6.457240in}{0.571254in}}%
\pgfpathlineto{\pgfqpoint{6.462200in}{0.574450in}}%
\pgfpathlineto{\pgfqpoint{6.464680in}{0.577076in}}%
\pgfpathlineto{\pgfqpoint{6.468400in}{0.575953in}}%
\pgfpathlineto{\pgfqpoint{6.472120in}{0.583580in}}%
\pgfpathlineto{\pgfqpoint{6.474600in}{0.580088in}}%
\pgfpathlineto{\pgfqpoint{6.475840in}{0.579422in}}%
\pgfpathlineto{\pgfqpoint{6.480800in}{0.581768in}}%
\pgfpathlineto{\pgfqpoint{6.482040in}{0.581127in}}%
\pgfpathlineto{\pgfqpoint{6.483280in}{0.578189in}}%
\pgfpathlineto{\pgfqpoint{6.485760in}{0.581563in}}%
\pgfpathlineto{\pgfqpoint{6.491960in}{0.577091in}}%
\pgfpathlineto{\pgfqpoint{6.494440in}{0.578593in}}%
\pgfpathlineto{\pgfqpoint{6.495680in}{0.576005in}}%
\pgfpathlineto{\pgfqpoint{6.499400in}{0.575824in}}%
\pgfpathlineto{\pgfqpoint{6.501880in}{0.578484in}}%
\pgfpathlineto{\pgfqpoint{6.503120in}{0.575644in}}%
\pgfpathlineto{\pgfqpoint{6.504360in}{0.577212in}}%
\pgfpathlineto{\pgfqpoint{6.508080in}{0.574829in}}%
\pgfpathlineto{\pgfqpoint{6.511800in}{0.579149in}}%
\pgfpathlineto{\pgfqpoint{6.513040in}{0.578940in}}%
\pgfpathlineto{\pgfqpoint{6.514280in}{0.576601in}}%
\pgfpathlineto{\pgfqpoint{6.518000in}{0.578104in}}%
\pgfpathlineto{\pgfqpoint{6.520480in}{0.580892in}}%
\pgfpathlineto{\pgfqpoint{6.524200in}{0.570889in}}%
\pgfpathlineto{\pgfqpoint{6.529160in}{0.567057in}}%
\pgfpathlineto{\pgfqpoint{6.531640in}{0.569564in}}%
\pgfpathlineto{\pgfqpoint{6.537840in}{0.564915in}}%
\pgfpathlineto{\pgfqpoint{6.539080in}{0.563560in}}%
\pgfpathlineto{\pgfqpoint{6.541560in}{0.565239in}}%
\pgfpathlineto{\pgfqpoint{6.542800in}{0.567297in}}%
\pgfpathlineto{\pgfqpoint{6.544040in}{0.567001in}}%
\pgfpathlineto{\pgfqpoint{6.549000in}{0.573748in}}%
\pgfpathlineto{\pgfqpoint{6.550240in}{0.573244in}}%
\pgfpathlineto{\pgfqpoint{6.552720in}{0.575163in}}%
\pgfpathlineto{\pgfqpoint{6.556440in}{0.574822in}}%
\pgfpathlineto{\pgfqpoint{6.558920in}{0.575889in}}%
\pgfpathlineto{\pgfqpoint{6.562640in}{0.575304in}}%
\pgfpathlineto{\pgfqpoint{6.566360in}{0.571455in}}%
\pgfpathlineto{\pgfqpoint{6.568840in}{0.572553in}}%
\pgfpathlineto{\pgfqpoint{6.570080in}{0.572906in}}%
\pgfpathlineto{\pgfqpoint{6.571320in}{0.571621in}}%
\pgfpathlineto{\pgfqpoint{6.575040in}{0.572270in}}%
\pgfpathlineto{\pgfqpoint{6.576280in}{0.574721in}}%
\pgfpathlineto{\pgfqpoint{6.580000in}{0.570153in}}%
\pgfpathlineto{\pgfqpoint{6.581240in}{0.569980in}}%
\pgfpathlineto{\pgfqpoint{6.584960in}{0.572429in}}%
\pgfpathlineto{\pgfqpoint{6.587440in}{0.574017in}}%
\pgfpathlineto{\pgfqpoint{6.589920in}{0.576084in}}%
\pgfpathlineto{\pgfqpoint{6.592400in}{0.577040in}}%
\pgfpathlineto{\pgfqpoint{6.594880in}{0.580768in}}%
\pgfpathlineto{\pgfqpoint{6.596120in}{0.582315in}}%
\pgfpathlineto{\pgfqpoint{6.598600in}{0.576808in}}%
\pgfpathlineto{\pgfqpoint{6.601080in}{0.578033in}}%
\pgfpathlineto{\pgfqpoint{6.603560in}{0.579613in}}%
\pgfpathlineto{\pgfqpoint{6.606040in}{0.579303in}}%
\pgfpathlineto{\pgfqpoint{6.607280in}{0.576479in}}%
\pgfpathlineto{\pgfqpoint{6.609760in}{0.581674in}}%
\pgfpathlineto{\pgfqpoint{6.614720in}{0.577197in}}%
\pgfpathlineto{\pgfqpoint{6.615960in}{0.577016in}}%
\pgfpathlineto{\pgfqpoint{6.618440in}{0.578847in}}%
\pgfpathlineto{\pgfqpoint{6.619680in}{0.576150in}}%
\pgfpathlineto{\pgfqpoint{6.622160in}{0.577381in}}%
\pgfpathlineto{\pgfqpoint{6.623400in}{0.575385in}}%
\pgfpathlineto{\pgfqpoint{6.625880in}{0.579288in}}%
\pgfpathlineto{\pgfqpoint{6.627120in}{0.575932in}}%
\pgfpathlineto{\pgfqpoint{6.628360in}{0.576668in}}%
\pgfpathlineto{\pgfqpoint{6.632080in}{0.571567in}}%
\pgfpathlineto{\pgfqpoint{6.634560in}{0.574206in}}%
\pgfpathlineto{\pgfqpoint{6.637040in}{0.573383in}}%
\pgfpathlineto{\pgfqpoint{6.638280in}{0.570807in}}%
\pgfpathlineto{\pgfqpoint{6.643240in}{0.575362in}}%
\pgfpathlineto{\pgfqpoint{6.644480in}{0.574608in}}%
\pgfpathlineto{\pgfqpoint{6.648200in}{0.564387in}}%
\pgfpathlineto{\pgfqpoint{6.651920in}{0.564206in}}%
\pgfpathlineto{\pgfqpoint{6.653160in}{0.563023in}}%
\pgfpathlineto{\pgfqpoint{6.654400in}{0.565347in}}%
\pgfpathlineto{\pgfqpoint{6.656880in}{0.564334in}}%
\pgfpathlineto{\pgfqpoint{6.658120in}{0.562295in}}%
\pgfpathlineto{\pgfqpoint{6.659360in}{0.562902in}}%
\pgfpathlineto{\pgfqpoint{6.664320in}{0.558134in}}%
\pgfpathlineto{\pgfqpoint{6.666800in}{0.561184in}}%
\pgfpathlineto{\pgfqpoint{6.668040in}{0.560137in}}%
\pgfpathlineto{\pgfqpoint{6.670520in}{0.562909in}}%
\pgfpathlineto{\pgfqpoint{6.673000in}{0.568472in}}%
\pgfpathlineto{\pgfqpoint{6.682920in}{0.572439in}}%
\pgfpathlineto{\pgfqpoint{6.687880in}{0.568851in}}%
\pgfpathlineto{\pgfqpoint{6.689120in}{0.565293in}}%
\pgfpathlineto{\pgfqpoint{6.690360in}{0.565483in}}%
\pgfpathlineto{\pgfqpoint{6.692840in}{0.567861in}}%
\pgfpathlineto{\pgfqpoint{6.694080in}{0.568521in}}%
\pgfpathlineto{\pgfqpoint{6.695320in}{0.565666in}}%
\pgfpathlineto{\pgfqpoint{6.699040in}{0.565702in}}%
\pgfpathlineto{\pgfqpoint{6.700280in}{0.568422in}}%
\pgfpathlineto{\pgfqpoint{6.704000in}{0.563654in}}%
\pgfpathlineto{\pgfqpoint{6.706480in}{0.566331in}}%
\pgfpathlineto{\pgfqpoint{6.708960in}{0.565825in}}%
\pgfpathlineto{\pgfqpoint{6.715160in}{0.571359in}}%
\pgfpathlineto{\pgfqpoint{6.716400in}{0.570554in}}%
\pgfpathlineto{\pgfqpoint{6.718880in}{0.575700in}}%
\pgfpathlineto{\pgfqpoint{6.720120in}{0.577128in}}%
\pgfpathlineto{\pgfqpoint{6.722600in}{0.571730in}}%
\pgfpathlineto{\pgfqpoint{6.723840in}{0.572546in}}%
\pgfpathlineto{\pgfqpoint{6.727560in}{0.577093in}}%
\pgfpathlineto{\pgfqpoint{6.730040in}{0.578908in}}%
\pgfpathlineto{\pgfqpoint{6.731280in}{0.577835in}}%
\pgfpathlineto{\pgfqpoint{6.733760in}{0.584527in}}%
\pgfpathlineto{\pgfqpoint{6.735000in}{0.583508in}}%
\pgfpathlineto{\pgfqpoint{6.737480in}{0.579626in}}%
\pgfpathlineto{\pgfqpoint{6.742440in}{0.580441in}}%
\pgfpathlineto{\pgfqpoint{6.743680in}{0.578050in}}%
\pgfpathlineto{\pgfqpoint{6.746160in}{0.582939in}}%
\pgfpathlineto{\pgfqpoint{6.747400in}{0.582024in}}%
\pgfpathlineto{\pgfqpoint{6.749880in}{0.586760in}}%
\pgfpathlineto{\pgfqpoint{6.751120in}{0.584099in}}%
\pgfpathlineto{\pgfqpoint{6.752360in}{0.584352in}}%
\pgfpathlineto{\pgfqpoint{6.756080in}{0.577015in}}%
\pgfpathlineto{\pgfqpoint{6.757320in}{0.576974in}}%
\pgfpathlineto{\pgfqpoint{6.758560in}{0.578781in}}%
\pgfpathlineto{\pgfqpoint{6.761040in}{0.577507in}}%
\pgfpathlineto{\pgfqpoint{6.762280in}{0.574955in}}%
\pgfpathlineto{\pgfqpoint{6.766000in}{0.577314in}}%
\pgfpathlineto{\pgfqpoint{6.767240in}{0.579305in}}%
\pgfpathlineto{\pgfqpoint{6.768480in}{0.576788in}}%
\pgfpathlineto{\pgfqpoint{6.770960in}{0.568391in}}%
\pgfpathlineto{\pgfqpoint{6.773440in}{0.566422in}}%
\pgfpathlineto{\pgfqpoint{6.774680in}{0.566413in}}%
\pgfpathlineto{\pgfqpoint{6.777160in}{0.562807in}}%
\pgfpathlineto{\pgfqpoint{6.779640in}{0.565303in}}%
\pgfpathlineto{\pgfqpoint{6.782120in}{0.564330in}}%
\pgfpathlineto{\pgfqpoint{6.785840in}{0.560043in}}%
\pgfpathlineto{\pgfqpoint{6.787080in}{0.557376in}}%
\pgfpathlineto{\pgfqpoint{6.790800in}{0.561491in}}%
\pgfpathlineto{\pgfqpoint{6.792040in}{0.559492in}}%
\pgfpathlineto{\pgfqpoint{6.795760in}{0.565387in}}%
\pgfpathlineto{\pgfqpoint{6.798240in}{0.567870in}}%
\pgfpathlineto{\pgfqpoint{6.799480in}{0.567756in}}%
\pgfpathlineto{\pgfqpoint{6.801960in}{0.570069in}}%
\pgfpathlineto{\pgfqpoint{6.803200in}{0.569664in}}%
\pgfpathlineto{\pgfqpoint{6.805680in}{0.572372in}}%
\pgfpathlineto{\pgfqpoint{6.806920in}{0.572798in}}%
\pgfpathlineto{\pgfqpoint{6.809400in}{0.570120in}}%
\pgfpathlineto{\pgfqpoint{6.810640in}{0.569685in}}%
\pgfpathlineto{\pgfqpoint{6.814360in}{0.564134in}}%
\pgfpathlineto{\pgfqpoint{6.815600in}{0.566305in}}%
\pgfpathlineto{\pgfqpoint{6.816840in}{0.566199in}}%
\pgfpathlineto{\pgfqpoint{6.818080in}{0.567930in}}%
\pgfpathlineto{\pgfqpoint{6.820560in}{0.567490in}}%
\pgfpathlineto{\pgfqpoint{6.823040in}{0.566277in}}%
\pgfpathlineto{\pgfqpoint{6.824280in}{0.568039in}}%
\pgfpathlineto{\pgfqpoint{6.825520in}{0.567486in}}%
\pgfpathlineto{\pgfqpoint{6.828000in}{0.563276in}}%
\pgfpathlineto{\pgfqpoint{6.829240in}{0.568036in}}%
\pgfpathlineto{\pgfqpoint{6.832960in}{0.562674in}}%
\pgfpathlineto{\pgfqpoint{6.839160in}{0.571050in}}%
\pgfpathlineto{\pgfqpoint{6.840400in}{0.569431in}}%
\pgfpathlineto{\pgfqpoint{6.842880in}{0.575876in}}%
\pgfpathlineto{\pgfqpoint{6.844120in}{0.577881in}}%
\pgfpathlineto{\pgfqpoint{6.845360in}{0.572767in}}%
\pgfpathlineto{\pgfqpoint{6.847840in}{0.573624in}}%
\pgfpathlineto{\pgfqpoint{6.850320in}{0.579099in}}%
\pgfpathlineto{\pgfqpoint{6.851560in}{0.578004in}}%
\pgfpathlineto{\pgfqpoint{6.855280in}{0.576922in}}%
\pgfpathlineto{\pgfqpoint{6.857760in}{0.585058in}}%
\pgfpathlineto{\pgfqpoint{6.859000in}{0.583839in}}%
\pgfpathlineto{\pgfqpoint{6.861480in}{0.579777in}}%
\pgfpathlineto{\pgfqpoint{6.863960in}{0.582025in}}%
\pgfpathlineto{\pgfqpoint{6.865200in}{0.583988in}}%
\pgfpathlineto{\pgfqpoint{6.867680in}{0.578570in}}%
\pgfpathlineto{\pgfqpoint{6.870160in}{0.583921in}}%
\pgfpathlineto{\pgfqpoint{6.871400in}{0.584090in}}%
\pgfpathlineto{\pgfqpoint{6.873880in}{0.590950in}}%
\pgfpathlineto{\pgfqpoint{6.880080in}{0.578172in}}%
\pgfpathlineto{\pgfqpoint{6.881320in}{0.577819in}}%
\pgfpathlineto{\pgfqpoint{6.882560in}{0.578975in}}%
\pgfpathlineto{\pgfqpoint{6.886280in}{0.569929in}}%
\pgfpathlineto{\pgfqpoint{6.888760in}{0.573384in}}%
\pgfpathlineto{\pgfqpoint{6.891240in}{0.577891in}}%
\pgfpathlineto{\pgfqpoint{6.892480in}{0.576637in}}%
\pgfpathlineto{\pgfqpoint{6.894960in}{0.572035in}}%
\pgfpathlineto{\pgfqpoint{6.897440in}{0.571089in}}%
\pgfpathlineto{\pgfqpoint{6.898680in}{0.572721in}}%
\pgfpathlineto{\pgfqpoint{6.901160in}{0.568159in}}%
\pgfpathlineto{\pgfqpoint{6.902400in}{0.569933in}}%
\pgfpathlineto{\pgfqpoint{6.907360in}{0.565536in}}%
\pgfpathlineto{\pgfqpoint{6.909840in}{0.565909in}}%
\pgfpathlineto{\pgfqpoint{6.911080in}{0.563031in}}%
\pgfpathlineto{\pgfqpoint{6.912320in}{0.564079in}}%
\pgfpathlineto{\pgfqpoint{6.913560in}{0.562782in}}%
\pgfpathlineto{\pgfqpoint{6.914800in}{0.565206in}}%
\pgfpathlineto{\pgfqpoint{6.916040in}{0.564319in}}%
\pgfpathlineto{\pgfqpoint{6.923480in}{0.572629in}}%
\pgfpathlineto{\pgfqpoint{6.927200in}{0.570137in}}%
\pgfpathlineto{\pgfqpoint{6.929680in}{0.572691in}}%
\pgfpathlineto{\pgfqpoint{6.933400in}{0.566444in}}%
\pgfpathlineto{\pgfqpoint{6.935880in}{0.565216in}}%
\pgfpathlineto{\pgfqpoint{6.938360in}{0.562325in}}%
\pgfpathlineto{\pgfqpoint{6.939600in}{0.563534in}}%
\pgfpathlineto{\pgfqpoint{6.940840in}{0.561416in}}%
\pgfpathlineto{\pgfqpoint{6.942080in}{0.562402in}}%
\pgfpathlineto{\pgfqpoint{6.944560in}{0.561508in}}%
\pgfpathlineto{\pgfqpoint{6.947040in}{0.564754in}}%
\pgfpathlineto{\pgfqpoint{6.948280in}{0.567958in}}%
\pgfpathlineto{\pgfqpoint{6.949520in}{0.567936in}}%
\pgfpathlineto{\pgfqpoint{6.952000in}{0.560954in}}%
\pgfpathlineto{\pgfqpoint{6.953240in}{0.568909in}}%
\pgfpathlineto{\pgfqpoint{6.955720in}{0.563282in}}%
\pgfpathlineto{\pgfqpoint{6.956960in}{0.559391in}}%
\pgfpathlineto{\pgfqpoint{6.960680in}{0.569415in}}%
\pgfpathlineto{\pgfqpoint{6.961920in}{0.569286in}}%
\pgfpathlineto{\pgfqpoint{6.963160in}{0.572447in}}%
\pgfpathlineto{\pgfqpoint{6.964400in}{0.572553in}}%
\pgfpathlineto{\pgfqpoint{6.966880in}{0.576352in}}%
\pgfpathlineto{\pgfqpoint{6.968120in}{0.577594in}}%
\pgfpathlineto{\pgfqpoint{6.970600in}{0.569373in}}%
\pgfpathlineto{\pgfqpoint{6.973080in}{0.572299in}}%
\pgfpathlineto{\pgfqpoint{6.974320in}{0.575225in}}%
\pgfpathlineto{\pgfqpoint{6.976800in}{0.569610in}}%
\pgfpathlineto{\pgfqpoint{6.979280in}{0.571718in}}%
\pgfpathlineto{\pgfqpoint{6.981760in}{0.578445in}}%
\pgfpathlineto{\pgfqpoint{6.985480in}{0.568123in}}%
\pgfpathlineto{\pgfqpoint{6.987960in}{0.569212in}}%
\pgfpathlineto{\pgfqpoint{6.989200in}{0.574391in}}%
\pgfpathlineto{\pgfqpoint{6.991680in}{0.570419in}}%
\pgfpathlineto{\pgfqpoint{6.997880in}{0.582447in}}%
\pgfpathlineto{\pgfqpoint{6.999120in}{0.579958in}}%
\pgfpathlineto{\pgfqpoint{7.000360in}{0.581977in}}%
\pgfpathlineto{\pgfqpoint{7.002840in}{0.575003in}}%
\pgfpathlineto{\pgfqpoint{7.005320in}{0.572147in}}%
\pgfpathlineto{\pgfqpoint{7.006560in}{0.572546in}}%
\pgfpathlineto{\pgfqpoint{7.007800in}{0.569013in}}%
\pgfpathlineto{\pgfqpoint{7.009040in}{0.570420in}}%
\pgfpathlineto{\pgfqpoint{7.010280in}{0.565849in}}%
\pgfpathlineto{\pgfqpoint{7.014000in}{0.573171in}}%
\pgfpathlineto{\pgfqpoint{7.015240in}{0.576946in}}%
\pgfpathlineto{\pgfqpoint{7.016480in}{0.574563in}}%
\pgfpathlineto{\pgfqpoint{7.018960in}{0.568068in}}%
\pgfpathlineto{\pgfqpoint{7.022680in}{0.572992in}}%
\pgfpathlineto{\pgfqpoint{7.023920in}{0.570610in}}%
\pgfpathlineto{\pgfqpoint{7.026400in}{0.573573in}}%
\pgfpathlineto{\pgfqpoint{7.030120in}{0.566959in}}%
\pgfpathlineto{\pgfqpoint{7.033840in}{0.563919in}}%
\pgfpathlineto{\pgfqpoint{7.037560in}{0.563556in}}%
\pgfpathlineto{\pgfqpoint{7.040040in}{0.570937in}}%
\pgfpathlineto{\pgfqpoint{7.041280in}{0.573523in}}%
\pgfpathlineto{\pgfqpoint{7.042520in}{0.573405in}}%
\pgfpathlineto{\pgfqpoint{7.048720in}{0.580895in}}%
\pgfpathlineto{\pgfqpoint{7.052440in}{0.577979in}}%
\pgfpathlineto{\pgfqpoint{7.053680in}{0.579558in}}%
\pgfpathlineto{\pgfqpoint{7.058640in}{0.567782in}}%
\pgfpathlineto{\pgfqpoint{7.059880in}{0.568395in}}%
\pgfpathlineto{\pgfqpoint{7.062360in}{0.563523in}}%
\pgfpathlineto{\pgfqpoint{7.063600in}{0.568032in}}%
\pgfpathlineto{\pgfqpoint{7.066080in}{0.564782in}}%
\pgfpathlineto{\pgfqpoint{7.067320in}{0.565655in}}%
\pgfpathlineto{\pgfqpoint{7.069800in}{0.562599in}}%
\pgfpathlineto{\pgfqpoint{7.071040in}{0.564397in}}%
\pgfpathlineto{\pgfqpoint{7.073520in}{0.570839in}}%
\pgfpathlineto{\pgfqpoint{7.074760in}{0.569699in}}%
\pgfpathlineto{\pgfqpoint{7.076000in}{0.565342in}}%
\pgfpathlineto{\pgfqpoint{7.077240in}{0.580177in}}%
\pgfpathlineto{\pgfqpoint{7.080960in}{0.568593in}}%
\pgfpathlineto{\pgfqpoint{7.083440in}{0.571294in}}%
\pgfpathlineto{\pgfqpoint{7.088400in}{0.581423in}}%
\pgfpathlineto{\pgfqpoint{7.090880in}{0.581433in}}%
\pgfpathlineto{\pgfqpoint{7.092120in}{0.586666in}}%
\pgfpathlineto{\pgfqpoint{7.094600in}{0.575103in}}%
\pgfpathlineto{\pgfqpoint{7.097080in}{0.578631in}}%
\pgfpathlineto{\pgfqpoint{7.098320in}{0.587615in}}%
\pgfpathlineto{\pgfqpoint{7.099560in}{0.586761in}}%
\pgfpathlineto{\pgfqpoint{7.100800in}{0.588365in}}%
\pgfpathlineto{\pgfqpoint{7.103280in}{0.588300in}}%
\pgfpathlineto{\pgfqpoint{7.104520in}{0.594651in}}%
\pgfpathlineto{\pgfqpoint{7.105760in}{0.593411in}}%
\pgfpathlineto{\pgfqpoint{7.109480in}{0.577758in}}%
\pgfpathlineto{\pgfqpoint{7.111960in}{0.581900in}}%
\pgfpathlineto{\pgfqpoint{7.113200in}{0.590427in}}%
\pgfpathlineto{\pgfqpoint{7.116920in}{0.579617in}}%
\pgfpathlineto{\pgfqpoint{7.118160in}{0.580791in}}%
\pgfpathlineto{\pgfqpoint{7.119400in}{0.578897in}}%
\pgfpathlineto{\pgfqpoint{7.121880in}{0.582107in}}%
\pgfpathlineto{\pgfqpoint{7.123120in}{0.581557in}}%
\pgfpathlineto{\pgfqpoint{7.124360in}{0.583818in}}%
\pgfpathlineto{\pgfqpoint{7.125600in}{0.576437in}}%
\pgfpathlineto{\pgfqpoint{7.128080in}{0.583461in}}%
\pgfpathlineto{\pgfqpoint{7.130560in}{0.584730in}}%
\pgfpathlineto{\pgfqpoint{7.134280in}{0.574089in}}%
\pgfpathlineto{\pgfqpoint{7.135520in}{0.574853in}}%
\pgfpathlineto{\pgfqpoint{7.139240in}{0.594400in}}%
\pgfpathlineto{\pgfqpoint{7.142960in}{0.576026in}}%
\pgfpathlineto{\pgfqpoint{7.146680in}{0.572148in}}%
\pgfpathlineto{\pgfqpoint{7.147920in}{0.565975in}}%
\pgfpathlineto{\pgfqpoint{7.150400in}{0.568341in}}%
\pgfpathlineto{\pgfqpoint{7.151640in}{0.566444in}}%
\pgfpathlineto{\pgfqpoint{7.152880in}{0.561357in}}%
\pgfpathlineto{\pgfqpoint{7.154120in}{0.561276in}}%
\pgfpathlineto{\pgfqpoint{7.156600in}{0.572542in}}%
\pgfpathlineto{\pgfqpoint{7.157840in}{0.574691in}}%
\pgfpathlineto{\pgfqpoint{7.159080in}{0.573555in}}%
\pgfpathlineto{\pgfqpoint{7.164040in}{0.589746in}}%
\pgfpathlineto{\pgfqpoint{7.165280in}{0.591567in}}%
\pgfpathlineto{\pgfqpoint{7.167760in}{0.588272in}}%
\pgfpathlineto{\pgfqpoint{7.169000in}{0.588432in}}%
\pgfpathlineto{\pgfqpoint{7.170240in}{0.594836in}}%
\pgfpathlineto{\pgfqpoint{7.171480in}{0.593080in}}%
\pgfpathlineto{\pgfqpoint{7.175200in}{0.598749in}}%
\pgfpathlineto{\pgfqpoint{7.181400in}{0.577817in}}%
\pgfpathlineto{\pgfqpoint{7.182640in}{0.575998in}}%
\pgfpathlineto{\pgfqpoint{7.183880in}{0.583065in}}%
\pgfpathlineto{\pgfqpoint{7.185120in}{0.581884in}}%
\pgfpathlineto{\pgfqpoint{7.186360in}{0.579085in}}%
\pgfpathlineto{\pgfqpoint{7.188840in}{0.589686in}}%
\pgfpathlineto{\pgfqpoint{7.191320in}{0.588228in}}%
\pgfpathlineto{\pgfqpoint{7.193800in}{0.583056in}}%
\pgfpathlineto{\pgfqpoint{7.197520in}{0.590614in}}%
\pgfpathlineto{\pgfqpoint{7.198760in}{0.591301in}}%
\pgfpathlineto{\pgfqpoint{7.200000in}{0.588987in}}%
\pgfpathlineto{\pgfqpoint{7.200000in}{0.588987in}}%
\pgfusepath{stroke}%
\end{pgfscope}%
\begin{pgfscope}%
\pgfpathrectangle{\pgfqpoint{1.000000in}{0.300000in}}{\pgfqpoint{6.200000in}{2.400000in}} %
\pgfusepath{clip}%
\pgfsetrectcap%
\pgfsetroundjoin%
\pgfsetlinewidth{2.007500pt}%
\definecolor{currentstroke}{rgb}{0.000000,0.500000,0.000000}%
\pgfsetstrokecolor{currentstroke}%
\pgfsetdash{}{0pt}%
\pgfpathmoveto{\pgfqpoint{2.241240in}{0.561243in}}%
\pgfpathlineto{\pgfqpoint{6.578760in}{0.561243in}}%
\pgfusepath{stroke}%
\end{pgfscope}%
\begin{pgfscope}%
\pgfpathrectangle{\pgfqpoint{1.000000in}{0.300000in}}{\pgfqpoint{6.200000in}{2.400000in}} %
\pgfusepath{clip}%
\pgfsetrectcap%
\pgfsetroundjoin%
\pgfsetlinewidth{1.003750pt}%
\definecolor{currentstroke}{rgb}{1.000000,0.000000,0.000000}%
\pgfsetstrokecolor{currentstroke}%
\pgfsetdash{}{0pt}%
\pgfpathmoveto{\pgfqpoint{1.001240in}{1.520756in}}%
\pgfpathlineto{\pgfqpoint{1.002480in}{2.151600in}}%
\pgfpathlineto{\pgfqpoint{1.003720in}{2.421570in}}%
\pgfpathlineto{\pgfqpoint{1.004960in}{2.474556in}}%
\pgfpathlineto{\pgfqpoint{1.008680in}{2.422400in}}%
\pgfpathlineto{\pgfqpoint{1.014880in}{2.305800in}}%
\pgfpathlineto{\pgfqpoint{1.017360in}{2.279517in}}%
\pgfpathlineto{\pgfqpoint{1.023560in}{2.216103in}}%
\pgfpathlineto{\pgfqpoint{1.024800in}{2.217151in}}%
\pgfpathlineto{\pgfqpoint{1.027280in}{2.214651in}}%
\pgfpathlineto{\pgfqpoint{1.029760in}{2.199063in}}%
\pgfpathlineto{\pgfqpoint{1.031000in}{2.199534in}}%
\pgfpathlineto{\pgfqpoint{1.032240in}{2.198756in}}%
\pgfpathlineto{\pgfqpoint{1.033480in}{2.195888in}}%
\pgfpathlineto{\pgfqpoint{1.038440in}{2.146624in}}%
\pgfpathlineto{\pgfqpoint{1.042160in}{2.128131in}}%
\pgfpathlineto{\pgfqpoint{1.044640in}{2.134889in}}%
\pgfpathlineto{\pgfqpoint{1.045880in}{2.135387in}}%
\pgfpathlineto{\pgfqpoint{1.050840in}{2.112683in}}%
\pgfpathlineto{\pgfqpoint{1.052080in}{2.111175in}}%
\pgfpathlineto{\pgfqpoint{1.057040in}{2.081005in}}%
\pgfpathlineto{\pgfqpoint{1.059520in}{2.067852in}}%
\pgfpathlineto{\pgfqpoint{1.060760in}{2.069741in}}%
\pgfpathlineto{\pgfqpoint{1.062000in}{2.068107in}}%
\pgfpathlineto{\pgfqpoint{1.066960in}{2.050979in}}%
\pgfpathlineto{\pgfqpoint{1.068200in}{2.052679in}}%
\pgfpathlineto{\pgfqpoint{1.069440in}{2.056667in}}%
\pgfpathlineto{\pgfqpoint{1.070680in}{2.054558in}}%
\pgfpathlineto{\pgfqpoint{1.073160in}{2.040640in}}%
\pgfpathlineto{\pgfqpoint{1.074400in}{2.039363in}}%
\pgfpathlineto{\pgfqpoint{1.075640in}{2.036423in}}%
\pgfpathlineto{\pgfqpoint{1.076880in}{2.037233in}}%
\pgfpathlineto{\pgfqpoint{1.078120in}{2.033545in}}%
\pgfpathlineto{\pgfqpoint{1.079360in}{2.036306in}}%
\pgfpathlineto{\pgfqpoint{1.081840in}{2.030061in}}%
\pgfpathlineto{\pgfqpoint{1.084320in}{2.036471in}}%
\pgfpathlineto{\pgfqpoint{1.085560in}{2.035317in}}%
\pgfpathlineto{\pgfqpoint{1.090520in}{2.026234in}}%
\pgfpathlineto{\pgfqpoint{1.091760in}{2.026078in}}%
\pgfpathlineto{\pgfqpoint{1.093000in}{2.027099in}}%
\pgfpathlineto{\pgfqpoint{1.094240in}{2.031135in}}%
\pgfpathlineto{\pgfqpoint{1.096720in}{2.020080in}}%
\pgfpathlineto{\pgfqpoint{1.100440in}{2.011638in}}%
\pgfpathlineto{\pgfqpoint{1.101680in}{2.012650in}}%
\pgfpathlineto{\pgfqpoint{1.102920in}{2.013960in}}%
\pgfpathlineto{\pgfqpoint{1.104160in}{2.012974in}}%
\pgfpathlineto{\pgfqpoint{1.105400in}{2.016486in}}%
\pgfpathlineto{\pgfqpoint{1.106640in}{2.014088in}}%
\pgfpathlineto{\pgfqpoint{1.110360in}{2.002342in}}%
\pgfpathlineto{\pgfqpoint{1.114080in}{2.002454in}}%
\pgfpathlineto{\pgfqpoint{1.119040in}{2.014454in}}%
\pgfpathlineto{\pgfqpoint{1.120280in}{2.015611in}}%
\pgfpathlineto{\pgfqpoint{1.122760in}{2.004709in}}%
\pgfpathlineto{\pgfqpoint{1.126480in}{1.987451in}}%
\pgfpathlineto{\pgfqpoint{1.128960in}{1.988641in}}%
\pgfpathlineto{\pgfqpoint{1.131440in}{1.987572in}}%
\pgfpathlineto{\pgfqpoint{1.133920in}{1.983136in}}%
\pgfpathlineto{\pgfqpoint{1.135160in}{1.983295in}}%
\pgfpathlineto{\pgfqpoint{1.137640in}{1.980144in}}%
\pgfpathlineto{\pgfqpoint{1.140120in}{1.987418in}}%
\pgfpathlineto{\pgfqpoint{1.141360in}{1.986292in}}%
\pgfpathlineto{\pgfqpoint{1.142600in}{1.982899in}}%
\pgfpathlineto{\pgfqpoint{1.145080in}{1.982613in}}%
\pgfpathlineto{\pgfqpoint{1.146320in}{1.988158in}}%
\pgfpathlineto{\pgfqpoint{1.148800in}{1.985259in}}%
\pgfpathlineto{\pgfqpoint{1.152520in}{1.986706in}}%
\pgfpathlineto{\pgfqpoint{1.155000in}{1.986492in}}%
\pgfpathlineto{\pgfqpoint{1.157480in}{1.993858in}}%
\pgfpathlineto{\pgfqpoint{1.161200in}{1.990386in}}%
\pgfpathlineto{\pgfqpoint{1.162440in}{1.991033in}}%
\pgfpathlineto{\pgfqpoint{1.166160in}{1.983045in}}%
\pgfpathlineto{\pgfqpoint{1.167400in}{1.988145in}}%
\pgfpathlineto{\pgfqpoint{1.168640in}{1.988477in}}%
\pgfpathlineto{\pgfqpoint{1.169880in}{1.992308in}}%
\pgfpathlineto{\pgfqpoint{1.172360in}{1.988978in}}%
\pgfpathlineto{\pgfqpoint{1.174840in}{1.986216in}}%
\pgfpathlineto{\pgfqpoint{1.176080in}{1.987092in}}%
\pgfpathlineto{\pgfqpoint{1.177320in}{1.985694in}}%
\pgfpathlineto{\pgfqpoint{1.179800in}{1.977940in}}%
\pgfpathlineto{\pgfqpoint{1.182280in}{1.977043in}}%
\pgfpathlineto{\pgfqpoint{1.183520in}{1.972120in}}%
\pgfpathlineto{\pgfqpoint{1.188480in}{1.972000in}}%
\pgfpathlineto{\pgfqpoint{1.189720in}{1.974306in}}%
\pgfpathlineto{\pgfqpoint{1.190960in}{1.973345in}}%
\pgfpathlineto{\pgfqpoint{1.192200in}{1.978010in}}%
\pgfpathlineto{\pgfqpoint{1.193440in}{1.978302in}}%
\pgfpathlineto{\pgfqpoint{1.194680in}{1.980091in}}%
\pgfpathlineto{\pgfqpoint{1.195920in}{1.978442in}}%
\pgfpathlineto{\pgfqpoint{1.197160in}{1.979010in}}%
\pgfpathlineto{\pgfqpoint{1.199640in}{1.976240in}}%
\pgfpathlineto{\pgfqpoint{1.200880in}{1.978343in}}%
\pgfpathlineto{\pgfqpoint{1.205840in}{1.976027in}}%
\pgfpathlineto{\pgfqpoint{1.207080in}{1.979808in}}%
\pgfpathlineto{\pgfqpoint{1.209560in}{1.979890in}}%
\pgfpathlineto{\pgfqpoint{1.213280in}{1.983638in}}%
\pgfpathlineto{\pgfqpoint{1.214520in}{1.980925in}}%
\pgfpathlineto{\pgfqpoint{1.217000in}{1.983662in}}%
\pgfpathlineto{\pgfqpoint{1.218240in}{1.986722in}}%
\pgfpathlineto{\pgfqpoint{1.219480in}{1.984620in}}%
\pgfpathlineto{\pgfqpoint{1.220720in}{1.985378in}}%
\pgfpathlineto{\pgfqpoint{1.223200in}{1.980909in}}%
\pgfpathlineto{\pgfqpoint{1.224440in}{1.976513in}}%
\pgfpathlineto{\pgfqpoint{1.225680in}{1.976337in}}%
\pgfpathlineto{\pgfqpoint{1.228160in}{1.973899in}}%
\pgfpathlineto{\pgfqpoint{1.229400in}{1.975080in}}%
\pgfpathlineto{\pgfqpoint{1.230640in}{1.974246in}}%
\pgfpathlineto{\pgfqpoint{1.234360in}{1.962994in}}%
\pgfpathlineto{\pgfqpoint{1.235600in}{1.963111in}}%
\pgfpathlineto{\pgfqpoint{1.236840in}{1.965345in}}%
\pgfpathlineto{\pgfqpoint{1.240560in}{1.961842in}}%
\pgfpathlineto{\pgfqpoint{1.244280in}{1.975389in}}%
\pgfpathlineto{\pgfqpoint{1.245520in}{1.974846in}}%
\pgfpathlineto{\pgfqpoint{1.250480in}{1.953127in}}%
\pgfpathlineto{\pgfqpoint{1.255440in}{1.955115in}}%
\pgfpathlineto{\pgfqpoint{1.256680in}{1.954304in}}%
\pgfpathlineto{\pgfqpoint{1.261640in}{1.958224in}}%
\pgfpathlineto{\pgfqpoint{1.264120in}{1.964359in}}%
\pgfpathlineto{\pgfqpoint{1.265360in}{1.963580in}}%
\pgfpathlineto{\pgfqpoint{1.269080in}{1.957610in}}%
\pgfpathlineto{\pgfqpoint{1.270320in}{1.959796in}}%
\pgfpathlineto{\pgfqpoint{1.274040in}{1.958840in}}%
\pgfpathlineto{\pgfqpoint{1.275280in}{1.960320in}}%
\pgfpathlineto{\pgfqpoint{1.277760in}{1.958012in}}%
\pgfpathlineto{\pgfqpoint{1.279000in}{1.955040in}}%
\pgfpathlineto{\pgfqpoint{1.280240in}{1.955827in}}%
\pgfpathlineto{\pgfqpoint{1.281480in}{1.953866in}}%
\pgfpathlineto{\pgfqpoint{1.283960in}{1.956008in}}%
\pgfpathlineto{\pgfqpoint{1.285200in}{1.955204in}}%
\pgfpathlineto{\pgfqpoint{1.286440in}{1.956038in}}%
\pgfpathlineto{\pgfqpoint{1.287680in}{1.953437in}}%
\pgfpathlineto{\pgfqpoint{1.290160in}{1.954245in}}%
\pgfpathlineto{\pgfqpoint{1.291400in}{1.956961in}}%
\pgfpathlineto{\pgfqpoint{1.292640in}{1.954885in}}%
\pgfpathlineto{\pgfqpoint{1.293880in}{1.956090in}}%
\pgfpathlineto{\pgfqpoint{1.297600in}{1.953489in}}%
\pgfpathlineto{\pgfqpoint{1.301320in}{1.954540in}}%
\pgfpathlineto{\pgfqpoint{1.302560in}{1.953268in}}%
\pgfpathlineto{\pgfqpoint{1.305040in}{1.955577in}}%
\pgfpathlineto{\pgfqpoint{1.306280in}{1.954334in}}%
\pgfpathlineto{\pgfqpoint{1.308760in}{1.947118in}}%
\pgfpathlineto{\pgfqpoint{1.312480in}{1.948684in}}%
\pgfpathlineto{\pgfqpoint{1.318680in}{1.961555in}}%
\pgfpathlineto{\pgfqpoint{1.319920in}{1.959535in}}%
\pgfpathlineto{\pgfqpoint{1.321160in}{1.960551in}}%
\pgfpathlineto{\pgfqpoint{1.323640in}{1.958582in}}%
\pgfpathlineto{\pgfqpoint{1.326120in}{1.959069in}}%
\pgfpathlineto{\pgfqpoint{1.327360in}{1.958566in}}%
\pgfpathlineto{\pgfqpoint{1.328600in}{1.955849in}}%
\pgfpathlineto{\pgfqpoint{1.329840in}{1.956608in}}%
\pgfpathlineto{\pgfqpoint{1.332320in}{1.959174in}}%
\pgfpathlineto{\pgfqpoint{1.333560in}{1.957008in}}%
\pgfpathlineto{\pgfqpoint{1.337280in}{1.960474in}}%
\pgfpathlineto{\pgfqpoint{1.338520in}{1.958077in}}%
\pgfpathlineto{\pgfqpoint{1.342240in}{1.962071in}}%
\pgfpathlineto{\pgfqpoint{1.343480in}{1.959135in}}%
\pgfpathlineto{\pgfqpoint{1.344720in}{1.959626in}}%
\pgfpathlineto{\pgfqpoint{1.345960in}{1.958896in}}%
\pgfpathlineto{\pgfqpoint{1.347200in}{1.956692in}}%
\pgfpathlineto{\pgfqpoint{1.349680in}{1.949466in}}%
\pgfpathlineto{\pgfqpoint{1.352160in}{1.946782in}}%
\pgfpathlineto{\pgfqpoint{1.354640in}{1.948485in}}%
\pgfpathlineto{\pgfqpoint{1.357120in}{1.945435in}}%
\pgfpathlineto{\pgfqpoint{1.358360in}{1.942153in}}%
\pgfpathlineto{\pgfqpoint{1.359600in}{1.942262in}}%
\pgfpathlineto{\pgfqpoint{1.360840in}{1.945819in}}%
\pgfpathlineto{\pgfqpoint{1.362080in}{1.945044in}}%
\pgfpathlineto{\pgfqpoint{1.364560in}{1.941533in}}%
\pgfpathlineto{\pgfqpoint{1.368280in}{1.951361in}}%
\pgfpathlineto{\pgfqpoint{1.369520in}{1.951093in}}%
\pgfpathlineto{\pgfqpoint{1.373240in}{1.938884in}}%
\pgfpathlineto{\pgfqpoint{1.378200in}{1.940041in}}%
\pgfpathlineto{\pgfqpoint{1.381920in}{1.938329in}}%
\pgfpathlineto{\pgfqpoint{1.383160in}{1.939862in}}%
\pgfpathlineto{\pgfqpoint{1.384400in}{1.938750in}}%
\pgfpathlineto{\pgfqpoint{1.388120in}{1.944618in}}%
\pgfpathlineto{\pgfqpoint{1.389360in}{1.944914in}}%
\pgfpathlineto{\pgfqpoint{1.393080in}{1.940283in}}%
\pgfpathlineto{\pgfqpoint{1.394320in}{1.942379in}}%
\pgfpathlineto{\pgfqpoint{1.396800in}{1.937936in}}%
\pgfpathlineto{\pgfqpoint{1.400520in}{1.940897in}}%
\pgfpathlineto{\pgfqpoint{1.401760in}{1.940332in}}%
\pgfpathlineto{\pgfqpoint{1.403000in}{1.937544in}}%
\pgfpathlineto{\pgfqpoint{1.404240in}{1.938893in}}%
\pgfpathlineto{\pgfqpoint{1.405480in}{1.937195in}}%
\pgfpathlineto{\pgfqpoint{1.407960in}{1.939608in}}%
\pgfpathlineto{\pgfqpoint{1.412920in}{1.933313in}}%
\pgfpathlineto{\pgfqpoint{1.415400in}{1.937625in}}%
\pgfpathlineto{\pgfqpoint{1.417880in}{1.935160in}}%
\pgfpathlineto{\pgfqpoint{1.421600in}{1.931634in}}%
\pgfpathlineto{\pgfqpoint{1.424080in}{1.931735in}}%
\pgfpathlineto{\pgfqpoint{1.425320in}{1.933664in}}%
\pgfpathlineto{\pgfqpoint{1.426560in}{1.933307in}}%
\pgfpathlineto{\pgfqpoint{1.429040in}{1.935890in}}%
\pgfpathlineto{\pgfqpoint{1.430280in}{1.935069in}}%
\pgfpathlineto{\pgfqpoint{1.432760in}{1.930003in}}%
\pgfpathlineto{\pgfqpoint{1.434000in}{1.929663in}}%
\pgfpathlineto{\pgfqpoint{1.436480in}{1.927854in}}%
\pgfpathlineto{\pgfqpoint{1.440200in}{1.935881in}}%
\pgfpathlineto{\pgfqpoint{1.442680in}{1.939988in}}%
\pgfpathlineto{\pgfqpoint{1.445160in}{1.937611in}}%
\pgfpathlineto{\pgfqpoint{1.447640in}{1.937156in}}%
\pgfpathlineto{\pgfqpoint{1.450120in}{1.936902in}}%
\pgfpathlineto{\pgfqpoint{1.453840in}{1.933030in}}%
\pgfpathlineto{\pgfqpoint{1.455080in}{1.934466in}}%
\pgfpathlineto{\pgfqpoint{1.458800in}{1.932505in}}%
\pgfpathlineto{\pgfqpoint{1.460040in}{1.934190in}}%
\pgfpathlineto{\pgfqpoint{1.461280in}{1.933429in}}%
\pgfpathlineto{\pgfqpoint{1.462520in}{1.929884in}}%
\pgfpathlineto{\pgfqpoint{1.466240in}{1.933097in}}%
\pgfpathlineto{\pgfqpoint{1.468720in}{1.929136in}}%
\pgfpathlineto{\pgfqpoint{1.469960in}{1.929223in}}%
\pgfpathlineto{\pgfqpoint{1.476160in}{1.921412in}}%
\pgfpathlineto{\pgfqpoint{1.477400in}{1.921841in}}%
\pgfpathlineto{\pgfqpoint{1.478640in}{1.923489in}}%
\pgfpathlineto{\pgfqpoint{1.481120in}{1.921214in}}%
\pgfpathlineto{\pgfqpoint{1.482360in}{1.918006in}}%
\pgfpathlineto{\pgfqpoint{1.483600in}{1.919282in}}%
\pgfpathlineto{\pgfqpoint{1.486080in}{1.923220in}}%
\pgfpathlineto{\pgfqpoint{1.488560in}{1.919041in}}%
\pgfpathlineto{\pgfqpoint{1.491040in}{1.922828in}}%
\pgfpathlineto{\pgfqpoint{1.492280in}{1.927092in}}%
\pgfpathlineto{\pgfqpoint{1.493520in}{1.925282in}}%
\pgfpathlineto{\pgfqpoint{1.497240in}{1.912059in}}%
\pgfpathlineto{\pgfqpoint{1.499720in}{1.914315in}}%
\pgfpathlineto{\pgfqpoint{1.500960in}{1.914059in}}%
\pgfpathlineto{\pgfqpoint{1.503440in}{1.911450in}}%
\pgfpathlineto{\pgfqpoint{1.505920in}{1.912745in}}%
\pgfpathlineto{\pgfqpoint{1.507160in}{1.914561in}}%
\pgfpathlineto{\pgfqpoint{1.508400in}{1.912748in}}%
\pgfpathlineto{\pgfqpoint{1.513360in}{1.915545in}}%
\pgfpathlineto{\pgfqpoint{1.515840in}{1.911731in}}%
\pgfpathlineto{\pgfqpoint{1.517080in}{1.911852in}}%
\pgfpathlineto{\pgfqpoint{1.518320in}{1.913695in}}%
\pgfpathlineto{\pgfqpoint{1.520800in}{1.909862in}}%
\pgfpathlineto{\pgfqpoint{1.524520in}{1.911726in}}%
\pgfpathlineto{\pgfqpoint{1.525760in}{1.911698in}}%
\pgfpathlineto{\pgfqpoint{1.527000in}{1.909746in}}%
\pgfpathlineto{\pgfqpoint{1.528240in}{1.911184in}}%
\pgfpathlineto{\pgfqpoint{1.529480in}{1.910804in}}%
\pgfpathlineto{\pgfqpoint{1.531960in}{1.913136in}}%
\pgfpathlineto{\pgfqpoint{1.533200in}{1.912320in}}%
\pgfpathlineto{\pgfqpoint{1.535680in}{1.909403in}}%
\pgfpathlineto{\pgfqpoint{1.538160in}{1.909606in}}%
\pgfpathlineto{\pgfqpoint{1.539400in}{1.911709in}}%
\pgfpathlineto{\pgfqpoint{1.540640in}{1.909664in}}%
\pgfpathlineto{\pgfqpoint{1.541880in}{1.910435in}}%
\pgfpathlineto{\pgfqpoint{1.544360in}{1.908342in}}%
\pgfpathlineto{\pgfqpoint{1.546840in}{1.909070in}}%
\pgfpathlineto{\pgfqpoint{1.548080in}{1.909633in}}%
\pgfpathlineto{\pgfqpoint{1.549320in}{1.911758in}}%
\pgfpathlineto{\pgfqpoint{1.550560in}{1.911091in}}%
\pgfpathlineto{\pgfqpoint{1.553040in}{1.912377in}}%
\pgfpathlineto{\pgfqpoint{1.560480in}{1.907355in}}%
\pgfpathlineto{\pgfqpoint{1.564200in}{1.915980in}}%
\pgfpathlineto{\pgfqpoint{1.566680in}{1.920170in}}%
\pgfpathlineto{\pgfqpoint{1.569160in}{1.915221in}}%
\pgfpathlineto{\pgfqpoint{1.574120in}{1.913957in}}%
\pgfpathlineto{\pgfqpoint{1.577840in}{1.910224in}}%
\pgfpathlineto{\pgfqpoint{1.580320in}{1.912726in}}%
\pgfpathlineto{\pgfqpoint{1.581560in}{1.910519in}}%
\pgfpathlineto{\pgfqpoint{1.585280in}{1.912768in}}%
\pgfpathlineto{\pgfqpoint{1.586520in}{1.909791in}}%
\pgfpathlineto{\pgfqpoint{1.589000in}{1.911621in}}%
\pgfpathlineto{\pgfqpoint{1.590240in}{1.913819in}}%
\pgfpathlineto{\pgfqpoint{1.592720in}{1.910060in}}%
\pgfpathlineto{\pgfqpoint{1.595200in}{1.910463in}}%
\pgfpathlineto{\pgfqpoint{1.597680in}{1.905459in}}%
\pgfpathlineto{\pgfqpoint{1.598920in}{1.903902in}}%
\pgfpathlineto{\pgfqpoint{1.603880in}{1.905293in}}%
\pgfpathlineto{\pgfqpoint{1.606360in}{1.900691in}}%
\pgfpathlineto{\pgfqpoint{1.607600in}{1.901977in}}%
\pgfpathlineto{\pgfqpoint{1.610080in}{1.905294in}}%
\pgfpathlineto{\pgfqpoint{1.612560in}{1.899676in}}%
\pgfpathlineto{\pgfqpoint{1.613800in}{1.900431in}}%
\pgfpathlineto{\pgfqpoint{1.616280in}{1.906135in}}%
\pgfpathlineto{\pgfqpoint{1.617520in}{1.904248in}}%
\pgfpathlineto{\pgfqpoint{1.620000in}{1.895733in}}%
\pgfpathlineto{\pgfqpoint{1.624960in}{1.896702in}}%
\pgfpathlineto{\pgfqpoint{1.628680in}{1.891865in}}%
\pgfpathlineto{\pgfqpoint{1.629920in}{1.891969in}}%
\pgfpathlineto{\pgfqpoint{1.631160in}{1.893608in}}%
\pgfpathlineto{\pgfqpoint{1.632400in}{1.892122in}}%
\pgfpathlineto{\pgfqpoint{1.636120in}{1.895304in}}%
\pgfpathlineto{\pgfqpoint{1.637360in}{1.895694in}}%
\pgfpathlineto{\pgfqpoint{1.639840in}{1.891869in}}%
\pgfpathlineto{\pgfqpoint{1.641080in}{1.891300in}}%
\pgfpathlineto{\pgfqpoint{1.642320in}{1.893119in}}%
\pgfpathlineto{\pgfqpoint{1.646040in}{1.889013in}}%
\pgfpathlineto{\pgfqpoint{1.648520in}{1.889123in}}%
\pgfpathlineto{\pgfqpoint{1.649760in}{1.889924in}}%
\pgfpathlineto{\pgfqpoint{1.651000in}{1.888842in}}%
\pgfpathlineto{\pgfqpoint{1.655960in}{1.894573in}}%
\pgfpathlineto{\pgfqpoint{1.658440in}{1.890927in}}%
\pgfpathlineto{\pgfqpoint{1.660920in}{1.891395in}}%
\pgfpathlineto{\pgfqpoint{1.662160in}{1.891827in}}%
\pgfpathlineto{\pgfqpoint{1.663400in}{1.893936in}}%
\pgfpathlineto{\pgfqpoint{1.664640in}{1.892504in}}%
\pgfpathlineto{\pgfqpoint{1.665880in}{1.893048in}}%
\pgfpathlineto{\pgfqpoint{1.668360in}{1.890616in}}%
\pgfpathlineto{\pgfqpoint{1.677040in}{1.896011in}}%
\pgfpathlineto{\pgfqpoint{1.683240in}{1.890595in}}%
\pgfpathlineto{\pgfqpoint{1.684480in}{1.891514in}}%
\pgfpathlineto{\pgfqpoint{1.688200in}{1.899135in}}%
\pgfpathlineto{\pgfqpoint{1.690680in}{1.902617in}}%
\pgfpathlineto{\pgfqpoint{1.694400in}{1.897238in}}%
\pgfpathlineto{\pgfqpoint{1.700600in}{1.896696in}}%
\pgfpathlineto{\pgfqpoint{1.701840in}{1.895351in}}%
\pgfpathlineto{\pgfqpoint{1.704320in}{1.896729in}}%
\pgfpathlineto{\pgfqpoint{1.705560in}{1.895032in}}%
\pgfpathlineto{\pgfqpoint{1.709280in}{1.898570in}}%
\pgfpathlineto{\pgfqpoint{1.710520in}{1.895858in}}%
\pgfpathlineto{\pgfqpoint{1.713000in}{1.897511in}}%
\pgfpathlineto{\pgfqpoint{1.714240in}{1.899861in}}%
\pgfpathlineto{\pgfqpoint{1.715480in}{1.897443in}}%
\pgfpathlineto{\pgfqpoint{1.719200in}{1.898743in}}%
\pgfpathlineto{\pgfqpoint{1.721680in}{1.894441in}}%
\pgfpathlineto{\pgfqpoint{1.724160in}{1.892740in}}%
\pgfpathlineto{\pgfqpoint{1.725400in}{1.892858in}}%
\pgfpathlineto{\pgfqpoint{1.727880in}{1.894821in}}%
\pgfpathlineto{\pgfqpoint{1.731600in}{1.893235in}}%
\pgfpathlineto{\pgfqpoint{1.734080in}{1.895916in}}%
\pgfpathlineto{\pgfqpoint{1.736560in}{1.891650in}}%
\pgfpathlineto{\pgfqpoint{1.737800in}{1.892919in}}%
\pgfpathlineto{\pgfqpoint{1.740280in}{1.899017in}}%
\pgfpathlineto{\pgfqpoint{1.741520in}{1.897755in}}%
\pgfpathlineto{\pgfqpoint{1.745240in}{1.885106in}}%
\pgfpathlineto{\pgfqpoint{1.747720in}{1.886615in}}%
\pgfpathlineto{\pgfqpoint{1.750200in}{1.883412in}}%
\pgfpathlineto{\pgfqpoint{1.752680in}{1.880480in}}%
\pgfpathlineto{\pgfqpoint{1.753920in}{1.879488in}}%
\pgfpathlineto{\pgfqpoint{1.755160in}{1.881084in}}%
\pgfpathlineto{\pgfqpoint{1.756400in}{1.880066in}}%
\pgfpathlineto{\pgfqpoint{1.758880in}{1.883355in}}%
\pgfpathlineto{\pgfqpoint{1.761360in}{1.882830in}}%
\pgfpathlineto{\pgfqpoint{1.763840in}{1.878038in}}%
\pgfpathlineto{\pgfqpoint{1.765080in}{1.877972in}}%
\pgfpathlineto{\pgfqpoint{1.766320in}{1.879346in}}%
\pgfpathlineto{\pgfqpoint{1.770040in}{1.875051in}}%
\pgfpathlineto{\pgfqpoint{1.776240in}{1.878388in}}%
\pgfpathlineto{\pgfqpoint{1.778720in}{1.881132in}}%
\pgfpathlineto{\pgfqpoint{1.779960in}{1.882533in}}%
\pgfpathlineto{\pgfqpoint{1.783680in}{1.878987in}}%
\pgfpathlineto{\pgfqpoint{1.786160in}{1.878689in}}%
\pgfpathlineto{\pgfqpoint{1.787400in}{1.881130in}}%
\pgfpathlineto{\pgfqpoint{1.788640in}{1.879553in}}%
\pgfpathlineto{\pgfqpoint{1.789880in}{1.880269in}}%
\pgfpathlineto{\pgfqpoint{1.792360in}{1.878745in}}%
\pgfpathlineto{\pgfqpoint{1.799800in}{1.885778in}}%
\pgfpathlineto{\pgfqpoint{1.801040in}{1.886777in}}%
\pgfpathlineto{\pgfqpoint{1.808480in}{1.880068in}}%
\pgfpathlineto{\pgfqpoint{1.812200in}{1.886185in}}%
\pgfpathlineto{\pgfqpoint{1.814680in}{1.890421in}}%
\pgfpathlineto{\pgfqpoint{1.817160in}{1.887604in}}%
\pgfpathlineto{\pgfqpoint{1.820880in}{1.886882in}}%
\pgfpathlineto{\pgfqpoint{1.822120in}{1.887598in}}%
\pgfpathlineto{\pgfqpoint{1.825840in}{1.883290in}}%
\pgfpathlineto{\pgfqpoint{1.827080in}{1.883275in}}%
\pgfpathlineto{\pgfqpoint{1.828320in}{1.884538in}}%
\pgfpathlineto{\pgfqpoint{1.829560in}{1.883219in}}%
\pgfpathlineto{\pgfqpoint{1.833280in}{1.886217in}}%
\pgfpathlineto{\pgfqpoint{1.834520in}{1.884243in}}%
\pgfpathlineto{\pgfqpoint{1.837000in}{1.885712in}}%
\pgfpathlineto{\pgfqpoint{1.838240in}{1.887587in}}%
\pgfpathlineto{\pgfqpoint{1.839480in}{1.885767in}}%
\pgfpathlineto{\pgfqpoint{1.841960in}{1.888532in}}%
\pgfpathlineto{\pgfqpoint{1.843200in}{1.888392in}}%
\pgfpathlineto{\pgfqpoint{1.846920in}{1.882483in}}%
\pgfpathlineto{\pgfqpoint{1.849400in}{1.881781in}}%
\pgfpathlineto{\pgfqpoint{1.851880in}{1.883827in}}%
\pgfpathlineto{\pgfqpoint{1.855600in}{1.882570in}}%
\pgfpathlineto{\pgfqpoint{1.858080in}{1.884727in}}%
\pgfpathlineto{\pgfqpoint{1.860560in}{1.879468in}}%
\pgfpathlineto{\pgfqpoint{1.861800in}{1.880022in}}%
\pgfpathlineto{\pgfqpoint{1.864280in}{1.884735in}}%
\pgfpathlineto{\pgfqpoint{1.865520in}{1.883903in}}%
\pgfpathlineto{\pgfqpoint{1.869240in}{1.872141in}}%
\pgfpathlineto{\pgfqpoint{1.872960in}{1.872412in}}%
\pgfpathlineto{\pgfqpoint{1.875440in}{1.869065in}}%
\pgfpathlineto{\pgfqpoint{1.877920in}{1.869002in}}%
\pgfpathlineto{\pgfqpoint{1.879160in}{1.870575in}}%
\pgfpathlineto{\pgfqpoint{1.880400in}{1.870041in}}%
\pgfpathlineto{\pgfqpoint{1.882880in}{1.873408in}}%
\pgfpathlineto{\pgfqpoint{1.885360in}{1.872263in}}%
\pgfpathlineto{\pgfqpoint{1.887840in}{1.867449in}}%
\pgfpathlineto{\pgfqpoint{1.889080in}{1.867135in}}%
\pgfpathlineto{\pgfqpoint{1.890320in}{1.869260in}}%
\pgfpathlineto{\pgfqpoint{1.894040in}{1.865381in}}%
\pgfpathlineto{\pgfqpoint{1.897760in}{1.867268in}}%
\pgfpathlineto{\pgfqpoint{1.899000in}{1.866566in}}%
\pgfpathlineto{\pgfqpoint{1.903960in}{1.871248in}}%
\pgfpathlineto{\pgfqpoint{1.908920in}{1.865669in}}%
\pgfpathlineto{\pgfqpoint{1.910160in}{1.865873in}}%
\pgfpathlineto{\pgfqpoint{1.911400in}{1.868278in}}%
\pgfpathlineto{\pgfqpoint{1.912640in}{1.867681in}}%
\pgfpathlineto{\pgfqpoint{1.913880in}{1.869052in}}%
\pgfpathlineto{\pgfqpoint{1.917600in}{1.868805in}}%
\pgfpathlineto{\pgfqpoint{1.920080in}{1.871596in}}%
\pgfpathlineto{\pgfqpoint{1.922560in}{1.873393in}}%
\pgfpathlineto{\pgfqpoint{1.925040in}{1.874346in}}%
\pgfpathlineto{\pgfqpoint{1.928760in}{1.871150in}}%
\pgfpathlineto{\pgfqpoint{1.931240in}{1.867794in}}%
\pgfpathlineto{\pgfqpoint{1.932480in}{1.868286in}}%
\pgfpathlineto{\pgfqpoint{1.936200in}{1.873757in}}%
\pgfpathlineto{\pgfqpoint{1.938680in}{1.877795in}}%
\pgfpathlineto{\pgfqpoint{1.941160in}{1.874680in}}%
\pgfpathlineto{\pgfqpoint{1.946120in}{1.873333in}}%
\pgfpathlineto{\pgfqpoint{1.948600in}{1.871056in}}%
\pgfpathlineto{\pgfqpoint{1.951080in}{1.870077in}}%
\pgfpathlineto{\pgfqpoint{1.952320in}{1.870995in}}%
\pgfpathlineto{\pgfqpoint{1.953560in}{1.869704in}}%
\pgfpathlineto{\pgfqpoint{1.957280in}{1.872792in}}%
\pgfpathlineto{\pgfqpoint{1.958520in}{1.870643in}}%
\pgfpathlineto{\pgfqpoint{1.961000in}{1.872909in}}%
\pgfpathlineto{\pgfqpoint{1.962240in}{1.874868in}}%
\pgfpathlineto{\pgfqpoint{1.963480in}{1.872876in}}%
\pgfpathlineto{\pgfqpoint{1.967200in}{1.875932in}}%
\pgfpathlineto{\pgfqpoint{1.969680in}{1.871537in}}%
\pgfpathlineto{\pgfqpoint{1.972160in}{1.869909in}}%
\pgfpathlineto{\pgfqpoint{1.973400in}{1.869438in}}%
\pgfpathlineto{\pgfqpoint{1.975880in}{1.870939in}}%
\pgfpathlineto{\pgfqpoint{1.979600in}{1.870194in}}%
\pgfpathlineto{\pgfqpoint{1.980840in}{1.872659in}}%
\pgfpathlineto{\pgfqpoint{1.982080in}{1.872536in}}%
\pgfpathlineto{\pgfqpoint{1.984560in}{1.869325in}}%
\pgfpathlineto{\pgfqpoint{1.985800in}{1.869923in}}%
\pgfpathlineto{\pgfqpoint{1.988280in}{1.874649in}}%
\pgfpathlineto{\pgfqpoint{1.989520in}{1.872988in}}%
\pgfpathlineto{\pgfqpoint{1.993240in}{1.862824in}}%
\pgfpathlineto{\pgfqpoint{1.995720in}{1.863194in}}%
\pgfpathlineto{\pgfqpoint{1.999440in}{1.861412in}}%
\pgfpathlineto{\pgfqpoint{2.001920in}{1.861309in}}%
\pgfpathlineto{\pgfqpoint{2.003160in}{1.863192in}}%
\pgfpathlineto{\pgfqpoint{2.004400in}{1.862546in}}%
\pgfpathlineto{\pgfqpoint{2.006880in}{1.865703in}}%
\pgfpathlineto{\pgfqpoint{2.009360in}{1.864292in}}%
\pgfpathlineto{\pgfqpoint{2.011840in}{1.859194in}}%
\pgfpathlineto{\pgfqpoint{2.013080in}{1.859059in}}%
\pgfpathlineto{\pgfqpoint{2.014320in}{1.860913in}}%
\pgfpathlineto{\pgfqpoint{2.018040in}{1.857786in}}%
\pgfpathlineto{\pgfqpoint{2.021760in}{1.859881in}}%
\pgfpathlineto{\pgfqpoint{2.023000in}{1.860070in}}%
\pgfpathlineto{\pgfqpoint{2.027960in}{1.864981in}}%
\pgfpathlineto{\pgfqpoint{2.032920in}{1.860507in}}%
\pgfpathlineto{\pgfqpoint{2.034160in}{1.860608in}}%
\pgfpathlineto{\pgfqpoint{2.036640in}{1.863360in}}%
\pgfpathlineto{\pgfqpoint{2.039120in}{1.865442in}}%
\pgfpathlineto{\pgfqpoint{2.041600in}{1.866286in}}%
\pgfpathlineto{\pgfqpoint{2.049040in}{1.871095in}}%
\pgfpathlineto{\pgfqpoint{2.051520in}{1.869948in}}%
\pgfpathlineto{\pgfqpoint{2.055240in}{1.866408in}}%
\pgfpathlineto{\pgfqpoint{2.056480in}{1.867387in}}%
\pgfpathlineto{\pgfqpoint{2.058960in}{1.871501in}}%
\pgfpathlineto{\pgfqpoint{2.060200in}{1.872013in}}%
\pgfpathlineto{\pgfqpoint{2.062680in}{1.875978in}}%
\pgfpathlineto{\pgfqpoint{2.065160in}{1.872899in}}%
\pgfpathlineto{\pgfqpoint{2.068880in}{1.871518in}}%
\pgfpathlineto{\pgfqpoint{2.070120in}{1.871067in}}%
\pgfpathlineto{\pgfqpoint{2.072600in}{1.868627in}}%
\pgfpathlineto{\pgfqpoint{2.073840in}{1.868170in}}%
\pgfpathlineto{\pgfqpoint{2.076320in}{1.870097in}}%
\pgfpathlineto{\pgfqpoint{2.077560in}{1.868925in}}%
\pgfpathlineto{\pgfqpoint{2.081280in}{1.870561in}}%
\pgfpathlineto{\pgfqpoint{2.083760in}{1.868513in}}%
\pgfpathlineto{\pgfqpoint{2.086240in}{1.871314in}}%
\pgfpathlineto{\pgfqpoint{2.087480in}{1.869522in}}%
\pgfpathlineto{\pgfqpoint{2.091200in}{1.873657in}}%
\pgfpathlineto{\pgfqpoint{2.093680in}{1.870209in}}%
\pgfpathlineto{\pgfqpoint{2.096160in}{1.868879in}}%
\pgfpathlineto{\pgfqpoint{2.097400in}{1.868482in}}%
\pgfpathlineto{\pgfqpoint{2.101120in}{1.869940in}}%
\pgfpathlineto{\pgfqpoint{2.102360in}{1.868064in}}%
\pgfpathlineto{\pgfqpoint{2.103600in}{1.868400in}}%
\pgfpathlineto{\pgfqpoint{2.104840in}{1.870540in}}%
\pgfpathlineto{\pgfqpoint{2.106080in}{1.870154in}}%
\pgfpathlineto{\pgfqpoint{2.108560in}{1.867487in}}%
\pgfpathlineto{\pgfqpoint{2.109800in}{1.867374in}}%
\pgfpathlineto{\pgfqpoint{2.112280in}{1.871708in}}%
\pgfpathlineto{\pgfqpoint{2.113520in}{1.870235in}}%
\pgfpathlineto{\pgfqpoint{2.116000in}{1.863299in}}%
\pgfpathlineto{\pgfqpoint{2.119720in}{1.863037in}}%
\pgfpathlineto{\pgfqpoint{2.124680in}{1.861534in}}%
\pgfpathlineto{\pgfqpoint{2.125920in}{1.861772in}}%
\pgfpathlineto{\pgfqpoint{2.127160in}{1.863877in}}%
\pgfpathlineto{\pgfqpoint{2.128400in}{1.863604in}}%
\pgfpathlineto{\pgfqpoint{2.132120in}{1.866654in}}%
\pgfpathlineto{\pgfqpoint{2.133360in}{1.864843in}}%
\pgfpathlineto{\pgfqpoint{2.134600in}{1.860782in}}%
\pgfpathlineto{\pgfqpoint{2.137080in}{1.860645in}}%
\pgfpathlineto{\pgfqpoint{2.138320in}{1.861878in}}%
\pgfpathlineto{\pgfqpoint{2.142040in}{1.858750in}}%
\pgfpathlineto{\pgfqpoint{2.145760in}{1.861154in}}%
\pgfpathlineto{\pgfqpoint{2.149480in}{1.863486in}}%
\pgfpathlineto{\pgfqpoint{2.151960in}{1.866626in}}%
\pgfpathlineto{\pgfqpoint{2.156920in}{1.861424in}}%
\pgfpathlineto{\pgfqpoint{2.158160in}{1.861770in}}%
\pgfpathlineto{\pgfqpoint{2.160640in}{1.864046in}}%
\pgfpathlineto{\pgfqpoint{2.161880in}{1.865953in}}%
\pgfpathlineto{\pgfqpoint{2.164360in}{1.865722in}}%
\pgfpathlineto{\pgfqpoint{2.166840in}{1.867590in}}%
\pgfpathlineto{\pgfqpoint{2.169320in}{1.870053in}}%
\pgfpathlineto{\pgfqpoint{2.171800in}{1.870051in}}%
\pgfpathlineto{\pgfqpoint{2.175520in}{1.870220in}}%
\pgfpathlineto{\pgfqpoint{2.179240in}{1.865594in}}%
\pgfpathlineto{\pgfqpoint{2.186680in}{1.874569in}}%
\pgfpathlineto{\pgfqpoint{2.189160in}{1.872786in}}%
\pgfpathlineto{\pgfqpoint{2.191640in}{1.872464in}}%
\pgfpathlineto{\pgfqpoint{2.195360in}{1.867838in}}%
\pgfpathlineto{\pgfqpoint{2.199080in}{1.867349in}}%
\pgfpathlineto{\pgfqpoint{2.200320in}{1.868531in}}%
\pgfpathlineto{\pgfqpoint{2.201560in}{1.867187in}}%
\pgfpathlineto{\pgfqpoint{2.205280in}{1.867865in}}%
\pgfpathlineto{\pgfqpoint{2.207760in}{1.865768in}}%
\pgfpathlineto{\pgfqpoint{2.210240in}{1.868761in}}%
\pgfpathlineto{\pgfqpoint{2.211480in}{1.867192in}}%
\pgfpathlineto{\pgfqpoint{2.215200in}{1.871501in}}%
\pgfpathlineto{\pgfqpoint{2.218920in}{1.867076in}}%
\pgfpathlineto{\pgfqpoint{2.221400in}{1.866430in}}%
\pgfpathlineto{\pgfqpoint{2.225120in}{1.868745in}}%
\pgfpathlineto{\pgfqpoint{2.226360in}{1.866763in}}%
\pgfpathlineto{\pgfqpoint{2.230080in}{1.869005in}}%
\pgfpathlineto{\pgfqpoint{2.232560in}{1.866385in}}%
\pgfpathlineto{\pgfqpoint{2.233800in}{1.866133in}}%
\pgfpathlineto{\pgfqpoint{2.236280in}{1.871194in}}%
\pgfpathlineto{\pgfqpoint{2.237520in}{1.869708in}}%
\pgfpathlineto{\pgfqpoint{2.240000in}{1.862471in}}%
\pgfpathlineto{\pgfqpoint{2.242480in}{1.861819in}}%
\pgfpathlineto{\pgfqpoint{2.247440in}{1.860486in}}%
\pgfpathlineto{\pgfqpoint{2.256120in}{1.866673in}}%
\pgfpathlineto{\pgfqpoint{2.257360in}{1.864944in}}%
\pgfpathlineto{\pgfqpoint{2.258600in}{1.860687in}}%
\pgfpathlineto{\pgfqpoint{2.262320in}{1.861877in}}%
\pgfpathlineto{\pgfqpoint{2.266040in}{1.858681in}}%
\pgfpathlineto{\pgfqpoint{2.274720in}{1.866010in}}%
\pgfpathlineto{\pgfqpoint{2.275960in}{1.867381in}}%
\pgfpathlineto{\pgfqpoint{2.282160in}{1.863060in}}%
\pgfpathlineto{\pgfqpoint{2.285880in}{1.869013in}}%
\pgfpathlineto{\pgfqpoint{2.288360in}{1.868234in}}%
\pgfpathlineto{\pgfqpoint{2.297040in}{1.874871in}}%
\pgfpathlineto{\pgfqpoint{2.298280in}{1.875167in}}%
\pgfpathlineto{\pgfqpoint{2.302000in}{1.870853in}}%
\pgfpathlineto{\pgfqpoint{2.303240in}{1.868775in}}%
\pgfpathlineto{\pgfqpoint{2.305720in}{1.871729in}}%
\pgfpathlineto{\pgfqpoint{2.309440in}{1.875606in}}%
\pgfpathlineto{\pgfqpoint{2.310680in}{1.877050in}}%
\pgfpathlineto{\pgfqpoint{2.313160in}{1.875642in}}%
\pgfpathlineto{\pgfqpoint{2.315640in}{1.876442in}}%
\pgfpathlineto{\pgfqpoint{2.320600in}{1.871330in}}%
\pgfpathlineto{\pgfqpoint{2.321840in}{1.870590in}}%
\pgfpathlineto{\pgfqpoint{2.324320in}{1.873242in}}%
\pgfpathlineto{\pgfqpoint{2.325560in}{1.872044in}}%
\pgfpathlineto{\pgfqpoint{2.328040in}{1.873076in}}%
\pgfpathlineto{\pgfqpoint{2.329280in}{1.872570in}}%
\pgfpathlineto{\pgfqpoint{2.330520in}{1.870368in}}%
\pgfpathlineto{\pgfqpoint{2.331760in}{1.870875in}}%
\pgfpathlineto{\pgfqpoint{2.334240in}{1.874614in}}%
\pgfpathlineto{\pgfqpoint{2.335480in}{1.873639in}}%
\pgfpathlineto{\pgfqpoint{2.339200in}{1.877588in}}%
\pgfpathlineto{\pgfqpoint{2.342920in}{1.873284in}}%
\pgfpathlineto{\pgfqpoint{2.346640in}{1.873448in}}%
\pgfpathlineto{\pgfqpoint{2.347880in}{1.874830in}}%
\pgfpathlineto{\pgfqpoint{2.349120in}{1.874126in}}%
\pgfpathlineto{\pgfqpoint{2.350360in}{1.871807in}}%
\pgfpathlineto{\pgfqpoint{2.354080in}{1.873105in}}%
\pgfpathlineto{\pgfqpoint{2.356560in}{1.871042in}}%
\pgfpathlineto{\pgfqpoint{2.357800in}{1.870849in}}%
\pgfpathlineto{\pgfqpoint{2.360280in}{1.875497in}}%
\pgfpathlineto{\pgfqpoint{2.361520in}{1.873746in}}%
\pgfpathlineto{\pgfqpoint{2.365240in}{1.864162in}}%
\pgfpathlineto{\pgfqpoint{2.367720in}{1.863840in}}%
\pgfpathlineto{\pgfqpoint{2.371440in}{1.863769in}}%
\pgfpathlineto{\pgfqpoint{2.373920in}{1.865142in}}%
\pgfpathlineto{\pgfqpoint{2.375160in}{1.868107in}}%
\pgfpathlineto{\pgfqpoint{2.377640in}{1.869245in}}%
\pgfpathlineto{\pgfqpoint{2.378880in}{1.871263in}}%
\pgfpathlineto{\pgfqpoint{2.380120in}{1.870834in}}%
\pgfpathlineto{\pgfqpoint{2.382600in}{1.864947in}}%
\pgfpathlineto{\pgfqpoint{2.383840in}{1.865878in}}%
\pgfpathlineto{\pgfqpoint{2.386320in}{1.866989in}}%
\pgfpathlineto{\pgfqpoint{2.390040in}{1.862740in}}%
\pgfpathlineto{\pgfqpoint{2.397480in}{1.868396in}}%
\pgfpathlineto{\pgfqpoint{2.399960in}{1.871806in}}%
\pgfpathlineto{\pgfqpoint{2.406160in}{1.866921in}}%
\pgfpathlineto{\pgfqpoint{2.409880in}{1.872020in}}%
\pgfpathlineto{\pgfqpoint{2.412360in}{1.870844in}}%
\pgfpathlineto{\pgfqpoint{2.414840in}{1.872515in}}%
\pgfpathlineto{\pgfqpoint{2.416080in}{1.875256in}}%
\pgfpathlineto{\pgfqpoint{2.424760in}{1.873194in}}%
\pgfpathlineto{\pgfqpoint{2.427240in}{1.870553in}}%
\pgfpathlineto{\pgfqpoint{2.428480in}{1.870980in}}%
\pgfpathlineto{\pgfqpoint{2.434680in}{1.879295in}}%
\pgfpathlineto{\pgfqpoint{2.437160in}{1.877743in}}%
\pgfpathlineto{\pgfqpoint{2.439640in}{1.878440in}}%
\pgfpathlineto{\pgfqpoint{2.444600in}{1.873196in}}%
\pgfpathlineto{\pgfqpoint{2.447080in}{1.873196in}}%
\pgfpathlineto{\pgfqpoint{2.448320in}{1.874186in}}%
\pgfpathlineto{\pgfqpoint{2.449560in}{1.872968in}}%
\pgfpathlineto{\pgfqpoint{2.452040in}{1.873720in}}%
\pgfpathlineto{\pgfqpoint{2.453280in}{1.873227in}}%
\pgfpathlineto{\pgfqpoint{2.454520in}{1.871287in}}%
\pgfpathlineto{\pgfqpoint{2.455760in}{1.872129in}}%
\pgfpathlineto{\pgfqpoint{2.458240in}{1.875714in}}%
\pgfpathlineto{\pgfqpoint{2.459480in}{1.874919in}}%
\pgfpathlineto{\pgfqpoint{2.464440in}{1.878391in}}%
\pgfpathlineto{\pgfqpoint{2.465680in}{1.877537in}}%
\pgfpathlineto{\pgfqpoint{2.468160in}{1.875225in}}%
\pgfpathlineto{\pgfqpoint{2.470640in}{1.874735in}}%
\pgfpathlineto{\pgfqpoint{2.471880in}{1.876014in}}%
\pgfpathlineto{\pgfqpoint{2.473120in}{1.875301in}}%
\pgfpathlineto{\pgfqpoint{2.474360in}{1.873259in}}%
\pgfpathlineto{\pgfqpoint{2.478080in}{1.874720in}}%
\pgfpathlineto{\pgfqpoint{2.480560in}{1.872321in}}%
\pgfpathlineto{\pgfqpoint{2.481800in}{1.871749in}}%
\pgfpathlineto{\pgfqpoint{2.484280in}{1.876125in}}%
\pgfpathlineto{\pgfqpoint{2.485520in}{1.874491in}}%
\pgfpathlineto{\pgfqpoint{2.488000in}{1.867578in}}%
\pgfpathlineto{\pgfqpoint{2.489240in}{1.866199in}}%
\pgfpathlineto{\pgfqpoint{2.490480in}{1.866648in}}%
\pgfpathlineto{\pgfqpoint{2.492960in}{1.866157in}}%
\pgfpathlineto{\pgfqpoint{2.497920in}{1.868102in}}%
\pgfpathlineto{\pgfqpoint{2.500400in}{1.871438in}}%
\pgfpathlineto{\pgfqpoint{2.504120in}{1.875004in}}%
\pgfpathlineto{\pgfqpoint{2.507840in}{1.870687in}}%
\pgfpathlineto{\pgfqpoint{2.510320in}{1.871487in}}%
\pgfpathlineto{\pgfqpoint{2.514040in}{1.867386in}}%
\pgfpathlineto{\pgfqpoint{2.525200in}{1.875126in}}%
\pgfpathlineto{\pgfqpoint{2.527680in}{1.872879in}}%
\pgfpathlineto{\pgfqpoint{2.530160in}{1.871799in}}%
\pgfpathlineto{\pgfqpoint{2.533880in}{1.876517in}}%
\pgfpathlineto{\pgfqpoint{2.536360in}{1.875220in}}%
\pgfpathlineto{\pgfqpoint{2.538840in}{1.876736in}}%
\pgfpathlineto{\pgfqpoint{2.540080in}{1.879846in}}%
\pgfpathlineto{\pgfqpoint{2.547520in}{1.878412in}}%
\pgfpathlineto{\pgfqpoint{2.552480in}{1.875023in}}%
\pgfpathlineto{\pgfqpoint{2.558680in}{1.881558in}}%
\pgfpathlineto{\pgfqpoint{2.561160in}{1.879822in}}%
\pgfpathlineto{\pgfqpoint{2.563640in}{1.880481in}}%
\pgfpathlineto{\pgfqpoint{2.567360in}{1.875774in}}%
\pgfpathlineto{\pgfqpoint{2.571080in}{1.874821in}}%
\pgfpathlineto{\pgfqpoint{2.572320in}{1.875689in}}%
\pgfpathlineto{\pgfqpoint{2.573560in}{1.873978in}}%
\pgfpathlineto{\pgfqpoint{2.576040in}{1.875061in}}%
\pgfpathlineto{\pgfqpoint{2.579760in}{1.873815in}}%
\pgfpathlineto{\pgfqpoint{2.582240in}{1.878098in}}%
\pgfpathlineto{\pgfqpoint{2.583480in}{1.877043in}}%
\pgfpathlineto{\pgfqpoint{2.584720in}{1.877993in}}%
\pgfpathlineto{\pgfqpoint{2.587200in}{1.882097in}}%
\pgfpathlineto{\pgfqpoint{2.589680in}{1.880731in}}%
\pgfpathlineto{\pgfqpoint{2.592160in}{1.877972in}}%
\pgfpathlineto{\pgfqpoint{2.594640in}{1.877218in}}%
\pgfpathlineto{\pgfqpoint{2.595880in}{1.878313in}}%
\pgfpathlineto{\pgfqpoint{2.599600in}{1.876262in}}%
\pgfpathlineto{\pgfqpoint{2.600840in}{1.877602in}}%
\pgfpathlineto{\pgfqpoint{2.603320in}{1.875409in}}%
\pgfpathlineto{\pgfqpoint{2.605800in}{1.873395in}}%
\pgfpathlineto{\pgfqpoint{2.608280in}{1.877685in}}%
\pgfpathlineto{\pgfqpoint{2.609520in}{1.875703in}}%
\pgfpathlineto{\pgfqpoint{2.612000in}{1.868862in}}%
\pgfpathlineto{\pgfqpoint{2.614480in}{1.870814in}}%
\pgfpathlineto{\pgfqpoint{2.615720in}{1.869786in}}%
\pgfpathlineto{\pgfqpoint{2.621920in}{1.872756in}}%
\pgfpathlineto{\pgfqpoint{2.624400in}{1.876163in}}%
\pgfpathlineto{\pgfqpoint{2.628120in}{1.880117in}}%
\pgfpathlineto{\pgfqpoint{2.629360in}{1.878813in}}%
\pgfpathlineto{\pgfqpoint{2.630600in}{1.875550in}}%
\pgfpathlineto{\pgfqpoint{2.634320in}{1.877043in}}%
\pgfpathlineto{\pgfqpoint{2.638040in}{1.873683in}}%
\pgfpathlineto{\pgfqpoint{2.640520in}{1.875188in}}%
\pgfpathlineto{\pgfqpoint{2.643000in}{1.877513in}}%
\pgfpathlineto{\pgfqpoint{2.645480in}{1.877748in}}%
\pgfpathlineto{\pgfqpoint{2.647960in}{1.882396in}}%
\pgfpathlineto{\pgfqpoint{2.649200in}{1.881781in}}%
\pgfpathlineto{\pgfqpoint{2.651680in}{1.879119in}}%
\pgfpathlineto{\pgfqpoint{2.654160in}{1.877405in}}%
\pgfpathlineto{\pgfqpoint{2.657880in}{1.881245in}}%
\pgfpathlineto{\pgfqpoint{2.660360in}{1.880051in}}%
\pgfpathlineto{\pgfqpoint{2.665320in}{1.884208in}}%
\pgfpathlineto{\pgfqpoint{2.667800in}{1.884707in}}%
\pgfpathlineto{\pgfqpoint{2.670280in}{1.884585in}}%
\pgfpathlineto{\pgfqpoint{2.674000in}{1.880458in}}%
\pgfpathlineto{\pgfqpoint{2.675240in}{1.878215in}}%
\pgfpathlineto{\pgfqpoint{2.676480in}{1.878646in}}%
\pgfpathlineto{\pgfqpoint{2.682680in}{1.885032in}}%
\pgfpathlineto{\pgfqpoint{2.685160in}{1.882405in}}%
\pgfpathlineto{\pgfqpoint{2.687640in}{1.882941in}}%
\pgfpathlineto{\pgfqpoint{2.690120in}{1.881355in}}%
\pgfpathlineto{\pgfqpoint{2.693840in}{1.878383in}}%
\pgfpathlineto{\pgfqpoint{2.701280in}{1.878484in}}%
\pgfpathlineto{\pgfqpoint{2.703760in}{1.877331in}}%
\pgfpathlineto{\pgfqpoint{2.706240in}{1.881421in}}%
\pgfpathlineto{\pgfqpoint{2.707480in}{1.880880in}}%
\pgfpathlineto{\pgfqpoint{2.713680in}{1.885483in}}%
\pgfpathlineto{\pgfqpoint{2.716160in}{1.882755in}}%
\pgfpathlineto{\pgfqpoint{2.718640in}{1.881472in}}%
\pgfpathlineto{\pgfqpoint{2.719880in}{1.882190in}}%
\pgfpathlineto{\pgfqpoint{2.722360in}{1.879886in}}%
\pgfpathlineto{\pgfqpoint{2.724840in}{1.883085in}}%
\pgfpathlineto{\pgfqpoint{2.726080in}{1.882439in}}%
\pgfpathlineto{\pgfqpoint{2.729800in}{1.877628in}}%
\pgfpathlineto{\pgfqpoint{2.732280in}{1.881083in}}%
\pgfpathlineto{\pgfqpoint{2.733520in}{1.879361in}}%
\pgfpathlineto{\pgfqpoint{2.736000in}{1.873429in}}%
\pgfpathlineto{\pgfqpoint{2.738480in}{1.874709in}}%
\pgfpathlineto{\pgfqpoint{2.740960in}{1.873416in}}%
\pgfpathlineto{\pgfqpoint{2.744680in}{1.874723in}}%
\pgfpathlineto{\pgfqpoint{2.745920in}{1.875808in}}%
\pgfpathlineto{\pgfqpoint{2.748400in}{1.880035in}}%
\pgfpathlineto{\pgfqpoint{2.752120in}{1.883417in}}%
\pgfpathlineto{\pgfqpoint{2.753360in}{1.882423in}}%
\pgfpathlineto{\pgfqpoint{2.754600in}{1.879226in}}%
\pgfpathlineto{\pgfqpoint{2.758320in}{1.880248in}}%
\pgfpathlineto{\pgfqpoint{2.762040in}{1.877011in}}%
\pgfpathlineto{\pgfqpoint{2.763280in}{1.876678in}}%
\pgfpathlineto{\pgfqpoint{2.773200in}{1.883944in}}%
\pgfpathlineto{\pgfqpoint{2.778160in}{1.879873in}}%
\pgfpathlineto{\pgfqpoint{2.781880in}{1.885575in}}%
\pgfpathlineto{\pgfqpoint{2.784360in}{1.883402in}}%
\pgfpathlineto{\pgfqpoint{2.793040in}{1.888376in}}%
\pgfpathlineto{\pgfqpoint{2.795520in}{1.886593in}}%
\pgfpathlineto{\pgfqpoint{2.800480in}{1.881414in}}%
\pgfpathlineto{\pgfqpoint{2.802960in}{1.885016in}}%
\pgfpathlineto{\pgfqpoint{2.805440in}{1.886245in}}%
\pgfpathlineto{\pgfqpoint{2.806680in}{1.887512in}}%
\pgfpathlineto{\pgfqpoint{2.809160in}{1.885335in}}%
\pgfpathlineto{\pgfqpoint{2.811640in}{1.886092in}}%
\pgfpathlineto{\pgfqpoint{2.815360in}{1.882811in}}%
\pgfpathlineto{\pgfqpoint{2.819080in}{1.881198in}}%
\pgfpathlineto{\pgfqpoint{2.820320in}{1.882240in}}%
\pgfpathlineto{\pgfqpoint{2.821560in}{1.881284in}}%
\pgfpathlineto{\pgfqpoint{2.824040in}{1.882790in}}%
\pgfpathlineto{\pgfqpoint{2.827760in}{1.880865in}}%
\pgfpathlineto{\pgfqpoint{2.830240in}{1.884113in}}%
\pgfpathlineto{\pgfqpoint{2.831480in}{1.883324in}}%
\pgfpathlineto{\pgfqpoint{2.837680in}{1.887236in}}%
\pgfpathlineto{\pgfqpoint{2.840160in}{1.884540in}}%
\pgfpathlineto{\pgfqpoint{2.842640in}{1.883700in}}%
\pgfpathlineto{\pgfqpoint{2.843880in}{1.884344in}}%
\pgfpathlineto{\pgfqpoint{2.846360in}{1.881148in}}%
\pgfpathlineto{\pgfqpoint{2.850080in}{1.883950in}}%
\pgfpathlineto{\pgfqpoint{2.853800in}{1.879981in}}%
\pgfpathlineto{\pgfqpoint{2.856280in}{1.883054in}}%
\pgfpathlineto{\pgfqpoint{2.857520in}{1.881356in}}%
\pgfpathlineto{\pgfqpoint{2.860000in}{1.875757in}}%
\pgfpathlineto{\pgfqpoint{2.862480in}{1.877843in}}%
\pgfpathlineto{\pgfqpoint{2.863720in}{1.876450in}}%
\pgfpathlineto{\pgfqpoint{2.869920in}{1.879752in}}%
\pgfpathlineto{\pgfqpoint{2.872400in}{1.883682in}}%
\pgfpathlineto{\pgfqpoint{2.876120in}{1.887116in}}%
\pgfpathlineto{\pgfqpoint{2.877360in}{1.886367in}}%
\pgfpathlineto{\pgfqpoint{2.878600in}{1.883789in}}%
\pgfpathlineto{\pgfqpoint{2.882320in}{1.885929in}}%
\pgfpathlineto{\pgfqpoint{2.887280in}{1.882397in}}%
\pgfpathlineto{\pgfqpoint{2.895960in}{1.889693in}}%
\pgfpathlineto{\pgfqpoint{2.898440in}{1.888266in}}%
\pgfpathlineto{\pgfqpoint{2.902160in}{1.886064in}}%
\pgfpathlineto{\pgfqpoint{2.905880in}{1.892418in}}%
\pgfpathlineto{\pgfqpoint{2.908360in}{1.889875in}}%
\pgfpathlineto{\pgfqpoint{2.910840in}{1.891353in}}%
\pgfpathlineto{\pgfqpoint{2.913320in}{1.894758in}}%
\pgfpathlineto{\pgfqpoint{2.919520in}{1.894361in}}%
\pgfpathlineto{\pgfqpoint{2.924480in}{1.888872in}}%
\pgfpathlineto{\pgfqpoint{2.926960in}{1.892518in}}%
\pgfpathlineto{\pgfqpoint{2.929440in}{1.892728in}}%
\pgfpathlineto{\pgfqpoint{2.930680in}{1.893526in}}%
\pgfpathlineto{\pgfqpoint{2.933160in}{1.891123in}}%
\pgfpathlineto{\pgfqpoint{2.935640in}{1.891794in}}%
\pgfpathlineto{\pgfqpoint{2.939360in}{1.887542in}}%
\pgfpathlineto{\pgfqpoint{2.941840in}{1.886048in}}%
\pgfpathlineto{\pgfqpoint{2.948040in}{1.888876in}}%
\pgfpathlineto{\pgfqpoint{2.951760in}{1.887254in}}%
\pgfpathlineto{\pgfqpoint{2.954240in}{1.890739in}}%
\pgfpathlineto{\pgfqpoint{2.955480in}{1.890108in}}%
\pgfpathlineto{\pgfqpoint{2.960440in}{1.893697in}}%
\pgfpathlineto{\pgfqpoint{2.961680in}{1.893095in}}%
\pgfpathlineto{\pgfqpoint{2.962920in}{1.890190in}}%
\pgfpathlineto{\pgfqpoint{2.964160in}{1.890370in}}%
\pgfpathlineto{\pgfqpoint{2.966640in}{1.889155in}}%
\pgfpathlineto{\pgfqpoint{2.967880in}{1.889218in}}%
\pgfpathlineto{\pgfqpoint{2.971600in}{1.887175in}}%
\pgfpathlineto{\pgfqpoint{2.972840in}{1.888943in}}%
\pgfpathlineto{\pgfqpoint{2.974080in}{1.888738in}}%
\pgfpathlineto{\pgfqpoint{2.976560in}{1.885462in}}%
\pgfpathlineto{\pgfqpoint{2.977800in}{1.885321in}}%
\pgfpathlineto{\pgfqpoint{2.980280in}{1.888879in}}%
\pgfpathlineto{\pgfqpoint{2.981520in}{1.887395in}}%
\pgfpathlineto{\pgfqpoint{2.984000in}{1.881989in}}%
\pgfpathlineto{\pgfqpoint{2.986480in}{1.884408in}}%
\pgfpathlineto{\pgfqpoint{2.988960in}{1.883170in}}%
\pgfpathlineto{\pgfqpoint{2.993920in}{1.885710in}}%
\pgfpathlineto{\pgfqpoint{2.996400in}{1.889836in}}%
\pgfpathlineto{\pgfqpoint{3.000120in}{1.893817in}}%
\pgfpathlineto{\pgfqpoint{3.003840in}{1.891073in}}%
\pgfpathlineto{\pgfqpoint{3.007560in}{1.891688in}}%
\pgfpathlineto{\pgfqpoint{3.010040in}{1.888512in}}%
\pgfpathlineto{\pgfqpoint{3.011280in}{1.888015in}}%
\pgfpathlineto{\pgfqpoint{3.016240in}{1.891548in}}%
\pgfpathlineto{\pgfqpoint{3.017480in}{1.891417in}}%
\pgfpathlineto{\pgfqpoint{3.019960in}{1.894500in}}%
\pgfpathlineto{\pgfqpoint{3.023680in}{1.893132in}}%
\pgfpathlineto{\pgfqpoint{3.026160in}{1.891340in}}%
\pgfpathlineto{\pgfqpoint{3.029880in}{1.897600in}}%
\pgfpathlineto{\pgfqpoint{3.032360in}{1.894973in}}%
\pgfpathlineto{\pgfqpoint{3.034840in}{1.896593in}}%
\pgfpathlineto{\pgfqpoint{3.037320in}{1.900130in}}%
\pgfpathlineto{\pgfqpoint{3.041040in}{1.900713in}}%
\pgfpathlineto{\pgfqpoint{3.043520in}{1.898883in}}%
\pgfpathlineto{\pgfqpoint{3.047240in}{1.892672in}}%
\pgfpathlineto{\pgfqpoint{3.048480in}{1.892779in}}%
\pgfpathlineto{\pgfqpoint{3.050960in}{1.895582in}}%
\pgfpathlineto{\pgfqpoint{3.053440in}{1.895556in}}%
\pgfpathlineto{\pgfqpoint{3.054680in}{1.896653in}}%
\pgfpathlineto{\pgfqpoint{3.057160in}{1.894369in}}%
\pgfpathlineto{\pgfqpoint{3.059640in}{1.895585in}}%
\pgfpathlineto{\pgfqpoint{3.064600in}{1.890556in}}%
\pgfpathlineto{\pgfqpoint{3.065840in}{1.890055in}}%
\pgfpathlineto{\pgfqpoint{3.069560in}{1.892783in}}%
\pgfpathlineto{\pgfqpoint{3.072040in}{1.893872in}}%
\pgfpathlineto{\pgfqpoint{3.075760in}{1.892040in}}%
\pgfpathlineto{\pgfqpoint{3.078240in}{1.894675in}}%
\pgfpathlineto{\pgfqpoint{3.079480in}{1.894048in}}%
\pgfpathlineto{\pgfqpoint{3.084440in}{1.898958in}}%
\pgfpathlineto{\pgfqpoint{3.085680in}{1.898277in}}%
\pgfpathlineto{\pgfqpoint{3.086920in}{1.895574in}}%
\pgfpathlineto{\pgfqpoint{3.088160in}{1.895598in}}%
\pgfpathlineto{\pgfqpoint{3.090640in}{1.893921in}}%
\pgfpathlineto{\pgfqpoint{3.093120in}{1.892749in}}%
\pgfpathlineto{\pgfqpoint{3.094360in}{1.890616in}}%
\pgfpathlineto{\pgfqpoint{3.098080in}{1.893680in}}%
\pgfpathlineto{\pgfqpoint{3.101800in}{1.890560in}}%
\pgfpathlineto{\pgfqpoint{3.104280in}{1.894306in}}%
\pgfpathlineto{\pgfqpoint{3.108000in}{1.885877in}}%
\pgfpathlineto{\pgfqpoint{3.109240in}{1.887223in}}%
\pgfpathlineto{\pgfqpoint{3.110480in}{1.887795in}}%
\pgfpathlineto{\pgfqpoint{3.111720in}{1.886753in}}%
\pgfpathlineto{\pgfqpoint{3.117920in}{1.889774in}}%
\pgfpathlineto{\pgfqpoint{3.120400in}{1.893497in}}%
\pgfpathlineto{\pgfqpoint{3.124120in}{1.896669in}}%
\pgfpathlineto{\pgfqpoint{3.127840in}{1.894286in}}%
\pgfpathlineto{\pgfqpoint{3.131560in}{1.895227in}}%
\pgfpathlineto{\pgfqpoint{3.134040in}{1.892764in}}%
\pgfpathlineto{\pgfqpoint{3.135280in}{1.891989in}}%
\pgfpathlineto{\pgfqpoint{3.145200in}{1.897570in}}%
\pgfpathlineto{\pgfqpoint{3.150160in}{1.894812in}}%
\pgfpathlineto{\pgfqpoint{3.153880in}{1.900839in}}%
\pgfpathlineto{\pgfqpoint{3.156360in}{1.898134in}}%
\pgfpathlineto{\pgfqpoint{3.158840in}{1.899535in}}%
\pgfpathlineto{\pgfqpoint{3.160080in}{1.902434in}}%
\pgfpathlineto{\pgfqpoint{3.166280in}{1.901532in}}%
\pgfpathlineto{\pgfqpoint{3.168760in}{1.897835in}}%
\pgfpathlineto{\pgfqpoint{3.172480in}{1.894531in}}%
\pgfpathlineto{\pgfqpoint{3.174960in}{1.897593in}}%
\pgfpathlineto{\pgfqpoint{3.177440in}{1.897692in}}%
\pgfpathlineto{\pgfqpoint{3.178680in}{1.899054in}}%
\pgfpathlineto{\pgfqpoint{3.181160in}{1.897064in}}%
\pgfpathlineto{\pgfqpoint{3.183640in}{1.897850in}}%
\pgfpathlineto{\pgfqpoint{3.188600in}{1.892647in}}%
\pgfpathlineto{\pgfqpoint{3.189840in}{1.892123in}}%
\pgfpathlineto{\pgfqpoint{3.192320in}{1.895510in}}%
\pgfpathlineto{\pgfqpoint{3.193560in}{1.894513in}}%
\pgfpathlineto{\pgfqpoint{3.196040in}{1.894931in}}%
\pgfpathlineto{\pgfqpoint{3.199760in}{1.892679in}}%
\pgfpathlineto{\pgfqpoint{3.202240in}{1.895055in}}%
\pgfpathlineto{\pgfqpoint{3.203480in}{1.894518in}}%
\pgfpathlineto{\pgfqpoint{3.207200in}{1.899001in}}%
\pgfpathlineto{\pgfqpoint{3.209680in}{1.898943in}}%
\pgfpathlineto{\pgfqpoint{3.210920in}{1.896570in}}%
\pgfpathlineto{\pgfqpoint{3.212160in}{1.896781in}}%
\pgfpathlineto{\pgfqpoint{3.214640in}{1.894963in}}%
\pgfpathlineto{\pgfqpoint{3.217120in}{1.894064in}}%
\pgfpathlineto{\pgfqpoint{3.218360in}{1.892120in}}%
\pgfpathlineto{\pgfqpoint{3.220840in}{1.894203in}}%
\pgfpathlineto{\pgfqpoint{3.222080in}{1.893993in}}%
\pgfpathlineto{\pgfqpoint{3.225800in}{1.890319in}}%
\pgfpathlineto{\pgfqpoint{3.228280in}{1.894124in}}%
\pgfpathlineto{\pgfqpoint{3.232000in}{1.885891in}}%
\pgfpathlineto{\pgfqpoint{3.233240in}{1.886763in}}%
\pgfpathlineto{\pgfqpoint{3.234480in}{1.887254in}}%
\pgfpathlineto{\pgfqpoint{3.235720in}{1.886432in}}%
\pgfpathlineto{\pgfqpoint{3.238200in}{1.887418in}}%
\pgfpathlineto{\pgfqpoint{3.239440in}{1.886598in}}%
\pgfpathlineto{\pgfqpoint{3.243160in}{1.892394in}}%
\pgfpathlineto{\pgfqpoint{3.244400in}{1.892516in}}%
\pgfpathlineto{\pgfqpoint{3.248120in}{1.896838in}}%
\pgfpathlineto{\pgfqpoint{3.251840in}{1.894425in}}%
\pgfpathlineto{\pgfqpoint{3.255560in}{1.894307in}}%
\pgfpathlineto{\pgfqpoint{3.259280in}{1.890989in}}%
\pgfpathlineto{\pgfqpoint{3.265480in}{1.893271in}}%
\pgfpathlineto{\pgfqpoint{3.267960in}{1.896180in}}%
\pgfpathlineto{\pgfqpoint{3.272920in}{1.894719in}}%
\pgfpathlineto{\pgfqpoint{3.274160in}{1.894039in}}%
\pgfpathlineto{\pgfqpoint{3.277880in}{1.900281in}}%
\pgfpathlineto{\pgfqpoint{3.280360in}{1.897641in}}%
\pgfpathlineto{\pgfqpoint{3.282840in}{1.899107in}}%
\pgfpathlineto{\pgfqpoint{3.284080in}{1.902196in}}%
\pgfpathlineto{\pgfqpoint{3.286560in}{1.902536in}}%
\pgfpathlineto{\pgfqpoint{3.289040in}{1.902769in}}%
\pgfpathlineto{\pgfqpoint{3.291520in}{1.900019in}}%
\pgfpathlineto{\pgfqpoint{3.295240in}{1.893267in}}%
\pgfpathlineto{\pgfqpoint{3.296480in}{1.893948in}}%
\pgfpathlineto{\pgfqpoint{3.298960in}{1.897742in}}%
\pgfpathlineto{\pgfqpoint{3.301440in}{1.897237in}}%
\pgfpathlineto{\pgfqpoint{3.302680in}{1.898684in}}%
\pgfpathlineto{\pgfqpoint{3.305160in}{1.897303in}}%
\pgfpathlineto{\pgfqpoint{3.307640in}{1.898051in}}%
\pgfpathlineto{\pgfqpoint{3.313840in}{1.892278in}}%
\pgfpathlineto{\pgfqpoint{3.316320in}{1.894970in}}%
\pgfpathlineto{\pgfqpoint{3.317560in}{1.893741in}}%
\pgfpathlineto{\pgfqpoint{3.321280in}{1.894848in}}%
\pgfpathlineto{\pgfqpoint{3.323760in}{1.893677in}}%
\pgfpathlineto{\pgfqpoint{3.326240in}{1.895901in}}%
\pgfpathlineto{\pgfqpoint{3.327480in}{1.895468in}}%
\pgfpathlineto{\pgfqpoint{3.331200in}{1.899574in}}%
\pgfpathlineto{\pgfqpoint{3.333680in}{1.899863in}}%
\pgfpathlineto{\pgfqpoint{3.336160in}{1.897193in}}%
\pgfpathlineto{\pgfqpoint{3.338640in}{1.894948in}}%
\pgfpathlineto{\pgfqpoint{3.341120in}{1.894181in}}%
\pgfpathlineto{\pgfqpoint{3.342360in}{1.892400in}}%
\pgfpathlineto{\pgfqpoint{3.344840in}{1.894162in}}%
\pgfpathlineto{\pgfqpoint{3.347320in}{1.892656in}}%
\pgfpathlineto{\pgfqpoint{3.349800in}{1.891294in}}%
\pgfpathlineto{\pgfqpoint{3.352280in}{1.894743in}}%
\pgfpathlineto{\pgfqpoint{3.357240in}{1.887183in}}%
\pgfpathlineto{\pgfqpoint{3.363440in}{1.886979in}}%
\pgfpathlineto{\pgfqpoint{3.367160in}{1.892382in}}%
\pgfpathlineto{\pgfqpoint{3.368400in}{1.892321in}}%
\pgfpathlineto{\pgfqpoint{3.372120in}{1.896018in}}%
\pgfpathlineto{\pgfqpoint{3.373360in}{1.895207in}}%
\pgfpathlineto{\pgfqpoint{3.374600in}{1.892466in}}%
\pgfpathlineto{\pgfqpoint{3.378320in}{1.895342in}}%
\pgfpathlineto{\pgfqpoint{3.383280in}{1.890958in}}%
\pgfpathlineto{\pgfqpoint{3.395680in}{1.895284in}}%
\pgfpathlineto{\pgfqpoint{3.398160in}{1.894485in}}%
\pgfpathlineto{\pgfqpoint{3.401880in}{1.900863in}}%
\pgfpathlineto{\pgfqpoint{3.404360in}{1.897969in}}%
\pgfpathlineto{\pgfqpoint{3.406840in}{1.900005in}}%
\pgfpathlineto{\pgfqpoint{3.408080in}{1.904028in}}%
\pgfpathlineto{\pgfqpoint{3.413040in}{1.903983in}}%
\pgfpathlineto{\pgfqpoint{3.416760in}{1.898285in}}%
\pgfpathlineto{\pgfqpoint{3.419240in}{1.894004in}}%
\pgfpathlineto{\pgfqpoint{3.420480in}{1.894900in}}%
\pgfpathlineto{\pgfqpoint{3.422960in}{1.898399in}}%
\pgfpathlineto{\pgfqpoint{3.425440in}{1.897496in}}%
\pgfpathlineto{\pgfqpoint{3.426680in}{1.899223in}}%
\pgfpathlineto{\pgfqpoint{3.429160in}{1.897922in}}%
\pgfpathlineto{\pgfqpoint{3.431640in}{1.899801in}}%
\pgfpathlineto{\pgfqpoint{3.437840in}{1.894308in}}%
\pgfpathlineto{\pgfqpoint{3.441560in}{1.896682in}}%
\pgfpathlineto{\pgfqpoint{3.445280in}{1.896962in}}%
\pgfpathlineto{\pgfqpoint{3.447760in}{1.895947in}}%
\pgfpathlineto{\pgfqpoint{3.452720in}{1.900036in}}%
\pgfpathlineto{\pgfqpoint{3.455200in}{1.901791in}}%
\pgfpathlineto{\pgfqpoint{3.457680in}{1.900901in}}%
\pgfpathlineto{\pgfqpoint{3.458920in}{1.898705in}}%
\pgfpathlineto{\pgfqpoint{3.460160in}{1.899158in}}%
\pgfpathlineto{\pgfqpoint{3.462640in}{1.897619in}}%
\pgfpathlineto{\pgfqpoint{3.465120in}{1.897188in}}%
\pgfpathlineto{\pgfqpoint{3.466360in}{1.895652in}}%
\pgfpathlineto{\pgfqpoint{3.468840in}{1.897411in}}%
\pgfpathlineto{\pgfqpoint{3.471320in}{1.895634in}}%
\pgfpathlineto{\pgfqpoint{3.473800in}{1.894751in}}%
\pgfpathlineto{\pgfqpoint{3.476280in}{1.898429in}}%
\pgfpathlineto{\pgfqpoint{3.478760in}{1.893348in}}%
\pgfpathlineto{\pgfqpoint{3.480000in}{1.890907in}}%
\pgfpathlineto{\pgfqpoint{3.482480in}{1.892172in}}%
\pgfpathlineto{\pgfqpoint{3.483720in}{1.891511in}}%
\pgfpathlineto{\pgfqpoint{3.486200in}{1.892366in}}%
\pgfpathlineto{\pgfqpoint{3.487440in}{1.891435in}}%
\pgfpathlineto{\pgfqpoint{3.491160in}{1.896770in}}%
\pgfpathlineto{\pgfqpoint{3.492400in}{1.896444in}}%
\pgfpathlineto{\pgfqpoint{3.496120in}{1.900380in}}%
\pgfpathlineto{\pgfqpoint{3.497360in}{1.899916in}}%
\pgfpathlineto{\pgfqpoint{3.498600in}{1.897119in}}%
\pgfpathlineto{\pgfqpoint{3.502320in}{1.899476in}}%
\pgfpathlineto{\pgfqpoint{3.508520in}{1.895000in}}%
\pgfpathlineto{\pgfqpoint{3.511000in}{1.896778in}}%
\pgfpathlineto{\pgfqpoint{3.513480in}{1.896720in}}%
\pgfpathlineto{\pgfqpoint{3.515960in}{1.899338in}}%
\pgfpathlineto{\pgfqpoint{3.517200in}{1.899558in}}%
\pgfpathlineto{\pgfqpoint{3.519680in}{1.898307in}}%
\pgfpathlineto{\pgfqpoint{3.522160in}{1.898143in}}%
\pgfpathlineto{\pgfqpoint{3.525880in}{1.904368in}}%
\pgfpathlineto{\pgfqpoint{3.528360in}{1.901020in}}%
\pgfpathlineto{\pgfqpoint{3.530840in}{1.903907in}}%
\pgfpathlineto{\pgfqpoint{3.532080in}{1.908197in}}%
\pgfpathlineto{\pgfqpoint{3.537040in}{1.907872in}}%
\pgfpathlineto{\pgfqpoint{3.540760in}{1.900933in}}%
\pgfpathlineto{\pgfqpoint{3.543240in}{1.896338in}}%
\pgfpathlineto{\pgfqpoint{3.544480in}{1.897470in}}%
\pgfpathlineto{\pgfqpoint{3.546960in}{1.901034in}}%
\pgfpathlineto{\pgfqpoint{3.549440in}{1.901146in}}%
\pgfpathlineto{\pgfqpoint{3.550680in}{1.903613in}}%
\pgfpathlineto{\pgfqpoint{3.553160in}{1.901994in}}%
\pgfpathlineto{\pgfqpoint{3.555640in}{1.903623in}}%
\pgfpathlineto{\pgfqpoint{3.561840in}{1.897801in}}%
\pgfpathlineto{\pgfqpoint{3.565560in}{1.900636in}}%
\pgfpathlineto{\pgfqpoint{3.573000in}{1.902549in}}%
\pgfpathlineto{\pgfqpoint{3.579200in}{1.908350in}}%
\pgfpathlineto{\pgfqpoint{3.580440in}{1.908731in}}%
\pgfpathlineto{\pgfqpoint{3.581680in}{1.907659in}}%
\pgfpathlineto{\pgfqpoint{3.582920in}{1.905061in}}%
\pgfpathlineto{\pgfqpoint{3.584160in}{1.905251in}}%
\pgfpathlineto{\pgfqpoint{3.586640in}{1.903814in}}%
\pgfpathlineto{\pgfqpoint{3.591600in}{1.903853in}}%
\pgfpathlineto{\pgfqpoint{3.594080in}{1.904011in}}%
\pgfpathlineto{\pgfqpoint{3.597800in}{1.901782in}}%
\pgfpathlineto{\pgfqpoint{3.600280in}{1.906176in}}%
\pgfpathlineto{\pgfqpoint{3.601520in}{1.904716in}}%
\pgfpathlineto{\pgfqpoint{3.604000in}{1.898921in}}%
\pgfpathlineto{\pgfqpoint{3.605240in}{1.901354in}}%
\pgfpathlineto{\pgfqpoint{3.611440in}{1.900730in}}%
\pgfpathlineto{\pgfqpoint{3.613920in}{1.904145in}}%
\pgfpathlineto{\pgfqpoint{3.615160in}{1.905622in}}%
\pgfpathlineto{\pgfqpoint{3.616400in}{1.904963in}}%
\pgfpathlineto{\pgfqpoint{3.620120in}{1.908733in}}%
\pgfpathlineto{\pgfqpoint{3.621360in}{1.907749in}}%
\pgfpathlineto{\pgfqpoint{3.622600in}{1.905291in}}%
\pgfpathlineto{\pgfqpoint{3.623840in}{1.906494in}}%
\pgfpathlineto{\pgfqpoint{3.625080in}{1.905946in}}%
\pgfpathlineto{\pgfqpoint{3.626320in}{1.907093in}}%
\pgfpathlineto{\pgfqpoint{3.632520in}{1.902006in}}%
\pgfpathlineto{\pgfqpoint{3.635000in}{1.903777in}}%
\pgfpathlineto{\pgfqpoint{3.637480in}{1.903535in}}%
\pgfpathlineto{\pgfqpoint{3.638720in}{1.905902in}}%
\pgfpathlineto{\pgfqpoint{3.642440in}{1.904130in}}%
\pgfpathlineto{\pgfqpoint{3.646160in}{1.903645in}}%
\pgfpathlineto{\pgfqpoint{3.649880in}{1.910087in}}%
\pgfpathlineto{\pgfqpoint{3.652360in}{1.906416in}}%
\pgfpathlineto{\pgfqpoint{3.654840in}{1.909400in}}%
\pgfpathlineto{\pgfqpoint{3.656080in}{1.913939in}}%
\pgfpathlineto{\pgfqpoint{3.661040in}{1.914617in}}%
\pgfpathlineto{\pgfqpoint{3.664760in}{1.908326in}}%
\pgfpathlineto{\pgfqpoint{3.667240in}{1.904062in}}%
\pgfpathlineto{\pgfqpoint{3.668480in}{1.905196in}}%
\pgfpathlineto{\pgfqpoint{3.670960in}{1.908303in}}%
\pgfpathlineto{\pgfqpoint{3.673440in}{1.908893in}}%
\pgfpathlineto{\pgfqpoint{3.674680in}{1.911370in}}%
\pgfpathlineto{\pgfqpoint{3.677160in}{1.910235in}}%
\pgfpathlineto{\pgfqpoint{3.679640in}{1.912776in}}%
\pgfpathlineto{\pgfqpoint{3.685840in}{1.907600in}}%
\pgfpathlineto{\pgfqpoint{3.690800in}{1.910536in}}%
\pgfpathlineto{\pgfqpoint{3.694520in}{1.910370in}}%
\pgfpathlineto{\pgfqpoint{3.699480in}{1.912960in}}%
\pgfpathlineto{\pgfqpoint{3.703200in}{1.917570in}}%
\pgfpathlineto{\pgfqpoint{3.704440in}{1.917892in}}%
\pgfpathlineto{\pgfqpoint{3.709400in}{1.913120in}}%
\pgfpathlineto{\pgfqpoint{3.713120in}{1.912850in}}%
\pgfpathlineto{\pgfqpoint{3.714360in}{1.911260in}}%
\pgfpathlineto{\pgfqpoint{3.718080in}{1.913406in}}%
\pgfpathlineto{\pgfqpoint{3.721800in}{1.910845in}}%
\pgfpathlineto{\pgfqpoint{3.724280in}{1.915229in}}%
\pgfpathlineto{\pgfqpoint{3.726760in}{1.910833in}}%
\pgfpathlineto{\pgfqpoint{3.728000in}{1.908788in}}%
\pgfpathlineto{\pgfqpoint{3.729240in}{1.910222in}}%
\pgfpathlineto{\pgfqpoint{3.732960in}{1.910272in}}%
\pgfpathlineto{\pgfqpoint{3.735440in}{1.909356in}}%
\pgfpathlineto{\pgfqpoint{3.737920in}{1.913200in}}%
\pgfpathlineto{\pgfqpoint{3.739160in}{1.914450in}}%
\pgfpathlineto{\pgfqpoint{3.740400in}{1.913703in}}%
\pgfpathlineto{\pgfqpoint{3.742880in}{1.918188in}}%
\pgfpathlineto{\pgfqpoint{3.745360in}{1.917292in}}%
\pgfpathlineto{\pgfqpoint{3.746600in}{1.914977in}}%
\pgfpathlineto{\pgfqpoint{3.747840in}{1.915576in}}%
\pgfpathlineto{\pgfqpoint{3.749080in}{1.914923in}}%
\pgfpathlineto{\pgfqpoint{3.750320in}{1.915782in}}%
\pgfpathlineto{\pgfqpoint{3.754040in}{1.913351in}}%
\pgfpathlineto{\pgfqpoint{3.756520in}{1.911732in}}%
\pgfpathlineto{\pgfqpoint{3.759000in}{1.913595in}}%
\pgfpathlineto{\pgfqpoint{3.761480in}{1.912888in}}%
\pgfpathlineto{\pgfqpoint{3.763960in}{1.915756in}}%
\pgfpathlineto{\pgfqpoint{3.765200in}{1.915936in}}%
\pgfpathlineto{\pgfqpoint{3.767680in}{1.914464in}}%
\pgfpathlineto{\pgfqpoint{3.770160in}{1.914674in}}%
\pgfpathlineto{\pgfqpoint{3.773880in}{1.921688in}}%
\pgfpathlineto{\pgfqpoint{3.776360in}{1.917428in}}%
\pgfpathlineto{\pgfqpoint{3.778840in}{1.920447in}}%
\pgfpathlineto{\pgfqpoint{3.781320in}{1.924871in}}%
\pgfpathlineto{\pgfqpoint{3.785040in}{1.925527in}}%
\pgfpathlineto{\pgfqpoint{3.791240in}{1.914216in}}%
\pgfpathlineto{\pgfqpoint{3.796200in}{1.918661in}}%
\pgfpathlineto{\pgfqpoint{3.797440in}{1.918952in}}%
\pgfpathlineto{\pgfqpoint{3.798680in}{1.920865in}}%
\pgfpathlineto{\pgfqpoint{3.801160in}{1.919179in}}%
\pgfpathlineto{\pgfqpoint{3.803640in}{1.921461in}}%
\pgfpathlineto{\pgfqpoint{3.807360in}{1.918363in}}%
\pgfpathlineto{\pgfqpoint{3.809840in}{1.917311in}}%
\pgfpathlineto{\pgfqpoint{3.813560in}{1.920093in}}%
\pgfpathlineto{\pgfqpoint{3.819760in}{1.920491in}}%
\pgfpathlineto{\pgfqpoint{3.822240in}{1.922425in}}%
\pgfpathlineto{\pgfqpoint{3.823480in}{1.922513in}}%
\pgfpathlineto{\pgfqpoint{3.827200in}{1.927216in}}%
\pgfpathlineto{\pgfqpoint{3.828440in}{1.927840in}}%
\pgfpathlineto{\pgfqpoint{3.833400in}{1.922617in}}%
\pgfpathlineto{\pgfqpoint{3.837120in}{1.922385in}}%
\pgfpathlineto{\pgfqpoint{3.838360in}{1.920737in}}%
\pgfpathlineto{\pgfqpoint{3.842080in}{1.923083in}}%
\pgfpathlineto{\pgfqpoint{3.843320in}{1.922554in}}%
\pgfpathlineto{\pgfqpoint{3.845800in}{1.920718in}}%
\pgfpathlineto{\pgfqpoint{3.848280in}{1.925722in}}%
\pgfpathlineto{\pgfqpoint{3.850760in}{1.921312in}}%
\pgfpathlineto{\pgfqpoint{3.852000in}{1.918910in}}%
\pgfpathlineto{\pgfqpoint{3.853240in}{1.922835in}}%
\pgfpathlineto{\pgfqpoint{3.859440in}{1.920183in}}%
\pgfpathlineto{\pgfqpoint{3.861920in}{1.924383in}}%
\pgfpathlineto{\pgfqpoint{3.863160in}{1.925631in}}%
\pgfpathlineto{\pgfqpoint{3.864400in}{1.924477in}}%
\pgfpathlineto{\pgfqpoint{3.866880in}{1.928451in}}%
\pgfpathlineto{\pgfqpoint{3.869360in}{1.927477in}}%
\pgfpathlineto{\pgfqpoint{3.870600in}{1.925315in}}%
\pgfpathlineto{\pgfqpoint{3.871840in}{1.926505in}}%
\pgfpathlineto{\pgfqpoint{3.874320in}{1.926398in}}%
\pgfpathlineto{\pgfqpoint{3.878040in}{1.924445in}}%
\pgfpathlineto{\pgfqpoint{3.879280in}{1.923447in}}%
\pgfpathlineto{\pgfqpoint{3.883000in}{1.925030in}}%
\pgfpathlineto{\pgfqpoint{3.885480in}{1.923867in}}%
\pgfpathlineto{\pgfqpoint{3.887960in}{1.926831in}}%
\pgfpathlineto{\pgfqpoint{3.891680in}{1.925695in}}%
\pgfpathlineto{\pgfqpoint{3.894160in}{1.925174in}}%
\pgfpathlineto{\pgfqpoint{3.897880in}{1.931155in}}%
\pgfpathlineto{\pgfqpoint{3.900360in}{1.927586in}}%
\pgfpathlineto{\pgfqpoint{3.902840in}{1.930521in}}%
\pgfpathlineto{\pgfqpoint{3.905320in}{1.935480in}}%
\pgfpathlineto{\pgfqpoint{3.909040in}{1.936131in}}%
\pgfpathlineto{\pgfqpoint{3.915240in}{1.925222in}}%
\pgfpathlineto{\pgfqpoint{3.922680in}{1.931788in}}%
\pgfpathlineto{\pgfqpoint{3.925160in}{1.929493in}}%
\pgfpathlineto{\pgfqpoint{3.927640in}{1.932864in}}%
\pgfpathlineto{\pgfqpoint{3.935080in}{1.929923in}}%
\pgfpathlineto{\pgfqpoint{3.936320in}{1.931877in}}%
\pgfpathlineto{\pgfqpoint{3.942520in}{1.930896in}}%
\pgfpathlineto{\pgfqpoint{3.946240in}{1.933913in}}%
\pgfpathlineto{\pgfqpoint{3.947480in}{1.934011in}}%
\pgfpathlineto{\pgfqpoint{3.951200in}{1.939429in}}%
\pgfpathlineto{\pgfqpoint{3.952440in}{1.939800in}}%
\pgfpathlineto{\pgfqpoint{3.957400in}{1.934091in}}%
\pgfpathlineto{\pgfqpoint{3.959880in}{1.934107in}}%
\pgfpathlineto{\pgfqpoint{3.963600in}{1.932186in}}%
\pgfpathlineto{\pgfqpoint{3.966080in}{1.933073in}}%
\pgfpathlineto{\pgfqpoint{3.969800in}{1.931202in}}%
\pgfpathlineto{\pgfqpoint{3.972280in}{1.936831in}}%
\pgfpathlineto{\pgfqpoint{3.976000in}{1.930485in}}%
\pgfpathlineto{\pgfqpoint{3.977240in}{1.933431in}}%
\pgfpathlineto{\pgfqpoint{3.980960in}{1.931669in}}%
\pgfpathlineto{\pgfqpoint{3.983440in}{1.930091in}}%
\pgfpathlineto{\pgfqpoint{3.987160in}{1.935356in}}%
\pgfpathlineto{\pgfqpoint{3.988400in}{1.933914in}}%
\pgfpathlineto{\pgfqpoint{3.992120in}{1.937095in}}%
\pgfpathlineto{\pgfqpoint{3.993360in}{1.936929in}}%
\pgfpathlineto{\pgfqpoint{3.994600in}{1.935033in}}%
\pgfpathlineto{\pgfqpoint{3.995840in}{1.936718in}}%
\pgfpathlineto{\pgfqpoint{3.997080in}{1.936236in}}%
\pgfpathlineto{\pgfqpoint{3.999560in}{1.936959in}}%
\pgfpathlineto{\pgfqpoint{4.004520in}{1.934503in}}%
\pgfpathlineto{\pgfqpoint{4.007000in}{1.935523in}}%
\pgfpathlineto{\pgfqpoint{4.009480in}{1.935237in}}%
\pgfpathlineto{\pgfqpoint{4.011960in}{1.938359in}}%
\pgfpathlineto{\pgfqpoint{4.015680in}{1.936184in}}%
\pgfpathlineto{\pgfqpoint{4.016920in}{1.935311in}}%
\pgfpathlineto{\pgfqpoint{4.018160in}{1.935870in}}%
\pgfpathlineto{\pgfqpoint{4.021880in}{1.942012in}}%
\pgfpathlineto{\pgfqpoint{4.024360in}{1.938420in}}%
\pgfpathlineto{\pgfqpoint{4.026840in}{1.941830in}}%
\pgfpathlineto{\pgfqpoint{4.029320in}{1.947148in}}%
\pgfpathlineto{\pgfqpoint{4.031800in}{1.948451in}}%
\pgfpathlineto{\pgfqpoint{4.033040in}{1.948122in}}%
\pgfpathlineto{\pgfqpoint{4.036760in}{1.941098in}}%
\pgfpathlineto{\pgfqpoint{4.039240in}{1.937869in}}%
\pgfpathlineto{\pgfqpoint{4.040480in}{1.938441in}}%
\pgfpathlineto{\pgfqpoint{4.042960in}{1.941108in}}%
\pgfpathlineto{\pgfqpoint{4.046680in}{1.943384in}}%
\pgfpathlineto{\pgfqpoint{4.049160in}{1.941043in}}%
\pgfpathlineto{\pgfqpoint{4.051640in}{1.944907in}}%
\pgfpathlineto{\pgfqpoint{4.059080in}{1.941584in}}%
\pgfpathlineto{\pgfqpoint{4.060320in}{1.943400in}}%
\pgfpathlineto{\pgfqpoint{4.064040in}{1.942139in}}%
\pgfpathlineto{\pgfqpoint{4.071480in}{1.945334in}}%
\pgfpathlineto{\pgfqpoint{4.075200in}{1.950301in}}%
\pgfpathlineto{\pgfqpoint{4.076440in}{1.950034in}}%
\pgfpathlineto{\pgfqpoint{4.081400in}{1.943028in}}%
\pgfpathlineto{\pgfqpoint{4.082640in}{1.943129in}}%
\pgfpathlineto{\pgfqpoint{4.087600in}{1.940221in}}%
\pgfpathlineto{\pgfqpoint{4.090080in}{1.940970in}}%
\pgfpathlineto{\pgfqpoint{4.093800in}{1.939364in}}%
\pgfpathlineto{\pgfqpoint{4.096280in}{1.945210in}}%
\pgfpathlineto{\pgfqpoint{4.100000in}{1.939413in}}%
\pgfpathlineto{\pgfqpoint{4.101240in}{1.940823in}}%
\pgfpathlineto{\pgfqpoint{4.107440in}{1.937836in}}%
\pgfpathlineto{\pgfqpoint{4.111160in}{1.943509in}}%
\pgfpathlineto{\pgfqpoint{4.112400in}{1.942451in}}%
\pgfpathlineto{\pgfqpoint{4.114880in}{1.945906in}}%
\pgfpathlineto{\pgfqpoint{4.118600in}{1.941292in}}%
\pgfpathlineto{\pgfqpoint{4.121080in}{1.942004in}}%
\pgfpathlineto{\pgfqpoint{4.123560in}{1.942087in}}%
\pgfpathlineto{\pgfqpoint{4.128520in}{1.939364in}}%
\pgfpathlineto{\pgfqpoint{4.131000in}{1.940569in}}%
\pgfpathlineto{\pgfqpoint{4.133480in}{1.940832in}}%
\pgfpathlineto{\pgfqpoint{4.135960in}{1.943439in}}%
\pgfpathlineto{\pgfqpoint{4.138440in}{1.942797in}}%
\pgfpathlineto{\pgfqpoint{4.142160in}{1.941859in}}%
\pgfpathlineto{\pgfqpoint{4.145880in}{1.947913in}}%
\pgfpathlineto{\pgfqpoint{4.148360in}{1.943611in}}%
\pgfpathlineto{\pgfqpoint{4.150840in}{1.947439in}}%
\pgfpathlineto{\pgfqpoint{4.153320in}{1.952648in}}%
\pgfpathlineto{\pgfqpoint{4.155800in}{1.953822in}}%
\pgfpathlineto{\pgfqpoint{4.159520in}{1.951394in}}%
\pgfpathlineto{\pgfqpoint{4.162000in}{1.944828in}}%
\pgfpathlineto{\pgfqpoint{4.163240in}{1.944180in}}%
\pgfpathlineto{\pgfqpoint{4.164480in}{1.944810in}}%
\pgfpathlineto{\pgfqpoint{4.166960in}{1.947789in}}%
\pgfpathlineto{\pgfqpoint{4.170680in}{1.951056in}}%
\pgfpathlineto{\pgfqpoint{4.173160in}{1.948849in}}%
\pgfpathlineto{\pgfqpoint{4.175640in}{1.952750in}}%
\pgfpathlineto{\pgfqpoint{4.176880in}{1.952390in}}%
\pgfpathlineto{\pgfqpoint{4.180600in}{1.949469in}}%
\pgfpathlineto{\pgfqpoint{4.183080in}{1.949767in}}%
\pgfpathlineto{\pgfqpoint{4.184320in}{1.951173in}}%
\pgfpathlineto{\pgfqpoint{4.188040in}{1.949444in}}%
\pgfpathlineto{\pgfqpoint{4.190520in}{1.949246in}}%
\pgfpathlineto{\pgfqpoint{4.193000in}{1.952854in}}%
\pgfpathlineto{\pgfqpoint{4.195480in}{1.954782in}}%
\pgfpathlineto{\pgfqpoint{4.197960in}{1.959403in}}%
\pgfpathlineto{\pgfqpoint{4.199200in}{1.959959in}}%
\pgfpathlineto{\pgfqpoint{4.200440in}{1.959233in}}%
\pgfpathlineto{\pgfqpoint{4.205400in}{1.950775in}}%
\pgfpathlineto{\pgfqpoint{4.206640in}{1.950645in}}%
\pgfpathlineto{\pgfqpoint{4.210360in}{1.947174in}}%
\pgfpathlineto{\pgfqpoint{4.214080in}{1.948721in}}%
\pgfpathlineto{\pgfqpoint{4.217800in}{1.946322in}}%
\pgfpathlineto{\pgfqpoint{4.220280in}{1.953172in}}%
\pgfpathlineto{\pgfqpoint{4.222760in}{1.949127in}}%
\pgfpathlineto{\pgfqpoint{4.224000in}{1.947429in}}%
\pgfpathlineto{\pgfqpoint{4.226480in}{1.948667in}}%
\pgfpathlineto{\pgfqpoint{4.227720in}{1.948648in}}%
\pgfpathlineto{\pgfqpoint{4.231440in}{1.945263in}}%
\pgfpathlineto{\pgfqpoint{4.235160in}{1.951071in}}%
\pgfpathlineto{\pgfqpoint{4.236400in}{1.950156in}}%
\pgfpathlineto{\pgfqpoint{4.238880in}{1.954497in}}%
\pgfpathlineto{\pgfqpoint{4.242600in}{1.948768in}}%
\pgfpathlineto{\pgfqpoint{4.245080in}{1.949886in}}%
\pgfpathlineto{\pgfqpoint{4.247560in}{1.949332in}}%
\pgfpathlineto{\pgfqpoint{4.250040in}{1.947738in}}%
\pgfpathlineto{\pgfqpoint{4.252520in}{1.945824in}}%
\pgfpathlineto{\pgfqpoint{4.255000in}{1.947407in}}%
\pgfpathlineto{\pgfqpoint{4.257480in}{1.947486in}}%
\pgfpathlineto{\pgfqpoint{4.259960in}{1.949895in}}%
\pgfpathlineto{\pgfqpoint{4.261200in}{1.949957in}}%
\pgfpathlineto{\pgfqpoint{4.264920in}{1.947446in}}%
\pgfpathlineto{\pgfqpoint{4.266160in}{1.947799in}}%
\pgfpathlineto{\pgfqpoint{4.268640in}{1.952686in}}%
\pgfpathlineto{\pgfqpoint{4.269880in}{1.953488in}}%
\pgfpathlineto{\pgfqpoint{4.272360in}{1.948736in}}%
\pgfpathlineto{\pgfqpoint{4.274840in}{1.952882in}}%
\pgfpathlineto{\pgfqpoint{4.277320in}{1.958062in}}%
\pgfpathlineto{\pgfqpoint{4.279800in}{1.959040in}}%
\pgfpathlineto{\pgfqpoint{4.283520in}{1.956291in}}%
\pgfpathlineto{\pgfqpoint{4.286000in}{1.950036in}}%
\pgfpathlineto{\pgfqpoint{4.288480in}{1.950478in}}%
\pgfpathlineto{\pgfqpoint{4.290960in}{1.954285in}}%
\pgfpathlineto{\pgfqpoint{4.294680in}{1.957408in}}%
\pgfpathlineto{\pgfqpoint{4.297160in}{1.955433in}}%
\pgfpathlineto{\pgfqpoint{4.299640in}{1.957796in}}%
\pgfpathlineto{\pgfqpoint{4.305840in}{1.954807in}}%
\pgfpathlineto{\pgfqpoint{4.308320in}{1.957321in}}%
\pgfpathlineto{\pgfqpoint{4.309560in}{1.956960in}}%
\pgfpathlineto{\pgfqpoint{4.310800in}{1.958184in}}%
\pgfpathlineto{\pgfqpoint{4.314520in}{1.956898in}}%
\pgfpathlineto{\pgfqpoint{4.318240in}{1.962538in}}%
\pgfpathlineto{\pgfqpoint{4.319480in}{1.963726in}}%
\pgfpathlineto{\pgfqpoint{4.321960in}{1.968536in}}%
\pgfpathlineto{\pgfqpoint{4.324440in}{1.969138in}}%
\pgfpathlineto{\pgfqpoint{4.329400in}{1.960825in}}%
\pgfpathlineto{\pgfqpoint{4.336840in}{1.955652in}}%
\pgfpathlineto{\pgfqpoint{4.339320in}{1.955116in}}%
\pgfpathlineto{\pgfqpoint{4.341800in}{1.953795in}}%
\pgfpathlineto{\pgfqpoint{4.344280in}{1.959757in}}%
\pgfpathlineto{\pgfqpoint{4.349240in}{1.953791in}}%
\pgfpathlineto{\pgfqpoint{4.351720in}{1.953909in}}%
\pgfpathlineto{\pgfqpoint{4.355440in}{1.950643in}}%
\pgfpathlineto{\pgfqpoint{4.359160in}{1.957071in}}%
\pgfpathlineto{\pgfqpoint{4.360400in}{1.955813in}}%
\pgfpathlineto{\pgfqpoint{4.362880in}{1.959462in}}%
\pgfpathlineto{\pgfqpoint{4.366600in}{1.953906in}}%
\pgfpathlineto{\pgfqpoint{4.367840in}{1.955140in}}%
\pgfpathlineto{\pgfqpoint{4.371560in}{1.954023in}}%
\pgfpathlineto{\pgfqpoint{4.374040in}{1.952716in}}%
\pgfpathlineto{\pgfqpoint{4.375280in}{1.950740in}}%
\pgfpathlineto{\pgfqpoint{4.376520in}{1.951068in}}%
\pgfpathlineto{\pgfqpoint{4.379000in}{1.952746in}}%
\pgfpathlineto{\pgfqpoint{4.381480in}{1.953043in}}%
\pgfpathlineto{\pgfqpoint{4.383960in}{1.955330in}}%
\pgfpathlineto{\pgfqpoint{4.385200in}{1.955521in}}%
\pgfpathlineto{\pgfqpoint{4.387680in}{1.953681in}}%
\pgfpathlineto{\pgfqpoint{4.390160in}{1.952475in}}%
\pgfpathlineto{\pgfqpoint{4.392640in}{1.957641in}}%
\pgfpathlineto{\pgfqpoint{4.393880in}{1.958611in}}%
\pgfpathlineto{\pgfqpoint{4.396360in}{1.954161in}}%
\pgfpathlineto{\pgfqpoint{4.398840in}{1.958486in}}%
\pgfpathlineto{\pgfqpoint{4.401320in}{1.964154in}}%
\pgfpathlineto{\pgfqpoint{4.403800in}{1.964696in}}%
\pgfpathlineto{\pgfqpoint{4.407520in}{1.963083in}}%
\pgfpathlineto{\pgfqpoint{4.410000in}{1.957021in}}%
\pgfpathlineto{\pgfqpoint{4.412480in}{1.957261in}}%
\pgfpathlineto{\pgfqpoint{4.414960in}{1.960453in}}%
\pgfpathlineto{\pgfqpoint{4.418680in}{1.963660in}}%
\pgfpathlineto{\pgfqpoint{4.421160in}{1.961890in}}%
\pgfpathlineto{\pgfqpoint{4.423640in}{1.964634in}}%
\pgfpathlineto{\pgfqpoint{4.429840in}{1.961939in}}%
\pgfpathlineto{\pgfqpoint{4.432320in}{1.964541in}}%
\pgfpathlineto{\pgfqpoint{4.434800in}{1.964854in}}%
\pgfpathlineto{\pgfqpoint{4.438520in}{1.964957in}}%
\pgfpathlineto{\pgfqpoint{4.441000in}{1.968715in}}%
\pgfpathlineto{\pgfqpoint{4.448440in}{1.975442in}}%
\pgfpathlineto{\pgfqpoint{4.454640in}{1.965480in}}%
\pgfpathlineto{\pgfqpoint{4.458360in}{1.961292in}}%
\pgfpathlineto{\pgfqpoint{4.462080in}{1.962241in}}%
\pgfpathlineto{\pgfqpoint{4.465800in}{1.960607in}}%
\pgfpathlineto{\pgfqpoint{4.468280in}{1.966252in}}%
\pgfpathlineto{\pgfqpoint{4.472000in}{1.960263in}}%
\pgfpathlineto{\pgfqpoint{4.475720in}{1.962701in}}%
\pgfpathlineto{\pgfqpoint{4.479440in}{1.960287in}}%
\pgfpathlineto{\pgfqpoint{4.483160in}{1.967557in}}%
\pgfpathlineto{\pgfqpoint{4.484400in}{1.966138in}}%
\pgfpathlineto{\pgfqpoint{4.486880in}{1.970267in}}%
\pgfpathlineto{\pgfqpoint{4.490600in}{1.965513in}}%
\pgfpathlineto{\pgfqpoint{4.491840in}{1.966089in}}%
\pgfpathlineto{\pgfqpoint{4.493080in}{1.964781in}}%
\pgfpathlineto{\pgfqpoint{4.495560in}{1.965224in}}%
\pgfpathlineto{\pgfqpoint{4.498040in}{1.963202in}}%
\pgfpathlineto{\pgfqpoint{4.499280in}{1.961129in}}%
\pgfpathlineto{\pgfqpoint{4.505480in}{1.962502in}}%
\pgfpathlineto{\pgfqpoint{4.507960in}{1.964274in}}%
\pgfpathlineto{\pgfqpoint{4.509200in}{1.963874in}}%
\pgfpathlineto{\pgfqpoint{4.511680in}{1.961497in}}%
\pgfpathlineto{\pgfqpoint{4.512920in}{1.960396in}}%
\pgfpathlineto{\pgfqpoint{4.514160in}{1.960899in}}%
\pgfpathlineto{\pgfqpoint{4.516640in}{1.966381in}}%
\pgfpathlineto{\pgfqpoint{4.517880in}{1.968247in}}%
\pgfpathlineto{\pgfqpoint{4.520360in}{1.963259in}}%
\pgfpathlineto{\pgfqpoint{4.522840in}{1.966637in}}%
\pgfpathlineto{\pgfqpoint{4.525320in}{1.972875in}}%
\pgfpathlineto{\pgfqpoint{4.529040in}{1.972598in}}%
\pgfpathlineto{\pgfqpoint{4.531520in}{1.971192in}}%
\pgfpathlineto{\pgfqpoint{4.534000in}{1.965067in}}%
\pgfpathlineto{\pgfqpoint{4.535240in}{1.964536in}}%
\pgfpathlineto{\pgfqpoint{4.541440in}{1.969328in}}%
\pgfpathlineto{\pgfqpoint{4.542680in}{1.971245in}}%
\pgfpathlineto{\pgfqpoint{4.545160in}{1.969075in}}%
\pgfpathlineto{\pgfqpoint{4.547640in}{1.971782in}}%
\pgfpathlineto{\pgfqpoint{4.552600in}{1.968542in}}%
\pgfpathlineto{\pgfqpoint{4.553840in}{1.968583in}}%
\pgfpathlineto{\pgfqpoint{4.556320in}{1.971285in}}%
\pgfpathlineto{\pgfqpoint{4.557560in}{1.970969in}}%
\pgfpathlineto{\pgfqpoint{4.560040in}{1.972223in}}%
\pgfpathlineto{\pgfqpoint{4.562520in}{1.971352in}}%
\pgfpathlineto{\pgfqpoint{4.565000in}{1.975709in}}%
\pgfpathlineto{\pgfqpoint{4.568720in}{1.981473in}}%
\pgfpathlineto{\pgfqpoint{4.571200in}{1.983340in}}%
\pgfpathlineto{\pgfqpoint{4.572440in}{1.982937in}}%
\pgfpathlineto{\pgfqpoint{4.578640in}{1.973658in}}%
\pgfpathlineto{\pgfqpoint{4.583600in}{1.969389in}}%
\pgfpathlineto{\pgfqpoint{4.587320in}{1.969563in}}%
\pgfpathlineto{\pgfqpoint{4.588560in}{1.968301in}}%
\pgfpathlineto{\pgfqpoint{4.589800in}{1.968983in}}%
\pgfpathlineto{\pgfqpoint{4.592280in}{1.974008in}}%
\pgfpathlineto{\pgfqpoint{4.596000in}{1.967537in}}%
\pgfpathlineto{\pgfqpoint{4.597240in}{1.970238in}}%
\pgfpathlineto{\pgfqpoint{4.602200in}{1.969642in}}%
\pgfpathlineto{\pgfqpoint{4.603440in}{1.967621in}}%
\pgfpathlineto{\pgfqpoint{4.607160in}{1.975988in}}%
\pgfpathlineto{\pgfqpoint{4.608400in}{1.975091in}}%
\pgfpathlineto{\pgfqpoint{4.610880in}{1.979905in}}%
\pgfpathlineto{\pgfqpoint{4.614600in}{1.974749in}}%
\pgfpathlineto{\pgfqpoint{4.618320in}{1.973752in}}%
\pgfpathlineto{\pgfqpoint{4.620800in}{1.972414in}}%
\pgfpathlineto{\pgfqpoint{4.624520in}{1.968973in}}%
\pgfpathlineto{\pgfqpoint{4.627000in}{1.969671in}}%
\pgfpathlineto{\pgfqpoint{4.629480in}{1.970510in}}%
\pgfpathlineto{\pgfqpoint{4.630720in}{1.972253in}}%
\pgfpathlineto{\pgfqpoint{4.638160in}{1.970030in}}%
\pgfpathlineto{\pgfqpoint{4.640640in}{1.975798in}}%
\pgfpathlineto{\pgfqpoint{4.641880in}{1.977521in}}%
\pgfpathlineto{\pgfqpoint{4.644360in}{1.972176in}}%
\pgfpathlineto{\pgfqpoint{4.646840in}{1.975584in}}%
\pgfpathlineto{\pgfqpoint{4.649320in}{1.982093in}}%
\pgfpathlineto{\pgfqpoint{4.651800in}{1.982599in}}%
\pgfpathlineto{\pgfqpoint{4.655520in}{1.979955in}}%
\pgfpathlineto{\pgfqpoint{4.658000in}{1.973407in}}%
\pgfpathlineto{\pgfqpoint{4.659240in}{1.972214in}}%
\pgfpathlineto{\pgfqpoint{4.662960in}{1.974307in}}%
\pgfpathlineto{\pgfqpoint{4.667920in}{1.976821in}}%
\pgfpathlineto{\pgfqpoint{4.669160in}{1.975467in}}%
\pgfpathlineto{\pgfqpoint{4.672880in}{1.977110in}}%
\pgfpathlineto{\pgfqpoint{4.675360in}{1.976071in}}%
\pgfpathlineto{\pgfqpoint{4.677840in}{1.976637in}}%
\pgfpathlineto{\pgfqpoint{4.680320in}{1.979841in}}%
\pgfpathlineto{\pgfqpoint{4.682800in}{1.979348in}}%
\pgfpathlineto{\pgfqpoint{4.684040in}{1.980075in}}%
\pgfpathlineto{\pgfqpoint{4.686520in}{1.978033in}}%
\pgfpathlineto{\pgfqpoint{4.690240in}{1.983886in}}%
\pgfpathlineto{\pgfqpoint{4.691480in}{1.985258in}}%
\pgfpathlineto{\pgfqpoint{4.693960in}{1.990822in}}%
\pgfpathlineto{\pgfqpoint{4.695200in}{1.991514in}}%
\pgfpathlineto{\pgfqpoint{4.697680in}{1.987196in}}%
\pgfpathlineto{\pgfqpoint{4.701400in}{1.981629in}}%
\pgfpathlineto{\pgfqpoint{4.707600in}{1.975279in}}%
\pgfpathlineto{\pgfqpoint{4.710080in}{1.975407in}}%
\pgfpathlineto{\pgfqpoint{4.713800in}{1.974210in}}%
\pgfpathlineto{\pgfqpoint{4.716280in}{1.979107in}}%
\pgfpathlineto{\pgfqpoint{4.721240in}{1.972897in}}%
\pgfpathlineto{\pgfqpoint{4.723720in}{1.974731in}}%
\pgfpathlineto{\pgfqpoint{4.726200in}{1.973874in}}%
\pgfpathlineto{\pgfqpoint{4.727440in}{1.971960in}}%
\pgfpathlineto{\pgfqpoint{4.731160in}{1.980476in}}%
\pgfpathlineto{\pgfqpoint{4.732400in}{1.979914in}}%
\pgfpathlineto{\pgfqpoint{4.734880in}{1.985657in}}%
\pgfpathlineto{\pgfqpoint{4.739840in}{1.981760in}}%
\pgfpathlineto{\pgfqpoint{4.742320in}{1.981252in}}%
\pgfpathlineto{\pgfqpoint{4.743560in}{1.981351in}}%
\pgfpathlineto{\pgfqpoint{4.748520in}{1.976063in}}%
\pgfpathlineto{\pgfqpoint{4.751000in}{1.977814in}}%
\pgfpathlineto{\pgfqpoint{4.753480in}{1.978394in}}%
\pgfpathlineto{\pgfqpoint{4.754720in}{1.979912in}}%
\pgfpathlineto{\pgfqpoint{4.759680in}{1.978297in}}%
\pgfpathlineto{\pgfqpoint{4.760920in}{1.976826in}}%
\pgfpathlineto{\pgfqpoint{4.762160in}{1.977202in}}%
\pgfpathlineto{\pgfqpoint{4.764640in}{1.982421in}}%
\pgfpathlineto{\pgfqpoint{4.765880in}{1.983743in}}%
\pgfpathlineto{\pgfqpoint{4.768360in}{1.978663in}}%
\pgfpathlineto{\pgfqpoint{4.770840in}{1.982748in}}%
\pgfpathlineto{\pgfqpoint{4.773320in}{1.988010in}}%
\pgfpathlineto{\pgfqpoint{4.775800in}{1.988524in}}%
\pgfpathlineto{\pgfqpoint{4.777040in}{1.988051in}}%
\pgfpathlineto{\pgfqpoint{4.783240in}{1.977856in}}%
\pgfpathlineto{\pgfqpoint{4.790680in}{1.981913in}}%
\pgfpathlineto{\pgfqpoint{4.793160in}{1.979143in}}%
\pgfpathlineto{\pgfqpoint{4.796880in}{1.981318in}}%
\pgfpathlineto{\pgfqpoint{4.799360in}{1.980641in}}%
\pgfpathlineto{\pgfqpoint{4.805560in}{1.983513in}}%
\pgfpathlineto{\pgfqpoint{4.809280in}{1.982613in}}%
\pgfpathlineto{\pgfqpoint{4.810520in}{1.981195in}}%
\pgfpathlineto{\pgfqpoint{4.813000in}{1.983986in}}%
\pgfpathlineto{\pgfqpoint{4.816720in}{1.990341in}}%
\pgfpathlineto{\pgfqpoint{4.819200in}{1.993668in}}%
\pgfpathlineto{\pgfqpoint{4.821680in}{1.989509in}}%
\pgfpathlineto{\pgfqpoint{4.822920in}{1.987764in}}%
\pgfpathlineto{\pgfqpoint{4.824160in}{1.988240in}}%
\pgfpathlineto{\pgfqpoint{4.829120in}{1.982124in}}%
\pgfpathlineto{\pgfqpoint{4.834080in}{1.979284in}}%
\pgfpathlineto{\pgfqpoint{4.837800in}{1.976614in}}%
\pgfpathlineto{\pgfqpoint{4.840280in}{1.982474in}}%
\pgfpathlineto{\pgfqpoint{4.844000in}{1.975894in}}%
\pgfpathlineto{\pgfqpoint{4.845240in}{1.973338in}}%
\pgfpathlineto{\pgfqpoint{4.847720in}{1.974268in}}%
\pgfpathlineto{\pgfqpoint{4.851440in}{1.971206in}}%
\pgfpathlineto{\pgfqpoint{4.855160in}{1.979002in}}%
\pgfpathlineto{\pgfqpoint{4.856400in}{1.979280in}}%
\pgfpathlineto{\pgfqpoint{4.858880in}{1.985190in}}%
\pgfpathlineto{\pgfqpoint{4.860120in}{1.983866in}}%
\pgfpathlineto{\pgfqpoint{4.861360in}{1.983868in}}%
\pgfpathlineto{\pgfqpoint{4.863840in}{1.981392in}}%
\pgfpathlineto{\pgfqpoint{4.865080in}{1.981064in}}%
\pgfpathlineto{\pgfqpoint{4.867560in}{1.982269in}}%
\pgfpathlineto{\pgfqpoint{4.871280in}{1.977913in}}%
\pgfpathlineto{\pgfqpoint{4.872520in}{1.978130in}}%
\pgfpathlineto{\pgfqpoint{4.875000in}{1.979488in}}%
\pgfpathlineto{\pgfqpoint{4.877480in}{1.979221in}}%
\pgfpathlineto{\pgfqpoint{4.879960in}{1.980153in}}%
\pgfpathlineto{\pgfqpoint{4.882440in}{1.979086in}}%
\pgfpathlineto{\pgfqpoint{4.883680in}{1.978856in}}%
\pgfpathlineto{\pgfqpoint{4.886160in}{1.977629in}}%
\pgfpathlineto{\pgfqpoint{4.888640in}{1.982846in}}%
\pgfpathlineto{\pgfqpoint{4.889880in}{1.984392in}}%
\pgfpathlineto{\pgfqpoint{4.892360in}{1.979754in}}%
\pgfpathlineto{\pgfqpoint{4.897320in}{1.989172in}}%
\pgfpathlineto{\pgfqpoint{4.899800in}{1.989700in}}%
\pgfpathlineto{\pgfqpoint{4.903520in}{1.986308in}}%
\pgfpathlineto{\pgfqpoint{4.907240in}{1.977190in}}%
\pgfpathlineto{\pgfqpoint{4.910960in}{1.977931in}}%
\pgfpathlineto{\pgfqpoint{4.913440in}{1.979031in}}%
\pgfpathlineto{\pgfqpoint{4.914680in}{1.980586in}}%
\pgfpathlineto{\pgfqpoint{4.917160in}{1.978686in}}%
\pgfpathlineto{\pgfqpoint{4.919640in}{1.981310in}}%
\pgfpathlineto{\pgfqpoint{4.923360in}{1.980155in}}%
\pgfpathlineto{\pgfqpoint{4.928320in}{1.984655in}}%
\pgfpathlineto{\pgfqpoint{4.930800in}{1.983312in}}%
\pgfpathlineto{\pgfqpoint{4.932040in}{1.984286in}}%
\pgfpathlineto{\pgfqpoint{4.934520in}{1.982076in}}%
\pgfpathlineto{\pgfqpoint{4.937000in}{1.985762in}}%
\pgfpathlineto{\pgfqpoint{4.939480in}{1.987560in}}%
\pgfpathlineto{\pgfqpoint{4.941960in}{1.994115in}}%
\pgfpathlineto{\pgfqpoint{4.943200in}{1.995022in}}%
\pgfpathlineto{\pgfqpoint{4.944440in}{1.993740in}}%
\pgfpathlineto{\pgfqpoint{4.946920in}{1.988867in}}%
\pgfpathlineto{\pgfqpoint{4.948160in}{1.989350in}}%
\pgfpathlineto{\pgfqpoint{4.951880in}{1.984613in}}%
\pgfpathlineto{\pgfqpoint{4.955600in}{1.982728in}}%
\pgfpathlineto{\pgfqpoint{4.959320in}{1.980610in}}%
\pgfpathlineto{\pgfqpoint{4.960560in}{1.978316in}}%
\pgfpathlineto{\pgfqpoint{4.961800in}{1.978475in}}%
\pgfpathlineto{\pgfqpoint{4.964280in}{1.983923in}}%
\pgfpathlineto{\pgfqpoint{4.966760in}{1.979403in}}%
\pgfpathlineto{\pgfqpoint{4.969240in}{1.975099in}}%
\pgfpathlineto{\pgfqpoint{4.971720in}{1.976445in}}%
\pgfpathlineto{\pgfqpoint{4.974200in}{1.974938in}}%
\pgfpathlineto{\pgfqpoint{4.975440in}{1.972583in}}%
\pgfpathlineto{\pgfqpoint{4.979160in}{1.979127in}}%
\pgfpathlineto{\pgfqpoint{4.980400in}{1.979560in}}%
\pgfpathlineto{\pgfqpoint{4.982880in}{1.984539in}}%
\pgfpathlineto{\pgfqpoint{4.984120in}{1.983223in}}%
\pgfpathlineto{\pgfqpoint{4.985360in}{1.984408in}}%
\pgfpathlineto{\pgfqpoint{4.989080in}{1.982053in}}%
\pgfpathlineto{\pgfqpoint{4.991560in}{1.982759in}}%
\pgfpathlineto{\pgfqpoint{4.995280in}{1.977807in}}%
\pgfpathlineto{\pgfqpoint{5.001480in}{1.976989in}}%
\pgfpathlineto{\pgfqpoint{5.003960in}{1.978593in}}%
\pgfpathlineto{\pgfqpoint{5.006440in}{1.978380in}}%
\pgfpathlineto{\pgfqpoint{5.010160in}{1.976856in}}%
\pgfpathlineto{\pgfqpoint{5.013880in}{1.983671in}}%
\pgfpathlineto{\pgfqpoint{5.016360in}{1.978653in}}%
\pgfpathlineto{\pgfqpoint{5.020080in}{1.988861in}}%
\pgfpathlineto{\pgfqpoint{5.025040in}{1.987410in}}%
\pgfpathlineto{\pgfqpoint{5.028760in}{1.981092in}}%
\pgfpathlineto{\pgfqpoint{5.031240in}{1.976394in}}%
\pgfpathlineto{\pgfqpoint{5.033720in}{1.977658in}}%
\pgfpathlineto{\pgfqpoint{5.034960in}{1.977389in}}%
\pgfpathlineto{\pgfqpoint{5.038680in}{1.982312in}}%
\pgfpathlineto{\pgfqpoint{5.041160in}{1.980434in}}%
\pgfpathlineto{\pgfqpoint{5.043640in}{1.983410in}}%
\pgfpathlineto{\pgfqpoint{5.046120in}{1.981125in}}%
\pgfpathlineto{\pgfqpoint{5.047360in}{1.980910in}}%
\pgfpathlineto{\pgfqpoint{5.048600in}{1.981892in}}%
\pgfpathlineto{\pgfqpoint{5.049840in}{1.981404in}}%
\pgfpathlineto{\pgfqpoint{5.052320in}{1.985870in}}%
\pgfpathlineto{\pgfqpoint{5.054800in}{1.985437in}}%
\pgfpathlineto{\pgfqpoint{5.056040in}{1.986498in}}%
\pgfpathlineto{\pgfqpoint{5.058520in}{1.984272in}}%
\pgfpathlineto{\pgfqpoint{5.059760in}{1.986635in}}%
\pgfpathlineto{\pgfqpoint{5.062240in}{1.986664in}}%
\pgfpathlineto{\pgfqpoint{5.063480in}{1.987967in}}%
\pgfpathlineto{\pgfqpoint{5.067200in}{1.996341in}}%
\pgfpathlineto{\pgfqpoint{5.077120in}{1.986212in}}%
\pgfpathlineto{\pgfqpoint{5.083320in}{1.984813in}}%
\pgfpathlineto{\pgfqpoint{5.085800in}{1.981983in}}%
\pgfpathlineto{\pgfqpoint{5.088280in}{1.986651in}}%
\pgfpathlineto{\pgfqpoint{5.090760in}{1.982800in}}%
\pgfpathlineto{\pgfqpoint{5.093240in}{1.978602in}}%
\pgfpathlineto{\pgfqpoint{5.095720in}{1.979389in}}%
\pgfpathlineto{\pgfqpoint{5.098200in}{1.977862in}}%
\pgfpathlineto{\pgfqpoint{5.099440in}{1.975096in}}%
\pgfpathlineto{\pgfqpoint{5.103160in}{1.981287in}}%
\pgfpathlineto{\pgfqpoint{5.104400in}{1.981477in}}%
\pgfpathlineto{\pgfqpoint{5.106880in}{1.988057in}}%
\pgfpathlineto{\pgfqpoint{5.108120in}{1.987360in}}%
\pgfpathlineto{\pgfqpoint{5.109360in}{1.989023in}}%
\pgfpathlineto{\pgfqpoint{5.114320in}{1.986424in}}%
\pgfpathlineto{\pgfqpoint{5.115560in}{1.986559in}}%
\pgfpathlineto{\pgfqpoint{5.119280in}{1.981959in}}%
\pgfpathlineto{\pgfqpoint{5.125480in}{1.982076in}}%
\pgfpathlineto{\pgfqpoint{5.127960in}{1.983602in}}%
\pgfpathlineto{\pgfqpoint{5.130440in}{1.984379in}}%
\pgfpathlineto{\pgfqpoint{5.134160in}{1.984250in}}%
\pgfpathlineto{\pgfqpoint{5.137880in}{1.991073in}}%
\pgfpathlineto{\pgfqpoint{5.140360in}{1.986440in}}%
\pgfpathlineto{\pgfqpoint{5.144080in}{1.993969in}}%
\pgfpathlineto{\pgfqpoint{5.149040in}{1.992449in}}%
\pgfpathlineto{\pgfqpoint{5.152760in}{1.986420in}}%
\pgfpathlineto{\pgfqpoint{5.155240in}{1.981064in}}%
\pgfpathlineto{\pgfqpoint{5.160200in}{1.983272in}}%
\pgfpathlineto{\pgfqpoint{5.162680in}{1.986453in}}%
\pgfpathlineto{\pgfqpoint{5.165160in}{1.985025in}}%
\pgfpathlineto{\pgfqpoint{5.167640in}{1.987725in}}%
\pgfpathlineto{\pgfqpoint{5.170120in}{1.984818in}}%
\pgfpathlineto{\pgfqpoint{5.171360in}{1.984397in}}%
\pgfpathlineto{\pgfqpoint{5.175080in}{1.986566in}}%
\pgfpathlineto{\pgfqpoint{5.176320in}{1.989601in}}%
\pgfpathlineto{\pgfqpoint{5.178800in}{1.988599in}}%
\pgfpathlineto{\pgfqpoint{5.180040in}{1.990181in}}%
\pgfpathlineto{\pgfqpoint{5.182520in}{1.987123in}}%
\pgfpathlineto{\pgfqpoint{5.183760in}{1.989758in}}%
\pgfpathlineto{\pgfqpoint{5.186240in}{1.990284in}}%
\pgfpathlineto{\pgfqpoint{5.188720in}{1.995941in}}%
\pgfpathlineto{\pgfqpoint{5.191200in}{2.000302in}}%
\pgfpathlineto{\pgfqpoint{5.194920in}{1.994827in}}%
\pgfpathlineto{\pgfqpoint{5.196160in}{1.996323in}}%
\pgfpathlineto{\pgfqpoint{5.198640in}{1.991593in}}%
\pgfpathlineto{\pgfqpoint{5.201120in}{1.989991in}}%
\pgfpathlineto{\pgfqpoint{5.203600in}{1.989361in}}%
\pgfpathlineto{\pgfqpoint{5.206080in}{1.988810in}}%
\pgfpathlineto{\pgfqpoint{5.209800in}{1.984333in}}%
\pgfpathlineto{\pgfqpoint{5.213520in}{1.988209in}}%
\pgfpathlineto{\pgfqpoint{5.216000in}{1.981673in}}%
\pgfpathlineto{\pgfqpoint{5.218480in}{1.987661in}}%
\pgfpathlineto{\pgfqpoint{5.220960in}{1.986460in}}%
\pgfpathlineto{\pgfqpoint{5.223440in}{1.981529in}}%
\pgfpathlineto{\pgfqpoint{5.227160in}{1.989071in}}%
\pgfpathlineto{\pgfqpoint{5.228400in}{1.988498in}}%
\pgfpathlineto{\pgfqpoint{5.233360in}{1.994955in}}%
\pgfpathlineto{\pgfqpoint{5.237080in}{1.992375in}}%
\pgfpathlineto{\pgfqpoint{5.239560in}{1.994063in}}%
\pgfpathlineto{\pgfqpoint{5.242040in}{1.991046in}}%
\pgfpathlineto{\pgfqpoint{5.244520in}{1.989720in}}%
\pgfpathlineto{\pgfqpoint{5.248240in}{1.989953in}}%
\pgfpathlineto{\pgfqpoint{5.251960in}{1.993011in}}%
\pgfpathlineto{\pgfqpoint{5.258160in}{1.994040in}}%
\pgfpathlineto{\pgfqpoint{5.261880in}{2.001794in}}%
\pgfpathlineto{\pgfqpoint{5.264360in}{1.995816in}}%
\pgfpathlineto{\pgfqpoint{5.268080in}{2.001569in}}%
\pgfpathlineto{\pgfqpoint{5.274280in}{1.999204in}}%
\pgfpathlineto{\pgfqpoint{5.275520in}{1.998041in}}%
\pgfpathlineto{\pgfqpoint{5.279240in}{1.988039in}}%
\pgfpathlineto{\pgfqpoint{5.282960in}{1.989764in}}%
\pgfpathlineto{\pgfqpoint{5.286680in}{1.994186in}}%
\pgfpathlineto{\pgfqpoint{5.289160in}{1.993078in}}%
\pgfpathlineto{\pgfqpoint{5.291640in}{1.995618in}}%
\pgfpathlineto{\pgfqpoint{5.294120in}{1.990763in}}%
\pgfpathlineto{\pgfqpoint{5.295360in}{1.990121in}}%
\pgfpathlineto{\pgfqpoint{5.296600in}{1.991249in}}%
\pgfpathlineto{\pgfqpoint{5.297840in}{1.990864in}}%
\pgfpathlineto{\pgfqpoint{5.300320in}{1.995092in}}%
\pgfpathlineto{\pgfqpoint{5.302800in}{1.993705in}}%
\pgfpathlineto{\pgfqpoint{5.304040in}{1.995470in}}%
\pgfpathlineto{\pgfqpoint{5.306520in}{1.991551in}}%
\pgfpathlineto{\pgfqpoint{5.307760in}{1.993648in}}%
\pgfpathlineto{\pgfqpoint{5.309000in}{1.993033in}}%
\pgfpathlineto{\pgfqpoint{5.311480in}{1.995178in}}%
\pgfpathlineto{\pgfqpoint{5.315200in}{2.003218in}}%
\pgfpathlineto{\pgfqpoint{5.318920in}{1.999378in}}%
\pgfpathlineto{\pgfqpoint{5.320160in}{2.001595in}}%
\pgfpathlineto{\pgfqpoint{5.322640in}{1.998145in}}%
\pgfpathlineto{\pgfqpoint{5.325120in}{1.995999in}}%
\pgfpathlineto{\pgfqpoint{5.328840in}{1.994134in}}%
\pgfpathlineto{\pgfqpoint{5.331320in}{1.992731in}}%
\pgfpathlineto{\pgfqpoint{5.333800in}{1.989957in}}%
\pgfpathlineto{\pgfqpoint{5.336280in}{1.993464in}}%
\pgfpathlineto{\pgfqpoint{5.337520in}{1.992574in}}%
\pgfpathlineto{\pgfqpoint{5.340000in}{1.987111in}}%
\pgfpathlineto{\pgfqpoint{5.342480in}{1.990615in}}%
\pgfpathlineto{\pgfqpoint{5.344960in}{1.990335in}}%
\pgfpathlineto{\pgfqpoint{5.347440in}{1.984427in}}%
\pgfpathlineto{\pgfqpoint{5.351160in}{1.991726in}}%
\pgfpathlineto{\pgfqpoint{5.352400in}{1.991472in}}%
\pgfpathlineto{\pgfqpoint{5.356120in}{1.997492in}}%
\pgfpathlineto{\pgfqpoint{5.357360in}{1.998704in}}%
\pgfpathlineto{\pgfqpoint{5.361080in}{1.994272in}}%
\pgfpathlineto{\pgfqpoint{5.363560in}{1.994673in}}%
\pgfpathlineto{\pgfqpoint{5.364800in}{1.992702in}}%
\pgfpathlineto{\pgfqpoint{5.366040in}{1.992979in}}%
\pgfpathlineto{\pgfqpoint{5.367280in}{1.991110in}}%
\pgfpathlineto{\pgfqpoint{5.371000in}{1.990906in}}%
\pgfpathlineto{\pgfqpoint{5.372240in}{1.990855in}}%
\pgfpathlineto{\pgfqpoint{5.379680in}{1.997244in}}%
\pgfpathlineto{\pgfqpoint{5.382160in}{1.997466in}}%
\pgfpathlineto{\pgfqpoint{5.385880in}{2.004068in}}%
\pgfpathlineto{\pgfqpoint{5.388360in}{1.999413in}}%
\pgfpathlineto{\pgfqpoint{5.389600in}{2.002021in}}%
\pgfpathlineto{\pgfqpoint{5.390840in}{2.001817in}}%
\pgfpathlineto{\pgfqpoint{5.392080in}{2.005728in}}%
\pgfpathlineto{\pgfqpoint{5.398280in}{2.004376in}}%
\pgfpathlineto{\pgfqpoint{5.399520in}{2.003320in}}%
\pgfpathlineto{\pgfqpoint{5.403240in}{1.993169in}}%
\pgfpathlineto{\pgfqpoint{5.405720in}{1.994218in}}%
\pgfpathlineto{\pgfqpoint{5.409440in}{1.996863in}}%
\pgfpathlineto{\pgfqpoint{5.410680in}{1.998135in}}%
\pgfpathlineto{\pgfqpoint{5.413160in}{1.995058in}}%
\pgfpathlineto{\pgfqpoint{5.415640in}{1.998449in}}%
\pgfpathlineto{\pgfqpoint{5.419360in}{1.992854in}}%
\pgfpathlineto{\pgfqpoint{5.420600in}{1.993527in}}%
\pgfpathlineto{\pgfqpoint{5.421840in}{1.992154in}}%
\pgfpathlineto{\pgfqpoint{5.425560in}{1.996211in}}%
\pgfpathlineto{\pgfqpoint{5.426800in}{1.995811in}}%
\pgfpathlineto{\pgfqpoint{5.428040in}{1.997726in}}%
\pgfpathlineto{\pgfqpoint{5.430520in}{1.993317in}}%
\pgfpathlineto{\pgfqpoint{5.433000in}{1.995179in}}%
\pgfpathlineto{\pgfqpoint{5.435480in}{1.997586in}}%
\pgfpathlineto{\pgfqpoint{5.439200in}{2.004953in}}%
\pgfpathlineto{\pgfqpoint{5.442920in}{2.000648in}}%
\pgfpathlineto{\pgfqpoint{5.444160in}{2.003175in}}%
\pgfpathlineto{\pgfqpoint{5.447880in}{1.998504in}}%
\pgfpathlineto{\pgfqpoint{5.450360in}{1.995427in}}%
\pgfpathlineto{\pgfqpoint{5.455320in}{1.993738in}}%
\pgfpathlineto{\pgfqpoint{5.457800in}{1.990570in}}%
\pgfpathlineto{\pgfqpoint{5.461520in}{1.993533in}}%
\pgfpathlineto{\pgfqpoint{5.464000in}{1.989383in}}%
\pgfpathlineto{\pgfqpoint{5.466480in}{1.993775in}}%
\pgfpathlineto{\pgfqpoint{5.467720in}{1.994292in}}%
\pgfpathlineto{\pgfqpoint{5.470200in}{1.991231in}}%
\pgfpathlineto{\pgfqpoint{5.471440in}{1.987089in}}%
\pgfpathlineto{\pgfqpoint{5.475160in}{1.994160in}}%
\pgfpathlineto{\pgfqpoint{5.476400in}{1.994404in}}%
\pgfpathlineto{\pgfqpoint{5.477640in}{1.996151in}}%
\pgfpathlineto{\pgfqpoint{5.478880in}{2.000504in}}%
\pgfpathlineto{\pgfqpoint{5.480120in}{2.000378in}}%
\pgfpathlineto{\pgfqpoint{5.481360in}{2.001752in}}%
\pgfpathlineto{\pgfqpoint{5.483840in}{1.997148in}}%
\pgfpathlineto{\pgfqpoint{5.485080in}{1.995672in}}%
\pgfpathlineto{\pgfqpoint{5.487560in}{1.996135in}}%
\pgfpathlineto{\pgfqpoint{5.488800in}{1.994193in}}%
\pgfpathlineto{\pgfqpoint{5.490040in}{1.994461in}}%
\pgfpathlineto{\pgfqpoint{5.491280in}{1.992889in}}%
\pgfpathlineto{\pgfqpoint{5.492520in}{1.994017in}}%
\pgfpathlineto{\pgfqpoint{5.496240in}{1.992517in}}%
\pgfpathlineto{\pgfqpoint{5.498720in}{1.995529in}}%
\pgfpathlineto{\pgfqpoint{5.501200in}{1.995305in}}%
\pgfpathlineto{\pgfqpoint{5.503680in}{1.995056in}}%
\pgfpathlineto{\pgfqpoint{5.506160in}{1.994784in}}%
\pgfpathlineto{\pgfqpoint{5.509880in}{2.000425in}}%
\pgfpathlineto{\pgfqpoint{5.512360in}{1.996534in}}%
\pgfpathlineto{\pgfqpoint{5.513600in}{1.999023in}}%
\pgfpathlineto{\pgfqpoint{5.514840in}{1.998386in}}%
\pgfpathlineto{\pgfqpoint{5.517320in}{2.002675in}}%
\pgfpathlineto{\pgfqpoint{5.518560in}{2.002354in}}%
\pgfpathlineto{\pgfqpoint{5.521040in}{2.000323in}}%
\pgfpathlineto{\pgfqpoint{5.523520in}{1.999610in}}%
\pgfpathlineto{\pgfqpoint{5.527240in}{1.990983in}}%
\pgfpathlineto{\pgfqpoint{5.528480in}{1.990812in}}%
\pgfpathlineto{\pgfqpoint{5.534680in}{1.997591in}}%
\pgfpathlineto{\pgfqpoint{5.537160in}{1.993966in}}%
\pgfpathlineto{\pgfqpoint{5.539640in}{1.996901in}}%
\pgfpathlineto{\pgfqpoint{5.543360in}{1.991121in}}%
\pgfpathlineto{\pgfqpoint{5.544600in}{1.991217in}}%
\pgfpathlineto{\pgfqpoint{5.545840in}{1.989350in}}%
\pgfpathlineto{\pgfqpoint{5.549560in}{1.994170in}}%
\pgfpathlineto{\pgfqpoint{5.550800in}{1.993345in}}%
\pgfpathlineto{\pgfqpoint{5.552040in}{1.994749in}}%
\pgfpathlineto{\pgfqpoint{5.554520in}{1.989688in}}%
\pgfpathlineto{\pgfqpoint{5.557000in}{1.991961in}}%
\pgfpathlineto{\pgfqpoint{5.558240in}{1.992081in}}%
\pgfpathlineto{\pgfqpoint{5.559480in}{1.993434in}}%
\pgfpathlineto{\pgfqpoint{5.561960in}{1.998771in}}%
\pgfpathlineto{\pgfqpoint{5.563200in}{1.999565in}}%
\pgfpathlineto{\pgfqpoint{5.566920in}{1.997143in}}%
\pgfpathlineto{\pgfqpoint{5.568160in}{1.999433in}}%
\pgfpathlineto{\pgfqpoint{5.570640in}{1.995593in}}%
\pgfpathlineto{\pgfqpoint{5.576840in}{1.991634in}}%
\pgfpathlineto{\pgfqpoint{5.578080in}{1.992043in}}%
\pgfpathlineto{\pgfqpoint{5.579320in}{1.991200in}}%
\pgfpathlineto{\pgfqpoint{5.581800in}{1.988696in}}%
\pgfpathlineto{\pgfqpoint{5.584280in}{1.991606in}}%
\pgfpathlineto{\pgfqpoint{5.585520in}{1.990620in}}%
\pgfpathlineto{\pgfqpoint{5.588000in}{1.985405in}}%
\pgfpathlineto{\pgfqpoint{5.589240in}{1.985671in}}%
\pgfpathlineto{\pgfqpoint{5.591720in}{1.988601in}}%
\pgfpathlineto{\pgfqpoint{5.592960in}{1.987569in}}%
\pgfpathlineto{\pgfqpoint{5.595440in}{1.980830in}}%
\pgfpathlineto{\pgfqpoint{5.599160in}{1.987595in}}%
\pgfpathlineto{\pgfqpoint{5.601640in}{1.989895in}}%
\pgfpathlineto{\pgfqpoint{5.602880in}{1.994277in}}%
\pgfpathlineto{\pgfqpoint{5.604120in}{1.994204in}}%
\pgfpathlineto{\pgfqpoint{5.605360in}{1.995997in}}%
\pgfpathlineto{\pgfqpoint{5.607840in}{1.991305in}}%
\pgfpathlineto{\pgfqpoint{5.609080in}{1.990443in}}%
\pgfpathlineto{\pgfqpoint{5.611560in}{1.990705in}}%
\pgfpathlineto{\pgfqpoint{5.615280in}{1.986238in}}%
\pgfpathlineto{\pgfqpoint{5.616520in}{1.986907in}}%
\pgfpathlineto{\pgfqpoint{5.620240in}{1.985143in}}%
\pgfpathlineto{\pgfqpoint{5.622720in}{1.989318in}}%
\pgfpathlineto{\pgfqpoint{5.625200in}{1.990819in}}%
\pgfpathlineto{\pgfqpoint{5.627680in}{1.991175in}}%
\pgfpathlineto{\pgfqpoint{5.630160in}{1.991865in}}%
\pgfpathlineto{\pgfqpoint{5.633880in}{1.995192in}}%
\pgfpathlineto{\pgfqpoint{5.636360in}{1.990838in}}%
\pgfpathlineto{\pgfqpoint{5.637600in}{1.993500in}}%
\pgfpathlineto{\pgfqpoint{5.638840in}{1.992904in}}%
\pgfpathlineto{\pgfqpoint{5.640080in}{1.998058in}}%
\pgfpathlineto{\pgfqpoint{5.641320in}{1.998222in}}%
\pgfpathlineto{\pgfqpoint{5.645040in}{1.994916in}}%
\pgfpathlineto{\pgfqpoint{5.647520in}{1.994296in}}%
\pgfpathlineto{\pgfqpoint{5.651240in}{1.986235in}}%
\pgfpathlineto{\pgfqpoint{5.652480in}{1.985501in}}%
\pgfpathlineto{\pgfqpoint{5.658680in}{1.992367in}}%
\pgfpathlineto{\pgfqpoint{5.661160in}{1.988345in}}%
\pgfpathlineto{\pgfqpoint{5.663640in}{1.992032in}}%
\pgfpathlineto{\pgfqpoint{5.667360in}{1.985701in}}%
\pgfpathlineto{\pgfqpoint{5.668600in}{1.986882in}}%
\pgfpathlineto{\pgfqpoint{5.669840in}{1.985466in}}%
\pgfpathlineto{\pgfqpoint{5.673560in}{1.989620in}}%
\pgfpathlineto{\pgfqpoint{5.674800in}{1.987639in}}%
\pgfpathlineto{\pgfqpoint{5.676040in}{1.988535in}}%
\pgfpathlineto{\pgfqpoint{5.678520in}{1.982976in}}%
\pgfpathlineto{\pgfqpoint{5.681000in}{1.986958in}}%
\pgfpathlineto{\pgfqpoint{5.683480in}{1.988277in}}%
\pgfpathlineto{\pgfqpoint{5.687200in}{1.993851in}}%
\pgfpathlineto{\pgfqpoint{5.688440in}{1.993614in}}%
\pgfpathlineto{\pgfqpoint{5.690920in}{1.991709in}}%
\pgfpathlineto{\pgfqpoint{5.692160in}{1.993141in}}%
\pgfpathlineto{\pgfqpoint{5.694640in}{1.987856in}}%
\pgfpathlineto{\pgfqpoint{5.695880in}{1.987269in}}%
\pgfpathlineto{\pgfqpoint{5.698360in}{1.984080in}}%
\pgfpathlineto{\pgfqpoint{5.703320in}{1.982163in}}%
\pgfpathlineto{\pgfqpoint{5.704560in}{1.979681in}}%
\pgfpathlineto{\pgfqpoint{5.705800in}{1.980059in}}%
\pgfpathlineto{\pgfqpoint{5.708280in}{1.982826in}}%
\pgfpathlineto{\pgfqpoint{5.709520in}{1.982710in}}%
\pgfpathlineto{\pgfqpoint{5.712000in}{1.978737in}}%
\pgfpathlineto{\pgfqpoint{5.714480in}{1.984985in}}%
\pgfpathlineto{\pgfqpoint{5.715720in}{1.987136in}}%
\pgfpathlineto{\pgfqpoint{5.716960in}{1.986108in}}%
\pgfpathlineto{\pgfqpoint{5.719440in}{1.978496in}}%
\pgfpathlineto{\pgfqpoint{5.723160in}{1.985622in}}%
\pgfpathlineto{\pgfqpoint{5.724400in}{1.985249in}}%
\pgfpathlineto{\pgfqpoint{5.725640in}{1.986603in}}%
\pgfpathlineto{\pgfqpoint{5.726880in}{1.990824in}}%
\pgfpathlineto{\pgfqpoint{5.729360in}{1.989611in}}%
\pgfpathlineto{\pgfqpoint{5.731840in}{1.984836in}}%
\pgfpathlineto{\pgfqpoint{5.734320in}{1.984396in}}%
\pgfpathlineto{\pgfqpoint{5.740520in}{1.981077in}}%
\pgfpathlineto{\pgfqpoint{5.744240in}{1.980223in}}%
\pgfpathlineto{\pgfqpoint{5.747960in}{1.986823in}}%
\pgfpathlineto{\pgfqpoint{5.750440in}{1.988170in}}%
\pgfpathlineto{\pgfqpoint{5.755400in}{1.988286in}}%
\pgfpathlineto{\pgfqpoint{5.757880in}{1.990742in}}%
\pgfpathlineto{\pgfqpoint{5.760360in}{1.986041in}}%
\pgfpathlineto{\pgfqpoint{5.761600in}{1.987083in}}%
\pgfpathlineto{\pgfqpoint{5.762840in}{1.985904in}}%
\pgfpathlineto{\pgfqpoint{5.765320in}{1.992590in}}%
\pgfpathlineto{\pgfqpoint{5.767800in}{1.990230in}}%
\pgfpathlineto{\pgfqpoint{5.771520in}{1.991799in}}%
\pgfpathlineto{\pgfqpoint{5.775240in}{1.982657in}}%
\pgfpathlineto{\pgfqpoint{5.776480in}{1.982428in}}%
\pgfpathlineto{\pgfqpoint{5.777720in}{1.984367in}}%
\pgfpathlineto{\pgfqpoint{5.778960in}{1.983815in}}%
\pgfpathlineto{\pgfqpoint{5.782680in}{1.989419in}}%
\pgfpathlineto{\pgfqpoint{5.785160in}{1.985595in}}%
\pgfpathlineto{\pgfqpoint{5.787640in}{1.989557in}}%
\pgfpathlineto{\pgfqpoint{5.791360in}{1.981113in}}%
\pgfpathlineto{\pgfqpoint{5.792600in}{1.982150in}}%
\pgfpathlineto{\pgfqpoint{5.793840in}{1.980106in}}%
\pgfpathlineto{\pgfqpoint{5.797560in}{1.984225in}}%
\pgfpathlineto{\pgfqpoint{5.802520in}{1.977184in}}%
\pgfpathlineto{\pgfqpoint{5.805000in}{1.980654in}}%
\pgfpathlineto{\pgfqpoint{5.807480in}{1.981804in}}%
\pgfpathlineto{\pgfqpoint{5.811200in}{1.988545in}}%
\pgfpathlineto{\pgfqpoint{5.814920in}{1.985020in}}%
\pgfpathlineto{\pgfqpoint{5.816160in}{1.987150in}}%
\pgfpathlineto{\pgfqpoint{5.821120in}{1.978967in}}%
\pgfpathlineto{\pgfqpoint{5.829800in}{1.975470in}}%
\pgfpathlineto{\pgfqpoint{5.832280in}{1.978990in}}%
\pgfpathlineto{\pgfqpoint{5.833520in}{1.979562in}}%
\pgfpathlineto{\pgfqpoint{5.836000in}{1.974758in}}%
\pgfpathlineto{\pgfqpoint{5.838480in}{1.981736in}}%
\pgfpathlineto{\pgfqpoint{5.839720in}{1.983878in}}%
\pgfpathlineto{\pgfqpoint{5.840960in}{1.982514in}}%
\pgfpathlineto{\pgfqpoint{5.843440in}{1.974643in}}%
\pgfpathlineto{\pgfqpoint{5.847160in}{1.982522in}}%
\pgfpathlineto{\pgfqpoint{5.848400in}{1.981296in}}%
\pgfpathlineto{\pgfqpoint{5.849640in}{1.982093in}}%
\pgfpathlineto{\pgfqpoint{5.850880in}{1.986503in}}%
\pgfpathlineto{\pgfqpoint{5.853360in}{1.984528in}}%
\pgfpathlineto{\pgfqpoint{5.855840in}{1.979809in}}%
\pgfpathlineto{\pgfqpoint{5.858320in}{1.980177in}}%
\pgfpathlineto{\pgfqpoint{5.859560in}{1.979544in}}%
\pgfpathlineto{\pgfqpoint{5.863280in}{1.974384in}}%
\pgfpathlineto{\pgfqpoint{5.864520in}{1.975249in}}%
\pgfpathlineto{\pgfqpoint{5.867000in}{1.973534in}}%
\pgfpathlineto{\pgfqpoint{5.868240in}{1.974851in}}%
\pgfpathlineto{\pgfqpoint{5.871960in}{1.981921in}}%
\pgfpathlineto{\pgfqpoint{5.874440in}{1.983552in}}%
\pgfpathlineto{\pgfqpoint{5.875680in}{1.982921in}}%
\pgfpathlineto{\pgfqpoint{5.878160in}{1.981003in}}%
\pgfpathlineto{\pgfqpoint{5.881880in}{1.985863in}}%
\pgfpathlineto{\pgfqpoint{5.884360in}{1.981532in}}%
\pgfpathlineto{\pgfqpoint{5.885600in}{1.982549in}}%
\pgfpathlineto{\pgfqpoint{5.886840in}{1.979732in}}%
\pgfpathlineto{\pgfqpoint{5.889320in}{1.984809in}}%
\pgfpathlineto{\pgfqpoint{5.891800in}{1.982760in}}%
\pgfpathlineto{\pgfqpoint{5.894280in}{1.983304in}}%
\pgfpathlineto{\pgfqpoint{5.895520in}{1.982877in}}%
\pgfpathlineto{\pgfqpoint{5.898000in}{1.975679in}}%
\pgfpathlineto{\pgfqpoint{5.900480in}{1.973707in}}%
\pgfpathlineto{\pgfqpoint{5.906680in}{1.980128in}}%
\pgfpathlineto{\pgfqpoint{5.909160in}{1.976293in}}%
\pgfpathlineto{\pgfqpoint{5.911640in}{1.979037in}}%
\pgfpathlineto{\pgfqpoint{5.914120in}{1.970452in}}%
\pgfpathlineto{\pgfqpoint{5.915360in}{1.969604in}}%
\pgfpathlineto{\pgfqpoint{5.916600in}{1.971111in}}%
\pgfpathlineto{\pgfqpoint{5.917840in}{1.969323in}}%
\pgfpathlineto{\pgfqpoint{5.921560in}{1.972965in}}%
\pgfpathlineto{\pgfqpoint{5.926520in}{1.964457in}}%
\pgfpathlineto{\pgfqpoint{5.929000in}{1.967198in}}%
\pgfpathlineto{\pgfqpoint{5.931480in}{1.968259in}}%
\pgfpathlineto{\pgfqpoint{5.935200in}{1.976427in}}%
\pgfpathlineto{\pgfqpoint{5.937680in}{1.974937in}}%
\pgfpathlineto{\pgfqpoint{5.938920in}{1.973602in}}%
\pgfpathlineto{\pgfqpoint{5.940160in}{1.975435in}}%
\pgfpathlineto{\pgfqpoint{5.943880in}{1.970540in}}%
\pgfpathlineto{\pgfqpoint{5.947600in}{1.966232in}}%
\pgfpathlineto{\pgfqpoint{5.948840in}{1.966260in}}%
\pgfpathlineto{\pgfqpoint{5.950080in}{1.968559in}}%
\pgfpathlineto{\pgfqpoint{5.953800in}{1.968263in}}%
\pgfpathlineto{\pgfqpoint{5.956280in}{1.972699in}}%
\pgfpathlineto{\pgfqpoint{5.957520in}{1.973548in}}%
\pgfpathlineto{\pgfqpoint{5.960000in}{1.967356in}}%
\pgfpathlineto{\pgfqpoint{5.963720in}{1.973145in}}%
\pgfpathlineto{\pgfqpoint{5.966200in}{1.967050in}}%
\pgfpathlineto{\pgfqpoint{5.967440in}{1.964659in}}%
\pgfpathlineto{\pgfqpoint{5.969920in}{1.969705in}}%
\pgfpathlineto{\pgfqpoint{5.971160in}{1.972060in}}%
\pgfpathlineto{\pgfqpoint{5.973640in}{1.970584in}}%
\pgfpathlineto{\pgfqpoint{5.974880in}{1.974776in}}%
\pgfpathlineto{\pgfqpoint{5.977360in}{1.973255in}}%
\pgfpathlineto{\pgfqpoint{5.979840in}{1.969987in}}%
\pgfpathlineto{\pgfqpoint{5.982320in}{1.971080in}}%
\pgfpathlineto{\pgfqpoint{5.986040in}{1.965545in}}%
\pgfpathlineto{\pgfqpoint{5.987280in}{1.963893in}}%
\pgfpathlineto{\pgfqpoint{5.988520in}{1.965004in}}%
\pgfpathlineto{\pgfqpoint{5.991000in}{1.961438in}}%
\pgfpathlineto{\pgfqpoint{5.993480in}{1.966087in}}%
\pgfpathlineto{\pgfqpoint{5.995960in}{1.969629in}}%
\pgfpathlineto{\pgfqpoint{5.998440in}{1.974335in}}%
\pgfpathlineto{\pgfqpoint{5.999680in}{1.974052in}}%
\pgfpathlineto{\pgfqpoint{6.002160in}{1.971063in}}%
\pgfpathlineto{\pgfqpoint{6.004640in}{1.973472in}}%
\pgfpathlineto{\pgfqpoint{6.005880in}{1.975656in}}%
\pgfpathlineto{\pgfqpoint{6.008360in}{1.971327in}}%
\pgfpathlineto{\pgfqpoint{6.009600in}{1.972233in}}%
\pgfpathlineto{\pgfqpoint{6.010840in}{1.968652in}}%
\pgfpathlineto{\pgfqpoint{6.013320in}{1.974620in}}%
\pgfpathlineto{\pgfqpoint{6.015800in}{1.972232in}}%
\pgfpathlineto{\pgfqpoint{6.018280in}{1.972591in}}%
\pgfpathlineto{\pgfqpoint{6.019520in}{1.971322in}}%
\pgfpathlineto{\pgfqpoint{6.022000in}{1.964878in}}%
\pgfpathlineto{\pgfqpoint{6.024480in}{1.962850in}}%
\pgfpathlineto{\pgfqpoint{6.026960in}{1.963257in}}%
\pgfpathlineto{\pgfqpoint{6.030680in}{1.969060in}}%
\pgfpathlineto{\pgfqpoint{6.035640in}{1.965950in}}%
\pgfpathlineto{\pgfqpoint{6.038120in}{1.955482in}}%
\pgfpathlineto{\pgfqpoint{6.039360in}{1.954615in}}%
\pgfpathlineto{\pgfqpoint{6.040600in}{1.956613in}}%
\pgfpathlineto{\pgfqpoint{6.041840in}{1.954763in}}%
\pgfpathlineto{\pgfqpoint{6.045560in}{1.958873in}}%
\pgfpathlineto{\pgfqpoint{6.046800in}{1.956915in}}%
\pgfpathlineto{\pgfqpoint{6.048040in}{1.957360in}}%
\pgfpathlineto{\pgfqpoint{6.050520in}{1.951657in}}%
\pgfpathlineto{\pgfqpoint{6.053000in}{1.954729in}}%
\pgfpathlineto{\pgfqpoint{6.055480in}{1.957319in}}%
\pgfpathlineto{\pgfqpoint{6.059200in}{1.967887in}}%
\pgfpathlineto{\pgfqpoint{6.060440in}{1.967914in}}%
\pgfpathlineto{\pgfqpoint{6.062920in}{1.964027in}}%
\pgfpathlineto{\pgfqpoint{6.064160in}{1.966797in}}%
\pgfpathlineto{\pgfqpoint{6.066640in}{1.962479in}}%
\pgfpathlineto{\pgfqpoint{6.067880in}{1.962224in}}%
\pgfpathlineto{\pgfqpoint{6.069120in}{1.958926in}}%
\pgfpathlineto{\pgfqpoint{6.070360in}{1.959433in}}%
\pgfpathlineto{\pgfqpoint{6.072840in}{1.957079in}}%
\pgfpathlineto{\pgfqpoint{6.074080in}{1.958977in}}%
\pgfpathlineto{\pgfqpoint{6.076560in}{1.958297in}}%
\pgfpathlineto{\pgfqpoint{6.081520in}{1.966374in}}%
\pgfpathlineto{\pgfqpoint{6.085240in}{1.958739in}}%
\pgfpathlineto{\pgfqpoint{6.087720in}{1.962892in}}%
\pgfpathlineto{\pgfqpoint{6.091440in}{1.957001in}}%
\pgfpathlineto{\pgfqpoint{6.093920in}{1.960695in}}%
\pgfpathlineto{\pgfqpoint{6.095160in}{1.962281in}}%
\pgfpathlineto{\pgfqpoint{6.097640in}{1.959310in}}%
\pgfpathlineto{\pgfqpoint{6.098880in}{1.963424in}}%
\pgfpathlineto{\pgfqpoint{6.101360in}{1.961017in}}%
\pgfpathlineto{\pgfqpoint{6.103840in}{1.957553in}}%
\pgfpathlineto{\pgfqpoint{6.106320in}{1.957396in}}%
\pgfpathlineto{\pgfqpoint{6.111280in}{1.951970in}}%
\pgfpathlineto{\pgfqpoint{6.112520in}{1.953445in}}%
\pgfpathlineto{\pgfqpoint{6.115000in}{1.951588in}}%
\pgfpathlineto{\pgfqpoint{6.116240in}{1.951855in}}%
\pgfpathlineto{\pgfqpoint{6.122440in}{1.965607in}}%
\pgfpathlineto{\pgfqpoint{6.123680in}{1.965161in}}%
\pgfpathlineto{\pgfqpoint{6.126160in}{1.962291in}}%
\pgfpathlineto{\pgfqpoint{6.127400in}{1.962555in}}%
\pgfpathlineto{\pgfqpoint{6.129880in}{1.967123in}}%
\pgfpathlineto{\pgfqpoint{6.132360in}{1.962108in}}%
\pgfpathlineto{\pgfqpoint{6.133600in}{1.962179in}}%
\pgfpathlineto{\pgfqpoint{6.134840in}{1.958365in}}%
\pgfpathlineto{\pgfqpoint{6.137320in}{1.964430in}}%
\pgfpathlineto{\pgfqpoint{6.139800in}{1.962147in}}%
\pgfpathlineto{\pgfqpoint{6.142280in}{1.963639in}}%
\pgfpathlineto{\pgfqpoint{6.143520in}{1.962659in}}%
\pgfpathlineto{\pgfqpoint{6.147240in}{1.954091in}}%
\pgfpathlineto{\pgfqpoint{6.149720in}{1.953081in}}%
\pgfpathlineto{\pgfqpoint{6.150960in}{1.953667in}}%
\pgfpathlineto{\pgfqpoint{6.153440in}{1.958518in}}%
\pgfpathlineto{\pgfqpoint{6.154680in}{1.960770in}}%
\pgfpathlineto{\pgfqpoint{6.158400in}{1.958312in}}%
\pgfpathlineto{\pgfqpoint{6.159640in}{1.958106in}}%
\pgfpathlineto{\pgfqpoint{6.163360in}{1.943889in}}%
\pgfpathlineto{\pgfqpoint{6.164600in}{1.945821in}}%
\pgfpathlineto{\pgfqpoint{6.165840in}{1.944444in}}%
\pgfpathlineto{\pgfqpoint{6.168320in}{1.946918in}}%
\pgfpathlineto{\pgfqpoint{6.169560in}{1.949338in}}%
\pgfpathlineto{\pgfqpoint{6.172040in}{1.949314in}}%
\pgfpathlineto{\pgfqpoint{6.174520in}{1.941750in}}%
\pgfpathlineto{\pgfqpoint{6.177000in}{1.945243in}}%
\pgfpathlineto{\pgfqpoint{6.179480in}{1.947881in}}%
\pgfpathlineto{\pgfqpoint{6.184440in}{1.957955in}}%
\pgfpathlineto{\pgfqpoint{6.185680in}{1.955029in}}%
\pgfpathlineto{\pgfqpoint{6.186920in}{1.955420in}}%
\pgfpathlineto{\pgfqpoint{6.188160in}{1.957298in}}%
\pgfpathlineto{\pgfqpoint{6.191880in}{1.954433in}}%
\pgfpathlineto{\pgfqpoint{6.193120in}{1.950366in}}%
\pgfpathlineto{\pgfqpoint{6.194360in}{1.951330in}}%
\pgfpathlineto{\pgfqpoint{6.196840in}{1.948358in}}%
\pgfpathlineto{\pgfqpoint{6.199320in}{1.949500in}}%
\pgfpathlineto{\pgfqpoint{6.200560in}{1.948620in}}%
\pgfpathlineto{\pgfqpoint{6.201800in}{1.950791in}}%
\pgfpathlineto{\pgfqpoint{6.204280in}{1.956824in}}%
\pgfpathlineto{\pgfqpoint{6.205520in}{1.959478in}}%
\pgfpathlineto{\pgfqpoint{6.209240in}{1.947925in}}%
\pgfpathlineto{\pgfqpoint{6.211720in}{1.953041in}}%
\pgfpathlineto{\pgfqpoint{6.215440in}{1.947467in}}%
\pgfpathlineto{\pgfqpoint{6.217920in}{1.950676in}}%
\pgfpathlineto{\pgfqpoint{6.219160in}{1.951994in}}%
\pgfpathlineto{\pgfqpoint{6.221640in}{1.948867in}}%
\pgfpathlineto{\pgfqpoint{6.222880in}{1.952675in}}%
\pgfpathlineto{\pgfqpoint{6.225360in}{1.950815in}}%
\pgfpathlineto{\pgfqpoint{6.227840in}{1.948071in}}%
\pgfpathlineto{\pgfqpoint{6.229080in}{1.947996in}}%
\pgfpathlineto{\pgfqpoint{6.230320in}{1.949481in}}%
\pgfpathlineto{\pgfqpoint{6.234040in}{1.943170in}}%
\pgfpathlineto{\pgfqpoint{6.236520in}{1.944813in}}%
\pgfpathlineto{\pgfqpoint{6.237760in}{1.942929in}}%
\pgfpathlineto{\pgfqpoint{6.240240in}{1.943435in}}%
\pgfpathlineto{\pgfqpoint{6.242720in}{1.948754in}}%
\pgfpathlineto{\pgfqpoint{6.243960in}{1.948070in}}%
\pgfpathlineto{\pgfqpoint{6.246440in}{1.953468in}}%
\pgfpathlineto{\pgfqpoint{6.250160in}{1.948748in}}%
\pgfpathlineto{\pgfqpoint{6.252640in}{1.948683in}}%
\pgfpathlineto{\pgfqpoint{6.253880in}{1.950958in}}%
\pgfpathlineto{\pgfqpoint{6.256360in}{1.944371in}}%
\pgfpathlineto{\pgfqpoint{6.257600in}{1.945828in}}%
\pgfpathlineto{\pgfqpoint{6.258840in}{1.942925in}}%
\pgfpathlineto{\pgfqpoint{6.261320in}{1.949504in}}%
\pgfpathlineto{\pgfqpoint{6.262560in}{1.946769in}}%
\pgfpathlineto{\pgfqpoint{6.266280in}{1.947134in}}%
\pgfpathlineto{\pgfqpoint{6.273720in}{1.939564in}}%
\pgfpathlineto{\pgfqpoint{6.274960in}{1.940307in}}%
\pgfpathlineto{\pgfqpoint{6.277440in}{1.946763in}}%
\pgfpathlineto{\pgfqpoint{6.279920in}{1.949308in}}%
\pgfpathlineto{\pgfqpoint{6.281160in}{1.946855in}}%
\pgfpathlineto{\pgfqpoint{6.283640in}{1.949274in}}%
\pgfpathlineto{\pgfqpoint{6.287360in}{1.933771in}}%
\pgfpathlineto{\pgfqpoint{6.288600in}{1.935660in}}%
\pgfpathlineto{\pgfqpoint{6.289840in}{1.934235in}}%
\pgfpathlineto{\pgfqpoint{6.291080in}{1.934914in}}%
\pgfpathlineto{\pgfqpoint{6.296040in}{1.942249in}}%
\pgfpathlineto{\pgfqpoint{6.298520in}{1.933471in}}%
\pgfpathlineto{\pgfqpoint{6.301000in}{1.935847in}}%
\pgfpathlineto{\pgfqpoint{6.302240in}{1.936129in}}%
\pgfpathlineto{\pgfqpoint{6.308440in}{1.945950in}}%
\pgfpathlineto{\pgfqpoint{6.309680in}{1.942763in}}%
\pgfpathlineto{\pgfqpoint{6.312160in}{1.943863in}}%
\pgfpathlineto{\pgfqpoint{6.314640in}{1.942682in}}%
\pgfpathlineto{\pgfqpoint{6.315880in}{1.942402in}}%
\pgfpathlineto{\pgfqpoint{6.317120in}{1.939283in}}%
\pgfpathlineto{\pgfqpoint{6.318360in}{1.941553in}}%
\pgfpathlineto{\pgfqpoint{6.320840in}{1.938137in}}%
\pgfpathlineto{\pgfqpoint{6.322080in}{1.939288in}}%
\pgfpathlineto{\pgfqpoint{6.324560in}{1.936682in}}%
\pgfpathlineto{\pgfqpoint{6.325800in}{1.939219in}}%
\pgfpathlineto{\pgfqpoint{6.328280in}{1.947268in}}%
\pgfpathlineto{\pgfqpoint{6.329520in}{1.949139in}}%
\pgfpathlineto{\pgfqpoint{6.333240in}{1.932117in}}%
\pgfpathlineto{\pgfqpoint{6.335720in}{1.938072in}}%
\pgfpathlineto{\pgfqpoint{6.338200in}{1.934688in}}%
\pgfpathlineto{\pgfqpoint{6.339440in}{1.933235in}}%
\pgfpathlineto{\pgfqpoint{6.343160in}{1.940865in}}%
\pgfpathlineto{\pgfqpoint{6.345640in}{1.936434in}}%
\pgfpathlineto{\pgfqpoint{6.346880in}{1.940528in}}%
\pgfpathlineto{\pgfqpoint{6.349360in}{1.939067in}}%
\pgfpathlineto{\pgfqpoint{6.351840in}{1.933364in}}%
\pgfpathlineto{\pgfqpoint{6.354320in}{1.936972in}}%
\pgfpathlineto{\pgfqpoint{6.356800in}{1.932490in}}%
\pgfpathlineto{\pgfqpoint{6.358040in}{1.932050in}}%
\pgfpathlineto{\pgfqpoint{6.359280in}{1.933153in}}%
\pgfpathlineto{\pgfqpoint{6.360520in}{1.932344in}}%
\pgfpathlineto{\pgfqpoint{6.361760in}{1.929787in}}%
\pgfpathlineto{\pgfqpoint{6.365480in}{1.931287in}}%
\pgfpathlineto{\pgfqpoint{6.366720in}{1.934956in}}%
\pgfpathlineto{\pgfqpoint{6.367960in}{1.935071in}}%
\pgfpathlineto{\pgfqpoint{6.370440in}{1.940104in}}%
\pgfpathlineto{\pgfqpoint{6.375400in}{1.934284in}}%
\pgfpathlineto{\pgfqpoint{6.377880in}{1.938982in}}%
\pgfpathlineto{\pgfqpoint{6.380360in}{1.934027in}}%
\pgfpathlineto{\pgfqpoint{6.381600in}{1.935965in}}%
\pgfpathlineto{\pgfqpoint{6.382840in}{1.933704in}}%
\pgfpathlineto{\pgfqpoint{6.385320in}{1.942265in}}%
\pgfpathlineto{\pgfqpoint{6.386560in}{1.940250in}}%
\pgfpathlineto{\pgfqpoint{6.387800in}{1.940854in}}%
\pgfpathlineto{\pgfqpoint{6.390280in}{1.938075in}}%
\pgfpathlineto{\pgfqpoint{6.395240in}{1.929439in}}%
\pgfpathlineto{\pgfqpoint{6.398960in}{1.931297in}}%
\pgfpathlineto{\pgfqpoint{6.401440in}{1.939853in}}%
\pgfpathlineto{\pgfqpoint{6.403920in}{1.941793in}}%
\pgfpathlineto{\pgfqpoint{6.406400in}{1.939001in}}%
\pgfpathlineto{\pgfqpoint{6.407640in}{1.939542in}}%
\pgfpathlineto{\pgfqpoint{6.411360in}{1.922854in}}%
\pgfpathlineto{\pgfqpoint{6.412600in}{1.924355in}}%
\pgfpathlineto{\pgfqpoint{6.415080in}{1.923876in}}%
\pgfpathlineto{\pgfqpoint{6.416320in}{1.925519in}}%
\pgfpathlineto{\pgfqpoint{6.418800in}{1.931264in}}%
\pgfpathlineto{\pgfqpoint{6.420040in}{1.933575in}}%
\pgfpathlineto{\pgfqpoint{6.422520in}{1.924372in}}%
\pgfpathlineto{\pgfqpoint{6.425000in}{1.927490in}}%
\pgfpathlineto{\pgfqpoint{6.426240in}{1.927158in}}%
\pgfpathlineto{\pgfqpoint{6.431200in}{1.936327in}}%
\pgfpathlineto{\pgfqpoint{6.432440in}{1.936947in}}%
\pgfpathlineto{\pgfqpoint{6.433680in}{1.931239in}}%
\pgfpathlineto{\pgfqpoint{6.436160in}{1.933942in}}%
\pgfpathlineto{\pgfqpoint{6.438640in}{1.931190in}}%
\pgfpathlineto{\pgfqpoint{6.439880in}{1.930798in}}%
\pgfpathlineto{\pgfqpoint{6.441120in}{1.925970in}}%
\pgfpathlineto{\pgfqpoint{6.442360in}{1.928212in}}%
\pgfpathlineto{\pgfqpoint{6.444840in}{1.925446in}}%
\pgfpathlineto{\pgfqpoint{6.447320in}{1.927822in}}%
\pgfpathlineto{\pgfqpoint{6.448560in}{1.926218in}}%
\pgfpathlineto{\pgfqpoint{6.449800in}{1.927627in}}%
\pgfpathlineto{\pgfqpoint{6.452280in}{1.934650in}}%
\pgfpathlineto{\pgfqpoint{6.453520in}{1.936138in}}%
\pgfpathlineto{\pgfqpoint{6.457240in}{1.924560in}}%
\pgfpathlineto{\pgfqpoint{6.459720in}{1.935955in}}%
\pgfpathlineto{\pgfqpoint{6.464680in}{1.930235in}}%
\pgfpathlineto{\pgfqpoint{6.467160in}{1.934596in}}%
\pgfpathlineto{\pgfqpoint{6.469640in}{1.929280in}}%
\pgfpathlineto{\pgfqpoint{6.472120in}{1.934121in}}%
\pgfpathlineto{\pgfqpoint{6.473360in}{1.933610in}}%
\pgfpathlineto{\pgfqpoint{6.475840in}{1.925604in}}%
\pgfpathlineto{\pgfqpoint{6.478320in}{1.929073in}}%
\pgfpathlineto{\pgfqpoint{6.480800in}{1.924173in}}%
\pgfpathlineto{\pgfqpoint{6.482040in}{1.922961in}}%
\pgfpathlineto{\pgfqpoint{6.484520in}{1.925965in}}%
\pgfpathlineto{\pgfqpoint{6.487000in}{1.926817in}}%
\pgfpathlineto{\pgfqpoint{6.489480in}{1.923633in}}%
\pgfpathlineto{\pgfqpoint{6.491960in}{1.929242in}}%
\pgfpathlineto{\pgfqpoint{6.494440in}{1.935467in}}%
\pgfpathlineto{\pgfqpoint{6.496920in}{1.929315in}}%
\pgfpathlineto{\pgfqpoint{6.500640in}{1.926049in}}%
\pgfpathlineto{\pgfqpoint{6.501880in}{1.927686in}}%
\pgfpathlineto{\pgfqpoint{6.504360in}{1.922528in}}%
\pgfpathlineto{\pgfqpoint{6.509320in}{1.930830in}}%
\pgfpathlineto{\pgfqpoint{6.510560in}{1.929283in}}%
\pgfpathlineto{\pgfqpoint{6.511800in}{1.929856in}}%
\pgfpathlineto{\pgfqpoint{6.515520in}{1.925699in}}%
\pgfpathlineto{\pgfqpoint{6.519240in}{1.914958in}}%
\pgfpathlineto{\pgfqpoint{6.520480in}{1.915653in}}%
\pgfpathlineto{\pgfqpoint{6.522960in}{1.913591in}}%
\pgfpathlineto{\pgfqpoint{6.525440in}{1.922103in}}%
\pgfpathlineto{\pgfqpoint{6.527920in}{1.924126in}}%
\pgfpathlineto{\pgfqpoint{6.530400in}{1.919312in}}%
\pgfpathlineto{\pgfqpoint{6.531640in}{1.920045in}}%
\pgfpathlineto{\pgfqpoint{6.534120in}{1.907489in}}%
\pgfpathlineto{\pgfqpoint{6.535360in}{1.907666in}}%
\pgfpathlineto{\pgfqpoint{6.536600in}{1.910830in}}%
\pgfpathlineto{\pgfqpoint{6.540320in}{1.911280in}}%
\pgfpathlineto{\pgfqpoint{6.542800in}{1.914457in}}%
\pgfpathlineto{\pgfqpoint{6.544040in}{1.916939in}}%
\pgfpathlineto{\pgfqpoint{6.546520in}{1.909479in}}%
\pgfpathlineto{\pgfqpoint{6.550240in}{1.913584in}}%
\pgfpathlineto{\pgfqpoint{6.553960in}{1.919885in}}%
\pgfpathlineto{\pgfqpoint{6.556440in}{1.923011in}}%
\pgfpathlineto{\pgfqpoint{6.557680in}{1.917413in}}%
\pgfpathlineto{\pgfqpoint{6.560160in}{1.922539in}}%
\pgfpathlineto{\pgfqpoint{6.561400in}{1.920077in}}%
\pgfpathlineto{\pgfqpoint{6.562640in}{1.921499in}}%
\pgfpathlineto{\pgfqpoint{6.563880in}{1.920345in}}%
\pgfpathlineto{\pgfqpoint{6.565120in}{1.915666in}}%
\pgfpathlineto{\pgfqpoint{6.566360in}{1.917564in}}%
\pgfpathlineto{\pgfqpoint{6.567600in}{1.915541in}}%
\pgfpathlineto{\pgfqpoint{6.568840in}{1.915953in}}%
\pgfpathlineto{\pgfqpoint{6.570080in}{1.919115in}}%
\pgfpathlineto{\pgfqpoint{6.573800in}{1.918747in}}%
\pgfpathlineto{\pgfqpoint{6.576280in}{1.921822in}}%
\pgfpathlineto{\pgfqpoint{6.577520in}{1.923062in}}%
\pgfpathlineto{\pgfqpoint{6.581240in}{1.910566in}}%
\pgfpathlineto{\pgfqpoint{6.583720in}{1.922819in}}%
\pgfpathlineto{\pgfqpoint{6.586200in}{1.915568in}}%
\pgfpathlineto{\pgfqpoint{6.588680in}{1.911972in}}%
\pgfpathlineto{\pgfqpoint{6.591160in}{1.916106in}}%
\pgfpathlineto{\pgfqpoint{6.593640in}{1.909813in}}%
\pgfpathlineto{\pgfqpoint{6.597360in}{1.912386in}}%
\pgfpathlineto{\pgfqpoint{6.599840in}{1.902559in}}%
\pgfpathlineto{\pgfqpoint{6.602320in}{1.906670in}}%
\pgfpathlineto{\pgfqpoint{6.603560in}{1.905114in}}%
\pgfpathlineto{\pgfqpoint{6.611000in}{1.912395in}}%
\pgfpathlineto{\pgfqpoint{6.613480in}{1.908345in}}%
\pgfpathlineto{\pgfqpoint{6.617200in}{1.919210in}}%
\pgfpathlineto{\pgfqpoint{6.618440in}{1.921697in}}%
\pgfpathlineto{\pgfqpoint{6.623400in}{1.915798in}}%
\pgfpathlineto{\pgfqpoint{6.625880in}{1.916810in}}%
\pgfpathlineto{\pgfqpoint{6.628360in}{1.913199in}}%
\pgfpathlineto{\pgfqpoint{6.629600in}{1.915154in}}%
\pgfpathlineto{\pgfqpoint{6.630840in}{1.912880in}}%
\pgfpathlineto{\pgfqpoint{6.633320in}{1.913716in}}%
\pgfpathlineto{\pgfqpoint{6.634560in}{1.910133in}}%
\pgfpathlineto{\pgfqpoint{6.635800in}{1.911324in}}%
\pgfpathlineto{\pgfqpoint{6.640760in}{1.906859in}}%
\pgfpathlineto{\pgfqpoint{6.643240in}{1.899180in}}%
\pgfpathlineto{\pgfqpoint{6.644480in}{1.899649in}}%
\pgfpathlineto{\pgfqpoint{6.645720in}{1.899668in}}%
\pgfpathlineto{\pgfqpoint{6.646960in}{1.900820in}}%
\pgfpathlineto{\pgfqpoint{6.649440in}{1.911257in}}%
\pgfpathlineto{\pgfqpoint{6.650680in}{1.910631in}}%
\pgfpathlineto{\pgfqpoint{6.651920in}{1.911331in}}%
\pgfpathlineto{\pgfqpoint{6.653160in}{1.910002in}}%
\pgfpathlineto{\pgfqpoint{6.654400in}{1.906901in}}%
\pgfpathlineto{\pgfqpoint{6.655640in}{1.908244in}}%
\pgfpathlineto{\pgfqpoint{6.658120in}{1.898091in}}%
\pgfpathlineto{\pgfqpoint{6.660600in}{1.901538in}}%
\pgfpathlineto{\pgfqpoint{6.661840in}{1.897177in}}%
\pgfpathlineto{\pgfqpoint{6.664320in}{1.898190in}}%
\pgfpathlineto{\pgfqpoint{6.668040in}{1.906631in}}%
\pgfpathlineto{\pgfqpoint{6.670520in}{1.897978in}}%
\pgfpathlineto{\pgfqpoint{6.673000in}{1.896736in}}%
\pgfpathlineto{\pgfqpoint{6.680440in}{1.905569in}}%
\pgfpathlineto{\pgfqpoint{6.681680in}{1.900795in}}%
\pgfpathlineto{\pgfqpoint{6.684160in}{1.907136in}}%
\pgfpathlineto{\pgfqpoint{6.687880in}{1.906298in}}%
\pgfpathlineto{\pgfqpoint{6.691600in}{1.897947in}}%
\pgfpathlineto{\pgfqpoint{6.692840in}{1.899284in}}%
\pgfpathlineto{\pgfqpoint{6.694080in}{1.902904in}}%
\pgfpathlineto{\pgfqpoint{6.695320in}{1.902826in}}%
\pgfpathlineto{\pgfqpoint{6.696560in}{1.901064in}}%
\pgfpathlineto{\pgfqpoint{6.697800in}{1.901586in}}%
\pgfpathlineto{\pgfqpoint{6.699040in}{1.905493in}}%
\pgfpathlineto{\pgfqpoint{6.702760in}{1.905682in}}%
\pgfpathlineto{\pgfqpoint{6.704000in}{1.901101in}}%
\pgfpathlineto{\pgfqpoint{6.705240in}{1.890205in}}%
\pgfpathlineto{\pgfqpoint{6.707720in}{1.901468in}}%
\pgfpathlineto{\pgfqpoint{6.711440in}{1.892049in}}%
\pgfpathlineto{\pgfqpoint{6.712680in}{1.891433in}}%
\pgfpathlineto{\pgfqpoint{6.715160in}{1.899441in}}%
\pgfpathlineto{\pgfqpoint{6.720120in}{1.889094in}}%
\pgfpathlineto{\pgfqpoint{6.721360in}{1.890522in}}%
\pgfpathlineto{\pgfqpoint{6.723840in}{1.876757in}}%
\pgfpathlineto{\pgfqpoint{6.727560in}{1.883123in}}%
\pgfpathlineto{\pgfqpoint{6.728800in}{1.885705in}}%
\pgfpathlineto{\pgfqpoint{6.731280in}{1.884288in}}%
\pgfpathlineto{\pgfqpoint{6.732520in}{1.884003in}}%
\pgfpathlineto{\pgfqpoint{6.735000in}{1.886008in}}%
\pgfpathlineto{\pgfqpoint{6.737480in}{1.884093in}}%
\pgfpathlineto{\pgfqpoint{6.741200in}{1.898387in}}%
\pgfpathlineto{\pgfqpoint{6.742440in}{1.899309in}}%
\pgfpathlineto{\pgfqpoint{6.743680in}{1.897947in}}%
\pgfpathlineto{\pgfqpoint{6.746160in}{1.899731in}}%
\pgfpathlineto{\pgfqpoint{6.748640in}{1.894382in}}%
\pgfpathlineto{\pgfqpoint{6.749880in}{1.896434in}}%
\pgfpathlineto{\pgfqpoint{6.752360in}{1.894482in}}%
\pgfpathlineto{\pgfqpoint{6.753600in}{1.894272in}}%
\pgfpathlineto{\pgfqpoint{6.761040in}{1.884369in}}%
\pgfpathlineto{\pgfqpoint{6.763520in}{1.887244in}}%
\pgfpathlineto{\pgfqpoint{6.767240in}{1.877544in}}%
\pgfpathlineto{\pgfqpoint{6.768480in}{1.878485in}}%
\pgfpathlineto{\pgfqpoint{6.769720in}{1.876732in}}%
\pgfpathlineto{\pgfqpoint{6.770960in}{1.877501in}}%
\pgfpathlineto{\pgfqpoint{6.773440in}{1.889996in}}%
\pgfpathlineto{\pgfqpoint{6.774680in}{1.890189in}}%
\pgfpathlineto{\pgfqpoint{6.775920in}{1.894008in}}%
\pgfpathlineto{\pgfqpoint{6.779640in}{1.889511in}}%
\pgfpathlineto{\pgfqpoint{6.782120in}{1.876763in}}%
\pgfpathlineto{\pgfqpoint{6.783360in}{1.877225in}}%
\pgfpathlineto{\pgfqpoint{6.784600in}{1.880088in}}%
\pgfpathlineto{\pgfqpoint{6.785840in}{1.874911in}}%
\pgfpathlineto{\pgfqpoint{6.787080in}{1.875754in}}%
\pgfpathlineto{\pgfqpoint{6.789560in}{1.883189in}}%
\pgfpathlineto{\pgfqpoint{6.792040in}{1.890580in}}%
\pgfpathlineto{\pgfqpoint{6.797000in}{1.870244in}}%
\pgfpathlineto{\pgfqpoint{6.799480in}{1.870827in}}%
\pgfpathlineto{\pgfqpoint{6.800720in}{1.868629in}}%
\pgfpathlineto{\pgfqpoint{6.804440in}{1.877764in}}%
\pgfpathlineto{\pgfqpoint{6.805680in}{1.872623in}}%
\pgfpathlineto{\pgfqpoint{6.809400in}{1.884025in}}%
\pgfpathlineto{\pgfqpoint{6.810640in}{1.881714in}}%
\pgfpathlineto{\pgfqpoint{6.811880in}{1.882021in}}%
\pgfpathlineto{\pgfqpoint{6.813120in}{1.878783in}}%
\pgfpathlineto{\pgfqpoint{6.814360in}{1.879299in}}%
\pgfpathlineto{\pgfqpoint{6.816840in}{1.875461in}}%
\pgfpathlineto{\pgfqpoint{6.818080in}{1.879881in}}%
\pgfpathlineto{\pgfqpoint{6.819320in}{1.879639in}}%
\pgfpathlineto{\pgfqpoint{6.821800in}{1.875634in}}%
\pgfpathlineto{\pgfqpoint{6.826760in}{1.884148in}}%
\pgfpathlineto{\pgfqpoint{6.829240in}{1.876043in}}%
\pgfpathlineto{\pgfqpoint{6.831720in}{1.885579in}}%
\pgfpathlineto{\pgfqpoint{6.834200in}{1.881599in}}%
\pgfpathlineto{\pgfqpoint{6.835440in}{1.881149in}}%
\pgfpathlineto{\pgfqpoint{6.837920in}{1.883612in}}%
\pgfpathlineto{\pgfqpoint{6.839160in}{1.887674in}}%
\pgfpathlineto{\pgfqpoint{6.841640in}{1.882792in}}%
\pgfpathlineto{\pgfqpoint{6.845360in}{1.879748in}}%
\pgfpathlineto{\pgfqpoint{6.847840in}{1.866450in}}%
\pgfpathlineto{\pgfqpoint{6.852800in}{1.885066in}}%
\pgfpathlineto{\pgfqpoint{6.855280in}{1.877693in}}%
\pgfpathlineto{\pgfqpoint{6.857760in}{1.878323in}}%
\pgfpathlineto{\pgfqpoint{6.859000in}{1.877632in}}%
\pgfpathlineto{\pgfqpoint{6.860240in}{1.875361in}}%
\pgfpathlineto{\pgfqpoint{6.861480in}{1.875517in}}%
\pgfpathlineto{\pgfqpoint{6.862720in}{1.878906in}}%
\pgfpathlineto{\pgfqpoint{6.863960in}{1.877745in}}%
\pgfpathlineto{\pgfqpoint{6.866440in}{1.884443in}}%
\pgfpathlineto{\pgfqpoint{6.868920in}{1.880825in}}%
\pgfpathlineto{\pgfqpoint{6.870160in}{1.880660in}}%
\pgfpathlineto{\pgfqpoint{6.871400in}{1.876384in}}%
\pgfpathlineto{\pgfqpoint{6.872640in}{1.877271in}}%
\pgfpathlineto{\pgfqpoint{6.873880in}{1.881141in}}%
\pgfpathlineto{\pgfqpoint{6.875120in}{1.878050in}}%
\pgfpathlineto{\pgfqpoint{6.876360in}{1.878129in}}%
\pgfpathlineto{\pgfqpoint{6.877600in}{1.880066in}}%
\pgfpathlineto{\pgfqpoint{6.882560in}{1.870751in}}%
\pgfpathlineto{\pgfqpoint{6.883800in}{1.874617in}}%
\pgfpathlineto{\pgfqpoint{6.885040in}{1.872449in}}%
\pgfpathlineto{\pgfqpoint{6.887520in}{1.875756in}}%
\pgfpathlineto{\pgfqpoint{6.890000in}{1.870695in}}%
\pgfpathlineto{\pgfqpoint{6.891240in}{1.866660in}}%
\pgfpathlineto{\pgfqpoint{6.893720in}{1.870825in}}%
\pgfpathlineto{\pgfqpoint{6.894960in}{1.869467in}}%
\pgfpathlineto{\pgfqpoint{6.897440in}{1.885254in}}%
\pgfpathlineto{\pgfqpoint{6.899920in}{1.888585in}}%
\pgfpathlineto{\pgfqpoint{6.904880in}{1.868902in}}%
\pgfpathlineto{\pgfqpoint{6.906120in}{1.859167in}}%
\pgfpathlineto{\pgfqpoint{6.907360in}{1.859642in}}%
\pgfpathlineto{\pgfqpoint{6.908600in}{1.863139in}}%
\pgfpathlineto{\pgfqpoint{6.909840in}{1.858652in}}%
\pgfpathlineto{\pgfqpoint{6.912320in}{1.865157in}}%
\pgfpathlineto{\pgfqpoint{6.916040in}{1.888188in}}%
\pgfpathlineto{\pgfqpoint{6.922240in}{1.863912in}}%
\pgfpathlineto{\pgfqpoint{6.923480in}{1.864478in}}%
\pgfpathlineto{\pgfqpoint{6.924720in}{1.863760in}}%
\pgfpathlineto{\pgfqpoint{6.925960in}{1.866148in}}%
\pgfpathlineto{\pgfqpoint{6.928440in}{1.879623in}}%
\pgfpathlineto{\pgfqpoint{6.929680in}{1.872545in}}%
\pgfpathlineto{\pgfqpoint{6.932160in}{1.887291in}}%
\pgfpathlineto{\pgfqpoint{6.937120in}{1.874492in}}%
\pgfpathlineto{\pgfqpoint{6.939600in}{1.876692in}}%
\pgfpathlineto{\pgfqpoint{6.940840in}{1.873524in}}%
\pgfpathlineto{\pgfqpoint{6.942080in}{1.875533in}}%
\pgfpathlineto{\pgfqpoint{6.944560in}{1.871353in}}%
\pgfpathlineto{\pgfqpoint{6.945800in}{1.872434in}}%
\pgfpathlineto{\pgfqpoint{6.950760in}{1.885145in}}%
\pgfpathlineto{\pgfqpoint{6.952000in}{1.884444in}}%
\pgfpathlineto{\pgfqpoint{6.954480in}{1.895076in}}%
\pgfpathlineto{\pgfqpoint{6.955720in}{1.899001in}}%
\pgfpathlineto{\pgfqpoint{6.956960in}{1.891877in}}%
\pgfpathlineto{\pgfqpoint{6.959440in}{1.896190in}}%
\pgfpathlineto{\pgfqpoint{6.961920in}{1.890091in}}%
\pgfpathlineto{\pgfqpoint{6.963160in}{1.892163in}}%
\pgfpathlineto{\pgfqpoint{6.965640in}{1.885699in}}%
\pgfpathlineto{\pgfqpoint{6.966880in}{1.885425in}}%
\pgfpathlineto{\pgfqpoint{6.969360in}{1.893297in}}%
\pgfpathlineto{\pgfqpoint{6.971840in}{1.882669in}}%
\pgfpathlineto{\pgfqpoint{6.976800in}{1.903061in}}%
\pgfpathlineto{\pgfqpoint{6.980520in}{1.889734in}}%
\pgfpathlineto{\pgfqpoint{6.981760in}{1.887802in}}%
\pgfpathlineto{\pgfqpoint{6.983000in}{1.889775in}}%
\pgfpathlineto{\pgfqpoint{6.985480in}{1.880996in}}%
\pgfpathlineto{\pgfqpoint{6.987960in}{1.891533in}}%
\pgfpathlineto{\pgfqpoint{6.990440in}{1.898050in}}%
\pgfpathlineto{\pgfqpoint{6.991680in}{1.894702in}}%
\pgfpathlineto{\pgfqpoint{6.992920in}{1.894648in}}%
\pgfpathlineto{\pgfqpoint{6.994160in}{1.897223in}}%
\pgfpathlineto{\pgfqpoint{6.995400in}{1.892168in}}%
\pgfpathlineto{\pgfqpoint{6.997880in}{1.903988in}}%
\pgfpathlineto{\pgfqpoint{6.999120in}{1.898071in}}%
\pgfpathlineto{\pgfqpoint{7.000360in}{1.898402in}}%
\pgfpathlineto{\pgfqpoint{7.005320in}{1.883536in}}%
\pgfpathlineto{\pgfqpoint{7.006560in}{1.877007in}}%
\pgfpathlineto{\pgfqpoint{7.007800in}{1.885378in}}%
\pgfpathlineto{\pgfqpoint{7.009040in}{1.883253in}}%
\pgfpathlineto{\pgfqpoint{7.010280in}{1.886631in}}%
\pgfpathlineto{\pgfqpoint{7.012760in}{1.879670in}}%
\pgfpathlineto{\pgfqpoint{7.015240in}{1.872386in}}%
\pgfpathlineto{\pgfqpoint{7.017720in}{1.880200in}}%
\pgfpathlineto{\pgfqpoint{7.018960in}{1.880882in}}%
\pgfpathlineto{\pgfqpoint{7.021440in}{1.901611in}}%
\pgfpathlineto{\pgfqpoint{7.022680in}{1.903166in}}%
\pgfpathlineto{\pgfqpoint{7.025160in}{1.894658in}}%
\pgfpathlineto{\pgfqpoint{7.027640in}{1.882372in}}%
\pgfpathlineto{\pgfqpoint{7.028880in}{1.873399in}}%
\pgfpathlineto{\pgfqpoint{7.030120in}{1.874611in}}%
\pgfpathlineto{\pgfqpoint{7.032600in}{1.883593in}}%
\pgfpathlineto{\pgfqpoint{7.033840in}{1.877367in}}%
\pgfpathlineto{\pgfqpoint{7.036320in}{1.881941in}}%
\pgfpathlineto{\pgfqpoint{7.040040in}{1.912208in}}%
\pgfpathlineto{\pgfqpoint{7.046240in}{1.863012in}}%
\pgfpathlineto{\pgfqpoint{7.047480in}{1.862545in}}%
\pgfpathlineto{\pgfqpoint{7.048720in}{1.863847in}}%
\pgfpathlineto{\pgfqpoint{7.052440in}{1.878019in}}%
\pgfpathlineto{\pgfqpoint{7.053680in}{1.872052in}}%
\pgfpathlineto{\pgfqpoint{7.056160in}{1.890682in}}%
\pgfpathlineto{\pgfqpoint{7.061120in}{1.875032in}}%
\pgfpathlineto{\pgfqpoint{7.062360in}{1.877037in}}%
\pgfpathlineto{\pgfqpoint{7.063600in}{1.876369in}}%
\pgfpathlineto{\pgfqpoint{7.064840in}{1.872648in}}%
\pgfpathlineto{\pgfqpoint{7.066080in}{1.877760in}}%
\pgfpathlineto{\pgfqpoint{7.068560in}{1.874481in}}%
\pgfpathlineto{\pgfqpoint{7.069800in}{1.875975in}}%
\pgfpathlineto{\pgfqpoint{7.071040in}{1.880719in}}%
\pgfpathlineto{\pgfqpoint{7.072280in}{1.879597in}}%
\pgfpathlineto{\pgfqpoint{7.074760in}{1.886355in}}%
\pgfpathlineto{\pgfqpoint{7.076000in}{1.881478in}}%
\pgfpathlineto{\pgfqpoint{7.079720in}{1.892893in}}%
\pgfpathlineto{\pgfqpoint{7.080960in}{1.886623in}}%
\pgfpathlineto{\pgfqpoint{7.082200in}{1.891392in}}%
\pgfpathlineto{\pgfqpoint{7.084680in}{1.888533in}}%
\pgfpathlineto{\pgfqpoint{7.085920in}{1.878086in}}%
\pgfpathlineto{\pgfqpoint{7.087160in}{1.881639in}}%
\pgfpathlineto{\pgfqpoint{7.088400in}{1.877099in}}%
\pgfpathlineto{\pgfqpoint{7.089640in}{1.877447in}}%
\pgfpathlineto{\pgfqpoint{7.093360in}{1.896057in}}%
\pgfpathlineto{\pgfqpoint{7.095840in}{1.886330in}}%
\pgfpathlineto{\pgfqpoint{7.099560in}{1.918613in}}%
\pgfpathlineto{\pgfqpoint{7.102040in}{1.936271in}}%
\pgfpathlineto{\pgfqpoint{7.105760in}{1.915832in}}%
\pgfpathlineto{\pgfqpoint{7.107000in}{1.920503in}}%
\pgfpathlineto{\pgfqpoint{7.109480in}{1.917103in}}%
\pgfpathlineto{\pgfqpoint{7.114440in}{1.950078in}}%
\pgfpathlineto{\pgfqpoint{7.116920in}{1.936417in}}%
\pgfpathlineto{\pgfqpoint{7.118160in}{1.944444in}}%
\pgfpathlineto{\pgfqpoint{7.119400in}{1.935195in}}%
\pgfpathlineto{\pgfqpoint{7.121880in}{1.952574in}}%
\pgfpathlineto{\pgfqpoint{7.126840in}{1.936088in}}%
\pgfpathlineto{\pgfqpoint{7.128080in}{1.929821in}}%
\pgfpathlineto{\pgfqpoint{7.129320in}{1.932891in}}%
\pgfpathlineto{\pgfqpoint{7.130560in}{1.920370in}}%
\pgfpathlineto{\pgfqpoint{7.131800in}{1.929210in}}%
\pgfpathlineto{\pgfqpoint{7.134280in}{1.922676in}}%
\pgfpathlineto{\pgfqpoint{7.136760in}{1.908549in}}%
\pgfpathlineto{\pgfqpoint{7.138000in}{1.907919in}}%
\pgfpathlineto{\pgfqpoint{7.141720in}{1.934080in}}%
\pgfpathlineto{\pgfqpoint{7.142960in}{1.929121in}}%
\pgfpathlineto{\pgfqpoint{7.144200in}{1.943592in}}%
\pgfpathlineto{\pgfqpoint{7.145440in}{1.944391in}}%
\pgfpathlineto{\pgfqpoint{7.146680in}{1.942122in}}%
\pgfpathlineto{\pgfqpoint{7.147920in}{1.943727in}}%
\pgfpathlineto{\pgfqpoint{7.149160in}{1.937122in}}%
\pgfpathlineto{\pgfqpoint{7.151640in}{1.909299in}}%
\pgfpathlineto{\pgfqpoint{7.152880in}{1.910283in}}%
\pgfpathlineto{\pgfqpoint{7.155360in}{1.934202in}}%
\pgfpathlineto{\pgfqpoint{7.156600in}{1.936522in}}%
\pgfpathlineto{\pgfqpoint{7.157840in}{1.930895in}}%
\pgfpathlineto{\pgfqpoint{7.164040in}{1.968951in}}%
\pgfpathlineto{\pgfqpoint{7.165280in}{1.957996in}}%
\pgfpathlineto{\pgfqpoint{7.169000in}{1.905771in}}%
\pgfpathlineto{\pgfqpoint{7.171480in}{1.890106in}}%
\pgfpathlineto{\pgfqpoint{7.175200in}{1.905801in}}%
\pgfpathlineto{\pgfqpoint{7.176440in}{1.915294in}}%
\pgfpathlineto{\pgfqpoint{7.177680in}{1.915594in}}%
\pgfpathlineto{\pgfqpoint{7.178920in}{1.924100in}}%
\pgfpathlineto{\pgfqpoint{7.180160in}{1.924452in}}%
\pgfpathlineto{\pgfqpoint{7.185120in}{1.891500in}}%
\pgfpathlineto{\pgfqpoint{7.186360in}{1.893746in}}%
\pgfpathlineto{\pgfqpoint{7.191320in}{1.925939in}}%
\pgfpathlineto{\pgfqpoint{7.193800in}{1.933224in}}%
\pgfpathlineto{\pgfqpoint{7.195040in}{1.927506in}}%
\pgfpathlineto{\pgfqpoint{7.196280in}{1.930810in}}%
\pgfpathlineto{\pgfqpoint{7.200000in}{1.906098in}}%
\pgfpathlineto{\pgfqpoint{7.200000in}{1.906098in}}%
\pgfusepath{stroke}%
\end{pgfscope}%
\begin{pgfscope}%
\pgfpathrectangle{\pgfqpoint{1.000000in}{0.300000in}}{\pgfqpoint{6.200000in}{2.400000in}} %
\pgfusepath{clip}%
\pgfsetrectcap%
\pgfsetroundjoin%
\pgfsetlinewidth{2.007500pt}%
\definecolor{currentstroke}{rgb}{1.000000,0.000000,0.000000}%
\pgfsetstrokecolor{currentstroke}%
\pgfsetdash{}{0pt}%
\pgfpathmoveto{\pgfqpoint{2.241240in}{1.938241in}}%
\pgfpathlineto{\pgfqpoint{6.578760in}{1.938241in}}%
\pgfusepath{stroke}%
\end{pgfscope}%
\begin{pgfscope}%
\pgfpathrectangle{\pgfqpoint{1.000000in}{0.300000in}}{\pgfqpoint{6.200000in}{2.400000in}} %
\pgfusepath{clip}%
\pgfsetbuttcap%
\pgfsetroundjoin%
\pgfsetlinewidth{0.501875pt}%
\definecolor{currentstroke}{rgb}{0.000000,0.000000,0.000000}%
\pgfsetstrokecolor{currentstroke}%
\pgfsetdash{{1.000000pt}{3.000000pt}}{0.000000pt}%
\pgfpathmoveto{\pgfqpoint{1.000000in}{0.300000in}}%
\pgfpathlineto{\pgfqpoint{1.000000in}{2.700000in}}%
\pgfusepath{stroke}%
\end{pgfscope}%
\begin{pgfscope}%
\pgfsetbuttcap%
\pgfsetroundjoin%
\definecolor{currentfill}{rgb}{0.000000,0.000000,0.000000}%
\pgfsetfillcolor{currentfill}%
\pgfsetlinewidth{0.501875pt}%
\definecolor{currentstroke}{rgb}{0.000000,0.000000,0.000000}%
\pgfsetstrokecolor{currentstroke}%
\pgfsetdash{}{0pt}%
\pgfsys@defobject{currentmarker}{\pgfqpoint{0.000000in}{0.000000in}}{\pgfqpoint{0.000000in}{0.055556in}}{%
\pgfpathmoveto{\pgfqpoint{0.000000in}{0.000000in}}%
\pgfpathlineto{\pgfqpoint{0.000000in}{0.055556in}}%
\pgfusepath{stroke,fill}%
}%
\begin{pgfscope}%
\pgfsys@transformshift{1.000000in}{0.300000in}%
\pgfsys@useobject{currentmarker}{}%
\end{pgfscope}%
\end{pgfscope}%
\begin{pgfscope}%
\pgfsetbuttcap%
\pgfsetroundjoin%
\definecolor{currentfill}{rgb}{0.000000,0.000000,0.000000}%
\pgfsetfillcolor{currentfill}%
\pgfsetlinewidth{0.501875pt}%
\definecolor{currentstroke}{rgb}{0.000000,0.000000,0.000000}%
\pgfsetstrokecolor{currentstroke}%
\pgfsetdash{}{0pt}%
\pgfsys@defobject{currentmarker}{\pgfqpoint{0.000000in}{-0.055556in}}{\pgfqpoint{0.000000in}{0.000000in}}{%
\pgfpathmoveto{\pgfqpoint{0.000000in}{0.000000in}}%
\pgfpathlineto{\pgfqpoint{0.000000in}{-0.055556in}}%
\pgfusepath{stroke,fill}%
}%
\begin{pgfscope}%
\pgfsys@transformshift{1.000000in}{2.700000in}%
\pgfsys@useobject{currentmarker}{}%
\end{pgfscope}%
\end{pgfscope}%
\begin{pgfscope}%
\pgftext[left,bottom,x=0.946981in,y=0.118387in,rotate=0.000000]{{\sffamily\fontsize{12.000000}{14.400000}\selectfont 0}}
%
\end{pgfscope}%
\begin{pgfscope}%
\pgfpathrectangle{\pgfqpoint{1.000000in}{0.300000in}}{\pgfqpoint{6.200000in}{2.400000in}} %
\pgfusepath{clip}%
\pgfsetbuttcap%
\pgfsetroundjoin%
\pgfsetlinewidth{0.501875pt}%
\definecolor{currentstroke}{rgb}{0.000000,0.000000,0.000000}%
\pgfsetstrokecolor{currentstroke}%
\pgfsetdash{{1.000000pt}{3.000000pt}}{0.000000pt}%
\pgfpathmoveto{\pgfqpoint{2.240000in}{0.300000in}}%
\pgfpathlineto{\pgfqpoint{2.240000in}{2.700000in}}%
\pgfusepath{stroke}%
\end{pgfscope}%
\begin{pgfscope}%
\pgfsetbuttcap%
\pgfsetroundjoin%
\definecolor{currentfill}{rgb}{0.000000,0.000000,0.000000}%
\pgfsetfillcolor{currentfill}%
\pgfsetlinewidth{0.501875pt}%
\definecolor{currentstroke}{rgb}{0.000000,0.000000,0.000000}%
\pgfsetstrokecolor{currentstroke}%
\pgfsetdash{}{0pt}%
\pgfsys@defobject{currentmarker}{\pgfqpoint{0.000000in}{0.000000in}}{\pgfqpoint{0.000000in}{0.055556in}}{%
\pgfpathmoveto{\pgfqpoint{0.000000in}{0.000000in}}%
\pgfpathlineto{\pgfqpoint{0.000000in}{0.055556in}}%
\pgfusepath{stroke,fill}%
}%
\begin{pgfscope}%
\pgfsys@transformshift{2.240000in}{0.300000in}%
\pgfsys@useobject{currentmarker}{}%
\end{pgfscope}%
\end{pgfscope}%
\begin{pgfscope}%
\pgfsetbuttcap%
\pgfsetroundjoin%
\definecolor{currentfill}{rgb}{0.000000,0.000000,0.000000}%
\pgfsetfillcolor{currentfill}%
\pgfsetlinewidth{0.501875pt}%
\definecolor{currentstroke}{rgb}{0.000000,0.000000,0.000000}%
\pgfsetstrokecolor{currentstroke}%
\pgfsetdash{}{0pt}%
\pgfsys@defobject{currentmarker}{\pgfqpoint{0.000000in}{-0.055556in}}{\pgfqpoint{0.000000in}{0.000000in}}{%
\pgfpathmoveto{\pgfqpoint{0.000000in}{0.000000in}}%
\pgfpathlineto{\pgfqpoint{0.000000in}{-0.055556in}}%
\pgfusepath{stroke,fill}%
}%
\begin{pgfscope}%
\pgfsys@transformshift{2.240000in}{2.700000in}%
\pgfsys@useobject{currentmarker}{}%
\end{pgfscope}%
\end{pgfscope}%
\begin{pgfscope}%
\pgftext[left,bottom,x=2.080942in,y=0.118387in,rotate=0.000000]{{\sffamily\fontsize{12.000000}{14.400000}\selectfont 100}}
%
\end{pgfscope}%
\begin{pgfscope}%
\pgfpathrectangle{\pgfqpoint{1.000000in}{0.300000in}}{\pgfqpoint{6.200000in}{2.400000in}} %
\pgfusepath{clip}%
\pgfsetbuttcap%
\pgfsetroundjoin%
\pgfsetlinewidth{0.501875pt}%
\definecolor{currentstroke}{rgb}{0.000000,0.000000,0.000000}%
\pgfsetstrokecolor{currentstroke}%
\pgfsetdash{{1.000000pt}{3.000000pt}}{0.000000pt}%
\pgfpathmoveto{\pgfqpoint{3.480000in}{0.300000in}}%
\pgfpathlineto{\pgfqpoint{3.480000in}{2.700000in}}%
\pgfusepath{stroke}%
\end{pgfscope}%
\begin{pgfscope}%
\pgfsetbuttcap%
\pgfsetroundjoin%
\definecolor{currentfill}{rgb}{0.000000,0.000000,0.000000}%
\pgfsetfillcolor{currentfill}%
\pgfsetlinewidth{0.501875pt}%
\definecolor{currentstroke}{rgb}{0.000000,0.000000,0.000000}%
\pgfsetstrokecolor{currentstroke}%
\pgfsetdash{}{0pt}%
\pgfsys@defobject{currentmarker}{\pgfqpoint{0.000000in}{0.000000in}}{\pgfqpoint{0.000000in}{0.055556in}}{%
\pgfpathmoveto{\pgfqpoint{0.000000in}{0.000000in}}%
\pgfpathlineto{\pgfqpoint{0.000000in}{0.055556in}}%
\pgfusepath{stroke,fill}%
}%
\begin{pgfscope}%
\pgfsys@transformshift{3.480000in}{0.300000in}%
\pgfsys@useobject{currentmarker}{}%
\end{pgfscope}%
\end{pgfscope}%
\begin{pgfscope}%
\pgfsetbuttcap%
\pgfsetroundjoin%
\definecolor{currentfill}{rgb}{0.000000,0.000000,0.000000}%
\pgfsetfillcolor{currentfill}%
\pgfsetlinewidth{0.501875pt}%
\definecolor{currentstroke}{rgb}{0.000000,0.000000,0.000000}%
\pgfsetstrokecolor{currentstroke}%
\pgfsetdash{}{0pt}%
\pgfsys@defobject{currentmarker}{\pgfqpoint{0.000000in}{-0.055556in}}{\pgfqpoint{0.000000in}{0.000000in}}{%
\pgfpathmoveto{\pgfqpoint{0.000000in}{0.000000in}}%
\pgfpathlineto{\pgfqpoint{0.000000in}{-0.055556in}}%
\pgfusepath{stroke,fill}%
}%
\begin{pgfscope}%
\pgfsys@transformshift{3.480000in}{2.700000in}%
\pgfsys@useobject{currentmarker}{}%
\end{pgfscope}%
\end{pgfscope}%
\begin{pgfscope}%
\pgftext[left,bottom,x=3.320942in,y=0.118387in,rotate=0.000000]{{\sffamily\fontsize{12.000000}{14.400000}\selectfont 200}}
%
\end{pgfscope}%
\begin{pgfscope}%
\pgfpathrectangle{\pgfqpoint{1.000000in}{0.300000in}}{\pgfqpoint{6.200000in}{2.400000in}} %
\pgfusepath{clip}%
\pgfsetbuttcap%
\pgfsetroundjoin%
\pgfsetlinewidth{0.501875pt}%
\definecolor{currentstroke}{rgb}{0.000000,0.000000,0.000000}%
\pgfsetstrokecolor{currentstroke}%
\pgfsetdash{{1.000000pt}{3.000000pt}}{0.000000pt}%
\pgfpathmoveto{\pgfqpoint{4.720000in}{0.300000in}}%
\pgfpathlineto{\pgfqpoint{4.720000in}{2.700000in}}%
\pgfusepath{stroke}%
\end{pgfscope}%
\begin{pgfscope}%
\pgfsetbuttcap%
\pgfsetroundjoin%
\definecolor{currentfill}{rgb}{0.000000,0.000000,0.000000}%
\pgfsetfillcolor{currentfill}%
\pgfsetlinewidth{0.501875pt}%
\definecolor{currentstroke}{rgb}{0.000000,0.000000,0.000000}%
\pgfsetstrokecolor{currentstroke}%
\pgfsetdash{}{0pt}%
\pgfsys@defobject{currentmarker}{\pgfqpoint{0.000000in}{0.000000in}}{\pgfqpoint{0.000000in}{0.055556in}}{%
\pgfpathmoveto{\pgfqpoint{0.000000in}{0.000000in}}%
\pgfpathlineto{\pgfqpoint{0.000000in}{0.055556in}}%
\pgfusepath{stroke,fill}%
}%
\begin{pgfscope}%
\pgfsys@transformshift{4.720000in}{0.300000in}%
\pgfsys@useobject{currentmarker}{}%
\end{pgfscope}%
\end{pgfscope}%
\begin{pgfscope}%
\pgfsetbuttcap%
\pgfsetroundjoin%
\definecolor{currentfill}{rgb}{0.000000,0.000000,0.000000}%
\pgfsetfillcolor{currentfill}%
\pgfsetlinewidth{0.501875pt}%
\definecolor{currentstroke}{rgb}{0.000000,0.000000,0.000000}%
\pgfsetstrokecolor{currentstroke}%
\pgfsetdash{}{0pt}%
\pgfsys@defobject{currentmarker}{\pgfqpoint{0.000000in}{-0.055556in}}{\pgfqpoint{0.000000in}{0.000000in}}{%
\pgfpathmoveto{\pgfqpoint{0.000000in}{0.000000in}}%
\pgfpathlineto{\pgfqpoint{0.000000in}{-0.055556in}}%
\pgfusepath{stroke,fill}%
}%
\begin{pgfscope}%
\pgfsys@transformshift{4.720000in}{2.700000in}%
\pgfsys@useobject{currentmarker}{}%
\end{pgfscope}%
\end{pgfscope}%
\begin{pgfscope}%
\pgftext[left,bottom,x=4.560942in,y=0.118387in,rotate=0.000000]{{\sffamily\fontsize{12.000000}{14.400000}\selectfont 300}}
%
\end{pgfscope}%
\begin{pgfscope}%
\pgfpathrectangle{\pgfqpoint{1.000000in}{0.300000in}}{\pgfqpoint{6.200000in}{2.400000in}} %
\pgfusepath{clip}%
\pgfsetbuttcap%
\pgfsetroundjoin%
\pgfsetlinewidth{0.501875pt}%
\definecolor{currentstroke}{rgb}{0.000000,0.000000,0.000000}%
\pgfsetstrokecolor{currentstroke}%
\pgfsetdash{{1.000000pt}{3.000000pt}}{0.000000pt}%
\pgfpathmoveto{\pgfqpoint{5.960000in}{0.300000in}}%
\pgfpathlineto{\pgfqpoint{5.960000in}{2.700000in}}%
\pgfusepath{stroke}%
\end{pgfscope}%
\begin{pgfscope}%
\pgfsetbuttcap%
\pgfsetroundjoin%
\definecolor{currentfill}{rgb}{0.000000,0.000000,0.000000}%
\pgfsetfillcolor{currentfill}%
\pgfsetlinewidth{0.501875pt}%
\definecolor{currentstroke}{rgb}{0.000000,0.000000,0.000000}%
\pgfsetstrokecolor{currentstroke}%
\pgfsetdash{}{0pt}%
\pgfsys@defobject{currentmarker}{\pgfqpoint{0.000000in}{0.000000in}}{\pgfqpoint{0.000000in}{0.055556in}}{%
\pgfpathmoveto{\pgfqpoint{0.000000in}{0.000000in}}%
\pgfpathlineto{\pgfqpoint{0.000000in}{0.055556in}}%
\pgfusepath{stroke,fill}%
}%
\begin{pgfscope}%
\pgfsys@transformshift{5.960000in}{0.300000in}%
\pgfsys@useobject{currentmarker}{}%
\end{pgfscope}%
\end{pgfscope}%
\begin{pgfscope}%
\pgfsetbuttcap%
\pgfsetroundjoin%
\definecolor{currentfill}{rgb}{0.000000,0.000000,0.000000}%
\pgfsetfillcolor{currentfill}%
\pgfsetlinewidth{0.501875pt}%
\definecolor{currentstroke}{rgb}{0.000000,0.000000,0.000000}%
\pgfsetstrokecolor{currentstroke}%
\pgfsetdash{}{0pt}%
\pgfsys@defobject{currentmarker}{\pgfqpoint{0.000000in}{-0.055556in}}{\pgfqpoint{0.000000in}{0.000000in}}{%
\pgfpathmoveto{\pgfqpoint{0.000000in}{0.000000in}}%
\pgfpathlineto{\pgfqpoint{0.000000in}{-0.055556in}}%
\pgfusepath{stroke,fill}%
}%
\begin{pgfscope}%
\pgfsys@transformshift{5.960000in}{2.700000in}%
\pgfsys@useobject{currentmarker}{}%
\end{pgfscope}%
\end{pgfscope}%
\begin{pgfscope}%
\pgftext[left,bottom,x=5.800942in,y=0.118387in,rotate=0.000000]{{\sffamily\fontsize{12.000000}{14.400000}\selectfont 400}}
%
\end{pgfscope}%
\begin{pgfscope}%
\pgfpathrectangle{\pgfqpoint{1.000000in}{0.300000in}}{\pgfqpoint{6.200000in}{2.400000in}} %
\pgfusepath{clip}%
\pgfsetbuttcap%
\pgfsetroundjoin%
\pgfsetlinewidth{0.501875pt}%
\definecolor{currentstroke}{rgb}{0.000000,0.000000,0.000000}%
\pgfsetstrokecolor{currentstroke}%
\pgfsetdash{{1.000000pt}{3.000000pt}}{0.000000pt}%
\pgfpathmoveto{\pgfqpoint{7.200000in}{0.300000in}}%
\pgfpathlineto{\pgfqpoint{7.200000in}{2.700000in}}%
\pgfusepath{stroke}%
\end{pgfscope}%
\begin{pgfscope}%
\pgfsetbuttcap%
\pgfsetroundjoin%
\definecolor{currentfill}{rgb}{0.000000,0.000000,0.000000}%
\pgfsetfillcolor{currentfill}%
\pgfsetlinewidth{0.501875pt}%
\definecolor{currentstroke}{rgb}{0.000000,0.000000,0.000000}%
\pgfsetstrokecolor{currentstroke}%
\pgfsetdash{}{0pt}%
\pgfsys@defobject{currentmarker}{\pgfqpoint{0.000000in}{0.000000in}}{\pgfqpoint{0.000000in}{0.055556in}}{%
\pgfpathmoveto{\pgfqpoint{0.000000in}{0.000000in}}%
\pgfpathlineto{\pgfqpoint{0.000000in}{0.055556in}}%
\pgfusepath{stroke,fill}%
}%
\begin{pgfscope}%
\pgfsys@transformshift{7.200000in}{0.300000in}%
\pgfsys@useobject{currentmarker}{}%
\end{pgfscope}%
\end{pgfscope}%
\begin{pgfscope}%
\pgfsetbuttcap%
\pgfsetroundjoin%
\definecolor{currentfill}{rgb}{0.000000,0.000000,0.000000}%
\pgfsetfillcolor{currentfill}%
\pgfsetlinewidth{0.501875pt}%
\definecolor{currentstroke}{rgb}{0.000000,0.000000,0.000000}%
\pgfsetstrokecolor{currentstroke}%
\pgfsetdash{}{0pt}%
\pgfsys@defobject{currentmarker}{\pgfqpoint{0.000000in}{-0.055556in}}{\pgfqpoint{0.000000in}{0.000000in}}{%
\pgfpathmoveto{\pgfqpoint{0.000000in}{0.000000in}}%
\pgfpathlineto{\pgfqpoint{0.000000in}{-0.055556in}}%
\pgfusepath{stroke,fill}%
}%
\begin{pgfscope}%
\pgfsys@transformshift{7.200000in}{2.700000in}%
\pgfsys@useobject{currentmarker}{}%
\end{pgfscope}%
\end{pgfscope}%
\begin{pgfscope}%
\pgftext[left,bottom,x=7.040942in,y=0.118387in,rotate=0.000000]{{\sffamily\fontsize{12.000000}{14.400000}\selectfont 500}}
%
\end{pgfscope}%
\begin{pgfscope}%
\pgftext[left,bottom,x=3.723901in,y=-0.112353in,rotate=0.000000]{{\sffamily\fontsize{12.000000}{14.400000}\selectfont time [ps]}}
%
\end{pgfscope}%
\begin{pgfscope}%
\pgfpathrectangle{\pgfqpoint{1.000000in}{0.300000in}}{\pgfqpoint{6.200000in}{2.400000in}} %
\pgfusepath{clip}%
\pgfsetbuttcap%
\pgfsetroundjoin%
\pgfsetlinewidth{0.501875pt}%
\definecolor{currentstroke}{rgb}{0.000000,0.000000,0.000000}%
\pgfsetstrokecolor{currentstroke}%
\pgfsetdash{{1.000000pt}{3.000000pt}}{0.000000pt}%
\pgfpathmoveto{\pgfqpoint{1.000000in}{0.300000in}}%
\pgfpathlineto{\pgfqpoint{7.200000in}{0.300000in}}%
\pgfusepath{stroke}%
\end{pgfscope}%
\begin{pgfscope}%
\pgfsetbuttcap%
\pgfsetroundjoin%
\definecolor{currentfill}{rgb}{0.000000,0.000000,0.000000}%
\pgfsetfillcolor{currentfill}%
\pgfsetlinewidth{0.501875pt}%
\definecolor{currentstroke}{rgb}{0.000000,0.000000,0.000000}%
\pgfsetstrokecolor{currentstroke}%
\pgfsetdash{}{0pt}%
\pgfsys@defobject{currentmarker}{\pgfqpoint{0.000000in}{0.000000in}}{\pgfqpoint{0.055556in}{0.000000in}}{%
\pgfpathmoveto{\pgfqpoint{0.000000in}{0.000000in}}%
\pgfpathlineto{\pgfqpoint{0.055556in}{0.000000in}}%
\pgfusepath{stroke,fill}%
}%
\begin{pgfscope}%
\pgfsys@transformshift{1.000000in}{0.300000in}%
\pgfsys@useobject{currentmarker}{}%
\end{pgfscope}%
\end{pgfscope}%
\begin{pgfscope}%
\pgfsetbuttcap%
\pgfsetroundjoin%
\definecolor{currentfill}{rgb}{0.000000,0.000000,0.000000}%
\pgfsetfillcolor{currentfill}%
\pgfsetlinewidth{0.501875pt}%
\definecolor{currentstroke}{rgb}{0.000000,0.000000,0.000000}%
\pgfsetstrokecolor{currentstroke}%
\pgfsetdash{}{0pt}%
\pgfsys@defobject{currentmarker}{\pgfqpoint{-0.055556in}{0.000000in}}{\pgfqpoint{0.000000in}{0.000000in}}{%
\pgfpathmoveto{\pgfqpoint{0.000000in}{0.000000in}}%
\pgfpathlineto{\pgfqpoint{-0.055556in}{0.000000in}}%
\pgfusepath{stroke,fill}%
}%
\begin{pgfscope}%
\pgfsys@transformshift{7.200000in}{0.300000in}%
\pgfsys@useobject{currentmarker}{}%
\end{pgfscope}%
\end{pgfscope}%
\begin{pgfscope}%
\pgftext[left,bottom,x=0.679389in,y=0.236971in,rotate=0.000000]{{\sffamily\fontsize{12.000000}{14.400000}\selectfont 2.0}}
%
\end{pgfscope}%
\begin{pgfscope}%
\pgfpathrectangle{\pgfqpoint{1.000000in}{0.300000in}}{\pgfqpoint{6.200000in}{2.400000in}} %
\pgfusepath{clip}%
\pgfsetbuttcap%
\pgfsetroundjoin%
\pgfsetlinewidth{0.501875pt}%
\definecolor{currentstroke}{rgb}{0.000000,0.000000,0.000000}%
\pgfsetstrokecolor{currentstroke}%
\pgfsetdash{{1.000000pt}{3.000000pt}}{0.000000pt}%
\pgfpathmoveto{\pgfqpoint{1.000000in}{0.566667in}}%
\pgfpathlineto{\pgfqpoint{7.200000in}{0.566667in}}%
\pgfusepath{stroke}%
\end{pgfscope}%
\begin{pgfscope}%
\pgfsetbuttcap%
\pgfsetroundjoin%
\definecolor{currentfill}{rgb}{0.000000,0.000000,0.000000}%
\pgfsetfillcolor{currentfill}%
\pgfsetlinewidth{0.501875pt}%
\definecolor{currentstroke}{rgb}{0.000000,0.000000,0.000000}%
\pgfsetstrokecolor{currentstroke}%
\pgfsetdash{}{0pt}%
\pgfsys@defobject{currentmarker}{\pgfqpoint{0.000000in}{0.000000in}}{\pgfqpoint{0.055556in}{0.000000in}}{%
\pgfpathmoveto{\pgfqpoint{0.000000in}{0.000000in}}%
\pgfpathlineto{\pgfqpoint{0.055556in}{0.000000in}}%
\pgfusepath{stroke,fill}%
}%
\begin{pgfscope}%
\pgfsys@transformshift{1.000000in}{0.566667in}%
\pgfsys@useobject{currentmarker}{}%
\end{pgfscope}%
\end{pgfscope}%
\begin{pgfscope}%
\pgfsetbuttcap%
\pgfsetroundjoin%
\definecolor{currentfill}{rgb}{0.000000,0.000000,0.000000}%
\pgfsetfillcolor{currentfill}%
\pgfsetlinewidth{0.501875pt}%
\definecolor{currentstroke}{rgb}{0.000000,0.000000,0.000000}%
\pgfsetstrokecolor{currentstroke}%
\pgfsetdash{}{0pt}%
\pgfsys@defobject{currentmarker}{\pgfqpoint{-0.055556in}{0.000000in}}{\pgfqpoint{0.000000in}{0.000000in}}{%
\pgfpathmoveto{\pgfqpoint{0.000000in}{0.000000in}}%
\pgfpathlineto{\pgfqpoint{-0.055556in}{0.000000in}}%
\pgfusepath{stroke,fill}%
}%
\begin{pgfscope}%
\pgfsys@transformshift{7.200000in}{0.566667in}%
\pgfsys@useobject{currentmarker}{}%
\end{pgfscope}%
\end{pgfscope}%
\begin{pgfscope}%
\pgftext[left,bottom,x=0.679389in,y=0.503638in,rotate=0.000000]{{\sffamily\fontsize{12.000000}{14.400000}\selectfont 2.5}}
%
\end{pgfscope}%
\begin{pgfscope}%
\pgfpathrectangle{\pgfqpoint{1.000000in}{0.300000in}}{\pgfqpoint{6.200000in}{2.400000in}} %
\pgfusepath{clip}%
\pgfsetbuttcap%
\pgfsetroundjoin%
\pgfsetlinewidth{0.501875pt}%
\definecolor{currentstroke}{rgb}{0.000000,0.000000,0.000000}%
\pgfsetstrokecolor{currentstroke}%
\pgfsetdash{{1.000000pt}{3.000000pt}}{0.000000pt}%
\pgfpathmoveto{\pgfqpoint{1.000000in}{0.833333in}}%
\pgfpathlineto{\pgfqpoint{7.200000in}{0.833333in}}%
\pgfusepath{stroke}%
\end{pgfscope}%
\begin{pgfscope}%
\pgfsetbuttcap%
\pgfsetroundjoin%
\definecolor{currentfill}{rgb}{0.000000,0.000000,0.000000}%
\pgfsetfillcolor{currentfill}%
\pgfsetlinewidth{0.501875pt}%
\definecolor{currentstroke}{rgb}{0.000000,0.000000,0.000000}%
\pgfsetstrokecolor{currentstroke}%
\pgfsetdash{}{0pt}%
\pgfsys@defobject{currentmarker}{\pgfqpoint{0.000000in}{0.000000in}}{\pgfqpoint{0.055556in}{0.000000in}}{%
\pgfpathmoveto{\pgfqpoint{0.000000in}{0.000000in}}%
\pgfpathlineto{\pgfqpoint{0.055556in}{0.000000in}}%
\pgfusepath{stroke,fill}%
}%
\begin{pgfscope}%
\pgfsys@transformshift{1.000000in}{0.833333in}%
\pgfsys@useobject{currentmarker}{}%
\end{pgfscope}%
\end{pgfscope}%
\begin{pgfscope}%
\pgfsetbuttcap%
\pgfsetroundjoin%
\definecolor{currentfill}{rgb}{0.000000,0.000000,0.000000}%
\pgfsetfillcolor{currentfill}%
\pgfsetlinewidth{0.501875pt}%
\definecolor{currentstroke}{rgb}{0.000000,0.000000,0.000000}%
\pgfsetstrokecolor{currentstroke}%
\pgfsetdash{}{0pt}%
\pgfsys@defobject{currentmarker}{\pgfqpoint{-0.055556in}{0.000000in}}{\pgfqpoint{0.000000in}{0.000000in}}{%
\pgfpathmoveto{\pgfqpoint{0.000000in}{0.000000in}}%
\pgfpathlineto{\pgfqpoint{-0.055556in}{0.000000in}}%
\pgfusepath{stroke,fill}%
}%
\begin{pgfscope}%
\pgfsys@transformshift{7.200000in}{0.833333in}%
\pgfsys@useobject{currentmarker}{}%
\end{pgfscope}%
\end{pgfscope}%
\begin{pgfscope}%
\pgftext[left,bottom,x=0.679389in,y=0.770305in,rotate=0.000000]{{\sffamily\fontsize{12.000000}{14.400000}\selectfont 3.0}}
%
\end{pgfscope}%
\begin{pgfscope}%
\pgfpathrectangle{\pgfqpoint{1.000000in}{0.300000in}}{\pgfqpoint{6.200000in}{2.400000in}} %
\pgfusepath{clip}%
\pgfsetbuttcap%
\pgfsetroundjoin%
\pgfsetlinewidth{0.501875pt}%
\definecolor{currentstroke}{rgb}{0.000000,0.000000,0.000000}%
\pgfsetstrokecolor{currentstroke}%
\pgfsetdash{{1.000000pt}{3.000000pt}}{0.000000pt}%
\pgfpathmoveto{\pgfqpoint{1.000000in}{1.100000in}}%
\pgfpathlineto{\pgfqpoint{7.200000in}{1.100000in}}%
\pgfusepath{stroke}%
\end{pgfscope}%
\begin{pgfscope}%
\pgfsetbuttcap%
\pgfsetroundjoin%
\definecolor{currentfill}{rgb}{0.000000,0.000000,0.000000}%
\pgfsetfillcolor{currentfill}%
\pgfsetlinewidth{0.501875pt}%
\definecolor{currentstroke}{rgb}{0.000000,0.000000,0.000000}%
\pgfsetstrokecolor{currentstroke}%
\pgfsetdash{}{0pt}%
\pgfsys@defobject{currentmarker}{\pgfqpoint{0.000000in}{0.000000in}}{\pgfqpoint{0.055556in}{0.000000in}}{%
\pgfpathmoveto{\pgfqpoint{0.000000in}{0.000000in}}%
\pgfpathlineto{\pgfqpoint{0.055556in}{0.000000in}}%
\pgfusepath{stroke,fill}%
}%
\begin{pgfscope}%
\pgfsys@transformshift{1.000000in}{1.100000in}%
\pgfsys@useobject{currentmarker}{}%
\end{pgfscope}%
\end{pgfscope}%
\begin{pgfscope}%
\pgfsetbuttcap%
\pgfsetroundjoin%
\definecolor{currentfill}{rgb}{0.000000,0.000000,0.000000}%
\pgfsetfillcolor{currentfill}%
\pgfsetlinewidth{0.501875pt}%
\definecolor{currentstroke}{rgb}{0.000000,0.000000,0.000000}%
\pgfsetstrokecolor{currentstroke}%
\pgfsetdash{}{0pt}%
\pgfsys@defobject{currentmarker}{\pgfqpoint{-0.055556in}{0.000000in}}{\pgfqpoint{0.000000in}{0.000000in}}{%
\pgfpathmoveto{\pgfqpoint{0.000000in}{0.000000in}}%
\pgfpathlineto{\pgfqpoint{-0.055556in}{0.000000in}}%
\pgfusepath{stroke,fill}%
}%
\begin{pgfscope}%
\pgfsys@transformshift{7.200000in}{1.100000in}%
\pgfsys@useobject{currentmarker}{}%
\end{pgfscope}%
\end{pgfscope}%
\begin{pgfscope}%
\pgftext[left,bottom,x=0.679389in,y=1.036971in,rotate=0.000000]{{\sffamily\fontsize{12.000000}{14.400000}\selectfont 3.5}}
%
\end{pgfscope}%
\begin{pgfscope}%
\pgfpathrectangle{\pgfqpoint{1.000000in}{0.300000in}}{\pgfqpoint{6.200000in}{2.400000in}} %
\pgfusepath{clip}%
\pgfsetbuttcap%
\pgfsetroundjoin%
\pgfsetlinewidth{0.501875pt}%
\definecolor{currentstroke}{rgb}{0.000000,0.000000,0.000000}%
\pgfsetstrokecolor{currentstroke}%
\pgfsetdash{{1.000000pt}{3.000000pt}}{0.000000pt}%
\pgfpathmoveto{\pgfqpoint{1.000000in}{1.366667in}}%
\pgfpathlineto{\pgfqpoint{7.200000in}{1.366667in}}%
\pgfusepath{stroke}%
\end{pgfscope}%
\begin{pgfscope}%
\pgfsetbuttcap%
\pgfsetroundjoin%
\definecolor{currentfill}{rgb}{0.000000,0.000000,0.000000}%
\pgfsetfillcolor{currentfill}%
\pgfsetlinewidth{0.501875pt}%
\definecolor{currentstroke}{rgb}{0.000000,0.000000,0.000000}%
\pgfsetstrokecolor{currentstroke}%
\pgfsetdash{}{0pt}%
\pgfsys@defobject{currentmarker}{\pgfqpoint{0.000000in}{0.000000in}}{\pgfqpoint{0.055556in}{0.000000in}}{%
\pgfpathmoveto{\pgfqpoint{0.000000in}{0.000000in}}%
\pgfpathlineto{\pgfqpoint{0.055556in}{0.000000in}}%
\pgfusepath{stroke,fill}%
}%
\begin{pgfscope}%
\pgfsys@transformshift{1.000000in}{1.366667in}%
\pgfsys@useobject{currentmarker}{}%
\end{pgfscope}%
\end{pgfscope}%
\begin{pgfscope}%
\pgfsetbuttcap%
\pgfsetroundjoin%
\definecolor{currentfill}{rgb}{0.000000,0.000000,0.000000}%
\pgfsetfillcolor{currentfill}%
\pgfsetlinewidth{0.501875pt}%
\definecolor{currentstroke}{rgb}{0.000000,0.000000,0.000000}%
\pgfsetstrokecolor{currentstroke}%
\pgfsetdash{}{0pt}%
\pgfsys@defobject{currentmarker}{\pgfqpoint{-0.055556in}{0.000000in}}{\pgfqpoint{0.000000in}{0.000000in}}{%
\pgfpathmoveto{\pgfqpoint{0.000000in}{0.000000in}}%
\pgfpathlineto{\pgfqpoint{-0.055556in}{0.000000in}}%
\pgfusepath{stroke,fill}%
}%
\begin{pgfscope}%
\pgfsys@transformshift{7.200000in}{1.366667in}%
\pgfsys@useobject{currentmarker}{}%
\end{pgfscope}%
\end{pgfscope}%
\begin{pgfscope}%
\pgftext[left,bottom,x=0.679389in,y=1.303638in,rotate=0.000000]{{\sffamily\fontsize{12.000000}{14.400000}\selectfont 4.0}}
%
\end{pgfscope}%
\begin{pgfscope}%
\pgfpathrectangle{\pgfqpoint{1.000000in}{0.300000in}}{\pgfqpoint{6.200000in}{2.400000in}} %
\pgfusepath{clip}%
\pgfsetbuttcap%
\pgfsetroundjoin%
\pgfsetlinewidth{0.501875pt}%
\definecolor{currentstroke}{rgb}{0.000000,0.000000,0.000000}%
\pgfsetstrokecolor{currentstroke}%
\pgfsetdash{{1.000000pt}{3.000000pt}}{0.000000pt}%
\pgfpathmoveto{\pgfqpoint{1.000000in}{1.633333in}}%
\pgfpathlineto{\pgfqpoint{7.200000in}{1.633333in}}%
\pgfusepath{stroke}%
\end{pgfscope}%
\begin{pgfscope}%
\pgfsetbuttcap%
\pgfsetroundjoin%
\definecolor{currentfill}{rgb}{0.000000,0.000000,0.000000}%
\pgfsetfillcolor{currentfill}%
\pgfsetlinewidth{0.501875pt}%
\definecolor{currentstroke}{rgb}{0.000000,0.000000,0.000000}%
\pgfsetstrokecolor{currentstroke}%
\pgfsetdash{}{0pt}%
\pgfsys@defobject{currentmarker}{\pgfqpoint{0.000000in}{0.000000in}}{\pgfqpoint{0.055556in}{0.000000in}}{%
\pgfpathmoveto{\pgfqpoint{0.000000in}{0.000000in}}%
\pgfpathlineto{\pgfqpoint{0.055556in}{0.000000in}}%
\pgfusepath{stroke,fill}%
}%
\begin{pgfscope}%
\pgfsys@transformshift{1.000000in}{1.633333in}%
\pgfsys@useobject{currentmarker}{}%
\end{pgfscope}%
\end{pgfscope}%
\begin{pgfscope}%
\pgfsetbuttcap%
\pgfsetroundjoin%
\definecolor{currentfill}{rgb}{0.000000,0.000000,0.000000}%
\pgfsetfillcolor{currentfill}%
\pgfsetlinewidth{0.501875pt}%
\definecolor{currentstroke}{rgb}{0.000000,0.000000,0.000000}%
\pgfsetstrokecolor{currentstroke}%
\pgfsetdash{}{0pt}%
\pgfsys@defobject{currentmarker}{\pgfqpoint{-0.055556in}{0.000000in}}{\pgfqpoint{0.000000in}{0.000000in}}{%
\pgfpathmoveto{\pgfqpoint{0.000000in}{0.000000in}}%
\pgfpathlineto{\pgfqpoint{-0.055556in}{0.000000in}}%
\pgfusepath{stroke,fill}%
}%
\begin{pgfscope}%
\pgfsys@transformshift{7.200000in}{1.633333in}%
\pgfsys@useobject{currentmarker}{}%
\end{pgfscope}%
\end{pgfscope}%
\begin{pgfscope}%
\pgftext[left,bottom,x=0.679389in,y=1.571403in,rotate=0.000000]{{\sffamily\fontsize{12.000000}{14.400000}\selectfont 4.5}}
%
\end{pgfscope}%
\begin{pgfscope}%
\pgfpathrectangle{\pgfqpoint{1.000000in}{0.300000in}}{\pgfqpoint{6.200000in}{2.400000in}} %
\pgfusepath{clip}%
\pgfsetbuttcap%
\pgfsetroundjoin%
\pgfsetlinewidth{0.501875pt}%
\definecolor{currentstroke}{rgb}{0.000000,0.000000,0.000000}%
\pgfsetstrokecolor{currentstroke}%
\pgfsetdash{{1.000000pt}{3.000000pt}}{0.000000pt}%
\pgfpathmoveto{\pgfqpoint{1.000000in}{1.900000in}}%
\pgfpathlineto{\pgfqpoint{7.200000in}{1.900000in}}%
\pgfusepath{stroke}%
\end{pgfscope}%
\begin{pgfscope}%
\pgfsetbuttcap%
\pgfsetroundjoin%
\definecolor{currentfill}{rgb}{0.000000,0.000000,0.000000}%
\pgfsetfillcolor{currentfill}%
\pgfsetlinewidth{0.501875pt}%
\definecolor{currentstroke}{rgb}{0.000000,0.000000,0.000000}%
\pgfsetstrokecolor{currentstroke}%
\pgfsetdash{}{0pt}%
\pgfsys@defobject{currentmarker}{\pgfqpoint{0.000000in}{0.000000in}}{\pgfqpoint{0.055556in}{0.000000in}}{%
\pgfpathmoveto{\pgfqpoint{0.000000in}{0.000000in}}%
\pgfpathlineto{\pgfqpoint{0.055556in}{0.000000in}}%
\pgfusepath{stroke,fill}%
}%
\begin{pgfscope}%
\pgfsys@transformshift{1.000000in}{1.900000in}%
\pgfsys@useobject{currentmarker}{}%
\end{pgfscope}%
\end{pgfscope}%
\begin{pgfscope}%
\pgfsetbuttcap%
\pgfsetroundjoin%
\definecolor{currentfill}{rgb}{0.000000,0.000000,0.000000}%
\pgfsetfillcolor{currentfill}%
\pgfsetlinewidth{0.501875pt}%
\definecolor{currentstroke}{rgb}{0.000000,0.000000,0.000000}%
\pgfsetstrokecolor{currentstroke}%
\pgfsetdash{}{0pt}%
\pgfsys@defobject{currentmarker}{\pgfqpoint{-0.055556in}{0.000000in}}{\pgfqpoint{0.000000in}{0.000000in}}{%
\pgfpathmoveto{\pgfqpoint{0.000000in}{0.000000in}}%
\pgfpathlineto{\pgfqpoint{-0.055556in}{0.000000in}}%
\pgfusepath{stroke,fill}%
}%
\begin{pgfscope}%
\pgfsys@transformshift{7.200000in}{1.900000in}%
\pgfsys@useobject{currentmarker}{}%
\end{pgfscope}%
\end{pgfscope}%
\begin{pgfscope}%
\pgftext[left,bottom,x=0.679389in,y=1.836971in,rotate=0.000000]{{\sffamily\fontsize{12.000000}{14.400000}\selectfont 5.0}}
%
\end{pgfscope}%
\begin{pgfscope}%
\pgfpathrectangle{\pgfqpoint{1.000000in}{0.300000in}}{\pgfqpoint{6.200000in}{2.400000in}} %
\pgfusepath{clip}%
\pgfsetbuttcap%
\pgfsetroundjoin%
\pgfsetlinewidth{0.501875pt}%
\definecolor{currentstroke}{rgb}{0.000000,0.000000,0.000000}%
\pgfsetstrokecolor{currentstroke}%
\pgfsetdash{{1.000000pt}{3.000000pt}}{0.000000pt}%
\pgfpathmoveto{\pgfqpoint{1.000000in}{2.166667in}}%
\pgfpathlineto{\pgfqpoint{7.200000in}{2.166667in}}%
\pgfusepath{stroke}%
\end{pgfscope}%
\begin{pgfscope}%
\pgfsetbuttcap%
\pgfsetroundjoin%
\definecolor{currentfill}{rgb}{0.000000,0.000000,0.000000}%
\pgfsetfillcolor{currentfill}%
\pgfsetlinewidth{0.501875pt}%
\definecolor{currentstroke}{rgb}{0.000000,0.000000,0.000000}%
\pgfsetstrokecolor{currentstroke}%
\pgfsetdash{}{0pt}%
\pgfsys@defobject{currentmarker}{\pgfqpoint{0.000000in}{0.000000in}}{\pgfqpoint{0.055556in}{0.000000in}}{%
\pgfpathmoveto{\pgfqpoint{0.000000in}{0.000000in}}%
\pgfpathlineto{\pgfqpoint{0.055556in}{0.000000in}}%
\pgfusepath{stroke,fill}%
}%
\begin{pgfscope}%
\pgfsys@transformshift{1.000000in}{2.166667in}%
\pgfsys@useobject{currentmarker}{}%
\end{pgfscope}%
\end{pgfscope}%
\begin{pgfscope}%
\pgfsetbuttcap%
\pgfsetroundjoin%
\definecolor{currentfill}{rgb}{0.000000,0.000000,0.000000}%
\pgfsetfillcolor{currentfill}%
\pgfsetlinewidth{0.501875pt}%
\definecolor{currentstroke}{rgb}{0.000000,0.000000,0.000000}%
\pgfsetstrokecolor{currentstroke}%
\pgfsetdash{}{0pt}%
\pgfsys@defobject{currentmarker}{\pgfqpoint{-0.055556in}{0.000000in}}{\pgfqpoint{0.000000in}{0.000000in}}{%
\pgfpathmoveto{\pgfqpoint{0.000000in}{0.000000in}}%
\pgfpathlineto{\pgfqpoint{-0.055556in}{0.000000in}}%
\pgfusepath{stroke,fill}%
}%
\begin{pgfscope}%
\pgfsys@transformshift{7.200000in}{2.166667in}%
\pgfsys@useobject{currentmarker}{}%
\end{pgfscope}%
\end{pgfscope}%
\begin{pgfscope}%
\pgftext[left,bottom,x=0.679389in,y=2.104736in,rotate=0.000000]{{\sffamily\fontsize{12.000000}{14.400000}\selectfont 5.5}}
%
\end{pgfscope}%
\begin{pgfscope}%
\pgfpathrectangle{\pgfqpoint{1.000000in}{0.300000in}}{\pgfqpoint{6.200000in}{2.400000in}} %
\pgfusepath{clip}%
\pgfsetbuttcap%
\pgfsetroundjoin%
\pgfsetlinewidth{0.501875pt}%
\definecolor{currentstroke}{rgb}{0.000000,0.000000,0.000000}%
\pgfsetstrokecolor{currentstroke}%
\pgfsetdash{{1.000000pt}{3.000000pt}}{0.000000pt}%
\pgfpathmoveto{\pgfqpoint{1.000000in}{2.433333in}}%
\pgfpathlineto{\pgfqpoint{7.200000in}{2.433333in}}%
\pgfusepath{stroke}%
\end{pgfscope}%
\begin{pgfscope}%
\pgfsetbuttcap%
\pgfsetroundjoin%
\definecolor{currentfill}{rgb}{0.000000,0.000000,0.000000}%
\pgfsetfillcolor{currentfill}%
\pgfsetlinewidth{0.501875pt}%
\definecolor{currentstroke}{rgb}{0.000000,0.000000,0.000000}%
\pgfsetstrokecolor{currentstroke}%
\pgfsetdash{}{0pt}%
\pgfsys@defobject{currentmarker}{\pgfqpoint{0.000000in}{0.000000in}}{\pgfqpoint{0.055556in}{0.000000in}}{%
\pgfpathmoveto{\pgfqpoint{0.000000in}{0.000000in}}%
\pgfpathlineto{\pgfqpoint{0.055556in}{0.000000in}}%
\pgfusepath{stroke,fill}%
}%
\begin{pgfscope}%
\pgfsys@transformshift{1.000000in}{2.433333in}%
\pgfsys@useobject{currentmarker}{}%
\end{pgfscope}%
\end{pgfscope}%
\begin{pgfscope}%
\pgfsetbuttcap%
\pgfsetroundjoin%
\definecolor{currentfill}{rgb}{0.000000,0.000000,0.000000}%
\pgfsetfillcolor{currentfill}%
\pgfsetlinewidth{0.501875pt}%
\definecolor{currentstroke}{rgb}{0.000000,0.000000,0.000000}%
\pgfsetstrokecolor{currentstroke}%
\pgfsetdash{}{0pt}%
\pgfsys@defobject{currentmarker}{\pgfqpoint{-0.055556in}{0.000000in}}{\pgfqpoint{0.000000in}{0.000000in}}{%
\pgfpathmoveto{\pgfqpoint{0.000000in}{0.000000in}}%
\pgfpathlineto{\pgfqpoint{-0.055556in}{0.000000in}}%
\pgfusepath{stroke,fill}%
}%
\begin{pgfscope}%
\pgfsys@transformshift{7.200000in}{2.433333in}%
\pgfsys@useobject{currentmarker}{}%
\end{pgfscope}%
\end{pgfscope}%
\begin{pgfscope}%
\pgftext[left,bottom,x=0.679389in,y=2.370305in,rotate=0.000000]{{\sffamily\fontsize{12.000000}{14.400000}\selectfont 6.0}}
%
\end{pgfscope}%
\begin{pgfscope}%
\pgfpathrectangle{\pgfqpoint{1.000000in}{0.300000in}}{\pgfqpoint{6.200000in}{2.400000in}} %
\pgfusepath{clip}%
\pgfsetbuttcap%
\pgfsetroundjoin%
\pgfsetlinewidth{0.501875pt}%
\definecolor{currentstroke}{rgb}{0.000000,0.000000,0.000000}%
\pgfsetstrokecolor{currentstroke}%
\pgfsetdash{{1.000000pt}{3.000000pt}}{0.000000pt}%
\pgfpathmoveto{\pgfqpoint{1.000000in}{2.700000in}}%
\pgfpathlineto{\pgfqpoint{7.200000in}{2.700000in}}%
\pgfusepath{stroke}%
\end{pgfscope}%
\begin{pgfscope}%
\pgfsetbuttcap%
\pgfsetroundjoin%
\definecolor{currentfill}{rgb}{0.000000,0.000000,0.000000}%
\pgfsetfillcolor{currentfill}%
\pgfsetlinewidth{0.501875pt}%
\definecolor{currentstroke}{rgb}{0.000000,0.000000,0.000000}%
\pgfsetstrokecolor{currentstroke}%
\pgfsetdash{}{0pt}%
\pgfsys@defobject{currentmarker}{\pgfqpoint{0.000000in}{0.000000in}}{\pgfqpoint{0.055556in}{0.000000in}}{%
\pgfpathmoveto{\pgfqpoint{0.000000in}{0.000000in}}%
\pgfpathlineto{\pgfqpoint{0.055556in}{0.000000in}}%
\pgfusepath{stroke,fill}%
}%
\begin{pgfscope}%
\pgfsys@transformshift{1.000000in}{2.700000in}%
\pgfsys@useobject{currentmarker}{}%
\end{pgfscope}%
\end{pgfscope}%
\begin{pgfscope}%
\pgfsetbuttcap%
\pgfsetroundjoin%
\definecolor{currentfill}{rgb}{0.000000,0.000000,0.000000}%
\pgfsetfillcolor{currentfill}%
\pgfsetlinewidth{0.501875pt}%
\definecolor{currentstroke}{rgb}{0.000000,0.000000,0.000000}%
\pgfsetstrokecolor{currentstroke}%
\pgfsetdash{}{0pt}%
\pgfsys@defobject{currentmarker}{\pgfqpoint{-0.055556in}{0.000000in}}{\pgfqpoint{0.000000in}{0.000000in}}{%
\pgfpathmoveto{\pgfqpoint{0.000000in}{0.000000in}}%
\pgfpathlineto{\pgfqpoint{-0.055556in}{0.000000in}}%
\pgfusepath{stroke,fill}%
}%
\begin{pgfscope}%
\pgfsys@transformshift{7.200000in}{2.700000in}%
\pgfsys@useobject{currentmarker}{}%
\end{pgfscope}%
\end{pgfscope}%
\begin{pgfscope}%
\pgftext[left,bottom,x=0.679389in,y=2.636971in,rotate=0.000000]{{\sffamily\fontsize{12.000000}{14.400000}\selectfont 6.5}}
%
\end{pgfscope}%
\begin{pgfscope}%
\pgftext[left,bottom,x=0.609945in,y=0.148085in,rotate=90.000000]{{\sffamily\fontsize{12.000000}{14.400000}\selectfont diffusion coefficient [10\(\displaystyle ^{-5}\) cm\(\displaystyle ^2\)/s]}}
%
\end{pgfscope}%
\begin{pgfscope}%
\pgfsetrectcap%
\pgfsetroundjoin%
\pgfsetlinewidth{1.003750pt}%
\definecolor{currentstroke}{rgb}{0.000000,0.000000,0.000000}%
\pgfsetstrokecolor{currentstroke}%
\pgfsetdash{}{0pt}%
\pgfpathmoveto{\pgfqpoint{1.000000in}{2.700000in}}%
\pgfpathlineto{\pgfqpoint{7.200000in}{2.700000in}}%
\pgfusepath{stroke}%
\end{pgfscope}%
\begin{pgfscope}%
\pgfsetrectcap%
\pgfsetroundjoin%
\pgfsetlinewidth{1.003750pt}%
\definecolor{currentstroke}{rgb}{0.000000,0.000000,0.000000}%
\pgfsetstrokecolor{currentstroke}%
\pgfsetdash{}{0pt}%
\pgfpathmoveto{\pgfqpoint{7.200000in}{0.300000in}}%
\pgfpathlineto{\pgfqpoint{7.200000in}{2.700000in}}%
\pgfusepath{stroke}%
\end{pgfscope}%
\begin{pgfscope}%
\pgfsetrectcap%
\pgfsetroundjoin%
\pgfsetlinewidth{1.003750pt}%
\definecolor{currentstroke}{rgb}{0.000000,0.000000,0.000000}%
\pgfsetstrokecolor{currentstroke}%
\pgfsetdash{}{0pt}%
\pgfpathmoveto{\pgfqpoint{1.000000in}{0.300000in}}%
\pgfpathlineto{\pgfqpoint{7.200000in}{0.300000in}}%
\pgfusepath{stroke}%
\end{pgfscope}%
\begin{pgfscope}%
\pgfsetrectcap%
\pgfsetroundjoin%
\pgfsetlinewidth{1.003750pt}%
\definecolor{currentstroke}{rgb}{0.000000,0.000000,0.000000}%
\pgfsetstrokecolor{currentstroke}%
\pgfsetdash{}{0pt}%
\pgfpathmoveto{\pgfqpoint{1.000000in}{0.300000in}}%
\pgfpathlineto{\pgfqpoint{1.000000in}{2.700000in}}%
\pgfusepath{stroke}%
\end{pgfscope}%
\begin{pgfscope}%
\pgfsetrectcap%
\pgfsetroundjoin%
\definecolor{currentfill}{rgb}{1.000000,1.000000,1.000000}%
\pgfsetfillcolor{currentfill}%
\pgfsetlinewidth{1.003750pt}%
\definecolor{currentstroke}{rgb}{0.000000,0.000000,0.000000}%
\pgfsetstrokecolor{currentstroke}%
\pgfsetdash{}{0pt}%
\pgfpathmoveto{\pgfqpoint{1.069417in}{1.977606in}}%
\pgfpathlineto{\pgfqpoint{3.053134in}{1.977606in}}%
\pgfpathlineto{\pgfqpoint{3.053134in}{2.630583in}}%
\pgfpathlineto{\pgfqpoint{1.069417in}{2.630583in}}%
\pgfpathlineto{\pgfqpoint{1.069417in}{1.977606in}}%
\pgfpathclose%
\pgfusepath{stroke,fill}%
\end{pgfscope}%
\begin{pgfscope}%
\pgfsetrectcap%
\pgfsetroundjoin%
\pgfsetlinewidth{1.003750pt}%
\definecolor{currentstroke}{rgb}{0.000000,0.000000,1.000000}%
\pgfsetstrokecolor{currentstroke}%
\pgfsetdash{}{0pt}%
\pgfpathmoveto{\pgfqpoint{1.166600in}{2.518161in}}%
\pgfpathlineto{\pgfqpoint{1.360967in}{2.518161in}}%
\pgfusepath{stroke}%
\end{pgfscope}%
\begin{pgfscope}%
\pgftext[left,bottom,x=1.513683in,y=2.440691in,rotate=0.000000]{{\sffamily\fontsize{9.996000}{11.995200}\selectfont spc, mean: 3.47378}}
%
\end{pgfscope}%
\begin{pgfscope}%
\pgfsetrectcap%
\pgfsetroundjoin%
\pgfsetlinewidth{1.003750pt}%
\definecolor{currentstroke}{rgb}{0.000000,0.500000,0.000000}%
\pgfsetstrokecolor{currentstroke}%
\pgfsetdash{}{0pt}%
\pgfpathmoveto{\pgfqpoint{1.166600in}{2.314385in}}%
\pgfpathlineto{\pgfqpoint{1.360967in}{2.314385in}}%
\pgfusepath{stroke}%
\end{pgfscope}%
\begin{pgfscope}%
\pgftext[left,bottom,x=1.513683in,y=2.236915in,rotate=0.000000]{{\sffamily\fontsize{9.996000}{11.995200}\selectfont spce, mean: 2.48983}}
%
\end{pgfscope}%
\begin{pgfscope}%
\pgfsetrectcap%
\pgfsetroundjoin%
\pgfsetlinewidth{1.003750pt}%
\definecolor{currentstroke}{rgb}{1.000000,0.000000,0.000000}%
\pgfsetstrokecolor{currentstroke}%
\pgfsetdash{}{0pt}%
\pgfpathmoveto{\pgfqpoint{1.166600in}{2.110609in}}%
\pgfpathlineto{\pgfqpoint{1.360967in}{2.110609in}}%
\pgfusepath{stroke}%
\end{pgfscope}%
\begin{pgfscope}%
\pgftext[left,bottom,x=1.513683in,y=2.033139in,rotate=0.000000]{{\sffamily\fontsize{9.996000}{11.995200}\selectfont tip3p, mean: 5.0717}}
%
\end{pgfscope}%
\end{pgfpicture}%
\makeatother%
\endgroup%
}
    		\caption{LONG}
		\end{subfigure}
        \caption{LONG} \label{fig:latticeLDA}
\end{figure}
