%%%%%%%% Klassen-Optionen
\documentclass[12pt,a4paper]{scrartcl}

%%%%%%%% PAKETE: unverzichtbare Pakete mit Einstellungen
\usepackage[left=2.5cm, right=2cm, top=3cm, bottom=3cm, a4paper]{geometry} %Seitenrände
\usepackage[utf8x]{inputenc} % utf8-Kodierung und direkte Eingabe von Sonderzeichen
\usepackage{fixltx2e} % Verbessert einige Kernkompetenzen von LaTeX2e

%%%%%%%% PAKETE: AMS-Pakete
\usepackage{amsmath} % Mathe-Erweiterung
\usepackage{amsfonts} % Schrift-Erweiterung
\usepackage{amssymb} % Sonderzeichen-Erweiterung

%%%%%%%% PAKETE: Sonstiges
\usepackage[colorlinks, citecolor=black, filecolor=black, linkcolor=black, urlcolor=black]{hyperref} % Links
\usepackage{wrapfig} % ausgeklügekte Floatumgebung
\usepackage{float} % normale Floatumgebung
\restylefloat{figure} % ermöglicht die Verwendung von "H" (ist noch stärker als "h!")
\usepackage[small,it,singlelinecheck=false]{caption} % Bildunterschriften formatieren
\usepackage{multirow} % ermöglich Verbinden von Tabellenzeilen
\usepackage{multicol} % ermöglicht Spalten
\usepackage{fancyhdr} % ermöglicht Kopf- und Fußzeilen
\usepackage{graphicx} % Einbinden von Bildern möglich
\usepackage{units} % Einheiten
\usepackage{subcaption}

%%%%%%%% DEFINITIONEN: Titelseite
\author{Patrick Kreissl und Sebastian Weber}
\title{Worksheet 1: Integrators}
\publishers{University of Stuttgart}
\date{\today}

%%%%%%%% ANPASSUNGEN: Kopf-und Fußzeile
\fancypagestyle{plain}{} % redefine the plain pagestyle to match the fancy layout
\pagestyle{fancy} % aktiviere eigenen Seitenstil
\fancyhf{} % alle Kopf- und Fußzeilen bereinigen
\fancyhead[L]{Worksheet 1: Integrators}
\fancyhead[R]{\today}
\renewcommand{\headrulewidth}{0.6pt} % obere Trennlinie
\fancyfoot[L]{Patrick Kreissl und Sebastian Weber}
\fancyfoot[R]{Seite \thepage}
\renewcommand{\footrulewidth}{0.6pt} % untere Trennlinie

%%%%%%%% ANPASSUNGEN: Absätze
\setlength{\parindent}{0em} % keine Absatzeinzüge
\setlength{\parskip}{0em} % Absatz-Abstand

%%%%%%%% ANPASSUNGEN: Abbildungsverzeichnis
\usepackage{tocloft} % Zum Anpassen der Verzeichnisse
\renewcommand{\cftfigpresnum}{Abb. }
\renewcommand{\cfttabpresnum}{Tab. }
\renewcommand{\cftfigaftersnum}{:}
\renewcommand{\cfttabaftersnum}{:}
\setlength{\cftfignumwidth}{2cm}
\setlength{\cfttabnumwidth}{2cm}
\setlength{\cftfigindent}{0cm}
\setlength{\cfttabindent}{0cm}

%%%%%%%% SONSTIGES
\usepackage{pdfpages}
\usepackage{pgf}



% NÜTZLICH: http://truben.no/latex/table/

% Anfang des eigentlichen Dokuments
\begin{document}

% Anpassungen
%\renewcommand{\figurename}{Abb.}
\renewcommand{\tablename}{Tab.}

\maketitle

\tableofcontents

\newpage
\section{Cannonball}

A program was written to simulate the flight of a cannonball thrown with the velocity $\vec{v} = (50, 50)^T \tfrac{m}{s}$ in two dimensions \footnote{for the code please take a look at the  appendix}. Therefore the Euler integration scheme was used.

\begin{figure}[H]
	\resizebox{1\textwidth}{!}{%% Creator: Matplotlib, PGF backend
%%
%% To include the figure in your LaTeX document, write
%%   \input{<filename>.pgf}
%%
%% Make sure the required packages are loaded in your preamble
%%   \usepackage{pgf}
%%
%% Figures using additional raster images can only be included by \input if
%% they are in the same directory as the main LaTeX file. For loading figures
%% from other directories you can use the `import` package
%%   \usepackage{import}
%% and then include the figures with
%%   \import{<path to file>}{<filename>.pgf}
%%
%% Matplotlib used the following preamble
%%   \usepackage{fontspec}
%%   \setmainfont{DejaVu Serif}
%%   \setsansfont{DejaVu Sans}
%%   \setmonofont{DejaVu Sans Mono}
%%
\begingroup%
\makeatletter%
\begin{pgfpicture}%
\pgfpathrectangle{\pgfpointorigin}{\pgfqpoint{10.000000in}{4.000000in}}%
\pgfusepath{use as bounding box}%
\begin{pgfscope}%
\pgfsetrectcap%
\pgfsetroundjoin%
\definecolor{currentfill}{rgb}{1.000000,1.000000,1.000000}%
\pgfsetfillcolor{currentfill}%
\pgfsetlinewidth{0.000000pt}%
\definecolor{currentstroke}{rgb}{1.000000,1.000000,1.000000}%
\pgfsetstrokecolor{currentstroke}%
\pgfsetdash{}{0pt}%
\pgfpathmoveto{\pgfqpoint{0.000000in}{0.000000in}}%
\pgfpathlineto{\pgfqpoint{10.000000in}{0.000000in}}%
\pgfpathlineto{\pgfqpoint{10.000000in}{4.000000in}}%
\pgfpathlineto{\pgfqpoint{0.000000in}{4.000000in}}%
\pgfpathclose%
\pgfusepath{fill}%
\end{pgfscope}%
\begin{pgfscope}%
\pgfsetrectcap%
\pgfsetroundjoin%
\definecolor{currentfill}{rgb}{1.000000,1.000000,1.000000}%
\pgfsetfillcolor{currentfill}%
\pgfsetlinewidth{0.000000pt}%
\definecolor{currentstroke}{rgb}{0.000000,0.000000,0.000000}%
\pgfsetstrokecolor{currentstroke}%
\pgfsetdash{}{0pt}%
\pgfpathmoveto{\pgfqpoint{1.250000in}{0.400000in}}%
\pgfpathlineto{\pgfqpoint{9.000000in}{0.400000in}}%
\pgfpathlineto{\pgfqpoint{9.000000in}{3.600000in}}%
\pgfpathlineto{\pgfqpoint{1.250000in}{3.600000in}}%
\pgfpathclose%
\pgfusepath{fill}%
\end{pgfscope}%
\begin{pgfscope}%
\pgfpathrectangle{\pgfqpoint{1.250000in}{0.400000in}}{\pgfqpoint{7.750000in}{3.200000in}} %
\pgfusepath{clip}%
\pgfsetrectcap%
\pgfsetroundjoin%
\pgfsetlinewidth{1.003750pt}%
\definecolor{currentstroke}{rgb}{1.000000,0.000000,0.000000}%
\pgfsetstrokecolor{currentstroke}%
\pgfsetdash{}{0pt}%
\pgfpathmoveto{\pgfqpoint{1.250000in}{0.800000in}}%
\pgfpathlineto{\pgfqpoint{1.314583in}{0.898038in}}%
\pgfpathlineto{\pgfqpoint{1.379167in}{0.994114in}}%
\pgfpathlineto{\pgfqpoint{1.443750in}{1.088228in}}%
\pgfpathlineto{\pgfqpoint{1.508333in}{1.180380in}}%
\pgfpathlineto{\pgfqpoint{1.572917in}{1.270570in}}%
\pgfpathlineto{\pgfqpoint{1.637500in}{1.358798in}}%
\pgfpathlineto{\pgfqpoint{1.702083in}{1.445064in}}%
\pgfpathlineto{\pgfqpoint{1.766667in}{1.529368in}}%
\pgfpathlineto{\pgfqpoint{1.831250in}{1.611710in}}%
\pgfpathlineto{\pgfqpoint{1.895833in}{1.692090in}}%
\pgfpathlineto{\pgfqpoint{1.960417in}{1.770508in}}%
\pgfpathlineto{\pgfqpoint{2.025000in}{1.846964in}}%
\pgfpathlineto{\pgfqpoint{2.089583in}{1.921458in}}%
\pgfpathlineto{\pgfqpoint{2.154167in}{1.993990in}}%
\pgfpathlineto{\pgfqpoint{2.218750in}{2.064560in}}%
\pgfpathlineto{\pgfqpoint{2.283333in}{2.133168in}}%
\pgfpathlineto{\pgfqpoint{2.347917in}{2.199814in}}%
\pgfpathlineto{\pgfqpoint{2.412500in}{2.264498in}}%
\pgfpathlineto{\pgfqpoint{2.477083in}{2.327220in}}%
\pgfpathlineto{\pgfqpoint{2.541667in}{2.387980in}}%
\pgfpathlineto{\pgfqpoint{2.606250in}{2.446778in}}%
\pgfpathlineto{\pgfqpoint{2.670833in}{2.503614in}}%
\pgfpathlineto{\pgfqpoint{2.735417in}{2.558488in}}%
\pgfpathlineto{\pgfqpoint{2.800000in}{2.611400in}}%
\pgfpathlineto{\pgfqpoint{2.864583in}{2.662350in}}%
\pgfpathlineto{\pgfqpoint{2.929167in}{2.711338in}}%
\pgfpathlineto{\pgfqpoint{2.993750in}{2.758364in}}%
\pgfpathlineto{\pgfqpoint{3.058333in}{2.803428in}}%
\pgfpathlineto{\pgfqpoint{3.122917in}{2.846530in}}%
\pgfpathlineto{\pgfqpoint{3.187500in}{2.887670in}}%
\pgfpathlineto{\pgfqpoint{3.252083in}{2.926848in}}%
\pgfpathlineto{\pgfqpoint{3.316667in}{2.964064in}}%
\pgfpathlineto{\pgfqpoint{3.381250in}{2.999318in}}%
\pgfpathlineto{\pgfqpoint{3.445833in}{3.032610in}}%
\pgfpathlineto{\pgfqpoint{3.510417in}{3.063940in}}%
\pgfpathlineto{\pgfqpoint{3.575000in}{3.093308in}}%
\pgfpathlineto{\pgfqpoint{3.639583in}{3.120714in}}%
\pgfpathlineto{\pgfqpoint{3.704167in}{3.146158in}}%
\pgfpathlineto{\pgfqpoint{3.768750in}{3.169640in}}%
\pgfpathlineto{\pgfqpoint{3.833333in}{3.191160in}}%
\pgfpathlineto{\pgfqpoint{3.897917in}{3.210718in}}%
\pgfpathlineto{\pgfqpoint{3.962500in}{3.228314in}}%
\pgfpathlineto{\pgfqpoint{4.027083in}{3.243948in}}%
\pgfpathlineto{\pgfqpoint{4.091667in}{3.257620in}}%
\pgfpathlineto{\pgfqpoint{4.156250in}{3.269330in}}%
\pgfpathlineto{\pgfqpoint{4.220833in}{3.279078in}}%
\pgfpathlineto{\pgfqpoint{4.285417in}{3.286864in}}%
\pgfpathlineto{\pgfqpoint{4.350000in}{3.292688in}}%
\pgfpathlineto{\pgfqpoint{4.414583in}{3.296550in}}%
\pgfpathlineto{\pgfqpoint{4.479167in}{3.298450in}}%
\pgfpathlineto{\pgfqpoint{4.543750in}{3.298388in}}%
\pgfpathlineto{\pgfqpoint{4.608333in}{3.296364in}}%
\pgfpathlineto{\pgfqpoint{4.672917in}{3.292378in}}%
\pgfpathlineto{\pgfqpoint{4.737500in}{3.286430in}}%
\pgfpathlineto{\pgfqpoint{4.802083in}{3.278520in}}%
\pgfpathlineto{\pgfqpoint{4.866667in}{3.268648in}}%
\pgfpathlineto{\pgfqpoint{4.931250in}{3.256814in}}%
\pgfpathlineto{\pgfqpoint{4.995833in}{3.243018in}}%
\pgfpathlineto{\pgfqpoint{5.060417in}{3.227260in}}%
\pgfpathlineto{\pgfqpoint{5.125000in}{3.209540in}}%
\pgfpathlineto{\pgfqpoint{5.189583in}{3.189858in}}%
\pgfpathlineto{\pgfqpoint{5.254167in}{3.168214in}}%
\pgfpathlineto{\pgfqpoint{5.318750in}{3.144608in}}%
\pgfpathlineto{\pgfqpoint{5.383333in}{3.119040in}}%
\pgfpathlineto{\pgfqpoint{5.447917in}{3.091510in}}%
\pgfpathlineto{\pgfqpoint{5.512500in}{3.062018in}}%
\pgfpathlineto{\pgfqpoint{5.577083in}{3.030564in}}%
\pgfpathlineto{\pgfqpoint{5.641667in}{2.997148in}}%
\pgfpathlineto{\pgfqpoint{5.706250in}{2.961770in}}%
\pgfpathlineto{\pgfqpoint{5.770833in}{2.924430in}}%
\pgfpathlineto{\pgfqpoint{5.835417in}{2.885128in}}%
\pgfpathlineto{\pgfqpoint{5.900000in}{2.843864in}}%
\pgfpathlineto{\pgfqpoint{5.964583in}{2.800638in}}%
\pgfpathlineto{\pgfqpoint{6.029167in}{2.755450in}}%
\pgfpathlineto{\pgfqpoint{6.093750in}{2.708300in}}%
\pgfpathlineto{\pgfqpoint{6.158333in}{2.659188in}}%
\pgfpathlineto{\pgfqpoint{6.222917in}{2.608114in}}%
\pgfpathlineto{\pgfqpoint{6.287500in}{2.555078in}}%
\pgfpathlineto{\pgfqpoint{6.352083in}{2.500080in}}%
\pgfpathlineto{\pgfqpoint{6.416667in}{2.443120in}}%
\pgfpathlineto{\pgfqpoint{6.481250in}{2.384198in}}%
\pgfpathlineto{\pgfqpoint{6.545833in}{2.323314in}}%
\pgfpathlineto{\pgfqpoint{6.610417in}{2.260468in}}%
\pgfpathlineto{\pgfqpoint{6.675000in}{2.195660in}}%
\pgfpathlineto{\pgfqpoint{6.739583in}{2.128890in}}%
\pgfpathlineto{\pgfqpoint{6.804167in}{2.060158in}}%
\pgfpathlineto{\pgfqpoint{6.868750in}{1.989464in}}%
\pgfpathlineto{\pgfqpoint{6.933333in}{1.916808in}}%
\pgfpathlineto{\pgfqpoint{6.997917in}{1.842190in}}%
\pgfpathlineto{\pgfqpoint{7.062500in}{1.765610in}}%
\pgfpathlineto{\pgfqpoint{7.127083in}{1.687068in}}%
\pgfpathlineto{\pgfqpoint{7.191667in}{1.606564in}}%
\pgfpathlineto{\pgfqpoint{7.256250in}{1.524098in}}%
\pgfpathlineto{\pgfqpoint{7.320833in}{1.439670in}}%
\pgfpathlineto{\pgfqpoint{7.385417in}{1.353280in}}%
\pgfpathlineto{\pgfqpoint{7.450000in}{1.264928in}}%
\pgfpathlineto{\pgfqpoint{7.514583in}{1.174614in}}%
\pgfpathlineto{\pgfqpoint{7.579167in}{1.082338in}}%
\pgfpathlineto{\pgfqpoint{7.643750in}{0.988100in}}%
\pgfpathlineto{\pgfqpoint{7.708333in}{0.891900in}}%
\pgfpathlineto{\pgfqpoint{7.772917in}{0.793738in}}%
\pgfusepath{stroke}%
\end{pgfscope}%
\begin{pgfscope}%
\pgfpathrectangle{\pgfqpoint{1.250000in}{0.400000in}}{\pgfqpoint{7.750000in}{3.200000in}} %
\pgfusepath{clip}%
\pgfsetbuttcap%
\pgfsetroundjoin%
\definecolor{currentfill}{rgb}{1.000000,0.000000,0.000000}%
\pgfsetfillcolor{currentfill}%
\pgfsetlinewidth{0.501875pt}%
\definecolor{currentstroke}{rgb}{1.000000,0.000000,0.000000}%
\pgfsetstrokecolor{currentstroke}%
\pgfsetdash{}{0pt}%
\pgfsys@defobject{currentmarker}{\pgfqpoint{-0.041667in}{-0.041667in}}{\pgfqpoint{0.041667in}{0.041667in}}{%
\pgfpathmoveto{\pgfqpoint{-0.041667in}{0.000000in}}%
\pgfpathlineto{\pgfqpoint{0.041667in}{0.000000in}}%
\pgfpathmoveto{\pgfqpoint{0.000000in}{-0.041667in}}%
\pgfpathlineto{\pgfqpoint{0.000000in}{0.041667in}}%
\pgfusepath{stroke,fill}%
}%
\begin{pgfscope}%
\pgfsys@transformshift{1.250000in}{0.800000in}%
\pgfsys@useobject{currentmarker}{}%
\end{pgfscope}%
\begin{pgfscope}%
\pgfsys@transformshift{1.314583in}{0.898038in}%
\pgfsys@useobject{currentmarker}{}%
\end{pgfscope}%
\begin{pgfscope}%
\pgfsys@transformshift{1.379167in}{0.994114in}%
\pgfsys@useobject{currentmarker}{}%
\end{pgfscope}%
\begin{pgfscope}%
\pgfsys@transformshift{1.443750in}{1.088228in}%
\pgfsys@useobject{currentmarker}{}%
\end{pgfscope}%
\begin{pgfscope}%
\pgfsys@transformshift{1.508333in}{1.180380in}%
\pgfsys@useobject{currentmarker}{}%
\end{pgfscope}%
\begin{pgfscope}%
\pgfsys@transformshift{1.572917in}{1.270570in}%
\pgfsys@useobject{currentmarker}{}%
\end{pgfscope}%
\begin{pgfscope}%
\pgfsys@transformshift{1.637500in}{1.358798in}%
\pgfsys@useobject{currentmarker}{}%
\end{pgfscope}%
\begin{pgfscope}%
\pgfsys@transformshift{1.702083in}{1.445064in}%
\pgfsys@useobject{currentmarker}{}%
\end{pgfscope}%
\begin{pgfscope}%
\pgfsys@transformshift{1.766667in}{1.529368in}%
\pgfsys@useobject{currentmarker}{}%
\end{pgfscope}%
\begin{pgfscope}%
\pgfsys@transformshift{1.831250in}{1.611710in}%
\pgfsys@useobject{currentmarker}{}%
\end{pgfscope}%
\begin{pgfscope}%
\pgfsys@transformshift{1.895833in}{1.692090in}%
\pgfsys@useobject{currentmarker}{}%
\end{pgfscope}%
\begin{pgfscope}%
\pgfsys@transformshift{1.960417in}{1.770508in}%
\pgfsys@useobject{currentmarker}{}%
\end{pgfscope}%
\begin{pgfscope}%
\pgfsys@transformshift{2.025000in}{1.846964in}%
\pgfsys@useobject{currentmarker}{}%
\end{pgfscope}%
\begin{pgfscope}%
\pgfsys@transformshift{2.089583in}{1.921458in}%
\pgfsys@useobject{currentmarker}{}%
\end{pgfscope}%
\begin{pgfscope}%
\pgfsys@transformshift{2.154167in}{1.993990in}%
\pgfsys@useobject{currentmarker}{}%
\end{pgfscope}%
\begin{pgfscope}%
\pgfsys@transformshift{2.218750in}{2.064560in}%
\pgfsys@useobject{currentmarker}{}%
\end{pgfscope}%
\begin{pgfscope}%
\pgfsys@transformshift{2.283333in}{2.133168in}%
\pgfsys@useobject{currentmarker}{}%
\end{pgfscope}%
\begin{pgfscope}%
\pgfsys@transformshift{2.347917in}{2.199814in}%
\pgfsys@useobject{currentmarker}{}%
\end{pgfscope}%
\begin{pgfscope}%
\pgfsys@transformshift{2.412500in}{2.264498in}%
\pgfsys@useobject{currentmarker}{}%
\end{pgfscope}%
\begin{pgfscope}%
\pgfsys@transformshift{2.477083in}{2.327220in}%
\pgfsys@useobject{currentmarker}{}%
\end{pgfscope}%
\begin{pgfscope}%
\pgfsys@transformshift{2.541667in}{2.387980in}%
\pgfsys@useobject{currentmarker}{}%
\end{pgfscope}%
\begin{pgfscope}%
\pgfsys@transformshift{2.606250in}{2.446778in}%
\pgfsys@useobject{currentmarker}{}%
\end{pgfscope}%
\begin{pgfscope}%
\pgfsys@transformshift{2.670833in}{2.503614in}%
\pgfsys@useobject{currentmarker}{}%
\end{pgfscope}%
\begin{pgfscope}%
\pgfsys@transformshift{2.735417in}{2.558488in}%
\pgfsys@useobject{currentmarker}{}%
\end{pgfscope}%
\begin{pgfscope}%
\pgfsys@transformshift{2.800000in}{2.611400in}%
\pgfsys@useobject{currentmarker}{}%
\end{pgfscope}%
\begin{pgfscope}%
\pgfsys@transformshift{2.864583in}{2.662350in}%
\pgfsys@useobject{currentmarker}{}%
\end{pgfscope}%
\begin{pgfscope}%
\pgfsys@transformshift{2.929167in}{2.711338in}%
\pgfsys@useobject{currentmarker}{}%
\end{pgfscope}%
\begin{pgfscope}%
\pgfsys@transformshift{2.993750in}{2.758364in}%
\pgfsys@useobject{currentmarker}{}%
\end{pgfscope}%
\begin{pgfscope}%
\pgfsys@transformshift{3.058333in}{2.803428in}%
\pgfsys@useobject{currentmarker}{}%
\end{pgfscope}%
\begin{pgfscope}%
\pgfsys@transformshift{3.122917in}{2.846530in}%
\pgfsys@useobject{currentmarker}{}%
\end{pgfscope}%
\begin{pgfscope}%
\pgfsys@transformshift{3.187500in}{2.887670in}%
\pgfsys@useobject{currentmarker}{}%
\end{pgfscope}%
\begin{pgfscope}%
\pgfsys@transformshift{3.252083in}{2.926848in}%
\pgfsys@useobject{currentmarker}{}%
\end{pgfscope}%
\begin{pgfscope}%
\pgfsys@transformshift{3.316667in}{2.964064in}%
\pgfsys@useobject{currentmarker}{}%
\end{pgfscope}%
\begin{pgfscope}%
\pgfsys@transformshift{3.381250in}{2.999318in}%
\pgfsys@useobject{currentmarker}{}%
\end{pgfscope}%
\begin{pgfscope}%
\pgfsys@transformshift{3.445833in}{3.032610in}%
\pgfsys@useobject{currentmarker}{}%
\end{pgfscope}%
\begin{pgfscope}%
\pgfsys@transformshift{3.510417in}{3.063940in}%
\pgfsys@useobject{currentmarker}{}%
\end{pgfscope}%
\begin{pgfscope}%
\pgfsys@transformshift{3.575000in}{3.093308in}%
\pgfsys@useobject{currentmarker}{}%
\end{pgfscope}%
\begin{pgfscope}%
\pgfsys@transformshift{3.639583in}{3.120714in}%
\pgfsys@useobject{currentmarker}{}%
\end{pgfscope}%
\begin{pgfscope}%
\pgfsys@transformshift{3.704167in}{3.146158in}%
\pgfsys@useobject{currentmarker}{}%
\end{pgfscope}%
\begin{pgfscope}%
\pgfsys@transformshift{3.768750in}{3.169640in}%
\pgfsys@useobject{currentmarker}{}%
\end{pgfscope}%
\begin{pgfscope}%
\pgfsys@transformshift{3.833333in}{3.191160in}%
\pgfsys@useobject{currentmarker}{}%
\end{pgfscope}%
\begin{pgfscope}%
\pgfsys@transformshift{3.897917in}{3.210718in}%
\pgfsys@useobject{currentmarker}{}%
\end{pgfscope}%
\begin{pgfscope}%
\pgfsys@transformshift{3.962500in}{3.228314in}%
\pgfsys@useobject{currentmarker}{}%
\end{pgfscope}%
\begin{pgfscope}%
\pgfsys@transformshift{4.027083in}{3.243948in}%
\pgfsys@useobject{currentmarker}{}%
\end{pgfscope}%
\begin{pgfscope}%
\pgfsys@transformshift{4.091667in}{3.257620in}%
\pgfsys@useobject{currentmarker}{}%
\end{pgfscope}%
\begin{pgfscope}%
\pgfsys@transformshift{4.156250in}{3.269330in}%
\pgfsys@useobject{currentmarker}{}%
\end{pgfscope}%
\begin{pgfscope}%
\pgfsys@transformshift{4.220833in}{3.279078in}%
\pgfsys@useobject{currentmarker}{}%
\end{pgfscope}%
\begin{pgfscope}%
\pgfsys@transformshift{4.285417in}{3.286864in}%
\pgfsys@useobject{currentmarker}{}%
\end{pgfscope}%
\begin{pgfscope}%
\pgfsys@transformshift{4.350000in}{3.292688in}%
\pgfsys@useobject{currentmarker}{}%
\end{pgfscope}%
\begin{pgfscope}%
\pgfsys@transformshift{4.414583in}{3.296550in}%
\pgfsys@useobject{currentmarker}{}%
\end{pgfscope}%
\begin{pgfscope}%
\pgfsys@transformshift{4.479167in}{3.298450in}%
\pgfsys@useobject{currentmarker}{}%
\end{pgfscope}%
\begin{pgfscope}%
\pgfsys@transformshift{4.543750in}{3.298388in}%
\pgfsys@useobject{currentmarker}{}%
\end{pgfscope}%
\begin{pgfscope}%
\pgfsys@transformshift{4.608333in}{3.296364in}%
\pgfsys@useobject{currentmarker}{}%
\end{pgfscope}%
\begin{pgfscope}%
\pgfsys@transformshift{4.672917in}{3.292378in}%
\pgfsys@useobject{currentmarker}{}%
\end{pgfscope}%
\begin{pgfscope}%
\pgfsys@transformshift{4.737500in}{3.286430in}%
\pgfsys@useobject{currentmarker}{}%
\end{pgfscope}%
\begin{pgfscope}%
\pgfsys@transformshift{4.802083in}{3.278520in}%
\pgfsys@useobject{currentmarker}{}%
\end{pgfscope}%
\begin{pgfscope}%
\pgfsys@transformshift{4.866667in}{3.268648in}%
\pgfsys@useobject{currentmarker}{}%
\end{pgfscope}%
\begin{pgfscope}%
\pgfsys@transformshift{4.931250in}{3.256814in}%
\pgfsys@useobject{currentmarker}{}%
\end{pgfscope}%
\begin{pgfscope}%
\pgfsys@transformshift{4.995833in}{3.243018in}%
\pgfsys@useobject{currentmarker}{}%
\end{pgfscope}%
\begin{pgfscope}%
\pgfsys@transformshift{5.060417in}{3.227260in}%
\pgfsys@useobject{currentmarker}{}%
\end{pgfscope}%
\begin{pgfscope}%
\pgfsys@transformshift{5.125000in}{3.209540in}%
\pgfsys@useobject{currentmarker}{}%
\end{pgfscope}%
\begin{pgfscope}%
\pgfsys@transformshift{5.189583in}{3.189858in}%
\pgfsys@useobject{currentmarker}{}%
\end{pgfscope}%
\begin{pgfscope}%
\pgfsys@transformshift{5.254167in}{3.168214in}%
\pgfsys@useobject{currentmarker}{}%
\end{pgfscope}%
\begin{pgfscope}%
\pgfsys@transformshift{5.318750in}{3.144608in}%
\pgfsys@useobject{currentmarker}{}%
\end{pgfscope}%
\begin{pgfscope}%
\pgfsys@transformshift{5.383333in}{3.119040in}%
\pgfsys@useobject{currentmarker}{}%
\end{pgfscope}%
\begin{pgfscope}%
\pgfsys@transformshift{5.447917in}{3.091510in}%
\pgfsys@useobject{currentmarker}{}%
\end{pgfscope}%
\begin{pgfscope}%
\pgfsys@transformshift{5.512500in}{3.062018in}%
\pgfsys@useobject{currentmarker}{}%
\end{pgfscope}%
\begin{pgfscope}%
\pgfsys@transformshift{5.577083in}{3.030564in}%
\pgfsys@useobject{currentmarker}{}%
\end{pgfscope}%
\begin{pgfscope}%
\pgfsys@transformshift{5.641667in}{2.997148in}%
\pgfsys@useobject{currentmarker}{}%
\end{pgfscope}%
\begin{pgfscope}%
\pgfsys@transformshift{5.706250in}{2.961770in}%
\pgfsys@useobject{currentmarker}{}%
\end{pgfscope}%
\begin{pgfscope}%
\pgfsys@transformshift{5.770833in}{2.924430in}%
\pgfsys@useobject{currentmarker}{}%
\end{pgfscope}%
\begin{pgfscope}%
\pgfsys@transformshift{5.835417in}{2.885128in}%
\pgfsys@useobject{currentmarker}{}%
\end{pgfscope}%
\begin{pgfscope}%
\pgfsys@transformshift{5.900000in}{2.843864in}%
\pgfsys@useobject{currentmarker}{}%
\end{pgfscope}%
\begin{pgfscope}%
\pgfsys@transformshift{5.964583in}{2.800638in}%
\pgfsys@useobject{currentmarker}{}%
\end{pgfscope}%
\begin{pgfscope}%
\pgfsys@transformshift{6.029167in}{2.755450in}%
\pgfsys@useobject{currentmarker}{}%
\end{pgfscope}%
\begin{pgfscope}%
\pgfsys@transformshift{6.093750in}{2.708300in}%
\pgfsys@useobject{currentmarker}{}%
\end{pgfscope}%
\begin{pgfscope}%
\pgfsys@transformshift{6.158333in}{2.659188in}%
\pgfsys@useobject{currentmarker}{}%
\end{pgfscope}%
\begin{pgfscope}%
\pgfsys@transformshift{6.222917in}{2.608114in}%
\pgfsys@useobject{currentmarker}{}%
\end{pgfscope}%
\begin{pgfscope}%
\pgfsys@transformshift{6.287500in}{2.555078in}%
\pgfsys@useobject{currentmarker}{}%
\end{pgfscope}%
\begin{pgfscope}%
\pgfsys@transformshift{6.352083in}{2.500080in}%
\pgfsys@useobject{currentmarker}{}%
\end{pgfscope}%
\begin{pgfscope}%
\pgfsys@transformshift{6.416667in}{2.443120in}%
\pgfsys@useobject{currentmarker}{}%
\end{pgfscope}%
\begin{pgfscope}%
\pgfsys@transformshift{6.481250in}{2.384198in}%
\pgfsys@useobject{currentmarker}{}%
\end{pgfscope}%
\begin{pgfscope}%
\pgfsys@transformshift{6.545833in}{2.323314in}%
\pgfsys@useobject{currentmarker}{}%
\end{pgfscope}%
\begin{pgfscope}%
\pgfsys@transformshift{6.610417in}{2.260468in}%
\pgfsys@useobject{currentmarker}{}%
\end{pgfscope}%
\begin{pgfscope}%
\pgfsys@transformshift{6.675000in}{2.195660in}%
\pgfsys@useobject{currentmarker}{}%
\end{pgfscope}%
\begin{pgfscope}%
\pgfsys@transformshift{6.739583in}{2.128890in}%
\pgfsys@useobject{currentmarker}{}%
\end{pgfscope}%
\begin{pgfscope}%
\pgfsys@transformshift{6.804167in}{2.060158in}%
\pgfsys@useobject{currentmarker}{}%
\end{pgfscope}%
\begin{pgfscope}%
\pgfsys@transformshift{6.868750in}{1.989464in}%
\pgfsys@useobject{currentmarker}{}%
\end{pgfscope}%
\begin{pgfscope}%
\pgfsys@transformshift{6.933333in}{1.916808in}%
\pgfsys@useobject{currentmarker}{}%
\end{pgfscope}%
\begin{pgfscope}%
\pgfsys@transformshift{6.997917in}{1.842190in}%
\pgfsys@useobject{currentmarker}{}%
\end{pgfscope}%
\begin{pgfscope}%
\pgfsys@transformshift{7.062500in}{1.765610in}%
\pgfsys@useobject{currentmarker}{}%
\end{pgfscope}%
\begin{pgfscope}%
\pgfsys@transformshift{7.127083in}{1.687068in}%
\pgfsys@useobject{currentmarker}{}%
\end{pgfscope}%
\begin{pgfscope}%
\pgfsys@transformshift{7.191667in}{1.606564in}%
\pgfsys@useobject{currentmarker}{}%
\end{pgfscope}%
\begin{pgfscope}%
\pgfsys@transformshift{7.256250in}{1.524098in}%
\pgfsys@useobject{currentmarker}{}%
\end{pgfscope}%
\begin{pgfscope}%
\pgfsys@transformshift{7.320833in}{1.439670in}%
\pgfsys@useobject{currentmarker}{}%
\end{pgfscope}%
\begin{pgfscope}%
\pgfsys@transformshift{7.385417in}{1.353280in}%
\pgfsys@useobject{currentmarker}{}%
\end{pgfscope}%
\begin{pgfscope}%
\pgfsys@transformshift{7.450000in}{1.264928in}%
\pgfsys@useobject{currentmarker}{}%
\end{pgfscope}%
\begin{pgfscope}%
\pgfsys@transformshift{7.514583in}{1.174614in}%
\pgfsys@useobject{currentmarker}{}%
\end{pgfscope}%
\begin{pgfscope}%
\pgfsys@transformshift{7.579167in}{1.082338in}%
\pgfsys@useobject{currentmarker}{}%
\end{pgfscope}%
\begin{pgfscope}%
\pgfsys@transformshift{7.643750in}{0.988100in}%
\pgfsys@useobject{currentmarker}{}%
\end{pgfscope}%
\begin{pgfscope}%
\pgfsys@transformshift{7.708333in}{0.891900in}%
\pgfsys@useobject{currentmarker}{}%
\end{pgfscope}%
\begin{pgfscope}%
\pgfsys@transformshift{7.772917in}{0.793738in}%
\pgfsys@useobject{currentmarker}{}%
\end{pgfscope}%
\end{pgfscope}%
\begin{pgfscope}%
\pgfpathrectangle{\pgfqpoint{1.250000in}{0.400000in}}{\pgfqpoint{7.750000in}{3.200000in}} %
\pgfusepath{clip}%
\pgfsetrectcap%
\pgfsetroundjoin%
\pgfsetlinewidth{1.003750pt}%
\definecolor{currentstroke}{rgb}{0.000000,0.000000,1.000000}%
\pgfsetstrokecolor{currentstroke}%
\pgfsetdash{}{0pt}%
\pgfpathmoveto{\pgfqpoint{1.250000in}{0.800000in}}%
\pgfpathlineto{\pgfqpoint{1.314260in}{0.897538in}}%
\pgfpathlineto{\pgfqpoint{1.378200in}{0.992626in}}%
\pgfpathlineto{\pgfqpoint{1.441819in}{1.085277in}}%
\pgfpathlineto{\pgfqpoint{1.505120in}{1.175503in}}%
\pgfpathlineto{\pgfqpoint{1.568105in}{1.263315in}}%
\pgfpathlineto{\pgfqpoint{1.630775in}{1.348727in}}%
\pgfpathlineto{\pgfqpoint{1.693132in}{1.431749in}}%
\pgfpathlineto{\pgfqpoint{1.755176in}{1.512394in}}%
\pgfpathlineto{\pgfqpoint{1.816911in}{1.590674in}}%
\pgfpathlineto{\pgfqpoint{1.878337in}{1.666601in}}%
\pgfpathlineto{\pgfqpoint{1.939455in}{1.740186in}}%
\pgfpathlineto{\pgfqpoint{2.000269in}{1.811441in}}%
\pgfpathlineto{\pgfqpoint{2.060778in}{1.880378in}}%
\pgfpathlineto{\pgfqpoint{2.120984in}{1.947008in}}%
\pgfpathlineto{\pgfqpoint{2.180890in}{2.011343in}}%
\pgfpathlineto{\pgfqpoint{2.240496in}{2.073394in}}%
\pgfpathlineto{\pgfqpoint{2.299804in}{2.133173in}}%
\pgfpathlineto{\pgfqpoint{2.358815in}{2.190691in}}%
\pgfpathlineto{\pgfqpoint{2.417531in}{2.245960in}}%
\pgfpathlineto{\pgfqpoint{2.475954in}{2.298990in}}%
\pgfpathlineto{\pgfqpoint{2.534085in}{2.349793in}}%
\pgfpathlineto{\pgfqpoint{2.591925in}{2.398380in}}%
\pgfpathlineto{\pgfqpoint{2.649475in}{2.444762in}}%
\pgfpathlineto{\pgfqpoint{2.706739in}{2.488950in}}%
\pgfpathlineto{\pgfqpoint{2.763715in}{2.530956in}}%
\pgfpathlineto{\pgfqpoint{2.820407in}{2.570789in}}%
\pgfpathlineto{\pgfqpoint{2.876815in}{2.608461in}}%
\pgfpathlineto{\pgfqpoint{2.932942in}{2.643983in}}%
\pgfpathlineto{\pgfqpoint{2.988788in}{2.677365in}}%
\pgfpathlineto{\pgfqpoint{3.044354in}{2.708618in}}%
\pgfpathlineto{\pgfqpoint{3.099643in}{2.737753in}}%
\pgfpathlineto{\pgfqpoint{3.154655in}{2.764780in}}%
\pgfpathlineto{\pgfqpoint{3.209392in}{2.789710in}}%
\pgfpathlineto{\pgfqpoint{3.263855in}{2.812554in}}%
\pgfpathlineto{\pgfqpoint{3.318047in}{2.833321in}}%
\pgfpathlineto{\pgfqpoint{3.371967in}{2.852022in}}%
\pgfpathlineto{\pgfqpoint{3.425617in}{2.868668in}}%
\pgfpathlineto{\pgfqpoint{3.479000in}{2.883269in}}%
\pgfpathlineto{\pgfqpoint{3.532115in}{2.895834in}}%
\pgfpathlineto{\pgfqpoint{3.584965in}{2.906375in}}%
\pgfpathlineto{\pgfqpoint{3.637551in}{2.914901in}}%
\pgfpathlineto{\pgfqpoint{3.689873in}{2.921423in}}%
\pgfpathlineto{\pgfqpoint{3.741934in}{2.925950in}}%
\pgfpathlineto{\pgfqpoint{3.793735in}{2.928492in}}%
\pgfpathlineto{\pgfqpoint{3.845277in}{2.929060in}}%
\pgfpathlineto{\pgfqpoint{3.896561in}{2.927662in}}%
\pgfpathlineto{\pgfqpoint{3.947588in}{2.924310in}}%
\pgfpathlineto{\pgfqpoint{3.998361in}{2.919012in}}%
\pgfpathlineto{\pgfqpoint{4.048879in}{2.911779in}}%
\pgfpathlineto{\pgfqpoint{4.099145in}{2.902620in}}%
\pgfpathlineto{\pgfqpoint{4.149160in}{2.891545in}}%
\pgfpathlineto{\pgfqpoint{4.198925in}{2.878564in}}%
\pgfpathlineto{\pgfqpoint{4.248441in}{2.863685in}}%
\pgfpathlineto{\pgfqpoint{4.297709in}{2.846918in}}%
\pgfpathlineto{\pgfqpoint{4.346731in}{2.828274in}}%
\pgfpathlineto{\pgfqpoint{4.395507in}{2.807760in}}%
\pgfpathlineto{\pgfqpoint{4.444040in}{2.785388in}}%
\pgfpathlineto{\pgfqpoint{4.492331in}{2.761165in}}%
\pgfpathlineto{\pgfqpoint{4.540379in}{2.735101in}}%
\pgfpathlineto{\pgfqpoint{4.588188in}{2.707205in}}%
\pgfpathlineto{\pgfqpoint{4.635757in}{2.677487in}}%
\pgfpathlineto{\pgfqpoint{4.683089in}{2.645956in}}%
\pgfpathlineto{\pgfqpoint{4.730184in}{2.612620in}}%
\pgfpathlineto{\pgfqpoint{4.777043in}{2.577489in}}%
\pgfpathlineto{\pgfqpoint{4.823669in}{2.540572in}}%
\pgfpathlineto{\pgfqpoint{4.870061in}{2.501877in}}%
\pgfpathlineto{\pgfqpoint{4.916221in}{2.461413in}}%
\pgfpathlineto{\pgfqpoint{4.962150in}{2.419190in}}%
\pgfpathlineto{\pgfqpoint{5.007850in}{2.375216in}}%
\pgfpathlineto{\pgfqpoint{5.053321in}{2.329500in}}%
\pgfpathlineto{\pgfqpoint{5.098565in}{2.282051in}}%
\pgfpathlineto{\pgfqpoint{5.143582in}{2.232876in}}%
\pgfpathlineto{\pgfqpoint{5.188375in}{2.181986in}}%
\pgfpathlineto{\pgfqpoint{5.232943in}{2.129388in}}%
\pgfpathlineto{\pgfqpoint{5.277289in}{2.075091in}}%
\pgfpathlineto{\pgfqpoint{5.321413in}{2.019104in}}%
\pgfpathlineto{\pgfqpoint{5.365316in}{1.961434in}}%
\pgfpathlineto{\pgfqpoint{5.409000in}{1.902091in}}%
\pgfpathlineto{\pgfqpoint{5.452466in}{1.841083in}}%
\pgfpathlineto{\pgfqpoint{5.495714in}{1.778417in}}%
\pgfpathlineto{\pgfqpoint{5.538746in}{1.714103in}}%
\pgfpathlineto{\pgfqpoint{5.581562in}{1.648149in}}%
\pgfpathlineto{\pgfqpoint{5.624165in}{1.580562in}}%
\pgfpathlineto{\pgfqpoint{5.666554in}{1.511351in}}%
\pgfpathlineto{\pgfqpoint{5.708732in}{1.440524in}}%
\pgfpathlineto{\pgfqpoint{5.750699in}{1.368090in}}%
\pgfpathlineto{\pgfqpoint{5.792456in}{1.294055in}}%
\pgfpathlineto{\pgfqpoint{5.834004in}{1.218429in}}%
\pgfpathlineto{\pgfqpoint{5.875344in}{1.141219in}}%
\pgfpathlineto{\pgfqpoint{5.916478in}{1.062433in}}%
\pgfpathlineto{\pgfqpoint{5.957406in}{0.982079in}}%
\pgfpathlineto{\pgfqpoint{5.998129in}{0.900164in}}%
\pgfpathlineto{\pgfqpoint{6.038649in}{0.816697in}}%
\pgfpathlineto{\pgfqpoint{6.078966in}{0.731686in}}%
\pgfusepath{stroke}%
\end{pgfscope}%
\begin{pgfscope}%
\pgfpathrectangle{\pgfqpoint{1.250000in}{0.400000in}}{\pgfqpoint{7.750000in}{3.200000in}} %
\pgfusepath{clip}%
\pgfsetbuttcap%
\pgfsetroundjoin%
\definecolor{currentfill}{rgb}{0.000000,0.000000,1.000000}%
\pgfsetfillcolor{currentfill}%
\pgfsetlinewidth{0.501875pt}%
\definecolor{currentstroke}{rgb}{0.000000,0.000000,1.000000}%
\pgfsetstrokecolor{currentstroke}%
\pgfsetdash{}{0pt}%
\pgfsys@defobject{currentmarker}{\pgfqpoint{-0.041667in}{-0.041667in}}{\pgfqpoint{0.041667in}{0.041667in}}{%
\pgfpathmoveto{\pgfqpoint{-0.041667in}{0.000000in}}%
\pgfpathlineto{\pgfqpoint{0.041667in}{0.000000in}}%
\pgfpathmoveto{\pgfqpoint{0.000000in}{-0.041667in}}%
\pgfpathlineto{\pgfqpoint{0.000000in}{0.041667in}}%
\pgfusepath{stroke,fill}%
}%
\begin{pgfscope}%
\pgfsys@transformshift{1.250000in}{0.800000in}%
\pgfsys@useobject{currentmarker}{}%
\end{pgfscope}%
\begin{pgfscope}%
\pgfsys@transformshift{1.314260in}{0.897538in}%
\pgfsys@useobject{currentmarker}{}%
\end{pgfscope}%
\begin{pgfscope}%
\pgfsys@transformshift{1.378200in}{0.992626in}%
\pgfsys@useobject{currentmarker}{}%
\end{pgfscope}%
\begin{pgfscope}%
\pgfsys@transformshift{1.441819in}{1.085277in}%
\pgfsys@useobject{currentmarker}{}%
\end{pgfscope}%
\begin{pgfscope}%
\pgfsys@transformshift{1.505120in}{1.175503in}%
\pgfsys@useobject{currentmarker}{}%
\end{pgfscope}%
\begin{pgfscope}%
\pgfsys@transformshift{1.568105in}{1.263315in}%
\pgfsys@useobject{currentmarker}{}%
\end{pgfscope}%
\begin{pgfscope}%
\pgfsys@transformshift{1.630775in}{1.348727in}%
\pgfsys@useobject{currentmarker}{}%
\end{pgfscope}%
\begin{pgfscope}%
\pgfsys@transformshift{1.693132in}{1.431749in}%
\pgfsys@useobject{currentmarker}{}%
\end{pgfscope}%
\begin{pgfscope}%
\pgfsys@transformshift{1.755176in}{1.512394in}%
\pgfsys@useobject{currentmarker}{}%
\end{pgfscope}%
\begin{pgfscope}%
\pgfsys@transformshift{1.816911in}{1.590674in}%
\pgfsys@useobject{currentmarker}{}%
\end{pgfscope}%
\begin{pgfscope}%
\pgfsys@transformshift{1.878337in}{1.666601in}%
\pgfsys@useobject{currentmarker}{}%
\end{pgfscope}%
\begin{pgfscope}%
\pgfsys@transformshift{1.939455in}{1.740186in}%
\pgfsys@useobject{currentmarker}{}%
\end{pgfscope}%
\begin{pgfscope}%
\pgfsys@transformshift{2.000269in}{1.811441in}%
\pgfsys@useobject{currentmarker}{}%
\end{pgfscope}%
\begin{pgfscope}%
\pgfsys@transformshift{2.060778in}{1.880378in}%
\pgfsys@useobject{currentmarker}{}%
\end{pgfscope}%
\begin{pgfscope}%
\pgfsys@transformshift{2.120984in}{1.947008in}%
\pgfsys@useobject{currentmarker}{}%
\end{pgfscope}%
\begin{pgfscope}%
\pgfsys@transformshift{2.180890in}{2.011343in}%
\pgfsys@useobject{currentmarker}{}%
\end{pgfscope}%
\begin{pgfscope}%
\pgfsys@transformshift{2.240496in}{2.073394in}%
\pgfsys@useobject{currentmarker}{}%
\end{pgfscope}%
\begin{pgfscope}%
\pgfsys@transformshift{2.299804in}{2.133173in}%
\pgfsys@useobject{currentmarker}{}%
\end{pgfscope}%
\begin{pgfscope}%
\pgfsys@transformshift{2.358815in}{2.190691in}%
\pgfsys@useobject{currentmarker}{}%
\end{pgfscope}%
\begin{pgfscope}%
\pgfsys@transformshift{2.417531in}{2.245960in}%
\pgfsys@useobject{currentmarker}{}%
\end{pgfscope}%
\begin{pgfscope}%
\pgfsys@transformshift{2.475954in}{2.298990in}%
\pgfsys@useobject{currentmarker}{}%
\end{pgfscope}%
\begin{pgfscope}%
\pgfsys@transformshift{2.534085in}{2.349793in}%
\pgfsys@useobject{currentmarker}{}%
\end{pgfscope}%
\begin{pgfscope}%
\pgfsys@transformshift{2.591925in}{2.398380in}%
\pgfsys@useobject{currentmarker}{}%
\end{pgfscope}%
\begin{pgfscope}%
\pgfsys@transformshift{2.649475in}{2.444762in}%
\pgfsys@useobject{currentmarker}{}%
\end{pgfscope}%
\begin{pgfscope}%
\pgfsys@transformshift{2.706739in}{2.488950in}%
\pgfsys@useobject{currentmarker}{}%
\end{pgfscope}%
\begin{pgfscope}%
\pgfsys@transformshift{2.763715in}{2.530956in}%
\pgfsys@useobject{currentmarker}{}%
\end{pgfscope}%
\begin{pgfscope}%
\pgfsys@transformshift{2.820407in}{2.570789in}%
\pgfsys@useobject{currentmarker}{}%
\end{pgfscope}%
\begin{pgfscope}%
\pgfsys@transformshift{2.876815in}{2.608461in}%
\pgfsys@useobject{currentmarker}{}%
\end{pgfscope}%
\begin{pgfscope}%
\pgfsys@transformshift{2.932942in}{2.643983in}%
\pgfsys@useobject{currentmarker}{}%
\end{pgfscope}%
\begin{pgfscope}%
\pgfsys@transformshift{2.988788in}{2.677365in}%
\pgfsys@useobject{currentmarker}{}%
\end{pgfscope}%
\begin{pgfscope}%
\pgfsys@transformshift{3.044354in}{2.708618in}%
\pgfsys@useobject{currentmarker}{}%
\end{pgfscope}%
\begin{pgfscope}%
\pgfsys@transformshift{3.099643in}{2.737753in}%
\pgfsys@useobject{currentmarker}{}%
\end{pgfscope}%
\begin{pgfscope}%
\pgfsys@transformshift{3.154655in}{2.764780in}%
\pgfsys@useobject{currentmarker}{}%
\end{pgfscope}%
\begin{pgfscope}%
\pgfsys@transformshift{3.209392in}{2.789710in}%
\pgfsys@useobject{currentmarker}{}%
\end{pgfscope}%
\begin{pgfscope}%
\pgfsys@transformshift{3.263855in}{2.812554in}%
\pgfsys@useobject{currentmarker}{}%
\end{pgfscope}%
\begin{pgfscope}%
\pgfsys@transformshift{3.318047in}{2.833321in}%
\pgfsys@useobject{currentmarker}{}%
\end{pgfscope}%
\begin{pgfscope}%
\pgfsys@transformshift{3.371967in}{2.852022in}%
\pgfsys@useobject{currentmarker}{}%
\end{pgfscope}%
\begin{pgfscope}%
\pgfsys@transformshift{3.425617in}{2.868668in}%
\pgfsys@useobject{currentmarker}{}%
\end{pgfscope}%
\begin{pgfscope}%
\pgfsys@transformshift{3.479000in}{2.883269in}%
\pgfsys@useobject{currentmarker}{}%
\end{pgfscope}%
\begin{pgfscope}%
\pgfsys@transformshift{3.532115in}{2.895834in}%
\pgfsys@useobject{currentmarker}{}%
\end{pgfscope}%
\begin{pgfscope}%
\pgfsys@transformshift{3.584965in}{2.906375in}%
\pgfsys@useobject{currentmarker}{}%
\end{pgfscope}%
\begin{pgfscope}%
\pgfsys@transformshift{3.637551in}{2.914901in}%
\pgfsys@useobject{currentmarker}{}%
\end{pgfscope}%
\begin{pgfscope}%
\pgfsys@transformshift{3.689873in}{2.921423in}%
\pgfsys@useobject{currentmarker}{}%
\end{pgfscope}%
\begin{pgfscope}%
\pgfsys@transformshift{3.741934in}{2.925950in}%
\pgfsys@useobject{currentmarker}{}%
\end{pgfscope}%
\begin{pgfscope}%
\pgfsys@transformshift{3.793735in}{2.928492in}%
\pgfsys@useobject{currentmarker}{}%
\end{pgfscope}%
\begin{pgfscope}%
\pgfsys@transformshift{3.845277in}{2.929060in}%
\pgfsys@useobject{currentmarker}{}%
\end{pgfscope}%
\begin{pgfscope}%
\pgfsys@transformshift{3.896561in}{2.927662in}%
\pgfsys@useobject{currentmarker}{}%
\end{pgfscope}%
\begin{pgfscope}%
\pgfsys@transformshift{3.947588in}{2.924310in}%
\pgfsys@useobject{currentmarker}{}%
\end{pgfscope}%
\begin{pgfscope}%
\pgfsys@transformshift{3.998361in}{2.919012in}%
\pgfsys@useobject{currentmarker}{}%
\end{pgfscope}%
\begin{pgfscope}%
\pgfsys@transformshift{4.048879in}{2.911779in}%
\pgfsys@useobject{currentmarker}{}%
\end{pgfscope}%
\begin{pgfscope}%
\pgfsys@transformshift{4.099145in}{2.902620in}%
\pgfsys@useobject{currentmarker}{}%
\end{pgfscope}%
\begin{pgfscope}%
\pgfsys@transformshift{4.149160in}{2.891545in}%
\pgfsys@useobject{currentmarker}{}%
\end{pgfscope}%
\begin{pgfscope}%
\pgfsys@transformshift{4.198925in}{2.878564in}%
\pgfsys@useobject{currentmarker}{}%
\end{pgfscope}%
\begin{pgfscope}%
\pgfsys@transformshift{4.248441in}{2.863685in}%
\pgfsys@useobject{currentmarker}{}%
\end{pgfscope}%
\begin{pgfscope}%
\pgfsys@transformshift{4.297709in}{2.846918in}%
\pgfsys@useobject{currentmarker}{}%
\end{pgfscope}%
\begin{pgfscope}%
\pgfsys@transformshift{4.346731in}{2.828274in}%
\pgfsys@useobject{currentmarker}{}%
\end{pgfscope}%
\begin{pgfscope}%
\pgfsys@transformshift{4.395507in}{2.807760in}%
\pgfsys@useobject{currentmarker}{}%
\end{pgfscope}%
\begin{pgfscope}%
\pgfsys@transformshift{4.444040in}{2.785388in}%
\pgfsys@useobject{currentmarker}{}%
\end{pgfscope}%
\begin{pgfscope}%
\pgfsys@transformshift{4.492331in}{2.761165in}%
\pgfsys@useobject{currentmarker}{}%
\end{pgfscope}%
\begin{pgfscope}%
\pgfsys@transformshift{4.540379in}{2.735101in}%
\pgfsys@useobject{currentmarker}{}%
\end{pgfscope}%
\begin{pgfscope}%
\pgfsys@transformshift{4.588188in}{2.707205in}%
\pgfsys@useobject{currentmarker}{}%
\end{pgfscope}%
\begin{pgfscope}%
\pgfsys@transformshift{4.635757in}{2.677487in}%
\pgfsys@useobject{currentmarker}{}%
\end{pgfscope}%
\begin{pgfscope}%
\pgfsys@transformshift{4.683089in}{2.645956in}%
\pgfsys@useobject{currentmarker}{}%
\end{pgfscope}%
\begin{pgfscope}%
\pgfsys@transformshift{4.730184in}{2.612620in}%
\pgfsys@useobject{currentmarker}{}%
\end{pgfscope}%
\begin{pgfscope}%
\pgfsys@transformshift{4.777043in}{2.577489in}%
\pgfsys@useobject{currentmarker}{}%
\end{pgfscope}%
\begin{pgfscope}%
\pgfsys@transformshift{4.823669in}{2.540572in}%
\pgfsys@useobject{currentmarker}{}%
\end{pgfscope}%
\begin{pgfscope}%
\pgfsys@transformshift{4.870061in}{2.501877in}%
\pgfsys@useobject{currentmarker}{}%
\end{pgfscope}%
\begin{pgfscope}%
\pgfsys@transformshift{4.916221in}{2.461413in}%
\pgfsys@useobject{currentmarker}{}%
\end{pgfscope}%
\begin{pgfscope}%
\pgfsys@transformshift{4.962150in}{2.419190in}%
\pgfsys@useobject{currentmarker}{}%
\end{pgfscope}%
\begin{pgfscope}%
\pgfsys@transformshift{5.007850in}{2.375216in}%
\pgfsys@useobject{currentmarker}{}%
\end{pgfscope}%
\begin{pgfscope}%
\pgfsys@transformshift{5.053321in}{2.329500in}%
\pgfsys@useobject{currentmarker}{}%
\end{pgfscope}%
\begin{pgfscope}%
\pgfsys@transformshift{5.098565in}{2.282051in}%
\pgfsys@useobject{currentmarker}{}%
\end{pgfscope}%
\begin{pgfscope}%
\pgfsys@transformshift{5.143582in}{2.232876in}%
\pgfsys@useobject{currentmarker}{}%
\end{pgfscope}%
\begin{pgfscope}%
\pgfsys@transformshift{5.188375in}{2.181986in}%
\pgfsys@useobject{currentmarker}{}%
\end{pgfscope}%
\begin{pgfscope}%
\pgfsys@transformshift{5.232943in}{2.129388in}%
\pgfsys@useobject{currentmarker}{}%
\end{pgfscope}%
\begin{pgfscope}%
\pgfsys@transformshift{5.277289in}{2.075091in}%
\pgfsys@useobject{currentmarker}{}%
\end{pgfscope}%
\begin{pgfscope}%
\pgfsys@transformshift{5.321413in}{2.019104in}%
\pgfsys@useobject{currentmarker}{}%
\end{pgfscope}%
\begin{pgfscope}%
\pgfsys@transformshift{5.365316in}{1.961434in}%
\pgfsys@useobject{currentmarker}{}%
\end{pgfscope}%
\begin{pgfscope}%
\pgfsys@transformshift{5.409000in}{1.902091in}%
\pgfsys@useobject{currentmarker}{}%
\end{pgfscope}%
\begin{pgfscope}%
\pgfsys@transformshift{5.452466in}{1.841083in}%
\pgfsys@useobject{currentmarker}{}%
\end{pgfscope}%
\begin{pgfscope}%
\pgfsys@transformshift{5.495714in}{1.778417in}%
\pgfsys@useobject{currentmarker}{}%
\end{pgfscope}%
\begin{pgfscope}%
\pgfsys@transformshift{5.538746in}{1.714103in}%
\pgfsys@useobject{currentmarker}{}%
\end{pgfscope}%
\begin{pgfscope}%
\pgfsys@transformshift{5.581562in}{1.648149in}%
\pgfsys@useobject{currentmarker}{}%
\end{pgfscope}%
\begin{pgfscope}%
\pgfsys@transformshift{5.624165in}{1.580562in}%
\pgfsys@useobject{currentmarker}{}%
\end{pgfscope}%
\begin{pgfscope}%
\pgfsys@transformshift{5.666554in}{1.511351in}%
\pgfsys@useobject{currentmarker}{}%
\end{pgfscope}%
\begin{pgfscope}%
\pgfsys@transformshift{5.708732in}{1.440524in}%
\pgfsys@useobject{currentmarker}{}%
\end{pgfscope}%
\begin{pgfscope}%
\pgfsys@transformshift{5.750699in}{1.368090in}%
\pgfsys@useobject{currentmarker}{}%
\end{pgfscope}%
\begin{pgfscope}%
\pgfsys@transformshift{5.792456in}{1.294055in}%
\pgfsys@useobject{currentmarker}{}%
\end{pgfscope}%
\begin{pgfscope}%
\pgfsys@transformshift{5.834004in}{1.218429in}%
\pgfsys@useobject{currentmarker}{}%
\end{pgfscope}%
\begin{pgfscope}%
\pgfsys@transformshift{5.875344in}{1.141219in}%
\pgfsys@useobject{currentmarker}{}%
\end{pgfscope}%
\begin{pgfscope}%
\pgfsys@transformshift{5.916478in}{1.062433in}%
\pgfsys@useobject{currentmarker}{}%
\end{pgfscope}%
\begin{pgfscope}%
\pgfsys@transformshift{5.957406in}{0.982079in}%
\pgfsys@useobject{currentmarker}{}%
\end{pgfscope}%
\begin{pgfscope}%
\pgfsys@transformshift{5.998129in}{0.900164in}%
\pgfsys@useobject{currentmarker}{}%
\end{pgfscope}%
\begin{pgfscope}%
\pgfsys@transformshift{6.038649in}{0.816697in}%
\pgfsys@useobject{currentmarker}{}%
\end{pgfscope}%
\begin{pgfscope}%
\pgfsys@transformshift{6.078966in}{0.731686in}%
\pgfsys@useobject{currentmarker}{}%
\end{pgfscope}%
\end{pgfscope}%
\begin{pgfscope}%
\pgfpathrectangle{\pgfqpoint{1.250000in}{0.400000in}}{\pgfqpoint{7.750000in}{3.200000in}} %
\pgfusepath{clip}%
\pgfsetrectcap%
\pgfsetroundjoin%
\pgfsetlinewidth{1.003750pt}%
\definecolor{currentstroke}{rgb}{0.000000,0.500000,0.000000}%
\pgfsetstrokecolor{currentstroke}%
\pgfsetdash{}{0pt}%
\pgfpathmoveto{\pgfqpoint{1.250000in}{0.800000in}}%
\pgfpathlineto{\pgfqpoint{1.313938in}{0.897538in}}%
\pgfpathlineto{\pgfqpoint{1.377232in}{0.992626in}}%
\pgfpathlineto{\pgfqpoint{1.439888in}{1.085277in}}%
\pgfpathlineto{\pgfqpoint{1.501907in}{1.175503in}}%
\pgfpathlineto{\pgfqpoint{1.563294in}{1.263315in}}%
\pgfpathlineto{\pgfqpoint{1.624050in}{1.348727in}}%
\pgfpathlineto{\pgfqpoint{1.684180in}{1.431749in}}%
\pgfpathlineto{\pgfqpoint{1.743686in}{1.512394in}}%
\pgfpathlineto{\pgfqpoint{1.802572in}{1.590674in}}%
\pgfpathlineto{\pgfqpoint{1.860840in}{1.666601in}}%
\pgfpathlineto{\pgfqpoint{1.918494in}{1.740186in}}%
\pgfpathlineto{\pgfqpoint{1.975537in}{1.811441in}}%
\pgfpathlineto{\pgfqpoint{2.031972in}{1.880378in}}%
\pgfpathlineto{\pgfqpoint{2.087802in}{1.947008in}}%
\pgfpathlineto{\pgfqpoint{2.143029in}{2.011343in}}%
\pgfpathlineto{\pgfqpoint{2.197658in}{2.073394in}}%
\pgfpathlineto{\pgfqpoint{2.251690in}{2.133173in}}%
\pgfpathlineto{\pgfqpoint{2.305130in}{2.190691in}}%
\pgfpathlineto{\pgfqpoint{2.357979in}{2.245960in}}%
\pgfpathlineto{\pgfqpoint{2.410241in}{2.298990in}}%
\pgfpathlineto{\pgfqpoint{2.461919in}{2.349793in}}%
\pgfpathlineto{\pgfqpoint{2.513016in}{2.398380in}}%
\pgfpathlineto{\pgfqpoint{2.563534in}{2.444762in}}%
\pgfpathlineto{\pgfqpoint{2.613477in}{2.488950in}}%
\pgfpathlineto{\pgfqpoint{2.662847in}{2.530956in}}%
\pgfpathlineto{\pgfqpoint{2.711648in}{2.570789in}}%
\pgfpathlineto{\pgfqpoint{2.759881in}{2.608461in}}%
\pgfpathlineto{\pgfqpoint{2.807550in}{2.643983in}}%
\pgfpathlineto{\pgfqpoint{2.854658in}{2.677365in}}%
\pgfpathlineto{\pgfqpoint{2.901208in}{2.708618in}}%
\pgfpathlineto{\pgfqpoint{2.947202in}{2.737753in}}%
\pgfpathlineto{\pgfqpoint{2.992643in}{2.764780in}}%
\pgfpathlineto{\pgfqpoint{3.037534in}{2.789710in}}%
\pgfpathlineto{\pgfqpoint{3.081878in}{2.812554in}}%
\pgfpathlineto{\pgfqpoint{3.125677in}{2.833321in}}%
\pgfpathlineto{\pgfqpoint{3.168934in}{2.852022in}}%
\pgfpathlineto{\pgfqpoint{3.211651in}{2.868668in}}%
\pgfpathlineto{\pgfqpoint{3.253833in}{2.883269in}}%
\pgfpathlineto{\pgfqpoint{3.295480in}{2.895834in}}%
\pgfpathlineto{\pgfqpoint{3.336597in}{2.906375in}}%
\pgfpathlineto{\pgfqpoint{3.377184in}{2.914901in}}%
\pgfpathlineto{\pgfqpoint{3.417246in}{2.921423in}}%
\pgfpathlineto{\pgfqpoint{3.456785in}{2.925950in}}%
\pgfpathlineto{\pgfqpoint{3.495803in}{2.928492in}}%
\pgfpathlineto{\pgfqpoint{3.534303in}{2.929060in}}%
\pgfpathlineto{\pgfqpoint{3.572288in}{2.927662in}}%
\pgfpathlineto{\pgfqpoint{3.609760in}{2.924310in}}%
\pgfpathlineto{\pgfqpoint{3.646722in}{2.919012in}}%
\pgfpathlineto{\pgfqpoint{3.683176in}{2.911779in}}%
\pgfpathlineto{\pgfqpoint{3.719124in}{2.902620in}}%
\pgfpathlineto{\pgfqpoint{3.754570in}{2.891545in}}%
\pgfpathlineto{\pgfqpoint{3.789516in}{2.878564in}}%
\pgfpathlineto{\pgfqpoint{3.823965in}{2.863685in}}%
\pgfpathlineto{\pgfqpoint{3.857918in}{2.846918in}}%
\pgfpathlineto{\pgfqpoint{3.891378in}{2.828274in}}%
\pgfpathlineto{\pgfqpoint{3.924348in}{2.807760in}}%
\pgfpathlineto{\pgfqpoint{3.956831in}{2.785388in}}%
\pgfpathlineto{\pgfqpoint{3.988828in}{2.761165in}}%
\pgfpathlineto{\pgfqpoint{4.020342in}{2.735101in}}%
\pgfpathlineto{\pgfqpoint{4.051376in}{2.707205in}}%
\pgfpathlineto{\pgfqpoint{4.081931in}{2.677487in}}%
\pgfpathlineto{\pgfqpoint{4.112011in}{2.645956in}}%
\pgfpathlineto{\pgfqpoint{4.141618in}{2.612620in}}%
\pgfpathlineto{\pgfqpoint{4.170753in}{2.577489in}}%
\pgfpathlineto{\pgfqpoint{4.199421in}{2.540572in}}%
\pgfpathlineto{\pgfqpoint{4.227621in}{2.501877in}}%
\pgfpathlineto{\pgfqpoint{4.255358in}{2.461413in}}%
\pgfpathlineto{\pgfqpoint{4.282634in}{2.419190in}}%
\pgfpathlineto{\pgfqpoint{4.309450in}{2.375216in}}%
\pgfpathlineto{\pgfqpoint{4.335809in}{2.329500in}}%
\pgfpathlineto{\pgfqpoint{4.361713in}{2.282051in}}%
\pgfpathlineto{\pgfqpoint{4.387165in}{2.232876in}}%
\pgfpathlineto{\pgfqpoint{4.412166in}{2.181986in}}%
\pgfpathlineto{\pgfqpoint{4.436720in}{2.129388in}}%
\pgfpathlineto{\pgfqpoint{4.460828in}{2.075091in}}%
\pgfpathlineto{\pgfqpoint{4.484493in}{2.019104in}}%
\pgfpathlineto{\pgfqpoint{4.507716in}{1.961434in}}%
\pgfpathlineto{\pgfqpoint{4.530500in}{1.902091in}}%
\pgfpathlineto{\pgfqpoint{4.552848in}{1.841083in}}%
\pgfpathlineto{\pgfqpoint{4.574761in}{1.778417in}}%
\pgfpathlineto{\pgfqpoint{4.596241in}{1.714103in}}%
\pgfpathlineto{\pgfqpoint{4.617291in}{1.648149in}}%
\pgfpathlineto{\pgfqpoint{4.637913in}{1.580562in}}%
\pgfpathlineto{\pgfqpoint{4.658109in}{1.511351in}}%
\pgfpathlineto{\pgfqpoint{4.677881in}{1.440524in}}%
\pgfpathlineto{\pgfqpoint{4.697231in}{1.368090in}}%
\pgfpathlineto{\pgfqpoint{4.716162in}{1.294055in}}%
\pgfpathlineto{\pgfqpoint{4.734675in}{1.218429in}}%
\pgfpathlineto{\pgfqpoint{4.752772in}{1.141219in}}%
\pgfpathlineto{\pgfqpoint{4.770456in}{1.062433in}}%
\pgfpathlineto{\pgfqpoint{4.787729in}{0.982079in}}%
\pgfpathlineto{\pgfqpoint{4.804592in}{0.900164in}}%
\pgfpathlineto{\pgfqpoint{4.821048in}{0.816697in}}%
\pgfpathlineto{\pgfqpoint{4.837099in}{0.731686in}}%
\pgfusepath{stroke}%
\end{pgfscope}%
\begin{pgfscope}%
\pgfpathrectangle{\pgfqpoint{1.250000in}{0.400000in}}{\pgfqpoint{7.750000in}{3.200000in}} %
\pgfusepath{clip}%
\pgfsetbuttcap%
\pgfsetroundjoin%
\definecolor{currentfill}{rgb}{0.000000,0.500000,0.000000}%
\pgfsetfillcolor{currentfill}%
\pgfsetlinewidth{0.501875pt}%
\definecolor{currentstroke}{rgb}{0.000000,0.500000,0.000000}%
\pgfsetstrokecolor{currentstroke}%
\pgfsetdash{}{0pt}%
\pgfsys@defobject{currentmarker}{\pgfqpoint{-0.041667in}{-0.041667in}}{\pgfqpoint{0.041667in}{0.041667in}}{%
\pgfpathmoveto{\pgfqpoint{-0.041667in}{0.000000in}}%
\pgfpathlineto{\pgfqpoint{0.041667in}{0.000000in}}%
\pgfpathmoveto{\pgfqpoint{0.000000in}{-0.041667in}}%
\pgfpathlineto{\pgfqpoint{0.000000in}{0.041667in}}%
\pgfusepath{stroke,fill}%
}%
\begin{pgfscope}%
\pgfsys@transformshift{1.250000in}{0.800000in}%
\pgfsys@useobject{currentmarker}{}%
\end{pgfscope}%
\begin{pgfscope}%
\pgfsys@transformshift{1.313938in}{0.897538in}%
\pgfsys@useobject{currentmarker}{}%
\end{pgfscope}%
\begin{pgfscope}%
\pgfsys@transformshift{1.377232in}{0.992626in}%
\pgfsys@useobject{currentmarker}{}%
\end{pgfscope}%
\begin{pgfscope}%
\pgfsys@transformshift{1.439888in}{1.085277in}%
\pgfsys@useobject{currentmarker}{}%
\end{pgfscope}%
\begin{pgfscope}%
\pgfsys@transformshift{1.501907in}{1.175503in}%
\pgfsys@useobject{currentmarker}{}%
\end{pgfscope}%
\begin{pgfscope}%
\pgfsys@transformshift{1.563294in}{1.263315in}%
\pgfsys@useobject{currentmarker}{}%
\end{pgfscope}%
\begin{pgfscope}%
\pgfsys@transformshift{1.624050in}{1.348727in}%
\pgfsys@useobject{currentmarker}{}%
\end{pgfscope}%
\begin{pgfscope}%
\pgfsys@transformshift{1.684180in}{1.431749in}%
\pgfsys@useobject{currentmarker}{}%
\end{pgfscope}%
\begin{pgfscope}%
\pgfsys@transformshift{1.743686in}{1.512394in}%
\pgfsys@useobject{currentmarker}{}%
\end{pgfscope}%
\begin{pgfscope}%
\pgfsys@transformshift{1.802572in}{1.590674in}%
\pgfsys@useobject{currentmarker}{}%
\end{pgfscope}%
\begin{pgfscope}%
\pgfsys@transformshift{1.860840in}{1.666601in}%
\pgfsys@useobject{currentmarker}{}%
\end{pgfscope}%
\begin{pgfscope}%
\pgfsys@transformshift{1.918494in}{1.740186in}%
\pgfsys@useobject{currentmarker}{}%
\end{pgfscope}%
\begin{pgfscope}%
\pgfsys@transformshift{1.975537in}{1.811441in}%
\pgfsys@useobject{currentmarker}{}%
\end{pgfscope}%
\begin{pgfscope}%
\pgfsys@transformshift{2.031972in}{1.880378in}%
\pgfsys@useobject{currentmarker}{}%
\end{pgfscope}%
\begin{pgfscope}%
\pgfsys@transformshift{2.087802in}{1.947008in}%
\pgfsys@useobject{currentmarker}{}%
\end{pgfscope}%
\begin{pgfscope}%
\pgfsys@transformshift{2.143029in}{2.011343in}%
\pgfsys@useobject{currentmarker}{}%
\end{pgfscope}%
\begin{pgfscope}%
\pgfsys@transformshift{2.197658in}{2.073394in}%
\pgfsys@useobject{currentmarker}{}%
\end{pgfscope}%
\begin{pgfscope}%
\pgfsys@transformshift{2.251690in}{2.133173in}%
\pgfsys@useobject{currentmarker}{}%
\end{pgfscope}%
\begin{pgfscope}%
\pgfsys@transformshift{2.305130in}{2.190691in}%
\pgfsys@useobject{currentmarker}{}%
\end{pgfscope}%
\begin{pgfscope}%
\pgfsys@transformshift{2.357979in}{2.245960in}%
\pgfsys@useobject{currentmarker}{}%
\end{pgfscope}%
\begin{pgfscope}%
\pgfsys@transformshift{2.410241in}{2.298990in}%
\pgfsys@useobject{currentmarker}{}%
\end{pgfscope}%
\begin{pgfscope}%
\pgfsys@transformshift{2.461919in}{2.349793in}%
\pgfsys@useobject{currentmarker}{}%
\end{pgfscope}%
\begin{pgfscope}%
\pgfsys@transformshift{2.513016in}{2.398380in}%
\pgfsys@useobject{currentmarker}{}%
\end{pgfscope}%
\begin{pgfscope}%
\pgfsys@transformshift{2.563534in}{2.444762in}%
\pgfsys@useobject{currentmarker}{}%
\end{pgfscope}%
\begin{pgfscope}%
\pgfsys@transformshift{2.613477in}{2.488950in}%
\pgfsys@useobject{currentmarker}{}%
\end{pgfscope}%
\begin{pgfscope}%
\pgfsys@transformshift{2.662847in}{2.530956in}%
\pgfsys@useobject{currentmarker}{}%
\end{pgfscope}%
\begin{pgfscope}%
\pgfsys@transformshift{2.711648in}{2.570789in}%
\pgfsys@useobject{currentmarker}{}%
\end{pgfscope}%
\begin{pgfscope}%
\pgfsys@transformshift{2.759881in}{2.608461in}%
\pgfsys@useobject{currentmarker}{}%
\end{pgfscope}%
\begin{pgfscope}%
\pgfsys@transformshift{2.807550in}{2.643983in}%
\pgfsys@useobject{currentmarker}{}%
\end{pgfscope}%
\begin{pgfscope}%
\pgfsys@transformshift{2.854658in}{2.677365in}%
\pgfsys@useobject{currentmarker}{}%
\end{pgfscope}%
\begin{pgfscope}%
\pgfsys@transformshift{2.901208in}{2.708618in}%
\pgfsys@useobject{currentmarker}{}%
\end{pgfscope}%
\begin{pgfscope}%
\pgfsys@transformshift{2.947202in}{2.737753in}%
\pgfsys@useobject{currentmarker}{}%
\end{pgfscope}%
\begin{pgfscope}%
\pgfsys@transformshift{2.992643in}{2.764780in}%
\pgfsys@useobject{currentmarker}{}%
\end{pgfscope}%
\begin{pgfscope}%
\pgfsys@transformshift{3.037534in}{2.789710in}%
\pgfsys@useobject{currentmarker}{}%
\end{pgfscope}%
\begin{pgfscope}%
\pgfsys@transformshift{3.081878in}{2.812554in}%
\pgfsys@useobject{currentmarker}{}%
\end{pgfscope}%
\begin{pgfscope}%
\pgfsys@transformshift{3.125677in}{2.833321in}%
\pgfsys@useobject{currentmarker}{}%
\end{pgfscope}%
\begin{pgfscope}%
\pgfsys@transformshift{3.168934in}{2.852022in}%
\pgfsys@useobject{currentmarker}{}%
\end{pgfscope}%
\begin{pgfscope}%
\pgfsys@transformshift{3.211651in}{2.868668in}%
\pgfsys@useobject{currentmarker}{}%
\end{pgfscope}%
\begin{pgfscope}%
\pgfsys@transformshift{3.253833in}{2.883269in}%
\pgfsys@useobject{currentmarker}{}%
\end{pgfscope}%
\begin{pgfscope}%
\pgfsys@transformshift{3.295480in}{2.895834in}%
\pgfsys@useobject{currentmarker}{}%
\end{pgfscope}%
\begin{pgfscope}%
\pgfsys@transformshift{3.336597in}{2.906375in}%
\pgfsys@useobject{currentmarker}{}%
\end{pgfscope}%
\begin{pgfscope}%
\pgfsys@transformshift{3.377184in}{2.914901in}%
\pgfsys@useobject{currentmarker}{}%
\end{pgfscope}%
\begin{pgfscope}%
\pgfsys@transformshift{3.417246in}{2.921423in}%
\pgfsys@useobject{currentmarker}{}%
\end{pgfscope}%
\begin{pgfscope}%
\pgfsys@transformshift{3.456785in}{2.925950in}%
\pgfsys@useobject{currentmarker}{}%
\end{pgfscope}%
\begin{pgfscope}%
\pgfsys@transformshift{3.495803in}{2.928492in}%
\pgfsys@useobject{currentmarker}{}%
\end{pgfscope}%
\begin{pgfscope}%
\pgfsys@transformshift{3.534303in}{2.929060in}%
\pgfsys@useobject{currentmarker}{}%
\end{pgfscope}%
\begin{pgfscope}%
\pgfsys@transformshift{3.572288in}{2.927662in}%
\pgfsys@useobject{currentmarker}{}%
\end{pgfscope}%
\begin{pgfscope}%
\pgfsys@transformshift{3.609760in}{2.924310in}%
\pgfsys@useobject{currentmarker}{}%
\end{pgfscope}%
\begin{pgfscope}%
\pgfsys@transformshift{3.646722in}{2.919012in}%
\pgfsys@useobject{currentmarker}{}%
\end{pgfscope}%
\begin{pgfscope}%
\pgfsys@transformshift{3.683176in}{2.911779in}%
\pgfsys@useobject{currentmarker}{}%
\end{pgfscope}%
\begin{pgfscope}%
\pgfsys@transformshift{3.719124in}{2.902620in}%
\pgfsys@useobject{currentmarker}{}%
\end{pgfscope}%
\begin{pgfscope}%
\pgfsys@transformshift{3.754570in}{2.891545in}%
\pgfsys@useobject{currentmarker}{}%
\end{pgfscope}%
\begin{pgfscope}%
\pgfsys@transformshift{3.789516in}{2.878564in}%
\pgfsys@useobject{currentmarker}{}%
\end{pgfscope}%
\begin{pgfscope}%
\pgfsys@transformshift{3.823965in}{2.863685in}%
\pgfsys@useobject{currentmarker}{}%
\end{pgfscope}%
\begin{pgfscope}%
\pgfsys@transformshift{3.857918in}{2.846918in}%
\pgfsys@useobject{currentmarker}{}%
\end{pgfscope}%
\begin{pgfscope}%
\pgfsys@transformshift{3.891378in}{2.828274in}%
\pgfsys@useobject{currentmarker}{}%
\end{pgfscope}%
\begin{pgfscope}%
\pgfsys@transformshift{3.924348in}{2.807760in}%
\pgfsys@useobject{currentmarker}{}%
\end{pgfscope}%
\begin{pgfscope}%
\pgfsys@transformshift{3.956831in}{2.785388in}%
\pgfsys@useobject{currentmarker}{}%
\end{pgfscope}%
\begin{pgfscope}%
\pgfsys@transformshift{3.988828in}{2.761165in}%
\pgfsys@useobject{currentmarker}{}%
\end{pgfscope}%
\begin{pgfscope}%
\pgfsys@transformshift{4.020342in}{2.735101in}%
\pgfsys@useobject{currentmarker}{}%
\end{pgfscope}%
\begin{pgfscope}%
\pgfsys@transformshift{4.051376in}{2.707205in}%
\pgfsys@useobject{currentmarker}{}%
\end{pgfscope}%
\begin{pgfscope}%
\pgfsys@transformshift{4.081931in}{2.677487in}%
\pgfsys@useobject{currentmarker}{}%
\end{pgfscope}%
\begin{pgfscope}%
\pgfsys@transformshift{4.112011in}{2.645956in}%
\pgfsys@useobject{currentmarker}{}%
\end{pgfscope}%
\begin{pgfscope}%
\pgfsys@transformshift{4.141618in}{2.612620in}%
\pgfsys@useobject{currentmarker}{}%
\end{pgfscope}%
\begin{pgfscope}%
\pgfsys@transformshift{4.170753in}{2.577489in}%
\pgfsys@useobject{currentmarker}{}%
\end{pgfscope}%
\begin{pgfscope}%
\pgfsys@transformshift{4.199421in}{2.540572in}%
\pgfsys@useobject{currentmarker}{}%
\end{pgfscope}%
\begin{pgfscope}%
\pgfsys@transformshift{4.227621in}{2.501877in}%
\pgfsys@useobject{currentmarker}{}%
\end{pgfscope}%
\begin{pgfscope}%
\pgfsys@transformshift{4.255358in}{2.461413in}%
\pgfsys@useobject{currentmarker}{}%
\end{pgfscope}%
\begin{pgfscope}%
\pgfsys@transformshift{4.282634in}{2.419190in}%
\pgfsys@useobject{currentmarker}{}%
\end{pgfscope}%
\begin{pgfscope}%
\pgfsys@transformshift{4.309450in}{2.375216in}%
\pgfsys@useobject{currentmarker}{}%
\end{pgfscope}%
\begin{pgfscope}%
\pgfsys@transformshift{4.335809in}{2.329500in}%
\pgfsys@useobject{currentmarker}{}%
\end{pgfscope}%
\begin{pgfscope}%
\pgfsys@transformshift{4.361713in}{2.282051in}%
\pgfsys@useobject{currentmarker}{}%
\end{pgfscope}%
\begin{pgfscope}%
\pgfsys@transformshift{4.387165in}{2.232876in}%
\pgfsys@useobject{currentmarker}{}%
\end{pgfscope}%
\begin{pgfscope}%
\pgfsys@transformshift{4.412166in}{2.181986in}%
\pgfsys@useobject{currentmarker}{}%
\end{pgfscope}%
\begin{pgfscope}%
\pgfsys@transformshift{4.436720in}{2.129388in}%
\pgfsys@useobject{currentmarker}{}%
\end{pgfscope}%
\begin{pgfscope}%
\pgfsys@transformshift{4.460828in}{2.075091in}%
\pgfsys@useobject{currentmarker}{}%
\end{pgfscope}%
\begin{pgfscope}%
\pgfsys@transformshift{4.484493in}{2.019104in}%
\pgfsys@useobject{currentmarker}{}%
\end{pgfscope}%
\begin{pgfscope}%
\pgfsys@transformshift{4.507716in}{1.961434in}%
\pgfsys@useobject{currentmarker}{}%
\end{pgfscope}%
\begin{pgfscope}%
\pgfsys@transformshift{4.530500in}{1.902091in}%
\pgfsys@useobject{currentmarker}{}%
\end{pgfscope}%
\begin{pgfscope}%
\pgfsys@transformshift{4.552848in}{1.841083in}%
\pgfsys@useobject{currentmarker}{}%
\end{pgfscope}%
\begin{pgfscope}%
\pgfsys@transformshift{4.574761in}{1.778417in}%
\pgfsys@useobject{currentmarker}{}%
\end{pgfscope}%
\begin{pgfscope}%
\pgfsys@transformshift{4.596241in}{1.714103in}%
\pgfsys@useobject{currentmarker}{}%
\end{pgfscope}%
\begin{pgfscope}%
\pgfsys@transformshift{4.617291in}{1.648149in}%
\pgfsys@useobject{currentmarker}{}%
\end{pgfscope}%
\begin{pgfscope}%
\pgfsys@transformshift{4.637913in}{1.580562in}%
\pgfsys@useobject{currentmarker}{}%
\end{pgfscope}%
\begin{pgfscope}%
\pgfsys@transformshift{4.658109in}{1.511351in}%
\pgfsys@useobject{currentmarker}{}%
\end{pgfscope}%
\begin{pgfscope}%
\pgfsys@transformshift{4.677881in}{1.440524in}%
\pgfsys@useobject{currentmarker}{}%
\end{pgfscope}%
\begin{pgfscope}%
\pgfsys@transformshift{4.697231in}{1.368090in}%
\pgfsys@useobject{currentmarker}{}%
\end{pgfscope}%
\begin{pgfscope}%
\pgfsys@transformshift{4.716162in}{1.294055in}%
\pgfsys@useobject{currentmarker}{}%
\end{pgfscope}%
\begin{pgfscope}%
\pgfsys@transformshift{4.734675in}{1.218429in}%
\pgfsys@useobject{currentmarker}{}%
\end{pgfscope}%
\begin{pgfscope}%
\pgfsys@transformshift{4.752772in}{1.141219in}%
\pgfsys@useobject{currentmarker}{}%
\end{pgfscope}%
\begin{pgfscope}%
\pgfsys@transformshift{4.770456in}{1.062433in}%
\pgfsys@useobject{currentmarker}{}%
\end{pgfscope}%
\begin{pgfscope}%
\pgfsys@transformshift{4.787729in}{0.982079in}%
\pgfsys@useobject{currentmarker}{}%
\end{pgfscope}%
\begin{pgfscope}%
\pgfsys@transformshift{4.804592in}{0.900164in}%
\pgfsys@useobject{currentmarker}{}%
\end{pgfscope}%
\begin{pgfscope}%
\pgfsys@transformshift{4.821048in}{0.816697in}%
\pgfsys@useobject{currentmarker}{}%
\end{pgfscope}%
\begin{pgfscope}%
\pgfsys@transformshift{4.837099in}{0.731686in}%
\pgfsys@useobject{currentmarker}{}%
\end{pgfscope}%
\end{pgfscope}%
\begin{pgfscope}%
\pgfpathrectangle{\pgfqpoint{1.250000in}{0.400000in}}{\pgfqpoint{7.750000in}{3.200000in}} %
\pgfusepath{clip}%
\pgfsetbuttcap%
\pgfsetroundjoin%
\pgfsetlinewidth{0.501875pt}%
\definecolor{currentstroke}{rgb}{0.000000,0.000000,0.000000}%
\pgfsetstrokecolor{currentstroke}%
\pgfsetdash{{1.000000pt}{3.000000pt}}{0.000000pt}%
\pgfpathmoveto{\pgfqpoint{1.250000in}{0.400000in}}%
\pgfpathlineto{\pgfqpoint{1.250000in}{3.600000in}}%
\pgfusepath{stroke}%
\end{pgfscope}%
\begin{pgfscope}%
\pgfsetbuttcap%
\pgfsetroundjoin%
\definecolor{currentfill}{rgb}{0.000000,0.000000,0.000000}%
\pgfsetfillcolor{currentfill}%
\pgfsetlinewidth{0.501875pt}%
\definecolor{currentstroke}{rgb}{0.000000,0.000000,0.000000}%
\pgfsetstrokecolor{currentstroke}%
\pgfsetdash{}{0pt}%
\pgfsys@defobject{currentmarker}{\pgfqpoint{0.000000in}{0.000000in}}{\pgfqpoint{0.000000in}{0.055556in}}{%
\pgfpathmoveto{\pgfqpoint{0.000000in}{0.000000in}}%
\pgfpathlineto{\pgfqpoint{0.000000in}{0.055556in}}%
\pgfusepath{stroke,fill}%
}%
\begin{pgfscope}%
\pgfsys@transformshift{1.250000in}{0.400000in}%
\pgfsys@useobject{currentmarker}{}%
\end{pgfscope}%
\end{pgfscope}%
\begin{pgfscope}%
\pgfsetbuttcap%
\pgfsetroundjoin%
\definecolor{currentfill}{rgb}{0.000000,0.000000,0.000000}%
\pgfsetfillcolor{currentfill}%
\pgfsetlinewidth{0.501875pt}%
\definecolor{currentstroke}{rgb}{0.000000,0.000000,0.000000}%
\pgfsetstrokecolor{currentstroke}%
\pgfsetdash{}{0pt}%
\pgfsys@defobject{currentmarker}{\pgfqpoint{0.000000in}{-0.055556in}}{\pgfqpoint{0.000000in}{0.000000in}}{%
\pgfpathmoveto{\pgfqpoint{0.000000in}{0.000000in}}%
\pgfpathlineto{\pgfqpoint{0.000000in}{-0.055556in}}%
\pgfusepath{stroke,fill}%
}%
\begin{pgfscope}%
\pgfsys@transformshift{1.250000in}{3.600000in}%
\pgfsys@useobject{currentmarker}{}%
\end{pgfscope}%
\end{pgfscope}%
\begin{pgfscope}%
\pgftext[left,bottom,x=1.196981in,y=0.218387in,rotate=0.000000]{{\sffamily\fontsize{12.000000}{14.400000}\selectfont 0}}
%
\end{pgfscope}%
\begin{pgfscope}%
\pgfpathrectangle{\pgfqpoint{1.250000in}{0.400000in}}{\pgfqpoint{7.750000in}{3.200000in}} %
\pgfusepath{clip}%
\pgfsetbuttcap%
\pgfsetroundjoin%
\pgfsetlinewidth{0.501875pt}%
\definecolor{currentstroke}{rgb}{0.000000,0.000000,0.000000}%
\pgfsetstrokecolor{currentstroke}%
\pgfsetdash{{1.000000pt}{3.000000pt}}{0.000000pt}%
\pgfpathmoveto{\pgfqpoint{2.541667in}{0.400000in}}%
\pgfpathlineto{\pgfqpoint{2.541667in}{3.600000in}}%
\pgfusepath{stroke}%
\end{pgfscope}%
\begin{pgfscope}%
\pgfsetbuttcap%
\pgfsetroundjoin%
\definecolor{currentfill}{rgb}{0.000000,0.000000,0.000000}%
\pgfsetfillcolor{currentfill}%
\pgfsetlinewidth{0.501875pt}%
\definecolor{currentstroke}{rgb}{0.000000,0.000000,0.000000}%
\pgfsetstrokecolor{currentstroke}%
\pgfsetdash{}{0pt}%
\pgfsys@defobject{currentmarker}{\pgfqpoint{0.000000in}{0.000000in}}{\pgfqpoint{0.000000in}{0.055556in}}{%
\pgfpathmoveto{\pgfqpoint{0.000000in}{0.000000in}}%
\pgfpathlineto{\pgfqpoint{0.000000in}{0.055556in}}%
\pgfusepath{stroke,fill}%
}%
\begin{pgfscope}%
\pgfsys@transformshift{2.541667in}{0.400000in}%
\pgfsys@useobject{currentmarker}{}%
\end{pgfscope}%
\end{pgfscope}%
\begin{pgfscope}%
\pgfsetbuttcap%
\pgfsetroundjoin%
\definecolor{currentfill}{rgb}{0.000000,0.000000,0.000000}%
\pgfsetfillcolor{currentfill}%
\pgfsetlinewidth{0.501875pt}%
\definecolor{currentstroke}{rgb}{0.000000,0.000000,0.000000}%
\pgfsetstrokecolor{currentstroke}%
\pgfsetdash{}{0pt}%
\pgfsys@defobject{currentmarker}{\pgfqpoint{0.000000in}{-0.055556in}}{\pgfqpoint{0.000000in}{0.000000in}}{%
\pgfpathmoveto{\pgfqpoint{0.000000in}{0.000000in}}%
\pgfpathlineto{\pgfqpoint{0.000000in}{-0.055556in}}%
\pgfusepath{stroke,fill}%
}%
\begin{pgfscope}%
\pgfsys@transformshift{2.541667in}{3.600000in}%
\pgfsys@useobject{currentmarker}{}%
\end{pgfscope}%
\end{pgfscope}%
\begin{pgfscope}%
\pgftext[left,bottom,x=2.382609in,y=0.218387in,rotate=0.000000]{{\sffamily\fontsize{12.000000}{14.400000}\selectfont 100}}
%
\end{pgfscope}%
\begin{pgfscope}%
\pgfpathrectangle{\pgfqpoint{1.250000in}{0.400000in}}{\pgfqpoint{7.750000in}{3.200000in}} %
\pgfusepath{clip}%
\pgfsetbuttcap%
\pgfsetroundjoin%
\pgfsetlinewidth{0.501875pt}%
\definecolor{currentstroke}{rgb}{0.000000,0.000000,0.000000}%
\pgfsetstrokecolor{currentstroke}%
\pgfsetdash{{1.000000pt}{3.000000pt}}{0.000000pt}%
\pgfpathmoveto{\pgfqpoint{3.833333in}{0.400000in}}%
\pgfpathlineto{\pgfqpoint{3.833333in}{3.600000in}}%
\pgfusepath{stroke}%
\end{pgfscope}%
\begin{pgfscope}%
\pgfsetbuttcap%
\pgfsetroundjoin%
\definecolor{currentfill}{rgb}{0.000000,0.000000,0.000000}%
\pgfsetfillcolor{currentfill}%
\pgfsetlinewidth{0.501875pt}%
\definecolor{currentstroke}{rgb}{0.000000,0.000000,0.000000}%
\pgfsetstrokecolor{currentstroke}%
\pgfsetdash{}{0pt}%
\pgfsys@defobject{currentmarker}{\pgfqpoint{0.000000in}{0.000000in}}{\pgfqpoint{0.000000in}{0.055556in}}{%
\pgfpathmoveto{\pgfqpoint{0.000000in}{0.000000in}}%
\pgfpathlineto{\pgfqpoint{0.000000in}{0.055556in}}%
\pgfusepath{stroke,fill}%
}%
\begin{pgfscope}%
\pgfsys@transformshift{3.833333in}{0.400000in}%
\pgfsys@useobject{currentmarker}{}%
\end{pgfscope}%
\end{pgfscope}%
\begin{pgfscope}%
\pgfsetbuttcap%
\pgfsetroundjoin%
\definecolor{currentfill}{rgb}{0.000000,0.000000,0.000000}%
\pgfsetfillcolor{currentfill}%
\pgfsetlinewidth{0.501875pt}%
\definecolor{currentstroke}{rgb}{0.000000,0.000000,0.000000}%
\pgfsetstrokecolor{currentstroke}%
\pgfsetdash{}{0pt}%
\pgfsys@defobject{currentmarker}{\pgfqpoint{0.000000in}{-0.055556in}}{\pgfqpoint{0.000000in}{0.000000in}}{%
\pgfpathmoveto{\pgfqpoint{0.000000in}{0.000000in}}%
\pgfpathlineto{\pgfqpoint{0.000000in}{-0.055556in}}%
\pgfusepath{stroke,fill}%
}%
\begin{pgfscope}%
\pgfsys@transformshift{3.833333in}{3.600000in}%
\pgfsys@useobject{currentmarker}{}%
\end{pgfscope}%
\end{pgfscope}%
\begin{pgfscope}%
\pgftext[left,bottom,x=3.674276in,y=0.218387in,rotate=0.000000]{{\sffamily\fontsize{12.000000}{14.400000}\selectfont 200}}
%
\end{pgfscope}%
\begin{pgfscope}%
\pgfpathrectangle{\pgfqpoint{1.250000in}{0.400000in}}{\pgfqpoint{7.750000in}{3.200000in}} %
\pgfusepath{clip}%
\pgfsetbuttcap%
\pgfsetroundjoin%
\pgfsetlinewidth{0.501875pt}%
\definecolor{currentstroke}{rgb}{0.000000,0.000000,0.000000}%
\pgfsetstrokecolor{currentstroke}%
\pgfsetdash{{1.000000pt}{3.000000pt}}{0.000000pt}%
\pgfpathmoveto{\pgfqpoint{5.125000in}{0.400000in}}%
\pgfpathlineto{\pgfqpoint{5.125000in}{3.600000in}}%
\pgfusepath{stroke}%
\end{pgfscope}%
\begin{pgfscope}%
\pgfsetbuttcap%
\pgfsetroundjoin%
\definecolor{currentfill}{rgb}{0.000000,0.000000,0.000000}%
\pgfsetfillcolor{currentfill}%
\pgfsetlinewidth{0.501875pt}%
\definecolor{currentstroke}{rgb}{0.000000,0.000000,0.000000}%
\pgfsetstrokecolor{currentstroke}%
\pgfsetdash{}{0pt}%
\pgfsys@defobject{currentmarker}{\pgfqpoint{0.000000in}{0.000000in}}{\pgfqpoint{0.000000in}{0.055556in}}{%
\pgfpathmoveto{\pgfqpoint{0.000000in}{0.000000in}}%
\pgfpathlineto{\pgfqpoint{0.000000in}{0.055556in}}%
\pgfusepath{stroke,fill}%
}%
\begin{pgfscope}%
\pgfsys@transformshift{5.125000in}{0.400000in}%
\pgfsys@useobject{currentmarker}{}%
\end{pgfscope}%
\end{pgfscope}%
\begin{pgfscope}%
\pgfsetbuttcap%
\pgfsetroundjoin%
\definecolor{currentfill}{rgb}{0.000000,0.000000,0.000000}%
\pgfsetfillcolor{currentfill}%
\pgfsetlinewidth{0.501875pt}%
\definecolor{currentstroke}{rgb}{0.000000,0.000000,0.000000}%
\pgfsetstrokecolor{currentstroke}%
\pgfsetdash{}{0pt}%
\pgfsys@defobject{currentmarker}{\pgfqpoint{0.000000in}{-0.055556in}}{\pgfqpoint{0.000000in}{0.000000in}}{%
\pgfpathmoveto{\pgfqpoint{0.000000in}{0.000000in}}%
\pgfpathlineto{\pgfqpoint{0.000000in}{-0.055556in}}%
\pgfusepath{stroke,fill}%
}%
\begin{pgfscope}%
\pgfsys@transformshift{5.125000in}{3.600000in}%
\pgfsys@useobject{currentmarker}{}%
\end{pgfscope}%
\end{pgfscope}%
\begin{pgfscope}%
\pgftext[left,bottom,x=4.965942in,y=0.218387in,rotate=0.000000]{{\sffamily\fontsize{12.000000}{14.400000}\selectfont 300}}
%
\end{pgfscope}%
\begin{pgfscope}%
\pgfpathrectangle{\pgfqpoint{1.250000in}{0.400000in}}{\pgfqpoint{7.750000in}{3.200000in}} %
\pgfusepath{clip}%
\pgfsetbuttcap%
\pgfsetroundjoin%
\pgfsetlinewidth{0.501875pt}%
\definecolor{currentstroke}{rgb}{0.000000,0.000000,0.000000}%
\pgfsetstrokecolor{currentstroke}%
\pgfsetdash{{1.000000pt}{3.000000pt}}{0.000000pt}%
\pgfpathmoveto{\pgfqpoint{6.416667in}{0.400000in}}%
\pgfpathlineto{\pgfqpoint{6.416667in}{3.600000in}}%
\pgfusepath{stroke}%
\end{pgfscope}%
\begin{pgfscope}%
\pgfsetbuttcap%
\pgfsetroundjoin%
\definecolor{currentfill}{rgb}{0.000000,0.000000,0.000000}%
\pgfsetfillcolor{currentfill}%
\pgfsetlinewidth{0.501875pt}%
\definecolor{currentstroke}{rgb}{0.000000,0.000000,0.000000}%
\pgfsetstrokecolor{currentstroke}%
\pgfsetdash{}{0pt}%
\pgfsys@defobject{currentmarker}{\pgfqpoint{0.000000in}{0.000000in}}{\pgfqpoint{0.000000in}{0.055556in}}{%
\pgfpathmoveto{\pgfqpoint{0.000000in}{0.000000in}}%
\pgfpathlineto{\pgfqpoint{0.000000in}{0.055556in}}%
\pgfusepath{stroke,fill}%
}%
\begin{pgfscope}%
\pgfsys@transformshift{6.416667in}{0.400000in}%
\pgfsys@useobject{currentmarker}{}%
\end{pgfscope}%
\end{pgfscope}%
\begin{pgfscope}%
\pgfsetbuttcap%
\pgfsetroundjoin%
\definecolor{currentfill}{rgb}{0.000000,0.000000,0.000000}%
\pgfsetfillcolor{currentfill}%
\pgfsetlinewidth{0.501875pt}%
\definecolor{currentstroke}{rgb}{0.000000,0.000000,0.000000}%
\pgfsetstrokecolor{currentstroke}%
\pgfsetdash{}{0pt}%
\pgfsys@defobject{currentmarker}{\pgfqpoint{0.000000in}{-0.055556in}}{\pgfqpoint{0.000000in}{0.000000in}}{%
\pgfpathmoveto{\pgfqpoint{0.000000in}{0.000000in}}%
\pgfpathlineto{\pgfqpoint{0.000000in}{-0.055556in}}%
\pgfusepath{stroke,fill}%
}%
\begin{pgfscope}%
\pgfsys@transformshift{6.416667in}{3.600000in}%
\pgfsys@useobject{currentmarker}{}%
\end{pgfscope}%
\end{pgfscope}%
\begin{pgfscope}%
\pgftext[left,bottom,x=6.257609in,y=0.218387in,rotate=0.000000]{{\sffamily\fontsize{12.000000}{14.400000}\selectfont 400}}
%
\end{pgfscope}%
\begin{pgfscope}%
\pgfpathrectangle{\pgfqpoint{1.250000in}{0.400000in}}{\pgfqpoint{7.750000in}{3.200000in}} %
\pgfusepath{clip}%
\pgfsetbuttcap%
\pgfsetroundjoin%
\pgfsetlinewidth{0.501875pt}%
\definecolor{currentstroke}{rgb}{0.000000,0.000000,0.000000}%
\pgfsetstrokecolor{currentstroke}%
\pgfsetdash{{1.000000pt}{3.000000pt}}{0.000000pt}%
\pgfpathmoveto{\pgfqpoint{7.708333in}{0.400000in}}%
\pgfpathlineto{\pgfqpoint{7.708333in}{3.600000in}}%
\pgfusepath{stroke}%
\end{pgfscope}%
\begin{pgfscope}%
\pgfsetbuttcap%
\pgfsetroundjoin%
\definecolor{currentfill}{rgb}{0.000000,0.000000,0.000000}%
\pgfsetfillcolor{currentfill}%
\pgfsetlinewidth{0.501875pt}%
\definecolor{currentstroke}{rgb}{0.000000,0.000000,0.000000}%
\pgfsetstrokecolor{currentstroke}%
\pgfsetdash{}{0pt}%
\pgfsys@defobject{currentmarker}{\pgfqpoint{0.000000in}{0.000000in}}{\pgfqpoint{0.000000in}{0.055556in}}{%
\pgfpathmoveto{\pgfqpoint{0.000000in}{0.000000in}}%
\pgfpathlineto{\pgfqpoint{0.000000in}{0.055556in}}%
\pgfusepath{stroke,fill}%
}%
\begin{pgfscope}%
\pgfsys@transformshift{7.708333in}{0.400000in}%
\pgfsys@useobject{currentmarker}{}%
\end{pgfscope}%
\end{pgfscope}%
\begin{pgfscope}%
\pgfsetbuttcap%
\pgfsetroundjoin%
\definecolor{currentfill}{rgb}{0.000000,0.000000,0.000000}%
\pgfsetfillcolor{currentfill}%
\pgfsetlinewidth{0.501875pt}%
\definecolor{currentstroke}{rgb}{0.000000,0.000000,0.000000}%
\pgfsetstrokecolor{currentstroke}%
\pgfsetdash{}{0pt}%
\pgfsys@defobject{currentmarker}{\pgfqpoint{0.000000in}{-0.055556in}}{\pgfqpoint{0.000000in}{0.000000in}}{%
\pgfpathmoveto{\pgfqpoint{0.000000in}{0.000000in}}%
\pgfpathlineto{\pgfqpoint{0.000000in}{-0.055556in}}%
\pgfusepath{stroke,fill}%
}%
\begin{pgfscope}%
\pgfsys@transformshift{7.708333in}{3.600000in}%
\pgfsys@useobject{currentmarker}{}%
\end{pgfscope}%
\end{pgfscope}%
\begin{pgfscope}%
\pgftext[left,bottom,x=7.549276in,y=0.218387in,rotate=0.000000]{{\sffamily\fontsize{12.000000}{14.400000}\selectfont 500}}
%
\end{pgfscope}%
\begin{pgfscope}%
\pgfpathrectangle{\pgfqpoint{1.250000in}{0.400000in}}{\pgfqpoint{7.750000in}{3.200000in}} %
\pgfusepath{clip}%
\pgfsetbuttcap%
\pgfsetroundjoin%
\pgfsetlinewidth{0.501875pt}%
\definecolor{currentstroke}{rgb}{0.000000,0.000000,0.000000}%
\pgfsetstrokecolor{currentstroke}%
\pgfsetdash{{1.000000pt}{3.000000pt}}{0.000000pt}%
\pgfpathmoveto{\pgfqpoint{9.000000in}{0.400000in}}%
\pgfpathlineto{\pgfqpoint{9.000000in}{3.600000in}}%
\pgfusepath{stroke}%
\end{pgfscope}%
\begin{pgfscope}%
\pgfsetbuttcap%
\pgfsetroundjoin%
\definecolor{currentfill}{rgb}{0.000000,0.000000,0.000000}%
\pgfsetfillcolor{currentfill}%
\pgfsetlinewidth{0.501875pt}%
\definecolor{currentstroke}{rgb}{0.000000,0.000000,0.000000}%
\pgfsetstrokecolor{currentstroke}%
\pgfsetdash{}{0pt}%
\pgfsys@defobject{currentmarker}{\pgfqpoint{0.000000in}{0.000000in}}{\pgfqpoint{0.000000in}{0.055556in}}{%
\pgfpathmoveto{\pgfqpoint{0.000000in}{0.000000in}}%
\pgfpathlineto{\pgfqpoint{0.000000in}{0.055556in}}%
\pgfusepath{stroke,fill}%
}%
\begin{pgfscope}%
\pgfsys@transformshift{9.000000in}{0.400000in}%
\pgfsys@useobject{currentmarker}{}%
\end{pgfscope}%
\end{pgfscope}%
\begin{pgfscope}%
\pgfsetbuttcap%
\pgfsetroundjoin%
\definecolor{currentfill}{rgb}{0.000000,0.000000,0.000000}%
\pgfsetfillcolor{currentfill}%
\pgfsetlinewidth{0.501875pt}%
\definecolor{currentstroke}{rgb}{0.000000,0.000000,0.000000}%
\pgfsetstrokecolor{currentstroke}%
\pgfsetdash{}{0pt}%
\pgfsys@defobject{currentmarker}{\pgfqpoint{0.000000in}{-0.055556in}}{\pgfqpoint{0.000000in}{0.000000in}}{%
\pgfpathmoveto{\pgfqpoint{0.000000in}{0.000000in}}%
\pgfpathlineto{\pgfqpoint{0.000000in}{-0.055556in}}%
\pgfusepath{stroke,fill}%
}%
\begin{pgfscope}%
\pgfsys@transformshift{9.000000in}{3.600000in}%
\pgfsys@useobject{currentmarker}{}%
\end{pgfscope}%
\end{pgfscope}%
\begin{pgfscope}%
\pgftext[left,bottom,x=8.840942in,y=0.218387in,rotate=0.000000]{{\sffamily\fontsize{12.000000}{14.400000}\selectfont 600}}
%
\end{pgfscope}%
\begin{pgfscope}%
\pgftext[left,bottom,x=4.865560in,y=0.022315in,rotate=0.000000]{{\sffamily\fontsize{12.000000}{14.400000}\selectfont x in m}}
%
\end{pgfscope}%
\begin{pgfscope}%
\pgfpathrectangle{\pgfqpoint{1.250000in}{0.400000in}}{\pgfqpoint{7.750000in}{3.200000in}} %
\pgfusepath{clip}%
\pgfsetbuttcap%
\pgfsetroundjoin%
\pgfsetlinewidth{0.501875pt}%
\definecolor{currentstroke}{rgb}{0.000000,0.000000,0.000000}%
\pgfsetstrokecolor{currentstroke}%
\pgfsetdash{{1.000000pt}{3.000000pt}}{0.000000pt}%
\pgfpathmoveto{\pgfqpoint{1.250000in}{0.400000in}}%
\pgfpathlineto{\pgfqpoint{9.000000in}{0.400000in}}%
\pgfusepath{stroke}%
\end{pgfscope}%
\begin{pgfscope}%
\pgfsetbuttcap%
\pgfsetroundjoin%
\definecolor{currentfill}{rgb}{0.000000,0.000000,0.000000}%
\pgfsetfillcolor{currentfill}%
\pgfsetlinewidth{0.501875pt}%
\definecolor{currentstroke}{rgb}{0.000000,0.000000,0.000000}%
\pgfsetstrokecolor{currentstroke}%
\pgfsetdash{}{0pt}%
\pgfsys@defobject{currentmarker}{\pgfqpoint{0.000000in}{0.000000in}}{\pgfqpoint{0.055556in}{0.000000in}}{%
\pgfpathmoveto{\pgfqpoint{0.000000in}{0.000000in}}%
\pgfpathlineto{\pgfqpoint{0.055556in}{0.000000in}}%
\pgfusepath{stroke,fill}%
}%
\begin{pgfscope}%
\pgfsys@transformshift{1.250000in}{0.400000in}%
\pgfsys@useobject{currentmarker}{}%
\end{pgfscope}%
\end{pgfscope}%
\begin{pgfscope}%
\pgfsetbuttcap%
\pgfsetroundjoin%
\definecolor{currentfill}{rgb}{0.000000,0.000000,0.000000}%
\pgfsetfillcolor{currentfill}%
\pgfsetlinewidth{0.501875pt}%
\definecolor{currentstroke}{rgb}{0.000000,0.000000,0.000000}%
\pgfsetstrokecolor{currentstroke}%
\pgfsetdash{}{0pt}%
\pgfsys@defobject{currentmarker}{\pgfqpoint{-0.055556in}{0.000000in}}{\pgfqpoint{0.000000in}{0.000000in}}{%
\pgfpathmoveto{\pgfqpoint{0.000000in}{0.000000in}}%
\pgfpathlineto{\pgfqpoint{-0.055556in}{0.000000in}}%
\pgfusepath{stroke,fill}%
}%
\begin{pgfscope}%
\pgfsys@transformshift{9.000000in}{0.400000in}%
\pgfsys@useobject{currentmarker}{}%
\end{pgfscope}%
\end{pgfscope}%
\begin{pgfscope}%
\pgftext[left,bottom,x=0.842719in,y=0.336971in,rotate=0.000000]{{\sffamily\fontsize{12.000000}{14.400000}\selectfont −20}}
%
\end{pgfscope}%
\begin{pgfscope}%
\pgfpathrectangle{\pgfqpoint{1.250000in}{0.400000in}}{\pgfqpoint{7.750000in}{3.200000in}} %
\pgfusepath{clip}%
\pgfsetbuttcap%
\pgfsetroundjoin%
\pgfsetlinewidth{0.501875pt}%
\definecolor{currentstroke}{rgb}{0.000000,0.000000,0.000000}%
\pgfsetstrokecolor{currentstroke}%
\pgfsetdash{{1.000000pt}{3.000000pt}}{0.000000pt}%
\pgfpathmoveto{\pgfqpoint{1.250000in}{0.800000in}}%
\pgfpathlineto{\pgfqpoint{9.000000in}{0.800000in}}%
\pgfusepath{stroke}%
\end{pgfscope}%
\begin{pgfscope}%
\pgfsetbuttcap%
\pgfsetroundjoin%
\definecolor{currentfill}{rgb}{0.000000,0.000000,0.000000}%
\pgfsetfillcolor{currentfill}%
\pgfsetlinewidth{0.501875pt}%
\definecolor{currentstroke}{rgb}{0.000000,0.000000,0.000000}%
\pgfsetstrokecolor{currentstroke}%
\pgfsetdash{}{0pt}%
\pgfsys@defobject{currentmarker}{\pgfqpoint{0.000000in}{0.000000in}}{\pgfqpoint{0.055556in}{0.000000in}}{%
\pgfpathmoveto{\pgfqpoint{0.000000in}{0.000000in}}%
\pgfpathlineto{\pgfqpoint{0.055556in}{0.000000in}}%
\pgfusepath{stroke,fill}%
}%
\begin{pgfscope}%
\pgfsys@transformshift{1.250000in}{0.800000in}%
\pgfsys@useobject{currentmarker}{}%
\end{pgfscope}%
\end{pgfscope}%
\begin{pgfscope}%
\pgfsetbuttcap%
\pgfsetroundjoin%
\definecolor{currentfill}{rgb}{0.000000,0.000000,0.000000}%
\pgfsetfillcolor{currentfill}%
\pgfsetlinewidth{0.501875pt}%
\definecolor{currentstroke}{rgb}{0.000000,0.000000,0.000000}%
\pgfsetstrokecolor{currentstroke}%
\pgfsetdash{}{0pt}%
\pgfsys@defobject{currentmarker}{\pgfqpoint{-0.055556in}{0.000000in}}{\pgfqpoint{0.000000in}{0.000000in}}{%
\pgfpathmoveto{\pgfqpoint{0.000000in}{0.000000in}}%
\pgfpathlineto{\pgfqpoint{-0.055556in}{0.000000in}}%
\pgfusepath{stroke,fill}%
}%
\begin{pgfscope}%
\pgfsys@transformshift{9.000000in}{0.800000in}%
\pgfsys@useobject{currentmarker}{}%
\end{pgfscope}%
\end{pgfscope}%
\begin{pgfscope}%
\pgftext[left,bottom,x=1.088406in,y=0.736971in,rotate=0.000000]{{\sffamily\fontsize{12.000000}{14.400000}\selectfont 0}}
%
\end{pgfscope}%
\begin{pgfscope}%
\pgfpathrectangle{\pgfqpoint{1.250000in}{0.400000in}}{\pgfqpoint{7.750000in}{3.200000in}} %
\pgfusepath{clip}%
\pgfsetbuttcap%
\pgfsetroundjoin%
\pgfsetlinewidth{0.501875pt}%
\definecolor{currentstroke}{rgb}{0.000000,0.000000,0.000000}%
\pgfsetstrokecolor{currentstroke}%
\pgfsetdash{{1.000000pt}{3.000000pt}}{0.000000pt}%
\pgfpathmoveto{\pgfqpoint{1.250000in}{1.200000in}}%
\pgfpathlineto{\pgfqpoint{9.000000in}{1.200000in}}%
\pgfusepath{stroke}%
\end{pgfscope}%
\begin{pgfscope}%
\pgfsetbuttcap%
\pgfsetroundjoin%
\definecolor{currentfill}{rgb}{0.000000,0.000000,0.000000}%
\pgfsetfillcolor{currentfill}%
\pgfsetlinewidth{0.501875pt}%
\definecolor{currentstroke}{rgb}{0.000000,0.000000,0.000000}%
\pgfsetstrokecolor{currentstroke}%
\pgfsetdash{}{0pt}%
\pgfsys@defobject{currentmarker}{\pgfqpoint{0.000000in}{0.000000in}}{\pgfqpoint{0.055556in}{0.000000in}}{%
\pgfpathmoveto{\pgfqpoint{0.000000in}{0.000000in}}%
\pgfpathlineto{\pgfqpoint{0.055556in}{0.000000in}}%
\pgfusepath{stroke,fill}%
}%
\begin{pgfscope}%
\pgfsys@transformshift{1.250000in}{1.200000in}%
\pgfsys@useobject{currentmarker}{}%
\end{pgfscope}%
\end{pgfscope}%
\begin{pgfscope}%
\pgfsetbuttcap%
\pgfsetroundjoin%
\definecolor{currentfill}{rgb}{0.000000,0.000000,0.000000}%
\pgfsetfillcolor{currentfill}%
\pgfsetlinewidth{0.501875pt}%
\definecolor{currentstroke}{rgb}{0.000000,0.000000,0.000000}%
\pgfsetstrokecolor{currentstroke}%
\pgfsetdash{}{0pt}%
\pgfsys@defobject{currentmarker}{\pgfqpoint{-0.055556in}{0.000000in}}{\pgfqpoint{0.000000in}{0.000000in}}{%
\pgfpathmoveto{\pgfqpoint{0.000000in}{0.000000in}}%
\pgfpathlineto{\pgfqpoint{-0.055556in}{0.000000in}}%
\pgfusepath{stroke,fill}%
}%
\begin{pgfscope}%
\pgfsys@transformshift{9.000000in}{1.200000in}%
\pgfsys@useobject{currentmarker}{}%
\end{pgfscope}%
\end{pgfscope}%
\begin{pgfscope}%
\pgftext[left,bottom,x=0.982368in,y=1.136971in,rotate=0.000000]{{\sffamily\fontsize{12.000000}{14.400000}\selectfont 20}}
%
\end{pgfscope}%
\begin{pgfscope}%
\pgfpathrectangle{\pgfqpoint{1.250000in}{0.400000in}}{\pgfqpoint{7.750000in}{3.200000in}} %
\pgfusepath{clip}%
\pgfsetbuttcap%
\pgfsetroundjoin%
\pgfsetlinewidth{0.501875pt}%
\definecolor{currentstroke}{rgb}{0.000000,0.000000,0.000000}%
\pgfsetstrokecolor{currentstroke}%
\pgfsetdash{{1.000000pt}{3.000000pt}}{0.000000pt}%
\pgfpathmoveto{\pgfqpoint{1.250000in}{1.600000in}}%
\pgfpathlineto{\pgfqpoint{9.000000in}{1.600000in}}%
\pgfusepath{stroke}%
\end{pgfscope}%
\begin{pgfscope}%
\pgfsetbuttcap%
\pgfsetroundjoin%
\definecolor{currentfill}{rgb}{0.000000,0.000000,0.000000}%
\pgfsetfillcolor{currentfill}%
\pgfsetlinewidth{0.501875pt}%
\definecolor{currentstroke}{rgb}{0.000000,0.000000,0.000000}%
\pgfsetstrokecolor{currentstroke}%
\pgfsetdash{}{0pt}%
\pgfsys@defobject{currentmarker}{\pgfqpoint{0.000000in}{0.000000in}}{\pgfqpoint{0.055556in}{0.000000in}}{%
\pgfpathmoveto{\pgfqpoint{0.000000in}{0.000000in}}%
\pgfpathlineto{\pgfqpoint{0.055556in}{0.000000in}}%
\pgfusepath{stroke,fill}%
}%
\begin{pgfscope}%
\pgfsys@transformshift{1.250000in}{1.600000in}%
\pgfsys@useobject{currentmarker}{}%
\end{pgfscope}%
\end{pgfscope}%
\begin{pgfscope}%
\pgfsetbuttcap%
\pgfsetroundjoin%
\definecolor{currentfill}{rgb}{0.000000,0.000000,0.000000}%
\pgfsetfillcolor{currentfill}%
\pgfsetlinewidth{0.501875pt}%
\definecolor{currentstroke}{rgb}{0.000000,0.000000,0.000000}%
\pgfsetstrokecolor{currentstroke}%
\pgfsetdash{}{0pt}%
\pgfsys@defobject{currentmarker}{\pgfqpoint{-0.055556in}{0.000000in}}{\pgfqpoint{0.000000in}{0.000000in}}{%
\pgfpathmoveto{\pgfqpoint{0.000000in}{0.000000in}}%
\pgfpathlineto{\pgfqpoint{-0.055556in}{0.000000in}}%
\pgfusepath{stroke,fill}%
}%
\begin{pgfscope}%
\pgfsys@transformshift{9.000000in}{1.600000in}%
\pgfsys@useobject{currentmarker}{}%
\end{pgfscope}%
\end{pgfscope}%
\begin{pgfscope}%
\pgftext[left,bottom,x=0.982368in,y=1.536971in,rotate=0.000000]{{\sffamily\fontsize{12.000000}{14.400000}\selectfont 40}}
%
\end{pgfscope}%
\begin{pgfscope}%
\pgfpathrectangle{\pgfqpoint{1.250000in}{0.400000in}}{\pgfqpoint{7.750000in}{3.200000in}} %
\pgfusepath{clip}%
\pgfsetbuttcap%
\pgfsetroundjoin%
\pgfsetlinewidth{0.501875pt}%
\definecolor{currentstroke}{rgb}{0.000000,0.000000,0.000000}%
\pgfsetstrokecolor{currentstroke}%
\pgfsetdash{{1.000000pt}{3.000000pt}}{0.000000pt}%
\pgfpathmoveto{\pgfqpoint{1.250000in}{2.000000in}}%
\pgfpathlineto{\pgfqpoint{9.000000in}{2.000000in}}%
\pgfusepath{stroke}%
\end{pgfscope}%
\begin{pgfscope}%
\pgfsetbuttcap%
\pgfsetroundjoin%
\definecolor{currentfill}{rgb}{0.000000,0.000000,0.000000}%
\pgfsetfillcolor{currentfill}%
\pgfsetlinewidth{0.501875pt}%
\definecolor{currentstroke}{rgb}{0.000000,0.000000,0.000000}%
\pgfsetstrokecolor{currentstroke}%
\pgfsetdash{}{0pt}%
\pgfsys@defobject{currentmarker}{\pgfqpoint{0.000000in}{0.000000in}}{\pgfqpoint{0.055556in}{0.000000in}}{%
\pgfpathmoveto{\pgfqpoint{0.000000in}{0.000000in}}%
\pgfpathlineto{\pgfqpoint{0.055556in}{0.000000in}}%
\pgfusepath{stroke,fill}%
}%
\begin{pgfscope}%
\pgfsys@transformshift{1.250000in}{2.000000in}%
\pgfsys@useobject{currentmarker}{}%
\end{pgfscope}%
\end{pgfscope}%
\begin{pgfscope}%
\pgfsetbuttcap%
\pgfsetroundjoin%
\definecolor{currentfill}{rgb}{0.000000,0.000000,0.000000}%
\pgfsetfillcolor{currentfill}%
\pgfsetlinewidth{0.501875pt}%
\definecolor{currentstroke}{rgb}{0.000000,0.000000,0.000000}%
\pgfsetstrokecolor{currentstroke}%
\pgfsetdash{}{0pt}%
\pgfsys@defobject{currentmarker}{\pgfqpoint{-0.055556in}{0.000000in}}{\pgfqpoint{0.000000in}{0.000000in}}{%
\pgfpathmoveto{\pgfqpoint{0.000000in}{0.000000in}}%
\pgfpathlineto{\pgfqpoint{-0.055556in}{0.000000in}}%
\pgfusepath{stroke,fill}%
}%
\begin{pgfscope}%
\pgfsys@transformshift{9.000000in}{2.000000in}%
\pgfsys@useobject{currentmarker}{}%
\end{pgfscope}%
\end{pgfscope}%
\begin{pgfscope}%
\pgftext[left,bottom,x=0.982368in,y=1.936971in,rotate=0.000000]{{\sffamily\fontsize{12.000000}{14.400000}\selectfont 60}}
%
\end{pgfscope}%
\begin{pgfscope}%
\pgfpathrectangle{\pgfqpoint{1.250000in}{0.400000in}}{\pgfqpoint{7.750000in}{3.200000in}} %
\pgfusepath{clip}%
\pgfsetbuttcap%
\pgfsetroundjoin%
\pgfsetlinewidth{0.501875pt}%
\definecolor{currentstroke}{rgb}{0.000000,0.000000,0.000000}%
\pgfsetstrokecolor{currentstroke}%
\pgfsetdash{{1.000000pt}{3.000000pt}}{0.000000pt}%
\pgfpathmoveto{\pgfqpoint{1.250000in}{2.400000in}}%
\pgfpathlineto{\pgfqpoint{9.000000in}{2.400000in}}%
\pgfusepath{stroke}%
\end{pgfscope}%
\begin{pgfscope}%
\pgfsetbuttcap%
\pgfsetroundjoin%
\definecolor{currentfill}{rgb}{0.000000,0.000000,0.000000}%
\pgfsetfillcolor{currentfill}%
\pgfsetlinewidth{0.501875pt}%
\definecolor{currentstroke}{rgb}{0.000000,0.000000,0.000000}%
\pgfsetstrokecolor{currentstroke}%
\pgfsetdash{}{0pt}%
\pgfsys@defobject{currentmarker}{\pgfqpoint{0.000000in}{0.000000in}}{\pgfqpoint{0.055556in}{0.000000in}}{%
\pgfpathmoveto{\pgfqpoint{0.000000in}{0.000000in}}%
\pgfpathlineto{\pgfqpoint{0.055556in}{0.000000in}}%
\pgfusepath{stroke,fill}%
}%
\begin{pgfscope}%
\pgfsys@transformshift{1.250000in}{2.400000in}%
\pgfsys@useobject{currentmarker}{}%
\end{pgfscope}%
\end{pgfscope}%
\begin{pgfscope}%
\pgfsetbuttcap%
\pgfsetroundjoin%
\definecolor{currentfill}{rgb}{0.000000,0.000000,0.000000}%
\pgfsetfillcolor{currentfill}%
\pgfsetlinewidth{0.501875pt}%
\definecolor{currentstroke}{rgb}{0.000000,0.000000,0.000000}%
\pgfsetstrokecolor{currentstroke}%
\pgfsetdash{}{0pt}%
\pgfsys@defobject{currentmarker}{\pgfqpoint{-0.055556in}{0.000000in}}{\pgfqpoint{0.000000in}{0.000000in}}{%
\pgfpathmoveto{\pgfqpoint{0.000000in}{0.000000in}}%
\pgfpathlineto{\pgfqpoint{-0.055556in}{0.000000in}}%
\pgfusepath{stroke,fill}%
}%
\begin{pgfscope}%
\pgfsys@transformshift{9.000000in}{2.400000in}%
\pgfsys@useobject{currentmarker}{}%
\end{pgfscope}%
\end{pgfscope}%
\begin{pgfscope}%
\pgftext[left,bottom,x=0.982368in,y=2.336971in,rotate=0.000000]{{\sffamily\fontsize{12.000000}{14.400000}\selectfont 80}}
%
\end{pgfscope}%
\begin{pgfscope}%
\pgfpathrectangle{\pgfqpoint{1.250000in}{0.400000in}}{\pgfqpoint{7.750000in}{3.200000in}} %
\pgfusepath{clip}%
\pgfsetbuttcap%
\pgfsetroundjoin%
\pgfsetlinewidth{0.501875pt}%
\definecolor{currentstroke}{rgb}{0.000000,0.000000,0.000000}%
\pgfsetstrokecolor{currentstroke}%
\pgfsetdash{{1.000000pt}{3.000000pt}}{0.000000pt}%
\pgfpathmoveto{\pgfqpoint{1.250000in}{2.800000in}}%
\pgfpathlineto{\pgfqpoint{9.000000in}{2.800000in}}%
\pgfusepath{stroke}%
\end{pgfscope}%
\begin{pgfscope}%
\pgfsetbuttcap%
\pgfsetroundjoin%
\definecolor{currentfill}{rgb}{0.000000,0.000000,0.000000}%
\pgfsetfillcolor{currentfill}%
\pgfsetlinewidth{0.501875pt}%
\definecolor{currentstroke}{rgb}{0.000000,0.000000,0.000000}%
\pgfsetstrokecolor{currentstroke}%
\pgfsetdash{}{0pt}%
\pgfsys@defobject{currentmarker}{\pgfqpoint{0.000000in}{0.000000in}}{\pgfqpoint{0.055556in}{0.000000in}}{%
\pgfpathmoveto{\pgfqpoint{0.000000in}{0.000000in}}%
\pgfpathlineto{\pgfqpoint{0.055556in}{0.000000in}}%
\pgfusepath{stroke,fill}%
}%
\begin{pgfscope}%
\pgfsys@transformshift{1.250000in}{2.800000in}%
\pgfsys@useobject{currentmarker}{}%
\end{pgfscope}%
\end{pgfscope}%
\begin{pgfscope}%
\pgfsetbuttcap%
\pgfsetroundjoin%
\definecolor{currentfill}{rgb}{0.000000,0.000000,0.000000}%
\pgfsetfillcolor{currentfill}%
\pgfsetlinewidth{0.501875pt}%
\definecolor{currentstroke}{rgb}{0.000000,0.000000,0.000000}%
\pgfsetstrokecolor{currentstroke}%
\pgfsetdash{}{0pt}%
\pgfsys@defobject{currentmarker}{\pgfqpoint{-0.055556in}{0.000000in}}{\pgfqpoint{0.000000in}{0.000000in}}{%
\pgfpathmoveto{\pgfqpoint{0.000000in}{0.000000in}}%
\pgfpathlineto{\pgfqpoint{-0.055556in}{0.000000in}}%
\pgfusepath{stroke,fill}%
}%
\begin{pgfscope}%
\pgfsys@transformshift{9.000000in}{2.800000in}%
\pgfsys@useobject{currentmarker}{}%
\end{pgfscope}%
\end{pgfscope}%
\begin{pgfscope}%
\pgftext[left,bottom,x=0.876329in,y=2.736971in,rotate=0.000000]{{\sffamily\fontsize{12.000000}{14.400000}\selectfont 100}}
%
\end{pgfscope}%
\begin{pgfscope}%
\pgfpathrectangle{\pgfqpoint{1.250000in}{0.400000in}}{\pgfqpoint{7.750000in}{3.200000in}} %
\pgfusepath{clip}%
\pgfsetbuttcap%
\pgfsetroundjoin%
\pgfsetlinewidth{0.501875pt}%
\definecolor{currentstroke}{rgb}{0.000000,0.000000,0.000000}%
\pgfsetstrokecolor{currentstroke}%
\pgfsetdash{{1.000000pt}{3.000000pt}}{0.000000pt}%
\pgfpathmoveto{\pgfqpoint{1.250000in}{3.200000in}}%
\pgfpathlineto{\pgfqpoint{9.000000in}{3.200000in}}%
\pgfusepath{stroke}%
\end{pgfscope}%
\begin{pgfscope}%
\pgfsetbuttcap%
\pgfsetroundjoin%
\definecolor{currentfill}{rgb}{0.000000,0.000000,0.000000}%
\pgfsetfillcolor{currentfill}%
\pgfsetlinewidth{0.501875pt}%
\definecolor{currentstroke}{rgb}{0.000000,0.000000,0.000000}%
\pgfsetstrokecolor{currentstroke}%
\pgfsetdash{}{0pt}%
\pgfsys@defobject{currentmarker}{\pgfqpoint{0.000000in}{0.000000in}}{\pgfqpoint{0.055556in}{0.000000in}}{%
\pgfpathmoveto{\pgfqpoint{0.000000in}{0.000000in}}%
\pgfpathlineto{\pgfqpoint{0.055556in}{0.000000in}}%
\pgfusepath{stroke,fill}%
}%
\begin{pgfscope}%
\pgfsys@transformshift{1.250000in}{3.200000in}%
\pgfsys@useobject{currentmarker}{}%
\end{pgfscope}%
\end{pgfscope}%
\begin{pgfscope}%
\pgfsetbuttcap%
\pgfsetroundjoin%
\definecolor{currentfill}{rgb}{0.000000,0.000000,0.000000}%
\pgfsetfillcolor{currentfill}%
\pgfsetlinewidth{0.501875pt}%
\definecolor{currentstroke}{rgb}{0.000000,0.000000,0.000000}%
\pgfsetstrokecolor{currentstroke}%
\pgfsetdash{}{0pt}%
\pgfsys@defobject{currentmarker}{\pgfqpoint{-0.055556in}{0.000000in}}{\pgfqpoint{0.000000in}{0.000000in}}{%
\pgfpathmoveto{\pgfqpoint{0.000000in}{0.000000in}}%
\pgfpathlineto{\pgfqpoint{-0.055556in}{0.000000in}}%
\pgfusepath{stroke,fill}%
}%
\begin{pgfscope}%
\pgfsys@transformshift{9.000000in}{3.200000in}%
\pgfsys@useobject{currentmarker}{}%
\end{pgfscope}%
\end{pgfscope}%
\begin{pgfscope}%
\pgftext[left,bottom,x=0.876329in,y=3.136971in,rotate=0.000000]{{\sffamily\fontsize{12.000000}{14.400000}\selectfont 120}}
%
\end{pgfscope}%
\begin{pgfscope}%
\pgfpathrectangle{\pgfqpoint{1.250000in}{0.400000in}}{\pgfqpoint{7.750000in}{3.200000in}} %
\pgfusepath{clip}%
\pgfsetbuttcap%
\pgfsetroundjoin%
\pgfsetlinewidth{0.501875pt}%
\definecolor{currentstroke}{rgb}{0.000000,0.000000,0.000000}%
\pgfsetstrokecolor{currentstroke}%
\pgfsetdash{{1.000000pt}{3.000000pt}}{0.000000pt}%
\pgfpathmoveto{\pgfqpoint{1.250000in}{3.600000in}}%
\pgfpathlineto{\pgfqpoint{9.000000in}{3.600000in}}%
\pgfusepath{stroke}%
\end{pgfscope}%
\begin{pgfscope}%
\pgfsetbuttcap%
\pgfsetroundjoin%
\definecolor{currentfill}{rgb}{0.000000,0.000000,0.000000}%
\pgfsetfillcolor{currentfill}%
\pgfsetlinewidth{0.501875pt}%
\definecolor{currentstroke}{rgb}{0.000000,0.000000,0.000000}%
\pgfsetstrokecolor{currentstroke}%
\pgfsetdash{}{0pt}%
\pgfsys@defobject{currentmarker}{\pgfqpoint{0.000000in}{0.000000in}}{\pgfqpoint{0.055556in}{0.000000in}}{%
\pgfpathmoveto{\pgfqpoint{0.000000in}{0.000000in}}%
\pgfpathlineto{\pgfqpoint{0.055556in}{0.000000in}}%
\pgfusepath{stroke,fill}%
}%
\begin{pgfscope}%
\pgfsys@transformshift{1.250000in}{3.600000in}%
\pgfsys@useobject{currentmarker}{}%
\end{pgfscope}%
\end{pgfscope}%
\begin{pgfscope}%
\pgfsetbuttcap%
\pgfsetroundjoin%
\definecolor{currentfill}{rgb}{0.000000,0.000000,0.000000}%
\pgfsetfillcolor{currentfill}%
\pgfsetlinewidth{0.501875pt}%
\definecolor{currentstroke}{rgb}{0.000000,0.000000,0.000000}%
\pgfsetstrokecolor{currentstroke}%
\pgfsetdash{}{0pt}%
\pgfsys@defobject{currentmarker}{\pgfqpoint{-0.055556in}{0.000000in}}{\pgfqpoint{0.000000in}{0.000000in}}{%
\pgfpathmoveto{\pgfqpoint{0.000000in}{0.000000in}}%
\pgfpathlineto{\pgfqpoint{-0.055556in}{0.000000in}}%
\pgfusepath{stroke,fill}%
}%
\begin{pgfscope}%
\pgfsys@transformshift{9.000000in}{3.600000in}%
\pgfsys@useobject{currentmarker}{}%
\end{pgfscope}%
\end{pgfscope}%
\begin{pgfscope}%
\pgftext[left,bottom,x=0.876329in,y=3.536971in,rotate=0.000000]{{\sffamily\fontsize{12.000000}{14.400000}\selectfont 140}}
%
\end{pgfscope}%
\begin{pgfscope}%
\pgftext[left,bottom,x=0.773275in,y=1.740560in,rotate=90.000000]{{\sffamily\fontsize{12.000000}{14.400000}\selectfont y in m}}
%
\end{pgfscope}%
\begin{pgfscope}%
\pgfsetrectcap%
\pgfsetroundjoin%
\pgfsetlinewidth{1.003750pt}%
\definecolor{currentstroke}{rgb}{0.000000,0.000000,0.000000}%
\pgfsetstrokecolor{currentstroke}%
\pgfsetdash{}{0pt}%
\pgfpathmoveto{\pgfqpoint{1.250000in}{3.600000in}}%
\pgfpathlineto{\pgfqpoint{9.000000in}{3.600000in}}%
\pgfusepath{stroke}%
\end{pgfscope}%
\begin{pgfscope}%
\pgfsetrectcap%
\pgfsetroundjoin%
\pgfsetlinewidth{1.003750pt}%
\definecolor{currentstroke}{rgb}{0.000000,0.000000,0.000000}%
\pgfsetstrokecolor{currentstroke}%
\pgfsetdash{}{0pt}%
\pgfpathmoveto{\pgfqpoint{9.000000in}{0.400000in}}%
\pgfpathlineto{\pgfqpoint{9.000000in}{3.600000in}}%
\pgfusepath{stroke}%
\end{pgfscope}%
\begin{pgfscope}%
\pgfsetrectcap%
\pgfsetroundjoin%
\pgfsetlinewidth{1.003750pt}%
\definecolor{currentstroke}{rgb}{0.000000,0.000000,0.000000}%
\pgfsetstrokecolor{currentstroke}%
\pgfsetdash{}{0pt}%
\pgfpathmoveto{\pgfqpoint{1.250000in}{0.400000in}}%
\pgfpathlineto{\pgfqpoint{9.000000in}{0.400000in}}%
\pgfusepath{stroke}%
\end{pgfscope}%
\begin{pgfscope}%
\pgfsetrectcap%
\pgfsetroundjoin%
\pgfsetlinewidth{1.003750pt}%
\definecolor{currentstroke}{rgb}{0.000000,0.000000,0.000000}%
\pgfsetstrokecolor{currentstroke}%
\pgfsetdash{}{0pt}%
\pgfpathmoveto{\pgfqpoint{1.250000in}{0.400000in}}%
\pgfpathlineto{\pgfqpoint{1.250000in}{3.600000in}}%
\pgfusepath{stroke}%
\end{pgfscope}%
\begin{pgfscope}%
\pgftext[left,bottom,x=3.748096in,y=3.627843in,rotate=0.000000]{{\sffamily\fontsize{14.400000}{17.280000}\selectfont Trajectories of a cannonball}}
%
\end{pgfscope}%
\begin{pgfscope}%
\pgfsetrectcap%
\pgfsetroundjoin%
\definecolor{currentfill}{rgb}{1.000000,1.000000,1.000000}%
\pgfsetfillcolor{currentfill}%
\pgfsetlinewidth{1.003750pt}%
\definecolor{currentstroke}{rgb}{0.000000,0.000000,0.000000}%
\pgfsetstrokecolor{currentstroke}%
\pgfsetdash{}{0pt}%
\pgfpathmoveto{\pgfqpoint{6.658692in}{2.393913in}}%
\pgfpathlineto{\pgfqpoint{8.930583in}{2.393913in}}%
\pgfpathlineto{\pgfqpoint{8.930583in}{3.530583in}}%
\pgfpathlineto{\pgfqpoint{6.658692in}{3.530583in}}%
\pgfpathlineto{\pgfqpoint{6.658692in}{2.393913in}}%
\pgfpathclose%
\pgfusepath{stroke,fill}%
\end{pgfscope}%
\begin{pgfscope}%
\pgfsetrectcap%
\pgfsetroundjoin%
\pgfsetlinewidth{1.003750pt}%
\definecolor{currentstroke}{rgb}{1.000000,0.000000,0.000000}%
\pgfsetstrokecolor{currentstroke}%
\pgfsetdash{}{0pt}%
\pgfpathmoveto{\pgfqpoint{6.755875in}{3.323106in}}%
\pgfpathlineto{\pgfqpoint{6.950242in}{3.323106in}}%
\pgfusepath{stroke}%
\end{pgfscope}%
\begin{pgfscope}%
\pgfsetbuttcap%
\pgfsetroundjoin%
\definecolor{currentfill}{rgb}{1.000000,0.000000,0.000000}%
\pgfsetfillcolor{currentfill}%
\pgfsetlinewidth{0.501875pt}%
\definecolor{currentstroke}{rgb}{1.000000,0.000000,0.000000}%
\pgfsetstrokecolor{currentstroke}%
\pgfsetdash{}{0pt}%
\pgfsys@defobject{currentmarker}{\pgfqpoint{-0.041667in}{-0.041667in}}{\pgfqpoint{0.041667in}{0.041667in}}{%
\pgfpathmoveto{\pgfqpoint{-0.041667in}{0.000000in}}%
\pgfpathlineto{\pgfqpoint{0.041667in}{0.000000in}}%
\pgfpathmoveto{\pgfqpoint{0.000000in}{-0.041667in}}%
\pgfpathlineto{\pgfqpoint{0.000000in}{0.041667in}}%
\pgfusepath{stroke,fill}%
}%
\begin{pgfscope}%
\pgfsys@transformshift{6.755875in}{3.323106in}%
\pgfsys@useobject{currentmarker}{}%
\end{pgfscope}%
\begin{pgfscope}%
\pgfsys@transformshift{6.950242in}{3.323106in}%
\pgfsys@useobject{currentmarker}{}%
\end{pgfscope}%
\end{pgfscope}%
\begin{pgfscope}%
\pgftext[left,bottom,x=7.102958in,y=3.367603in,rotate=0.000000]{{\sffamily\fontsize{9.996000}{11.995200}\selectfont no friction }}
%
\end{pgfscope}%
\begin{pgfscope}%
\pgftext[left,bottom,x=7.102958in,y=3.179460in,rotate=0.000000]{{\sffamily\fontsize{9.996000}{11.995200}\selectfont  flying time: 10.100000 s}}
%
\end{pgfscope}%
\begin{pgfscope}%
\pgfsetrectcap%
\pgfsetroundjoin%
\pgfsetlinewidth{1.003750pt}%
\definecolor{currentstroke}{rgb}{0.000000,0.000000,1.000000}%
\pgfsetstrokecolor{currentstroke}%
\pgfsetdash{}{0pt}%
\pgfpathmoveto{\pgfqpoint{6.755875in}{2.958099in}}%
\pgfpathlineto{\pgfqpoint{6.950242in}{2.958099in}}%
\pgfusepath{stroke}%
\end{pgfscope}%
\begin{pgfscope}%
\pgfsetbuttcap%
\pgfsetroundjoin%
\definecolor{currentfill}{rgb}{0.000000,0.000000,1.000000}%
\pgfsetfillcolor{currentfill}%
\pgfsetlinewidth{0.501875pt}%
\definecolor{currentstroke}{rgb}{0.000000,0.000000,1.000000}%
\pgfsetstrokecolor{currentstroke}%
\pgfsetdash{}{0pt}%
\pgfsys@defobject{currentmarker}{\pgfqpoint{-0.041667in}{-0.041667in}}{\pgfqpoint{0.041667in}{0.041667in}}{%
\pgfpathmoveto{\pgfqpoint{-0.041667in}{0.000000in}}%
\pgfpathlineto{\pgfqpoint{0.041667in}{0.000000in}}%
\pgfpathmoveto{\pgfqpoint{0.000000in}{-0.041667in}}%
\pgfpathlineto{\pgfqpoint{0.000000in}{0.041667in}}%
\pgfusepath{stroke,fill}%
}%
\begin{pgfscope}%
\pgfsys@transformshift{6.755875in}{2.958099in}%
\pgfsys@useobject{currentmarker}{}%
\end{pgfscope}%
\begin{pgfscope}%
\pgfsys@transformshift{6.950242in}{2.958099in}%
\pgfsys@useobject{currentmarker}{}%
\end{pgfscope}%
\end{pgfscope}%
\begin{pgfscope}%
\pgftext[left,bottom,x=7.102958in,y=2.988428in,rotate=0.000000]{{\sffamily\fontsize{9.996000}{11.995200}\selectfont with friction, no wind }}
%
\end{pgfscope}%
\begin{pgfscope}%
\pgftext[left,bottom,x=7.102958in,y=2.814453in,rotate=0.000000]{{\sffamily\fontsize{9.996000}{11.995200}\selectfont  flying time: 9.400000 s}}
%
\end{pgfscope}%
\begin{pgfscope}%
\pgfsetrectcap%
\pgfsetroundjoin%
\pgfsetlinewidth{1.003750pt}%
\definecolor{currentstroke}{rgb}{0.000000,0.500000,0.000000}%
\pgfsetstrokecolor{currentstroke}%
\pgfsetdash{}{0pt}%
\pgfpathmoveto{\pgfqpoint{6.755875in}{2.593093in}}%
\pgfpathlineto{\pgfqpoint{6.950242in}{2.593093in}}%
\pgfusepath{stroke}%
\end{pgfscope}%
\begin{pgfscope}%
\pgfsetbuttcap%
\pgfsetroundjoin%
\definecolor{currentfill}{rgb}{0.000000,0.500000,0.000000}%
\pgfsetfillcolor{currentfill}%
\pgfsetlinewidth{0.501875pt}%
\definecolor{currentstroke}{rgb}{0.000000,0.500000,0.000000}%
\pgfsetstrokecolor{currentstroke}%
\pgfsetdash{}{0pt}%
\pgfsys@defobject{currentmarker}{\pgfqpoint{-0.041667in}{-0.041667in}}{\pgfqpoint{0.041667in}{0.041667in}}{%
\pgfpathmoveto{\pgfqpoint{-0.041667in}{0.000000in}}%
\pgfpathlineto{\pgfqpoint{0.041667in}{0.000000in}}%
\pgfpathmoveto{\pgfqpoint{0.000000in}{-0.041667in}}%
\pgfpathlineto{\pgfqpoint{0.000000in}{0.041667in}}%
\pgfusepath{stroke,fill}%
}%
\begin{pgfscope}%
\pgfsys@transformshift{6.755875in}{2.593093in}%
\pgfsys@useobject{currentmarker}{}%
\end{pgfscope}%
\begin{pgfscope}%
\pgfsys@transformshift{6.950242in}{2.593093in}%
\pgfsys@useobject{currentmarker}{}%
\end{pgfscope}%
\end{pgfscope}%
\begin{pgfscope}%
\pgftext[left,bottom,x=7.102958in,y=2.610677in,rotate=0.000000]{{\sffamily\fontsize{9.996000}{11.995200}\selectfont with friction, strong wind }}
%
\end{pgfscope}%
\begin{pgfscope}%
\pgftext[left,bottom,x=7.102958in,y=2.449446in,rotate=0.000000]{{\sffamily\fontsize{9.996000}{11.995200}\selectfont  flying time: 9.400000 s}}
%
\end{pgfscope}%
\end{pgfpicture}%
\makeatother%
\endgroup%
}
	\caption{Trajectories}\label{fig:c1}
\end{figure}

As a result we get a parabolic trajectory (see figure \ref{fig:c1}) if we neglect fricition. This meets one's expectations. The question what happens with friction is much more interesting because of the lack of a analytic solution. It's also possible to examine the influence of wind. 

\begin{figure}[H]
	\resizebox{1\textwidth}{!}{%% Creator: Matplotlib, PGF backend
%%
%% To include the figure in your LaTeX document, write
%%   \input{<filename>.pgf}
%%
%% Make sure the required packages are loaded in your preamble
%%   \usepackage{pgf}
%%
%% Figures using additional raster images can only be included by \input if
%% they are in the same directory as the main LaTeX file. For loading figures
%% from other directories you can use the `import` package
%%   \usepackage{import}
%% and then include the figures with
%%   \import{<path to file>}{<filename>.pgf}
%%
%% Matplotlib used the following preamble
%%   \usepackage{fontspec}
%%   \setmainfont{DejaVu Serif}
%%   \setsansfont{DejaVu Sans}
%%   \setmonofont{DejaVu Sans Mono}
%%
\begingroup%
\makeatletter%
\begin{pgfpicture}%
\pgfpathrectangle{\pgfpointorigin}{\pgfqpoint{10.000000in}{4.000000in}}%
\pgfusepath{use as bounding box}%
\begin{pgfscope}%
\pgfsetrectcap%
\pgfsetroundjoin%
\definecolor{currentfill}{rgb}{1.000000,1.000000,1.000000}%
\pgfsetfillcolor{currentfill}%
\pgfsetlinewidth{0.000000pt}%
\definecolor{currentstroke}{rgb}{1.000000,1.000000,1.000000}%
\pgfsetstrokecolor{currentstroke}%
\pgfsetdash{}{0pt}%
\pgfpathmoveto{\pgfqpoint{0.000000in}{0.000000in}}%
\pgfpathlineto{\pgfqpoint{10.000000in}{0.000000in}}%
\pgfpathlineto{\pgfqpoint{10.000000in}{4.000000in}}%
\pgfpathlineto{\pgfqpoint{0.000000in}{4.000000in}}%
\pgfpathclose%
\pgfusepath{fill}%
\end{pgfscope}%
\begin{pgfscope}%
\pgfsetrectcap%
\pgfsetroundjoin%
\definecolor{currentfill}{rgb}{1.000000,1.000000,1.000000}%
\pgfsetfillcolor{currentfill}%
\pgfsetlinewidth{0.000000pt}%
\definecolor{currentstroke}{rgb}{0.000000,0.000000,0.000000}%
\pgfsetstrokecolor{currentstroke}%
\pgfsetdash{}{0pt}%
\pgfpathmoveto{\pgfqpoint{1.250000in}{0.400000in}}%
\pgfpathlineto{\pgfqpoint{9.000000in}{0.400000in}}%
\pgfpathlineto{\pgfqpoint{9.000000in}{3.600000in}}%
\pgfpathlineto{\pgfqpoint{1.250000in}{3.600000in}}%
\pgfpathclose%
\pgfusepath{fill}%
\end{pgfscope}%
\begin{pgfscope}%
\pgfpathrectangle{\pgfqpoint{1.250000in}{0.400000in}}{\pgfqpoint{7.750000in}{3.200000in}} %
\pgfusepath{clip}%
\pgfsetrectcap%
\pgfsetroundjoin%
\pgfsetlinewidth{1.003750pt}%
\definecolor{currentstroke}{rgb}{0.000000,0.000000,1.000000}%
\pgfsetstrokecolor{currentstroke}%
\pgfsetdash{}{0pt}%
\pgfpathmoveto{\pgfqpoint{4.571429in}{0.857143in}}%
\pgfpathlineto{\pgfqpoint{4.624848in}{0.968615in}}%
\pgfpathlineto{\pgfqpoint{4.676340in}{1.077287in}}%
\pgfpathlineto{\pgfqpoint{4.725914in}{1.183174in}}%
\pgfpathlineto{\pgfqpoint{4.773579in}{1.286289in}}%
\pgfpathlineto{\pgfqpoint{4.819345in}{1.386646in}}%
\pgfpathlineto{\pgfqpoint{4.863221in}{1.484259in}}%
\pgfpathlineto{\pgfqpoint{4.905218in}{1.579142in}}%
\pgfpathlineto{\pgfqpoint{4.945343in}{1.671308in}}%
\pgfpathlineto{\pgfqpoint{4.983608in}{1.760771in}}%
\pgfpathlineto{\pgfqpoint{5.020020in}{1.847544in}}%
\pgfpathlineto{\pgfqpoint{5.054590in}{1.931641in}}%
\pgfpathlineto{\pgfqpoint{5.087326in}{2.013075in}}%
\pgfpathlineto{\pgfqpoint{5.118237in}{2.091860in}}%
\pgfpathlineto{\pgfqpoint{5.147333in}{2.168009in}}%
\pgfpathlineto{\pgfqpoint{5.174624in}{2.241535in}}%
\pgfpathlineto{\pgfqpoint{5.200117in}{2.312451in}}%
\pgfpathlineto{\pgfqpoint{5.223821in}{2.380769in}}%
\pgfpathlineto{\pgfqpoint{5.245747in}{2.446504in}}%
\pgfpathlineto{\pgfqpoint{5.265902in}{2.509668in}}%
\pgfpathlineto{\pgfqpoint{5.284296in}{2.570274in}}%
\pgfpathlineto{\pgfqpoint{5.300937in}{2.628335in}}%
\pgfpathlineto{\pgfqpoint{5.315834in}{2.683863in}}%
\pgfpathlineto{\pgfqpoint{5.328996in}{2.736871in}}%
\pgfpathlineto{\pgfqpoint{5.340431in}{2.787372in}}%
\pgfpathlineto{\pgfqpoint{5.350149in}{2.835378in}}%
\pgfpathlineto{\pgfqpoint{5.358157in}{2.880902in}}%
\pgfpathlineto{\pgfqpoint{5.364464in}{2.923955in}}%
\pgfpathlineto{\pgfqpoint{5.369079in}{2.964552in}}%
\pgfpathlineto{\pgfqpoint{5.372011in}{3.002703in}}%
\pgfpathlineto{\pgfqpoint{5.373267in}{3.038421in}}%
\pgfpathlineto{\pgfqpoint{5.372856in}{3.071718in}}%
\pgfpathlineto{\pgfqpoint{5.370786in}{3.102606in}}%
\pgfpathlineto{\pgfqpoint{5.367066in}{3.131097in}}%
\pgfpathlineto{\pgfqpoint{5.361704in}{3.157204in}}%
\pgfpathlineto{\pgfqpoint{5.354708in}{3.180938in}}%
\pgfpathlineto{\pgfqpoint{5.346086in}{3.202311in}}%
\pgfpathlineto{\pgfqpoint{5.335847in}{3.221335in}}%
\pgfpathlineto{\pgfqpoint{5.323998in}{3.238022in}}%
\pgfpathlineto{\pgfqpoint{5.310548in}{3.252382in}}%
\pgfpathlineto{\pgfqpoint{5.295504in}{3.264429in}}%
\pgfpathlineto{\pgfqpoint{5.278875in}{3.274173in}}%
\pgfpathlineto{\pgfqpoint{5.260668in}{3.281626in}}%
\pgfpathlineto{\pgfqpoint{5.240891in}{3.286800in}}%
\pgfpathlineto{\pgfqpoint{5.219553in}{3.289705in}}%
\pgfpathlineto{\pgfqpoint{5.196660in}{3.290354in}}%
\pgfpathlineto{\pgfqpoint{5.172222in}{3.288757in}}%
\pgfpathlineto{\pgfqpoint{5.146245in}{3.284926in}}%
\pgfpathlineto{\pgfqpoint{5.118737in}{3.278871in}}%
\pgfpathlineto{\pgfqpoint{5.089705in}{3.270605in}}%
\pgfpathlineto{\pgfqpoint{5.059159in}{3.260138in}}%
\pgfpathlineto{\pgfqpoint{5.027104in}{3.247480in}}%
\pgfpathlineto{\pgfqpoint{4.993549in}{3.232644in}}%
\pgfpathlineto{\pgfqpoint{4.958501in}{3.215640in}}%
\pgfpathlineto{\pgfqpoint{4.921967in}{3.196478in}}%
\pgfpathlineto{\pgfqpoint{4.883955in}{3.175170in}}%
\pgfpathlineto{\pgfqpoint{4.844473in}{3.151726in}}%
\pgfpathlineto{\pgfqpoint{4.803528in}{3.126157in}}%
\pgfpathlineto{\pgfqpoint{4.761126in}{3.098474in}}%
\pgfpathlineto{\pgfqpoint{4.717276in}{3.068687in}}%
\pgfpathlineto{\pgfqpoint{4.671984in}{3.036806in}}%
\pgfpathlineto{\pgfqpoint{4.625258in}{3.002843in}}%
\pgfpathlineto{\pgfqpoint{4.577105in}{2.966807in}}%
\pgfpathlineto{\pgfqpoint{4.527532in}{2.928709in}}%
\pgfpathlineto{\pgfqpoint{4.476546in}{2.888559in}}%
\pgfpathlineto{\pgfqpoint{4.424154in}{2.846368in}}%
\pgfpathlineto{\pgfqpoint{4.370364in}{2.802145in}}%
\pgfpathlineto{\pgfqpoint{4.315182in}{2.755901in}}%
\pgfpathlineto{\pgfqpoint{4.258615in}{2.707646in}}%
\pgfpathlineto{\pgfqpoint{4.200670in}{2.657390in}}%
\pgfpathlineto{\pgfqpoint{4.141354in}{2.605143in}}%
\pgfpathlineto{\pgfqpoint{4.080674in}{2.550915in}}%
\pgfpathlineto{\pgfqpoint{4.018637in}{2.494716in}}%
\pgfpathlineto{\pgfqpoint{3.955249in}{2.436556in}}%
\pgfpathlineto{\pgfqpoint{3.890517in}{2.376444in}}%
\pgfpathlineto{\pgfqpoint{3.824449in}{2.314390in}}%
\pgfpathlineto{\pgfqpoint{3.757050in}{2.250404in}}%
\pgfpathlineto{\pgfqpoint{3.688327in}{2.184496in}}%
\pgfpathlineto{\pgfqpoint{3.618287in}{2.116676in}}%
\pgfpathlineto{\pgfqpoint{3.546937in}{2.046952in}}%
\pgfpathlineto{\pgfqpoint{3.474282in}{1.975334in}}%
\pgfpathlineto{\pgfqpoint{3.400331in}{1.901832in}}%
\pgfpathlineto{\pgfqpoint{3.325088in}{1.826456in}}%
\pgfpathlineto{\pgfqpoint{3.248561in}{1.749214in}}%
\pgfpathlineto{\pgfqpoint{3.170755in}{1.670116in}}%
\pgfpathlineto{\pgfqpoint{3.091678in}{1.589171in}}%
\pgfpathlineto{\pgfqpoint{3.011336in}{1.506388in}}%
\pgfpathlineto{\pgfqpoint{2.929735in}{1.421777in}}%
\pgfpathlineto{\pgfqpoint{2.846881in}{1.335347in}}%
\pgfpathlineto{\pgfqpoint{2.762780in}{1.247107in}}%
\pgfpathlineto{\pgfqpoint{2.677440in}{1.157066in}}%
\pgfpathlineto{\pgfqpoint{2.590865in}{1.065233in}}%
\pgfpathlineto{\pgfqpoint{2.503062in}{0.971616in}}%
\pgfpathlineto{\pgfqpoint{2.414038in}{0.876226in}}%
\pgfpathlineto{\pgfqpoint{2.323798in}{0.779070in}}%
\pgfusepath{stroke}%
\end{pgfscope}%
\begin{pgfscope}%
\pgfpathrectangle{\pgfqpoint{1.250000in}{0.400000in}}{\pgfqpoint{7.750000in}{3.200000in}} %
\pgfusepath{clip}%
\pgfsetbuttcap%
\pgfsetroundjoin%
\definecolor{currentfill}{rgb}{0.000000,0.000000,1.000000}%
\pgfsetfillcolor{currentfill}%
\pgfsetlinewidth{0.501875pt}%
\definecolor{currentstroke}{rgb}{0.000000,0.000000,1.000000}%
\pgfsetstrokecolor{currentstroke}%
\pgfsetdash{}{0pt}%
\pgfsys@defobject{currentmarker}{\pgfqpoint{-0.041667in}{-0.041667in}}{\pgfqpoint{0.041667in}{0.041667in}}{%
\pgfpathmoveto{\pgfqpoint{-0.041667in}{0.000000in}}%
\pgfpathlineto{\pgfqpoint{0.041667in}{0.000000in}}%
\pgfpathmoveto{\pgfqpoint{0.000000in}{-0.041667in}}%
\pgfpathlineto{\pgfqpoint{0.000000in}{0.041667in}}%
\pgfusepath{stroke,fill}%
}%
\begin{pgfscope}%
\pgfsys@transformshift{4.571429in}{0.857143in}%
\pgfsys@useobject{currentmarker}{}%
\end{pgfscope}%
\begin{pgfscope}%
\pgfsys@transformshift{4.624848in}{0.968615in}%
\pgfsys@useobject{currentmarker}{}%
\end{pgfscope}%
\begin{pgfscope}%
\pgfsys@transformshift{4.676340in}{1.077287in}%
\pgfsys@useobject{currentmarker}{}%
\end{pgfscope}%
\begin{pgfscope}%
\pgfsys@transformshift{4.725914in}{1.183174in}%
\pgfsys@useobject{currentmarker}{}%
\end{pgfscope}%
\begin{pgfscope}%
\pgfsys@transformshift{4.773579in}{1.286289in}%
\pgfsys@useobject{currentmarker}{}%
\end{pgfscope}%
\begin{pgfscope}%
\pgfsys@transformshift{4.819345in}{1.386646in}%
\pgfsys@useobject{currentmarker}{}%
\end{pgfscope}%
\begin{pgfscope}%
\pgfsys@transformshift{4.863221in}{1.484259in}%
\pgfsys@useobject{currentmarker}{}%
\end{pgfscope}%
\begin{pgfscope}%
\pgfsys@transformshift{4.905218in}{1.579142in}%
\pgfsys@useobject{currentmarker}{}%
\end{pgfscope}%
\begin{pgfscope}%
\pgfsys@transformshift{4.945343in}{1.671308in}%
\pgfsys@useobject{currentmarker}{}%
\end{pgfscope}%
\begin{pgfscope}%
\pgfsys@transformshift{4.983608in}{1.760771in}%
\pgfsys@useobject{currentmarker}{}%
\end{pgfscope}%
\begin{pgfscope}%
\pgfsys@transformshift{5.020020in}{1.847544in}%
\pgfsys@useobject{currentmarker}{}%
\end{pgfscope}%
\begin{pgfscope}%
\pgfsys@transformshift{5.054590in}{1.931641in}%
\pgfsys@useobject{currentmarker}{}%
\end{pgfscope}%
\begin{pgfscope}%
\pgfsys@transformshift{5.087326in}{2.013075in}%
\pgfsys@useobject{currentmarker}{}%
\end{pgfscope}%
\begin{pgfscope}%
\pgfsys@transformshift{5.118237in}{2.091860in}%
\pgfsys@useobject{currentmarker}{}%
\end{pgfscope}%
\begin{pgfscope}%
\pgfsys@transformshift{5.147333in}{2.168009in}%
\pgfsys@useobject{currentmarker}{}%
\end{pgfscope}%
\begin{pgfscope}%
\pgfsys@transformshift{5.174624in}{2.241535in}%
\pgfsys@useobject{currentmarker}{}%
\end{pgfscope}%
\begin{pgfscope}%
\pgfsys@transformshift{5.200117in}{2.312451in}%
\pgfsys@useobject{currentmarker}{}%
\end{pgfscope}%
\begin{pgfscope}%
\pgfsys@transformshift{5.223821in}{2.380769in}%
\pgfsys@useobject{currentmarker}{}%
\end{pgfscope}%
\begin{pgfscope}%
\pgfsys@transformshift{5.245747in}{2.446504in}%
\pgfsys@useobject{currentmarker}{}%
\end{pgfscope}%
\begin{pgfscope}%
\pgfsys@transformshift{5.265902in}{2.509668in}%
\pgfsys@useobject{currentmarker}{}%
\end{pgfscope}%
\begin{pgfscope}%
\pgfsys@transformshift{5.284296in}{2.570274in}%
\pgfsys@useobject{currentmarker}{}%
\end{pgfscope}%
\begin{pgfscope}%
\pgfsys@transformshift{5.300937in}{2.628335in}%
\pgfsys@useobject{currentmarker}{}%
\end{pgfscope}%
\begin{pgfscope}%
\pgfsys@transformshift{5.315834in}{2.683863in}%
\pgfsys@useobject{currentmarker}{}%
\end{pgfscope}%
\begin{pgfscope}%
\pgfsys@transformshift{5.328996in}{2.736871in}%
\pgfsys@useobject{currentmarker}{}%
\end{pgfscope}%
\begin{pgfscope}%
\pgfsys@transformshift{5.340431in}{2.787372in}%
\pgfsys@useobject{currentmarker}{}%
\end{pgfscope}%
\begin{pgfscope}%
\pgfsys@transformshift{5.350149in}{2.835378in}%
\pgfsys@useobject{currentmarker}{}%
\end{pgfscope}%
\begin{pgfscope}%
\pgfsys@transformshift{5.358157in}{2.880902in}%
\pgfsys@useobject{currentmarker}{}%
\end{pgfscope}%
\begin{pgfscope}%
\pgfsys@transformshift{5.364464in}{2.923955in}%
\pgfsys@useobject{currentmarker}{}%
\end{pgfscope}%
\begin{pgfscope}%
\pgfsys@transformshift{5.369079in}{2.964552in}%
\pgfsys@useobject{currentmarker}{}%
\end{pgfscope}%
\begin{pgfscope}%
\pgfsys@transformshift{5.372011in}{3.002703in}%
\pgfsys@useobject{currentmarker}{}%
\end{pgfscope}%
\begin{pgfscope}%
\pgfsys@transformshift{5.373267in}{3.038421in}%
\pgfsys@useobject{currentmarker}{}%
\end{pgfscope}%
\begin{pgfscope}%
\pgfsys@transformshift{5.372856in}{3.071718in}%
\pgfsys@useobject{currentmarker}{}%
\end{pgfscope}%
\begin{pgfscope}%
\pgfsys@transformshift{5.370786in}{3.102606in}%
\pgfsys@useobject{currentmarker}{}%
\end{pgfscope}%
\begin{pgfscope}%
\pgfsys@transformshift{5.367066in}{3.131097in}%
\pgfsys@useobject{currentmarker}{}%
\end{pgfscope}%
\begin{pgfscope}%
\pgfsys@transformshift{5.361704in}{3.157204in}%
\pgfsys@useobject{currentmarker}{}%
\end{pgfscope}%
\begin{pgfscope}%
\pgfsys@transformshift{5.354708in}{3.180938in}%
\pgfsys@useobject{currentmarker}{}%
\end{pgfscope}%
\begin{pgfscope}%
\pgfsys@transformshift{5.346086in}{3.202311in}%
\pgfsys@useobject{currentmarker}{}%
\end{pgfscope}%
\begin{pgfscope}%
\pgfsys@transformshift{5.335847in}{3.221335in}%
\pgfsys@useobject{currentmarker}{}%
\end{pgfscope}%
\begin{pgfscope}%
\pgfsys@transformshift{5.323998in}{3.238022in}%
\pgfsys@useobject{currentmarker}{}%
\end{pgfscope}%
\begin{pgfscope}%
\pgfsys@transformshift{5.310548in}{3.252382in}%
\pgfsys@useobject{currentmarker}{}%
\end{pgfscope}%
\begin{pgfscope}%
\pgfsys@transformshift{5.295504in}{3.264429in}%
\pgfsys@useobject{currentmarker}{}%
\end{pgfscope}%
\begin{pgfscope}%
\pgfsys@transformshift{5.278875in}{3.274173in}%
\pgfsys@useobject{currentmarker}{}%
\end{pgfscope}%
\begin{pgfscope}%
\pgfsys@transformshift{5.260668in}{3.281626in}%
\pgfsys@useobject{currentmarker}{}%
\end{pgfscope}%
\begin{pgfscope}%
\pgfsys@transformshift{5.240891in}{3.286800in}%
\pgfsys@useobject{currentmarker}{}%
\end{pgfscope}%
\begin{pgfscope}%
\pgfsys@transformshift{5.219553in}{3.289705in}%
\pgfsys@useobject{currentmarker}{}%
\end{pgfscope}%
\begin{pgfscope}%
\pgfsys@transformshift{5.196660in}{3.290354in}%
\pgfsys@useobject{currentmarker}{}%
\end{pgfscope}%
\begin{pgfscope}%
\pgfsys@transformshift{5.172222in}{3.288757in}%
\pgfsys@useobject{currentmarker}{}%
\end{pgfscope}%
\begin{pgfscope}%
\pgfsys@transformshift{5.146245in}{3.284926in}%
\pgfsys@useobject{currentmarker}{}%
\end{pgfscope}%
\begin{pgfscope}%
\pgfsys@transformshift{5.118737in}{3.278871in}%
\pgfsys@useobject{currentmarker}{}%
\end{pgfscope}%
\begin{pgfscope}%
\pgfsys@transformshift{5.089705in}{3.270605in}%
\pgfsys@useobject{currentmarker}{}%
\end{pgfscope}%
\begin{pgfscope}%
\pgfsys@transformshift{5.059159in}{3.260138in}%
\pgfsys@useobject{currentmarker}{}%
\end{pgfscope}%
\begin{pgfscope}%
\pgfsys@transformshift{5.027104in}{3.247480in}%
\pgfsys@useobject{currentmarker}{}%
\end{pgfscope}%
\begin{pgfscope}%
\pgfsys@transformshift{4.993549in}{3.232644in}%
\pgfsys@useobject{currentmarker}{}%
\end{pgfscope}%
\begin{pgfscope}%
\pgfsys@transformshift{4.958501in}{3.215640in}%
\pgfsys@useobject{currentmarker}{}%
\end{pgfscope}%
\begin{pgfscope}%
\pgfsys@transformshift{4.921967in}{3.196478in}%
\pgfsys@useobject{currentmarker}{}%
\end{pgfscope}%
\begin{pgfscope}%
\pgfsys@transformshift{4.883955in}{3.175170in}%
\pgfsys@useobject{currentmarker}{}%
\end{pgfscope}%
\begin{pgfscope}%
\pgfsys@transformshift{4.844473in}{3.151726in}%
\pgfsys@useobject{currentmarker}{}%
\end{pgfscope}%
\begin{pgfscope}%
\pgfsys@transformshift{4.803528in}{3.126157in}%
\pgfsys@useobject{currentmarker}{}%
\end{pgfscope}%
\begin{pgfscope}%
\pgfsys@transformshift{4.761126in}{3.098474in}%
\pgfsys@useobject{currentmarker}{}%
\end{pgfscope}%
\begin{pgfscope}%
\pgfsys@transformshift{4.717276in}{3.068687in}%
\pgfsys@useobject{currentmarker}{}%
\end{pgfscope}%
\begin{pgfscope}%
\pgfsys@transformshift{4.671984in}{3.036806in}%
\pgfsys@useobject{currentmarker}{}%
\end{pgfscope}%
\begin{pgfscope}%
\pgfsys@transformshift{4.625258in}{3.002843in}%
\pgfsys@useobject{currentmarker}{}%
\end{pgfscope}%
\begin{pgfscope}%
\pgfsys@transformshift{4.577105in}{2.966807in}%
\pgfsys@useobject{currentmarker}{}%
\end{pgfscope}%
\begin{pgfscope}%
\pgfsys@transformshift{4.527532in}{2.928709in}%
\pgfsys@useobject{currentmarker}{}%
\end{pgfscope}%
\begin{pgfscope}%
\pgfsys@transformshift{4.476546in}{2.888559in}%
\pgfsys@useobject{currentmarker}{}%
\end{pgfscope}%
\begin{pgfscope}%
\pgfsys@transformshift{4.424154in}{2.846368in}%
\pgfsys@useobject{currentmarker}{}%
\end{pgfscope}%
\begin{pgfscope}%
\pgfsys@transformshift{4.370364in}{2.802145in}%
\pgfsys@useobject{currentmarker}{}%
\end{pgfscope}%
\begin{pgfscope}%
\pgfsys@transformshift{4.315182in}{2.755901in}%
\pgfsys@useobject{currentmarker}{}%
\end{pgfscope}%
\begin{pgfscope}%
\pgfsys@transformshift{4.258615in}{2.707646in}%
\pgfsys@useobject{currentmarker}{}%
\end{pgfscope}%
\begin{pgfscope}%
\pgfsys@transformshift{4.200670in}{2.657390in}%
\pgfsys@useobject{currentmarker}{}%
\end{pgfscope}%
\begin{pgfscope}%
\pgfsys@transformshift{4.141354in}{2.605143in}%
\pgfsys@useobject{currentmarker}{}%
\end{pgfscope}%
\begin{pgfscope}%
\pgfsys@transformshift{4.080674in}{2.550915in}%
\pgfsys@useobject{currentmarker}{}%
\end{pgfscope}%
\begin{pgfscope}%
\pgfsys@transformshift{4.018637in}{2.494716in}%
\pgfsys@useobject{currentmarker}{}%
\end{pgfscope}%
\begin{pgfscope}%
\pgfsys@transformshift{3.955249in}{2.436556in}%
\pgfsys@useobject{currentmarker}{}%
\end{pgfscope}%
\begin{pgfscope}%
\pgfsys@transformshift{3.890517in}{2.376444in}%
\pgfsys@useobject{currentmarker}{}%
\end{pgfscope}%
\begin{pgfscope}%
\pgfsys@transformshift{3.824449in}{2.314390in}%
\pgfsys@useobject{currentmarker}{}%
\end{pgfscope}%
\begin{pgfscope}%
\pgfsys@transformshift{3.757050in}{2.250404in}%
\pgfsys@useobject{currentmarker}{}%
\end{pgfscope}%
\begin{pgfscope}%
\pgfsys@transformshift{3.688327in}{2.184496in}%
\pgfsys@useobject{currentmarker}{}%
\end{pgfscope}%
\begin{pgfscope}%
\pgfsys@transformshift{3.618287in}{2.116676in}%
\pgfsys@useobject{currentmarker}{}%
\end{pgfscope}%
\begin{pgfscope}%
\pgfsys@transformshift{3.546937in}{2.046952in}%
\pgfsys@useobject{currentmarker}{}%
\end{pgfscope}%
\begin{pgfscope}%
\pgfsys@transformshift{3.474282in}{1.975334in}%
\pgfsys@useobject{currentmarker}{}%
\end{pgfscope}%
\begin{pgfscope}%
\pgfsys@transformshift{3.400331in}{1.901832in}%
\pgfsys@useobject{currentmarker}{}%
\end{pgfscope}%
\begin{pgfscope}%
\pgfsys@transformshift{3.325088in}{1.826456in}%
\pgfsys@useobject{currentmarker}{}%
\end{pgfscope}%
\begin{pgfscope}%
\pgfsys@transformshift{3.248561in}{1.749214in}%
\pgfsys@useobject{currentmarker}{}%
\end{pgfscope}%
\begin{pgfscope}%
\pgfsys@transformshift{3.170755in}{1.670116in}%
\pgfsys@useobject{currentmarker}{}%
\end{pgfscope}%
\begin{pgfscope}%
\pgfsys@transformshift{3.091678in}{1.589171in}%
\pgfsys@useobject{currentmarker}{}%
\end{pgfscope}%
\begin{pgfscope}%
\pgfsys@transformshift{3.011336in}{1.506388in}%
\pgfsys@useobject{currentmarker}{}%
\end{pgfscope}%
\begin{pgfscope}%
\pgfsys@transformshift{2.929735in}{1.421777in}%
\pgfsys@useobject{currentmarker}{}%
\end{pgfscope}%
\begin{pgfscope}%
\pgfsys@transformshift{2.846881in}{1.335347in}%
\pgfsys@useobject{currentmarker}{}%
\end{pgfscope}%
\begin{pgfscope}%
\pgfsys@transformshift{2.762780in}{1.247107in}%
\pgfsys@useobject{currentmarker}{}%
\end{pgfscope}%
\begin{pgfscope}%
\pgfsys@transformshift{2.677440in}{1.157066in}%
\pgfsys@useobject{currentmarker}{}%
\end{pgfscope}%
\begin{pgfscope}%
\pgfsys@transformshift{2.590865in}{1.065233in}%
\pgfsys@useobject{currentmarker}{}%
\end{pgfscope}%
\begin{pgfscope}%
\pgfsys@transformshift{2.503062in}{0.971616in}%
\pgfsys@useobject{currentmarker}{}%
\end{pgfscope}%
\begin{pgfscope}%
\pgfsys@transformshift{2.414038in}{0.876226in}%
\pgfsys@useobject{currentmarker}{}%
\end{pgfscope}%
\begin{pgfscope}%
\pgfsys@transformshift{2.323798in}{0.779070in}%
\pgfsys@useobject{currentmarker}{}%
\end{pgfscope}%
\end{pgfscope}%
\begin{pgfscope}%
\pgfpathrectangle{\pgfqpoint{1.250000in}{0.400000in}}{\pgfqpoint{7.750000in}{3.200000in}} %
\pgfusepath{clip}%
\pgfsetrectcap%
\pgfsetroundjoin%
\pgfsetlinewidth{1.003750pt}%
\definecolor{currentstroke}{rgb}{0.000000,0.500000,0.000000}%
\pgfsetstrokecolor{currentstroke}%
\pgfsetdash{}{0pt}%
\pgfpathmoveto{\pgfqpoint{4.571429in}{0.857143in}}%
\pgfpathlineto{\pgfqpoint{4.625039in}{0.968615in}}%
\pgfpathlineto{\pgfqpoint{4.676911in}{1.077287in}}%
\pgfpathlineto{\pgfqpoint{4.727054in}{1.183174in}}%
\pgfpathlineto{\pgfqpoint{4.775476in}{1.286289in}}%
\pgfpathlineto{\pgfqpoint{4.822186in}{1.386646in}}%
\pgfpathlineto{\pgfqpoint{4.867192in}{1.484259in}}%
\pgfpathlineto{\pgfqpoint{4.910504in}{1.579142in}}%
\pgfpathlineto{\pgfqpoint{4.952128in}{1.671308in}}%
\pgfpathlineto{\pgfqpoint{4.992075in}{1.760771in}}%
\pgfpathlineto{\pgfqpoint{5.030351in}{1.847544in}}%
\pgfpathlineto{\pgfqpoint{5.066967in}{1.931641in}}%
\pgfpathlineto{\pgfqpoint{5.101929in}{2.013075in}}%
\pgfpathlineto{\pgfqpoint{5.135246in}{2.091860in}}%
\pgfpathlineto{\pgfqpoint{5.166927in}{2.168009in}}%
\pgfpathlineto{\pgfqpoint{5.196979in}{2.241535in}}%
\pgfpathlineto{\pgfqpoint{5.225411in}{2.312451in}}%
\pgfpathlineto{\pgfqpoint{5.252231in}{2.380769in}}%
\pgfpathlineto{\pgfqpoint{5.277447in}{2.446504in}}%
\pgfpathlineto{\pgfqpoint{5.301066in}{2.509668in}}%
\pgfpathlineto{\pgfqpoint{5.323098in}{2.570274in}}%
\pgfpathlineto{\pgfqpoint{5.343549in}{2.628335in}}%
\pgfpathlineto{\pgfqpoint{5.362428in}{2.683863in}}%
\pgfpathlineto{\pgfqpoint{5.379742in}{2.736871in}}%
\pgfpathlineto{\pgfqpoint{5.395500in}{2.787372in}}%
\pgfpathlineto{\pgfqpoint{5.409709in}{2.835378in}}%
\pgfpathlineto{\pgfqpoint{5.422377in}{2.880902in}}%
\pgfpathlineto{\pgfqpoint{5.433511in}{2.923955in}}%
\pgfpathlineto{\pgfqpoint{5.443120in}{2.964552in}}%
\pgfpathlineto{\pgfqpoint{5.451211in}{3.002703in}}%
\pgfpathlineto{\pgfqpoint{5.457791in}{3.038421in}}%
\pgfpathlineto{\pgfqpoint{5.462868in}{3.071718in}}%
\pgfpathlineto{\pgfqpoint{5.466450in}{3.102606in}}%
\pgfpathlineto{\pgfqpoint{5.468544in}{3.131097in}}%
\pgfpathlineto{\pgfqpoint{5.469158in}{3.157204in}}%
\pgfpathlineto{\pgfqpoint{5.468298in}{3.180938in}}%
\pgfpathlineto{\pgfqpoint{5.465973in}{3.202311in}}%
\pgfpathlineto{\pgfqpoint{5.462189in}{3.221335in}}%
\pgfpathlineto{\pgfqpoint{5.456954in}{3.238022in}}%
\pgfpathlineto{\pgfqpoint{5.450275in}{3.252382in}}%
\pgfpathlineto{\pgfqpoint{5.442160in}{3.264429in}}%
\pgfpathlineto{\pgfqpoint{5.432615in}{3.274173in}}%
\pgfpathlineto{\pgfqpoint{5.421647in}{3.281626in}}%
\pgfpathlineto{\pgfqpoint{5.409265in}{3.286800in}}%
\pgfpathlineto{\pgfqpoint{5.395474in}{3.289705in}}%
\pgfpathlineto{\pgfqpoint{5.380283in}{3.290354in}}%
\pgfpathlineto{\pgfqpoint{5.363697in}{3.288757in}}%
\pgfpathlineto{\pgfqpoint{5.345724in}{3.284926in}}%
\pgfpathlineto{\pgfqpoint{5.326371in}{3.278871in}}%
\pgfpathlineto{\pgfqpoint{5.305645in}{3.270605in}}%
\pgfpathlineto{\pgfqpoint{5.283552in}{3.260138in}}%
\pgfpathlineto{\pgfqpoint{5.260100in}{3.247480in}}%
\pgfpathlineto{\pgfqpoint{5.235295in}{3.232644in}}%
\pgfpathlineto{\pgfqpoint{5.209144in}{3.215640in}}%
\pgfpathlineto{\pgfqpoint{5.181653in}{3.196478in}}%
\pgfpathlineto{\pgfqpoint{5.152830in}{3.175170in}}%
\pgfpathlineto{\pgfqpoint{5.122682in}{3.151726in}}%
\pgfpathlineto{\pgfqpoint{5.091213in}{3.126157in}}%
\pgfpathlineto{\pgfqpoint{5.058432in}{3.098474in}}%
\pgfpathlineto{\pgfqpoint{5.024345in}{3.068687in}}%
\pgfpathlineto{\pgfqpoint{4.988959in}{3.036806in}}%
\pgfpathlineto{\pgfqpoint{4.952279in}{3.002843in}}%
\pgfpathlineto{\pgfqpoint{4.914313in}{2.966807in}}%
\pgfpathlineto{\pgfqpoint{4.875066in}{2.928709in}}%
\pgfpathlineto{\pgfqpoint{4.834546in}{2.888559in}}%
\pgfpathlineto{\pgfqpoint{4.792758in}{2.846368in}}%
\pgfpathlineto{\pgfqpoint{4.749709in}{2.802145in}}%
\pgfpathlineto{\pgfqpoint{4.705405in}{2.755901in}}%
\pgfpathlineto{\pgfqpoint{4.659853in}{2.707646in}}%
\pgfpathlineto{\pgfqpoint{4.613059in}{2.657390in}}%
\pgfpathlineto{\pgfqpoint{4.565028in}{2.605143in}}%
\pgfpathlineto{\pgfqpoint{4.515768in}{2.550915in}}%
\pgfpathlineto{\pgfqpoint{4.465283in}{2.494716in}}%
\pgfpathlineto{\pgfqpoint{4.413582in}{2.436556in}}%
\pgfpathlineto{\pgfqpoint{4.360668in}{2.376444in}}%
\pgfpathlineto{\pgfqpoint{4.306549in}{2.314390in}}%
\pgfpathlineto{\pgfqpoint{4.251231in}{2.250404in}}%
\pgfpathlineto{\pgfqpoint{4.194720in}{2.184496in}}%
\pgfpathlineto{\pgfqpoint{4.137020in}{2.116676in}}%
\pgfpathlineto{\pgfqpoint{4.078140in}{2.046952in}}%
\pgfpathlineto{\pgfqpoint{4.018083in}{1.975334in}}%
\pgfpathlineto{\pgfqpoint{3.956857in}{1.901832in}}%
\pgfpathlineto{\pgfqpoint{3.894467in}{1.826456in}}%
\pgfpathlineto{\pgfqpoint{3.830919in}{1.749214in}}%
\pgfpathlineto{\pgfqpoint{3.766218in}{1.670116in}}%
\pgfpathlineto{\pgfqpoint{3.700372in}{1.589171in}}%
\pgfpathlineto{\pgfqpoint{3.633384in}{1.506388in}}%
\pgfpathlineto{\pgfqpoint{3.565261in}{1.421777in}}%
\pgfpathlineto{\pgfqpoint{3.496009in}{1.335347in}}%
\pgfpathlineto{\pgfqpoint{3.425633in}{1.247107in}}%
\pgfpathlineto{\pgfqpoint{3.354138in}{1.157066in}}%
\pgfpathlineto{\pgfqpoint{3.281531in}{1.065233in}}%
\pgfpathlineto{\pgfqpoint{3.207818in}{0.971616in}}%
\pgfpathlineto{\pgfqpoint{3.133002in}{0.876226in}}%
\pgfpathlineto{\pgfqpoint{3.057091in}{0.779070in}}%
\pgfusepath{stroke}%
\end{pgfscope}%
\begin{pgfscope}%
\pgfpathrectangle{\pgfqpoint{1.250000in}{0.400000in}}{\pgfqpoint{7.750000in}{3.200000in}} %
\pgfusepath{clip}%
\pgfsetbuttcap%
\pgfsetroundjoin%
\definecolor{currentfill}{rgb}{0.000000,0.500000,0.000000}%
\pgfsetfillcolor{currentfill}%
\pgfsetlinewidth{0.501875pt}%
\definecolor{currentstroke}{rgb}{0.000000,0.500000,0.000000}%
\pgfsetstrokecolor{currentstroke}%
\pgfsetdash{}{0pt}%
\pgfsys@defobject{currentmarker}{\pgfqpoint{-0.041667in}{-0.041667in}}{\pgfqpoint{0.041667in}{0.041667in}}{%
\pgfpathmoveto{\pgfqpoint{-0.041667in}{0.000000in}}%
\pgfpathlineto{\pgfqpoint{0.041667in}{0.000000in}}%
\pgfpathmoveto{\pgfqpoint{0.000000in}{-0.041667in}}%
\pgfpathlineto{\pgfqpoint{0.000000in}{0.041667in}}%
\pgfusepath{stroke,fill}%
}%
\begin{pgfscope}%
\pgfsys@transformshift{4.571429in}{0.857143in}%
\pgfsys@useobject{currentmarker}{}%
\end{pgfscope}%
\begin{pgfscope}%
\pgfsys@transformshift{4.625039in}{0.968615in}%
\pgfsys@useobject{currentmarker}{}%
\end{pgfscope}%
\begin{pgfscope}%
\pgfsys@transformshift{4.676911in}{1.077287in}%
\pgfsys@useobject{currentmarker}{}%
\end{pgfscope}%
\begin{pgfscope}%
\pgfsys@transformshift{4.727054in}{1.183174in}%
\pgfsys@useobject{currentmarker}{}%
\end{pgfscope}%
\begin{pgfscope}%
\pgfsys@transformshift{4.775476in}{1.286289in}%
\pgfsys@useobject{currentmarker}{}%
\end{pgfscope}%
\begin{pgfscope}%
\pgfsys@transformshift{4.822186in}{1.386646in}%
\pgfsys@useobject{currentmarker}{}%
\end{pgfscope}%
\begin{pgfscope}%
\pgfsys@transformshift{4.867192in}{1.484259in}%
\pgfsys@useobject{currentmarker}{}%
\end{pgfscope}%
\begin{pgfscope}%
\pgfsys@transformshift{4.910504in}{1.579142in}%
\pgfsys@useobject{currentmarker}{}%
\end{pgfscope}%
\begin{pgfscope}%
\pgfsys@transformshift{4.952128in}{1.671308in}%
\pgfsys@useobject{currentmarker}{}%
\end{pgfscope}%
\begin{pgfscope}%
\pgfsys@transformshift{4.992075in}{1.760771in}%
\pgfsys@useobject{currentmarker}{}%
\end{pgfscope}%
\begin{pgfscope}%
\pgfsys@transformshift{5.030351in}{1.847544in}%
\pgfsys@useobject{currentmarker}{}%
\end{pgfscope}%
\begin{pgfscope}%
\pgfsys@transformshift{5.066967in}{1.931641in}%
\pgfsys@useobject{currentmarker}{}%
\end{pgfscope}%
\begin{pgfscope}%
\pgfsys@transformshift{5.101929in}{2.013075in}%
\pgfsys@useobject{currentmarker}{}%
\end{pgfscope}%
\begin{pgfscope}%
\pgfsys@transformshift{5.135246in}{2.091860in}%
\pgfsys@useobject{currentmarker}{}%
\end{pgfscope}%
\begin{pgfscope}%
\pgfsys@transformshift{5.166927in}{2.168009in}%
\pgfsys@useobject{currentmarker}{}%
\end{pgfscope}%
\begin{pgfscope}%
\pgfsys@transformshift{5.196979in}{2.241535in}%
\pgfsys@useobject{currentmarker}{}%
\end{pgfscope}%
\begin{pgfscope}%
\pgfsys@transformshift{5.225411in}{2.312451in}%
\pgfsys@useobject{currentmarker}{}%
\end{pgfscope}%
\begin{pgfscope}%
\pgfsys@transformshift{5.252231in}{2.380769in}%
\pgfsys@useobject{currentmarker}{}%
\end{pgfscope}%
\begin{pgfscope}%
\pgfsys@transformshift{5.277447in}{2.446504in}%
\pgfsys@useobject{currentmarker}{}%
\end{pgfscope}%
\begin{pgfscope}%
\pgfsys@transformshift{5.301066in}{2.509668in}%
\pgfsys@useobject{currentmarker}{}%
\end{pgfscope}%
\begin{pgfscope}%
\pgfsys@transformshift{5.323098in}{2.570274in}%
\pgfsys@useobject{currentmarker}{}%
\end{pgfscope}%
\begin{pgfscope}%
\pgfsys@transformshift{5.343549in}{2.628335in}%
\pgfsys@useobject{currentmarker}{}%
\end{pgfscope}%
\begin{pgfscope}%
\pgfsys@transformshift{5.362428in}{2.683863in}%
\pgfsys@useobject{currentmarker}{}%
\end{pgfscope}%
\begin{pgfscope}%
\pgfsys@transformshift{5.379742in}{2.736871in}%
\pgfsys@useobject{currentmarker}{}%
\end{pgfscope}%
\begin{pgfscope}%
\pgfsys@transformshift{5.395500in}{2.787372in}%
\pgfsys@useobject{currentmarker}{}%
\end{pgfscope}%
\begin{pgfscope}%
\pgfsys@transformshift{5.409709in}{2.835378in}%
\pgfsys@useobject{currentmarker}{}%
\end{pgfscope}%
\begin{pgfscope}%
\pgfsys@transformshift{5.422377in}{2.880902in}%
\pgfsys@useobject{currentmarker}{}%
\end{pgfscope}%
\begin{pgfscope}%
\pgfsys@transformshift{5.433511in}{2.923955in}%
\pgfsys@useobject{currentmarker}{}%
\end{pgfscope}%
\begin{pgfscope}%
\pgfsys@transformshift{5.443120in}{2.964552in}%
\pgfsys@useobject{currentmarker}{}%
\end{pgfscope}%
\begin{pgfscope}%
\pgfsys@transformshift{5.451211in}{3.002703in}%
\pgfsys@useobject{currentmarker}{}%
\end{pgfscope}%
\begin{pgfscope}%
\pgfsys@transformshift{5.457791in}{3.038421in}%
\pgfsys@useobject{currentmarker}{}%
\end{pgfscope}%
\begin{pgfscope}%
\pgfsys@transformshift{5.462868in}{3.071718in}%
\pgfsys@useobject{currentmarker}{}%
\end{pgfscope}%
\begin{pgfscope}%
\pgfsys@transformshift{5.466450in}{3.102606in}%
\pgfsys@useobject{currentmarker}{}%
\end{pgfscope}%
\begin{pgfscope}%
\pgfsys@transformshift{5.468544in}{3.131097in}%
\pgfsys@useobject{currentmarker}{}%
\end{pgfscope}%
\begin{pgfscope}%
\pgfsys@transformshift{5.469158in}{3.157204in}%
\pgfsys@useobject{currentmarker}{}%
\end{pgfscope}%
\begin{pgfscope}%
\pgfsys@transformshift{5.468298in}{3.180938in}%
\pgfsys@useobject{currentmarker}{}%
\end{pgfscope}%
\begin{pgfscope}%
\pgfsys@transformshift{5.465973in}{3.202311in}%
\pgfsys@useobject{currentmarker}{}%
\end{pgfscope}%
\begin{pgfscope}%
\pgfsys@transformshift{5.462189in}{3.221335in}%
\pgfsys@useobject{currentmarker}{}%
\end{pgfscope}%
\begin{pgfscope}%
\pgfsys@transformshift{5.456954in}{3.238022in}%
\pgfsys@useobject{currentmarker}{}%
\end{pgfscope}%
\begin{pgfscope}%
\pgfsys@transformshift{5.450275in}{3.252382in}%
\pgfsys@useobject{currentmarker}{}%
\end{pgfscope}%
\begin{pgfscope}%
\pgfsys@transformshift{5.442160in}{3.264429in}%
\pgfsys@useobject{currentmarker}{}%
\end{pgfscope}%
\begin{pgfscope}%
\pgfsys@transformshift{5.432615in}{3.274173in}%
\pgfsys@useobject{currentmarker}{}%
\end{pgfscope}%
\begin{pgfscope}%
\pgfsys@transformshift{5.421647in}{3.281626in}%
\pgfsys@useobject{currentmarker}{}%
\end{pgfscope}%
\begin{pgfscope}%
\pgfsys@transformshift{5.409265in}{3.286800in}%
\pgfsys@useobject{currentmarker}{}%
\end{pgfscope}%
\begin{pgfscope}%
\pgfsys@transformshift{5.395474in}{3.289705in}%
\pgfsys@useobject{currentmarker}{}%
\end{pgfscope}%
\begin{pgfscope}%
\pgfsys@transformshift{5.380283in}{3.290354in}%
\pgfsys@useobject{currentmarker}{}%
\end{pgfscope}%
\begin{pgfscope}%
\pgfsys@transformshift{5.363697in}{3.288757in}%
\pgfsys@useobject{currentmarker}{}%
\end{pgfscope}%
\begin{pgfscope}%
\pgfsys@transformshift{5.345724in}{3.284926in}%
\pgfsys@useobject{currentmarker}{}%
\end{pgfscope}%
\begin{pgfscope}%
\pgfsys@transformshift{5.326371in}{3.278871in}%
\pgfsys@useobject{currentmarker}{}%
\end{pgfscope}%
\begin{pgfscope}%
\pgfsys@transformshift{5.305645in}{3.270605in}%
\pgfsys@useobject{currentmarker}{}%
\end{pgfscope}%
\begin{pgfscope}%
\pgfsys@transformshift{5.283552in}{3.260138in}%
\pgfsys@useobject{currentmarker}{}%
\end{pgfscope}%
\begin{pgfscope}%
\pgfsys@transformshift{5.260100in}{3.247480in}%
\pgfsys@useobject{currentmarker}{}%
\end{pgfscope}%
\begin{pgfscope}%
\pgfsys@transformshift{5.235295in}{3.232644in}%
\pgfsys@useobject{currentmarker}{}%
\end{pgfscope}%
\begin{pgfscope}%
\pgfsys@transformshift{5.209144in}{3.215640in}%
\pgfsys@useobject{currentmarker}{}%
\end{pgfscope}%
\begin{pgfscope}%
\pgfsys@transformshift{5.181653in}{3.196478in}%
\pgfsys@useobject{currentmarker}{}%
\end{pgfscope}%
\begin{pgfscope}%
\pgfsys@transformshift{5.152830in}{3.175170in}%
\pgfsys@useobject{currentmarker}{}%
\end{pgfscope}%
\begin{pgfscope}%
\pgfsys@transformshift{5.122682in}{3.151726in}%
\pgfsys@useobject{currentmarker}{}%
\end{pgfscope}%
\begin{pgfscope}%
\pgfsys@transformshift{5.091213in}{3.126157in}%
\pgfsys@useobject{currentmarker}{}%
\end{pgfscope}%
\begin{pgfscope}%
\pgfsys@transformshift{5.058432in}{3.098474in}%
\pgfsys@useobject{currentmarker}{}%
\end{pgfscope}%
\begin{pgfscope}%
\pgfsys@transformshift{5.024345in}{3.068687in}%
\pgfsys@useobject{currentmarker}{}%
\end{pgfscope}%
\begin{pgfscope}%
\pgfsys@transformshift{4.988959in}{3.036806in}%
\pgfsys@useobject{currentmarker}{}%
\end{pgfscope}%
\begin{pgfscope}%
\pgfsys@transformshift{4.952279in}{3.002843in}%
\pgfsys@useobject{currentmarker}{}%
\end{pgfscope}%
\begin{pgfscope}%
\pgfsys@transformshift{4.914313in}{2.966807in}%
\pgfsys@useobject{currentmarker}{}%
\end{pgfscope}%
\begin{pgfscope}%
\pgfsys@transformshift{4.875066in}{2.928709in}%
\pgfsys@useobject{currentmarker}{}%
\end{pgfscope}%
\begin{pgfscope}%
\pgfsys@transformshift{4.834546in}{2.888559in}%
\pgfsys@useobject{currentmarker}{}%
\end{pgfscope}%
\begin{pgfscope}%
\pgfsys@transformshift{4.792758in}{2.846368in}%
\pgfsys@useobject{currentmarker}{}%
\end{pgfscope}%
\begin{pgfscope}%
\pgfsys@transformshift{4.749709in}{2.802145in}%
\pgfsys@useobject{currentmarker}{}%
\end{pgfscope}%
\begin{pgfscope}%
\pgfsys@transformshift{4.705405in}{2.755901in}%
\pgfsys@useobject{currentmarker}{}%
\end{pgfscope}%
\begin{pgfscope}%
\pgfsys@transformshift{4.659853in}{2.707646in}%
\pgfsys@useobject{currentmarker}{}%
\end{pgfscope}%
\begin{pgfscope}%
\pgfsys@transformshift{4.613059in}{2.657390in}%
\pgfsys@useobject{currentmarker}{}%
\end{pgfscope}%
\begin{pgfscope}%
\pgfsys@transformshift{4.565028in}{2.605143in}%
\pgfsys@useobject{currentmarker}{}%
\end{pgfscope}%
\begin{pgfscope}%
\pgfsys@transformshift{4.515768in}{2.550915in}%
\pgfsys@useobject{currentmarker}{}%
\end{pgfscope}%
\begin{pgfscope}%
\pgfsys@transformshift{4.465283in}{2.494716in}%
\pgfsys@useobject{currentmarker}{}%
\end{pgfscope}%
\begin{pgfscope}%
\pgfsys@transformshift{4.413582in}{2.436556in}%
\pgfsys@useobject{currentmarker}{}%
\end{pgfscope}%
\begin{pgfscope}%
\pgfsys@transformshift{4.360668in}{2.376444in}%
\pgfsys@useobject{currentmarker}{}%
\end{pgfscope}%
\begin{pgfscope}%
\pgfsys@transformshift{4.306549in}{2.314390in}%
\pgfsys@useobject{currentmarker}{}%
\end{pgfscope}%
\begin{pgfscope}%
\pgfsys@transformshift{4.251231in}{2.250404in}%
\pgfsys@useobject{currentmarker}{}%
\end{pgfscope}%
\begin{pgfscope}%
\pgfsys@transformshift{4.194720in}{2.184496in}%
\pgfsys@useobject{currentmarker}{}%
\end{pgfscope}%
\begin{pgfscope}%
\pgfsys@transformshift{4.137020in}{2.116676in}%
\pgfsys@useobject{currentmarker}{}%
\end{pgfscope}%
\begin{pgfscope}%
\pgfsys@transformshift{4.078140in}{2.046952in}%
\pgfsys@useobject{currentmarker}{}%
\end{pgfscope}%
\begin{pgfscope}%
\pgfsys@transformshift{4.018083in}{1.975334in}%
\pgfsys@useobject{currentmarker}{}%
\end{pgfscope}%
\begin{pgfscope}%
\pgfsys@transformshift{3.956857in}{1.901832in}%
\pgfsys@useobject{currentmarker}{}%
\end{pgfscope}%
\begin{pgfscope}%
\pgfsys@transformshift{3.894467in}{1.826456in}%
\pgfsys@useobject{currentmarker}{}%
\end{pgfscope}%
\begin{pgfscope}%
\pgfsys@transformshift{3.830919in}{1.749214in}%
\pgfsys@useobject{currentmarker}{}%
\end{pgfscope}%
\begin{pgfscope}%
\pgfsys@transformshift{3.766218in}{1.670116in}%
\pgfsys@useobject{currentmarker}{}%
\end{pgfscope}%
\begin{pgfscope}%
\pgfsys@transformshift{3.700372in}{1.589171in}%
\pgfsys@useobject{currentmarker}{}%
\end{pgfscope}%
\begin{pgfscope}%
\pgfsys@transformshift{3.633384in}{1.506388in}%
\pgfsys@useobject{currentmarker}{}%
\end{pgfscope}%
\begin{pgfscope}%
\pgfsys@transformshift{3.565261in}{1.421777in}%
\pgfsys@useobject{currentmarker}{}%
\end{pgfscope}%
\begin{pgfscope}%
\pgfsys@transformshift{3.496009in}{1.335347in}%
\pgfsys@useobject{currentmarker}{}%
\end{pgfscope}%
\begin{pgfscope}%
\pgfsys@transformshift{3.425633in}{1.247107in}%
\pgfsys@useobject{currentmarker}{}%
\end{pgfscope}%
\begin{pgfscope}%
\pgfsys@transformshift{3.354138in}{1.157066in}%
\pgfsys@useobject{currentmarker}{}%
\end{pgfscope}%
\begin{pgfscope}%
\pgfsys@transformshift{3.281531in}{1.065233in}%
\pgfsys@useobject{currentmarker}{}%
\end{pgfscope}%
\begin{pgfscope}%
\pgfsys@transformshift{3.207818in}{0.971616in}%
\pgfsys@useobject{currentmarker}{}%
\end{pgfscope}%
\begin{pgfscope}%
\pgfsys@transformshift{3.133002in}{0.876226in}%
\pgfsys@useobject{currentmarker}{}%
\end{pgfscope}%
\begin{pgfscope}%
\pgfsys@transformshift{3.057091in}{0.779070in}%
\pgfsys@useobject{currentmarker}{}%
\end{pgfscope}%
\end{pgfscope}%
\begin{pgfscope}%
\pgfpathrectangle{\pgfqpoint{1.250000in}{0.400000in}}{\pgfqpoint{7.750000in}{3.200000in}} %
\pgfusepath{clip}%
\pgfsetrectcap%
\pgfsetroundjoin%
\pgfsetlinewidth{1.003750pt}%
\definecolor{currentstroke}{rgb}{1.000000,0.000000,0.000000}%
\pgfsetstrokecolor{currentstroke}%
\pgfsetdash{}{0pt}%
\pgfpathmoveto{\pgfqpoint{4.571429in}{0.857143in}}%
\pgfpathlineto{\pgfqpoint{4.625230in}{0.968615in}}%
\pgfpathlineto{\pgfqpoint{4.677482in}{1.077287in}}%
\pgfpathlineto{\pgfqpoint{4.728194in}{1.183174in}}%
\pgfpathlineto{\pgfqpoint{4.777373in}{1.286289in}}%
\pgfpathlineto{\pgfqpoint{4.825027in}{1.386646in}}%
\pgfpathlineto{\pgfqpoint{4.871163in}{1.484259in}}%
\pgfpathlineto{\pgfqpoint{4.915789in}{1.579142in}}%
\pgfpathlineto{\pgfqpoint{4.958913in}{1.671308in}}%
\pgfpathlineto{\pgfqpoint{5.000542in}{1.760771in}}%
\pgfpathlineto{\pgfqpoint{5.040683in}{1.847544in}}%
\pgfpathlineto{\pgfqpoint{5.079344in}{1.931641in}}%
\pgfpathlineto{\pgfqpoint{5.116532in}{2.013075in}}%
\pgfpathlineto{\pgfqpoint{5.152255in}{2.091860in}}%
\pgfpathlineto{\pgfqpoint{5.186520in}{2.168009in}}%
\pgfpathlineto{\pgfqpoint{5.219335in}{2.241535in}}%
\pgfpathlineto{\pgfqpoint{5.250706in}{2.312451in}}%
\pgfpathlineto{\pgfqpoint{5.280641in}{2.380769in}}%
\pgfpathlineto{\pgfqpoint{5.309146in}{2.446504in}}%
\pgfpathlineto{\pgfqpoint{5.336230in}{2.509668in}}%
\pgfpathlineto{\pgfqpoint{5.361899in}{2.570274in}}%
\pgfpathlineto{\pgfqpoint{5.386161in}{2.628335in}}%
\pgfpathlineto{\pgfqpoint{5.409021in}{2.683863in}}%
\pgfpathlineto{\pgfqpoint{5.430488in}{2.736871in}}%
\pgfpathlineto{\pgfqpoint{5.450569in}{2.787372in}}%
\pgfpathlineto{\pgfqpoint{5.469269in}{2.835378in}}%
\pgfpathlineto{\pgfqpoint{5.486597in}{2.880902in}}%
\pgfpathlineto{\pgfqpoint{5.502558in}{2.923955in}}%
\pgfpathlineto{\pgfqpoint{5.517161in}{2.964552in}}%
\pgfpathlineto{\pgfqpoint{5.530411in}{3.002703in}}%
\pgfpathlineto{\pgfqpoint{5.542315in}{3.038421in}}%
\pgfpathlineto{\pgfqpoint{5.552881in}{3.071718in}}%
\pgfpathlineto{\pgfqpoint{5.562114in}{3.102606in}}%
\pgfpathlineto{\pgfqpoint{5.570022in}{3.131097in}}%
\pgfpathlineto{\pgfqpoint{5.576611in}{3.157204in}}%
\pgfpathlineto{\pgfqpoint{5.581888in}{3.180938in}}%
\pgfpathlineto{\pgfqpoint{5.585859in}{3.202311in}}%
\pgfpathlineto{\pgfqpoint{5.588531in}{3.221335in}}%
\pgfpathlineto{\pgfqpoint{5.589910in}{3.238022in}}%
\pgfpathlineto{\pgfqpoint{5.590002in}{3.252382in}}%
\pgfpathlineto{\pgfqpoint{5.588815in}{3.264429in}}%
\pgfpathlineto{\pgfqpoint{5.586355in}{3.274173in}}%
\pgfpathlineto{\pgfqpoint{5.582627in}{3.281626in}}%
\pgfpathlineto{\pgfqpoint{5.577639in}{3.286800in}}%
\pgfpathlineto{\pgfqpoint{5.571396in}{3.289705in}}%
\pgfpathlineto{\pgfqpoint{5.563905in}{3.290354in}}%
\pgfpathlineto{\pgfqpoint{5.555172in}{3.288757in}}%
\pgfpathlineto{\pgfqpoint{5.545204in}{3.284926in}}%
\pgfpathlineto{\pgfqpoint{5.534006in}{3.278871in}}%
\pgfpathlineto{\pgfqpoint{5.521584in}{3.270605in}}%
\pgfpathlineto{\pgfqpoint{5.507946in}{3.260138in}}%
\pgfpathlineto{\pgfqpoint{5.493096in}{3.247480in}}%
\pgfpathlineto{\pgfqpoint{5.477041in}{3.232644in}}%
\pgfpathlineto{\pgfqpoint{5.459787in}{3.215640in}}%
\pgfpathlineto{\pgfqpoint{5.441340in}{3.196478in}}%
\pgfpathlineto{\pgfqpoint{5.421705in}{3.175170in}}%
\pgfpathlineto{\pgfqpoint{5.400890in}{3.151726in}}%
\pgfpathlineto{\pgfqpoint{5.378899in}{3.126157in}}%
\pgfpathlineto{\pgfqpoint{5.355739in}{3.098474in}}%
\pgfpathlineto{\pgfqpoint{5.331415in}{3.068687in}}%
\pgfpathlineto{\pgfqpoint{5.305934in}{3.036806in}}%
\pgfpathlineto{\pgfqpoint{5.279300in}{3.002843in}}%
\pgfpathlineto{\pgfqpoint{5.251521in}{2.966807in}}%
\pgfpathlineto{\pgfqpoint{5.222600in}{2.928709in}}%
\pgfpathlineto{\pgfqpoint{5.192546in}{2.888559in}}%
\pgfpathlineto{\pgfqpoint{5.161362in}{2.846368in}}%
\pgfpathlineto{\pgfqpoint{5.129054in}{2.802145in}}%
\pgfpathlineto{\pgfqpoint{5.095629in}{2.755901in}}%
\pgfpathlineto{\pgfqpoint{5.061092in}{2.707646in}}%
\pgfpathlineto{\pgfqpoint{5.025447in}{2.657390in}}%
\pgfpathlineto{\pgfqpoint{4.988702in}{2.605143in}}%
\pgfpathlineto{\pgfqpoint{4.950861in}{2.550915in}}%
\pgfpathlineto{\pgfqpoint{4.911930in}{2.494716in}}%
\pgfpathlineto{\pgfqpoint{4.871914in}{2.436556in}}%
\pgfpathlineto{\pgfqpoint{4.830819in}{2.376444in}}%
\pgfpathlineto{\pgfqpoint{4.788650in}{2.314390in}}%
\pgfpathlineto{\pgfqpoint{4.745413in}{2.250404in}}%
\pgfpathlineto{\pgfqpoint{4.701112in}{2.184496in}}%
\pgfpathlineto{\pgfqpoint{4.655754in}{2.116676in}}%
\pgfpathlineto{\pgfqpoint{4.609342in}{2.046952in}}%
\pgfpathlineto{\pgfqpoint{4.561884in}{1.975334in}}%
\pgfpathlineto{\pgfqpoint{4.513384in}{1.901832in}}%
\pgfpathlineto{\pgfqpoint{4.463846in}{1.826456in}}%
\pgfpathlineto{\pgfqpoint{4.413277in}{1.749214in}}%
\pgfpathlineto{\pgfqpoint{4.361682in}{1.670116in}}%
\pgfpathlineto{\pgfqpoint{4.309065in}{1.589171in}}%
\pgfpathlineto{\pgfqpoint{4.255431in}{1.506388in}}%
\pgfpathlineto{\pgfqpoint{4.200787in}{1.421777in}}%
\pgfpathlineto{\pgfqpoint{4.145136in}{1.335347in}}%
\pgfpathlineto{\pgfqpoint{4.088485in}{1.247107in}}%
\pgfpathlineto{\pgfqpoint{4.030837in}{1.157066in}}%
\pgfpathlineto{\pgfqpoint{3.972198in}{1.065233in}}%
\pgfpathlineto{\pgfqpoint{3.912573in}{0.971616in}}%
\pgfpathlineto{\pgfqpoint{3.851967in}{0.876226in}}%
\pgfpathlineto{\pgfqpoint{3.790384in}{0.779070in}}%
\pgfusepath{stroke}%
\end{pgfscope}%
\begin{pgfscope}%
\pgfpathrectangle{\pgfqpoint{1.250000in}{0.400000in}}{\pgfqpoint{7.750000in}{3.200000in}} %
\pgfusepath{clip}%
\pgfsetbuttcap%
\pgfsetroundjoin%
\definecolor{currentfill}{rgb}{1.000000,0.000000,0.000000}%
\pgfsetfillcolor{currentfill}%
\pgfsetlinewidth{0.501875pt}%
\definecolor{currentstroke}{rgb}{1.000000,0.000000,0.000000}%
\pgfsetstrokecolor{currentstroke}%
\pgfsetdash{}{0pt}%
\pgfsys@defobject{currentmarker}{\pgfqpoint{-0.041667in}{-0.041667in}}{\pgfqpoint{0.041667in}{0.041667in}}{%
\pgfpathmoveto{\pgfqpoint{-0.041667in}{0.000000in}}%
\pgfpathlineto{\pgfqpoint{0.041667in}{0.000000in}}%
\pgfpathmoveto{\pgfqpoint{0.000000in}{-0.041667in}}%
\pgfpathlineto{\pgfqpoint{0.000000in}{0.041667in}}%
\pgfusepath{stroke,fill}%
}%
\begin{pgfscope}%
\pgfsys@transformshift{4.571429in}{0.857143in}%
\pgfsys@useobject{currentmarker}{}%
\end{pgfscope}%
\begin{pgfscope}%
\pgfsys@transformshift{4.625230in}{0.968615in}%
\pgfsys@useobject{currentmarker}{}%
\end{pgfscope}%
\begin{pgfscope}%
\pgfsys@transformshift{4.677482in}{1.077287in}%
\pgfsys@useobject{currentmarker}{}%
\end{pgfscope}%
\begin{pgfscope}%
\pgfsys@transformshift{4.728194in}{1.183174in}%
\pgfsys@useobject{currentmarker}{}%
\end{pgfscope}%
\begin{pgfscope}%
\pgfsys@transformshift{4.777373in}{1.286289in}%
\pgfsys@useobject{currentmarker}{}%
\end{pgfscope}%
\begin{pgfscope}%
\pgfsys@transformshift{4.825027in}{1.386646in}%
\pgfsys@useobject{currentmarker}{}%
\end{pgfscope}%
\begin{pgfscope}%
\pgfsys@transformshift{4.871163in}{1.484259in}%
\pgfsys@useobject{currentmarker}{}%
\end{pgfscope}%
\begin{pgfscope}%
\pgfsys@transformshift{4.915789in}{1.579142in}%
\pgfsys@useobject{currentmarker}{}%
\end{pgfscope}%
\begin{pgfscope}%
\pgfsys@transformshift{4.958913in}{1.671308in}%
\pgfsys@useobject{currentmarker}{}%
\end{pgfscope}%
\begin{pgfscope}%
\pgfsys@transformshift{5.000542in}{1.760771in}%
\pgfsys@useobject{currentmarker}{}%
\end{pgfscope}%
\begin{pgfscope}%
\pgfsys@transformshift{5.040683in}{1.847544in}%
\pgfsys@useobject{currentmarker}{}%
\end{pgfscope}%
\begin{pgfscope}%
\pgfsys@transformshift{5.079344in}{1.931641in}%
\pgfsys@useobject{currentmarker}{}%
\end{pgfscope}%
\begin{pgfscope}%
\pgfsys@transformshift{5.116532in}{2.013075in}%
\pgfsys@useobject{currentmarker}{}%
\end{pgfscope}%
\begin{pgfscope}%
\pgfsys@transformshift{5.152255in}{2.091860in}%
\pgfsys@useobject{currentmarker}{}%
\end{pgfscope}%
\begin{pgfscope}%
\pgfsys@transformshift{5.186520in}{2.168009in}%
\pgfsys@useobject{currentmarker}{}%
\end{pgfscope}%
\begin{pgfscope}%
\pgfsys@transformshift{5.219335in}{2.241535in}%
\pgfsys@useobject{currentmarker}{}%
\end{pgfscope}%
\begin{pgfscope}%
\pgfsys@transformshift{5.250706in}{2.312451in}%
\pgfsys@useobject{currentmarker}{}%
\end{pgfscope}%
\begin{pgfscope}%
\pgfsys@transformshift{5.280641in}{2.380769in}%
\pgfsys@useobject{currentmarker}{}%
\end{pgfscope}%
\begin{pgfscope}%
\pgfsys@transformshift{5.309146in}{2.446504in}%
\pgfsys@useobject{currentmarker}{}%
\end{pgfscope}%
\begin{pgfscope}%
\pgfsys@transformshift{5.336230in}{2.509668in}%
\pgfsys@useobject{currentmarker}{}%
\end{pgfscope}%
\begin{pgfscope}%
\pgfsys@transformshift{5.361899in}{2.570274in}%
\pgfsys@useobject{currentmarker}{}%
\end{pgfscope}%
\begin{pgfscope}%
\pgfsys@transformshift{5.386161in}{2.628335in}%
\pgfsys@useobject{currentmarker}{}%
\end{pgfscope}%
\begin{pgfscope}%
\pgfsys@transformshift{5.409021in}{2.683863in}%
\pgfsys@useobject{currentmarker}{}%
\end{pgfscope}%
\begin{pgfscope}%
\pgfsys@transformshift{5.430488in}{2.736871in}%
\pgfsys@useobject{currentmarker}{}%
\end{pgfscope}%
\begin{pgfscope}%
\pgfsys@transformshift{5.450569in}{2.787372in}%
\pgfsys@useobject{currentmarker}{}%
\end{pgfscope}%
\begin{pgfscope}%
\pgfsys@transformshift{5.469269in}{2.835378in}%
\pgfsys@useobject{currentmarker}{}%
\end{pgfscope}%
\begin{pgfscope}%
\pgfsys@transformshift{5.486597in}{2.880902in}%
\pgfsys@useobject{currentmarker}{}%
\end{pgfscope}%
\begin{pgfscope}%
\pgfsys@transformshift{5.502558in}{2.923955in}%
\pgfsys@useobject{currentmarker}{}%
\end{pgfscope}%
\begin{pgfscope}%
\pgfsys@transformshift{5.517161in}{2.964552in}%
\pgfsys@useobject{currentmarker}{}%
\end{pgfscope}%
\begin{pgfscope}%
\pgfsys@transformshift{5.530411in}{3.002703in}%
\pgfsys@useobject{currentmarker}{}%
\end{pgfscope}%
\begin{pgfscope}%
\pgfsys@transformshift{5.542315in}{3.038421in}%
\pgfsys@useobject{currentmarker}{}%
\end{pgfscope}%
\begin{pgfscope}%
\pgfsys@transformshift{5.552881in}{3.071718in}%
\pgfsys@useobject{currentmarker}{}%
\end{pgfscope}%
\begin{pgfscope}%
\pgfsys@transformshift{5.562114in}{3.102606in}%
\pgfsys@useobject{currentmarker}{}%
\end{pgfscope}%
\begin{pgfscope}%
\pgfsys@transformshift{5.570022in}{3.131097in}%
\pgfsys@useobject{currentmarker}{}%
\end{pgfscope}%
\begin{pgfscope}%
\pgfsys@transformshift{5.576611in}{3.157204in}%
\pgfsys@useobject{currentmarker}{}%
\end{pgfscope}%
\begin{pgfscope}%
\pgfsys@transformshift{5.581888in}{3.180938in}%
\pgfsys@useobject{currentmarker}{}%
\end{pgfscope}%
\begin{pgfscope}%
\pgfsys@transformshift{5.585859in}{3.202311in}%
\pgfsys@useobject{currentmarker}{}%
\end{pgfscope}%
\begin{pgfscope}%
\pgfsys@transformshift{5.588531in}{3.221335in}%
\pgfsys@useobject{currentmarker}{}%
\end{pgfscope}%
\begin{pgfscope}%
\pgfsys@transformshift{5.589910in}{3.238022in}%
\pgfsys@useobject{currentmarker}{}%
\end{pgfscope}%
\begin{pgfscope}%
\pgfsys@transformshift{5.590002in}{3.252382in}%
\pgfsys@useobject{currentmarker}{}%
\end{pgfscope}%
\begin{pgfscope}%
\pgfsys@transformshift{5.588815in}{3.264429in}%
\pgfsys@useobject{currentmarker}{}%
\end{pgfscope}%
\begin{pgfscope}%
\pgfsys@transformshift{5.586355in}{3.274173in}%
\pgfsys@useobject{currentmarker}{}%
\end{pgfscope}%
\begin{pgfscope}%
\pgfsys@transformshift{5.582627in}{3.281626in}%
\pgfsys@useobject{currentmarker}{}%
\end{pgfscope}%
\begin{pgfscope}%
\pgfsys@transformshift{5.577639in}{3.286800in}%
\pgfsys@useobject{currentmarker}{}%
\end{pgfscope}%
\begin{pgfscope}%
\pgfsys@transformshift{5.571396in}{3.289705in}%
\pgfsys@useobject{currentmarker}{}%
\end{pgfscope}%
\begin{pgfscope}%
\pgfsys@transformshift{5.563905in}{3.290354in}%
\pgfsys@useobject{currentmarker}{}%
\end{pgfscope}%
\begin{pgfscope}%
\pgfsys@transformshift{5.555172in}{3.288757in}%
\pgfsys@useobject{currentmarker}{}%
\end{pgfscope}%
\begin{pgfscope}%
\pgfsys@transformshift{5.545204in}{3.284926in}%
\pgfsys@useobject{currentmarker}{}%
\end{pgfscope}%
\begin{pgfscope}%
\pgfsys@transformshift{5.534006in}{3.278871in}%
\pgfsys@useobject{currentmarker}{}%
\end{pgfscope}%
\begin{pgfscope}%
\pgfsys@transformshift{5.521584in}{3.270605in}%
\pgfsys@useobject{currentmarker}{}%
\end{pgfscope}%
\begin{pgfscope}%
\pgfsys@transformshift{5.507946in}{3.260138in}%
\pgfsys@useobject{currentmarker}{}%
\end{pgfscope}%
\begin{pgfscope}%
\pgfsys@transformshift{5.493096in}{3.247480in}%
\pgfsys@useobject{currentmarker}{}%
\end{pgfscope}%
\begin{pgfscope}%
\pgfsys@transformshift{5.477041in}{3.232644in}%
\pgfsys@useobject{currentmarker}{}%
\end{pgfscope}%
\begin{pgfscope}%
\pgfsys@transformshift{5.459787in}{3.215640in}%
\pgfsys@useobject{currentmarker}{}%
\end{pgfscope}%
\begin{pgfscope}%
\pgfsys@transformshift{5.441340in}{3.196478in}%
\pgfsys@useobject{currentmarker}{}%
\end{pgfscope}%
\begin{pgfscope}%
\pgfsys@transformshift{5.421705in}{3.175170in}%
\pgfsys@useobject{currentmarker}{}%
\end{pgfscope}%
\begin{pgfscope}%
\pgfsys@transformshift{5.400890in}{3.151726in}%
\pgfsys@useobject{currentmarker}{}%
\end{pgfscope}%
\begin{pgfscope}%
\pgfsys@transformshift{5.378899in}{3.126157in}%
\pgfsys@useobject{currentmarker}{}%
\end{pgfscope}%
\begin{pgfscope}%
\pgfsys@transformshift{5.355739in}{3.098474in}%
\pgfsys@useobject{currentmarker}{}%
\end{pgfscope}%
\begin{pgfscope}%
\pgfsys@transformshift{5.331415in}{3.068687in}%
\pgfsys@useobject{currentmarker}{}%
\end{pgfscope}%
\begin{pgfscope}%
\pgfsys@transformshift{5.305934in}{3.036806in}%
\pgfsys@useobject{currentmarker}{}%
\end{pgfscope}%
\begin{pgfscope}%
\pgfsys@transformshift{5.279300in}{3.002843in}%
\pgfsys@useobject{currentmarker}{}%
\end{pgfscope}%
\begin{pgfscope}%
\pgfsys@transformshift{5.251521in}{2.966807in}%
\pgfsys@useobject{currentmarker}{}%
\end{pgfscope}%
\begin{pgfscope}%
\pgfsys@transformshift{5.222600in}{2.928709in}%
\pgfsys@useobject{currentmarker}{}%
\end{pgfscope}%
\begin{pgfscope}%
\pgfsys@transformshift{5.192546in}{2.888559in}%
\pgfsys@useobject{currentmarker}{}%
\end{pgfscope}%
\begin{pgfscope}%
\pgfsys@transformshift{5.161362in}{2.846368in}%
\pgfsys@useobject{currentmarker}{}%
\end{pgfscope}%
\begin{pgfscope}%
\pgfsys@transformshift{5.129054in}{2.802145in}%
\pgfsys@useobject{currentmarker}{}%
\end{pgfscope}%
\begin{pgfscope}%
\pgfsys@transformshift{5.095629in}{2.755901in}%
\pgfsys@useobject{currentmarker}{}%
\end{pgfscope}%
\begin{pgfscope}%
\pgfsys@transformshift{5.061092in}{2.707646in}%
\pgfsys@useobject{currentmarker}{}%
\end{pgfscope}%
\begin{pgfscope}%
\pgfsys@transformshift{5.025447in}{2.657390in}%
\pgfsys@useobject{currentmarker}{}%
\end{pgfscope}%
\begin{pgfscope}%
\pgfsys@transformshift{4.988702in}{2.605143in}%
\pgfsys@useobject{currentmarker}{}%
\end{pgfscope}%
\begin{pgfscope}%
\pgfsys@transformshift{4.950861in}{2.550915in}%
\pgfsys@useobject{currentmarker}{}%
\end{pgfscope}%
\begin{pgfscope}%
\pgfsys@transformshift{4.911930in}{2.494716in}%
\pgfsys@useobject{currentmarker}{}%
\end{pgfscope}%
\begin{pgfscope}%
\pgfsys@transformshift{4.871914in}{2.436556in}%
\pgfsys@useobject{currentmarker}{}%
\end{pgfscope}%
\begin{pgfscope}%
\pgfsys@transformshift{4.830819in}{2.376444in}%
\pgfsys@useobject{currentmarker}{}%
\end{pgfscope}%
\begin{pgfscope}%
\pgfsys@transformshift{4.788650in}{2.314390in}%
\pgfsys@useobject{currentmarker}{}%
\end{pgfscope}%
\begin{pgfscope}%
\pgfsys@transformshift{4.745413in}{2.250404in}%
\pgfsys@useobject{currentmarker}{}%
\end{pgfscope}%
\begin{pgfscope}%
\pgfsys@transformshift{4.701112in}{2.184496in}%
\pgfsys@useobject{currentmarker}{}%
\end{pgfscope}%
\begin{pgfscope}%
\pgfsys@transformshift{4.655754in}{2.116676in}%
\pgfsys@useobject{currentmarker}{}%
\end{pgfscope}%
\begin{pgfscope}%
\pgfsys@transformshift{4.609342in}{2.046952in}%
\pgfsys@useobject{currentmarker}{}%
\end{pgfscope}%
\begin{pgfscope}%
\pgfsys@transformshift{4.561884in}{1.975334in}%
\pgfsys@useobject{currentmarker}{}%
\end{pgfscope}%
\begin{pgfscope}%
\pgfsys@transformshift{4.513384in}{1.901832in}%
\pgfsys@useobject{currentmarker}{}%
\end{pgfscope}%
\begin{pgfscope}%
\pgfsys@transformshift{4.463846in}{1.826456in}%
\pgfsys@useobject{currentmarker}{}%
\end{pgfscope}%
\begin{pgfscope}%
\pgfsys@transformshift{4.413277in}{1.749214in}%
\pgfsys@useobject{currentmarker}{}%
\end{pgfscope}%
\begin{pgfscope}%
\pgfsys@transformshift{4.361682in}{1.670116in}%
\pgfsys@useobject{currentmarker}{}%
\end{pgfscope}%
\begin{pgfscope}%
\pgfsys@transformshift{4.309065in}{1.589171in}%
\pgfsys@useobject{currentmarker}{}%
\end{pgfscope}%
\begin{pgfscope}%
\pgfsys@transformshift{4.255431in}{1.506388in}%
\pgfsys@useobject{currentmarker}{}%
\end{pgfscope}%
\begin{pgfscope}%
\pgfsys@transformshift{4.200787in}{1.421777in}%
\pgfsys@useobject{currentmarker}{}%
\end{pgfscope}%
\begin{pgfscope}%
\pgfsys@transformshift{4.145136in}{1.335347in}%
\pgfsys@useobject{currentmarker}{}%
\end{pgfscope}%
\begin{pgfscope}%
\pgfsys@transformshift{4.088485in}{1.247107in}%
\pgfsys@useobject{currentmarker}{}%
\end{pgfscope}%
\begin{pgfscope}%
\pgfsys@transformshift{4.030837in}{1.157066in}%
\pgfsys@useobject{currentmarker}{}%
\end{pgfscope}%
\begin{pgfscope}%
\pgfsys@transformshift{3.972198in}{1.065233in}%
\pgfsys@useobject{currentmarker}{}%
\end{pgfscope}%
\begin{pgfscope}%
\pgfsys@transformshift{3.912573in}{0.971616in}%
\pgfsys@useobject{currentmarker}{}%
\end{pgfscope}%
\begin{pgfscope}%
\pgfsys@transformshift{3.851967in}{0.876226in}%
\pgfsys@useobject{currentmarker}{}%
\end{pgfscope}%
\begin{pgfscope}%
\pgfsys@transformshift{3.790384in}{0.779070in}%
\pgfsys@useobject{currentmarker}{}%
\end{pgfscope}%
\end{pgfscope}%
\begin{pgfscope}%
\pgfpathrectangle{\pgfqpoint{1.250000in}{0.400000in}}{\pgfqpoint{7.750000in}{3.200000in}} %
\pgfusepath{clip}%
\pgfsetrectcap%
\pgfsetroundjoin%
\pgfsetlinewidth{1.003750pt}%
\definecolor{currentstroke}{rgb}{0.000000,0.750000,0.750000}%
\pgfsetstrokecolor{currentstroke}%
\pgfsetdash{}{0pt}%
\pgfpathmoveto{\pgfqpoint{4.571429in}{0.857143in}}%
\pgfpathlineto{\pgfqpoint{4.625420in}{0.968615in}}%
\pgfpathlineto{\pgfqpoint{4.678053in}{1.077287in}}%
\pgfpathlineto{\pgfqpoint{4.729334in}{1.183174in}}%
\pgfpathlineto{\pgfqpoint{4.779270in}{1.286289in}}%
\pgfpathlineto{\pgfqpoint{4.827868in}{1.386646in}}%
\pgfpathlineto{\pgfqpoint{4.875134in}{1.484259in}}%
\pgfpathlineto{\pgfqpoint{4.921075in}{1.579142in}}%
\pgfpathlineto{\pgfqpoint{4.965698in}{1.671308in}}%
\pgfpathlineto{\pgfqpoint{5.009009in}{1.760771in}}%
\pgfpathlineto{\pgfqpoint{5.051014in}{1.847544in}}%
\pgfpathlineto{\pgfqpoint{5.091721in}{1.931641in}}%
\pgfpathlineto{\pgfqpoint{5.131136in}{2.013075in}}%
\pgfpathlineto{\pgfqpoint{5.169264in}{2.091860in}}%
\pgfpathlineto{\pgfqpoint{5.206114in}{2.168009in}}%
\pgfpathlineto{\pgfqpoint{5.241691in}{2.241535in}}%
\pgfpathlineto{\pgfqpoint{5.276001in}{2.312451in}}%
\pgfpathlineto{\pgfqpoint{5.309050in}{2.380769in}}%
\pgfpathlineto{\pgfqpoint{5.340846in}{2.446504in}}%
\pgfpathlineto{\pgfqpoint{5.371394in}{2.509668in}}%
\pgfpathlineto{\pgfqpoint{5.400701in}{2.570274in}}%
\pgfpathlineto{\pgfqpoint{5.428772in}{2.628335in}}%
\pgfpathlineto{\pgfqpoint{5.455615in}{2.683863in}}%
\pgfpathlineto{\pgfqpoint{5.481234in}{2.736871in}}%
\pgfpathlineto{\pgfqpoint{5.505637in}{2.787372in}}%
\pgfpathlineto{\pgfqpoint{5.528829in}{2.835378in}}%
\pgfpathlineto{\pgfqpoint{5.550817in}{2.880902in}}%
\pgfpathlineto{\pgfqpoint{5.571605in}{2.923955in}}%
\pgfpathlineto{\pgfqpoint{5.591202in}{2.964552in}}%
\pgfpathlineto{\pgfqpoint{5.609611in}{3.002703in}}%
\pgfpathlineto{\pgfqpoint{5.626840in}{3.038421in}}%
\pgfpathlineto{\pgfqpoint{5.642894in}{3.071718in}}%
\pgfpathlineto{\pgfqpoint{5.657779in}{3.102606in}}%
\pgfpathlineto{\pgfqpoint{5.671500in}{3.131097in}}%
\pgfpathlineto{\pgfqpoint{5.684065in}{3.157204in}}%
\pgfpathlineto{\pgfqpoint{5.695478in}{3.180938in}}%
\pgfpathlineto{\pgfqpoint{5.705745in}{3.202311in}}%
\pgfpathlineto{\pgfqpoint{5.714872in}{3.221335in}}%
\pgfpathlineto{\pgfqpoint{5.722865in}{3.238022in}}%
\pgfpathlineto{\pgfqpoint{5.729730in}{3.252382in}}%
\pgfpathlineto{\pgfqpoint{5.735471in}{3.264429in}}%
\pgfpathlineto{\pgfqpoint{5.740095in}{3.274173in}}%
\pgfpathlineto{\pgfqpoint{5.743607in}{3.281626in}}%
\pgfpathlineto{\pgfqpoint{5.746012in}{3.286800in}}%
\pgfpathlineto{\pgfqpoint{5.747318in}{3.289705in}}%
\pgfpathlineto{\pgfqpoint{5.747527in}{3.290354in}}%
\pgfpathlineto{\pgfqpoint{5.746647in}{3.288757in}}%
\pgfpathlineto{\pgfqpoint{5.744683in}{3.284926in}}%
\pgfpathlineto{\pgfqpoint{5.741640in}{3.278871in}}%
\pgfpathlineto{\pgfqpoint{5.737524in}{3.270605in}}%
\pgfpathlineto{\pgfqpoint{5.732339in}{3.260138in}}%
\pgfpathlineto{\pgfqpoint{5.726092in}{3.247480in}}%
\pgfpathlineto{\pgfqpoint{5.718787in}{3.232644in}}%
\pgfpathlineto{\pgfqpoint{5.710430in}{3.215640in}}%
\pgfpathlineto{\pgfqpoint{5.701026in}{3.196478in}}%
\pgfpathlineto{\pgfqpoint{5.690580in}{3.175170in}}%
\pgfpathlineto{\pgfqpoint{5.679098in}{3.151726in}}%
\pgfpathlineto{\pgfqpoint{5.666585in}{3.126157in}}%
\pgfpathlineto{\pgfqpoint{5.653045in}{3.098474in}}%
\pgfpathlineto{\pgfqpoint{5.638485in}{3.068687in}}%
\pgfpathlineto{\pgfqpoint{5.622908in}{3.036806in}}%
\pgfpathlineto{\pgfqpoint{5.606321in}{3.002843in}}%
\pgfpathlineto{\pgfqpoint{5.588728in}{2.966807in}}%
\pgfpathlineto{\pgfqpoint{5.570135in}{2.928709in}}%
\pgfpathlineto{\pgfqpoint{5.550545in}{2.888559in}}%
\pgfpathlineto{\pgfqpoint{5.529965in}{2.846368in}}%
\pgfpathlineto{\pgfqpoint{5.508399in}{2.802145in}}%
\pgfpathlineto{\pgfqpoint{5.485853in}{2.755901in}}%
\pgfpathlineto{\pgfqpoint{5.462330in}{2.707646in}}%
\pgfpathlineto{\pgfqpoint{5.437836in}{2.657390in}}%
\pgfpathlineto{\pgfqpoint{5.412376in}{2.605143in}}%
\pgfpathlineto{\pgfqpoint{5.385955in}{2.550915in}}%
\pgfpathlineto{\pgfqpoint{5.358577in}{2.494716in}}%
\pgfpathlineto{\pgfqpoint{5.330247in}{2.436556in}}%
\pgfpathlineto{\pgfqpoint{5.300970in}{2.376444in}}%
\pgfpathlineto{\pgfqpoint{5.270751in}{2.314390in}}%
\pgfpathlineto{\pgfqpoint{5.239594in}{2.250404in}}%
\pgfpathlineto{\pgfqpoint{5.207505in}{2.184496in}}%
\pgfpathlineto{\pgfqpoint{5.174487in}{2.116676in}}%
\pgfpathlineto{\pgfqpoint{5.140545in}{2.046952in}}%
\pgfpathlineto{\pgfqpoint{5.105685in}{1.975334in}}%
\pgfpathlineto{\pgfqpoint{5.069910in}{1.901832in}}%
\pgfpathlineto{\pgfqpoint{5.033225in}{1.826456in}}%
\pgfpathlineto{\pgfqpoint{4.995635in}{1.749214in}}%
\pgfpathlineto{\pgfqpoint{4.957145in}{1.670116in}}%
\pgfpathlineto{\pgfqpoint{4.917758in}{1.589171in}}%
\pgfpathlineto{\pgfqpoint{4.877479in}{1.506388in}}%
\pgfpathlineto{\pgfqpoint{4.836313in}{1.421777in}}%
\pgfpathlineto{\pgfqpoint{4.794264in}{1.335347in}}%
\pgfpathlineto{\pgfqpoint{4.751337in}{1.247107in}}%
\pgfpathlineto{\pgfqpoint{4.707536in}{1.157066in}}%
\pgfpathlineto{\pgfqpoint{4.662865in}{1.065233in}}%
\pgfpathlineto{\pgfqpoint{4.617328in}{0.971616in}}%
\pgfpathlineto{\pgfqpoint{4.570931in}{0.876226in}}%
\pgfpathlineto{\pgfqpoint{4.523677in}{0.779070in}}%
\pgfusepath{stroke}%
\end{pgfscope}%
\begin{pgfscope}%
\pgfpathrectangle{\pgfqpoint{1.250000in}{0.400000in}}{\pgfqpoint{7.750000in}{3.200000in}} %
\pgfusepath{clip}%
\pgfsetbuttcap%
\pgfsetroundjoin%
\definecolor{currentfill}{rgb}{0.000000,0.750000,0.750000}%
\pgfsetfillcolor{currentfill}%
\pgfsetlinewidth{0.501875pt}%
\definecolor{currentstroke}{rgb}{0.000000,0.750000,0.750000}%
\pgfsetstrokecolor{currentstroke}%
\pgfsetdash{}{0pt}%
\pgfsys@defobject{currentmarker}{\pgfqpoint{-0.041667in}{-0.041667in}}{\pgfqpoint{0.041667in}{0.041667in}}{%
\pgfpathmoveto{\pgfqpoint{-0.041667in}{0.000000in}}%
\pgfpathlineto{\pgfqpoint{0.041667in}{0.000000in}}%
\pgfpathmoveto{\pgfqpoint{0.000000in}{-0.041667in}}%
\pgfpathlineto{\pgfqpoint{0.000000in}{0.041667in}}%
\pgfusepath{stroke,fill}%
}%
\begin{pgfscope}%
\pgfsys@transformshift{4.571429in}{0.857143in}%
\pgfsys@useobject{currentmarker}{}%
\end{pgfscope}%
\begin{pgfscope}%
\pgfsys@transformshift{4.625420in}{0.968615in}%
\pgfsys@useobject{currentmarker}{}%
\end{pgfscope}%
\begin{pgfscope}%
\pgfsys@transformshift{4.678053in}{1.077287in}%
\pgfsys@useobject{currentmarker}{}%
\end{pgfscope}%
\begin{pgfscope}%
\pgfsys@transformshift{4.729334in}{1.183174in}%
\pgfsys@useobject{currentmarker}{}%
\end{pgfscope}%
\begin{pgfscope}%
\pgfsys@transformshift{4.779270in}{1.286289in}%
\pgfsys@useobject{currentmarker}{}%
\end{pgfscope}%
\begin{pgfscope}%
\pgfsys@transformshift{4.827868in}{1.386646in}%
\pgfsys@useobject{currentmarker}{}%
\end{pgfscope}%
\begin{pgfscope}%
\pgfsys@transformshift{4.875134in}{1.484259in}%
\pgfsys@useobject{currentmarker}{}%
\end{pgfscope}%
\begin{pgfscope}%
\pgfsys@transformshift{4.921075in}{1.579142in}%
\pgfsys@useobject{currentmarker}{}%
\end{pgfscope}%
\begin{pgfscope}%
\pgfsys@transformshift{4.965698in}{1.671308in}%
\pgfsys@useobject{currentmarker}{}%
\end{pgfscope}%
\begin{pgfscope}%
\pgfsys@transformshift{5.009009in}{1.760771in}%
\pgfsys@useobject{currentmarker}{}%
\end{pgfscope}%
\begin{pgfscope}%
\pgfsys@transformshift{5.051014in}{1.847544in}%
\pgfsys@useobject{currentmarker}{}%
\end{pgfscope}%
\begin{pgfscope}%
\pgfsys@transformshift{5.091721in}{1.931641in}%
\pgfsys@useobject{currentmarker}{}%
\end{pgfscope}%
\begin{pgfscope}%
\pgfsys@transformshift{5.131136in}{2.013075in}%
\pgfsys@useobject{currentmarker}{}%
\end{pgfscope}%
\begin{pgfscope}%
\pgfsys@transformshift{5.169264in}{2.091860in}%
\pgfsys@useobject{currentmarker}{}%
\end{pgfscope}%
\begin{pgfscope}%
\pgfsys@transformshift{5.206114in}{2.168009in}%
\pgfsys@useobject{currentmarker}{}%
\end{pgfscope}%
\begin{pgfscope}%
\pgfsys@transformshift{5.241691in}{2.241535in}%
\pgfsys@useobject{currentmarker}{}%
\end{pgfscope}%
\begin{pgfscope}%
\pgfsys@transformshift{5.276001in}{2.312451in}%
\pgfsys@useobject{currentmarker}{}%
\end{pgfscope}%
\begin{pgfscope}%
\pgfsys@transformshift{5.309050in}{2.380769in}%
\pgfsys@useobject{currentmarker}{}%
\end{pgfscope}%
\begin{pgfscope}%
\pgfsys@transformshift{5.340846in}{2.446504in}%
\pgfsys@useobject{currentmarker}{}%
\end{pgfscope}%
\begin{pgfscope}%
\pgfsys@transformshift{5.371394in}{2.509668in}%
\pgfsys@useobject{currentmarker}{}%
\end{pgfscope}%
\begin{pgfscope}%
\pgfsys@transformshift{5.400701in}{2.570274in}%
\pgfsys@useobject{currentmarker}{}%
\end{pgfscope}%
\begin{pgfscope}%
\pgfsys@transformshift{5.428772in}{2.628335in}%
\pgfsys@useobject{currentmarker}{}%
\end{pgfscope}%
\begin{pgfscope}%
\pgfsys@transformshift{5.455615in}{2.683863in}%
\pgfsys@useobject{currentmarker}{}%
\end{pgfscope}%
\begin{pgfscope}%
\pgfsys@transformshift{5.481234in}{2.736871in}%
\pgfsys@useobject{currentmarker}{}%
\end{pgfscope}%
\begin{pgfscope}%
\pgfsys@transformshift{5.505637in}{2.787372in}%
\pgfsys@useobject{currentmarker}{}%
\end{pgfscope}%
\begin{pgfscope}%
\pgfsys@transformshift{5.528829in}{2.835378in}%
\pgfsys@useobject{currentmarker}{}%
\end{pgfscope}%
\begin{pgfscope}%
\pgfsys@transformshift{5.550817in}{2.880902in}%
\pgfsys@useobject{currentmarker}{}%
\end{pgfscope}%
\begin{pgfscope}%
\pgfsys@transformshift{5.571605in}{2.923955in}%
\pgfsys@useobject{currentmarker}{}%
\end{pgfscope}%
\begin{pgfscope}%
\pgfsys@transformshift{5.591202in}{2.964552in}%
\pgfsys@useobject{currentmarker}{}%
\end{pgfscope}%
\begin{pgfscope}%
\pgfsys@transformshift{5.609611in}{3.002703in}%
\pgfsys@useobject{currentmarker}{}%
\end{pgfscope}%
\begin{pgfscope}%
\pgfsys@transformshift{5.626840in}{3.038421in}%
\pgfsys@useobject{currentmarker}{}%
\end{pgfscope}%
\begin{pgfscope}%
\pgfsys@transformshift{5.642894in}{3.071718in}%
\pgfsys@useobject{currentmarker}{}%
\end{pgfscope}%
\begin{pgfscope}%
\pgfsys@transformshift{5.657779in}{3.102606in}%
\pgfsys@useobject{currentmarker}{}%
\end{pgfscope}%
\begin{pgfscope}%
\pgfsys@transformshift{5.671500in}{3.131097in}%
\pgfsys@useobject{currentmarker}{}%
\end{pgfscope}%
\begin{pgfscope}%
\pgfsys@transformshift{5.684065in}{3.157204in}%
\pgfsys@useobject{currentmarker}{}%
\end{pgfscope}%
\begin{pgfscope}%
\pgfsys@transformshift{5.695478in}{3.180938in}%
\pgfsys@useobject{currentmarker}{}%
\end{pgfscope}%
\begin{pgfscope}%
\pgfsys@transformshift{5.705745in}{3.202311in}%
\pgfsys@useobject{currentmarker}{}%
\end{pgfscope}%
\begin{pgfscope}%
\pgfsys@transformshift{5.714872in}{3.221335in}%
\pgfsys@useobject{currentmarker}{}%
\end{pgfscope}%
\begin{pgfscope}%
\pgfsys@transformshift{5.722865in}{3.238022in}%
\pgfsys@useobject{currentmarker}{}%
\end{pgfscope}%
\begin{pgfscope}%
\pgfsys@transformshift{5.729730in}{3.252382in}%
\pgfsys@useobject{currentmarker}{}%
\end{pgfscope}%
\begin{pgfscope}%
\pgfsys@transformshift{5.735471in}{3.264429in}%
\pgfsys@useobject{currentmarker}{}%
\end{pgfscope}%
\begin{pgfscope}%
\pgfsys@transformshift{5.740095in}{3.274173in}%
\pgfsys@useobject{currentmarker}{}%
\end{pgfscope}%
\begin{pgfscope}%
\pgfsys@transformshift{5.743607in}{3.281626in}%
\pgfsys@useobject{currentmarker}{}%
\end{pgfscope}%
\begin{pgfscope}%
\pgfsys@transformshift{5.746012in}{3.286800in}%
\pgfsys@useobject{currentmarker}{}%
\end{pgfscope}%
\begin{pgfscope}%
\pgfsys@transformshift{5.747318in}{3.289705in}%
\pgfsys@useobject{currentmarker}{}%
\end{pgfscope}%
\begin{pgfscope}%
\pgfsys@transformshift{5.747527in}{3.290354in}%
\pgfsys@useobject{currentmarker}{}%
\end{pgfscope}%
\begin{pgfscope}%
\pgfsys@transformshift{5.746647in}{3.288757in}%
\pgfsys@useobject{currentmarker}{}%
\end{pgfscope}%
\begin{pgfscope}%
\pgfsys@transformshift{5.744683in}{3.284926in}%
\pgfsys@useobject{currentmarker}{}%
\end{pgfscope}%
\begin{pgfscope}%
\pgfsys@transformshift{5.741640in}{3.278871in}%
\pgfsys@useobject{currentmarker}{}%
\end{pgfscope}%
\begin{pgfscope}%
\pgfsys@transformshift{5.737524in}{3.270605in}%
\pgfsys@useobject{currentmarker}{}%
\end{pgfscope}%
\begin{pgfscope}%
\pgfsys@transformshift{5.732339in}{3.260138in}%
\pgfsys@useobject{currentmarker}{}%
\end{pgfscope}%
\begin{pgfscope}%
\pgfsys@transformshift{5.726092in}{3.247480in}%
\pgfsys@useobject{currentmarker}{}%
\end{pgfscope}%
\begin{pgfscope}%
\pgfsys@transformshift{5.718787in}{3.232644in}%
\pgfsys@useobject{currentmarker}{}%
\end{pgfscope}%
\begin{pgfscope}%
\pgfsys@transformshift{5.710430in}{3.215640in}%
\pgfsys@useobject{currentmarker}{}%
\end{pgfscope}%
\begin{pgfscope}%
\pgfsys@transformshift{5.701026in}{3.196478in}%
\pgfsys@useobject{currentmarker}{}%
\end{pgfscope}%
\begin{pgfscope}%
\pgfsys@transformshift{5.690580in}{3.175170in}%
\pgfsys@useobject{currentmarker}{}%
\end{pgfscope}%
\begin{pgfscope}%
\pgfsys@transformshift{5.679098in}{3.151726in}%
\pgfsys@useobject{currentmarker}{}%
\end{pgfscope}%
\begin{pgfscope}%
\pgfsys@transformshift{5.666585in}{3.126157in}%
\pgfsys@useobject{currentmarker}{}%
\end{pgfscope}%
\begin{pgfscope}%
\pgfsys@transformshift{5.653045in}{3.098474in}%
\pgfsys@useobject{currentmarker}{}%
\end{pgfscope}%
\begin{pgfscope}%
\pgfsys@transformshift{5.638485in}{3.068687in}%
\pgfsys@useobject{currentmarker}{}%
\end{pgfscope}%
\begin{pgfscope}%
\pgfsys@transformshift{5.622908in}{3.036806in}%
\pgfsys@useobject{currentmarker}{}%
\end{pgfscope}%
\begin{pgfscope}%
\pgfsys@transformshift{5.606321in}{3.002843in}%
\pgfsys@useobject{currentmarker}{}%
\end{pgfscope}%
\begin{pgfscope}%
\pgfsys@transformshift{5.588728in}{2.966807in}%
\pgfsys@useobject{currentmarker}{}%
\end{pgfscope}%
\begin{pgfscope}%
\pgfsys@transformshift{5.570135in}{2.928709in}%
\pgfsys@useobject{currentmarker}{}%
\end{pgfscope}%
\begin{pgfscope}%
\pgfsys@transformshift{5.550545in}{2.888559in}%
\pgfsys@useobject{currentmarker}{}%
\end{pgfscope}%
\begin{pgfscope}%
\pgfsys@transformshift{5.529965in}{2.846368in}%
\pgfsys@useobject{currentmarker}{}%
\end{pgfscope}%
\begin{pgfscope}%
\pgfsys@transformshift{5.508399in}{2.802145in}%
\pgfsys@useobject{currentmarker}{}%
\end{pgfscope}%
\begin{pgfscope}%
\pgfsys@transformshift{5.485853in}{2.755901in}%
\pgfsys@useobject{currentmarker}{}%
\end{pgfscope}%
\begin{pgfscope}%
\pgfsys@transformshift{5.462330in}{2.707646in}%
\pgfsys@useobject{currentmarker}{}%
\end{pgfscope}%
\begin{pgfscope}%
\pgfsys@transformshift{5.437836in}{2.657390in}%
\pgfsys@useobject{currentmarker}{}%
\end{pgfscope}%
\begin{pgfscope}%
\pgfsys@transformshift{5.412376in}{2.605143in}%
\pgfsys@useobject{currentmarker}{}%
\end{pgfscope}%
\begin{pgfscope}%
\pgfsys@transformshift{5.385955in}{2.550915in}%
\pgfsys@useobject{currentmarker}{}%
\end{pgfscope}%
\begin{pgfscope}%
\pgfsys@transformshift{5.358577in}{2.494716in}%
\pgfsys@useobject{currentmarker}{}%
\end{pgfscope}%
\begin{pgfscope}%
\pgfsys@transformshift{5.330247in}{2.436556in}%
\pgfsys@useobject{currentmarker}{}%
\end{pgfscope}%
\begin{pgfscope}%
\pgfsys@transformshift{5.300970in}{2.376444in}%
\pgfsys@useobject{currentmarker}{}%
\end{pgfscope}%
\begin{pgfscope}%
\pgfsys@transformshift{5.270751in}{2.314390in}%
\pgfsys@useobject{currentmarker}{}%
\end{pgfscope}%
\begin{pgfscope}%
\pgfsys@transformshift{5.239594in}{2.250404in}%
\pgfsys@useobject{currentmarker}{}%
\end{pgfscope}%
\begin{pgfscope}%
\pgfsys@transformshift{5.207505in}{2.184496in}%
\pgfsys@useobject{currentmarker}{}%
\end{pgfscope}%
\begin{pgfscope}%
\pgfsys@transformshift{5.174487in}{2.116676in}%
\pgfsys@useobject{currentmarker}{}%
\end{pgfscope}%
\begin{pgfscope}%
\pgfsys@transformshift{5.140545in}{2.046952in}%
\pgfsys@useobject{currentmarker}{}%
\end{pgfscope}%
\begin{pgfscope}%
\pgfsys@transformshift{5.105685in}{1.975334in}%
\pgfsys@useobject{currentmarker}{}%
\end{pgfscope}%
\begin{pgfscope}%
\pgfsys@transformshift{5.069910in}{1.901832in}%
\pgfsys@useobject{currentmarker}{}%
\end{pgfscope}%
\begin{pgfscope}%
\pgfsys@transformshift{5.033225in}{1.826456in}%
\pgfsys@useobject{currentmarker}{}%
\end{pgfscope}%
\begin{pgfscope}%
\pgfsys@transformshift{4.995635in}{1.749214in}%
\pgfsys@useobject{currentmarker}{}%
\end{pgfscope}%
\begin{pgfscope}%
\pgfsys@transformshift{4.957145in}{1.670116in}%
\pgfsys@useobject{currentmarker}{}%
\end{pgfscope}%
\begin{pgfscope}%
\pgfsys@transformshift{4.917758in}{1.589171in}%
\pgfsys@useobject{currentmarker}{}%
\end{pgfscope}%
\begin{pgfscope}%
\pgfsys@transformshift{4.877479in}{1.506388in}%
\pgfsys@useobject{currentmarker}{}%
\end{pgfscope}%
\begin{pgfscope}%
\pgfsys@transformshift{4.836313in}{1.421777in}%
\pgfsys@useobject{currentmarker}{}%
\end{pgfscope}%
\begin{pgfscope}%
\pgfsys@transformshift{4.794264in}{1.335347in}%
\pgfsys@useobject{currentmarker}{}%
\end{pgfscope}%
\begin{pgfscope}%
\pgfsys@transformshift{4.751337in}{1.247107in}%
\pgfsys@useobject{currentmarker}{}%
\end{pgfscope}%
\begin{pgfscope}%
\pgfsys@transformshift{4.707536in}{1.157066in}%
\pgfsys@useobject{currentmarker}{}%
\end{pgfscope}%
\begin{pgfscope}%
\pgfsys@transformshift{4.662865in}{1.065233in}%
\pgfsys@useobject{currentmarker}{}%
\end{pgfscope}%
\begin{pgfscope}%
\pgfsys@transformshift{4.617328in}{0.971616in}%
\pgfsys@useobject{currentmarker}{}%
\end{pgfscope}%
\begin{pgfscope}%
\pgfsys@transformshift{4.570931in}{0.876226in}%
\pgfsys@useobject{currentmarker}{}%
\end{pgfscope}%
\begin{pgfscope}%
\pgfsys@transformshift{4.523677in}{0.779070in}%
\pgfsys@useobject{currentmarker}{}%
\end{pgfscope}%
\end{pgfscope}%
\begin{pgfscope}%
\pgfpathrectangle{\pgfqpoint{1.250000in}{0.400000in}}{\pgfqpoint{7.750000in}{3.200000in}} %
\pgfusepath{clip}%
\pgfsetrectcap%
\pgfsetroundjoin%
\pgfsetlinewidth{1.003750pt}%
\definecolor{currentstroke}{rgb}{0.750000,0.000000,0.750000}%
\pgfsetstrokecolor{currentstroke}%
\pgfsetdash{}{0pt}%
\pgfpathmoveto{\pgfqpoint{4.571429in}{0.857143in}}%
\pgfpathlineto{\pgfqpoint{4.625611in}{0.968615in}}%
\pgfpathlineto{\pgfqpoint{4.678624in}{1.077287in}}%
\pgfpathlineto{\pgfqpoint{4.730475in}{1.183174in}}%
\pgfpathlineto{\pgfqpoint{4.781168in}{1.286289in}}%
\pgfpathlineto{\pgfqpoint{4.830709in}{1.386646in}}%
\pgfpathlineto{\pgfqpoint{4.879105in}{1.484259in}}%
\pgfpathlineto{\pgfqpoint{4.926361in}{1.579142in}}%
\pgfpathlineto{\pgfqpoint{4.972483in}{1.671308in}}%
\pgfpathlineto{\pgfqpoint{5.017476in}{1.760771in}}%
\pgfpathlineto{\pgfqpoint{5.061345in}{1.847544in}}%
\pgfpathlineto{\pgfqpoint{5.104098in}{1.931641in}}%
\pgfpathlineto{\pgfqpoint{5.145739in}{2.013075in}}%
\pgfpathlineto{\pgfqpoint{5.186274in}{2.091860in}}%
\pgfpathlineto{\pgfqpoint{5.225707in}{2.168009in}}%
\pgfpathlineto{\pgfqpoint{5.264046in}{2.241535in}}%
\pgfpathlineto{\pgfqpoint{5.301295in}{2.312451in}}%
\pgfpathlineto{\pgfqpoint{5.337460in}{2.380769in}}%
\pgfpathlineto{\pgfqpoint{5.372546in}{2.446504in}}%
\pgfpathlineto{\pgfqpoint{5.406558in}{2.509668in}}%
\pgfpathlineto{\pgfqpoint{5.439503in}{2.570274in}}%
\pgfpathlineto{\pgfqpoint{5.471384in}{2.628335in}}%
\pgfpathlineto{\pgfqpoint{5.502209in}{2.683863in}}%
\pgfpathlineto{\pgfqpoint{5.531981in}{2.736871in}}%
\pgfpathlineto{\pgfqpoint{5.560706in}{2.787372in}}%
\pgfpathlineto{\pgfqpoint{5.588389in}{2.835378in}}%
\pgfpathlineto{\pgfqpoint{5.615037in}{2.880902in}}%
\pgfpathlineto{\pgfqpoint{5.640652in}{2.923955in}}%
\pgfpathlineto{\pgfqpoint{5.665242in}{2.964552in}}%
\pgfpathlineto{\pgfqpoint{5.688811in}{3.002703in}}%
\pgfpathlineto{\pgfqpoint{5.711364in}{3.038421in}}%
\pgfpathlineto{\pgfqpoint{5.732906in}{3.071718in}}%
\pgfpathlineto{\pgfqpoint{5.753443in}{3.102606in}}%
\pgfpathlineto{\pgfqpoint{5.772978in}{3.131097in}}%
\pgfpathlineto{\pgfqpoint{5.791519in}{3.157204in}}%
\pgfpathlineto{\pgfqpoint{5.809068in}{3.180938in}}%
\pgfpathlineto{\pgfqpoint{5.825631in}{3.202311in}}%
\pgfpathlineto{\pgfqpoint{5.841214in}{3.221335in}}%
\pgfpathlineto{\pgfqpoint{5.855821in}{3.238022in}}%
\pgfpathlineto{\pgfqpoint{5.869457in}{3.252382in}}%
\pgfpathlineto{\pgfqpoint{5.882126in}{3.264429in}}%
\pgfpathlineto{\pgfqpoint{5.893835in}{3.274173in}}%
\pgfpathlineto{\pgfqpoint{5.904586in}{3.281626in}}%
\pgfpathlineto{\pgfqpoint{5.914386in}{3.286800in}}%
\pgfpathlineto{\pgfqpoint{5.923239in}{3.289705in}}%
\pgfpathlineto{\pgfqpoint{5.931150in}{3.290354in}}%
\pgfpathlineto{\pgfqpoint{5.938123in}{3.288757in}}%
\pgfpathlineto{\pgfqpoint{5.944163in}{3.284926in}}%
\pgfpathlineto{\pgfqpoint{5.949275in}{3.278871in}}%
\pgfpathlineto{\pgfqpoint{5.953463in}{3.270605in}}%
\pgfpathlineto{\pgfqpoint{5.956732in}{3.260138in}}%
\pgfpathlineto{\pgfqpoint{5.959088in}{3.247480in}}%
\pgfpathlineto{\pgfqpoint{5.960533in}{3.232644in}}%
\pgfpathlineto{\pgfqpoint{5.961073in}{3.215640in}}%
\pgfpathlineto{\pgfqpoint{5.960712in}{3.196478in}}%
\pgfpathlineto{\pgfqpoint{5.959455in}{3.175170in}}%
\pgfpathlineto{\pgfqpoint{5.957306in}{3.151726in}}%
\pgfpathlineto{\pgfqpoint{5.954270in}{3.126157in}}%
\pgfpathlineto{\pgfqpoint{5.950352in}{3.098474in}}%
\pgfpathlineto{\pgfqpoint{5.945555in}{3.068687in}}%
\pgfpathlineto{\pgfqpoint{5.939883in}{3.036806in}}%
\pgfpathlineto{\pgfqpoint{5.933342in}{3.002843in}}%
\pgfpathlineto{\pgfqpoint{5.925936in}{2.966807in}}%
\pgfpathlineto{\pgfqpoint{5.917669in}{2.928709in}}%
\pgfpathlineto{\pgfqpoint{5.908545in}{2.888559in}}%
\pgfpathlineto{\pgfqpoint{5.898569in}{2.846368in}}%
\pgfpathlineto{\pgfqpoint{5.887745in}{2.802145in}}%
\pgfpathlineto{\pgfqpoint{5.876076in}{2.755901in}}%
\pgfpathlineto{\pgfqpoint{5.863568in}{2.707646in}}%
\pgfpathlineto{\pgfqpoint{5.850225in}{2.657390in}}%
\pgfpathlineto{\pgfqpoint{5.836050in}{2.605143in}}%
\pgfpathlineto{\pgfqpoint{5.821048in}{2.550915in}}%
\pgfpathlineto{\pgfqpoint{5.805223in}{2.494716in}}%
\pgfpathlineto{\pgfqpoint{5.788580in}{2.436556in}}%
\pgfpathlineto{\pgfqpoint{5.771121in}{2.376444in}}%
\pgfpathlineto{\pgfqpoint{5.752852in}{2.314390in}}%
\pgfpathlineto{\pgfqpoint{5.733776in}{2.250404in}}%
\pgfpathlineto{\pgfqpoint{5.713897in}{2.184496in}}%
\pgfpathlineto{\pgfqpoint{5.693220in}{2.116676in}}%
\pgfpathlineto{\pgfqpoint{5.671748in}{2.046952in}}%
\pgfpathlineto{\pgfqpoint{5.649486in}{1.975334in}}%
\pgfpathlineto{\pgfqpoint{5.626436in}{1.901832in}}%
\pgfpathlineto{\pgfqpoint{5.602604in}{1.826456in}}%
\pgfpathlineto{\pgfqpoint{5.577994in}{1.749214in}}%
\pgfpathlineto{\pgfqpoint{5.552608in}{1.670116in}}%
\pgfpathlineto{\pgfqpoint{5.526451in}{1.589171in}}%
\pgfpathlineto{\pgfqpoint{5.499527in}{1.506388in}}%
\pgfpathlineto{\pgfqpoint{5.471839in}{1.421777in}}%
\pgfpathlineto{\pgfqpoint{5.443392in}{1.335347in}}%
\pgfpathlineto{\pgfqpoint{5.414189in}{1.247107in}}%
\pgfpathlineto{\pgfqpoint{5.384234in}{1.157066in}}%
\pgfpathlineto{\pgfqpoint{5.353531in}{1.065233in}}%
\pgfpathlineto{\pgfqpoint{5.322084in}{0.971616in}}%
\pgfpathlineto{\pgfqpoint{5.289895in}{0.876226in}}%
\pgfpathlineto{\pgfqpoint{5.256970in}{0.779070in}}%
\pgfusepath{stroke}%
\end{pgfscope}%
\begin{pgfscope}%
\pgfpathrectangle{\pgfqpoint{1.250000in}{0.400000in}}{\pgfqpoint{7.750000in}{3.200000in}} %
\pgfusepath{clip}%
\pgfsetbuttcap%
\pgfsetroundjoin%
\definecolor{currentfill}{rgb}{0.750000,0.000000,0.750000}%
\pgfsetfillcolor{currentfill}%
\pgfsetlinewidth{0.501875pt}%
\definecolor{currentstroke}{rgb}{0.750000,0.000000,0.750000}%
\pgfsetstrokecolor{currentstroke}%
\pgfsetdash{}{0pt}%
\pgfsys@defobject{currentmarker}{\pgfqpoint{-0.041667in}{-0.041667in}}{\pgfqpoint{0.041667in}{0.041667in}}{%
\pgfpathmoveto{\pgfqpoint{-0.041667in}{0.000000in}}%
\pgfpathlineto{\pgfqpoint{0.041667in}{0.000000in}}%
\pgfpathmoveto{\pgfqpoint{0.000000in}{-0.041667in}}%
\pgfpathlineto{\pgfqpoint{0.000000in}{0.041667in}}%
\pgfusepath{stroke,fill}%
}%
\begin{pgfscope}%
\pgfsys@transformshift{4.571429in}{0.857143in}%
\pgfsys@useobject{currentmarker}{}%
\end{pgfscope}%
\begin{pgfscope}%
\pgfsys@transformshift{4.625611in}{0.968615in}%
\pgfsys@useobject{currentmarker}{}%
\end{pgfscope}%
\begin{pgfscope}%
\pgfsys@transformshift{4.678624in}{1.077287in}%
\pgfsys@useobject{currentmarker}{}%
\end{pgfscope}%
\begin{pgfscope}%
\pgfsys@transformshift{4.730475in}{1.183174in}%
\pgfsys@useobject{currentmarker}{}%
\end{pgfscope}%
\begin{pgfscope}%
\pgfsys@transformshift{4.781168in}{1.286289in}%
\pgfsys@useobject{currentmarker}{}%
\end{pgfscope}%
\begin{pgfscope}%
\pgfsys@transformshift{4.830709in}{1.386646in}%
\pgfsys@useobject{currentmarker}{}%
\end{pgfscope}%
\begin{pgfscope}%
\pgfsys@transformshift{4.879105in}{1.484259in}%
\pgfsys@useobject{currentmarker}{}%
\end{pgfscope}%
\begin{pgfscope}%
\pgfsys@transformshift{4.926361in}{1.579142in}%
\pgfsys@useobject{currentmarker}{}%
\end{pgfscope}%
\begin{pgfscope}%
\pgfsys@transformshift{4.972483in}{1.671308in}%
\pgfsys@useobject{currentmarker}{}%
\end{pgfscope}%
\begin{pgfscope}%
\pgfsys@transformshift{5.017476in}{1.760771in}%
\pgfsys@useobject{currentmarker}{}%
\end{pgfscope}%
\begin{pgfscope}%
\pgfsys@transformshift{5.061345in}{1.847544in}%
\pgfsys@useobject{currentmarker}{}%
\end{pgfscope}%
\begin{pgfscope}%
\pgfsys@transformshift{5.104098in}{1.931641in}%
\pgfsys@useobject{currentmarker}{}%
\end{pgfscope}%
\begin{pgfscope}%
\pgfsys@transformshift{5.145739in}{2.013075in}%
\pgfsys@useobject{currentmarker}{}%
\end{pgfscope}%
\begin{pgfscope}%
\pgfsys@transformshift{5.186274in}{2.091860in}%
\pgfsys@useobject{currentmarker}{}%
\end{pgfscope}%
\begin{pgfscope}%
\pgfsys@transformshift{5.225707in}{2.168009in}%
\pgfsys@useobject{currentmarker}{}%
\end{pgfscope}%
\begin{pgfscope}%
\pgfsys@transformshift{5.264046in}{2.241535in}%
\pgfsys@useobject{currentmarker}{}%
\end{pgfscope}%
\begin{pgfscope}%
\pgfsys@transformshift{5.301295in}{2.312451in}%
\pgfsys@useobject{currentmarker}{}%
\end{pgfscope}%
\begin{pgfscope}%
\pgfsys@transformshift{5.337460in}{2.380769in}%
\pgfsys@useobject{currentmarker}{}%
\end{pgfscope}%
\begin{pgfscope}%
\pgfsys@transformshift{5.372546in}{2.446504in}%
\pgfsys@useobject{currentmarker}{}%
\end{pgfscope}%
\begin{pgfscope}%
\pgfsys@transformshift{5.406558in}{2.509668in}%
\pgfsys@useobject{currentmarker}{}%
\end{pgfscope}%
\begin{pgfscope}%
\pgfsys@transformshift{5.439503in}{2.570274in}%
\pgfsys@useobject{currentmarker}{}%
\end{pgfscope}%
\begin{pgfscope}%
\pgfsys@transformshift{5.471384in}{2.628335in}%
\pgfsys@useobject{currentmarker}{}%
\end{pgfscope}%
\begin{pgfscope}%
\pgfsys@transformshift{5.502209in}{2.683863in}%
\pgfsys@useobject{currentmarker}{}%
\end{pgfscope}%
\begin{pgfscope}%
\pgfsys@transformshift{5.531981in}{2.736871in}%
\pgfsys@useobject{currentmarker}{}%
\end{pgfscope}%
\begin{pgfscope}%
\pgfsys@transformshift{5.560706in}{2.787372in}%
\pgfsys@useobject{currentmarker}{}%
\end{pgfscope}%
\begin{pgfscope}%
\pgfsys@transformshift{5.588389in}{2.835378in}%
\pgfsys@useobject{currentmarker}{}%
\end{pgfscope}%
\begin{pgfscope}%
\pgfsys@transformshift{5.615037in}{2.880902in}%
\pgfsys@useobject{currentmarker}{}%
\end{pgfscope}%
\begin{pgfscope}%
\pgfsys@transformshift{5.640652in}{2.923955in}%
\pgfsys@useobject{currentmarker}{}%
\end{pgfscope}%
\begin{pgfscope}%
\pgfsys@transformshift{5.665242in}{2.964552in}%
\pgfsys@useobject{currentmarker}{}%
\end{pgfscope}%
\begin{pgfscope}%
\pgfsys@transformshift{5.688811in}{3.002703in}%
\pgfsys@useobject{currentmarker}{}%
\end{pgfscope}%
\begin{pgfscope}%
\pgfsys@transformshift{5.711364in}{3.038421in}%
\pgfsys@useobject{currentmarker}{}%
\end{pgfscope}%
\begin{pgfscope}%
\pgfsys@transformshift{5.732906in}{3.071718in}%
\pgfsys@useobject{currentmarker}{}%
\end{pgfscope}%
\begin{pgfscope}%
\pgfsys@transformshift{5.753443in}{3.102606in}%
\pgfsys@useobject{currentmarker}{}%
\end{pgfscope}%
\begin{pgfscope}%
\pgfsys@transformshift{5.772978in}{3.131097in}%
\pgfsys@useobject{currentmarker}{}%
\end{pgfscope}%
\begin{pgfscope}%
\pgfsys@transformshift{5.791519in}{3.157204in}%
\pgfsys@useobject{currentmarker}{}%
\end{pgfscope}%
\begin{pgfscope}%
\pgfsys@transformshift{5.809068in}{3.180938in}%
\pgfsys@useobject{currentmarker}{}%
\end{pgfscope}%
\begin{pgfscope}%
\pgfsys@transformshift{5.825631in}{3.202311in}%
\pgfsys@useobject{currentmarker}{}%
\end{pgfscope}%
\begin{pgfscope}%
\pgfsys@transformshift{5.841214in}{3.221335in}%
\pgfsys@useobject{currentmarker}{}%
\end{pgfscope}%
\begin{pgfscope}%
\pgfsys@transformshift{5.855821in}{3.238022in}%
\pgfsys@useobject{currentmarker}{}%
\end{pgfscope}%
\begin{pgfscope}%
\pgfsys@transformshift{5.869457in}{3.252382in}%
\pgfsys@useobject{currentmarker}{}%
\end{pgfscope}%
\begin{pgfscope}%
\pgfsys@transformshift{5.882126in}{3.264429in}%
\pgfsys@useobject{currentmarker}{}%
\end{pgfscope}%
\begin{pgfscope}%
\pgfsys@transformshift{5.893835in}{3.274173in}%
\pgfsys@useobject{currentmarker}{}%
\end{pgfscope}%
\begin{pgfscope}%
\pgfsys@transformshift{5.904586in}{3.281626in}%
\pgfsys@useobject{currentmarker}{}%
\end{pgfscope}%
\begin{pgfscope}%
\pgfsys@transformshift{5.914386in}{3.286800in}%
\pgfsys@useobject{currentmarker}{}%
\end{pgfscope}%
\begin{pgfscope}%
\pgfsys@transformshift{5.923239in}{3.289705in}%
\pgfsys@useobject{currentmarker}{}%
\end{pgfscope}%
\begin{pgfscope}%
\pgfsys@transformshift{5.931150in}{3.290354in}%
\pgfsys@useobject{currentmarker}{}%
\end{pgfscope}%
\begin{pgfscope}%
\pgfsys@transformshift{5.938123in}{3.288757in}%
\pgfsys@useobject{currentmarker}{}%
\end{pgfscope}%
\begin{pgfscope}%
\pgfsys@transformshift{5.944163in}{3.284926in}%
\pgfsys@useobject{currentmarker}{}%
\end{pgfscope}%
\begin{pgfscope}%
\pgfsys@transformshift{5.949275in}{3.278871in}%
\pgfsys@useobject{currentmarker}{}%
\end{pgfscope}%
\begin{pgfscope}%
\pgfsys@transformshift{5.953463in}{3.270605in}%
\pgfsys@useobject{currentmarker}{}%
\end{pgfscope}%
\begin{pgfscope}%
\pgfsys@transformshift{5.956732in}{3.260138in}%
\pgfsys@useobject{currentmarker}{}%
\end{pgfscope}%
\begin{pgfscope}%
\pgfsys@transformshift{5.959088in}{3.247480in}%
\pgfsys@useobject{currentmarker}{}%
\end{pgfscope}%
\begin{pgfscope}%
\pgfsys@transformshift{5.960533in}{3.232644in}%
\pgfsys@useobject{currentmarker}{}%
\end{pgfscope}%
\begin{pgfscope}%
\pgfsys@transformshift{5.961073in}{3.215640in}%
\pgfsys@useobject{currentmarker}{}%
\end{pgfscope}%
\begin{pgfscope}%
\pgfsys@transformshift{5.960712in}{3.196478in}%
\pgfsys@useobject{currentmarker}{}%
\end{pgfscope}%
\begin{pgfscope}%
\pgfsys@transformshift{5.959455in}{3.175170in}%
\pgfsys@useobject{currentmarker}{}%
\end{pgfscope}%
\begin{pgfscope}%
\pgfsys@transformshift{5.957306in}{3.151726in}%
\pgfsys@useobject{currentmarker}{}%
\end{pgfscope}%
\begin{pgfscope}%
\pgfsys@transformshift{5.954270in}{3.126157in}%
\pgfsys@useobject{currentmarker}{}%
\end{pgfscope}%
\begin{pgfscope}%
\pgfsys@transformshift{5.950352in}{3.098474in}%
\pgfsys@useobject{currentmarker}{}%
\end{pgfscope}%
\begin{pgfscope}%
\pgfsys@transformshift{5.945555in}{3.068687in}%
\pgfsys@useobject{currentmarker}{}%
\end{pgfscope}%
\begin{pgfscope}%
\pgfsys@transformshift{5.939883in}{3.036806in}%
\pgfsys@useobject{currentmarker}{}%
\end{pgfscope}%
\begin{pgfscope}%
\pgfsys@transformshift{5.933342in}{3.002843in}%
\pgfsys@useobject{currentmarker}{}%
\end{pgfscope}%
\begin{pgfscope}%
\pgfsys@transformshift{5.925936in}{2.966807in}%
\pgfsys@useobject{currentmarker}{}%
\end{pgfscope}%
\begin{pgfscope}%
\pgfsys@transformshift{5.917669in}{2.928709in}%
\pgfsys@useobject{currentmarker}{}%
\end{pgfscope}%
\begin{pgfscope}%
\pgfsys@transformshift{5.908545in}{2.888559in}%
\pgfsys@useobject{currentmarker}{}%
\end{pgfscope}%
\begin{pgfscope}%
\pgfsys@transformshift{5.898569in}{2.846368in}%
\pgfsys@useobject{currentmarker}{}%
\end{pgfscope}%
\begin{pgfscope}%
\pgfsys@transformshift{5.887745in}{2.802145in}%
\pgfsys@useobject{currentmarker}{}%
\end{pgfscope}%
\begin{pgfscope}%
\pgfsys@transformshift{5.876076in}{2.755901in}%
\pgfsys@useobject{currentmarker}{}%
\end{pgfscope}%
\begin{pgfscope}%
\pgfsys@transformshift{5.863568in}{2.707646in}%
\pgfsys@useobject{currentmarker}{}%
\end{pgfscope}%
\begin{pgfscope}%
\pgfsys@transformshift{5.850225in}{2.657390in}%
\pgfsys@useobject{currentmarker}{}%
\end{pgfscope}%
\begin{pgfscope}%
\pgfsys@transformshift{5.836050in}{2.605143in}%
\pgfsys@useobject{currentmarker}{}%
\end{pgfscope}%
\begin{pgfscope}%
\pgfsys@transformshift{5.821048in}{2.550915in}%
\pgfsys@useobject{currentmarker}{}%
\end{pgfscope}%
\begin{pgfscope}%
\pgfsys@transformshift{5.805223in}{2.494716in}%
\pgfsys@useobject{currentmarker}{}%
\end{pgfscope}%
\begin{pgfscope}%
\pgfsys@transformshift{5.788580in}{2.436556in}%
\pgfsys@useobject{currentmarker}{}%
\end{pgfscope}%
\begin{pgfscope}%
\pgfsys@transformshift{5.771121in}{2.376444in}%
\pgfsys@useobject{currentmarker}{}%
\end{pgfscope}%
\begin{pgfscope}%
\pgfsys@transformshift{5.752852in}{2.314390in}%
\pgfsys@useobject{currentmarker}{}%
\end{pgfscope}%
\begin{pgfscope}%
\pgfsys@transformshift{5.733776in}{2.250404in}%
\pgfsys@useobject{currentmarker}{}%
\end{pgfscope}%
\begin{pgfscope}%
\pgfsys@transformshift{5.713897in}{2.184496in}%
\pgfsys@useobject{currentmarker}{}%
\end{pgfscope}%
\begin{pgfscope}%
\pgfsys@transformshift{5.693220in}{2.116676in}%
\pgfsys@useobject{currentmarker}{}%
\end{pgfscope}%
\begin{pgfscope}%
\pgfsys@transformshift{5.671748in}{2.046952in}%
\pgfsys@useobject{currentmarker}{}%
\end{pgfscope}%
\begin{pgfscope}%
\pgfsys@transformshift{5.649486in}{1.975334in}%
\pgfsys@useobject{currentmarker}{}%
\end{pgfscope}%
\begin{pgfscope}%
\pgfsys@transformshift{5.626436in}{1.901832in}%
\pgfsys@useobject{currentmarker}{}%
\end{pgfscope}%
\begin{pgfscope}%
\pgfsys@transformshift{5.602604in}{1.826456in}%
\pgfsys@useobject{currentmarker}{}%
\end{pgfscope}%
\begin{pgfscope}%
\pgfsys@transformshift{5.577994in}{1.749214in}%
\pgfsys@useobject{currentmarker}{}%
\end{pgfscope}%
\begin{pgfscope}%
\pgfsys@transformshift{5.552608in}{1.670116in}%
\pgfsys@useobject{currentmarker}{}%
\end{pgfscope}%
\begin{pgfscope}%
\pgfsys@transformshift{5.526451in}{1.589171in}%
\pgfsys@useobject{currentmarker}{}%
\end{pgfscope}%
\begin{pgfscope}%
\pgfsys@transformshift{5.499527in}{1.506388in}%
\pgfsys@useobject{currentmarker}{}%
\end{pgfscope}%
\begin{pgfscope}%
\pgfsys@transformshift{5.471839in}{1.421777in}%
\pgfsys@useobject{currentmarker}{}%
\end{pgfscope}%
\begin{pgfscope}%
\pgfsys@transformshift{5.443392in}{1.335347in}%
\pgfsys@useobject{currentmarker}{}%
\end{pgfscope}%
\begin{pgfscope}%
\pgfsys@transformshift{5.414189in}{1.247107in}%
\pgfsys@useobject{currentmarker}{}%
\end{pgfscope}%
\begin{pgfscope}%
\pgfsys@transformshift{5.384234in}{1.157066in}%
\pgfsys@useobject{currentmarker}{}%
\end{pgfscope}%
\begin{pgfscope}%
\pgfsys@transformshift{5.353531in}{1.065233in}%
\pgfsys@useobject{currentmarker}{}%
\end{pgfscope}%
\begin{pgfscope}%
\pgfsys@transformshift{5.322084in}{0.971616in}%
\pgfsys@useobject{currentmarker}{}%
\end{pgfscope}%
\begin{pgfscope}%
\pgfsys@transformshift{5.289895in}{0.876226in}%
\pgfsys@useobject{currentmarker}{}%
\end{pgfscope}%
\begin{pgfscope}%
\pgfsys@transformshift{5.256970in}{0.779070in}%
\pgfsys@useobject{currentmarker}{}%
\end{pgfscope}%
\end{pgfscope}%
\begin{pgfscope}%
\pgfpathrectangle{\pgfqpoint{1.250000in}{0.400000in}}{\pgfqpoint{7.750000in}{3.200000in}} %
\pgfusepath{clip}%
\pgfsetrectcap%
\pgfsetroundjoin%
\pgfsetlinewidth{1.003750pt}%
\definecolor{currentstroke}{rgb}{0.750000,0.750000,0.000000}%
\pgfsetstrokecolor{currentstroke}%
\pgfsetdash{}{0pt}%
\pgfpathmoveto{\pgfqpoint{4.571429in}{0.857143in}}%
\pgfpathlineto{\pgfqpoint{4.625802in}{0.968615in}}%
\pgfpathlineto{\pgfqpoint{4.679195in}{1.077287in}}%
\pgfpathlineto{\pgfqpoint{4.731615in}{1.183174in}}%
\pgfpathlineto{\pgfqpoint{4.783065in}{1.286289in}}%
\pgfpathlineto{\pgfqpoint{4.833550in}{1.386646in}}%
\pgfpathlineto{\pgfqpoint{4.883076in}{1.484259in}}%
\pgfpathlineto{\pgfqpoint{4.931647in}{1.579142in}}%
\pgfpathlineto{\pgfqpoint{4.979267in}{1.671308in}}%
\pgfpathlineto{\pgfqpoint{5.025942in}{1.760771in}}%
\pgfpathlineto{\pgfqpoint{5.071677in}{1.847544in}}%
\pgfpathlineto{\pgfqpoint{5.116475in}{1.931641in}}%
\pgfpathlineto{\pgfqpoint{5.160342in}{2.013075in}}%
\pgfpathlineto{\pgfqpoint{5.203283in}{2.091860in}}%
\pgfpathlineto{\pgfqpoint{5.245301in}{2.168009in}}%
\pgfpathlineto{\pgfqpoint{5.286402in}{2.241535in}}%
\pgfpathlineto{\pgfqpoint{5.326590in}{2.312451in}}%
\pgfpathlineto{\pgfqpoint{5.365870in}{2.380769in}}%
\pgfpathlineto{\pgfqpoint{5.404246in}{2.446504in}}%
\pgfpathlineto{\pgfqpoint{5.441722in}{2.509668in}}%
\pgfpathlineto{\pgfqpoint{5.478304in}{2.570274in}}%
\pgfpathlineto{\pgfqpoint{5.513996in}{2.628335in}}%
\pgfpathlineto{\pgfqpoint{5.548802in}{2.683863in}}%
\pgfpathlineto{\pgfqpoint{5.582727in}{2.736871in}}%
\pgfpathlineto{\pgfqpoint{5.615775in}{2.787372in}}%
\pgfpathlineto{\pgfqpoint{5.647950in}{2.835378in}}%
\pgfpathlineto{\pgfqpoint{5.679257in}{2.880902in}}%
\pgfpathlineto{\pgfqpoint{5.709700in}{2.923955in}}%
\pgfpathlineto{\pgfqpoint{5.739283in}{2.964552in}}%
\pgfpathlineto{\pgfqpoint{5.768011in}{3.002703in}}%
\pgfpathlineto{\pgfqpoint{5.795888in}{3.038421in}}%
\pgfpathlineto{\pgfqpoint{5.822919in}{3.071718in}}%
\pgfpathlineto{\pgfqpoint{5.849107in}{3.102606in}}%
\pgfpathlineto{\pgfqpoint{5.874457in}{3.131097in}}%
\pgfpathlineto{\pgfqpoint{5.898972in}{3.157204in}}%
\pgfpathlineto{\pgfqpoint{5.922658in}{3.180938in}}%
\pgfpathlineto{\pgfqpoint{5.945518in}{3.202311in}}%
\pgfpathlineto{\pgfqpoint{5.967556in}{3.221335in}}%
\pgfpathlineto{\pgfqpoint{5.988777in}{3.238022in}}%
\pgfpathlineto{\pgfqpoint{6.009184in}{3.252382in}}%
\pgfpathlineto{\pgfqpoint{6.028782in}{3.264429in}}%
\pgfpathlineto{\pgfqpoint{6.047575in}{3.274173in}}%
\pgfpathlineto{\pgfqpoint{6.065566in}{3.281626in}}%
\pgfpathlineto{\pgfqpoint{6.082760in}{3.286800in}}%
\pgfpathlineto{\pgfqpoint{6.099161in}{3.289705in}}%
\pgfpathlineto{\pgfqpoint{6.114772in}{3.290354in}}%
\pgfpathlineto{\pgfqpoint{6.129598in}{3.288757in}}%
\pgfpathlineto{\pgfqpoint{6.143642in}{3.284926in}}%
\pgfpathlineto{\pgfqpoint{6.156909in}{3.278871in}}%
\pgfpathlineto{\pgfqpoint{6.169403in}{3.270605in}}%
\pgfpathlineto{\pgfqpoint{6.181126in}{3.260138in}}%
\pgfpathlineto{\pgfqpoint{6.192083in}{3.247480in}}%
\pgfpathlineto{\pgfqpoint{6.202279in}{3.232644in}}%
\pgfpathlineto{\pgfqpoint{6.211716in}{3.215640in}}%
\pgfpathlineto{\pgfqpoint{6.220398in}{3.196478in}}%
\pgfpathlineto{\pgfqpoint{6.228330in}{3.175170in}}%
\pgfpathlineto{\pgfqpoint{6.235515in}{3.151726in}}%
\pgfpathlineto{\pgfqpoint{6.241956in}{3.126157in}}%
\pgfpathlineto{\pgfqpoint{6.247658in}{3.098474in}}%
\pgfpathlineto{\pgfqpoint{6.252624in}{3.068687in}}%
\pgfpathlineto{\pgfqpoint{6.256858in}{3.036806in}}%
\pgfpathlineto{\pgfqpoint{6.260363in}{3.002843in}}%
\pgfpathlineto{\pgfqpoint{6.263144in}{2.966807in}}%
\pgfpathlineto{\pgfqpoint{6.265203in}{2.928709in}}%
\pgfpathlineto{\pgfqpoint{6.266545in}{2.888559in}}%
\pgfpathlineto{\pgfqpoint{6.267173in}{2.846368in}}%
\pgfpathlineto{\pgfqpoint{6.267090in}{2.802145in}}%
\pgfpathlineto{\pgfqpoint{6.266300in}{2.755901in}}%
\pgfpathlineto{\pgfqpoint{6.264807in}{2.707646in}}%
\pgfpathlineto{\pgfqpoint{6.262614in}{2.657390in}}%
\pgfpathlineto{\pgfqpoint{6.259724in}{2.605143in}}%
\pgfpathlineto{\pgfqpoint{6.256142in}{2.550915in}}%
\pgfpathlineto{\pgfqpoint{6.251870in}{2.494716in}}%
\pgfpathlineto{\pgfqpoint{6.246912in}{2.436556in}}%
\pgfpathlineto{\pgfqpoint{6.241272in}{2.376444in}}%
\pgfpathlineto{\pgfqpoint{6.234952in}{2.314390in}}%
\pgfpathlineto{\pgfqpoint{6.227957in}{2.250404in}}%
\pgfpathlineto{\pgfqpoint{6.220290in}{2.184496in}}%
\pgfpathlineto{\pgfqpoint{6.211953in}{2.116676in}}%
\pgfpathlineto{\pgfqpoint{6.202951in}{2.046952in}}%
\pgfpathlineto{\pgfqpoint{6.193286in}{1.975334in}}%
\pgfpathlineto{\pgfqpoint{6.182963in}{1.901832in}}%
\pgfpathlineto{\pgfqpoint{6.171983in}{1.826456in}}%
\pgfpathlineto{\pgfqpoint{6.160352in}{1.749214in}}%
\pgfpathlineto{\pgfqpoint{6.148071in}{1.670116in}}%
\pgfpathlineto{\pgfqpoint{6.135144in}{1.589171in}}%
\pgfpathlineto{\pgfqpoint{6.121574in}{1.506388in}}%
\pgfpathlineto{\pgfqpoint{6.107365in}{1.421777in}}%
\pgfpathlineto{\pgfqpoint{6.092520in}{1.335347in}}%
\pgfpathlineto{\pgfqpoint{6.077041in}{1.247107in}}%
\pgfpathlineto{\pgfqpoint{6.060933in}{1.157066in}}%
\pgfpathlineto{\pgfqpoint{6.044198in}{1.065233in}}%
\pgfpathlineto{\pgfqpoint{6.026839in}{0.971616in}}%
\pgfpathlineto{\pgfqpoint{6.008859in}{0.876226in}}%
\pgfpathlineto{\pgfqpoint{5.990263in}{0.779070in}}%
\pgfusepath{stroke}%
\end{pgfscope}%
\begin{pgfscope}%
\pgfpathrectangle{\pgfqpoint{1.250000in}{0.400000in}}{\pgfqpoint{7.750000in}{3.200000in}} %
\pgfusepath{clip}%
\pgfsetbuttcap%
\pgfsetroundjoin%
\definecolor{currentfill}{rgb}{0.750000,0.750000,0.000000}%
\pgfsetfillcolor{currentfill}%
\pgfsetlinewidth{0.501875pt}%
\definecolor{currentstroke}{rgb}{0.750000,0.750000,0.000000}%
\pgfsetstrokecolor{currentstroke}%
\pgfsetdash{}{0pt}%
\pgfsys@defobject{currentmarker}{\pgfqpoint{-0.041667in}{-0.041667in}}{\pgfqpoint{0.041667in}{0.041667in}}{%
\pgfpathmoveto{\pgfqpoint{-0.041667in}{0.000000in}}%
\pgfpathlineto{\pgfqpoint{0.041667in}{0.000000in}}%
\pgfpathmoveto{\pgfqpoint{0.000000in}{-0.041667in}}%
\pgfpathlineto{\pgfqpoint{0.000000in}{0.041667in}}%
\pgfusepath{stroke,fill}%
}%
\begin{pgfscope}%
\pgfsys@transformshift{4.571429in}{0.857143in}%
\pgfsys@useobject{currentmarker}{}%
\end{pgfscope}%
\begin{pgfscope}%
\pgfsys@transformshift{4.625802in}{0.968615in}%
\pgfsys@useobject{currentmarker}{}%
\end{pgfscope}%
\begin{pgfscope}%
\pgfsys@transformshift{4.679195in}{1.077287in}%
\pgfsys@useobject{currentmarker}{}%
\end{pgfscope}%
\begin{pgfscope}%
\pgfsys@transformshift{4.731615in}{1.183174in}%
\pgfsys@useobject{currentmarker}{}%
\end{pgfscope}%
\begin{pgfscope}%
\pgfsys@transformshift{4.783065in}{1.286289in}%
\pgfsys@useobject{currentmarker}{}%
\end{pgfscope}%
\begin{pgfscope}%
\pgfsys@transformshift{4.833550in}{1.386646in}%
\pgfsys@useobject{currentmarker}{}%
\end{pgfscope}%
\begin{pgfscope}%
\pgfsys@transformshift{4.883076in}{1.484259in}%
\pgfsys@useobject{currentmarker}{}%
\end{pgfscope}%
\begin{pgfscope}%
\pgfsys@transformshift{4.931647in}{1.579142in}%
\pgfsys@useobject{currentmarker}{}%
\end{pgfscope}%
\begin{pgfscope}%
\pgfsys@transformshift{4.979267in}{1.671308in}%
\pgfsys@useobject{currentmarker}{}%
\end{pgfscope}%
\begin{pgfscope}%
\pgfsys@transformshift{5.025942in}{1.760771in}%
\pgfsys@useobject{currentmarker}{}%
\end{pgfscope}%
\begin{pgfscope}%
\pgfsys@transformshift{5.071677in}{1.847544in}%
\pgfsys@useobject{currentmarker}{}%
\end{pgfscope}%
\begin{pgfscope}%
\pgfsys@transformshift{5.116475in}{1.931641in}%
\pgfsys@useobject{currentmarker}{}%
\end{pgfscope}%
\begin{pgfscope}%
\pgfsys@transformshift{5.160342in}{2.013075in}%
\pgfsys@useobject{currentmarker}{}%
\end{pgfscope}%
\begin{pgfscope}%
\pgfsys@transformshift{5.203283in}{2.091860in}%
\pgfsys@useobject{currentmarker}{}%
\end{pgfscope}%
\begin{pgfscope}%
\pgfsys@transformshift{5.245301in}{2.168009in}%
\pgfsys@useobject{currentmarker}{}%
\end{pgfscope}%
\begin{pgfscope}%
\pgfsys@transformshift{5.286402in}{2.241535in}%
\pgfsys@useobject{currentmarker}{}%
\end{pgfscope}%
\begin{pgfscope}%
\pgfsys@transformshift{5.326590in}{2.312451in}%
\pgfsys@useobject{currentmarker}{}%
\end{pgfscope}%
\begin{pgfscope}%
\pgfsys@transformshift{5.365870in}{2.380769in}%
\pgfsys@useobject{currentmarker}{}%
\end{pgfscope}%
\begin{pgfscope}%
\pgfsys@transformshift{5.404246in}{2.446504in}%
\pgfsys@useobject{currentmarker}{}%
\end{pgfscope}%
\begin{pgfscope}%
\pgfsys@transformshift{5.441722in}{2.509668in}%
\pgfsys@useobject{currentmarker}{}%
\end{pgfscope}%
\begin{pgfscope}%
\pgfsys@transformshift{5.478304in}{2.570274in}%
\pgfsys@useobject{currentmarker}{}%
\end{pgfscope}%
\begin{pgfscope}%
\pgfsys@transformshift{5.513996in}{2.628335in}%
\pgfsys@useobject{currentmarker}{}%
\end{pgfscope}%
\begin{pgfscope}%
\pgfsys@transformshift{5.548802in}{2.683863in}%
\pgfsys@useobject{currentmarker}{}%
\end{pgfscope}%
\begin{pgfscope}%
\pgfsys@transformshift{5.582727in}{2.736871in}%
\pgfsys@useobject{currentmarker}{}%
\end{pgfscope}%
\begin{pgfscope}%
\pgfsys@transformshift{5.615775in}{2.787372in}%
\pgfsys@useobject{currentmarker}{}%
\end{pgfscope}%
\begin{pgfscope}%
\pgfsys@transformshift{5.647950in}{2.835378in}%
\pgfsys@useobject{currentmarker}{}%
\end{pgfscope}%
\begin{pgfscope}%
\pgfsys@transformshift{5.679257in}{2.880902in}%
\pgfsys@useobject{currentmarker}{}%
\end{pgfscope}%
\begin{pgfscope}%
\pgfsys@transformshift{5.709700in}{2.923955in}%
\pgfsys@useobject{currentmarker}{}%
\end{pgfscope}%
\begin{pgfscope}%
\pgfsys@transformshift{5.739283in}{2.964552in}%
\pgfsys@useobject{currentmarker}{}%
\end{pgfscope}%
\begin{pgfscope}%
\pgfsys@transformshift{5.768011in}{3.002703in}%
\pgfsys@useobject{currentmarker}{}%
\end{pgfscope}%
\begin{pgfscope}%
\pgfsys@transformshift{5.795888in}{3.038421in}%
\pgfsys@useobject{currentmarker}{}%
\end{pgfscope}%
\begin{pgfscope}%
\pgfsys@transformshift{5.822919in}{3.071718in}%
\pgfsys@useobject{currentmarker}{}%
\end{pgfscope}%
\begin{pgfscope}%
\pgfsys@transformshift{5.849107in}{3.102606in}%
\pgfsys@useobject{currentmarker}{}%
\end{pgfscope}%
\begin{pgfscope}%
\pgfsys@transformshift{5.874457in}{3.131097in}%
\pgfsys@useobject{currentmarker}{}%
\end{pgfscope}%
\begin{pgfscope}%
\pgfsys@transformshift{5.898972in}{3.157204in}%
\pgfsys@useobject{currentmarker}{}%
\end{pgfscope}%
\begin{pgfscope}%
\pgfsys@transformshift{5.922658in}{3.180938in}%
\pgfsys@useobject{currentmarker}{}%
\end{pgfscope}%
\begin{pgfscope}%
\pgfsys@transformshift{5.945518in}{3.202311in}%
\pgfsys@useobject{currentmarker}{}%
\end{pgfscope}%
\begin{pgfscope}%
\pgfsys@transformshift{5.967556in}{3.221335in}%
\pgfsys@useobject{currentmarker}{}%
\end{pgfscope}%
\begin{pgfscope}%
\pgfsys@transformshift{5.988777in}{3.238022in}%
\pgfsys@useobject{currentmarker}{}%
\end{pgfscope}%
\begin{pgfscope}%
\pgfsys@transformshift{6.009184in}{3.252382in}%
\pgfsys@useobject{currentmarker}{}%
\end{pgfscope}%
\begin{pgfscope}%
\pgfsys@transformshift{6.028782in}{3.264429in}%
\pgfsys@useobject{currentmarker}{}%
\end{pgfscope}%
\begin{pgfscope}%
\pgfsys@transformshift{6.047575in}{3.274173in}%
\pgfsys@useobject{currentmarker}{}%
\end{pgfscope}%
\begin{pgfscope}%
\pgfsys@transformshift{6.065566in}{3.281626in}%
\pgfsys@useobject{currentmarker}{}%
\end{pgfscope}%
\begin{pgfscope}%
\pgfsys@transformshift{6.082760in}{3.286800in}%
\pgfsys@useobject{currentmarker}{}%
\end{pgfscope}%
\begin{pgfscope}%
\pgfsys@transformshift{6.099161in}{3.289705in}%
\pgfsys@useobject{currentmarker}{}%
\end{pgfscope}%
\begin{pgfscope}%
\pgfsys@transformshift{6.114772in}{3.290354in}%
\pgfsys@useobject{currentmarker}{}%
\end{pgfscope}%
\begin{pgfscope}%
\pgfsys@transformshift{6.129598in}{3.288757in}%
\pgfsys@useobject{currentmarker}{}%
\end{pgfscope}%
\begin{pgfscope}%
\pgfsys@transformshift{6.143642in}{3.284926in}%
\pgfsys@useobject{currentmarker}{}%
\end{pgfscope}%
\begin{pgfscope}%
\pgfsys@transformshift{6.156909in}{3.278871in}%
\pgfsys@useobject{currentmarker}{}%
\end{pgfscope}%
\begin{pgfscope}%
\pgfsys@transformshift{6.169403in}{3.270605in}%
\pgfsys@useobject{currentmarker}{}%
\end{pgfscope}%
\begin{pgfscope}%
\pgfsys@transformshift{6.181126in}{3.260138in}%
\pgfsys@useobject{currentmarker}{}%
\end{pgfscope}%
\begin{pgfscope}%
\pgfsys@transformshift{6.192083in}{3.247480in}%
\pgfsys@useobject{currentmarker}{}%
\end{pgfscope}%
\begin{pgfscope}%
\pgfsys@transformshift{6.202279in}{3.232644in}%
\pgfsys@useobject{currentmarker}{}%
\end{pgfscope}%
\begin{pgfscope}%
\pgfsys@transformshift{6.211716in}{3.215640in}%
\pgfsys@useobject{currentmarker}{}%
\end{pgfscope}%
\begin{pgfscope}%
\pgfsys@transformshift{6.220398in}{3.196478in}%
\pgfsys@useobject{currentmarker}{}%
\end{pgfscope}%
\begin{pgfscope}%
\pgfsys@transformshift{6.228330in}{3.175170in}%
\pgfsys@useobject{currentmarker}{}%
\end{pgfscope}%
\begin{pgfscope}%
\pgfsys@transformshift{6.235515in}{3.151726in}%
\pgfsys@useobject{currentmarker}{}%
\end{pgfscope}%
\begin{pgfscope}%
\pgfsys@transformshift{6.241956in}{3.126157in}%
\pgfsys@useobject{currentmarker}{}%
\end{pgfscope}%
\begin{pgfscope}%
\pgfsys@transformshift{6.247658in}{3.098474in}%
\pgfsys@useobject{currentmarker}{}%
\end{pgfscope}%
\begin{pgfscope}%
\pgfsys@transformshift{6.252624in}{3.068687in}%
\pgfsys@useobject{currentmarker}{}%
\end{pgfscope}%
\begin{pgfscope}%
\pgfsys@transformshift{6.256858in}{3.036806in}%
\pgfsys@useobject{currentmarker}{}%
\end{pgfscope}%
\begin{pgfscope}%
\pgfsys@transformshift{6.260363in}{3.002843in}%
\pgfsys@useobject{currentmarker}{}%
\end{pgfscope}%
\begin{pgfscope}%
\pgfsys@transformshift{6.263144in}{2.966807in}%
\pgfsys@useobject{currentmarker}{}%
\end{pgfscope}%
\begin{pgfscope}%
\pgfsys@transformshift{6.265203in}{2.928709in}%
\pgfsys@useobject{currentmarker}{}%
\end{pgfscope}%
\begin{pgfscope}%
\pgfsys@transformshift{6.266545in}{2.888559in}%
\pgfsys@useobject{currentmarker}{}%
\end{pgfscope}%
\begin{pgfscope}%
\pgfsys@transformshift{6.267173in}{2.846368in}%
\pgfsys@useobject{currentmarker}{}%
\end{pgfscope}%
\begin{pgfscope}%
\pgfsys@transformshift{6.267090in}{2.802145in}%
\pgfsys@useobject{currentmarker}{}%
\end{pgfscope}%
\begin{pgfscope}%
\pgfsys@transformshift{6.266300in}{2.755901in}%
\pgfsys@useobject{currentmarker}{}%
\end{pgfscope}%
\begin{pgfscope}%
\pgfsys@transformshift{6.264807in}{2.707646in}%
\pgfsys@useobject{currentmarker}{}%
\end{pgfscope}%
\begin{pgfscope}%
\pgfsys@transformshift{6.262614in}{2.657390in}%
\pgfsys@useobject{currentmarker}{}%
\end{pgfscope}%
\begin{pgfscope}%
\pgfsys@transformshift{6.259724in}{2.605143in}%
\pgfsys@useobject{currentmarker}{}%
\end{pgfscope}%
\begin{pgfscope}%
\pgfsys@transformshift{6.256142in}{2.550915in}%
\pgfsys@useobject{currentmarker}{}%
\end{pgfscope}%
\begin{pgfscope}%
\pgfsys@transformshift{6.251870in}{2.494716in}%
\pgfsys@useobject{currentmarker}{}%
\end{pgfscope}%
\begin{pgfscope}%
\pgfsys@transformshift{6.246912in}{2.436556in}%
\pgfsys@useobject{currentmarker}{}%
\end{pgfscope}%
\begin{pgfscope}%
\pgfsys@transformshift{6.241272in}{2.376444in}%
\pgfsys@useobject{currentmarker}{}%
\end{pgfscope}%
\begin{pgfscope}%
\pgfsys@transformshift{6.234952in}{2.314390in}%
\pgfsys@useobject{currentmarker}{}%
\end{pgfscope}%
\begin{pgfscope}%
\pgfsys@transformshift{6.227957in}{2.250404in}%
\pgfsys@useobject{currentmarker}{}%
\end{pgfscope}%
\begin{pgfscope}%
\pgfsys@transformshift{6.220290in}{2.184496in}%
\pgfsys@useobject{currentmarker}{}%
\end{pgfscope}%
\begin{pgfscope}%
\pgfsys@transformshift{6.211953in}{2.116676in}%
\pgfsys@useobject{currentmarker}{}%
\end{pgfscope}%
\begin{pgfscope}%
\pgfsys@transformshift{6.202951in}{2.046952in}%
\pgfsys@useobject{currentmarker}{}%
\end{pgfscope}%
\begin{pgfscope}%
\pgfsys@transformshift{6.193286in}{1.975334in}%
\pgfsys@useobject{currentmarker}{}%
\end{pgfscope}%
\begin{pgfscope}%
\pgfsys@transformshift{6.182963in}{1.901832in}%
\pgfsys@useobject{currentmarker}{}%
\end{pgfscope}%
\begin{pgfscope}%
\pgfsys@transformshift{6.171983in}{1.826456in}%
\pgfsys@useobject{currentmarker}{}%
\end{pgfscope}%
\begin{pgfscope}%
\pgfsys@transformshift{6.160352in}{1.749214in}%
\pgfsys@useobject{currentmarker}{}%
\end{pgfscope}%
\begin{pgfscope}%
\pgfsys@transformshift{6.148071in}{1.670116in}%
\pgfsys@useobject{currentmarker}{}%
\end{pgfscope}%
\begin{pgfscope}%
\pgfsys@transformshift{6.135144in}{1.589171in}%
\pgfsys@useobject{currentmarker}{}%
\end{pgfscope}%
\begin{pgfscope}%
\pgfsys@transformshift{6.121574in}{1.506388in}%
\pgfsys@useobject{currentmarker}{}%
\end{pgfscope}%
\begin{pgfscope}%
\pgfsys@transformshift{6.107365in}{1.421777in}%
\pgfsys@useobject{currentmarker}{}%
\end{pgfscope}%
\begin{pgfscope}%
\pgfsys@transformshift{6.092520in}{1.335347in}%
\pgfsys@useobject{currentmarker}{}%
\end{pgfscope}%
\begin{pgfscope}%
\pgfsys@transformshift{6.077041in}{1.247107in}%
\pgfsys@useobject{currentmarker}{}%
\end{pgfscope}%
\begin{pgfscope}%
\pgfsys@transformshift{6.060933in}{1.157066in}%
\pgfsys@useobject{currentmarker}{}%
\end{pgfscope}%
\begin{pgfscope}%
\pgfsys@transformshift{6.044198in}{1.065233in}%
\pgfsys@useobject{currentmarker}{}%
\end{pgfscope}%
\begin{pgfscope}%
\pgfsys@transformshift{6.026839in}{0.971616in}%
\pgfsys@useobject{currentmarker}{}%
\end{pgfscope}%
\begin{pgfscope}%
\pgfsys@transformshift{6.008859in}{0.876226in}%
\pgfsys@useobject{currentmarker}{}%
\end{pgfscope}%
\begin{pgfscope}%
\pgfsys@transformshift{5.990263in}{0.779070in}%
\pgfsys@useobject{currentmarker}{}%
\end{pgfscope}%
\end{pgfscope}%
\begin{pgfscope}%
\pgfpathrectangle{\pgfqpoint{1.250000in}{0.400000in}}{\pgfqpoint{7.750000in}{3.200000in}} %
\pgfusepath{clip}%
\pgfsetrectcap%
\pgfsetroundjoin%
\pgfsetlinewidth{1.003750pt}%
\definecolor{currentstroke}{rgb}{0.000000,0.000000,0.000000}%
\pgfsetstrokecolor{currentstroke}%
\pgfsetdash{}{0pt}%
\pgfpathmoveto{\pgfqpoint{4.571429in}{0.857143in}}%
\pgfpathlineto{\pgfqpoint{4.625992in}{0.968615in}}%
\pgfpathlineto{\pgfqpoint{4.679766in}{1.077287in}}%
\pgfpathlineto{\pgfqpoint{4.732755in}{1.183174in}}%
\pgfpathlineto{\pgfqpoint{4.784962in}{1.286289in}}%
\pgfpathlineto{\pgfqpoint{4.836392in}{1.386646in}}%
\pgfpathlineto{\pgfqpoint{4.887047in}{1.484259in}}%
\pgfpathlineto{\pgfqpoint{4.936933in}{1.579142in}}%
\pgfpathlineto{\pgfqpoint{4.986052in}{1.671308in}}%
\pgfpathlineto{\pgfqpoint{5.034409in}{1.760771in}}%
\pgfpathlineto{\pgfqpoint{5.082008in}{1.847544in}}%
\pgfpathlineto{\pgfqpoint{5.128852in}{1.931641in}}%
\pgfpathlineto{\pgfqpoint{5.174946in}{2.013075in}}%
\pgfpathlineto{\pgfqpoint{5.220292in}{2.091860in}}%
\pgfpathlineto{\pgfqpoint{5.264894in}{2.168009in}}%
\pgfpathlineto{\pgfqpoint{5.308757in}{2.241535in}}%
\pgfpathlineto{\pgfqpoint{5.351884in}{2.312451in}}%
\pgfpathlineto{\pgfqpoint{5.394279in}{2.380769in}}%
\pgfpathlineto{\pgfqpoint{5.435945in}{2.446504in}}%
\pgfpathlineto{\pgfqpoint{5.476886in}{2.509668in}}%
\pgfpathlineto{\pgfqpoint{5.517106in}{2.570274in}}%
\pgfpathlineto{\pgfqpoint{5.556608in}{2.628335in}}%
\pgfpathlineto{\pgfqpoint{5.595396in}{2.683863in}}%
\pgfpathlineto{\pgfqpoint{5.633473in}{2.736871in}}%
\pgfpathlineto{\pgfqpoint{5.670843in}{2.787372in}}%
\pgfpathlineto{\pgfqpoint{5.707510in}{2.835378in}}%
\pgfpathlineto{\pgfqpoint{5.743476in}{2.880902in}}%
\pgfpathlineto{\pgfqpoint{5.778747in}{2.923955in}}%
\pgfpathlineto{\pgfqpoint{5.813324in}{2.964552in}}%
\pgfpathlineto{\pgfqpoint{5.847211in}{3.002703in}}%
\pgfpathlineto{\pgfqpoint{5.880413in}{3.038421in}}%
\pgfpathlineto{\pgfqpoint{5.912931in}{3.071718in}}%
\pgfpathlineto{\pgfqpoint{5.944771in}{3.102606in}}%
\pgfpathlineto{\pgfqpoint{5.975935in}{3.131097in}}%
\pgfpathlineto{\pgfqpoint{6.006426in}{3.157204in}}%
\pgfpathlineto{\pgfqpoint{6.036248in}{3.180938in}}%
\pgfpathlineto{\pgfqpoint{6.065404in}{3.202311in}}%
\pgfpathlineto{\pgfqpoint{6.093898in}{3.221335in}}%
\pgfpathlineto{\pgfqpoint{6.121733in}{3.238022in}}%
\pgfpathlineto{\pgfqpoint{6.148911in}{3.252382in}}%
\pgfpathlineto{\pgfqpoint{6.175438in}{3.264429in}}%
\pgfpathlineto{\pgfqpoint{6.201315in}{3.274173in}}%
\pgfpathlineto{\pgfqpoint{6.226546in}{3.281626in}}%
\pgfpathlineto{\pgfqpoint{6.251134in}{3.286800in}}%
\pgfpathlineto{\pgfqpoint{6.275082in}{3.289705in}}%
\pgfpathlineto{\pgfqpoint{6.298394in}{3.290354in}}%
\pgfpathlineto{\pgfqpoint{6.321073in}{3.288757in}}%
\pgfpathlineto{\pgfqpoint{6.343122in}{3.284926in}}%
\pgfpathlineto{\pgfqpoint{6.364544in}{3.278871in}}%
\pgfpathlineto{\pgfqpoint{6.385342in}{3.270605in}}%
\pgfpathlineto{\pgfqpoint{6.405519in}{3.260138in}}%
\pgfpathlineto{\pgfqpoint{6.425079in}{3.247480in}}%
\pgfpathlineto{\pgfqpoint{6.444025in}{3.232644in}}%
\pgfpathlineto{\pgfqpoint{6.462359in}{3.215640in}}%
\pgfpathlineto{\pgfqpoint{6.480084in}{3.196478in}}%
\pgfpathlineto{\pgfqpoint{6.497205in}{3.175170in}}%
\pgfpathlineto{\pgfqpoint{6.513723in}{3.151726in}}%
\pgfpathlineto{\pgfqpoint{6.529642in}{3.126157in}}%
\pgfpathlineto{\pgfqpoint{6.544965in}{3.098474in}}%
\pgfpathlineto{\pgfqpoint{6.559694in}{3.068687in}}%
\pgfpathlineto{\pgfqpoint{6.573833in}{3.036806in}}%
\pgfpathlineto{\pgfqpoint{6.587385in}{3.002843in}}%
\pgfpathlineto{\pgfqpoint{6.600352in}{2.966807in}}%
\pgfpathlineto{\pgfqpoint{6.612738in}{2.928709in}}%
\pgfpathlineto{\pgfqpoint{6.624545in}{2.888559in}}%
\pgfpathlineto{\pgfqpoint{6.635776in}{2.846368in}}%
\pgfpathlineto{\pgfqpoint{6.646435in}{2.802145in}}%
\pgfpathlineto{\pgfqpoint{6.656523in}{2.755901in}}%
\pgfpathlineto{\pgfqpoint{6.666045in}{2.707646in}}%
\pgfpathlineto{\pgfqpoint{6.675002in}{2.657390in}}%
\pgfpathlineto{\pgfqpoint{6.683398in}{2.605143in}}%
\pgfpathlineto{\pgfqpoint{6.691235in}{2.550915in}}%
\pgfpathlineto{\pgfqpoint{6.698517in}{2.494716in}}%
\pgfpathlineto{\pgfqpoint{6.705245in}{2.436556in}}%
\pgfpathlineto{\pgfqpoint{6.711423in}{2.376444in}}%
\pgfpathlineto{\pgfqpoint{6.717053in}{2.314390in}}%
\pgfpathlineto{\pgfqpoint{6.722139in}{2.250404in}}%
\pgfpathlineto{\pgfqpoint{6.726682in}{2.184496in}}%
\pgfpathlineto{\pgfqpoint{6.730686in}{2.116676in}}%
\pgfpathlineto{\pgfqpoint{6.734154in}{2.046952in}}%
\pgfpathlineto{\pgfqpoint{6.737087in}{1.975334in}}%
\pgfpathlineto{\pgfqpoint{6.739489in}{1.901832in}}%
\pgfpathlineto{\pgfqpoint{6.741363in}{1.826456in}}%
\pgfpathlineto{\pgfqpoint{6.742710in}{1.749214in}}%
\pgfpathlineto{\pgfqpoint{6.743534in}{1.670116in}}%
\pgfpathlineto{\pgfqpoint{6.743837in}{1.589171in}}%
\pgfpathlineto{\pgfqpoint{6.743622in}{1.506388in}}%
\pgfpathlineto{\pgfqpoint{6.742891in}{1.421777in}}%
\pgfpathlineto{\pgfqpoint{6.741648in}{1.335347in}}%
\pgfpathlineto{\pgfqpoint{6.739894in}{1.247107in}}%
\pgfpathlineto{\pgfqpoint{6.737632in}{1.157066in}}%
\pgfpathlineto{\pgfqpoint{6.734864in}{1.065233in}}%
\pgfpathlineto{\pgfqpoint{6.731594in}{0.971616in}}%
\pgfpathlineto{\pgfqpoint{6.727824in}{0.876226in}}%
\pgfpathlineto{\pgfqpoint{6.723555in}{0.779070in}}%
\pgfusepath{stroke}%
\end{pgfscope}%
\begin{pgfscope}%
\pgfpathrectangle{\pgfqpoint{1.250000in}{0.400000in}}{\pgfqpoint{7.750000in}{3.200000in}} %
\pgfusepath{clip}%
\pgfsetbuttcap%
\pgfsetroundjoin%
\definecolor{currentfill}{rgb}{0.000000,0.000000,0.000000}%
\pgfsetfillcolor{currentfill}%
\pgfsetlinewidth{0.501875pt}%
\definecolor{currentstroke}{rgb}{0.000000,0.000000,0.000000}%
\pgfsetstrokecolor{currentstroke}%
\pgfsetdash{}{0pt}%
\pgfsys@defobject{currentmarker}{\pgfqpoint{-0.041667in}{-0.041667in}}{\pgfqpoint{0.041667in}{0.041667in}}{%
\pgfpathmoveto{\pgfqpoint{-0.041667in}{0.000000in}}%
\pgfpathlineto{\pgfqpoint{0.041667in}{0.000000in}}%
\pgfpathmoveto{\pgfqpoint{0.000000in}{-0.041667in}}%
\pgfpathlineto{\pgfqpoint{0.000000in}{0.041667in}}%
\pgfusepath{stroke,fill}%
}%
\begin{pgfscope}%
\pgfsys@transformshift{4.571429in}{0.857143in}%
\pgfsys@useobject{currentmarker}{}%
\end{pgfscope}%
\begin{pgfscope}%
\pgfsys@transformshift{4.625992in}{0.968615in}%
\pgfsys@useobject{currentmarker}{}%
\end{pgfscope}%
\begin{pgfscope}%
\pgfsys@transformshift{4.679766in}{1.077287in}%
\pgfsys@useobject{currentmarker}{}%
\end{pgfscope}%
\begin{pgfscope}%
\pgfsys@transformshift{4.732755in}{1.183174in}%
\pgfsys@useobject{currentmarker}{}%
\end{pgfscope}%
\begin{pgfscope}%
\pgfsys@transformshift{4.784962in}{1.286289in}%
\pgfsys@useobject{currentmarker}{}%
\end{pgfscope}%
\begin{pgfscope}%
\pgfsys@transformshift{4.836392in}{1.386646in}%
\pgfsys@useobject{currentmarker}{}%
\end{pgfscope}%
\begin{pgfscope}%
\pgfsys@transformshift{4.887047in}{1.484259in}%
\pgfsys@useobject{currentmarker}{}%
\end{pgfscope}%
\begin{pgfscope}%
\pgfsys@transformshift{4.936933in}{1.579142in}%
\pgfsys@useobject{currentmarker}{}%
\end{pgfscope}%
\begin{pgfscope}%
\pgfsys@transformshift{4.986052in}{1.671308in}%
\pgfsys@useobject{currentmarker}{}%
\end{pgfscope}%
\begin{pgfscope}%
\pgfsys@transformshift{5.034409in}{1.760771in}%
\pgfsys@useobject{currentmarker}{}%
\end{pgfscope}%
\begin{pgfscope}%
\pgfsys@transformshift{5.082008in}{1.847544in}%
\pgfsys@useobject{currentmarker}{}%
\end{pgfscope}%
\begin{pgfscope}%
\pgfsys@transformshift{5.128852in}{1.931641in}%
\pgfsys@useobject{currentmarker}{}%
\end{pgfscope}%
\begin{pgfscope}%
\pgfsys@transformshift{5.174946in}{2.013075in}%
\pgfsys@useobject{currentmarker}{}%
\end{pgfscope}%
\begin{pgfscope}%
\pgfsys@transformshift{5.220292in}{2.091860in}%
\pgfsys@useobject{currentmarker}{}%
\end{pgfscope}%
\begin{pgfscope}%
\pgfsys@transformshift{5.264894in}{2.168009in}%
\pgfsys@useobject{currentmarker}{}%
\end{pgfscope}%
\begin{pgfscope}%
\pgfsys@transformshift{5.308757in}{2.241535in}%
\pgfsys@useobject{currentmarker}{}%
\end{pgfscope}%
\begin{pgfscope}%
\pgfsys@transformshift{5.351884in}{2.312451in}%
\pgfsys@useobject{currentmarker}{}%
\end{pgfscope}%
\begin{pgfscope}%
\pgfsys@transformshift{5.394279in}{2.380769in}%
\pgfsys@useobject{currentmarker}{}%
\end{pgfscope}%
\begin{pgfscope}%
\pgfsys@transformshift{5.435945in}{2.446504in}%
\pgfsys@useobject{currentmarker}{}%
\end{pgfscope}%
\begin{pgfscope}%
\pgfsys@transformshift{5.476886in}{2.509668in}%
\pgfsys@useobject{currentmarker}{}%
\end{pgfscope}%
\begin{pgfscope}%
\pgfsys@transformshift{5.517106in}{2.570274in}%
\pgfsys@useobject{currentmarker}{}%
\end{pgfscope}%
\begin{pgfscope}%
\pgfsys@transformshift{5.556608in}{2.628335in}%
\pgfsys@useobject{currentmarker}{}%
\end{pgfscope}%
\begin{pgfscope}%
\pgfsys@transformshift{5.595396in}{2.683863in}%
\pgfsys@useobject{currentmarker}{}%
\end{pgfscope}%
\begin{pgfscope}%
\pgfsys@transformshift{5.633473in}{2.736871in}%
\pgfsys@useobject{currentmarker}{}%
\end{pgfscope}%
\begin{pgfscope}%
\pgfsys@transformshift{5.670843in}{2.787372in}%
\pgfsys@useobject{currentmarker}{}%
\end{pgfscope}%
\begin{pgfscope}%
\pgfsys@transformshift{5.707510in}{2.835378in}%
\pgfsys@useobject{currentmarker}{}%
\end{pgfscope}%
\begin{pgfscope}%
\pgfsys@transformshift{5.743476in}{2.880902in}%
\pgfsys@useobject{currentmarker}{}%
\end{pgfscope}%
\begin{pgfscope}%
\pgfsys@transformshift{5.778747in}{2.923955in}%
\pgfsys@useobject{currentmarker}{}%
\end{pgfscope}%
\begin{pgfscope}%
\pgfsys@transformshift{5.813324in}{2.964552in}%
\pgfsys@useobject{currentmarker}{}%
\end{pgfscope}%
\begin{pgfscope}%
\pgfsys@transformshift{5.847211in}{3.002703in}%
\pgfsys@useobject{currentmarker}{}%
\end{pgfscope}%
\begin{pgfscope}%
\pgfsys@transformshift{5.880413in}{3.038421in}%
\pgfsys@useobject{currentmarker}{}%
\end{pgfscope}%
\begin{pgfscope}%
\pgfsys@transformshift{5.912931in}{3.071718in}%
\pgfsys@useobject{currentmarker}{}%
\end{pgfscope}%
\begin{pgfscope}%
\pgfsys@transformshift{5.944771in}{3.102606in}%
\pgfsys@useobject{currentmarker}{}%
\end{pgfscope}%
\begin{pgfscope}%
\pgfsys@transformshift{5.975935in}{3.131097in}%
\pgfsys@useobject{currentmarker}{}%
\end{pgfscope}%
\begin{pgfscope}%
\pgfsys@transformshift{6.006426in}{3.157204in}%
\pgfsys@useobject{currentmarker}{}%
\end{pgfscope}%
\begin{pgfscope}%
\pgfsys@transformshift{6.036248in}{3.180938in}%
\pgfsys@useobject{currentmarker}{}%
\end{pgfscope}%
\begin{pgfscope}%
\pgfsys@transformshift{6.065404in}{3.202311in}%
\pgfsys@useobject{currentmarker}{}%
\end{pgfscope}%
\begin{pgfscope}%
\pgfsys@transformshift{6.093898in}{3.221335in}%
\pgfsys@useobject{currentmarker}{}%
\end{pgfscope}%
\begin{pgfscope}%
\pgfsys@transformshift{6.121733in}{3.238022in}%
\pgfsys@useobject{currentmarker}{}%
\end{pgfscope}%
\begin{pgfscope}%
\pgfsys@transformshift{6.148911in}{3.252382in}%
\pgfsys@useobject{currentmarker}{}%
\end{pgfscope}%
\begin{pgfscope}%
\pgfsys@transformshift{6.175438in}{3.264429in}%
\pgfsys@useobject{currentmarker}{}%
\end{pgfscope}%
\begin{pgfscope}%
\pgfsys@transformshift{6.201315in}{3.274173in}%
\pgfsys@useobject{currentmarker}{}%
\end{pgfscope}%
\begin{pgfscope}%
\pgfsys@transformshift{6.226546in}{3.281626in}%
\pgfsys@useobject{currentmarker}{}%
\end{pgfscope}%
\begin{pgfscope}%
\pgfsys@transformshift{6.251134in}{3.286800in}%
\pgfsys@useobject{currentmarker}{}%
\end{pgfscope}%
\begin{pgfscope}%
\pgfsys@transformshift{6.275082in}{3.289705in}%
\pgfsys@useobject{currentmarker}{}%
\end{pgfscope}%
\begin{pgfscope}%
\pgfsys@transformshift{6.298394in}{3.290354in}%
\pgfsys@useobject{currentmarker}{}%
\end{pgfscope}%
\begin{pgfscope}%
\pgfsys@transformshift{6.321073in}{3.288757in}%
\pgfsys@useobject{currentmarker}{}%
\end{pgfscope}%
\begin{pgfscope}%
\pgfsys@transformshift{6.343122in}{3.284926in}%
\pgfsys@useobject{currentmarker}{}%
\end{pgfscope}%
\begin{pgfscope}%
\pgfsys@transformshift{6.364544in}{3.278871in}%
\pgfsys@useobject{currentmarker}{}%
\end{pgfscope}%
\begin{pgfscope}%
\pgfsys@transformshift{6.385342in}{3.270605in}%
\pgfsys@useobject{currentmarker}{}%
\end{pgfscope}%
\begin{pgfscope}%
\pgfsys@transformshift{6.405519in}{3.260138in}%
\pgfsys@useobject{currentmarker}{}%
\end{pgfscope}%
\begin{pgfscope}%
\pgfsys@transformshift{6.425079in}{3.247480in}%
\pgfsys@useobject{currentmarker}{}%
\end{pgfscope}%
\begin{pgfscope}%
\pgfsys@transformshift{6.444025in}{3.232644in}%
\pgfsys@useobject{currentmarker}{}%
\end{pgfscope}%
\begin{pgfscope}%
\pgfsys@transformshift{6.462359in}{3.215640in}%
\pgfsys@useobject{currentmarker}{}%
\end{pgfscope}%
\begin{pgfscope}%
\pgfsys@transformshift{6.480084in}{3.196478in}%
\pgfsys@useobject{currentmarker}{}%
\end{pgfscope}%
\begin{pgfscope}%
\pgfsys@transformshift{6.497205in}{3.175170in}%
\pgfsys@useobject{currentmarker}{}%
\end{pgfscope}%
\begin{pgfscope}%
\pgfsys@transformshift{6.513723in}{3.151726in}%
\pgfsys@useobject{currentmarker}{}%
\end{pgfscope}%
\begin{pgfscope}%
\pgfsys@transformshift{6.529642in}{3.126157in}%
\pgfsys@useobject{currentmarker}{}%
\end{pgfscope}%
\begin{pgfscope}%
\pgfsys@transformshift{6.544965in}{3.098474in}%
\pgfsys@useobject{currentmarker}{}%
\end{pgfscope}%
\begin{pgfscope}%
\pgfsys@transformshift{6.559694in}{3.068687in}%
\pgfsys@useobject{currentmarker}{}%
\end{pgfscope}%
\begin{pgfscope}%
\pgfsys@transformshift{6.573833in}{3.036806in}%
\pgfsys@useobject{currentmarker}{}%
\end{pgfscope}%
\begin{pgfscope}%
\pgfsys@transformshift{6.587385in}{3.002843in}%
\pgfsys@useobject{currentmarker}{}%
\end{pgfscope}%
\begin{pgfscope}%
\pgfsys@transformshift{6.600352in}{2.966807in}%
\pgfsys@useobject{currentmarker}{}%
\end{pgfscope}%
\begin{pgfscope}%
\pgfsys@transformshift{6.612738in}{2.928709in}%
\pgfsys@useobject{currentmarker}{}%
\end{pgfscope}%
\begin{pgfscope}%
\pgfsys@transformshift{6.624545in}{2.888559in}%
\pgfsys@useobject{currentmarker}{}%
\end{pgfscope}%
\begin{pgfscope}%
\pgfsys@transformshift{6.635776in}{2.846368in}%
\pgfsys@useobject{currentmarker}{}%
\end{pgfscope}%
\begin{pgfscope}%
\pgfsys@transformshift{6.646435in}{2.802145in}%
\pgfsys@useobject{currentmarker}{}%
\end{pgfscope}%
\begin{pgfscope}%
\pgfsys@transformshift{6.656523in}{2.755901in}%
\pgfsys@useobject{currentmarker}{}%
\end{pgfscope}%
\begin{pgfscope}%
\pgfsys@transformshift{6.666045in}{2.707646in}%
\pgfsys@useobject{currentmarker}{}%
\end{pgfscope}%
\begin{pgfscope}%
\pgfsys@transformshift{6.675002in}{2.657390in}%
\pgfsys@useobject{currentmarker}{}%
\end{pgfscope}%
\begin{pgfscope}%
\pgfsys@transformshift{6.683398in}{2.605143in}%
\pgfsys@useobject{currentmarker}{}%
\end{pgfscope}%
\begin{pgfscope}%
\pgfsys@transformshift{6.691235in}{2.550915in}%
\pgfsys@useobject{currentmarker}{}%
\end{pgfscope}%
\begin{pgfscope}%
\pgfsys@transformshift{6.698517in}{2.494716in}%
\pgfsys@useobject{currentmarker}{}%
\end{pgfscope}%
\begin{pgfscope}%
\pgfsys@transformshift{6.705245in}{2.436556in}%
\pgfsys@useobject{currentmarker}{}%
\end{pgfscope}%
\begin{pgfscope}%
\pgfsys@transformshift{6.711423in}{2.376444in}%
\pgfsys@useobject{currentmarker}{}%
\end{pgfscope}%
\begin{pgfscope}%
\pgfsys@transformshift{6.717053in}{2.314390in}%
\pgfsys@useobject{currentmarker}{}%
\end{pgfscope}%
\begin{pgfscope}%
\pgfsys@transformshift{6.722139in}{2.250404in}%
\pgfsys@useobject{currentmarker}{}%
\end{pgfscope}%
\begin{pgfscope}%
\pgfsys@transformshift{6.726682in}{2.184496in}%
\pgfsys@useobject{currentmarker}{}%
\end{pgfscope}%
\begin{pgfscope}%
\pgfsys@transformshift{6.730686in}{2.116676in}%
\pgfsys@useobject{currentmarker}{}%
\end{pgfscope}%
\begin{pgfscope}%
\pgfsys@transformshift{6.734154in}{2.046952in}%
\pgfsys@useobject{currentmarker}{}%
\end{pgfscope}%
\begin{pgfscope}%
\pgfsys@transformshift{6.737087in}{1.975334in}%
\pgfsys@useobject{currentmarker}{}%
\end{pgfscope}%
\begin{pgfscope}%
\pgfsys@transformshift{6.739489in}{1.901832in}%
\pgfsys@useobject{currentmarker}{}%
\end{pgfscope}%
\begin{pgfscope}%
\pgfsys@transformshift{6.741363in}{1.826456in}%
\pgfsys@useobject{currentmarker}{}%
\end{pgfscope}%
\begin{pgfscope}%
\pgfsys@transformshift{6.742710in}{1.749214in}%
\pgfsys@useobject{currentmarker}{}%
\end{pgfscope}%
\begin{pgfscope}%
\pgfsys@transformshift{6.743534in}{1.670116in}%
\pgfsys@useobject{currentmarker}{}%
\end{pgfscope}%
\begin{pgfscope}%
\pgfsys@transformshift{6.743837in}{1.589171in}%
\pgfsys@useobject{currentmarker}{}%
\end{pgfscope}%
\begin{pgfscope}%
\pgfsys@transformshift{6.743622in}{1.506388in}%
\pgfsys@useobject{currentmarker}{}%
\end{pgfscope}%
\begin{pgfscope}%
\pgfsys@transformshift{6.742891in}{1.421777in}%
\pgfsys@useobject{currentmarker}{}%
\end{pgfscope}%
\begin{pgfscope}%
\pgfsys@transformshift{6.741648in}{1.335347in}%
\pgfsys@useobject{currentmarker}{}%
\end{pgfscope}%
\begin{pgfscope}%
\pgfsys@transformshift{6.739894in}{1.247107in}%
\pgfsys@useobject{currentmarker}{}%
\end{pgfscope}%
\begin{pgfscope}%
\pgfsys@transformshift{6.737632in}{1.157066in}%
\pgfsys@useobject{currentmarker}{}%
\end{pgfscope}%
\begin{pgfscope}%
\pgfsys@transformshift{6.734864in}{1.065233in}%
\pgfsys@useobject{currentmarker}{}%
\end{pgfscope}%
\begin{pgfscope}%
\pgfsys@transformshift{6.731594in}{0.971616in}%
\pgfsys@useobject{currentmarker}{}%
\end{pgfscope}%
\begin{pgfscope}%
\pgfsys@transformshift{6.727824in}{0.876226in}%
\pgfsys@useobject{currentmarker}{}%
\end{pgfscope}%
\begin{pgfscope}%
\pgfsys@transformshift{6.723555in}{0.779070in}%
\pgfsys@useobject{currentmarker}{}%
\end{pgfscope}%
\end{pgfscope}%
\begin{pgfscope}%
\pgfpathrectangle{\pgfqpoint{1.250000in}{0.400000in}}{\pgfqpoint{7.750000in}{3.200000in}} %
\pgfusepath{clip}%
\pgfsetrectcap%
\pgfsetroundjoin%
\pgfsetlinewidth{1.003750pt}%
\definecolor{currentstroke}{rgb}{0.000000,0.000000,1.000000}%
\pgfsetstrokecolor{currentstroke}%
\pgfsetdash{}{0pt}%
\pgfpathmoveto{\pgfqpoint{4.571429in}{0.857143in}}%
\pgfpathlineto{\pgfqpoint{4.626183in}{0.968615in}}%
\pgfpathlineto{\pgfqpoint{4.680338in}{1.077287in}}%
\pgfpathlineto{\pgfqpoint{4.733895in}{1.183174in}}%
\pgfpathlineto{\pgfqpoint{4.786859in}{1.286289in}}%
\pgfpathlineto{\pgfqpoint{4.839233in}{1.386646in}}%
\pgfpathlineto{\pgfqpoint{4.891018in}{1.484259in}}%
\pgfpathlineto{\pgfqpoint{4.942219in}{1.579142in}}%
\pgfpathlineto{\pgfqpoint{4.992837in}{1.671308in}}%
\pgfpathlineto{\pgfqpoint{5.042876in}{1.760771in}}%
\pgfpathlineto{\pgfqpoint{5.092340in}{1.847544in}}%
\pgfpathlineto{\pgfqpoint{5.141229in}{1.931641in}}%
\pgfpathlineto{\pgfqpoint{5.189549in}{2.013075in}}%
\pgfpathlineto{\pgfqpoint{5.237301in}{2.091860in}}%
\pgfpathlineto{\pgfqpoint{5.284488in}{2.168009in}}%
\pgfpathlineto{\pgfqpoint{5.331113in}{2.241535in}}%
\pgfpathlineto{\pgfqpoint{5.377179in}{2.312451in}}%
\pgfpathlineto{\pgfqpoint{5.422689in}{2.380769in}}%
\pgfpathlineto{\pgfqpoint{5.467645in}{2.446504in}}%
\pgfpathlineto{\pgfqpoint{5.512050in}{2.509668in}}%
\pgfpathlineto{\pgfqpoint{5.555908in}{2.570274in}}%
\pgfpathlineto{\pgfqpoint{5.599220in}{2.628335in}}%
\pgfpathlineto{\pgfqpoint{5.641990in}{2.683863in}}%
\pgfpathlineto{\pgfqpoint{5.684219in}{2.736871in}}%
\pgfpathlineto{\pgfqpoint{5.725912in}{2.787372in}}%
\pgfpathlineto{\pgfqpoint{5.767070in}{2.835378in}}%
\pgfpathlineto{\pgfqpoint{5.807696in}{2.880902in}}%
\pgfpathlineto{\pgfqpoint{5.847794in}{2.923955in}}%
\pgfpathlineto{\pgfqpoint{5.887364in}{2.964552in}}%
\pgfpathlineto{\pgfqpoint{5.926411in}{3.002703in}}%
\pgfpathlineto{\pgfqpoint{5.964937in}{3.038421in}}%
\pgfpathlineto{\pgfqpoint{6.002944in}{3.071718in}}%
\pgfpathlineto{\pgfqpoint{6.040435in}{3.102606in}}%
\pgfpathlineto{\pgfqpoint{6.077413in}{3.131097in}}%
\pgfpathlineto{\pgfqpoint{6.113879in}{3.157204in}}%
\pgfpathlineto{\pgfqpoint{6.149838in}{3.180938in}}%
\pgfpathlineto{\pgfqpoint{6.185290in}{3.202311in}}%
\pgfpathlineto{\pgfqpoint{6.220240in}{3.221335in}}%
\pgfpathlineto{\pgfqpoint{6.254688in}{3.238022in}}%
\pgfpathlineto{\pgfqpoint{6.288639in}{3.252382in}}%
\pgfpathlineto{\pgfqpoint{6.322093in}{3.264429in}}%
\pgfpathlineto{\pgfqpoint{6.355055in}{3.274173in}}%
\pgfpathlineto{\pgfqpoint{6.387525in}{3.281626in}}%
\pgfpathlineto{\pgfqpoint{6.419507in}{3.286800in}}%
\pgfpathlineto{\pgfqpoint{6.451004in}{3.289705in}}%
\pgfpathlineto{\pgfqpoint{6.482017in}{3.290354in}}%
\pgfpathlineto{\pgfqpoint{6.512548in}{3.288757in}}%
\pgfpathlineto{\pgfqpoint{6.542601in}{3.284926in}}%
\pgfpathlineto{\pgfqpoint{6.572178in}{3.278871in}}%
\pgfpathlineto{\pgfqpoint{6.601281in}{3.270605in}}%
\pgfpathlineto{\pgfqpoint{6.629913in}{3.260138in}}%
\pgfpathlineto{\pgfqpoint{6.658075in}{3.247480in}}%
\pgfpathlineto{\pgfqpoint{6.685771in}{3.232644in}}%
\pgfpathlineto{\pgfqpoint{6.713002in}{3.215640in}}%
\pgfpathlineto{\pgfqpoint{6.739771in}{3.196478in}}%
\pgfpathlineto{\pgfqpoint{6.766080in}{3.175170in}}%
\pgfpathlineto{\pgfqpoint{6.791931in}{3.151726in}}%
\pgfpathlineto{\pgfqpoint{6.817328in}{3.126157in}}%
\pgfpathlineto{\pgfqpoint{6.842271in}{3.098474in}}%
\pgfpathlineto{\pgfqpoint{6.866764in}{3.068687in}}%
\pgfpathlineto{\pgfqpoint{6.890808in}{3.036806in}}%
\pgfpathlineto{\pgfqpoint{6.914406in}{3.002843in}}%
\pgfpathlineto{\pgfqpoint{6.937560in}{2.966807in}}%
\pgfpathlineto{\pgfqpoint{6.960272in}{2.928709in}}%
\pgfpathlineto{\pgfqpoint{6.982544in}{2.888559in}}%
\pgfpathlineto{\pgfqpoint{7.004380in}{2.846368in}}%
\pgfpathlineto{\pgfqpoint{7.025780in}{2.802145in}}%
\pgfpathlineto{\pgfqpoint{7.046747in}{2.755901in}}%
\pgfpathlineto{\pgfqpoint{7.067283in}{2.707646in}}%
\pgfpathlineto{\pgfqpoint{7.087391in}{2.657390in}}%
\pgfpathlineto{\pgfqpoint{7.107072in}{2.605143in}}%
\pgfpathlineto{\pgfqpoint{7.126329in}{2.550915in}}%
\pgfpathlineto{\pgfqpoint{7.145163in}{2.494716in}}%
\pgfpathlineto{\pgfqpoint{7.163577in}{2.436556in}}%
\pgfpathlineto{\pgfqpoint{7.181574in}{2.376444in}}%
\pgfpathlineto{\pgfqpoint{7.199154in}{2.314390in}}%
\pgfpathlineto{\pgfqpoint{7.216320in}{2.250404in}}%
\pgfpathlineto{\pgfqpoint{7.233075in}{2.184496in}}%
\pgfpathlineto{\pgfqpoint{7.249419in}{2.116676in}}%
\pgfpathlineto{\pgfqpoint{7.265357in}{2.046952in}}%
\pgfpathlineto{\pgfqpoint{7.280888in}{1.975334in}}%
\pgfpathlineto{\pgfqpoint{7.296016in}{1.901832in}}%
\pgfpathlineto{\pgfqpoint{7.310742in}{1.826456in}}%
\pgfpathlineto{\pgfqpoint{7.325068in}{1.749214in}}%
\pgfpathlineto{\pgfqpoint{7.338997in}{1.670116in}}%
\pgfpathlineto{\pgfqpoint{7.352530in}{1.589171in}}%
\pgfpathlineto{\pgfqpoint{7.365670in}{1.506388in}}%
\pgfpathlineto{\pgfqpoint{7.378417in}{1.421777in}}%
\pgfpathlineto{\pgfqpoint{7.390776in}{1.335347in}}%
\pgfpathlineto{\pgfqpoint{7.402746in}{1.247107in}}%
\pgfpathlineto{\pgfqpoint{7.414330in}{1.157066in}}%
\pgfpathlineto{\pgfqpoint{7.425531in}{1.065233in}}%
\pgfpathlineto{\pgfqpoint{7.436350in}{0.971616in}}%
\pgfpathlineto{\pgfqpoint{7.446788in}{0.876226in}}%
\pgfpathlineto{\pgfqpoint{7.456848in}{0.779070in}}%
\pgfusepath{stroke}%
\end{pgfscope}%
\begin{pgfscope}%
\pgfpathrectangle{\pgfqpoint{1.250000in}{0.400000in}}{\pgfqpoint{7.750000in}{3.200000in}} %
\pgfusepath{clip}%
\pgfsetbuttcap%
\pgfsetroundjoin%
\definecolor{currentfill}{rgb}{0.000000,0.000000,1.000000}%
\pgfsetfillcolor{currentfill}%
\pgfsetlinewidth{0.501875pt}%
\definecolor{currentstroke}{rgb}{0.000000,0.000000,1.000000}%
\pgfsetstrokecolor{currentstroke}%
\pgfsetdash{}{0pt}%
\pgfsys@defobject{currentmarker}{\pgfqpoint{-0.041667in}{-0.041667in}}{\pgfqpoint{0.041667in}{0.041667in}}{%
\pgfpathmoveto{\pgfqpoint{-0.041667in}{0.000000in}}%
\pgfpathlineto{\pgfqpoint{0.041667in}{0.000000in}}%
\pgfpathmoveto{\pgfqpoint{0.000000in}{-0.041667in}}%
\pgfpathlineto{\pgfqpoint{0.000000in}{0.041667in}}%
\pgfusepath{stroke,fill}%
}%
\begin{pgfscope}%
\pgfsys@transformshift{4.571429in}{0.857143in}%
\pgfsys@useobject{currentmarker}{}%
\end{pgfscope}%
\begin{pgfscope}%
\pgfsys@transformshift{4.626183in}{0.968615in}%
\pgfsys@useobject{currentmarker}{}%
\end{pgfscope}%
\begin{pgfscope}%
\pgfsys@transformshift{4.680338in}{1.077287in}%
\pgfsys@useobject{currentmarker}{}%
\end{pgfscope}%
\begin{pgfscope}%
\pgfsys@transformshift{4.733895in}{1.183174in}%
\pgfsys@useobject{currentmarker}{}%
\end{pgfscope}%
\begin{pgfscope}%
\pgfsys@transformshift{4.786859in}{1.286289in}%
\pgfsys@useobject{currentmarker}{}%
\end{pgfscope}%
\begin{pgfscope}%
\pgfsys@transformshift{4.839233in}{1.386646in}%
\pgfsys@useobject{currentmarker}{}%
\end{pgfscope}%
\begin{pgfscope}%
\pgfsys@transformshift{4.891018in}{1.484259in}%
\pgfsys@useobject{currentmarker}{}%
\end{pgfscope}%
\begin{pgfscope}%
\pgfsys@transformshift{4.942219in}{1.579142in}%
\pgfsys@useobject{currentmarker}{}%
\end{pgfscope}%
\begin{pgfscope}%
\pgfsys@transformshift{4.992837in}{1.671308in}%
\pgfsys@useobject{currentmarker}{}%
\end{pgfscope}%
\begin{pgfscope}%
\pgfsys@transformshift{5.042876in}{1.760771in}%
\pgfsys@useobject{currentmarker}{}%
\end{pgfscope}%
\begin{pgfscope}%
\pgfsys@transformshift{5.092340in}{1.847544in}%
\pgfsys@useobject{currentmarker}{}%
\end{pgfscope}%
\begin{pgfscope}%
\pgfsys@transformshift{5.141229in}{1.931641in}%
\pgfsys@useobject{currentmarker}{}%
\end{pgfscope}%
\begin{pgfscope}%
\pgfsys@transformshift{5.189549in}{2.013075in}%
\pgfsys@useobject{currentmarker}{}%
\end{pgfscope}%
\begin{pgfscope}%
\pgfsys@transformshift{5.237301in}{2.091860in}%
\pgfsys@useobject{currentmarker}{}%
\end{pgfscope}%
\begin{pgfscope}%
\pgfsys@transformshift{5.284488in}{2.168009in}%
\pgfsys@useobject{currentmarker}{}%
\end{pgfscope}%
\begin{pgfscope}%
\pgfsys@transformshift{5.331113in}{2.241535in}%
\pgfsys@useobject{currentmarker}{}%
\end{pgfscope}%
\begin{pgfscope}%
\pgfsys@transformshift{5.377179in}{2.312451in}%
\pgfsys@useobject{currentmarker}{}%
\end{pgfscope}%
\begin{pgfscope}%
\pgfsys@transformshift{5.422689in}{2.380769in}%
\pgfsys@useobject{currentmarker}{}%
\end{pgfscope}%
\begin{pgfscope}%
\pgfsys@transformshift{5.467645in}{2.446504in}%
\pgfsys@useobject{currentmarker}{}%
\end{pgfscope}%
\begin{pgfscope}%
\pgfsys@transformshift{5.512050in}{2.509668in}%
\pgfsys@useobject{currentmarker}{}%
\end{pgfscope}%
\begin{pgfscope}%
\pgfsys@transformshift{5.555908in}{2.570274in}%
\pgfsys@useobject{currentmarker}{}%
\end{pgfscope}%
\begin{pgfscope}%
\pgfsys@transformshift{5.599220in}{2.628335in}%
\pgfsys@useobject{currentmarker}{}%
\end{pgfscope}%
\begin{pgfscope}%
\pgfsys@transformshift{5.641990in}{2.683863in}%
\pgfsys@useobject{currentmarker}{}%
\end{pgfscope}%
\begin{pgfscope}%
\pgfsys@transformshift{5.684219in}{2.736871in}%
\pgfsys@useobject{currentmarker}{}%
\end{pgfscope}%
\begin{pgfscope}%
\pgfsys@transformshift{5.725912in}{2.787372in}%
\pgfsys@useobject{currentmarker}{}%
\end{pgfscope}%
\begin{pgfscope}%
\pgfsys@transformshift{5.767070in}{2.835378in}%
\pgfsys@useobject{currentmarker}{}%
\end{pgfscope}%
\begin{pgfscope}%
\pgfsys@transformshift{5.807696in}{2.880902in}%
\pgfsys@useobject{currentmarker}{}%
\end{pgfscope}%
\begin{pgfscope}%
\pgfsys@transformshift{5.847794in}{2.923955in}%
\pgfsys@useobject{currentmarker}{}%
\end{pgfscope}%
\begin{pgfscope}%
\pgfsys@transformshift{5.887364in}{2.964552in}%
\pgfsys@useobject{currentmarker}{}%
\end{pgfscope}%
\begin{pgfscope}%
\pgfsys@transformshift{5.926411in}{3.002703in}%
\pgfsys@useobject{currentmarker}{}%
\end{pgfscope}%
\begin{pgfscope}%
\pgfsys@transformshift{5.964937in}{3.038421in}%
\pgfsys@useobject{currentmarker}{}%
\end{pgfscope}%
\begin{pgfscope}%
\pgfsys@transformshift{6.002944in}{3.071718in}%
\pgfsys@useobject{currentmarker}{}%
\end{pgfscope}%
\begin{pgfscope}%
\pgfsys@transformshift{6.040435in}{3.102606in}%
\pgfsys@useobject{currentmarker}{}%
\end{pgfscope}%
\begin{pgfscope}%
\pgfsys@transformshift{6.077413in}{3.131097in}%
\pgfsys@useobject{currentmarker}{}%
\end{pgfscope}%
\begin{pgfscope}%
\pgfsys@transformshift{6.113879in}{3.157204in}%
\pgfsys@useobject{currentmarker}{}%
\end{pgfscope}%
\begin{pgfscope}%
\pgfsys@transformshift{6.149838in}{3.180938in}%
\pgfsys@useobject{currentmarker}{}%
\end{pgfscope}%
\begin{pgfscope}%
\pgfsys@transformshift{6.185290in}{3.202311in}%
\pgfsys@useobject{currentmarker}{}%
\end{pgfscope}%
\begin{pgfscope}%
\pgfsys@transformshift{6.220240in}{3.221335in}%
\pgfsys@useobject{currentmarker}{}%
\end{pgfscope}%
\begin{pgfscope}%
\pgfsys@transformshift{6.254688in}{3.238022in}%
\pgfsys@useobject{currentmarker}{}%
\end{pgfscope}%
\begin{pgfscope}%
\pgfsys@transformshift{6.288639in}{3.252382in}%
\pgfsys@useobject{currentmarker}{}%
\end{pgfscope}%
\begin{pgfscope}%
\pgfsys@transformshift{6.322093in}{3.264429in}%
\pgfsys@useobject{currentmarker}{}%
\end{pgfscope}%
\begin{pgfscope}%
\pgfsys@transformshift{6.355055in}{3.274173in}%
\pgfsys@useobject{currentmarker}{}%
\end{pgfscope}%
\begin{pgfscope}%
\pgfsys@transformshift{6.387525in}{3.281626in}%
\pgfsys@useobject{currentmarker}{}%
\end{pgfscope}%
\begin{pgfscope}%
\pgfsys@transformshift{6.419507in}{3.286800in}%
\pgfsys@useobject{currentmarker}{}%
\end{pgfscope}%
\begin{pgfscope}%
\pgfsys@transformshift{6.451004in}{3.289705in}%
\pgfsys@useobject{currentmarker}{}%
\end{pgfscope}%
\begin{pgfscope}%
\pgfsys@transformshift{6.482017in}{3.290354in}%
\pgfsys@useobject{currentmarker}{}%
\end{pgfscope}%
\begin{pgfscope}%
\pgfsys@transformshift{6.512548in}{3.288757in}%
\pgfsys@useobject{currentmarker}{}%
\end{pgfscope}%
\begin{pgfscope}%
\pgfsys@transformshift{6.542601in}{3.284926in}%
\pgfsys@useobject{currentmarker}{}%
\end{pgfscope}%
\begin{pgfscope}%
\pgfsys@transformshift{6.572178in}{3.278871in}%
\pgfsys@useobject{currentmarker}{}%
\end{pgfscope}%
\begin{pgfscope}%
\pgfsys@transformshift{6.601281in}{3.270605in}%
\pgfsys@useobject{currentmarker}{}%
\end{pgfscope}%
\begin{pgfscope}%
\pgfsys@transformshift{6.629913in}{3.260138in}%
\pgfsys@useobject{currentmarker}{}%
\end{pgfscope}%
\begin{pgfscope}%
\pgfsys@transformshift{6.658075in}{3.247480in}%
\pgfsys@useobject{currentmarker}{}%
\end{pgfscope}%
\begin{pgfscope}%
\pgfsys@transformshift{6.685771in}{3.232644in}%
\pgfsys@useobject{currentmarker}{}%
\end{pgfscope}%
\begin{pgfscope}%
\pgfsys@transformshift{6.713002in}{3.215640in}%
\pgfsys@useobject{currentmarker}{}%
\end{pgfscope}%
\begin{pgfscope}%
\pgfsys@transformshift{6.739771in}{3.196478in}%
\pgfsys@useobject{currentmarker}{}%
\end{pgfscope}%
\begin{pgfscope}%
\pgfsys@transformshift{6.766080in}{3.175170in}%
\pgfsys@useobject{currentmarker}{}%
\end{pgfscope}%
\begin{pgfscope}%
\pgfsys@transformshift{6.791931in}{3.151726in}%
\pgfsys@useobject{currentmarker}{}%
\end{pgfscope}%
\begin{pgfscope}%
\pgfsys@transformshift{6.817328in}{3.126157in}%
\pgfsys@useobject{currentmarker}{}%
\end{pgfscope}%
\begin{pgfscope}%
\pgfsys@transformshift{6.842271in}{3.098474in}%
\pgfsys@useobject{currentmarker}{}%
\end{pgfscope}%
\begin{pgfscope}%
\pgfsys@transformshift{6.866764in}{3.068687in}%
\pgfsys@useobject{currentmarker}{}%
\end{pgfscope}%
\begin{pgfscope}%
\pgfsys@transformshift{6.890808in}{3.036806in}%
\pgfsys@useobject{currentmarker}{}%
\end{pgfscope}%
\begin{pgfscope}%
\pgfsys@transformshift{6.914406in}{3.002843in}%
\pgfsys@useobject{currentmarker}{}%
\end{pgfscope}%
\begin{pgfscope}%
\pgfsys@transformshift{6.937560in}{2.966807in}%
\pgfsys@useobject{currentmarker}{}%
\end{pgfscope}%
\begin{pgfscope}%
\pgfsys@transformshift{6.960272in}{2.928709in}%
\pgfsys@useobject{currentmarker}{}%
\end{pgfscope}%
\begin{pgfscope}%
\pgfsys@transformshift{6.982544in}{2.888559in}%
\pgfsys@useobject{currentmarker}{}%
\end{pgfscope}%
\begin{pgfscope}%
\pgfsys@transformshift{7.004380in}{2.846368in}%
\pgfsys@useobject{currentmarker}{}%
\end{pgfscope}%
\begin{pgfscope}%
\pgfsys@transformshift{7.025780in}{2.802145in}%
\pgfsys@useobject{currentmarker}{}%
\end{pgfscope}%
\begin{pgfscope}%
\pgfsys@transformshift{7.046747in}{2.755901in}%
\pgfsys@useobject{currentmarker}{}%
\end{pgfscope}%
\begin{pgfscope}%
\pgfsys@transformshift{7.067283in}{2.707646in}%
\pgfsys@useobject{currentmarker}{}%
\end{pgfscope}%
\begin{pgfscope}%
\pgfsys@transformshift{7.087391in}{2.657390in}%
\pgfsys@useobject{currentmarker}{}%
\end{pgfscope}%
\begin{pgfscope}%
\pgfsys@transformshift{7.107072in}{2.605143in}%
\pgfsys@useobject{currentmarker}{}%
\end{pgfscope}%
\begin{pgfscope}%
\pgfsys@transformshift{7.126329in}{2.550915in}%
\pgfsys@useobject{currentmarker}{}%
\end{pgfscope}%
\begin{pgfscope}%
\pgfsys@transformshift{7.145163in}{2.494716in}%
\pgfsys@useobject{currentmarker}{}%
\end{pgfscope}%
\begin{pgfscope}%
\pgfsys@transformshift{7.163577in}{2.436556in}%
\pgfsys@useobject{currentmarker}{}%
\end{pgfscope}%
\begin{pgfscope}%
\pgfsys@transformshift{7.181574in}{2.376444in}%
\pgfsys@useobject{currentmarker}{}%
\end{pgfscope}%
\begin{pgfscope}%
\pgfsys@transformshift{7.199154in}{2.314390in}%
\pgfsys@useobject{currentmarker}{}%
\end{pgfscope}%
\begin{pgfscope}%
\pgfsys@transformshift{7.216320in}{2.250404in}%
\pgfsys@useobject{currentmarker}{}%
\end{pgfscope}%
\begin{pgfscope}%
\pgfsys@transformshift{7.233075in}{2.184496in}%
\pgfsys@useobject{currentmarker}{}%
\end{pgfscope}%
\begin{pgfscope}%
\pgfsys@transformshift{7.249419in}{2.116676in}%
\pgfsys@useobject{currentmarker}{}%
\end{pgfscope}%
\begin{pgfscope}%
\pgfsys@transformshift{7.265357in}{2.046952in}%
\pgfsys@useobject{currentmarker}{}%
\end{pgfscope}%
\begin{pgfscope}%
\pgfsys@transformshift{7.280888in}{1.975334in}%
\pgfsys@useobject{currentmarker}{}%
\end{pgfscope}%
\begin{pgfscope}%
\pgfsys@transformshift{7.296016in}{1.901832in}%
\pgfsys@useobject{currentmarker}{}%
\end{pgfscope}%
\begin{pgfscope}%
\pgfsys@transformshift{7.310742in}{1.826456in}%
\pgfsys@useobject{currentmarker}{}%
\end{pgfscope}%
\begin{pgfscope}%
\pgfsys@transformshift{7.325068in}{1.749214in}%
\pgfsys@useobject{currentmarker}{}%
\end{pgfscope}%
\begin{pgfscope}%
\pgfsys@transformshift{7.338997in}{1.670116in}%
\pgfsys@useobject{currentmarker}{}%
\end{pgfscope}%
\begin{pgfscope}%
\pgfsys@transformshift{7.352530in}{1.589171in}%
\pgfsys@useobject{currentmarker}{}%
\end{pgfscope}%
\begin{pgfscope}%
\pgfsys@transformshift{7.365670in}{1.506388in}%
\pgfsys@useobject{currentmarker}{}%
\end{pgfscope}%
\begin{pgfscope}%
\pgfsys@transformshift{7.378417in}{1.421777in}%
\pgfsys@useobject{currentmarker}{}%
\end{pgfscope}%
\begin{pgfscope}%
\pgfsys@transformshift{7.390776in}{1.335347in}%
\pgfsys@useobject{currentmarker}{}%
\end{pgfscope}%
\begin{pgfscope}%
\pgfsys@transformshift{7.402746in}{1.247107in}%
\pgfsys@useobject{currentmarker}{}%
\end{pgfscope}%
\begin{pgfscope}%
\pgfsys@transformshift{7.414330in}{1.157066in}%
\pgfsys@useobject{currentmarker}{}%
\end{pgfscope}%
\begin{pgfscope}%
\pgfsys@transformshift{7.425531in}{1.065233in}%
\pgfsys@useobject{currentmarker}{}%
\end{pgfscope}%
\begin{pgfscope}%
\pgfsys@transformshift{7.436350in}{0.971616in}%
\pgfsys@useobject{currentmarker}{}%
\end{pgfscope}%
\begin{pgfscope}%
\pgfsys@transformshift{7.446788in}{0.876226in}%
\pgfsys@useobject{currentmarker}{}%
\end{pgfscope}%
\begin{pgfscope}%
\pgfsys@transformshift{7.456848in}{0.779070in}%
\pgfsys@useobject{currentmarker}{}%
\end{pgfscope}%
\end{pgfscope}%
\begin{pgfscope}%
\pgfpathrectangle{\pgfqpoint{1.250000in}{0.400000in}}{\pgfqpoint{7.750000in}{3.200000in}} %
\pgfusepath{clip}%
\pgfsetrectcap%
\pgfsetroundjoin%
\pgfsetlinewidth{1.003750pt}%
\definecolor{currentstroke}{rgb}{0.000000,0.500000,0.000000}%
\pgfsetstrokecolor{currentstroke}%
\pgfsetdash{}{0pt}%
\pgfpathmoveto{\pgfqpoint{4.571429in}{0.857143in}}%
\pgfpathlineto{\pgfqpoint{4.626374in}{0.968615in}}%
\pgfpathlineto{\pgfqpoint{4.680909in}{1.077287in}}%
\pgfpathlineto{\pgfqpoint{4.735036in}{1.183174in}}%
\pgfpathlineto{\pgfqpoint{4.788757in}{1.286289in}}%
\pgfpathlineto{\pgfqpoint{4.842074in}{1.386646in}}%
\pgfpathlineto{\pgfqpoint{4.894989in}{1.484259in}}%
\pgfpathlineto{\pgfqpoint{4.947504in}{1.579142in}}%
\pgfpathlineto{\pgfqpoint{4.999622in}{1.671308in}}%
\pgfpathlineto{\pgfqpoint{5.051343in}{1.760771in}}%
\pgfpathlineto{\pgfqpoint{5.102671in}{1.847544in}}%
\pgfpathlineto{\pgfqpoint{5.153607in}{1.931641in}}%
\pgfpathlineto{\pgfqpoint{5.204152in}{2.013075in}}%
\pgfpathlineto{\pgfqpoint{5.254310in}{2.091860in}}%
\pgfpathlineto{\pgfqpoint{5.304081in}{2.168009in}}%
\pgfpathlineto{\pgfqpoint{5.353469in}{2.241535in}}%
\pgfpathlineto{\pgfqpoint{5.402474in}{2.312451in}}%
\pgfpathlineto{\pgfqpoint{5.451099in}{2.380769in}}%
\pgfpathlineto{\pgfqpoint{5.499345in}{2.446504in}}%
\pgfpathlineto{\pgfqpoint{5.547215in}{2.509668in}}%
\pgfpathlineto{\pgfqpoint{5.594710in}{2.570274in}}%
\pgfpathlineto{\pgfqpoint{5.641832in}{2.628335in}}%
\pgfpathlineto{\pgfqpoint{5.688583in}{2.683863in}}%
\pgfpathlineto{\pgfqpoint{5.734966in}{2.736871in}}%
\pgfpathlineto{\pgfqpoint{5.780981in}{2.787372in}}%
\pgfpathlineto{\pgfqpoint{5.826630in}{2.835378in}}%
\pgfpathlineto{\pgfqpoint{5.871916in}{2.880902in}}%
\pgfpathlineto{\pgfqpoint{5.916841in}{2.923955in}}%
\pgfpathlineto{\pgfqpoint{5.961405in}{2.964552in}}%
\pgfpathlineto{\pgfqpoint{6.005611in}{3.002703in}}%
\pgfpathlineto{\pgfqpoint{6.049461in}{3.038421in}}%
\pgfpathlineto{\pgfqpoint{6.092957in}{3.071718in}}%
\pgfpathlineto{\pgfqpoint{6.136099in}{3.102606in}}%
\pgfpathlineto{\pgfqpoint{6.178891in}{3.131097in}}%
\pgfpathlineto{\pgfqpoint{6.221333in}{3.157204in}}%
\pgfpathlineto{\pgfqpoint{6.263428in}{3.180938in}}%
\pgfpathlineto{\pgfqpoint{6.305177in}{3.202311in}}%
\pgfpathlineto{\pgfqpoint{6.346582in}{3.221335in}}%
\pgfpathlineto{\pgfqpoint{6.387644in}{3.238022in}}%
\pgfpathlineto{\pgfqpoint{6.428366in}{3.252382in}}%
\pgfpathlineto{\pgfqpoint{6.468749in}{3.264429in}}%
\pgfpathlineto{\pgfqpoint{6.508795in}{3.274173in}}%
\pgfpathlineto{\pgfqpoint{6.548505in}{3.281626in}}%
\pgfpathlineto{\pgfqpoint{6.587881in}{3.286800in}}%
\pgfpathlineto{\pgfqpoint{6.626925in}{3.289705in}}%
\pgfpathlineto{\pgfqpoint{6.665639in}{3.290354in}}%
\pgfpathlineto{\pgfqpoint{6.704024in}{3.288757in}}%
\pgfpathlineto{\pgfqpoint{6.742081in}{3.284926in}}%
\pgfpathlineto{\pgfqpoint{6.779813in}{3.278871in}}%
\pgfpathlineto{\pgfqpoint{6.817221in}{3.270605in}}%
\pgfpathlineto{\pgfqpoint{6.854306in}{3.260138in}}%
\pgfpathlineto{\pgfqpoint{6.891071in}{3.247480in}}%
\pgfpathlineto{\pgfqpoint{6.927517in}{3.232644in}}%
\pgfpathlineto{\pgfqpoint{6.963645in}{3.215640in}}%
\pgfpathlineto{\pgfqpoint{6.999457in}{3.196478in}}%
\pgfpathlineto{\pgfqpoint{7.034955in}{3.175170in}}%
\pgfpathlineto{\pgfqpoint{7.070140in}{3.151726in}}%
\pgfpathlineto{\pgfqpoint{7.105013in}{3.126157in}}%
\pgfpathlineto{\pgfqpoint{7.139577in}{3.098474in}}%
\pgfpathlineto{\pgfqpoint{7.173833in}{3.068687in}}%
\pgfpathlineto{\pgfqpoint{7.207783in}{3.036806in}}%
\pgfpathlineto{\pgfqpoint{7.241427in}{3.002843in}}%
\pgfpathlineto{\pgfqpoint{7.274767in}{2.966807in}}%
\pgfpathlineto{\pgfqpoint{7.307806in}{2.928709in}}%
\pgfpathlineto{\pgfqpoint{7.340544in}{2.888559in}}%
\pgfpathlineto{\pgfqpoint{7.372983in}{2.846368in}}%
\pgfpathlineto{\pgfqpoint{7.405125in}{2.802145in}}%
\pgfpathlineto{\pgfqpoint{7.436971in}{2.755901in}}%
\pgfpathlineto{\pgfqpoint{7.468522in}{2.707646in}}%
\pgfpathlineto{\pgfqpoint{7.499780in}{2.657390in}}%
\pgfpathlineto{\pgfqpoint{7.530746in}{2.605143in}}%
\pgfpathlineto{\pgfqpoint{7.561422in}{2.550915in}}%
\pgfpathlineto{\pgfqpoint{7.591810in}{2.494716in}}%
\pgfpathlineto{\pgfqpoint{7.621910in}{2.436556in}}%
\pgfpathlineto{\pgfqpoint{7.651725in}{2.376444in}}%
\pgfpathlineto{\pgfqpoint{7.681255in}{2.314390in}}%
\pgfpathlineto{\pgfqpoint{7.710502in}{2.250404in}}%
\pgfpathlineto{\pgfqpoint{7.739467in}{2.184496in}}%
\pgfpathlineto{\pgfqpoint{7.768153in}{2.116676in}}%
\pgfpathlineto{\pgfqpoint{7.796559in}{2.046952in}}%
\pgfpathlineto{\pgfqpoint{7.824689in}{1.975334in}}%
\pgfpathlineto{\pgfqpoint{7.852542in}{1.901832in}}%
\pgfpathlineto{\pgfqpoint{7.880121in}{1.826456in}}%
\pgfpathlineto{\pgfqpoint{7.907426in}{1.749214in}}%
\pgfpathlineto{\pgfqpoint{7.934460in}{1.670116in}}%
\pgfpathlineto{\pgfqpoint{7.961223in}{1.589171in}}%
\pgfpathlineto{\pgfqpoint{7.987717in}{1.506388in}}%
\pgfpathlineto{\pgfqpoint{8.013944in}{1.421777in}}%
\pgfpathlineto{\pgfqpoint{8.039903in}{1.335347in}}%
\pgfpathlineto{\pgfqpoint{8.065598in}{1.247107in}}%
\pgfpathlineto{\pgfqpoint{8.091029in}{1.157066in}}%
\pgfpathlineto{\pgfqpoint{8.116198in}{1.065233in}}%
\pgfpathlineto{\pgfqpoint{8.141105in}{0.971616in}}%
\pgfpathlineto{\pgfqpoint{8.165752in}{0.876226in}}%
\pgfpathlineto{\pgfqpoint{8.190141in}{0.779070in}}%
\pgfusepath{stroke}%
\end{pgfscope}%
\begin{pgfscope}%
\pgfpathrectangle{\pgfqpoint{1.250000in}{0.400000in}}{\pgfqpoint{7.750000in}{3.200000in}} %
\pgfusepath{clip}%
\pgfsetbuttcap%
\pgfsetroundjoin%
\definecolor{currentfill}{rgb}{0.000000,0.500000,0.000000}%
\pgfsetfillcolor{currentfill}%
\pgfsetlinewidth{0.501875pt}%
\definecolor{currentstroke}{rgb}{0.000000,0.500000,0.000000}%
\pgfsetstrokecolor{currentstroke}%
\pgfsetdash{}{0pt}%
\pgfsys@defobject{currentmarker}{\pgfqpoint{-0.041667in}{-0.041667in}}{\pgfqpoint{0.041667in}{0.041667in}}{%
\pgfpathmoveto{\pgfqpoint{-0.041667in}{0.000000in}}%
\pgfpathlineto{\pgfqpoint{0.041667in}{0.000000in}}%
\pgfpathmoveto{\pgfqpoint{0.000000in}{-0.041667in}}%
\pgfpathlineto{\pgfqpoint{0.000000in}{0.041667in}}%
\pgfusepath{stroke,fill}%
}%
\begin{pgfscope}%
\pgfsys@transformshift{4.571429in}{0.857143in}%
\pgfsys@useobject{currentmarker}{}%
\end{pgfscope}%
\begin{pgfscope}%
\pgfsys@transformshift{4.626374in}{0.968615in}%
\pgfsys@useobject{currentmarker}{}%
\end{pgfscope}%
\begin{pgfscope}%
\pgfsys@transformshift{4.680909in}{1.077287in}%
\pgfsys@useobject{currentmarker}{}%
\end{pgfscope}%
\begin{pgfscope}%
\pgfsys@transformshift{4.735036in}{1.183174in}%
\pgfsys@useobject{currentmarker}{}%
\end{pgfscope}%
\begin{pgfscope}%
\pgfsys@transformshift{4.788757in}{1.286289in}%
\pgfsys@useobject{currentmarker}{}%
\end{pgfscope}%
\begin{pgfscope}%
\pgfsys@transformshift{4.842074in}{1.386646in}%
\pgfsys@useobject{currentmarker}{}%
\end{pgfscope}%
\begin{pgfscope}%
\pgfsys@transformshift{4.894989in}{1.484259in}%
\pgfsys@useobject{currentmarker}{}%
\end{pgfscope}%
\begin{pgfscope}%
\pgfsys@transformshift{4.947504in}{1.579142in}%
\pgfsys@useobject{currentmarker}{}%
\end{pgfscope}%
\begin{pgfscope}%
\pgfsys@transformshift{4.999622in}{1.671308in}%
\pgfsys@useobject{currentmarker}{}%
\end{pgfscope}%
\begin{pgfscope}%
\pgfsys@transformshift{5.051343in}{1.760771in}%
\pgfsys@useobject{currentmarker}{}%
\end{pgfscope}%
\begin{pgfscope}%
\pgfsys@transformshift{5.102671in}{1.847544in}%
\pgfsys@useobject{currentmarker}{}%
\end{pgfscope}%
\begin{pgfscope}%
\pgfsys@transformshift{5.153607in}{1.931641in}%
\pgfsys@useobject{currentmarker}{}%
\end{pgfscope}%
\begin{pgfscope}%
\pgfsys@transformshift{5.204152in}{2.013075in}%
\pgfsys@useobject{currentmarker}{}%
\end{pgfscope}%
\begin{pgfscope}%
\pgfsys@transformshift{5.254310in}{2.091860in}%
\pgfsys@useobject{currentmarker}{}%
\end{pgfscope}%
\begin{pgfscope}%
\pgfsys@transformshift{5.304081in}{2.168009in}%
\pgfsys@useobject{currentmarker}{}%
\end{pgfscope}%
\begin{pgfscope}%
\pgfsys@transformshift{5.353469in}{2.241535in}%
\pgfsys@useobject{currentmarker}{}%
\end{pgfscope}%
\begin{pgfscope}%
\pgfsys@transformshift{5.402474in}{2.312451in}%
\pgfsys@useobject{currentmarker}{}%
\end{pgfscope}%
\begin{pgfscope}%
\pgfsys@transformshift{5.451099in}{2.380769in}%
\pgfsys@useobject{currentmarker}{}%
\end{pgfscope}%
\begin{pgfscope}%
\pgfsys@transformshift{5.499345in}{2.446504in}%
\pgfsys@useobject{currentmarker}{}%
\end{pgfscope}%
\begin{pgfscope}%
\pgfsys@transformshift{5.547215in}{2.509668in}%
\pgfsys@useobject{currentmarker}{}%
\end{pgfscope}%
\begin{pgfscope}%
\pgfsys@transformshift{5.594710in}{2.570274in}%
\pgfsys@useobject{currentmarker}{}%
\end{pgfscope}%
\begin{pgfscope}%
\pgfsys@transformshift{5.641832in}{2.628335in}%
\pgfsys@useobject{currentmarker}{}%
\end{pgfscope}%
\begin{pgfscope}%
\pgfsys@transformshift{5.688583in}{2.683863in}%
\pgfsys@useobject{currentmarker}{}%
\end{pgfscope}%
\begin{pgfscope}%
\pgfsys@transformshift{5.734966in}{2.736871in}%
\pgfsys@useobject{currentmarker}{}%
\end{pgfscope}%
\begin{pgfscope}%
\pgfsys@transformshift{5.780981in}{2.787372in}%
\pgfsys@useobject{currentmarker}{}%
\end{pgfscope}%
\begin{pgfscope}%
\pgfsys@transformshift{5.826630in}{2.835378in}%
\pgfsys@useobject{currentmarker}{}%
\end{pgfscope}%
\begin{pgfscope}%
\pgfsys@transformshift{5.871916in}{2.880902in}%
\pgfsys@useobject{currentmarker}{}%
\end{pgfscope}%
\begin{pgfscope}%
\pgfsys@transformshift{5.916841in}{2.923955in}%
\pgfsys@useobject{currentmarker}{}%
\end{pgfscope}%
\begin{pgfscope}%
\pgfsys@transformshift{5.961405in}{2.964552in}%
\pgfsys@useobject{currentmarker}{}%
\end{pgfscope}%
\begin{pgfscope}%
\pgfsys@transformshift{6.005611in}{3.002703in}%
\pgfsys@useobject{currentmarker}{}%
\end{pgfscope}%
\begin{pgfscope}%
\pgfsys@transformshift{6.049461in}{3.038421in}%
\pgfsys@useobject{currentmarker}{}%
\end{pgfscope}%
\begin{pgfscope}%
\pgfsys@transformshift{6.092957in}{3.071718in}%
\pgfsys@useobject{currentmarker}{}%
\end{pgfscope}%
\begin{pgfscope}%
\pgfsys@transformshift{6.136099in}{3.102606in}%
\pgfsys@useobject{currentmarker}{}%
\end{pgfscope}%
\begin{pgfscope}%
\pgfsys@transformshift{6.178891in}{3.131097in}%
\pgfsys@useobject{currentmarker}{}%
\end{pgfscope}%
\begin{pgfscope}%
\pgfsys@transformshift{6.221333in}{3.157204in}%
\pgfsys@useobject{currentmarker}{}%
\end{pgfscope}%
\begin{pgfscope}%
\pgfsys@transformshift{6.263428in}{3.180938in}%
\pgfsys@useobject{currentmarker}{}%
\end{pgfscope}%
\begin{pgfscope}%
\pgfsys@transformshift{6.305177in}{3.202311in}%
\pgfsys@useobject{currentmarker}{}%
\end{pgfscope}%
\begin{pgfscope}%
\pgfsys@transformshift{6.346582in}{3.221335in}%
\pgfsys@useobject{currentmarker}{}%
\end{pgfscope}%
\begin{pgfscope}%
\pgfsys@transformshift{6.387644in}{3.238022in}%
\pgfsys@useobject{currentmarker}{}%
\end{pgfscope}%
\begin{pgfscope}%
\pgfsys@transformshift{6.428366in}{3.252382in}%
\pgfsys@useobject{currentmarker}{}%
\end{pgfscope}%
\begin{pgfscope}%
\pgfsys@transformshift{6.468749in}{3.264429in}%
\pgfsys@useobject{currentmarker}{}%
\end{pgfscope}%
\begin{pgfscope}%
\pgfsys@transformshift{6.508795in}{3.274173in}%
\pgfsys@useobject{currentmarker}{}%
\end{pgfscope}%
\begin{pgfscope}%
\pgfsys@transformshift{6.548505in}{3.281626in}%
\pgfsys@useobject{currentmarker}{}%
\end{pgfscope}%
\begin{pgfscope}%
\pgfsys@transformshift{6.587881in}{3.286800in}%
\pgfsys@useobject{currentmarker}{}%
\end{pgfscope}%
\begin{pgfscope}%
\pgfsys@transformshift{6.626925in}{3.289705in}%
\pgfsys@useobject{currentmarker}{}%
\end{pgfscope}%
\begin{pgfscope}%
\pgfsys@transformshift{6.665639in}{3.290354in}%
\pgfsys@useobject{currentmarker}{}%
\end{pgfscope}%
\begin{pgfscope}%
\pgfsys@transformshift{6.704024in}{3.288757in}%
\pgfsys@useobject{currentmarker}{}%
\end{pgfscope}%
\begin{pgfscope}%
\pgfsys@transformshift{6.742081in}{3.284926in}%
\pgfsys@useobject{currentmarker}{}%
\end{pgfscope}%
\begin{pgfscope}%
\pgfsys@transformshift{6.779813in}{3.278871in}%
\pgfsys@useobject{currentmarker}{}%
\end{pgfscope}%
\begin{pgfscope}%
\pgfsys@transformshift{6.817221in}{3.270605in}%
\pgfsys@useobject{currentmarker}{}%
\end{pgfscope}%
\begin{pgfscope}%
\pgfsys@transformshift{6.854306in}{3.260138in}%
\pgfsys@useobject{currentmarker}{}%
\end{pgfscope}%
\begin{pgfscope}%
\pgfsys@transformshift{6.891071in}{3.247480in}%
\pgfsys@useobject{currentmarker}{}%
\end{pgfscope}%
\begin{pgfscope}%
\pgfsys@transformshift{6.927517in}{3.232644in}%
\pgfsys@useobject{currentmarker}{}%
\end{pgfscope}%
\begin{pgfscope}%
\pgfsys@transformshift{6.963645in}{3.215640in}%
\pgfsys@useobject{currentmarker}{}%
\end{pgfscope}%
\begin{pgfscope}%
\pgfsys@transformshift{6.999457in}{3.196478in}%
\pgfsys@useobject{currentmarker}{}%
\end{pgfscope}%
\begin{pgfscope}%
\pgfsys@transformshift{7.034955in}{3.175170in}%
\pgfsys@useobject{currentmarker}{}%
\end{pgfscope}%
\begin{pgfscope}%
\pgfsys@transformshift{7.070140in}{3.151726in}%
\pgfsys@useobject{currentmarker}{}%
\end{pgfscope}%
\begin{pgfscope}%
\pgfsys@transformshift{7.105013in}{3.126157in}%
\pgfsys@useobject{currentmarker}{}%
\end{pgfscope}%
\begin{pgfscope}%
\pgfsys@transformshift{7.139577in}{3.098474in}%
\pgfsys@useobject{currentmarker}{}%
\end{pgfscope}%
\begin{pgfscope}%
\pgfsys@transformshift{7.173833in}{3.068687in}%
\pgfsys@useobject{currentmarker}{}%
\end{pgfscope}%
\begin{pgfscope}%
\pgfsys@transformshift{7.207783in}{3.036806in}%
\pgfsys@useobject{currentmarker}{}%
\end{pgfscope}%
\begin{pgfscope}%
\pgfsys@transformshift{7.241427in}{3.002843in}%
\pgfsys@useobject{currentmarker}{}%
\end{pgfscope}%
\begin{pgfscope}%
\pgfsys@transformshift{7.274767in}{2.966807in}%
\pgfsys@useobject{currentmarker}{}%
\end{pgfscope}%
\begin{pgfscope}%
\pgfsys@transformshift{7.307806in}{2.928709in}%
\pgfsys@useobject{currentmarker}{}%
\end{pgfscope}%
\begin{pgfscope}%
\pgfsys@transformshift{7.340544in}{2.888559in}%
\pgfsys@useobject{currentmarker}{}%
\end{pgfscope}%
\begin{pgfscope}%
\pgfsys@transformshift{7.372983in}{2.846368in}%
\pgfsys@useobject{currentmarker}{}%
\end{pgfscope}%
\begin{pgfscope}%
\pgfsys@transformshift{7.405125in}{2.802145in}%
\pgfsys@useobject{currentmarker}{}%
\end{pgfscope}%
\begin{pgfscope}%
\pgfsys@transformshift{7.436971in}{2.755901in}%
\pgfsys@useobject{currentmarker}{}%
\end{pgfscope}%
\begin{pgfscope}%
\pgfsys@transformshift{7.468522in}{2.707646in}%
\pgfsys@useobject{currentmarker}{}%
\end{pgfscope}%
\begin{pgfscope}%
\pgfsys@transformshift{7.499780in}{2.657390in}%
\pgfsys@useobject{currentmarker}{}%
\end{pgfscope}%
\begin{pgfscope}%
\pgfsys@transformshift{7.530746in}{2.605143in}%
\pgfsys@useobject{currentmarker}{}%
\end{pgfscope}%
\begin{pgfscope}%
\pgfsys@transformshift{7.561422in}{2.550915in}%
\pgfsys@useobject{currentmarker}{}%
\end{pgfscope}%
\begin{pgfscope}%
\pgfsys@transformshift{7.591810in}{2.494716in}%
\pgfsys@useobject{currentmarker}{}%
\end{pgfscope}%
\begin{pgfscope}%
\pgfsys@transformshift{7.621910in}{2.436556in}%
\pgfsys@useobject{currentmarker}{}%
\end{pgfscope}%
\begin{pgfscope}%
\pgfsys@transformshift{7.651725in}{2.376444in}%
\pgfsys@useobject{currentmarker}{}%
\end{pgfscope}%
\begin{pgfscope}%
\pgfsys@transformshift{7.681255in}{2.314390in}%
\pgfsys@useobject{currentmarker}{}%
\end{pgfscope}%
\begin{pgfscope}%
\pgfsys@transformshift{7.710502in}{2.250404in}%
\pgfsys@useobject{currentmarker}{}%
\end{pgfscope}%
\begin{pgfscope}%
\pgfsys@transformshift{7.739467in}{2.184496in}%
\pgfsys@useobject{currentmarker}{}%
\end{pgfscope}%
\begin{pgfscope}%
\pgfsys@transformshift{7.768153in}{2.116676in}%
\pgfsys@useobject{currentmarker}{}%
\end{pgfscope}%
\begin{pgfscope}%
\pgfsys@transformshift{7.796559in}{2.046952in}%
\pgfsys@useobject{currentmarker}{}%
\end{pgfscope}%
\begin{pgfscope}%
\pgfsys@transformshift{7.824689in}{1.975334in}%
\pgfsys@useobject{currentmarker}{}%
\end{pgfscope}%
\begin{pgfscope}%
\pgfsys@transformshift{7.852542in}{1.901832in}%
\pgfsys@useobject{currentmarker}{}%
\end{pgfscope}%
\begin{pgfscope}%
\pgfsys@transformshift{7.880121in}{1.826456in}%
\pgfsys@useobject{currentmarker}{}%
\end{pgfscope}%
\begin{pgfscope}%
\pgfsys@transformshift{7.907426in}{1.749214in}%
\pgfsys@useobject{currentmarker}{}%
\end{pgfscope}%
\begin{pgfscope}%
\pgfsys@transformshift{7.934460in}{1.670116in}%
\pgfsys@useobject{currentmarker}{}%
\end{pgfscope}%
\begin{pgfscope}%
\pgfsys@transformshift{7.961223in}{1.589171in}%
\pgfsys@useobject{currentmarker}{}%
\end{pgfscope}%
\begin{pgfscope}%
\pgfsys@transformshift{7.987717in}{1.506388in}%
\pgfsys@useobject{currentmarker}{}%
\end{pgfscope}%
\begin{pgfscope}%
\pgfsys@transformshift{8.013944in}{1.421777in}%
\pgfsys@useobject{currentmarker}{}%
\end{pgfscope}%
\begin{pgfscope}%
\pgfsys@transformshift{8.039903in}{1.335347in}%
\pgfsys@useobject{currentmarker}{}%
\end{pgfscope}%
\begin{pgfscope}%
\pgfsys@transformshift{8.065598in}{1.247107in}%
\pgfsys@useobject{currentmarker}{}%
\end{pgfscope}%
\begin{pgfscope}%
\pgfsys@transformshift{8.091029in}{1.157066in}%
\pgfsys@useobject{currentmarker}{}%
\end{pgfscope}%
\begin{pgfscope}%
\pgfsys@transformshift{8.116198in}{1.065233in}%
\pgfsys@useobject{currentmarker}{}%
\end{pgfscope}%
\begin{pgfscope}%
\pgfsys@transformshift{8.141105in}{0.971616in}%
\pgfsys@useobject{currentmarker}{}%
\end{pgfscope}%
\begin{pgfscope}%
\pgfsys@transformshift{8.165752in}{0.876226in}%
\pgfsys@useobject{currentmarker}{}%
\end{pgfscope}%
\begin{pgfscope}%
\pgfsys@transformshift{8.190141in}{0.779070in}%
\pgfsys@useobject{currentmarker}{}%
\end{pgfscope}%
\end{pgfscope}%
\begin{pgfscope}%
\pgfpathrectangle{\pgfqpoint{1.250000in}{0.400000in}}{\pgfqpoint{7.750000in}{3.200000in}} %
\pgfusepath{clip}%
\pgfsetrectcap%
\pgfsetroundjoin%
\pgfsetlinewidth{1.003750pt}%
\definecolor{currentstroke}{rgb}{1.000000,0.000000,0.000000}%
\pgfsetstrokecolor{currentstroke}%
\pgfsetdash{}{0pt}%
\pgfpathmoveto{\pgfqpoint{4.571429in}{0.857143in}}%
\pgfpathlineto{\pgfqpoint{4.626564in}{0.968615in}}%
\pgfpathlineto{\pgfqpoint{4.681480in}{1.077287in}}%
\pgfpathlineto{\pgfqpoint{4.736176in}{1.183174in}}%
\pgfpathlineto{\pgfqpoint{4.790654in}{1.286289in}}%
\pgfpathlineto{\pgfqpoint{4.844915in}{1.386646in}}%
\pgfpathlineto{\pgfqpoint{4.898960in}{1.484259in}}%
\pgfpathlineto{\pgfqpoint{4.952790in}{1.579142in}}%
\pgfpathlineto{\pgfqpoint{5.006407in}{1.671308in}}%
\pgfpathlineto{\pgfqpoint{5.059810in}{1.760771in}}%
\pgfpathlineto{\pgfqpoint{5.113002in}{1.847544in}}%
\pgfpathlineto{\pgfqpoint{5.165984in}{1.931641in}}%
\pgfpathlineto{\pgfqpoint{5.218756in}{2.013075in}}%
\pgfpathlineto{\pgfqpoint{5.271319in}{2.091860in}}%
\pgfpathlineto{\pgfqpoint{5.323675in}{2.168009in}}%
\pgfpathlineto{\pgfqpoint{5.375824in}{2.241535in}}%
\pgfpathlineto{\pgfqpoint{5.427768in}{2.312451in}}%
\pgfpathlineto{\pgfqpoint{5.479508in}{2.380769in}}%
\pgfpathlineto{\pgfqpoint{5.531045in}{2.446504in}}%
\pgfpathlineto{\pgfqpoint{5.582379in}{2.509668in}}%
\pgfpathlineto{\pgfqpoint{5.633511in}{2.570274in}}%
\pgfpathlineto{\pgfqpoint{5.684444in}{2.628335in}}%
\pgfpathlineto{\pgfqpoint{5.735177in}{2.683863in}}%
\pgfpathlineto{\pgfqpoint{5.785712in}{2.736871in}}%
\pgfpathlineto{\pgfqpoint{5.836049in}{2.787372in}}%
\pgfpathlineto{\pgfqpoint{5.886190in}{2.835378in}}%
\pgfpathlineto{\pgfqpoint{5.936136in}{2.880902in}}%
\pgfpathlineto{\pgfqpoint{5.985888in}{2.923955in}}%
\pgfpathlineto{\pgfqpoint{6.035446in}{2.964552in}}%
\pgfpathlineto{\pgfqpoint{6.084811in}{3.002703in}}%
\pgfpathlineto{\pgfqpoint{6.133986in}{3.038421in}}%
\pgfpathlineto{\pgfqpoint{6.182969in}{3.071718in}}%
\pgfpathlineto{\pgfqpoint{6.231763in}{3.102606in}}%
\pgfpathlineto{\pgfqpoint{6.280369in}{3.131097in}}%
\pgfpathlineto{\pgfqpoint{6.328787in}{3.157204in}}%
\pgfpathlineto{\pgfqpoint{6.377018in}{3.180938in}}%
\pgfpathlineto{\pgfqpoint{6.425063in}{3.202311in}}%
\pgfpathlineto{\pgfqpoint{6.472923in}{3.221335in}}%
\pgfpathlineto{\pgfqpoint{6.520600in}{3.238022in}}%
\pgfpathlineto{\pgfqpoint{6.568093in}{3.252382in}}%
\pgfpathlineto{\pgfqpoint{6.615405in}{3.264429in}}%
\pgfpathlineto{\pgfqpoint{6.662535in}{3.274173in}}%
\pgfpathlineto{\pgfqpoint{6.709484in}{3.281626in}}%
\pgfpathlineto{\pgfqpoint{6.756255in}{3.286800in}}%
\pgfpathlineto{\pgfqpoint{6.802847in}{3.289705in}}%
\pgfpathlineto{\pgfqpoint{6.849261in}{3.290354in}}%
\pgfpathlineto{\pgfqpoint{6.895499in}{3.288757in}}%
\pgfpathlineto{\pgfqpoint{6.941561in}{3.284926in}}%
\pgfpathlineto{\pgfqpoint{6.987447in}{3.278871in}}%
\pgfpathlineto{\pgfqpoint{7.033160in}{3.270605in}}%
\pgfpathlineto{\pgfqpoint{7.078700in}{3.260138in}}%
\pgfpathlineto{\pgfqpoint{7.124067in}{3.247480in}}%
\pgfpathlineto{\pgfqpoint{7.169263in}{3.232644in}}%
\pgfpathlineto{\pgfqpoint{7.214288in}{3.215640in}}%
\pgfpathlineto{\pgfqpoint{7.259143in}{3.196478in}}%
\pgfpathlineto{\pgfqpoint{7.303830in}{3.175170in}}%
\pgfpathlineto{\pgfqpoint{7.348348in}{3.151726in}}%
\pgfpathlineto{\pgfqpoint{7.392699in}{3.126157in}}%
\pgfpathlineto{\pgfqpoint{7.436884in}{3.098474in}}%
\pgfpathlineto{\pgfqpoint{7.480903in}{3.068687in}}%
\pgfpathlineto{\pgfqpoint{7.524757in}{3.036806in}}%
\pgfpathlineto{\pgfqpoint{7.568448in}{3.002843in}}%
\pgfpathlineto{\pgfqpoint{7.611975in}{2.966807in}}%
\pgfpathlineto{\pgfqpoint{7.655340in}{2.928709in}}%
\pgfpathlineto{\pgfqpoint{7.698544in}{2.888559in}}%
\pgfpathlineto{\pgfqpoint{7.741587in}{2.846368in}}%
\pgfpathlineto{\pgfqpoint{7.784470in}{2.802145in}}%
\pgfpathlineto{\pgfqpoint{7.827194in}{2.755901in}}%
\pgfpathlineto{\pgfqpoint{7.869760in}{2.707646in}}%
\pgfpathlineto{\pgfqpoint{7.912168in}{2.657390in}}%
\pgfpathlineto{\pgfqpoint{7.954420in}{2.605143in}}%
\pgfpathlineto{\pgfqpoint{7.996516in}{2.550915in}}%
\pgfpathlineto{\pgfqpoint{8.038456in}{2.494716in}}%
\pgfpathlineto{\pgfqpoint{8.080243in}{2.436556in}}%
\pgfpathlineto{\pgfqpoint{8.121875in}{2.376444in}}%
\pgfpathlineto{\pgfqpoint{8.163355in}{2.314390in}}%
\pgfpathlineto{\pgfqpoint{8.204683in}{2.250404in}}%
\pgfpathlineto{\pgfqpoint{8.245860in}{2.184496in}}%
\pgfpathlineto{\pgfqpoint{8.286886in}{2.116676in}}%
\pgfpathlineto{\pgfqpoint{8.327762in}{2.046952in}}%
\pgfpathlineto{\pgfqpoint{8.368489in}{1.975334in}}%
\pgfpathlineto{\pgfqpoint{8.409068in}{1.901832in}}%
\pgfpathlineto{\pgfqpoint{8.449500in}{1.826456in}}%
\pgfpathlineto{\pgfqpoint{8.489784in}{1.749214in}}%
\pgfpathlineto{\pgfqpoint{8.529923in}{1.670116in}}%
\pgfpathlineto{\pgfqpoint{8.569916in}{1.589171in}}%
\pgfpathlineto{\pgfqpoint{8.609765in}{1.506388in}}%
\pgfpathlineto{\pgfqpoint{8.649470in}{1.421777in}}%
\pgfpathlineto{\pgfqpoint{8.689031in}{1.335347in}}%
\pgfpathlineto{\pgfqpoint{8.728450in}{1.247107in}}%
\pgfpathlineto{\pgfqpoint{8.767728in}{1.157066in}}%
\pgfpathlineto{\pgfqpoint{8.806864in}{1.065233in}}%
\pgfpathlineto{\pgfqpoint{8.845860in}{0.971616in}}%
\pgfpathlineto{\pgfqpoint{8.884717in}{0.876226in}}%
\pgfpathlineto{\pgfqpoint{8.923434in}{0.779070in}}%
\pgfusepath{stroke}%
\end{pgfscope}%
\begin{pgfscope}%
\pgfpathrectangle{\pgfqpoint{1.250000in}{0.400000in}}{\pgfqpoint{7.750000in}{3.200000in}} %
\pgfusepath{clip}%
\pgfsetbuttcap%
\pgfsetroundjoin%
\definecolor{currentfill}{rgb}{1.000000,0.000000,0.000000}%
\pgfsetfillcolor{currentfill}%
\pgfsetlinewidth{0.501875pt}%
\definecolor{currentstroke}{rgb}{1.000000,0.000000,0.000000}%
\pgfsetstrokecolor{currentstroke}%
\pgfsetdash{}{0pt}%
\pgfsys@defobject{currentmarker}{\pgfqpoint{-0.041667in}{-0.041667in}}{\pgfqpoint{0.041667in}{0.041667in}}{%
\pgfpathmoveto{\pgfqpoint{-0.041667in}{0.000000in}}%
\pgfpathlineto{\pgfqpoint{0.041667in}{0.000000in}}%
\pgfpathmoveto{\pgfqpoint{0.000000in}{-0.041667in}}%
\pgfpathlineto{\pgfqpoint{0.000000in}{0.041667in}}%
\pgfusepath{stroke,fill}%
}%
\begin{pgfscope}%
\pgfsys@transformshift{4.571429in}{0.857143in}%
\pgfsys@useobject{currentmarker}{}%
\end{pgfscope}%
\begin{pgfscope}%
\pgfsys@transformshift{4.626564in}{0.968615in}%
\pgfsys@useobject{currentmarker}{}%
\end{pgfscope}%
\begin{pgfscope}%
\pgfsys@transformshift{4.681480in}{1.077287in}%
\pgfsys@useobject{currentmarker}{}%
\end{pgfscope}%
\begin{pgfscope}%
\pgfsys@transformshift{4.736176in}{1.183174in}%
\pgfsys@useobject{currentmarker}{}%
\end{pgfscope}%
\begin{pgfscope}%
\pgfsys@transformshift{4.790654in}{1.286289in}%
\pgfsys@useobject{currentmarker}{}%
\end{pgfscope}%
\begin{pgfscope}%
\pgfsys@transformshift{4.844915in}{1.386646in}%
\pgfsys@useobject{currentmarker}{}%
\end{pgfscope}%
\begin{pgfscope}%
\pgfsys@transformshift{4.898960in}{1.484259in}%
\pgfsys@useobject{currentmarker}{}%
\end{pgfscope}%
\begin{pgfscope}%
\pgfsys@transformshift{4.952790in}{1.579142in}%
\pgfsys@useobject{currentmarker}{}%
\end{pgfscope}%
\begin{pgfscope}%
\pgfsys@transformshift{5.006407in}{1.671308in}%
\pgfsys@useobject{currentmarker}{}%
\end{pgfscope}%
\begin{pgfscope}%
\pgfsys@transformshift{5.059810in}{1.760771in}%
\pgfsys@useobject{currentmarker}{}%
\end{pgfscope}%
\begin{pgfscope}%
\pgfsys@transformshift{5.113002in}{1.847544in}%
\pgfsys@useobject{currentmarker}{}%
\end{pgfscope}%
\begin{pgfscope}%
\pgfsys@transformshift{5.165984in}{1.931641in}%
\pgfsys@useobject{currentmarker}{}%
\end{pgfscope}%
\begin{pgfscope}%
\pgfsys@transformshift{5.218756in}{2.013075in}%
\pgfsys@useobject{currentmarker}{}%
\end{pgfscope}%
\begin{pgfscope}%
\pgfsys@transformshift{5.271319in}{2.091860in}%
\pgfsys@useobject{currentmarker}{}%
\end{pgfscope}%
\begin{pgfscope}%
\pgfsys@transformshift{5.323675in}{2.168009in}%
\pgfsys@useobject{currentmarker}{}%
\end{pgfscope}%
\begin{pgfscope}%
\pgfsys@transformshift{5.375824in}{2.241535in}%
\pgfsys@useobject{currentmarker}{}%
\end{pgfscope}%
\begin{pgfscope}%
\pgfsys@transformshift{5.427768in}{2.312451in}%
\pgfsys@useobject{currentmarker}{}%
\end{pgfscope}%
\begin{pgfscope}%
\pgfsys@transformshift{5.479508in}{2.380769in}%
\pgfsys@useobject{currentmarker}{}%
\end{pgfscope}%
\begin{pgfscope}%
\pgfsys@transformshift{5.531045in}{2.446504in}%
\pgfsys@useobject{currentmarker}{}%
\end{pgfscope}%
\begin{pgfscope}%
\pgfsys@transformshift{5.582379in}{2.509668in}%
\pgfsys@useobject{currentmarker}{}%
\end{pgfscope}%
\begin{pgfscope}%
\pgfsys@transformshift{5.633511in}{2.570274in}%
\pgfsys@useobject{currentmarker}{}%
\end{pgfscope}%
\begin{pgfscope}%
\pgfsys@transformshift{5.684444in}{2.628335in}%
\pgfsys@useobject{currentmarker}{}%
\end{pgfscope}%
\begin{pgfscope}%
\pgfsys@transformshift{5.735177in}{2.683863in}%
\pgfsys@useobject{currentmarker}{}%
\end{pgfscope}%
\begin{pgfscope}%
\pgfsys@transformshift{5.785712in}{2.736871in}%
\pgfsys@useobject{currentmarker}{}%
\end{pgfscope}%
\begin{pgfscope}%
\pgfsys@transformshift{5.836049in}{2.787372in}%
\pgfsys@useobject{currentmarker}{}%
\end{pgfscope}%
\begin{pgfscope}%
\pgfsys@transformshift{5.886190in}{2.835378in}%
\pgfsys@useobject{currentmarker}{}%
\end{pgfscope}%
\begin{pgfscope}%
\pgfsys@transformshift{5.936136in}{2.880902in}%
\pgfsys@useobject{currentmarker}{}%
\end{pgfscope}%
\begin{pgfscope}%
\pgfsys@transformshift{5.985888in}{2.923955in}%
\pgfsys@useobject{currentmarker}{}%
\end{pgfscope}%
\begin{pgfscope}%
\pgfsys@transformshift{6.035446in}{2.964552in}%
\pgfsys@useobject{currentmarker}{}%
\end{pgfscope}%
\begin{pgfscope}%
\pgfsys@transformshift{6.084811in}{3.002703in}%
\pgfsys@useobject{currentmarker}{}%
\end{pgfscope}%
\begin{pgfscope}%
\pgfsys@transformshift{6.133986in}{3.038421in}%
\pgfsys@useobject{currentmarker}{}%
\end{pgfscope}%
\begin{pgfscope}%
\pgfsys@transformshift{6.182969in}{3.071718in}%
\pgfsys@useobject{currentmarker}{}%
\end{pgfscope}%
\begin{pgfscope}%
\pgfsys@transformshift{6.231763in}{3.102606in}%
\pgfsys@useobject{currentmarker}{}%
\end{pgfscope}%
\begin{pgfscope}%
\pgfsys@transformshift{6.280369in}{3.131097in}%
\pgfsys@useobject{currentmarker}{}%
\end{pgfscope}%
\begin{pgfscope}%
\pgfsys@transformshift{6.328787in}{3.157204in}%
\pgfsys@useobject{currentmarker}{}%
\end{pgfscope}%
\begin{pgfscope}%
\pgfsys@transformshift{6.377018in}{3.180938in}%
\pgfsys@useobject{currentmarker}{}%
\end{pgfscope}%
\begin{pgfscope}%
\pgfsys@transformshift{6.425063in}{3.202311in}%
\pgfsys@useobject{currentmarker}{}%
\end{pgfscope}%
\begin{pgfscope}%
\pgfsys@transformshift{6.472923in}{3.221335in}%
\pgfsys@useobject{currentmarker}{}%
\end{pgfscope}%
\begin{pgfscope}%
\pgfsys@transformshift{6.520600in}{3.238022in}%
\pgfsys@useobject{currentmarker}{}%
\end{pgfscope}%
\begin{pgfscope}%
\pgfsys@transformshift{6.568093in}{3.252382in}%
\pgfsys@useobject{currentmarker}{}%
\end{pgfscope}%
\begin{pgfscope}%
\pgfsys@transformshift{6.615405in}{3.264429in}%
\pgfsys@useobject{currentmarker}{}%
\end{pgfscope}%
\begin{pgfscope}%
\pgfsys@transformshift{6.662535in}{3.274173in}%
\pgfsys@useobject{currentmarker}{}%
\end{pgfscope}%
\begin{pgfscope}%
\pgfsys@transformshift{6.709484in}{3.281626in}%
\pgfsys@useobject{currentmarker}{}%
\end{pgfscope}%
\begin{pgfscope}%
\pgfsys@transformshift{6.756255in}{3.286800in}%
\pgfsys@useobject{currentmarker}{}%
\end{pgfscope}%
\begin{pgfscope}%
\pgfsys@transformshift{6.802847in}{3.289705in}%
\pgfsys@useobject{currentmarker}{}%
\end{pgfscope}%
\begin{pgfscope}%
\pgfsys@transformshift{6.849261in}{3.290354in}%
\pgfsys@useobject{currentmarker}{}%
\end{pgfscope}%
\begin{pgfscope}%
\pgfsys@transformshift{6.895499in}{3.288757in}%
\pgfsys@useobject{currentmarker}{}%
\end{pgfscope}%
\begin{pgfscope}%
\pgfsys@transformshift{6.941561in}{3.284926in}%
\pgfsys@useobject{currentmarker}{}%
\end{pgfscope}%
\begin{pgfscope}%
\pgfsys@transformshift{6.987447in}{3.278871in}%
\pgfsys@useobject{currentmarker}{}%
\end{pgfscope}%
\begin{pgfscope}%
\pgfsys@transformshift{7.033160in}{3.270605in}%
\pgfsys@useobject{currentmarker}{}%
\end{pgfscope}%
\begin{pgfscope}%
\pgfsys@transformshift{7.078700in}{3.260138in}%
\pgfsys@useobject{currentmarker}{}%
\end{pgfscope}%
\begin{pgfscope}%
\pgfsys@transformshift{7.124067in}{3.247480in}%
\pgfsys@useobject{currentmarker}{}%
\end{pgfscope}%
\begin{pgfscope}%
\pgfsys@transformshift{7.169263in}{3.232644in}%
\pgfsys@useobject{currentmarker}{}%
\end{pgfscope}%
\begin{pgfscope}%
\pgfsys@transformshift{7.214288in}{3.215640in}%
\pgfsys@useobject{currentmarker}{}%
\end{pgfscope}%
\begin{pgfscope}%
\pgfsys@transformshift{7.259143in}{3.196478in}%
\pgfsys@useobject{currentmarker}{}%
\end{pgfscope}%
\begin{pgfscope}%
\pgfsys@transformshift{7.303830in}{3.175170in}%
\pgfsys@useobject{currentmarker}{}%
\end{pgfscope}%
\begin{pgfscope}%
\pgfsys@transformshift{7.348348in}{3.151726in}%
\pgfsys@useobject{currentmarker}{}%
\end{pgfscope}%
\begin{pgfscope}%
\pgfsys@transformshift{7.392699in}{3.126157in}%
\pgfsys@useobject{currentmarker}{}%
\end{pgfscope}%
\begin{pgfscope}%
\pgfsys@transformshift{7.436884in}{3.098474in}%
\pgfsys@useobject{currentmarker}{}%
\end{pgfscope}%
\begin{pgfscope}%
\pgfsys@transformshift{7.480903in}{3.068687in}%
\pgfsys@useobject{currentmarker}{}%
\end{pgfscope}%
\begin{pgfscope}%
\pgfsys@transformshift{7.524757in}{3.036806in}%
\pgfsys@useobject{currentmarker}{}%
\end{pgfscope}%
\begin{pgfscope}%
\pgfsys@transformshift{7.568448in}{3.002843in}%
\pgfsys@useobject{currentmarker}{}%
\end{pgfscope}%
\begin{pgfscope}%
\pgfsys@transformshift{7.611975in}{2.966807in}%
\pgfsys@useobject{currentmarker}{}%
\end{pgfscope}%
\begin{pgfscope}%
\pgfsys@transformshift{7.655340in}{2.928709in}%
\pgfsys@useobject{currentmarker}{}%
\end{pgfscope}%
\begin{pgfscope}%
\pgfsys@transformshift{7.698544in}{2.888559in}%
\pgfsys@useobject{currentmarker}{}%
\end{pgfscope}%
\begin{pgfscope}%
\pgfsys@transformshift{7.741587in}{2.846368in}%
\pgfsys@useobject{currentmarker}{}%
\end{pgfscope}%
\begin{pgfscope}%
\pgfsys@transformshift{7.784470in}{2.802145in}%
\pgfsys@useobject{currentmarker}{}%
\end{pgfscope}%
\begin{pgfscope}%
\pgfsys@transformshift{7.827194in}{2.755901in}%
\pgfsys@useobject{currentmarker}{}%
\end{pgfscope}%
\begin{pgfscope}%
\pgfsys@transformshift{7.869760in}{2.707646in}%
\pgfsys@useobject{currentmarker}{}%
\end{pgfscope}%
\begin{pgfscope}%
\pgfsys@transformshift{7.912168in}{2.657390in}%
\pgfsys@useobject{currentmarker}{}%
\end{pgfscope}%
\begin{pgfscope}%
\pgfsys@transformshift{7.954420in}{2.605143in}%
\pgfsys@useobject{currentmarker}{}%
\end{pgfscope}%
\begin{pgfscope}%
\pgfsys@transformshift{7.996516in}{2.550915in}%
\pgfsys@useobject{currentmarker}{}%
\end{pgfscope}%
\begin{pgfscope}%
\pgfsys@transformshift{8.038456in}{2.494716in}%
\pgfsys@useobject{currentmarker}{}%
\end{pgfscope}%
\begin{pgfscope}%
\pgfsys@transformshift{8.080243in}{2.436556in}%
\pgfsys@useobject{currentmarker}{}%
\end{pgfscope}%
\begin{pgfscope}%
\pgfsys@transformshift{8.121875in}{2.376444in}%
\pgfsys@useobject{currentmarker}{}%
\end{pgfscope}%
\begin{pgfscope}%
\pgfsys@transformshift{8.163355in}{2.314390in}%
\pgfsys@useobject{currentmarker}{}%
\end{pgfscope}%
\begin{pgfscope}%
\pgfsys@transformshift{8.204683in}{2.250404in}%
\pgfsys@useobject{currentmarker}{}%
\end{pgfscope}%
\begin{pgfscope}%
\pgfsys@transformshift{8.245860in}{2.184496in}%
\pgfsys@useobject{currentmarker}{}%
\end{pgfscope}%
\begin{pgfscope}%
\pgfsys@transformshift{8.286886in}{2.116676in}%
\pgfsys@useobject{currentmarker}{}%
\end{pgfscope}%
\begin{pgfscope}%
\pgfsys@transformshift{8.327762in}{2.046952in}%
\pgfsys@useobject{currentmarker}{}%
\end{pgfscope}%
\begin{pgfscope}%
\pgfsys@transformshift{8.368489in}{1.975334in}%
\pgfsys@useobject{currentmarker}{}%
\end{pgfscope}%
\begin{pgfscope}%
\pgfsys@transformshift{8.409068in}{1.901832in}%
\pgfsys@useobject{currentmarker}{}%
\end{pgfscope}%
\begin{pgfscope}%
\pgfsys@transformshift{8.449500in}{1.826456in}%
\pgfsys@useobject{currentmarker}{}%
\end{pgfscope}%
\begin{pgfscope}%
\pgfsys@transformshift{8.489784in}{1.749214in}%
\pgfsys@useobject{currentmarker}{}%
\end{pgfscope}%
\begin{pgfscope}%
\pgfsys@transformshift{8.529923in}{1.670116in}%
\pgfsys@useobject{currentmarker}{}%
\end{pgfscope}%
\begin{pgfscope}%
\pgfsys@transformshift{8.569916in}{1.589171in}%
\pgfsys@useobject{currentmarker}{}%
\end{pgfscope}%
\begin{pgfscope}%
\pgfsys@transformshift{8.609765in}{1.506388in}%
\pgfsys@useobject{currentmarker}{}%
\end{pgfscope}%
\begin{pgfscope}%
\pgfsys@transformshift{8.649470in}{1.421777in}%
\pgfsys@useobject{currentmarker}{}%
\end{pgfscope}%
\begin{pgfscope}%
\pgfsys@transformshift{8.689031in}{1.335347in}%
\pgfsys@useobject{currentmarker}{}%
\end{pgfscope}%
\begin{pgfscope}%
\pgfsys@transformshift{8.728450in}{1.247107in}%
\pgfsys@useobject{currentmarker}{}%
\end{pgfscope}%
\begin{pgfscope}%
\pgfsys@transformshift{8.767728in}{1.157066in}%
\pgfsys@useobject{currentmarker}{}%
\end{pgfscope}%
\begin{pgfscope}%
\pgfsys@transformshift{8.806864in}{1.065233in}%
\pgfsys@useobject{currentmarker}{}%
\end{pgfscope}%
\begin{pgfscope}%
\pgfsys@transformshift{8.845860in}{0.971616in}%
\pgfsys@useobject{currentmarker}{}%
\end{pgfscope}%
\begin{pgfscope}%
\pgfsys@transformshift{8.884717in}{0.876226in}%
\pgfsys@useobject{currentmarker}{}%
\end{pgfscope}%
\begin{pgfscope}%
\pgfsys@transformshift{8.923434in}{0.779070in}%
\pgfsys@useobject{currentmarker}{}%
\end{pgfscope}%
\end{pgfscope}%
\begin{pgfscope}%
\pgfpathrectangle{\pgfqpoint{1.250000in}{0.400000in}}{\pgfqpoint{7.750000in}{3.200000in}} %
\pgfusepath{clip}%
\pgfsetbuttcap%
\pgfsetroundjoin%
\pgfsetlinewidth{0.501875pt}%
\definecolor{currentstroke}{rgb}{0.000000,0.000000,0.000000}%
\pgfsetstrokecolor{currentstroke}%
\pgfsetdash{{1.000000pt}{3.000000pt}}{0.000000pt}%
\pgfpathmoveto{\pgfqpoint{1.250000in}{0.400000in}}%
\pgfpathlineto{\pgfqpoint{1.250000in}{3.600000in}}%
\pgfusepath{stroke}%
\end{pgfscope}%
\begin{pgfscope}%
\pgfsetbuttcap%
\pgfsetroundjoin%
\definecolor{currentfill}{rgb}{0.000000,0.000000,0.000000}%
\pgfsetfillcolor{currentfill}%
\pgfsetlinewidth{0.501875pt}%
\definecolor{currentstroke}{rgb}{0.000000,0.000000,0.000000}%
\pgfsetstrokecolor{currentstroke}%
\pgfsetdash{}{0pt}%
\pgfsys@defobject{currentmarker}{\pgfqpoint{0.000000in}{0.000000in}}{\pgfqpoint{0.000000in}{0.055556in}}{%
\pgfpathmoveto{\pgfqpoint{0.000000in}{0.000000in}}%
\pgfpathlineto{\pgfqpoint{0.000000in}{0.055556in}}%
\pgfusepath{stroke,fill}%
}%
\begin{pgfscope}%
\pgfsys@transformshift{1.250000in}{0.400000in}%
\pgfsys@useobject{currentmarker}{}%
\end{pgfscope}%
\end{pgfscope}%
\begin{pgfscope}%
\pgfsetbuttcap%
\pgfsetroundjoin%
\definecolor{currentfill}{rgb}{0.000000,0.000000,0.000000}%
\pgfsetfillcolor{currentfill}%
\pgfsetlinewidth{0.501875pt}%
\definecolor{currentstroke}{rgb}{0.000000,0.000000,0.000000}%
\pgfsetstrokecolor{currentstroke}%
\pgfsetdash{}{0pt}%
\pgfsys@defobject{currentmarker}{\pgfqpoint{0.000000in}{-0.055556in}}{\pgfqpoint{0.000000in}{0.000000in}}{%
\pgfpathmoveto{\pgfqpoint{0.000000in}{0.000000in}}%
\pgfpathlineto{\pgfqpoint{0.000000in}{-0.055556in}}%
\pgfusepath{stroke,fill}%
}%
\begin{pgfscope}%
\pgfsys@transformshift{1.250000in}{3.600000in}%
\pgfsys@useobject{currentmarker}{}%
\end{pgfscope}%
\end{pgfscope}%
\begin{pgfscope}%
\pgftext[left,bottom,x=1.021118in,y=0.218387in,rotate=0.000000]{{\sffamily\fontsize{12.000000}{14.400000}\selectfont −300}}
%
\end{pgfscope}%
\begin{pgfscope}%
\pgfpathrectangle{\pgfqpoint{1.250000in}{0.400000in}}{\pgfqpoint{7.750000in}{3.200000in}} %
\pgfusepath{clip}%
\pgfsetbuttcap%
\pgfsetroundjoin%
\pgfsetlinewidth{0.501875pt}%
\definecolor{currentstroke}{rgb}{0.000000,0.000000,0.000000}%
\pgfsetstrokecolor{currentstroke}%
\pgfsetdash{{1.000000pt}{3.000000pt}}{0.000000pt}%
\pgfpathmoveto{\pgfqpoint{2.357143in}{0.400000in}}%
\pgfpathlineto{\pgfqpoint{2.357143in}{3.600000in}}%
\pgfusepath{stroke}%
\end{pgfscope}%
\begin{pgfscope}%
\pgfsetbuttcap%
\pgfsetroundjoin%
\definecolor{currentfill}{rgb}{0.000000,0.000000,0.000000}%
\pgfsetfillcolor{currentfill}%
\pgfsetlinewidth{0.501875pt}%
\definecolor{currentstroke}{rgb}{0.000000,0.000000,0.000000}%
\pgfsetstrokecolor{currentstroke}%
\pgfsetdash{}{0pt}%
\pgfsys@defobject{currentmarker}{\pgfqpoint{0.000000in}{0.000000in}}{\pgfqpoint{0.000000in}{0.055556in}}{%
\pgfpathmoveto{\pgfqpoint{0.000000in}{0.000000in}}%
\pgfpathlineto{\pgfqpoint{0.000000in}{0.055556in}}%
\pgfusepath{stroke,fill}%
}%
\begin{pgfscope}%
\pgfsys@transformshift{2.357143in}{0.400000in}%
\pgfsys@useobject{currentmarker}{}%
\end{pgfscope}%
\end{pgfscope}%
\begin{pgfscope}%
\pgfsetbuttcap%
\pgfsetroundjoin%
\definecolor{currentfill}{rgb}{0.000000,0.000000,0.000000}%
\pgfsetfillcolor{currentfill}%
\pgfsetlinewidth{0.501875pt}%
\definecolor{currentstroke}{rgb}{0.000000,0.000000,0.000000}%
\pgfsetstrokecolor{currentstroke}%
\pgfsetdash{}{0pt}%
\pgfsys@defobject{currentmarker}{\pgfqpoint{0.000000in}{-0.055556in}}{\pgfqpoint{0.000000in}{0.000000in}}{%
\pgfpathmoveto{\pgfqpoint{0.000000in}{0.000000in}}%
\pgfpathlineto{\pgfqpoint{0.000000in}{-0.055556in}}%
\pgfusepath{stroke,fill}%
}%
\begin{pgfscope}%
\pgfsys@transformshift{2.357143in}{3.600000in}%
\pgfsys@useobject{currentmarker}{}%
\end{pgfscope}%
\end{pgfscope}%
\begin{pgfscope}%
\pgftext[left,bottom,x=2.128261in,y=0.218387in,rotate=0.000000]{{\sffamily\fontsize{12.000000}{14.400000}\selectfont −200}}
%
\end{pgfscope}%
\begin{pgfscope}%
\pgfpathrectangle{\pgfqpoint{1.250000in}{0.400000in}}{\pgfqpoint{7.750000in}{3.200000in}} %
\pgfusepath{clip}%
\pgfsetbuttcap%
\pgfsetroundjoin%
\pgfsetlinewidth{0.501875pt}%
\definecolor{currentstroke}{rgb}{0.000000,0.000000,0.000000}%
\pgfsetstrokecolor{currentstroke}%
\pgfsetdash{{1.000000pt}{3.000000pt}}{0.000000pt}%
\pgfpathmoveto{\pgfqpoint{3.464286in}{0.400000in}}%
\pgfpathlineto{\pgfqpoint{3.464286in}{3.600000in}}%
\pgfusepath{stroke}%
\end{pgfscope}%
\begin{pgfscope}%
\pgfsetbuttcap%
\pgfsetroundjoin%
\definecolor{currentfill}{rgb}{0.000000,0.000000,0.000000}%
\pgfsetfillcolor{currentfill}%
\pgfsetlinewidth{0.501875pt}%
\definecolor{currentstroke}{rgb}{0.000000,0.000000,0.000000}%
\pgfsetstrokecolor{currentstroke}%
\pgfsetdash{}{0pt}%
\pgfsys@defobject{currentmarker}{\pgfqpoint{0.000000in}{0.000000in}}{\pgfqpoint{0.000000in}{0.055556in}}{%
\pgfpathmoveto{\pgfqpoint{0.000000in}{0.000000in}}%
\pgfpathlineto{\pgfqpoint{0.000000in}{0.055556in}}%
\pgfusepath{stroke,fill}%
}%
\begin{pgfscope}%
\pgfsys@transformshift{3.464286in}{0.400000in}%
\pgfsys@useobject{currentmarker}{}%
\end{pgfscope}%
\end{pgfscope}%
\begin{pgfscope}%
\pgfsetbuttcap%
\pgfsetroundjoin%
\definecolor{currentfill}{rgb}{0.000000,0.000000,0.000000}%
\pgfsetfillcolor{currentfill}%
\pgfsetlinewidth{0.501875pt}%
\definecolor{currentstroke}{rgb}{0.000000,0.000000,0.000000}%
\pgfsetstrokecolor{currentstroke}%
\pgfsetdash{}{0pt}%
\pgfsys@defobject{currentmarker}{\pgfqpoint{0.000000in}{-0.055556in}}{\pgfqpoint{0.000000in}{0.000000in}}{%
\pgfpathmoveto{\pgfqpoint{0.000000in}{0.000000in}}%
\pgfpathlineto{\pgfqpoint{0.000000in}{-0.055556in}}%
\pgfusepath{stroke,fill}%
}%
\begin{pgfscope}%
\pgfsys@transformshift{3.464286in}{3.600000in}%
\pgfsys@useobject{currentmarker}{}%
\end{pgfscope}%
\end{pgfscope}%
\begin{pgfscope}%
\pgftext[left,bottom,x=3.235404in,y=0.218387in,rotate=0.000000]{{\sffamily\fontsize{12.000000}{14.400000}\selectfont −100}}
%
\end{pgfscope}%
\begin{pgfscope}%
\pgfpathrectangle{\pgfqpoint{1.250000in}{0.400000in}}{\pgfqpoint{7.750000in}{3.200000in}} %
\pgfusepath{clip}%
\pgfsetbuttcap%
\pgfsetroundjoin%
\pgfsetlinewidth{0.501875pt}%
\definecolor{currentstroke}{rgb}{0.000000,0.000000,0.000000}%
\pgfsetstrokecolor{currentstroke}%
\pgfsetdash{{1.000000pt}{3.000000pt}}{0.000000pt}%
\pgfpathmoveto{\pgfqpoint{4.571429in}{0.400000in}}%
\pgfpathlineto{\pgfqpoint{4.571429in}{3.600000in}}%
\pgfusepath{stroke}%
\end{pgfscope}%
\begin{pgfscope}%
\pgfsetbuttcap%
\pgfsetroundjoin%
\definecolor{currentfill}{rgb}{0.000000,0.000000,0.000000}%
\pgfsetfillcolor{currentfill}%
\pgfsetlinewidth{0.501875pt}%
\definecolor{currentstroke}{rgb}{0.000000,0.000000,0.000000}%
\pgfsetstrokecolor{currentstroke}%
\pgfsetdash{}{0pt}%
\pgfsys@defobject{currentmarker}{\pgfqpoint{0.000000in}{0.000000in}}{\pgfqpoint{0.000000in}{0.055556in}}{%
\pgfpathmoveto{\pgfqpoint{0.000000in}{0.000000in}}%
\pgfpathlineto{\pgfqpoint{0.000000in}{0.055556in}}%
\pgfusepath{stroke,fill}%
}%
\begin{pgfscope}%
\pgfsys@transformshift{4.571429in}{0.400000in}%
\pgfsys@useobject{currentmarker}{}%
\end{pgfscope}%
\end{pgfscope}%
\begin{pgfscope}%
\pgfsetbuttcap%
\pgfsetroundjoin%
\definecolor{currentfill}{rgb}{0.000000,0.000000,0.000000}%
\pgfsetfillcolor{currentfill}%
\pgfsetlinewidth{0.501875pt}%
\definecolor{currentstroke}{rgb}{0.000000,0.000000,0.000000}%
\pgfsetstrokecolor{currentstroke}%
\pgfsetdash{}{0pt}%
\pgfsys@defobject{currentmarker}{\pgfqpoint{0.000000in}{-0.055556in}}{\pgfqpoint{0.000000in}{0.000000in}}{%
\pgfpathmoveto{\pgfqpoint{0.000000in}{0.000000in}}%
\pgfpathlineto{\pgfqpoint{0.000000in}{-0.055556in}}%
\pgfusepath{stroke,fill}%
}%
\begin{pgfscope}%
\pgfsys@transformshift{4.571429in}{3.600000in}%
\pgfsys@useobject{currentmarker}{}%
\end{pgfscope}%
\end{pgfscope}%
\begin{pgfscope}%
\pgftext[left,bottom,x=4.518409in,y=0.218387in,rotate=0.000000]{{\sffamily\fontsize{12.000000}{14.400000}\selectfont 0}}
%
\end{pgfscope}%
\begin{pgfscope}%
\pgfpathrectangle{\pgfqpoint{1.250000in}{0.400000in}}{\pgfqpoint{7.750000in}{3.200000in}} %
\pgfusepath{clip}%
\pgfsetbuttcap%
\pgfsetroundjoin%
\pgfsetlinewidth{0.501875pt}%
\definecolor{currentstroke}{rgb}{0.000000,0.000000,0.000000}%
\pgfsetstrokecolor{currentstroke}%
\pgfsetdash{{1.000000pt}{3.000000pt}}{0.000000pt}%
\pgfpathmoveto{\pgfqpoint{5.678571in}{0.400000in}}%
\pgfpathlineto{\pgfqpoint{5.678571in}{3.600000in}}%
\pgfusepath{stroke}%
\end{pgfscope}%
\begin{pgfscope}%
\pgfsetbuttcap%
\pgfsetroundjoin%
\definecolor{currentfill}{rgb}{0.000000,0.000000,0.000000}%
\pgfsetfillcolor{currentfill}%
\pgfsetlinewidth{0.501875pt}%
\definecolor{currentstroke}{rgb}{0.000000,0.000000,0.000000}%
\pgfsetstrokecolor{currentstroke}%
\pgfsetdash{}{0pt}%
\pgfsys@defobject{currentmarker}{\pgfqpoint{0.000000in}{0.000000in}}{\pgfqpoint{0.000000in}{0.055556in}}{%
\pgfpathmoveto{\pgfqpoint{0.000000in}{0.000000in}}%
\pgfpathlineto{\pgfqpoint{0.000000in}{0.055556in}}%
\pgfusepath{stroke,fill}%
}%
\begin{pgfscope}%
\pgfsys@transformshift{5.678571in}{0.400000in}%
\pgfsys@useobject{currentmarker}{}%
\end{pgfscope}%
\end{pgfscope}%
\begin{pgfscope}%
\pgfsetbuttcap%
\pgfsetroundjoin%
\definecolor{currentfill}{rgb}{0.000000,0.000000,0.000000}%
\pgfsetfillcolor{currentfill}%
\pgfsetlinewidth{0.501875pt}%
\definecolor{currentstroke}{rgb}{0.000000,0.000000,0.000000}%
\pgfsetstrokecolor{currentstroke}%
\pgfsetdash{}{0pt}%
\pgfsys@defobject{currentmarker}{\pgfqpoint{0.000000in}{-0.055556in}}{\pgfqpoint{0.000000in}{0.000000in}}{%
\pgfpathmoveto{\pgfqpoint{0.000000in}{0.000000in}}%
\pgfpathlineto{\pgfqpoint{0.000000in}{-0.055556in}}%
\pgfusepath{stroke,fill}%
}%
\begin{pgfscope}%
\pgfsys@transformshift{5.678571in}{3.600000in}%
\pgfsys@useobject{currentmarker}{}%
\end{pgfscope}%
\end{pgfscope}%
\begin{pgfscope}%
\pgftext[left,bottom,x=5.519514in,y=0.218387in,rotate=0.000000]{{\sffamily\fontsize{12.000000}{14.400000}\selectfont 100}}
%
\end{pgfscope}%
\begin{pgfscope}%
\pgfpathrectangle{\pgfqpoint{1.250000in}{0.400000in}}{\pgfqpoint{7.750000in}{3.200000in}} %
\pgfusepath{clip}%
\pgfsetbuttcap%
\pgfsetroundjoin%
\pgfsetlinewidth{0.501875pt}%
\definecolor{currentstroke}{rgb}{0.000000,0.000000,0.000000}%
\pgfsetstrokecolor{currentstroke}%
\pgfsetdash{{1.000000pt}{3.000000pt}}{0.000000pt}%
\pgfpathmoveto{\pgfqpoint{6.785714in}{0.400000in}}%
\pgfpathlineto{\pgfqpoint{6.785714in}{3.600000in}}%
\pgfusepath{stroke}%
\end{pgfscope}%
\begin{pgfscope}%
\pgfsetbuttcap%
\pgfsetroundjoin%
\definecolor{currentfill}{rgb}{0.000000,0.000000,0.000000}%
\pgfsetfillcolor{currentfill}%
\pgfsetlinewidth{0.501875pt}%
\definecolor{currentstroke}{rgb}{0.000000,0.000000,0.000000}%
\pgfsetstrokecolor{currentstroke}%
\pgfsetdash{}{0pt}%
\pgfsys@defobject{currentmarker}{\pgfqpoint{0.000000in}{0.000000in}}{\pgfqpoint{0.000000in}{0.055556in}}{%
\pgfpathmoveto{\pgfqpoint{0.000000in}{0.000000in}}%
\pgfpathlineto{\pgfqpoint{0.000000in}{0.055556in}}%
\pgfusepath{stroke,fill}%
}%
\begin{pgfscope}%
\pgfsys@transformshift{6.785714in}{0.400000in}%
\pgfsys@useobject{currentmarker}{}%
\end{pgfscope}%
\end{pgfscope}%
\begin{pgfscope}%
\pgfsetbuttcap%
\pgfsetroundjoin%
\definecolor{currentfill}{rgb}{0.000000,0.000000,0.000000}%
\pgfsetfillcolor{currentfill}%
\pgfsetlinewidth{0.501875pt}%
\definecolor{currentstroke}{rgb}{0.000000,0.000000,0.000000}%
\pgfsetstrokecolor{currentstroke}%
\pgfsetdash{}{0pt}%
\pgfsys@defobject{currentmarker}{\pgfqpoint{0.000000in}{-0.055556in}}{\pgfqpoint{0.000000in}{0.000000in}}{%
\pgfpathmoveto{\pgfqpoint{0.000000in}{0.000000in}}%
\pgfpathlineto{\pgfqpoint{0.000000in}{-0.055556in}}%
\pgfusepath{stroke,fill}%
}%
\begin{pgfscope}%
\pgfsys@transformshift{6.785714in}{3.600000in}%
\pgfsys@useobject{currentmarker}{}%
\end{pgfscope}%
\end{pgfscope}%
\begin{pgfscope}%
\pgftext[left,bottom,x=6.626657in,y=0.218387in,rotate=0.000000]{{\sffamily\fontsize{12.000000}{14.400000}\selectfont 200}}
%
\end{pgfscope}%
\begin{pgfscope}%
\pgfpathrectangle{\pgfqpoint{1.250000in}{0.400000in}}{\pgfqpoint{7.750000in}{3.200000in}} %
\pgfusepath{clip}%
\pgfsetbuttcap%
\pgfsetroundjoin%
\pgfsetlinewidth{0.501875pt}%
\definecolor{currentstroke}{rgb}{0.000000,0.000000,0.000000}%
\pgfsetstrokecolor{currentstroke}%
\pgfsetdash{{1.000000pt}{3.000000pt}}{0.000000pt}%
\pgfpathmoveto{\pgfqpoint{7.892857in}{0.400000in}}%
\pgfpathlineto{\pgfqpoint{7.892857in}{3.600000in}}%
\pgfusepath{stroke}%
\end{pgfscope}%
\begin{pgfscope}%
\pgfsetbuttcap%
\pgfsetroundjoin%
\definecolor{currentfill}{rgb}{0.000000,0.000000,0.000000}%
\pgfsetfillcolor{currentfill}%
\pgfsetlinewidth{0.501875pt}%
\definecolor{currentstroke}{rgb}{0.000000,0.000000,0.000000}%
\pgfsetstrokecolor{currentstroke}%
\pgfsetdash{}{0pt}%
\pgfsys@defobject{currentmarker}{\pgfqpoint{0.000000in}{0.000000in}}{\pgfqpoint{0.000000in}{0.055556in}}{%
\pgfpathmoveto{\pgfqpoint{0.000000in}{0.000000in}}%
\pgfpathlineto{\pgfqpoint{0.000000in}{0.055556in}}%
\pgfusepath{stroke,fill}%
}%
\begin{pgfscope}%
\pgfsys@transformshift{7.892857in}{0.400000in}%
\pgfsys@useobject{currentmarker}{}%
\end{pgfscope}%
\end{pgfscope}%
\begin{pgfscope}%
\pgfsetbuttcap%
\pgfsetroundjoin%
\definecolor{currentfill}{rgb}{0.000000,0.000000,0.000000}%
\pgfsetfillcolor{currentfill}%
\pgfsetlinewidth{0.501875pt}%
\definecolor{currentstroke}{rgb}{0.000000,0.000000,0.000000}%
\pgfsetstrokecolor{currentstroke}%
\pgfsetdash{}{0pt}%
\pgfsys@defobject{currentmarker}{\pgfqpoint{0.000000in}{-0.055556in}}{\pgfqpoint{0.000000in}{0.000000in}}{%
\pgfpathmoveto{\pgfqpoint{0.000000in}{0.000000in}}%
\pgfpathlineto{\pgfqpoint{0.000000in}{-0.055556in}}%
\pgfusepath{stroke,fill}%
}%
\begin{pgfscope}%
\pgfsys@transformshift{7.892857in}{3.600000in}%
\pgfsys@useobject{currentmarker}{}%
\end{pgfscope}%
\end{pgfscope}%
\begin{pgfscope}%
\pgftext[left,bottom,x=7.733800in,y=0.218387in,rotate=0.000000]{{\sffamily\fontsize{12.000000}{14.400000}\selectfont 300}}
%
\end{pgfscope}%
\begin{pgfscope}%
\pgfpathrectangle{\pgfqpoint{1.250000in}{0.400000in}}{\pgfqpoint{7.750000in}{3.200000in}} %
\pgfusepath{clip}%
\pgfsetbuttcap%
\pgfsetroundjoin%
\pgfsetlinewidth{0.501875pt}%
\definecolor{currentstroke}{rgb}{0.000000,0.000000,0.000000}%
\pgfsetstrokecolor{currentstroke}%
\pgfsetdash{{1.000000pt}{3.000000pt}}{0.000000pt}%
\pgfpathmoveto{\pgfqpoint{9.000000in}{0.400000in}}%
\pgfpathlineto{\pgfqpoint{9.000000in}{3.600000in}}%
\pgfusepath{stroke}%
\end{pgfscope}%
\begin{pgfscope}%
\pgfsetbuttcap%
\pgfsetroundjoin%
\definecolor{currentfill}{rgb}{0.000000,0.000000,0.000000}%
\pgfsetfillcolor{currentfill}%
\pgfsetlinewidth{0.501875pt}%
\definecolor{currentstroke}{rgb}{0.000000,0.000000,0.000000}%
\pgfsetstrokecolor{currentstroke}%
\pgfsetdash{}{0pt}%
\pgfsys@defobject{currentmarker}{\pgfqpoint{0.000000in}{0.000000in}}{\pgfqpoint{0.000000in}{0.055556in}}{%
\pgfpathmoveto{\pgfqpoint{0.000000in}{0.000000in}}%
\pgfpathlineto{\pgfqpoint{0.000000in}{0.055556in}}%
\pgfusepath{stroke,fill}%
}%
\begin{pgfscope}%
\pgfsys@transformshift{9.000000in}{0.400000in}%
\pgfsys@useobject{currentmarker}{}%
\end{pgfscope}%
\end{pgfscope}%
\begin{pgfscope}%
\pgfsetbuttcap%
\pgfsetroundjoin%
\definecolor{currentfill}{rgb}{0.000000,0.000000,0.000000}%
\pgfsetfillcolor{currentfill}%
\pgfsetlinewidth{0.501875pt}%
\definecolor{currentstroke}{rgb}{0.000000,0.000000,0.000000}%
\pgfsetstrokecolor{currentstroke}%
\pgfsetdash{}{0pt}%
\pgfsys@defobject{currentmarker}{\pgfqpoint{0.000000in}{-0.055556in}}{\pgfqpoint{0.000000in}{0.000000in}}{%
\pgfpathmoveto{\pgfqpoint{0.000000in}{0.000000in}}%
\pgfpathlineto{\pgfqpoint{0.000000in}{-0.055556in}}%
\pgfusepath{stroke,fill}%
}%
\begin{pgfscope}%
\pgfsys@transformshift{9.000000in}{3.600000in}%
\pgfsys@useobject{currentmarker}{}%
\end{pgfscope}%
\end{pgfscope}%
\begin{pgfscope}%
\pgftext[left,bottom,x=8.840942in,y=0.218387in,rotate=0.000000]{{\sffamily\fontsize{12.000000}{14.400000}\selectfont 400}}
%
\end{pgfscope}%
\begin{pgfscope}%
\pgftext[left,bottom,x=4.865560in,y=0.022315in,rotate=0.000000]{{\sffamily\fontsize{12.000000}{14.400000}\selectfont x in m}}
%
\end{pgfscope}%
\begin{pgfscope}%
\pgfpathrectangle{\pgfqpoint{1.250000in}{0.400000in}}{\pgfqpoint{7.750000in}{3.200000in}} %
\pgfusepath{clip}%
\pgfsetbuttcap%
\pgfsetroundjoin%
\pgfsetlinewidth{0.501875pt}%
\definecolor{currentstroke}{rgb}{0.000000,0.000000,0.000000}%
\pgfsetstrokecolor{currentstroke}%
\pgfsetdash{{1.000000pt}{3.000000pt}}{0.000000pt}%
\pgfpathmoveto{\pgfqpoint{1.250000in}{0.400000in}}%
\pgfpathlineto{\pgfqpoint{9.000000in}{0.400000in}}%
\pgfusepath{stroke}%
\end{pgfscope}%
\begin{pgfscope}%
\pgfsetbuttcap%
\pgfsetroundjoin%
\definecolor{currentfill}{rgb}{0.000000,0.000000,0.000000}%
\pgfsetfillcolor{currentfill}%
\pgfsetlinewidth{0.501875pt}%
\definecolor{currentstroke}{rgb}{0.000000,0.000000,0.000000}%
\pgfsetstrokecolor{currentstroke}%
\pgfsetdash{}{0pt}%
\pgfsys@defobject{currentmarker}{\pgfqpoint{0.000000in}{0.000000in}}{\pgfqpoint{0.055556in}{0.000000in}}{%
\pgfpathmoveto{\pgfqpoint{0.000000in}{0.000000in}}%
\pgfpathlineto{\pgfqpoint{0.055556in}{0.000000in}}%
\pgfusepath{stroke,fill}%
}%
\begin{pgfscope}%
\pgfsys@transformshift{1.250000in}{0.400000in}%
\pgfsys@useobject{currentmarker}{}%
\end{pgfscope}%
\end{pgfscope}%
\begin{pgfscope}%
\pgfsetbuttcap%
\pgfsetroundjoin%
\definecolor{currentfill}{rgb}{0.000000,0.000000,0.000000}%
\pgfsetfillcolor{currentfill}%
\pgfsetlinewidth{0.501875pt}%
\definecolor{currentstroke}{rgb}{0.000000,0.000000,0.000000}%
\pgfsetstrokecolor{currentstroke}%
\pgfsetdash{}{0pt}%
\pgfsys@defobject{currentmarker}{\pgfqpoint{-0.055556in}{0.000000in}}{\pgfqpoint{0.000000in}{0.000000in}}{%
\pgfpathmoveto{\pgfqpoint{0.000000in}{0.000000in}}%
\pgfpathlineto{\pgfqpoint{-0.055556in}{0.000000in}}%
\pgfusepath{stroke,fill}%
}%
\begin{pgfscope}%
\pgfsys@transformshift{9.000000in}{0.400000in}%
\pgfsys@useobject{currentmarker}{}%
\end{pgfscope}%
\end{pgfscope}%
\begin{pgfscope}%
\pgftext[left,bottom,x=0.842719in,y=0.336971in,rotate=0.000000]{{\sffamily\fontsize{12.000000}{14.400000}\selectfont −20}}
%
\end{pgfscope}%
\begin{pgfscope}%
\pgfpathrectangle{\pgfqpoint{1.250000in}{0.400000in}}{\pgfqpoint{7.750000in}{3.200000in}} %
\pgfusepath{clip}%
\pgfsetbuttcap%
\pgfsetroundjoin%
\pgfsetlinewidth{0.501875pt}%
\definecolor{currentstroke}{rgb}{0.000000,0.000000,0.000000}%
\pgfsetstrokecolor{currentstroke}%
\pgfsetdash{{1.000000pt}{3.000000pt}}{0.000000pt}%
\pgfpathmoveto{\pgfqpoint{1.250000in}{0.857143in}}%
\pgfpathlineto{\pgfqpoint{9.000000in}{0.857143in}}%
\pgfusepath{stroke}%
\end{pgfscope}%
\begin{pgfscope}%
\pgfsetbuttcap%
\pgfsetroundjoin%
\definecolor{currentfill}{rgb}{0.000000,0.000000,0.000000}%
\pgfsetfillcolor{currentfill}%
\pgfsetlinewidth{0.501875pt}%
\definecolor{currentstroke}{rgb}{0.000000,0.000000,0.000000}%
\pgfsetstrokecolor{currentstroke}%
\pgfsetdash{}{0pt}%
\pgfsys@defobject{currentmarker}{\pgfqpoint{0.000000in}{0.000000in}}{\pgfqpoint{0.055556in}{0.000000in}}{%
\pgfpathmoveto{\pgfqpoint{0.000000in}{0.000000in}}%
\pgfpathlineto{\pgfqpoint{0.055556in}{0.000000in}}%
\pgfusepath{stroke,fill}%
}%
\begin{pgfscope}%
\pgfsys@transformshift{1.250000in}{0.857143in}%
\pgfsys@useobject{currentmarker}{}%
\end{pgfscope}%
\end{pgfscope}%
\begin{pgfscope}%
\pgfsetbuttcap%
\pgfsetroundjoin%
\definecolor{currentfill}{rgb}{0.000000,0.000000,0.000000}%
\pgfsetfillcolor{currentfill}%
\pgfsetlinewidth{0.501875pt}%
\definecolor{currentstroke}{rgb}{0.000000,0.000000,0.000000}%
\pgfsetstrokecolor{currentstroke}%
\pgfsetdash{}{0pt}%
\pgfsys@defobject{currentmarker}{\pgfqpoint{-0.055556in}{0.000000in}}{\pgfqpoint{0.000000in}{0.000000in}}{%
\pgfpathmoveto{\pgfqpoint{0.000000in}{0.000000in}}%
\pgfpathlineto{\pgfqpoint{-0.055556in}{0.000000in}}%
\pgfusepath{stroke,fill}%
}%
\begin{pgfscope}%
\pgfsys@transformshift{9.000000in}{0.857143in}%
\pgfsys@useobject{currentmarker}{}%
\end{pgfscope}%
\end{pgfscope}%
\begin{pgfscope}%
\pgftext[left,bottom,x=1.088406in,y=0.794114in,rotate=0.000000]{{\sffamily\fontsize{12.000000}{14.400000}\selectfont 0}}
%
\end{pgfscope}%
\begin{pgfscope}%
\pgfpathrectangle{\pgfqpoint{1.250000in}{0.400000in}}{\pgfqpoint{7.750000in}{3.200000in}} %
\pgfusepath{clip}%
\pgfsetbuttcap%
\pgfsetroundjoin%
\pgfsetlinewidth{0.501875pt}%
\definecolor{currentstroke}{rgb}{0.000000,0.000000,0.000000}%
\pgfsetstrokecolor{currentstroke}%
\pgfsetdash{{1.000000pt}{3.000000pt}}{0.000000pt}%
\pgfpathmoveto{\pgfqpoint{1.250000in}{1.314286in}}%
\pgfpathlineto{\pgfqpoint{9.000000in}{1.314286in}}%
\pgfusepath{stroke}%
\end{pgfscope}%
\begin{pgfscope}%
\pgfsetbuttcap%
\pgfsetroundjoin%
\definecolor{currentfill}{rgb}{0.000000,0.000000,0.000000}%
\pgfsetfillcolor{currentfill}%
\pgfsetlinewidth{0.501875pt}%
\definecolor{currentstroke}{rgb}{0.000000,0.000000,0.000000}%
\pgfsetstrokecolor{currentstroke}%
\pgfsetdash{}{0pt}%
\pgfsys@defobject{currentmarker}{\pgfqpoint{0.000000in}{0.000000in}}{\pgfqpoint{0.055556in}{0.000000in}}{%
\pgfpathmoveto{\pgfqpoint{0.000000in}{0.000000in}}%
\pgfpathlineto{\pgfqpoint{0.055556in}{0.000000in}}%
\pgfusepath{stroke,fill}%
}%
\begin{pgfscope}%
\pgfsys@transformshift{1.250000in}{1.314286in}%
\pgfsys@useobject{currentmarker}{}%
\end{pgfscope}%
\end{pgfscope}%
\begin{pgfscope}%
\pgfsetbuttcap%
\pgfsetroundjoin%
\definecolor{currentfill}{rgb}{0.000000,0.000000,0.000000}%
\pgfsetfillcolor{currentfill}%
\pgfsetlinewidth{0.501875pt}%
\definecolor{currentstroke}{rgb}{0.000000,0.000000,0.000000}%
\pgfsetstrokecolor{currentstroke}%
\pgfsetdash{}{0pt}%
\pgfsys@defobject{currentmarker}{\pgfqpoint{-0.055556in}{0.000000in}}{\pgfqpoint{0.000000in}{0.000000in}}{%
\pgfpathmoveto{\pgfqpoint{0.000000in}{0.000000in}}%
\pgfpathlineto{\pgfqpoint{-0.055556in}{0.000000in}}%
\pgfusepath{stroke,fill}%
}%
\begin{pgfscope}%
\pgfsys@transformshift{9.000000in}{1.314286in}%
\pgfsys@useobject{currentmarker}{}%
\end{pgfscope}%
\end{pgfscope}%
\begin{pgfscope}%
\pgftext[left,bottom,x=0.982368in,y=1.251257in,rotate=0.000000]{{\sffamily\fontsize{12.000000}{14.400000}\selectfont 20}}
%
\end{pgfscope}%
\begin{pgfscope}%
\pgfpathrectangle{\pgfqpoint{1.250000in}{0.400000in}}{\pgfqpoint{7.750000in}{3.200000in}} %
\pgfusepath{clip}%
\pgfsetbuttcap%
\pgfsetroundjoin%
\pgfsetlinewidth{0.501875pt}%
\definecolor{currentstroke}{rgb}{0.000000,0.000000,0.000000}%
\pgfsetstrokecolor{currentstroke}%
\pgfsetdash{{1.000000pt}{3.000000pt}}{0.000000pt}%
\pgfpathmoveto{\pgfqpoint{1.250000in}{1.771429in}}%
\pgfpathlineto{\pgfqpoint{9.000000in}{1.771429in}}%
\pgfusepath{stroke}%
\end{pgfscope}%
\begin{pgfscope}%
\pgfsetbuttcap%
\pgfsetroundjoin%
\definecolor{currentfill}{rgb}{0.000000,0.000000,0.000000}%
\pgfsetfillcolor{currentfill}%
\pgfsetlinewidth{0.501875pt}%
\definecolor{currentstroke}{rgb}{0.000000,0.000000,0.000000}%
\pgfsetstrokecolor{currentstroke}%
\pgfsetdash{}{0pt}%
\pgfsys@defobject{currentmarker}{\pgfqpoint{0.000000in}{0.000000in}}{\pgfqpoint{0.055556in}{0.000000in}}{%
\pgfpathmoveto{\pgfqpoint{0.000000in}{0.000000in}}%
\pgfpathlineto{\pgfqpoint{0.055556in}{0.000000in}}%
\pgfusepath{stroke,fill}%
}%
\begin{pgfscope}%
\pgfsys@transformshift{1.250000in}{1.771429in}%
\pgfsys@useobject{currentmarker}{}%
\end{pgfscope}%
\end{pgfscope}%
\begin{pgfscope}%
\pgfsetbuttcap%
\pgfsetroundjoin%
\definecolor{currentfill}{rgb}{0.000000,0.000000,0.000000}%
\pgfsetfillcolor{currentfill}%
\pgfsetlinewidth{0.501875pt}%
\definecolor{currentstroke}{rgb}{0.000000,0.000000,0.000000}%
\pgfsetstrokecolor{currentstroke}%
\pgfsetdash{}{0pt}%
\pgfsys@defobject{currentmarker}{\pgfqpoint{-0.055556in}{0.000000in}}{\pgfqpoint{0.000000in}{0.000000in}}{%
\pgfpathmoveto{\pgfqpoint{0.000000in}{0.000000in}}%
\pgfpathlineto{\pgfqpoint{-0.055556in}{0.000000in}}%
\pgfusepath{stroke,fill}%
}%
\begin{pgfscope}%
\pgfsys@transformshift{9.000000in}{1.771429in}%
\pgfsys@useobject{currentmarker}{}%
\end{pgfscope}%
\end{pgfscope}%
\begin{pgfscope}%
\pgftext[left,bottom,x=0.982368in,y=1.708400in,rotate=0.000000]{{\sffamily\fontsize{12.000000}{14.400000}\selectfont 40}}
%
\end{pgfscope}%
\begin{pgfscope}%
\pgfpathrectangle{\pgfqpoint{1.250000in}{0.400000in}}{\pgfqpoint{7.750000in}{3.200000in}} %
\pgfusepath{clip}%
\pgfsetbuttcap%
\pgfsetroundjoin%
\pgfsetlinewidth{0.501875pt}%
\definecolor{currentstroke}{rgb}{0.000000,0.000000,0.000000}%
\pgfsetstrokecolor{currentstroke}%
\pgfsetdash{{1.000000pt}{3.000000pt}}{0.000000pt}%
\pgfpathmoveto{\pgfqpoint{1.250000in}{2.228571in}}%
\pgfpathlineto{\pgfqpoint{9.000000in}{2.228571in}}%
\pgfusepath{stroke}%
\end{pgfscope}%
\begin{pgfscope}%
\pgfsetbuttcap%
\pgfsetroundjoin%
\definecolor{currentfill}{rgb}{0.000000,0.000000,0.000000}%
\pgfsetfillcolor{currentfill}%
\pgfsetlinewidth{0.501875pt}%
\definecolor{currentstroke}{rgb}{0.000000,0.000000,0.000000}%
\pgfsetstrokecolor{currentstroke}%
\pgfsetdash{}{0pt}%
\pgfsys@defobject{currentmarker}{\pgfqpoint{0.000000in}{0.000000in}}{\pgfqpoint{0.055556in}{0.000000in}}{%
\pgfpathmoveto{\pgfqpoint{0.000000in}{0.000000in}}%
\pgfpathlineto{\pgfqpoint{0.055556in}{0.000000in}}%
\pgfusepath{stroke,fill}%
}%
\begin{pgfscope}%
\pgfsys@transformshift{1.250000in}{2.228571in}%
\pgfsys@useobject{currentmarker}{}%
\end{pgfscope}%
\end{pgfscope}%
\begin{pgfscope}%
\pgfsetbuttcap%
\pgfsetroundjoin%
\definecolor{currentfill}{rgb}{0.000000,0.000000,0.000000}%
\pgfsetfillcolor{currentfill}%
\pgfsetlinewidth{0.501875pt}%
\definecolor{currentstroke}{rgb}{0.000000,0.000000,0.000000}%
\pgfsetstrokecolor{currentstroke}%
\pgfsetdash{}{0pt}%
\pgfsys@defobject{currentmarker}{\pgfqpoint{-0.055556in}{0.000000in}}{\pgfqpoint{0.000000in}{0.000000in}}{%
\pgfpathmoveto{\pgfqpoint{0.000000in}{0.000000in}}%
\pgfpathlineto{\pgfqpoint{-0.055556in}{0.000000in}}%
\pgfusepath{stroke,fill}%
}%
\begin{pgfscope}%
\pgfsys@transformshift{9.000000in}{2.228571in}%
\pgfsys@useobject{currentmarker}{}%
\end{pgfscope}%
\end{pgfscope}%
\begin{pgfscope}%
\pgftext[left,bottom,x=0.982368in,y=2.165543in,rotate=0.000000]{{\sffamily\fontsize{12.000000}{14.400000}\selectfont 60}}
%
\end{pgfscope}%
\begin{pgfscope}%
\pgfpathrectangle{\pgfqpoint{1.250000in}{0.400000in}}{\pgfqpoint{7.750000in}{3.200000in}} %
\pgfusepath{clip}%
\pgfsetbuttcap%
\pgfsetroundjoin%
\pgfsetlinewidth{0.501875pt}%
\definecolor{currentstroke}{rgb}{0.000000,0.000000,0.000000}%
\pgfsetstrokecolor{currentstroke}%
\pgfsetdash{{1.000000pt}{3.000000pt}}{0.000000pt}%
\pgfpathmoveto{\pgfqpoint{1.250000in}{2.685714in}}%
\pgfpathlineto{\pgfqpoint{9.000000in}{2.685714in}}%
\pgfusepath{stroke}%
\end{pgfscope}%
\begin{pgfscope}%
\pgfsetbuttcap%
\pgfsetroundjoin%
\definecolor{currentfill}{rgb}{0.000000,0.000000,0.000000}%
\pgfsetfillcolor{currentfill}%
\pgfsetlinewidth{0.501875pt}%
\definecolor{currentstroke}{rgb}{0.000000,0.000000,0.000000}%
\pgfsetstrokecolor{currentstroke}%
\pgfsetdash{}{0pt}%
\pgfsys@defobject{currentmarker}{\pgfqpoint{0.000000in}{0.000000in}}{\pgfqpoint{0.055556in}{0.000000in}}{%
\pgfpathmoveto{\pgfqpoint{0.000000in}{0.000000in}}%
\pgfpathlineto{\pgfqpoint{0.055556in}{0.000000in}}%
\pgfusepath{stroke,fill}%
}%
\begin{pgfscope}%
\pgfsys@transformshift{1.250000in}{2.685714in}%
\pgfsys@useobject{currentmarker}{}%
\end{pgfscope}%
\end{pgfscope}%
\begin{pgfscope}%
\pgfsetbuttcap%
\pgfsetroundjoin%
\definecolor{currentfill}{rgb}{0.000000,0.000000,0.000000}%
\pgfsetfillcolor{currentfill}%
\pgfsetlinewidth{0.501875pt}%
\definecolor{currentstroke}{rgb}{0.000000,0.000000,0.000000}%
\pgfsetstrokecolor{currentstroke}%
\pgfsetdash{}{0pt}%
\pgfsys@defobject{currentmarker}{\pgfqpoint{-0.055556in}{0.000000in}}{\pgfqpoint{0.000000in}{0.000000in}}{%
\pgfpathmoveto{\pgfqpoint{0.000000in}{0.000000in}}%
\pgfpathlineto{\pgfqpoint{-0.055556in}{0.000000in}}%
\pgfusepath{stroke,fill}%
}%
\begin{pgfscope}%
\pgfsys@transformshift{9.000000in}{2.685714in}%
\pgfsys@useobject{currentmarker}{}%
\end{pgfscope}%
\end{pgfscope}%
\begin{pgfscope}%
\pgftext[left,bottom,x=0.982368in,y=2.622685in,rotate=0.000000]{{\sffamily\fontsize{12.000000}{14.400000}\selectfont 80}}
%
\end{pgfscope}%
\begin{pgfscope}%
\pgfpathrectangle{\pgfqpoint{1.250000in}{0.400000in}}{\pgfqpoint{7.750000in}{3.200000in}} %
\pgfusepath{clip}%
\pgfsetbuttcap%
\pgfsetroundjoin%
\pgfsetlinewidth{0.501875pt}%
\definecolor{currentstroke}{rgb}{0.000000,0.000000,0.000000}%
\pgfsetstrokecolor{currentstroke}%
\pgfsetdash{{1.000000pt}{3.000000pt}}{0.000000pt}%
\pgfpathmoveto{\pgfqpoint{1.250000in}{3.142857in}}%
\pgfpathlineto{\pgfqpoint{9.000000in}{3.142857in}}%
\pgfusepath{stroke}%
\end{pgfscope}%
\begin{pgfscope}%
\pgfsetbuttcap%
\pgfsetroundjoin%
\definecolor{currentfill}{rgb}{0.000000,0.000000,0.000000}%
\pgfsetfillcolor{currentfill}%
\pgfsetlinewidth{0.501875pt}%
\definecolor{currentstroke}{rgb}{0.000000,0.000000,0.000000}%
\pgfsetstrokecolor{currentstroke}%
\pgfsetdash{}{0pt}%
\pgfsys@defobject{currentmarker}{\pgfqpoint{0.000000in}{0.000000in}}{\pgfqpoint{0.055556in}{0.000000in}}{%
\pgfpathmoveto{\pgfqpoint{0.000000in}{0.000000in}}%
\pgfpathlineto{\pgfqpoint{0.055556in}{0.000000in}}%
\pgfusepath{stroke,fill}%
}%
\begin{pgfscope}%
\pgfsys@transformshift{1.250000in}{3.142857in}%
\pgfsys@useobject{currentmarker}{}%
\end{pgfscope}%
\end{pgfscope}%
\begin{pgfscope}%
\pgfsetbuttcap%
\pgfsetroundjoin%
\definecolor{currentfill}{rgb}{0.000000,0.000000,0.000000}%
\pgfsetfillcolor{currentfill}%
\pgfsetlinewidth{0.501875pt}%
\definecolor{currentstroke}{rgb}{0.000000,0.000000,0.000000}%
\pgfsetstrokecolor{currentstroke}%
\pgfsetdash{}{0pt}%
\pgfsys@defobject{currentmarker}{\pgfqpoint{-0.055556in}{0.000000in}}{\pgfqpoint{0.000000in}{0.000000in}}{%
\pgfpathmoveto{\pgfqpoint{0.000000in}{0.000000in}}%
\pgfpathlineto{\pgfqpoint{-0.055556in}{0.000000in}}%
\pgfusepath{stroke,fill}%
}%
\begin{pgfscope}%
\pgfsys@transformshift{9.000000in}{3.142857in}%
\pgfsys@useobject{currentmarker}{}%
\end{pgfscope}%
\end{pgfscope}%
\begin{pgfscope}%
\pgftext[left,bottom,x=0.876329in,y=3.079828in,rotate=0.000000]{{\sffamily\fontsize{12.000000}{14.400000}\selectfont 100}}
%
\end{pgfscope}%
\begin{pgfscope}%
\pgfpathrectangle{\pgfqpoint{1.250000in}{0.400000in}}{\pgfqpoint{7.750000in}{3.200000in}} %
\pgfusepath{clip}%
\pgfsetbuttcap%
\pgfsetroundjoin%
\pgfsetlinewidth{0.501875pt}%
\definecolor{currentstroke}{rgb}{0.000000,0.000000,0.000000}%
\pgfsetstrokecolor{currentstroke}%
\pgfsetdash{{1.000000pt}{3.000000pt}}{0.000000pt}%
\pgfpathmoveto{\pgfqpoint{1.250000in}{3.600000in}}%
\pgfpathlineto{\pgfqpoint{9.000000in}{3.600000in}}%
\pgfusepath{stroke}%
\end{pgfscope}%
\begin{pgfscope}%
\pgfsetbuttcap%
\pgfsetroundjoin%
\definecolor{currentfill}{rgb}{0.000000,0.000000,0.000000}%
\pgfsetfillcolor{currentfill}%
\pgfsetlinewidth{0.501875pt}%
\definecolor{currentstroke}{rgb}{0.000000,0.000000,0.000000}%
\pgfsetstrokecolor{currentstroke}%
\pgfsetdash{}{0pt}%
\pgfsys@defobject{currentmarker}{\pgfqpoint{0.000000in}{0.000000in}}{\pgfqpoint{0.055556in}{0.000000in}}{%
\pgfpathmoveto{\pgfqpoint{0.000000in}{0.000000in}}%
\pgfpathlineto{\pgfqpoint{0.055556in}{0.000000in}}%
\pgfusepath{stroke,fill}%
}%
\begin{pgfscope}%
\pgfsys@transformshift{1.250000in}{3.600000in}%
\pgfsys@useobject{currentmarker}{}%
\end{pgfscope}%
\end{pgfscope}%
\begin{pgfscope}%
\pgfsetbuttcap%
\pgfsetroundjoin%
\definecolor{currentfill}{rgb}{0.000000,0.000000,0.000000}%
\pgfsetfillcolor{currentfill}%
\pgfsetlinewidth{0.501875pt}%
\definecolor{currentstroke}{rgb}{0.000000,0.000000,0.000000}%
\pgfsetstrokecolor{currentstroke}%
\pgfsetdash{}{0pt}%
\pgfsys@defobject{currentmarker}{\pgfqpoint{-0.055556in}{0.000000in}}{\pgfqpoint{0.000000in}{0.000000in}}{%
\pgfpathmoveto{\pgfqpoint{0.000000in}{0.000000in}}%
\pgfpathlineto{\pgfqpoint{-0.055556in}{0.000000in}}%
\pgfusepath{stroke,fill}%
}%
\begin{pgfscope}%
\pgfsys@transformshift{9.000000in}{3.600000in}%
\pgfsys@useobject{currentmarker}{}%
\end{pgfscope}%
\end{pgfscope}%
\begin{pgfscope}%
\pgftext[left,bottom,x=0.876329in,y=3.536971in,rotate=0.000000]{{\sffamily\fontsize{12.000000}{14.400000}\selectfont 120}}
%
\end{pgfscope}%
\begin{pgfscope}%
\pgftext[left,bottom,x=0.773275in,y=1.740560in,rotate=90.000000]{{\sffamily\fontsize{12.000000}{14.400000}\selectfont y in m}}
%
\end{pgfscope}%
\begin{pgfscope}%
\pgfsetrectcap%
\pgfsetroundjoin%
\pgfsetlinewidth{1.003750pt}%
\definecolor{currentstroke}{rgb}{0.000000,0.000000,0.000000}%
\pgfsetstrokecolor{currentstroke}%
\pgfsetdash{}{0pt}%
\pgfpathmoveto{\pgfqpoint{1.250000in}{3.600000in}}%
\pgfpathlineto{\pgfqpoint{9.000000in}{3.600000in}}%
\pgfusepath{stroke}%
\end{pgfscope}%
\begin{pgfscope}%
\pgfsetrectcap%
\pgfsetroundjoin%
\pgfsetlinewidth{1.003750pt}%
\definecolor{currentstroke}{rgb}{0.000000,0.000000,0.000000}%
\pgfsetstrokecolor{currentstroke}%
\pgfsetdash{}{0pt}%
\pgfpathmoveto{\pgfqpoint{9.000000in}{0.400000in}}%
\pgfpathlineto{\pgfqpoint{9.000000in}{3.600000in}}%
\pgfusepath{stroke}%
\end{pgfscope}%
\begin{pgfscope}%
\pgfsetrectcap%
\pgfsetroundjoin%
\pgfsetlinewidth{1.003750pt}%
\definecolor{currentstroke}{rgb}{0.000000,0.000000,0.000000}%
\pgfsetstrokecolor{currentstroke}%
\pgfsetdash{}{0pt}%
\pgfpathmoveto{\pgfqpoint{1.250000in}{0.400000in}}%
\pgfpathlineto{\pgfqpoint{9.000000in}{0.400000in}}%
\pgfusepath{stroke}%
\end{pgfscope}%
\begin{pgfscope}%
\pgfsetrectcap%
\pgfsetroundjoin%
\pgfsetlinewidth{1.003750pt}%
\definecolor{currentstroke}{rgb}{0.000000,0.000000,0.000000}%
\pgfsetstrokecolor{currentstroke}%
\pgfsetdash{}{0pt}%
\pgfpathmoveto{\pgfqpoint{1.250000in}{0.400000in}}%
\pgfpathlineto{\pgfqpoint{1.250000in}{3.600000in}}%
\pgfusepath{stroke}%
\end{pgfscope}%
\begin{pgfscope}%
\pgftext[left,bottom,x=2.391065in,y=3.627843in,rotate=0.000000]{{\sffamily\fontsize{14.400000}{17.280000}\selectfont Trajectories of a cannonball at various wind speeds vw}}
%
\end{pgfscope}%
\begin{pgfscope}%
\pgfsetrectcap%
\pgfsetroundjoin%
\definecolor{currentfill}{rgb}{1.000000,1.000000,1.000000}%
\pgfsetfillcolor{currentfill}%
\pgfsetlinewidth{1.003750pt}%
\definecolor{currentstroke}{rgb}{0.000000,0.000000,0.000000}%
\pgfsetstrokecolor{currentstroke}%
\pgfsetdash{}{0pt}%
\pgfpathmoveto{\pgfqpoint{1.319417in}{1.478088in}}%
\pgfpathlineto{\pgfqpoint{2.711262in}{1.478088in}}%
\pgfpathlineto{\pgfqpoint{2.711262in}{3.530583in}}%
\pgfpathlineto{\pgfqpoint{1.319417in}{3.530583in}}%
\pgfpathlineto{\pgfqpoint{1.319417in}{1.478088in}}%
\pgfpathclose%
\pgfusepath{stroke,fill}%
\end{pgfscope}%
\begin{pgfscope}%
\pgfsetrectcap%
\pgfsetroundjoin%
\pgfsetlinewidth{1.003750pt}%
\definecolor{currentstroke}{rgb}{0.000000,0.000000,1.000000}%
\pgfsetstrokecolor{currentstroke}%
\pgfsetdash{}{0pt}%
\pgfpathmoveto{\pgfqpoint{1.416600in}{3.418161in}}%
\pgfpathlineto{\pgfqpoint{1.610967in}{3.418161in}}%
\pgfusepath{stroke}%
\end{pgfscope}%
\begin{pgfscope}%
\pgfsetbuttcap%
\pgfsetroundjoin%
\definecolor{currentfill}{rgb}{0.000000,0.000000,1.000000}%
\pgfsetfillcolor{currentfill}%
\pgfsetlinewidth{0.501875pt}%
\definecolor{currentstroke}{rgb}{0.000000,0.000000,1.000000}%
\pgfsetstrokecolor{currentstroke}%
\pgfsetdash{}{0pt}%
\pgfsys@defobject{currentmarker}{\pgfqpoint{-0.041667in}{-0.041667in}}{\pgfqpoint{0.041667in}{0.041667in}}{%
\pgfpathmoveto{\pgfqpoint{-0.041667in}{0.000000in}}%
\pgfpathlineto{\pgfqpoint{0.041667in}{0.000000in}}%
\pgfpathmoveto{\pgfqpoint{0.000000in}{-0.041667in}}%
\pgfpathlineto{\pgfqpoint{0.000000in}{0.041667in}}%
\pgfusepath{stroke,fill}%
}%
\begin{pgfscope}%
\pgfsys@transformshift{1.416600in}{3.418161in}%
\pgfsys@useobject{currentmarker}{}%
\end{pgfscope}%
\begin{pgfscope}%
\pgfsys@transformshift{1.610967in}{3.418161in}%
\pgfsys@useobject{currentmarker}{}%
\end{pgfscope}%
\end{pgfscope}%
\begin{pgfscope}%
\pgftext[left,bottom,x=1.763683in,y=3.367603in,rotate=0.000000]{{\sffamily\fontsize{9.996000}{11.995200}\selectfont vw =-300.00}}
%
\end{pgfscope}%
\begin{pgfscope}%
\pgfsetrectcap%
\pgfsetroundjoin%
\pgfsetlinewidth{1.003750pt}%
\definecolor{currentstroke}{rgb}{0.000000,0.500000,0.000000}%
\pgfsetstrokecolor{currentstroke}%
\pgfsetdash{}{0pt}%
\pgfpathmoveto{\pgfqpoint{1.416600in}{3.214385in}}%
\pgfpathlineto{\pgfqpoint{1.610967in}{3.214385in}}%
\pgfusepath{stroke}%
\end{pgfscope}%
\begin{pgfscope}%
\pgfsetbuttcap%
\pgfsetroundjoin%
\definecolor{currentfill}{rgb}{0.000000,0.500000,0.000000}%
\pgfsetfillcolor{currentfill}%
\pgfsetlinewidth{0.501875pt}%
\definecolor{currentstroke}{rgb}{0.000000,0.500000,0.000000}%
\pgfsetstrokecolor{currentstroke}%
\pgfsetdash{}{0pt}%
\pgfsys@defobject{currentmarker}{\pgfqpoint{-0.041667in}{-0.041667in}}{\pgfqpoint{0.041667in}{0.041667in}}{%
\pgfpathmoveto{\pgfqpoint{-0.041667in}{0.000000in}}%
\pgfpathlineto{\pgfqpoint{0.041667in}{0.000000in}}%
\pgfpathmoveto{\pgfqpoint{0.000000in}{-0.041667in}}%
\pgfpathlineto{\pgfqpoint{0.000000in}{0.041667in}}%
\pgfusepath{stroke,fill}%
}%
\begin{pgfscope}%
\pgfsys@transformshift{1.416600in}{3.214385in}%
\pgfsys@useobject{currentmarker}{}%
\end{pgfscope}%
\begin{pgfscope}%
\pgfsys@transformshift{1.610967in}{3.214385in}%
\pgfsys@useobject{currentmarker}{}%
\end{pgfscope}%
\end{pgfscope}%
\begin{pgfscope}%
\pgftext[left,bottom,x=1.763683in,y=3.163828in,rotate=0.000000]{{\sffamily\fontsize{9.996000}{11.995200}\selectfont vw =-265.56}}
%
\end{pgfscope}%
\begin{pgfscope}%
\pgfsetrectcap%
\pgfsetroundjoin%
\pgfsetlinewidth{1.003750pt}%
\definecolor{currentstroke}{rgb}{1.000000,0.000000,0.000000}%
\pgfsetstrokecolor{currentstroke}%
\pgfsetdash{}{0pt}%
\pgfpathmoveto{\pgfqpoint{1.416600in}{3.010609in}}%
\pgfpathlineto{\pgfqpoint{1.610967in}{3.010609in}}%
\pgfusepath{stroke}%
\end{pgfscope}%
\begin{pgfscope}%
\pgfsetbuttcap%
\pgfsetroundjoin%
\definecolor{currentfill}{rgb}{1.000000,0.000000,0.000000}%
\pgfsetfillcolor{currentfill}%
\pgfsetlinewidth{0.501875pt}%
\definecolor{currentstroke}{rgb}{1.000000,0.000000,0.000000}%
\pgfsetstrokecolor{currentstroke}%
\pgfsetdash{}{0pt}%
\pgfsys@defobject{currentmarker}{\pgfqpoint{-0.041667in}{-0.041667in}}{\pgfqpoint{0.041667in}{0.041667in}}{%
\pgfpathmoveto{\pgfqpoint{-0.041667in}{0.000000in}}%
\pgfpathlineto{\pgfqpoint{0.041667in}{0.000000in}}%
\pgfpathmoveto{\pgfqpoint{0.000000in}{-0.041667in}}%
\pgfpathlineto{\pgfqpoint{0.000000in}{0.041667in}}%
\pgfusepath{stroke,fill}%
}%
\begin{pgfscope}%
\pgfsys@transformshift{1.416600in}{3.010609in}%
\pgfsys@useobject{currentmarker}{}%
\end{pgfscope}%
\begin{pgfscope}%
\pgfsys@transformshift{1.610967in}{3.010609in}%
\pgfsys@useobject{currentmarker}{}%
\end{pgfscope}%
\end{pgfscope}%
\begin{pgfscope}%
\pgftext[left,bottom,x=1.763683in,y=2.960052in,rotate=0.000000]{{\sffamily\fontsize{9.996000}{11.995200}\selectfont vw =-231.11}}
%
\end{pgfscope}%
\begin{pgfscope}%
\pgfsetrectcap%
\pgfsetroundjoin%
\pgfsetlinewidth{1.003750pt}%
\definecolor{currentstroke}{rgb}{0.000000,0.750000,0.750000}%
\pgfsetstrokecolor{currentstroke}%
\pgfsetdash{}{0pt}%
\pgfpathmoveto{\pgfqpoint{1.416600in}{2.806833in}}%
\pgfpathlineto{\pgfqpoint{1.610967in}{2.806833in}}%
\pgfusepath{stroke}%
\end{pgfscope}%
\begin{pgfscope}%
\pgfsetbuttcap%
\pgfsetroundjoin%
\definecolor{currentfill}{rgb}{0.000000,0.750000,0.750000}%
\pgfsetfillcolor{currentfill}%
\pgfsetlinewidth{0.501875pt}%
\definecolor{currentstroke}{rgb}{0.000000,0.750000,0.750000}%
\pgfsetstrokecolor{currentstroke}%
\pgfsetdash{}{0pt}%
\pgfsys@defobject{currentmarker}{\pgfqpoint{-0.041667in}{-0.041667in}}{\pgfqpoint{0.041667in}{0.041667in}}{%
\pgfpathmoveto{\pgfqpoint{-0.041667in}{0.000000in}}%
\pgfpathlineto{\pgfqpoint{0.041667in}{0.000000in}}%
\pgfpathmoveto{\pgfqpoint{0.000000in}{-0.041667in}}%
\pgfpathlineto{\pgfqpoint{0.000000in}{0.041667in}}%
\pgfusepath{stroke,fill}%
}%
\begin{pgfscope}%
\pgfsys@transformshift{1.416600in}{2.806833in}%
\pgfsys@useobject{currentmarker}{}%
\end{pgfscope}%
\begin{pgfscope}%
\pgfsys@transformshift{1.610967in}{2.806833in}%
\pgfsys@useobject{currentmarker}{}%
\end{pgfscope}%
\end{pgfscope}%
\begin{pgfscope}%
\pgftext[left,bottom,x=1.763683in,y=2.756276in,rotate=0.000000]{{\sffamily\fontsize{9.996000}{11.995200}\selectfont vw =-196.67}}
%
\end{pgfscope}%
\begin{pgfscope}%
\pgfsetrectcap%
\pgfsetroundjoin%
\pgfsetlinewidth{1.003750pt}%
\definecolor{currentstroke}{rgb}{0.750000,0.000000,0.750000}%
\pgfsetstrokecolor{currentstroke}%
\pgfsetdash{}{0pt}%
\pgfpathmoveto{\pgfqpoint{1.416600in}{2.603058in}}%
\pgfpathlineto{\pgfqpoint{1.610967in}{2.603058in}}%
\pgfusepath{stroke}%
\end{pgfscope}%
\begin{pgfscope}%
\pgfsetbuttcap%
\pgfsetroundjoin%
\definecolor{currentfill}{rgb}{0.750000,0.000000,0.750000}%
\pgfsetfillcolor{currentfill}%
\pgfsetlinewidth{0.501875pt}%
\definecolor{currentstroke}{rgb}{0.750000,0.000000,0.750000}%
\pgfsetstrokecolor{currentstroke}%
\pgfsetdash{}{0pt}%
\pgfsys@defobject{currentmarker}{\pgfqpoint{-0.041667in}{-0.041667in}}{\pgfqpoint{0.041667in}{0.041667in}}{%
\pgfpathmoveto{\pgfqpoint{-0.041667in}{0.000000in}}%
\pgfpathlineto{\pgfqpoint{0.041667in}{0.000000in}}%
\pgfpathmoveto{\pgfqpoint{0.000000in}{-0.041667in}}%
\pgfpathlineto{\pgfqpoint{0.000000in}{0.041667in}}%
\pgfusepath{stroke,fill}%
}%
\begin{pgfscope}%
\pgfsys@transformshift{1.416600in}{2.603058in}%
\pgfsys@useobject{currentmarker}{}%
\end{pgfscope}%
\begin{pgfscope}%
\pgfsys@transformshift{1.610967in}{2.603058in}%
\pgfsys@useobject{currentmarker}{}%
\end{pgfscope}%
\end{pgfscope}%
\begin{pgfscope}%
\pgftext[left,bottom,x=1.763683in,y=2.552500in,rotate=0.000000]{{\sffamily\fontsize{9.996000}{11.995200}\selectfont vw =-162.22}}
%
\end{pgfscope}%
\begin{pgfscope}%
\pgfsetrectcap%
\pgfsetroundjoin%
\pgfsetlinewidth{1.003750pt}%
\definecolor{currentstroke}{rgb}{0.750000,0.750000,0.000000}%
\pgfsetstrokecolor{currentstroke}%
\pgfsetdash{}{0pt}%
\pgfpathmoveto{\pgfqpoint{1.416600in}{2.399282in}}%
\pgfpathlineto{\pgfqpoint{1.610967in}{2.399282in}}%
\pgfusepath{stroke}%
\end{pgfscope}%
\begin{pgfscope}%
\pgfsetbuttcap%
\pgfsetroundjoin%
\definecolor{currentfill}{rgb}{0.750000,0.750000,0.000000}%
\pgfsetfillcolor{currentfill}%
\pgfsetlinewidth{0.501875pt}%
\definecolor{currentstroke}{rgb}{0.750000,0.750000,0.000000}%
\pgfsetstrokecolor{currentstroke}%
\pgfsetdash{}{0pt}%
\pgfsys@defobject{currentmarker}{\pgfqpoint{-0.041667in}{-0.041667in}}{\pgfqpoint{0.041667in}{0.041667in}}{%
\pgfpathmoveto{\pgfqpoint{-0.041667in}{0.000000in}}%
\pgfpathlineto{\pgfqpoint{0.041667in}{0.000000in}}%
\pgfpathmoveto{\pgfqpoint{0.000000in}{-0.041667in}}%
\pgfpathlineto{\pgfqpoint{0.000000in}{0.041667in}}%
\pgfusepath{stroke,fill}%
}%
\begin{pgfscope}%
\pgfsys@transformshift{1.416600in}{2.399282in}%
\pgfsys@useobject{currentmarker}{}%
\end{pgfscope}%
\begin{pgfscope}%
\pgfsys@transformshift{1.610967in}{2.399282in}%
\pgfsys@useobject{currentmarker}{}%
\end{pgfscope}%
\end{pgfscope}%
\begin{pgfscope}%
\pgftext[left,bottom,x=1.763683in,y=2.348724in,rotate=0.000000]{{\sffamily\fontsize{9.996000}{11.995200}\selectfont vw =-127.78}}
%
\end{pgfscope}%
\begin{pgfscope}%
\pgfsetrectcap%
\pgfsetroundjoin%
\pgfsetlinewidth{1.003750pt}%
\definecolor{currentstroke}{rgb}{0.000000,0.000000,0.000000}%
\pgfsetstrokecolor{currentstroke}%
\pgfsetdash{}{0pt}%
\pgfpathmoveto{\pgfqpoint{1.416600in}{2.195506in}}%
\pgfpathlineto{\pgfqpoint{1.610967in}{2.195506in}}%
\pgfusepath{stroke}%
\end{pgfscope}%
\begin{pgfscope}%
\pgfsetbuttcap%
\pgfsetroundjoin%
\definecolor{currentfill}{rgb}{0.000000,0.000000,0.000000}%
\pgfsetfillcolor{currentfill}%
\pgfsetlinewidth{0.501875pt}%
\definecolor{currentstroke}{rgb}{0.000000,0.000000,0.000000}%
\pgfsetstrokecolor{currentstroke}%
\pgfsetdash{}{0pt}%
\pgfsys@defobject{currentmarker}{\pgfqpoint{-0.041667in}{-0.041667in}}{\pgfqpoint{0.041667in}{0.041667in}}{%
\pgfpathmoveto{\pgfqpoint{-0.041667in}{0.000000in}}%
\pgfpathlineto{\pgfqpoint{0.041667in}{0.000000in}}%
\pgfpathmoveto{\pgfqpoint{0.000000in}{-0.041667in}}%
\pgfpathlineto{\pgfqpoint{0.000000in}{0.041667in}}%
\pgfusepath{stroke,fill}%
}%
\begin{pgfscope}%
\pgfsys@transformshift{1.416600in}{2.195506in}%
\pgfsys@useobject{currentmarker}{}%
\end{pgfscope}%
\begin{pgfscope}%
\pgfsys@transformshift{1.610967in}{2.195506in}%
\pgfsys@useobject{currentmarker}{}%
\end{pgfscope}%
\end{pgfscope}%
\begin{pgfscope}%
\pgftext[left,bottom,x=1.763683in,y=2.144948in,rotate=0.000000]{{\sffamily\fontsize{9.996000}{11.995200}\selectfont vw =-93.33}}
%
\end{pgfscope}%
\begin{pgfscope}%
\pgfsetrectcap%
\pgfsetroundjoin%
\pgfsetlinewidth{1.003750pt}%
\definecolor{currentstroke}{rgb}{0.000000,0.000000,1.000000}%
\pgfsetstrokecolor{currentstroke}%
\pgfsetdash{}{0pt}%
\pgfpathmoveto{\pgfqpoint{1.416600in}{1.991730in}}%
\pgfpathlineto{\pgfqpoint{1.610967in}{1.991730in}}%
\pgfusepath{stroke}%
\end{pgfscope}%
\begin{pgfscope}%
\pgfsetbuttcap%
\pgfsetroundjoin%
\definecolor{currentfill}{rgb}{0.000000,0.000000,1.000000}%
\pgfsetfillcolor{currentfill}%
\pgfsetlinewidth{0.501875pt}%
\definecolor{currentstroke}{rgb}{0.000000,0.000000,1.000000}%
\pgfsetstrokecolor{currentstroke}%
\pgfsetdash{}{0pt}%
\pgfsys@defobject{currentmarker}{\pgfqpoint{-0.041667in}{-0.041667in}}{\pgfqpoint{0.041667in}{0.041667in}}{%
\pgfpathmoveto{\pgfqpoint{-0.041667in}{0.000000in}}%
\pgfpathlineto{\pgfqpoint{0.041667in}{0.000000in}}%
\pgfpathmoveto{\pgfqpoint{0.000000in}{-0.041667in}}%
\pgfpathlineto{\pgfqpoint{0.000000in}{0.041667in}}%
\pgfusepath{stroke,fill}%
}%
\begin{pgfscope}%
\pgfsys@transformshift{1.416600in}{1.991730in}%
\pgfsys@useobject{currentmarker}{}%
\end{pgfscope}%
\begin{pgfscope}%
\pgfsys@transformshift{1.610967in}{1.991730in}%
\pgfsys@useobject{currentmarker}{}%
\end{pgfscope}%
\end{pgfscope}%
\begin{pgfscope}%
\pgftext[left,bottom,x=1.763683in,y=1.941173in,rotate=0.000000]{{\sffamily\fontsize{9.996000}{11.995200}\selectfont vw =-58.89}}
%
\end{pgfscope}%
\begin{pgfscope}%
\pgfsetrectcap%
\pgfsetroundjoin%
\pgfsetlinewidth{1.003750pt}%
\definecolor{currentstroke}{rgb}{0.000000,0.500000,0.000000}%
\pgfsetstrokecolor{currentstroke}%
\pgfsetdash{}{0pt}%
\pgfpathmoveto{\pgfqpoint{1.416600in}{1.787954in}}%
\pgfpathlineto{\pgfqpoint{1.610967in}{1.787954in}}%
\pgfusepath{stroke}%
\end{pgfscope}%
\begin{pgfscope}%
\pgfsetbuttcap%
\pgfsetroundjoin%
\definecolor{currentfill}{rgb}{0.000000,0.500000,0.000000}%
\pgfsetfillcolor{currentfill}%
\pgfsetlinewidth{0.501875pt}%
\definecolor{currentstroke}{rgb}{0.000000,0.500000,0.000000}%
\pgfsetstrokecolor{currentstroke}%
\pgfsetdash{}{0pt}%
\pgfsys@defobject{currentmarker}{\pgfqpoint{-0.041667in}{-0.041667in}}{\pgfqpoint{0.041667in}{0.041667in}}{%
\pgfpathmoveto{\pgfqpoint{-0.041667in}{0.000000in}}%
\pgfpathlineto{\pgfqpoint{0.041667in}{0.000000in}}%
\pgfpathmoveto{\pgfqpoint{0.000000in}{-0.041667in}}%
\pgfpathlineto{\pgfqpoint{0.000000in}{0.041667in}}%
\pgfusepath{stroke,fill}%
}%
\begin{pgfscope}%
\pgfsys@transformshift{1.416600in}{1.787954in}%
\pgfsys@useobject{currentmarker}{}%
\end{pgfscope}%
\begin{pgfscope}%
\pgfsys@transformshift{1.610967in}{1.787954in}%
\pgfsys@useobject{currentmarker}{}%
\end{pgfscope}%
\end{pgfscope}%
\begin{pgfscope}%
\pgftext[left,bottom,x=1.763683in,y=1.739363in,rotate=0.000000]{{\sffamily\fontsize{9.996000}{11.995200}\selectfont vw =-24.44}}
%
\end{pgfscope}%
\begin{pgfscope}%
\pgfsetrectcap%
\pgfsetroundjoin%
\pgfsetlinewidth{1.003750pt}%
\definecolor{currentstroke}{rgb}{1.000000,0.000000,0.000000}%
\pgfsetstrokecolor{currentstroke}%
\pgfsetdash{}{0pt}%
\pgfpathmoveto{\pgfqpoint{1.416600in}{1.584178in}}%
\pgfpathlineto{\pgfqpoint{1.610967in}{1.584178in}}%
\pgfusepath{stroke}%
\end{pgfscope}%
\begin{pgfscope}%
\pgfsetbuttcap%
\pgfsetroundjoin%
\definecolor{currentfill}{rgb}{1.000000,0.000000,0.000000}%
\pgfsetfillcolor{currentfill}%
\pgfsetlinewidth{0.501875pt}%
\definecolor{currentstroke}{rgb}{1.000000,0.000000,0.000000}%
\pgfsetstrokecolor{currentstroke}%
\pgfsetdash{}{0pt}%
\pgfsys@defobject{currentmarker}{\pgfqpoint{-0.041667in}{-0.041667in}}{\pgfqpoint{0.041667in}{0.041667in}}{%
\pgfpathmoveto{\pgfqpoint{-0.041667in}{0.000000in}}%
\pgfpathlineto{\pgfqpoint{0.041667in}{0.000000in}}%
\pgfpathmoveto{\pgfqpoint{0.000000in}{-0.041667in}}%
\pgfpathlineto{\pgfqpoint{0.000000in}{0.041667in}}%
\pgfusepath{stroke,fill}%
}%
\begin{pgfscope}%
\pgfsys@transformshift{1.416600in}{1.584178in}%
\pgfsys@useobject{currentmarker}{}%
\end{pgfscope}%
\begin{pgfscope}%
\pgfsys@transformshift{1.610967in}{1.584178in}%
\pgfsys@useobject{currentmarker}{}%
\end{pgfscope}%
\end{pgfscope}%
\begin{pgfscope}%
\pgftext[left,bottom,x=1.763683in,y=1.533621in,rotate=0.000000]{{\sffamily\fontsize{9.996000}{11.995200}\selectfont vw = 10.00}}
%
\end{pgfscope}%
\end{pgfpicture}%
\makeatother%
\endgroup%
}
	\caption{Influence of different wind speeds}\label{fig:c2}
\end{figure}

As you can see in figure \ref{fig:c2} the cannonball comes back close to his initial position if the wind speed is about $\vec{v}_W = (0, -196.67)^T \tfrac{m}{s}$.

\section{Theoretical part \label{theo}}
\subsection{Derivation of the Velocity Verlet alogorithm}
Let's start with a Taylor expansion of $x(t + \Delta t)$ and $v(t + \Delta t)$.
\begin{align}
x(t + \Delta t) &= x(t) + \dot{x}(t) \Delta t + \frac{1}{2} \ddot{x}(t) \Delta t ^2 + \mathcal O(\Delta t ^3) \\
&= x(t) + v(t) \Delta t + \frac{1}{2} a(t) \Delta t ^2 + \mathcal O(\Delta t ^3) \\
v(t + \Delta t) &= v(t) + a(t) \Delta t + \frac{1}{2} \dot{a}(t) \Delta t ^2 + \mathcal O(\Delta t ^3)
\end{align}
We get rid of the derivative of $a(t)$ by the use of Taylor's formular once more.
\begin{equation}
\dot{a}(t) \Delta t = a(t+\Delta t) - a(t) + O(\Delta t ^2)
\end{equation}
Considering $\Delta t \cdot O(\Delta t ^2) = O(\Delta t ^3)$ and putting everything together we obtain the Velocity Verlet alogorithm.
\begin{align}
x(t + \Delta t) &= x(t) + v(t) \Delta t + \frac{a(t)}{2} \Delta t ^2 + \mathcal O(\Delta t ^3) \label{eqn:1}\\
v(t + \Delta t) &= v(t) + \frac{a(t)+a(t+\Delta t)}{2} \Delta t + \mathcal O(\Delta t ^3)
\label{eqn:2}
\end{align}

\subsection{Velocity Verlet is equivalent to the standard Verlet algorithm}
We can express x(t) by rearranging equation \ref{eqn:1}.
\begin{equation}
x(t) = x(t + \Delta t) - v(t) \Delta t - \frac{a(t)}{2} \Delta t ^2 + \mathcal O(\Delta t ^3)
\label{eqn:3}
\end{equation}
Also with equation \ref{eqn:1}, we can calculate $x(t+2\Delta t)$ by substitution of $t$ with $t + \Delta t$. Then you have to add the received equation to equation \ref{eqn:3}.

\begin{equation}
x(t + 2 \Delta t) + x(t) = 2 x(t + \Delta t) \left( v(t+\Delta t)- v(t) \right) \Delta t + \frac{a(t+\Delta t)-a(t)}{2} \Delta t ^2 + \mathcal O(\Delta t ^3)
\end{equation}

We use equation \ref{eqn:2} to get an expression for $\left( v(t+\Delta t)- v(t) \right) \Delta t$.
\begin{equation}
\left( v(t+\Delta t)- v(t) \right) \Delta t = \frac{a(t)+a(t+\Delta t)}{2} \Delta t^2 + \mathcal O(\Delta t ^4)
\end{equation}

Plugging in this expression leads to the standard Verlet algorithm.
\begin{equation}
x(t+\Delta t) = 2\cdot x(t) - x(t - \Delta t) + a(t) \cdot \Delta t^2 + \mathcal O(\Delta t ^3)
\end{equation}

\newpage
\subsection{Difficulties with the standard Verlet algorithm}
The standard Verlet algorithm needs the two predecessors $x(t-\Delta t)$ and $x(t)$ to calculate $x(t+\Delta t)$. However the initial conditions give us only one starting value for $x$. Therefore we have to calculate the missing predecessors via Taylor's formula at the very first step. This results in a slightly bigger error than $\mathcal O(\Delta t ^4)$.

\section{Advanced integrators: Solar system}

\subsection{Simulating the solar system with the Euler scheme}
A program was written to simulate a part of the solar system with given initial position \verb x_init {} and velocity \verb v_init {} by only taking account of Newton's law of universial gravitation. For this simulation again the Euler integration scheme was used.

\begin{figure}[H]
	\resizebox{1\textwidth}{!}{%% Creator: Matplotlib, PGF backend
%%
%% To include the figure in your LaTeX document, write
%%   \input{<filename>.pgf}
%%
%% Make sure the required packages are loaded in your preamble
%%   \usepackage{pgf}
%%
%% Figures using additional raster images can only be included by \input if
%% they are in the same directory as the main LaTeX file. For loading figures
%% from other directories you can use the `import` package
%%   \usepackage{import}
%% and then include the figures with
%%   \import{<path to file>}{<filename>.pgf}
%%
%% Matplotlib used the following preamble
%%   \usepackage{fontspec}
%%   \setmainfont{DejaVu Serif}
%%   \setsansfont{DejaVu Sans}
%%   \setmonofont{DejaVu Sans Mono}
%%
\begingroup%
\makeatletter%
\begin{pgfpicture}%
\pgfpathrectangle{\pgfpointorigin}{\pgfqpoint{10.000000in}{4.000000in}}%
\pgfusepath{use as bounding box}%
\begin{pgfscope}%
\pgfsetrectcap%
\pgfsetroundjoin%
\definecolor{currentfill}{rgb}{1.000000,1.000000,1.000000}%
\pgfsetfillcolor{currentfill}%
\pgfsetlinewidth{0.000000pt}%
\definecolor{currentstroke}{rgb}{1.000000,1.000000,1.000000}%
\pgfsetstrokecolor{currentstroke}%
\pgfsetdash{}{0pt}%
\pgfpathmoveto{\pgfqpoint{0.000000in}{0.000000in}}%
\pgfpathlineto{\pgfqpoint{10.000000in}{0.000000in}}%
\pgfpathlineto{\pgfqpoint{10.000000in}{4.000000in}}%
\pgfpathlineto{\pgfqpoint{0.000000in}{4.000000in}}%
\pgfpathclose%
\pgfusepath{fill}%
\end{pgfscope}%
\begin{pgfscope}%
\pgfsetrectcap%
\pgfsetroundjoin%
\definecolor{currentfill}{rgb}{1.000000,1.000000,1.000000}%
\pgfsetfillcolor{currentfill}%
\pgfsetlinewidth{0.000000pt}%
\definecolor{currentstroke}{rgb}{0.000000,0.000000,0.000000}%
\pgfsetstrokecolor{currentstroke}%
\pgfsetdash{}{0pt}%
\pgfpathmoveto{\pgfqpoint{1.250000in}{0.400000in}}%
\pgfpathlineto{\pgfqpoint{9.000000in}{0.400000in}}%
\pgfpathlineto{\pgfqpoint{9.000000in}{3.600000in}}%
\pgfpathlineto{\pgfqpoint{1.250000in}{3.600000in}}%
\pgfpathclose%
\pgfusepath{fill}%
\end{pgfscope}%
\begin{pgfscope}%
\pgfpathrectangle{\pgfqpoint{1.250000in}{0.400000in}}{\pgfqpoint{7.750000in}{3.200000in}} %
\pgfusepath{clip}%
\pgfsetrectcap%
\pgfsetroundjoin%
\pgfsetlinewidth{1.003750pt}%
\definecolor{currentstroke}{rgb}{0.000000,0.000000,1.000000}%
\pgfsetstrokecolor{currentstroke}%
\pgfsetdash{}{0pt}%
\pgfpathmoveto{\pgfqpoint{3.187500in}{1.466667in}}%
\pgfpathlineto{\pgfqpoint{3.188162in}{1.466781in}}%
\pgfpathlineto{\pgfqpoint{3.188162in}{1.466781in}}%
\pgfusepath{stroke}%
\end{pgfscope}%
\begin{pgfscope}%
\pgfpathrectangle{\pgfqpoint{1.250000in}{0.400000in}}{\pgfqpoint{7.750000in}{3.200000in}} %
\pgfusepath{clip}%
\pgfsetrectcap%
\pgfsetroundjoin%
\pgfsetlinewidth{1.003750pt}%
\definecolor{currentstroke}{rgb}{0.000000,0.500000,0.000000}%
\pgfsetstrokecolor{currentstroke}%
\pgfsetdash{}{0pt}%
\pgfpathmoveto{\pgfqpoint{4.156250in}{1.466667in}}%
\pgfpathlineto{\pgfqpoint{4.155110in}{1.501058in}}%
\pgfpathlineto{\pgfqpoint{4.151777in}{1.534930in}}%
\pgfpathlineto{\pgfqpoint{4.146253in}{1.568656in}}%
\pgfpathlineto{\pgfqpoint{4.138424in}{1.602597in}}%
\pgfpathlineto{\pgfqpoint{4.128361in}{1.636226in}}%
\pgfpathlineto{\pgfqpoint{4.116085in}{1.669454in}}%
\pgfpathlineto{\pgfqpoint{4.101633in}{1.702200in}}%
\pgfpathlineto{\pgfqpoint{4.084819in}{1.734800in}}%
\pgfpathlineto{\pgfqpoint{4.065607in}{1.767156in}}%
\pgfpathlineto{\pgfqpoint{4.044250in}{1.798772in}}%
\pgfpathlineto{\pgfqpoint{4.020486in}{1.829963in}}%
\pgfpathlineto{\pgfqpoint{3.994626in}{1.860248in}}%
\pgfpathlineto{\pgfqpoint{3.966363in}{1.889910in}}%
\pgfpathlineto{\pgfqpoint{3.936073in}{1.918483in}}%
\pgfpathlineto{\pgfqpoint{3.903425in}{1.946213in}}%
\pgfpathlineto{\pgfqpoint{3.868446in}{1.972979in}}%
\pgfpathlineto{\pgfqpoint{3.831626in}{1.998361in}}%
\pgfpathlineto{\pgfqpoint{3.792600in}{2.022577in}}%
\pgfpathlineto{\pgfqpoint{3.751414in}{2.045513in}}%
\pgfpathlineto{\pgfqpoint{3.708123in}{2.067055in}}%
\pgfpathlineto{\pgfqpoint{3.663314in}{2.086864in}}%
\pgfpathlineto{\pgfqpoint{3.617109in}{2.104876in}}%
\pgfpathlineto{\pgfqpoint{3.569079in}{2.121209in}}%
\pgfpathlineto{\pgfqpoint{3.519906in}{2.135583in}}%
\pgfpathlineto{\pgfqpoint{3.469169in}{2.148089in}}%
\pgfpathlineto{\pgfqpoint{3.417590in}{2.158518in}}%
\pgfpathlineto{\pgfqpoint{3.364731in}{2.166929in}}%
\pgfpathlineto{\pgfqpoint{3.311331in}{2.173176in}}%
\pgfpathlineto{\pgfqpoint{3.257543in}{2.177241in}}%
\pgfpathlineto{\pgfqpoint{3.203522in}{2.179107in}}%
\pgfpathlineto{\pgfqpoint{3.149433in}{2.178759in}}%
\pgfpathlineto{\pgfqpoint{3.095455in}{2.176189in}}%
\pgfpathlineto{\pgfqpoint{3.041770in}{2.171407in}}%
\pgfpathlineto{\pgfqpoint{2.988551in}{2.164436in}}%
\pgfpathlineto{\pgfqpoint{2.935965in}{2.155308in}}%
\pgfpathlineto{\pgfqpoint{2.884167in}{2.144057in}}%
\pgfpathlineto{\pgfqpoint{2.833302in}{2.130723in}}%
\pgfpathlineto{\pgfqpoint{2.784069in}{2.115522in}}%
\pgfpathlineto{\pgfqpoint{2.736578in}{2.098559in}}%
\pgfpathlineto{\pgfqpoint{2.690414in}{2.079717in}}%
\pgfpathlineto{\pgfqpoint{2.646232in}{2.059301in}}%
\pgfpathlineto{\pgfqpoint{2.603626in}{2.037170in}}%
\pgfpathlineto{\pgfqpoint{2.563185in}{2.013684in}}%
\pgfpathlineto{\pgfqpoint{2.524510in}{1.988666in}}%
\pgfpathlineto{\pgfqpoint{2.488121in}{1.962511in}}%
\pgfpathlineto{\pgfqpoint{2.453640in}{1.934997in}}%
\pgfpathlineto{\pgfqpoint{2.421534in}{1.906543in}}%
\pgfpathlineto{\pgfqpoint{2.391827in}{1.877262in}}%
\pgfpathlineto{\pgfqpoint{2.364531in}{1.847272in}}%
\pgfpathlineto{\pgfqpoint{2.339348in}{1.816304in}}%
\pgfpathlineto{\pgfqpoint{2.316616in}{1.784843in}}%
\pgfpathlineto{\pgfqpoint{2.296063in}{1.752597in}}%
\pgfpathlineto{\pgfqpoint{2.277948in}{1.720064in}}%
\pgfpathlineto{\pgfqpoint{2.262032in}{1.686926in}}%
\pgfpathlineto{\pgfqpoint{2.248508in}{1.653685in}}%
\pgfpathlineto{\pgfqpoint{2.237192in}{1.619995in}}%
\pgfpathlineto{\pgfqpoint{2.228224in}{1.586372in}}%
\pgfpathlineto{\pgfqpoint{2.221472in}{1.552464in}}%
\pgfpathlineto{\pgfqpoint{2.216958in}{1.518353in}}%
\pgfpathlineto{\pgfqpoint{2.214693in}{1.484125in}}%
\pgfpathlineto{\pgfqpoint{2.214680in}{1.449864in}}%
\pgfpathlineto{\pgfqpoint{2.216910in}{1.415650in}}%
\pgfpathlineto{\pgfqpoint{2.221370in}{1.381563in}}%
\pgfpathlineto{\pgfqpoint{2.228043in}{1.347675in}}%
\pgfpathlineto{\pgfqpoint{2.236915in}{1.314057in}}%
\pgfpathlineto{\pgfqpoint{2.247972in}{1.280780in}}%
\pgfpathlineto{\pgfqpoint{2.261387in}{1.247499in}}%
\pgfpathlineto{\pgfqpoint{2.277003in}{1.214731in}}%
\pgfpathlineto{\pgfqpoint{2.294785in}{1.182560in}}%
\pgfpathlineto{\pgfqpoint{2.314958in}{1.150670in}}%
\pgfpathlineto{\pgfqpoint{2.337545in}{1.119167in}}%
\pgfpathlineto{\pgfqpoint{2.362239in}{1.088538in}}%
\pgfpathlineto{\pgfqpoint{2.389320in}{1.058488in}}%
\pgfpathlineto{\pgfqpoint{2.418797in}{1.029122in}}%
\pgfpathlineto{\pgfqpoint{2.450281in}{1.000890in}}%
\pgfpathlineto{\pgfqpoint{2.484123in}{0.973544in}}%
\pgfpathlineto{\pgfqpoint{2.519864in}{0.947518in}}%
\pgfpathlineto{\pgfqpoint{2.557873in}{0.922600in}}%
\pgfpathlineto{\pgfqpoint{2.598102in}{0.898916in}}%
\pgfpathlineto{\pgfqpoint{2.639986in}{0.876842in}}%
\pgfpathlineto{\pgfqpoint{2.683916in}{0.856204in}}%
\pgfpathlineto{\pgfqpoint{2.729817in}{0.837113in}}%
\pgfpathlineto{\pgfqpoint{2.777062in}{0.819867in}}%
\pgfpathlineto{\pgfqpoint{2.826088in}{0.804353in}}%
\pgfpathlineto{\pgfqpoint{2.876222in}{0.790835in}}%
\pgfpathlineto{\pgfqpoint{2.927313in}{0.779362in}}%
\pgfpathlineto{\pgfqpoint{2.979200in}{0.769976in}}%
\pgfpathlineto{\pgfqpoint{3.032312in}{0.762640in}}%
\pgfpathlineto{\pgfqpoint{3.085890in}{0.757487in}}%
\pgfpathlineto{\pgfqpoint{3.139765in}{0.754525in}}%
\pgfpathlineto{\pgfqpoint{3.194390in}{0.753759in}}%
\pgfpathlineto{\pgfqpoint{3.248396in}{0.755210in}}%
\pgfpathlineto{\pgfqpoint{3.302231in}{0.758861in}}%
\pgfpathlineto{\pgfqpoint{3.355726in}{0.764712in}}%
\pgfpathlineto{\pgfqpoint{3.408708in}{0.772750in}}%
\pgfpathlineto{\pgfqpoint{3.460416in}{0.782827in}}%
\pgfpathlineto{\pgfqpoint{3.511279in}{0.794983in}}%
\pgfpathlineto{\pgfqpoint{3.561145in}{0.809172in}}%
\pgfpathlineto{\pgfqpoint{3.609874in}{0.825344in}}%
\pgfpathlineto{\pgfqpoint{3.656798in}{0.843237in}}%
\pgfpathlineto{\pgfqpoint{3.702346in}{0.862978in}}%
\pgfpathlineto{\pgfqpoint{3.745883in}{0.884261in}}%
\pgfpathlineto{\pgfqpoint{3.787328in}{0.906969in}}%
\pgfpathlineto{\pgfqpoint{3.827067in}{0.931276in}}%
\pgfpathlineto{\pgfqpoint{3.864542in}{0.956796in}}%
\pgfpathlineto{\pgfqpoint{3.900125in}{0.983725in}}%
\pgfpathlineto{\pgfqpoint{3.933339in}{1.011638in}}%
\pgfpathlineto{\pgfqpoint{3.964540in}{1.040774in}}%
\pgfpathlineto{\pgfqpoint{3.993312in}{1.070688in}}%
\pgfpathlineto{\pgfqpoint{4.019656in}{1.101269in}}%
\pgfpathlineto{\pgfqpoint{4.043576in}{1.132406in}}%
\pgfpathlineto{\pgfqpoint{4.065341in}{1.164397in}}%
\pgfpathlineto{\pgfqpoint{4.084655in}{1.196743in}}%
\pgfpathlineto{\pgfqpoint{4.101548in}{1.229332in}}%
\pgfpathlineto{\pgfqpoint{4.116241in}{1.262482in}}%
\pgfpathlineto{\pgfqpoint{4.128709in}{1.296112in}}%
\pgfpathlineto{\pgfqpoint{4.138814in}{1.329704in}}%
\pgfpathlineto{\pgfqpoint{4.146709in}{1.363613in}}%
\pgfpathlineto{\pgfqpoint{4.152377in}{1.397766in}}%
\pgfpathlineto{\pgfqpoint{4.155800in}{1.432088in}}%
\pgfpathlineto{\pgfqpoint{4.156625in}{1.448171in}}%
\pgfpathlineto{\pgfqpoint{4.156625in}{1.448171in}}%
\pgfusepath{stroke}%
\end{pgfscope}%
\begin{pgfscope}%
\pgfpathrectangle{\pgfqpoint{1.250000in}{0.400000in}}{\pgfqpoint{7.750000in}{3.200000in}} %
\pgfusepath{clip}%
\pgfsetrectcap%
\pgfsetroundjoin%
\pgfsetlinewidth{1.003750pt}%
\definecolor{currentstroke}{rgb}{1.000000,0.000000,0.000000}%
\pgfsetstrokecolor{currentstroke}%
\pgfsetdash{}{0pt}%
\pgfpathmoveto{\pgfqpoint{4.158733in}{1.466667in}}%
\pgfpathlineto{\pgfqpoint{4.157586in}{1.496174in}}%
\pgfpathlineto{\pgfqpoint{4.154208in}{1.525821in}}%
\pgfpathlineto{\pgfqpoint{4.148396in}{1.557088in}}%
\pgfpathlineto{\pgfqpoint{4.139550in}{1.592225in}}%
\pgfpathlineto{\pgfqpoint{4.126105in}{1.636083in}}%
\pgfpathlineto{\pgfqpoint{4.110264in}{1.680579in}}%
\pgfpathlineto{\pgfqpoint{4.095589in}{1.715357in}}%
\pgfpathlineto{\pgfqpoint{4.080262in}{1.745982in}}%
\pgfpathlineto{\pgfqpoint{4.063654in}{1.774229in}}%
\pgfpathlineto{\pgfqpoint{4.045043in}{1.801465in}}%
\pgfpathlineto{\pgfqpoint{4.024050in}{1.828176in}}%
\pgfpathlineto{\pgfqpoint{3.999927in}{1.855174in}}%
\pgfpathlineto{\pgfqpoint{3.971190in}{1.883891in}}%
\pgfpathlineto{\pgfqpoint{3.933669in}{1.918075in}}%
\pgfpathlineto{\pgfqpoint{3.888142in}{1.956579in}}%
\pgfpathlineto{\pgfqpoint{3.849646in}{1.986325in}}%
\pgfpathlineto{\pgfqpoint{3.814336in}{2.010809in}}%
\pgfpathlineto{\pgfqpoint{3.779406in}{2.032316in}}%
\pgfpathlineto{\pgfqpoint{3.744005in}{2.051512in}}%
\pgfpathlineto{\pgfqpoint{3.706639in}{2.069242in}}%
\pgfpathlineto{\pgfqpoint{3.665730in}{2.086163in}}%
\pgfpathlineto{\pgfqpoint{3.617579in}{2.103587in}}%
\pgfpathlineto{\pgfqpoint{3.554991in}{2.123731in}}%
\pgfpathlineto{\pgfqpoint{3.491151in}{2.142077in}}%
\pgfpathlineto{\pgfqpoint{3.438322in}{2.155018in}}%
\pgfpathlineto{\pgfqpoint{3.390544in}{2.164475in}}%
\pgfpathlineto{\pgfqpoint{3.344471in}{2.171328in}}%
\pgfpathlineto{\pgfqpoint{3.299018in}{2.175827in}}%
\pgfpathlineto{\pgfqpoint{3.251858in}{2.178223in}}%
\pgfpathlineto{\pgfqpoint{3.199524in}{2.178572in}}%
\pgfpathlineto{\pgfqpoint{3.135582in}{2.176650in}}%
\pgfpathlineto{\pgfqpoint{3.062933in}{2.172261in}}%
\pgfpathlineto{\pgfqpoint{3.002904in}{2.166494in}}%
\pgfpathlineto{\pgfqpoint{2.952236in}{2.159452in}}%
\pgfpathlineto{\pgfqpoint{2.906250in}{2.150851in}}%
\pgfpathlineto{\pgfqpoint{2.863232in}{2.140563in}}%
\pgfpathlineto{\pgfqpoint{2.820905in}{2.128149in}}%
\pgfpathlineto{\pgfqpoint{2.776565in}{2.112793in}}%
\pgfpathlineto{\pgfqpoint{2.725543in}{2.092672in}}%
\pgfpathlineto{\pgfqpoint{2.665967in}{2.066729in}}%
\pgfpathlineto{\pgfqpoint{2.612365in}{2.041067in}}%
\pgfpathlineto{\pgfqpoint{2.569210in}{2.018012in}}%
\pgfpathlineto{\pgfqpoint{2.532808in}{1.996110in}}%
\pgfpathlineto{\pgfqpoint{2.500645in}{1.974188in}}%
\pgfpathlineto{\pgfqpoint{2.471276in}{1.951470in}}%
\pgfpathlineto{\pgfqpoint{2.442958in}{1.926715in}}%
\pgfpathlineto{\pgfqpoint{2.413800in}{1.898200in}}%
\pgfpathlineto{\pgfqpoint{2.381753in}{1.863664in}}%
\pgfpathlineto{\pgfqpoint{2.347532in}{1.823516in}}%
\pgfpathlineto{\pgfqpoint{2.318434in}{1.786069in}}%
\pgfpathlineto{\pgfqpoint{2.295889in}{1.753527in}}%
\pgfpathlineto{\pgfqpoint{2.278100in}{1.724033in}}%
\pgfpathlineto{\pgfqpoint{2.263788in}{1.696042in}}%
\pgfpathlineto{\pgfqpoint{2.252102in}{1.668308in}}%
\pgfpathlineto{\pgfqpoint{2.242380in}{1.639505in}}%
\pgfpathlineto{\pgfqpoint{2.233944in}{1.607375in}}%
\pgfpathlineto{\pgfqpoint{2.226635in}{1.570503in}}%
\pgfpathlineto{\pgfqpoint{2.220446in}{1.527619in}}%
\pgfpathlineto{\pgfqpoint{2.216360in}{1.484100in}}%
\pgfpathlineto{\pgfqpoint{2.214964in}{1.446293in}}%
\pgfpathlineto{\pgfqpoint{2.215911in}{1.413485in}}%
\pgfpathlineto{\pgfqpoint{2.218937in}{1.383845in}}%
\pgfpathlineto{\pgfqpoint{2.224057in}{1.355513in}}%
\pgfpathlineto{\pgfqpoint{2.231497in}{1.327077in}}%
\pgfpathlineto{\pgfqpoint{2.241695in}{1.297154in}}%
\pgfpathlineto{\pgfqpoint{2.255263in}{1.264407in}}%
\pgfpathlineto{\pgfqpoint{2.273407in}{1.226644in}}%
\pgfpathlineto{\pgfqpoint{2.296101in}{1.184624in}}%
\pgfpathlineto{\pgfqpoint{2.319792in}{1.145282in}}%
\pgfpathlineto{\pgfqpoint{2.342203in}{1.112269in}}%
\pgfpathlineto{\pgfqpoint{2.363872in}{1.084221in}}%
\pgfpathlineto{\pgfqpoint{2.386359in}{1.058730in}}%
\pgfpathlineto{\pgfqpoint{2.410958in}{1.034264in}}%
\pgfpathlineto{\pgfqpoint{2.438816in}{1.009758in}}%
\pgfpathlineto{\pgfqpoint{2.471641in}{0.983909in}}%
\pgfpathlineto{\pgfqpoint{2.511781in}{0.955235in}}%
\pgfpathlineto{\pgfqpoint{2.560355in}{0.923348in}}%
\pgfpathlineto{\pgfqpoint{2.610119in}{0.893270in}}%
\pgfpathlineto{\pgfqpoint{2.652919in}{0.869898in}}%
\pgfpathlineto{\pgfqpoint{2.691793in}{0.851140in}}%
\pgfpathlineto{\pgfqpoint{2.729971in}{0.835165in}}%
\pgfpathlineto{\pgfqpoint{2.769984in}{0.820853in}}%
\pgfpathlineto{\pgfqpoint{2.814014in}{0.807518in}}%
\pgfpathlineto{\pgfqpoint{2.864340in}{0.794671in}}%
\pgfpathlineto{\pgfqpoint{2.923990in}{0.781818in}}%
\pgfpathlineto{\pgfqpoint{2.992981in}{0.769248in}}%
\pgfpathlineto{\pgfqpoint{3.053454in}{0.760360in}}%
\pgfpathlineto{\pgfqpoint{3.104022in}{0.755117in}}%
\pgfpathlineto{\pgfqpoint{3.150223in}{0.752537in}}%
\pgfpathlineto{\pgfqpoint{3.196097in}{0.752228in}}%
\pgfpathlineto{\pgfqpoint{3.243349in}{0.754160in}}%
\pgfpathlineto{\pgfqpoint{3.294349in}{0.758501in}}%
\pgfpathlineto{\pgfqpoint{3.352045in}{0.765693in}}%
\pgfpathlineto{\pgfqpoint{3.420404in}{0.776510in}}%
\pgfpathlineto{\pgfqpoint{3.485395in}{0.788933in}}%
\pgfpathlineto{\pgfqpoint{3.536057in}{0.800816in}}%
\pgfpathlineto{\pgfqpoint{3.579665in}{0.813295in}}%
\pgfpathlineto{\pgfqpoint{3.620692in}{0.827358in}}%
\pgfpathlineto{\pgfqpoint{3.660775in}{0.843477in}}%
\pgfpathlineto{\pgfqpoint{3.701515in}{0.862298in}}%
\pgfpathlineto{\pgfqpoint{3.744409in}{0.884607in}}%
\pgfpathlineto{\pgfqpoint{3.793338in}{0.912652in}}%
\pgfpathlineto{\pgfqpoint{3.846864in}{0.945872in}}%
\pgfpathlineto{\pgfqpoint{3.887987in}{0.973853in}}%
\pgfpathlineto{\pgfqpoint{3.920593in}{0.998669in}}%
\pgfpathlineto{\pgfqpoint{3.948701in}{1.022804in}}%
\pgfpathlineto{\pgfqpoint{3.974467in}{1.047867in}}%
\pgfpathlineto{\pgfqpoint{3.998360in}{1.074231in}}%
\pgfpathlineto{\pgfqpoint{4.021153in}{1.102710in}}%
\pgfpathlineto{\pgfqpoint{4.044038in}{1.134959in}}%
\pgfpathlineto{\pgfqpoint{4.068809in}{1.173897in}}%
\pgfpathlineto{\pgfqpoint{4.094792in}{1.218864in}}%
\pgfpathlineto{\pgfqpoint{4.113408in}{1.255255in}}%
\pgfpathlineto{\pgfqpoint{4.126857in}{1.286282in}}%
\pgfpathlineto{\pgfqpoint{4.137185in}{1.315612in}}%
\pgfpathlineto{\pgfqpoint{4.145036in}{1.344597in}}%
\pgfpathlineto{\pgfqpoint{4.150645in}{1.373792in}}%
\pgfpathlineto{\pgfqpoint{4.154210in}{1.404244in}}%
\pgfpathlineto{\pgfqpoint{4.155734in}{1.436532in}}%
\pgfpathlineto{\pgfqpoint{4.155781in}{1.449806in}}%
\pgfpathlineto{\pgfqpoint{4.155781in}{1.449806in}}%
\pgfusepath{stroke}%
\end{pgfscope}%
\begin{pgfscope}%
\pgfpathrectangle{\pgfqpoint{1.250000in}{0.400000in}}{\pgfqpoint{7.750000in}{3.200000in}} %
\pgfusepath{clip}%
\pgfsetrectcap%
\pgfsetroundjoin%
\pgfsetlinewidth{1.003750pt}%
\definecolor{currentstroke}{rgb}{0.000000,0.750000,0.750000}%
\pgfsetstrokecolor{currentstroke}%
\pgfsetdash{}{0pt}%
\pgfpathmoveto{\pgfqpoint{4.663972in}{1.466667in}}%
\pgfpathlineto{\pgfqpoint{4.662857in}{1.509818in}}%
\pgfpathlineto{\pgfqpoint{4.659532in}{1.552905in}}%
\pgfpathlineto{\pgfqpoint{4.654003in}{1.595863in}}%
\pgfpathlineto{\pgfqpoint{4.646203in}{1.638995in}}%
\pgfpathlineto{\pgfqpoint{4.636186in}{1.681865in}}%
\pgfpathlineto{\pgfqpoint{4.623968in}{1.724408in}}%
\pgfpathlineto{\pgfqpoint{4.609439in}{1.766919in}}%
\pgfpathlineto{\pgfqpoint{4.592720in}{1.808967in}}%
\pgfpathlineto{\pgfqpoint{4.573672in}{1.850838in}}%
\pgfpathlineto{\pgfqpoint{4.552461in}{1.892110in}}%
\pgfpathlineto{\pgfqpoint{4.528922in}{1.933058in}}%
\pgfpathlineto{\pgfqpoint{4.503044in}{1.973603in}}%
\pgfpathlineto{\pgfqpoint{4.474822in}{2.013661in}}%
\pgfpathlineto{\pgfqpoint{4.444257in}{2.053149in}}%
\pgfpathlineto{\pgfqpoint{4.411629in}{2.091675in}}%
\pgfpathlineto{\pgfqpoint{4.376710in}{2.129480in}}%
\pgfpathlineto{\pgfqpoint{4.339518in}{2.166479in}}%
\pgfpathlineto{\pgfqpoint{4.300077in}{2.202588in}}%
\pgfpathlineto{\pgfqpoint{4.258416in}{2.237722in}}%
\pgfpathlineto{\pgfqpoint{4.214573in}{2.271797in}}%
\pgfpathlineto{\pgfqpoint{4.168589in}{2.304728in}}%
\pgfpathlineto{\pgfqpoint{4.120514in}{2.336432in}}%
\pgfpathlineto{\pgfqpoint{4.070403in}{2.366826in}}%
\pgfpathlineto{\pgfqpoint{4.018319in}{2.395829in}}%
\pgfpathlineto{\pgfqpoint{3.964329in}{2.423362in}}%
\pgfpathlineto{\pgfqpoint{3.908508in}{2.449345in}}%
\pgfpathlineto{\pgfqpoint{3.850937in}{2.473704in}}%
\pgfpathlineto{\pgfqpoint{3.791701in}{2.496363in}}%
\pgfpathlineto{\pgfqpoint{3.730895in}{2.517253in}}%
\pgfpathlineto{\pgfqpoint{3.668615in}{2.536303in}}%
\pgfpathlineto{\pgfqpoint{3.604966in}{2.553450in}}%
\pgfpathlineto{\pgfqpoint{3.540057in}{2.568630in}}%
\pgfpathlineto{\pgfqpoint{3.474002in}{2.581785in}}%
\pgfpathlineto{\pgfqpoint{3.407396in}{2.592789in}}%
\pgfpathlineto{\pgfqpoint{3.339892in}{2.601694in}}%
\pgfpathlineto{\pgfqpoint{3.271617in}{2.608451in}}%
\pgfpathlineto{\pgfqpoint{3.203179in}{2.612995in}}%
\pgfpathlineto{\pgfqpoint{3.134233in}{2.615341in}}%
\pgfpathlineto{\pgfqpoint{3.065391in}{2.615465in}}%
\pgfpathlineto{\pgfqpoint{2.996308in}{2.613359in}}%
\pgfpathlineto{\pgfqpoint{2.927595in}{2.609037in}}%
\pgfpathlineto{\pgfqpoint{2.859382in}{2.602524in}}%
\pgfpathlineto{\pgfqpoint{2.791793in}{2.593848in}}%
\pgfpathlineto{\pgfqpoint{2.724953in}{2.583043in}}%
\pgfpathlineto{\pgfqpoint{2.658980in}{2.570144in}}%
\pgfpathlineto{\pgfqpoint{2.593992in}{2.555191in}}%
\pgfpathlineto{\pgfqpoint{2.530103in}{2.538225in}}%
\pgfpathlineto{\pgfqpoint{2.467420in}{2.519292in}}%
\pgfpathlineto{\pgfqpoint{2.406051in}{2.498440in}}%
\pgfpathlineto{\pgfqpoint{2.346098in}{2.475719in}}%
\pgfpathlineto{\pgfqpoint{2.288057in}{2.451358in}}%
\pgfpathlineto{\pgfqpoint{2.231603in}{2.425260in}}%
\pgfpathlineto{\pgfqpoint{2.177199in}{2.397682in}}%
\pgfpathlineto{\pgfqpoint{2.124532in}{2.368503in}}%
\pgfpathlineto{\pgfqpoint{2.074031in}{2.338004in}}%
\pgfpathlineto{\pgfqpoint{2.025396in}{2.306045in}}%
\pgfpathlineto{\pgfqpoint{1.979021in}{2.272927in}}%
\pgfpathlineto{\pgfqpoint{1.934623in}{2.238490in}}%
\pgfpathlineto{\pgfqpoint{1.892557in}{2.203055in}}%
\pgfpathlineto{\pgfqpoint{1.852839in}{2.166709in}}%
\pgfpathlineto{\pgfqpoint{1.815214in}{2.129266in}}%
\pgfpathlineto{\pgfqpoint{1.779981in}{2.091068in}}%
\pgfpathlineto{\pgfqpoint{1.747138in}{2.052201in}}%
\pgfpathlineto{\pgfqpoint{1.716680in}{2.012745in}}%
\pgfpathlineto{\pgfqpoint{1.688400in}{1.972481in}}%
\pgfpathlineto{\pgfqpoint{1.662514in}{1.931776in}}%
\pgfpathlineto{\pgfqpoint{1.639007in}{1.890705in}}%
\pgfpathlineto{\pgfqpoint{1.617705in}{1.849031in}}%
\pgfpathlineto{\pgfqpoint{1.598769in}{1.807131in}}%
\pgfpathlineto{\pgfqpoint{1.582054in}{1.764755in}}%
\pgfpathlineto{\pgfqpoint{1.567678in}{1.722287in}}%
\pgfpathlineto{\pgfqpoint{1.555525in}{1.679468in}}%
\pgfpathlineto{\pgfqpoint{1.545609in}{1.636358in}}%
\pgfpathlineto{\pgfqpoint{1.537990in}{1.593345in}}%
\pgfpathlineto{\pgfqpoint{1.534390in}{1.566660in}}%
\pgfpathlineto{\pgfqpoint{1.534390in}{1.566660in}}%
\pgfusepath{stroke}%
\end{pgfscope}%
\begin{pgfscope}%
\pgfpathrectangle{\pgfqpoint{1.250000in}{0.400000in}}{\pgfqpoint{7.750000in}{3.200000in}} %
\pgfusepath{clip}%
\pgfsetrectcap%
\pgfsetroundjoin%
\pgfsetlinewidth{1.003750pt}%
\definecolor{currentstroke}{rgb}{0.750000,0.000000,0.750000}%
\pgfsetstrokecolor{currentstroke}%
\pgfsetdash{}{0pt}%
\pgfpathmoveto{\pgfqpoint{3.887906in}{1.466667in}}%
\pgfpathlineto{\pgfqpoint{3.886782in}{1.495602in}}%
\pgfpathlineto{\pgfqpoint{3.883451in}{1.524447in}}%
\pgfpathlineto{\pgfqpoint{3.877925in}{1.553109in}}%
\pgfpathlineto{\pgfqpoint{3.870221in}{1.581499in}}%
\pgfpathlineto{\pgfqpoint{3.860165in}{1.610032in}}%
\pgfpathlineto{\pgfqpoint{3.847909in}{1.638096in}}%
\pgfpathlineto{\pgfqpoint{3.833495in}{1.665600in}}%
\pgfpathlineto{\pgfqpoint{3.816655in}{1.692926in}}%
\pgfpathlineto{\pgfqpoint{3.797686in}{1.719487in}}%
\pgfpathlineto{\pgfqpoint{3.776265in}{1.745635in}}%
\pgfpathlineto{\pgfqpoint{3.752782in}{1.770806in}}%
\pgfpathlineto{\pgfqpoint{3.726864in}{1.795318in}}%
\pgfpathlineto{\pgfqpoint{3.698993in}{1.818640in}}%
\pgfpathlineto{\pgfqpoint{3.668753in}{1.841050in}}%
\pgfpathlineto{\pgfqpoint{3.636163in}{1.862398in}}%
\pgfpathlineto{\pgfqpoint{3.601840in}{1.882217in}}%
\pgfpathlineto{\pgfqpoint{3.565316in}{1.900716in}}%
\pgfpathlineto{\pgfqpoint{3.527286in}{1.917485in}}%
\pgfpathlineto{\pgfqpoint{3.487257in}{1.932685in}}%
\pgfpathlineto{\pgfqpoint{3.445996in}{1.945969in}}%
\pgfpathlineto{\pgfqpoint{3.402995in}{1.957450in}}%
\pgfpathlineto{\pgfqpoint{3.359081in}{1.966852in}}%
\pgfpathlineto{\pgfqpoint{3.314442in}{1.974139in}}%
\pgfpathlineto{\pgfqpoint{3.269267in}{1.979282in}}%
\pgfpathlineto{\pgfqpoint{3.223034in}{1.982289in}}%
\pgfpathlineto{\pgfqpoint{3.176646in}{1.983053in}}%
\pgfpathlineto{\pgfqpoint{3.130305in}{1.981573in}}%
\pgfpathlineto{\pgfqpoint{3.084213in}{1.977857in}}%
\pgfpathlineto{\pgfqpoint{3.038569in}{1.971924in}}%
\pgfpathlineto{\pgfqpoint{2.994257in}{1.963942in}}%
\pgfpathlineto{\pgfqpoint{2.950756in}{1.953873in}}%
\pgfpathlineto{\pgfqpoint{2.908249in}{1.941761in}}%
\pgfpathlineto{\pgfqpoint{2.866911in}{1.927659in}}%
\pgfpathlineto{\pgfqpoint{2.827529in}{1.911891in}}%
\pgfpathlineto{\pgfqpoint{2.789606in}{1.894320in}}%
\pgfpathlineto{\pgfqpoint{2.753858in}{1.875336in}}%
\pgfpathlineto{\pgfqpoint{2.719809in}{1.854751in}}%
\pgfpathlineto{\pgfqpoint{2.688096in}{1.833015in}}%
\pgfpathlineto{\pgfqpoint{2.658274in}{1.809891in}}%
\pgfpathlineto{\pgfqpoint{2.630896in}{1.785878in}}%
\pgfpathlineto{\pgfqpoint{2.605554in}{1.760697in}}%
\pgfpathlineto{\pgfqpoint{2.582710in}{1.734887in}}%
\pgfpathlineto{\pgfqpoint{2.562004in}{1.708128in}}%
\pgfpathlineto{\pgfqpoint{2.543803in}{1.680991in}}%
\pgfpathlineto{\pgfqpoint{2.527798in}{1.653123in}}%
\pgfpathlineto{\pgfqpoint{2.514262in}{1.625115in}}%
\pgfpathlineto{\pgfqpoint{2.502943in}{1.596585in}}%
\pgfpathlineto{\pgfqpoint{2.494018in}{1.568139in}}%
\pgfpathlineto{\pgfqpoint{2.487296in}{1.539370in}}%
\pgfpathlineto{\pgfqpoint{2.482799in}{1.510371in}}%
\pgfpathlineto{\pgfqpoint{2.480540in}{1.481232in}}%
\pgfpathlineto{\pgfqpoint{2.480526in}{1.452048in}}%
\pgfpathlineto{\pgfqpoint{2.482757in}{1.422910in}}%
\pgfpathlineto{\pgfqpoint{2.487225in}{1.393910in}}%
\pgfpathlineto{\pgfqpoint{2.493918in}{1.365142in}}%
\pgfpathlineto{\pgfqpoint{2.502813in}{1.336697in}}%
\pgfpathlineto{\pgfqpoint{2.513883in}{1.308664in}}%
\pgfpathlineto{\pgfqpoint{2.527347in}{1.280646in}}%
\pgfpathlineto{\pgfqpoint{2.542984in}{1.253241in}}%
\pgfpathlineto{\pgfqpoint{2.561074in}{1.226076in}}%
\pgfpathlineto{\pgfqpoint{2.581300in}{1.199732in}}%
\pgfpathlineto{\pgfqpoint{2.603996in}{1.173869in}}%
\pgfpathlineto{\pgfqpoint{2.628752in}{1.149039in}}%
\pgfpathlineto{\pgfqpoint{2.655952in}{1.124940in}}%
\pgfpathlineto{\pgfqpoint{2.685095in}{1.102089in}}%
\pgfpathlineto{\pgfqpoint{2.716607in}{1.080225in}}%
\pgfpathlineto{\pgfqpoint{2.749900in}{1.059823in}}%
\pgfpathlineto{\pgfqpoint{2.785436in}{1.040664in}}%
\pgfpathlineto{\pgfqpoint{2.822545in}{1.023172in}}%
\pgfpathlineto{\pgfqpoint{2.861716in}{1.007177in}}%
\pgfpathlineto{\pgfqpoint{2.902205in}{0.993040in}}%
\pgfpathlineto{\pgfqpoint{2.944519in}{0.980643in}}%
\pgfpathlineto{\pgfqpoint{2.987850in}{0.970279in}}%
\pgfpathlineto{\pgfqpoint{3.032017in}{0.961992in}}%
\pgfpathlineto{\pgfqpoint{3.076836in}{0.955819in}}%
\pgfpathlineto{\pgfqpoint{3.122830in}{0.951742in}}%
\pgfpathlineto{\pgfqpoint{3.169104in}{0.949895in}}%
\pgfpathlineto{\pgfqpoint{3.215459in}{0.950287in}}%
\pgfpathlineto{\pgfqpoint{3.261693in}{0.952919in}}%
\pgfpathlineto{\pgfqpoint{3.307606in}{0.957781in}}%
\pgfpathlineto{\pgfqpoint{3.352304in}{0.964730in}}%
\pgfpathlineto{\pgfqpoint{3.396308in}{0.973795in}}%
\pgfpathlineto{\pgfqpoint{3.439432in}{0.984941in}}%
\pgfpathlineto{\pgfqpoint{3.481491in}{0.998123in}}%
\pgfpathlineto{\pgfqpoint{3.521681in}{1.013034in}}%
\pgfpathlineto{\pgfqpoint{3.560501in}{1.029806in}}%
\pgfpathlineto{\pgfqpoint{3.597211in}{1.048064in}}%
\pgfpathlineto{\pgfqpoint{3.632292in}{1.067987in}}%
\pgfpathlineto{\pgfqpoint{3.665079in}{1.089139in}}%
\pgfpathlineto{\pgfqpoint{3.696025in}{1.111747in}}%
\pgfpathlineto{\pgfqpoint{3.724550in}{1.135322in}}%
\pgfpathlineto{\pgfqpoint{3.751069in}{1.160135in}}%
\pgfpathlineto{\pgfqpoint{3.775092in}{1.185654in}}%
\pgfpathlineto{\pgfqpoint{3.796636in}{1.211732in}}%
\pgfpathlineto{\pgfqpoint{3.816050in}{1.238702in}}%
\pgfpathlineto{\pgfqpoint{3.832987in}{1.265985in}}%
\pgfpathlineto{\pgfqpoint{3.847741in}{1.293947in}}%
\pgfpathlineto{\pgfqpoint{3.860062in}{1.321988in}}%
\pgfpathlineto{\pgfqpoint{3.870187in}{1.350501in}}%
\pgfpathlineto{\pgfqpoint{3.878082in}{1.379395in}}%
\pgfpathlineto{\pgfqpoint{3.883639in}{1.408051in}}%
\pgfpathlineto{\pgfqpoint{3.887002in}{1.436892in}}%
\pgfpathlineto{\pgfqpoint{3.888159in}{1.465827in}}%
\pgfpathlineto{\pgfqpoint{3.887107in}{1.494765in}}%
\pgfpathlineto{\pgfqpoint{3.883849in}{1.523614in}}%
\pgfpathlineto{\pgfqpoint{3.878395in}{1.552284in}}%
\pgfpathlineto{\pgfqpoint{3.870762in}{1.580684in}}%
\pgfpathlineto{\pgfqpoint{3.860777in}{1.609230in}}%
\pgfpathlineto{\pgfqpoint{3.848592in}{1.637311in}}%
\pgfpathlineto{\pgfqpoint{3.834246in}{1.664833in}}%
\pgfpathlineto{\pgfqpoint{3.817475in}{1.692182in}}%
\pgfpathlineto{\pgfqpoint{3.798573in}{1.718767in}}%
\pgfpathlineto{\pgfqpoint{3.777217in}{1.744943in}}%
\pgfpathlineto{\pgfqpoint{3.753798in}{1.770144in}}%
\pgfpathlineto{\pgfqpoint{3.727942in}{1.794690in}}%
\pgfpathlineto{\pgfqpoint{3.700131in}{1.818047in}}%
\pgfpathlineto{\pgfqpoint{3.669947in}{1.840497in}}%
\pgfpathlineto{\pgfqpoint{3.637412in}{1.861886in}}%
\pgfpathlineto{\pgfqpoint{3.603140in}{1.881749in}}%
\pgfpathlineto{\pgfqpoint{3.566664in}{1.900295in}}%
\pgfpathlineto{\pgfqpoint{3.528678in}{1.917112in}}%
\pgfpathlineto{\pgfqpoint{3.488689in}{1.932363in}}%
\pgfpathlineto{\pgfqpoint{3.447465in}{1.945701in}}%
\pgfpathlineto{\pgfqpoint{3.404495in}{1.957236in}}%
\pgfpathlineto{\pgfqpoint{3.360609in}{1.966695in}}%
\pgfpathlineto{\pgfqpoint{3.315992in}{1.974039in}}%
\pgfpathlineto{\pgfqpoint{3.270834in}{1.979239in}}%
\pgfpathlineto{\pgfqpoint{3.224613in}{1.982306in}}%
\pgfpathlineto{\pgfqpoint{3.178232in}{1.983129in}}%
\pgfpathlineto{\pgfqpoint{3.131893in}{1.981709in}}%
\pgfpathlineto{\pgfqpoint{3.085797in}{1.978052in}}%
\pgfpathlineto{\pgfqpoint{3.040145in}{1.972177in}}%
\pgfpathlineto{\pgfqpoint{2.995819in}{1.964252in}}%
\pgfpathlineto{\pgfqpoint{2.952300in}{1.954239in}}%
\pgfpathlineto{\pgfqpoint{2.909769in}{1.942182in}}%
\pgfpathlineto{\pgfqpoint{2.868404in}{1.928132in}}%
\pgfpathlineto{\pgfqpoint{2.828990in}{1.912416in}}%
\pgfpathlineto{\pgfqpoint{2.791031in}{1.894894in}}%
\pgfpathlineto{\pgfqpoint{2.755243in}{1.875956in}}%
\pgfpathlineto{\pgfqpoint{2.721151in}{1.855416in}}%
\pgfpathlineto{\pgfqpoint{2.689392in}{1.833720in}}%
\pgfpathlineto{\pgfqpoint{2.659521in}{1.810635in}}%
\pgfpathlineto{\pgfqpoint{2.632091in}{1.786659in}}%
\pgfpathlineto{\pgfqpoint{2.606695in}{1.761511in}}%
\pgfpathlineto{\pgfqpoint{2.583795in}{1.735731in}}%
\pgfpathlineto{\pgfqpoint{2.563031in}{1.709000in}}%
\pgfpathlineto{\pgfqpoint{2.544770in}{1.681887in}}%
\pgfpathlineto{\pgfqpoint{2.528705in}{1.654041in}}%
\pgfpathlineto{\pgfqpoint{2.515107in}{1.626051in}}%
\pgfpathlineto{\pgfqpoint{2.503726in}{1.597537in}}%
\pgfpathlineto{\pgfqpoint{2.494738in}{1.569103in}}%
\pgfpathlineto{\pgfqpoint{2.487952in}{1.540344in}}%
\pgfpathlineto{\pgfqpoint{2.483391in}{1.511352in}}%
\pgfpathlineto{\pgfqpoint{2.481067in}{1.482218in}}%
\pgfpathlineto{\pgfqpoint{2.480989in}{1.453035in}}%
\pgfpathlineto{\pgfqpoint{2.483155in}{1.423895in}}%
\pgfpathlineto{\pgfqpoint{2.487559in}{1.394891in}}%
\pgfpathlineto{\pgfqpoint{2.494188in}{1.366116in}}%
\pgfpathlineto{\pgfqpoint{2.503020in}{1.337660in}}%
\pgfpathlineto{\pgfqpoint{2.514027in}{1.309615in}}%
\pgfpathlineto{\pgfqpoint{2.527430in}{1.281581in}}%
\pgfpathlineto{\pgfqpoint{2.543006in}{1.254157in}}%
\pgfpathlineto{\pgfqpoint{2.561036in}{1.226971in}}%
\pgfpathlineto{\pgfqpoint{2.581203in}{1.200603in}}%
\pgfpathlineto{\pgfqpoint{2.603842in}{1.174712in}}%
\pgfpathlineto{\pgfqpoint{2.628544in}{1.149852in}}%
\pgfpathlineto{\pgfqpoint{2.651956in}{1.128866in}}%
\pgfpathlineto{\pgfqpoint{2.651956in}{1.128866in}}%
\pgfusepath{stroke}%
\end{pgfscope}%
\begin{pgfscope}%
\pgfpathrectangle{\pgfqpoint{1.250000in}{0.400000in}}{\pgfqpoint{7.750000in}{3.200000in}} %
\pgfusepath{clip}%
\pgfsetrectcap%
\pgfsetroundjoin%
\pgfsetlinewidth{1.003750pt}%
\definecolor{currentstroke}{rgb}{0.750000,0.750000,0.000000}%
\pgfsetstrokecolor{currentstroke}%
\pgfsetdash{}{0pt}%
\pgfpathmoveto{\pgfqpoint{8.227906in}{1.466667in}}%
\pgfpathlineto{\pgfqpoint{8.226792in}{1.544478in}}%
\pgfpathlineto{\pgfqpoint{8.223455in}{1.622256in}}%
\pgfpathlineto{\pgfqpoint{8.217896in}{1.699965in}}%
\pgfpathlineto{\pgfqpoint{8.210118in}{1.777570in}}%
\pgfpathlineto{\pgfqpoint{8.200098in}{1.855234in}}%
\pgfpathlineto{\pgfqpoint{8.187855in}{1.932726in}}%
\pgfpathlineto{\pgfqpoint{8.173395in}{2.010010in}}%
\pgfpathlineto{\pgfqpoint{8.156681in}{2.087248in}}%
\pgfpathlineto{\pgfqpoint{8.137752in}{2.164208in}}%
\pgfpathlineto{\pgfqpoint{8.116563in}{2.241050in}}%
\pgfpathlineto{\pgfqpoint{8.093105in}{2.317736in}}%
\pgfpathlineto{\pgfqpoint{8.067440in}{2.394038in}}%
\pgfpathlineto{\pgfqpoint{8.039507in}{2.470113in}}%
\pgfpathlineto{\pgfqpoint{8.009303in}{2.545921in}}%
\pgfpathlineto{\pgfqpoint{7.976824in}{2.621424in}}%
\pgfpathlineto{\pgfqpoint{7.942070in}{2.696586in}}%
\pgfpathlineto{\pgfqpoint{7.905040in}{2.771366in}}%
\pgfpathlineto{\pgfqpoint{7.865736in}{2.845727in}}%
\pgfpathlineto{\pgfqpoint{7.824159in}{2.919629in}}%
\pgfpathlineto{\pgfqpoint{7.780313in}{2.993033in}}%
\pgfpathlineto{\pgfqpoint{7.734088in}{3.066076in}}%
\pgfpathlineto{\pgfqpoint{7.685594in}{3.138539in}}%
\pgfpathlineto{\pgfqpoint{7.634840in}{3.210380in}}%
\pgfpathlineto{\pgfqpoint{7.581703in}{3.281732in}}%
\pgfpathlineto{\pgfqpoint{7.538746in}{3.336832in}}%
\pgfpathlineto{\pgfqpoint{7.538746in}{3.336832in}}%
\pgfusepath{stroke}%
\end{pgfscope}%
\begin{pgfscope}%
\pgfpathrectangle{\pgfqpoint{1.250000in}{0.400000in}}{\pgfqpoint{7.750000in}{3.200000in}} %
\pgfusepath{clip}%
\pgfsetbuttcap%
\pgfsetroundjoin%
\pgfsetlinewidth{0.501875pt}%
\definecolor{currentstroke}{rgb}{0.000000,0.000000,0.000000}%
\pgfsetstrokecolor{currentstroke}%
\pgfsetdash{{1.000000pt}{3.000000pt}}{0.000000pt}%
\pgfpathmoveto{\pgfqpoint{1.250000in}{0.400000in}}%
\pgfpathlineto{\pgfqpoint{1.250000in}{3.600000in}}%
\pgfusepath{stroke}%
\end{pgfscope}%
\begin{pgfscope}%
\pgfsetbuttcap%
\pgfsetroundjoin%
\definecolor{currentfill}{rgb}{0.000000,0.000000,0.000000}%
\pgfsetfillcolor{currentfill}%
\pgfsetlinewidth{0.501875pt}%
\definecolor{currentstroke}{rgb}{0.000000,0.000000,0.000000}%
\pgfsetstrokecolor{currentstroke}%
\pgfsetdash{}{0pt}%
\pgfsys@defobject{currentmarker}{\pgfqpoint{0.000000in}{0.000000in}}{\pgfqpoint{0.000000in}{0.055556in}}{%
\pgfpathmoveto{\pgfqpoint{0.000000in}{0.000000in}}%
\pgfpathlineto{\pgfqpoint{0.000000in}{0.055556in}}%
\pgfusepath{stroke,fill}%
}%
\begin{pgfscope}%
\pgfsys@transformshift{1.250000in}{0.400000in}%
\pgfsys@useobject{currentmarker}{}%
\end{pgfscope}%
\end{pgfscope}%
\begin{pgfscope}%
\pgfsetbuttcap%
\pgfsetroundjoin%
\definecolor{currentfill}{rgb}{0.000000,0.000000,0.000000}%
\pgfsetfillcolor{currentfill}%
\pgfsetlinewidth{0.501875pt}%
\definecolor{currentstroke}{rgb}{0.000000,0.000000,0.000000}%
\pgfsetstrokecolor{currentstroke}%
\pgfsetdash{}{0pt}%
\pgfsys@defobject{currentmarker}{\pgfqpoint{0.000000in}{-0.055556in}}{\pgfqpoint{0.000000in}{0.000000in}}{%
\pgfpathmoveto{\pgfqpoint{0.000000in}{0.000000in}}%
\pgfpathlineto{\pgfqpoint{0.000000in}{-0.055556in}}%
\pgfusepath{stroke,fill}%
}%
\begin{pgfscope}%
\pgfsys@transformshift{1.250000in}{3.600000in}%
\pgfsys@useobject{currentmarker}{}%
\end{pgfscope}%
\end{pgfscope}%
\begin{pgfscope}%
\pgftext[left,bottom,x=1.127157in,y=0.220747in,rotate=0.000000]{{\sffamily\fontsize{12.000000}{14.400000}\selectfont −2}}
%
\end{pgfscope}%
\begin{pgfscope}%
\pgfpathrectangle{\pgfqpoint{1.250000in}{0.400000in}}{\pgfqpoint{7.750000in}{3.200000in}} %
\pgfusepath{clip}%
\pgfsetbuttcap%
\pgfsetroundjoin%
\pgfsetlinewidth{0.501875pt}%
\definecolor{currentstroke}{rgb}{0.000000,0.000000,0.000000}%
\pgfsetstrokecolor{currentstroke}%
\pgfsetdash{{1.000000pt}{3.000000pt}}{0.000000pt}%
\pgfpathmoveto{\pgfqpoint{2.218750in}{0.400000in}}%
\pgfpathlineto{\pgfqpoint{2.218750in}{3.600000in}}%
\pgfusepath{stroke}%
\end{pgfscope}%
\begin{pgfscope}%
\pgfsetbuttcap%
\pgfsetroundjoin%
\definecolor{currentfill}{rgb}{0.000000,0.000000,0.000000}%
\pgfsetfillcolor{currentfill}%
\pgfsetlinewidth{0.501875pt}%
\definecolor{currentstroke}{rgb}{0.000000,0.000000,0.000000}%
\pgfsetstrokecolor{currentstroke}%
\pgfsetdash{}{0pt}%
\pgfsys@defobject{currentmarker}{\pgfqpoint{0.000000in}{0.000000in}}{\pgfqpoint{0.000000in}{0.055556in}}{%
\pgfpathmoveto{\pgfqpoint{0.000000in}{0.000000in}}%
\pgfpathlineto{\pgfqpoint{0.000000in}{0.055556in}}%
\pgfusepath{stroke,fill}%
}%
\begin{pgfscope}%
\pgfsys@transformshift{2.218750in}{0.400000in}%
\pgfsys@useobject{currentmarker}{}%
\end{pgfscope}%
\end{pgfscope}%
\begin{pgfscope}%
\pgfsetbuttcap%
\pgfsetroundjoin%
\definecolor{currentfill}{rgb}{0.000000,0.000000,0.000000}%
\pgfsetfillcolor{currentfill}%
\pgfsetlinewidth{0.501875pt}%
\definecolor{currentstroke}{rgb}{0.000000,0.000000,0.000000}%
\pgfsetstrokecolor{currentstroke}%
\pgfsetdash{}{0pt}%
\pgfsys@defobject{currentmarker}{\pgfqpoint{0.000000in}{-0.055556in}}{\pgfqpoint{0.000000in}{0.000000in}}{%
\pgfpathmoveto{\pgfqpoint{0.000000in}{0.000000in}}%
\pgfpathlineto{\pgfqpoint{0.000000in}{-0.055556in}}%
\pgfusepath{stroke,fill}%
}%
\begin{pgfscope}%
\pgfsys@transformshift{2.218750in}{3.600000in}%
\pgfsys@useobject{currentmarker}{}%
\end{pgfscope}%
\end{pgfscope}%
\begin{pgfscope}%
\pgftext[left,bottom,x=2.095907in,y=0.222944in,rotate=0.000000]{{\sffamily\fontsize{12.000000}{14.400000}\selectfont −1}}
%
\end{pgfscope}%
\begin{pgfscope}%
\pgfpathrectangle{\pgfqpoint{1.250000in}{0.400000in}}{\pgfqpoint{7.750000in}{3.200000in}} %
\pgfusepath{clip}%
\pgfsetbuttcap%
\pgfsetroundjoin%
\pgfsetlinewidth{0.501875pt}%
\definecolor{currentstroke}{rgb}{0.000000,0.000000,0.000000}%
\pgfsetstrokecolor{currentstroke}%
\pgfsetdash{{1.000000pt}{3.000000pt}}{0.000000pt}%
\pgfpathmoveto{\pgfqpoint{3.187500in}{0.400000in}}%
\pgfpathlineto{\pgfqpoint{3.187500in}{3.600000in}}%
\pgfusepath{stroke}%
\end{pgfscope}%
\begin{pgfscope}%
\pgfsetbuttcap%
\pgfsetroundjoin%
\definecolor{currentfill}{rgb}{0.000000,0.000000,0.000000}%
\pgfsetfillcolor{currentfill}%
\pgfsetlinewidth{0.501875pt}%
\definecolor{currentstroke}{rgb}{0.000000,0.000000,0.000000}%
\pgfsetstrokecolor{currentstroke}%
\pgfsetdash{}{0pt}%
\pgfsys@defobject{currentmarker}{\pgfqpoint{0.000000in}{0.000000in}}{\pgfqpoint{0.000000in}{0.055556in}}{%
\pgfpathmoveto{\pgfqpoint{0.000000in}{0.000000in}}%
\pgfpathlineto{\pgfqpoint{0.000000in}{0.055556in}}%
\pgfusepath{stroke,fill}%
}%
\begin{pgfscope}%
\pgfsys@transformshift{3.187500in}{0.400000in}%
\pgfsys@useobject{currentmarker}{}%
\end{pgfscope}%
\end{pgfscope}%
\begin{pgfscope}%
\pgfsetbuttcap%
\pgfsetroundjoin%
\definecolor{currentfill}{rgb}{0.000000,0.000000,0.000000}%
\pgfsetfillcolor{currentfill}%
\pgfsetlinewidth{0.501875pt}%
\definecolor{currentstroke}{rgb}{0.000000,0.000000,0.000000}%
\pgfsetstrokecolor{currentstroke}%
\pgfsetdash{}{0pt}%
\pgfsys@defobject{currentmarker}{\pgfqpoint{0.000000in}{-0.055556in}}{\pgfqpoint{0.000000in}{0.000000in}}{%
\pgfpathmoveto{\pgfqpoint{0.000000in}{0.000000in}}%
\pgfpathlineto{\pgfqpoint{0.000000in}{-0.055556in}}%
\pgfusepath{stroke,fill}%
}%
\begin{pgfscope}%
\pgfsys@transformshift{3.187500in}{3.600000in}%
\pgfsys@useobject{currentmarker}{}%
\end{pgfscope}%
\end{pgfscope}%
\begin{pgfscope}%
\pgftext[left,bottom,x=3.134481in,y=0.218387in,rotate=0.000000]{{\sffamily\fontsize{12.000000}{14.400000}\selectfont 0}}
%
\end{pgfscope}%
\begin{pgfscope}%
\pgfpathrectangle{\pgfqpoint{1.250000in}{0.400000in}}{\pgfqpoint{7.750000in}{3.200000in}} %
\pgfusepath{clip}%
\pgfsetbuttcap%
\pgfsetroundjoin%
\pgfsetlinewidth{0.501875pt}%
\definecolor{currentstroke}{rgb}{0.000000,0.000000,0.000000}%
\pgfsetstrokecolor{currentstroke}%
\pgfsetdash{{1.000000pt}{3.000000pt}}{0.000000pt}%
\pgfpathmoveto{\pgfqpoint{4.156250in}{0.400000in}}%
\pgfpathlineto{\pgfqpoint{4.156250in}{3.600000in}}%
\pgfusepath{stroke}%
\end{pgfscope}%
\begin{pgfscope}%
\pgfsetbuttcap%
\pgfsetroundjoin%
\definecolor{currentfill}{rgb}{0.000000,0.000000,0.000000}%
\pgfsetfillcolor{currentfill}%
\pgfsetlinewidth{0.501875pt}%
\definecolor{currentstroke}{rgb}{0.000000,0.000000,0.000000}%
\pgfsetstrokecolor{currentstroke}%
\pgfsetdash{}{0pt}%
\pgfsys@defobject{currentmarker}{\pgfqpoint{0.000000in}{0.000000in}}{\pgfqpoint{0.000000in}{0.055556in}}{%
\pgfpathmoveto{\pgfqpoint{0.000000in}{0.000000in}}%
\pgfpathlineto{\pgfqpoint{0.000000in}{0.055556in}}%
\pgfusepath{stroke,fill}%
}%
\begin{pgfscope}%
\pgfsys@transformshift{4.156250in}{0.400000in}%
\pgfsys@useobject{currentmarker}{}%
\end{pgfscope}%
\end{pgfscope}%
\begin{pgfscope}%
\pgfsetbuttcap%
\pgfsetroundjoin%
\definecolor{currentfill}{rgb}{0.000000,0.000000,0.000000}%
\pgfsetfillcolor{currentfill}%
\pgfsetlinewidth{0.501875pt}%
\definecolor{currentstroke}{rgb}{0.000000,0.000000,0.000000}%
\pgfsetstrokecolor{currentstroke}%
\pgfsetdash{}{0pt}%
\pgfsys@defobject{currentmarker}{\pgfqpoint{0.000000in}{-0.055556in}}{\pgfqpoint{0.000000in}{0.000000in}}{%
\pgfpathmoveto{\pgfqpoint{0.000000in}{0.000000in}}%
\pgfpathlineto{\pgfqpoint{0.000000in}{-0.055556in}}%
\pgfusepath{stroke,fill}%
}%
\begin{pgfscope}%
\pgfsys@transformshift{4.156250in}{3.600000in}%
\pgfsys@useobject{currentmarker}{}%
\end{pgfscope}%
\end{pgfscope}%
\begin{pgfscope}%
\pgftext[left,bottom,x=4.103231in,y=0.222944in,rotate=0.000000]{{\sffamily\fontsize{12.000000}{14.400000}\selectfont 1}}
%
\end{pgfscope}%
\begin{pgfscope}%
\pgfpathrectangle{\pgfqpoint{1.250000in}{0.400000in}}{\pgfqpoint{7.750000in}{3.200000in}} %
\pgfusepath{clip}%
\pgfsetbuttcap%
\pgfsetroundjoin%
\pgfsetlinewidth{0.501875pt}%
\definecolor{currentstroke}{rgb}{0.000000,0.000000,0.000000}%
\pgfsetstrokecolor{currentstroke}%
\pgfsetdash{{1.000000pt}{3.000000pt}}{0.000000pt}%
\pgfpathmoveto{\pgfqpoint{5.125000in}{0.400000in}}%
\pgfpathlineto{\pgfqpoint{5.125000in}{3.600000in}}%
\pgfusepath{stroke}%
\end{pgfscope}%
\begin{pgfscope}%
\pgfsetbuttcap%
\pgfsetroundjoin%
\definecolor{currentfill}{rgb}{0.000000,0.000000,0.000000}%
\pgfsetfillcolor{currentfill}%
\pgfsetlinewidth{0.501875pt}%
\definecolor{currentstroke}{rgb}{0.000000,0.000000,0.000000}%
\pgfsetstrokecolor{currentstroke}%
\pgfsetdash{}{0pt}%
\pgfsys@defobject{currentmarker}{\pgfqpoint{0.000000in}{0.000000in}}{\pgfqpoint{0.000000in}{0.055556in}}{%
\pgfpathmoveto{\pgfqpoint{0.000000in}{0.000000in}}%
\pgfpathlineto{\pgfqpoint{0.000000in}{0.055556in}}%
\pgfusepath{stroke,fill}%
}%
\begin{pgfscope}%
\pgfsys@transformshift{5.125000in}{0.400000in}%
\pgfsys@useobject{currentmarker}{}%
\end{pgfscope}%
\end{pgfscope}%
\begin{pgfscope}%
\pgfsetbuttcap%
\pgfsetroundjoin%
\definecolor{currentfill}{rgb}{0.000000,0.000000,0.000000}%
\pgfsetfillcolor{currentfill}%
\pgfsetlinewidth{0.501875pt}%
\definecolor{currentstroke}{rgb}{0.000000,0.000000,0.000000}%
\pgfsetstrokecolor{currentstroke}%
\pgfsetdash{}{0pt}%
\pgfsys@defobject{currentmarker}{\pgfqpoint{0.000000in}{-0.055556in}}{\pgfqpoint{0.000000in}{0.000000in}}{%
\pgfpathmoveto{\pgfqpoint{0.000000in}{0.000000in}}%
\pgfpathlineto{\pgfqpoint{0.000000in}{-0.055556in}}%
\pgfusepath{stroke,fill}%
}%
\begin{pgfscope}%
\pgfsys@transformshift{5.125000in}{3.600000in}%
\pgfsys@useobject{currentmarker}{}%
\end{pgfscope}%
\end{pgfscope}%
\begin{pgfscope}%
\pgftext[left,bottom,x=5.071981in,y=0.220747in,rotate=0.000000]{{\sffamily\fontsize{12.000000}{14.400000}\selectfont 2}}
%
\end{pgfscope}%
\begin{pgfscope}%
\pgfpathrectangle{\pgfqpoint{1.250000in}{0.400000in}}{\pgfqpoint{7.750000in}{3.200000in}} %
\pgfusepath{clip}%
\pgfsetbuttcap%
\pgfsetroundjoin%
\pgfsetlinewidth{0.501875pt}%
\definecolor{currentstroke}{rgb}{0.000000,0.000000,0.000000}%
\pgfsetstrokecolor{currentstroke}%
\pgfsetdash{{1.000000pt}{3.000000pt}}{0.000000pt}%
\pgfpathmoveto{\pgfqpoint{6.093750in}{0.400000in}}%
\pgfpathlineto{\pgfqpoint{6.093750in}{3.600000in}}%
\pgfusepath{stroke}%
\end{pgfscope}%
\begin{pgfscope}%
\pgfsetbuttcap%
\pgfsetroundjoin%
\definecolor{currentfill}{rgb}{0.000000,0.000000,0.000000}%
\pgfsetfillcolor{currentfill}%
\pgfsetlinewidth{0.501875pt}%
\definecolor{currentstroke}{rgb}{0.000000,0.000000,0.000000}%
\pgfsetstrokecolor{currentstroke}%
\pgfsetdash{}{0pt}%
\pgfsys@defobject{currentmarker}{\pgfqpoint{0.000000in}{0.000000in}}{\pgfqpoint{0.000000in}{0.055556in}}{%
\pgfpathmoveto{\pgfqpoint{0.000000in}{0.000000in}}%
\pgfpathlineto{\pgfqpoint{0.000000in}{0.055556in}}%
\pgfusepath{stroke,fill}%
}%
\begin{pgfscope}%
\pgfsys@transformshift{6.093750in}{0.400000in}%
\pgfsys@useobject{currentmarker}{}%
\end{pgfscope}%
\end{pgfscope}%
\begin{pgfscope}%
\pgfsetbuttcap%
\pgfsetroundjoin%
\definecolor{currentfill}{rgb}{0.000000,0.000000,0.000000}%
\pgfsetfillcolor{currentfill}%
\pgfsetlinewidth{0.501875pt}%
\definecolor{currentstroke}{rgb}{0.000000,0.000000,0.000000}%
\pgfsetstrokecolor{currentstroke}%
\pgfsetdash{}{0pt}%
\pgfsys@defobject{currentmarker}{\pgfqpoint{0.000000in}{-0.055556in}}{\pgfqpoint{0.000000in}{0.000000in}}{%
\pgfpathmoveto{\pgfqpoint{0.000000in}{0.000000in}}%
\pgfpathlineto{\pgfqpoint{0.000000in}{-0.055556in}}%
\pgfusepath{stroke,fill}%
}%
\begin{pgfscope}%
\pgfsys@transformshift{6.093750in}{3.600000in}%
\pgfsys@useobject{currentmarker}{}%
\end{pgfscope}%
\end{pgfscope}%
\begin{pgfscope}%
\pgftext[left,bottom,x=6.040731in,y=0.218387in,rotate=0.000000]{{\sffamily\fontsize{12.000000}{14.400000}\selectfont 3}}
%
\end{pgfscope}%
\begin{pgfscope}%
\pgfpathrectangle{\pgfqpoint{1.250000in}{0.400000in}}{\pgfqpoint{7.750000in}{3.200000in}} %
\pgfusepath{clip}%
\pgfsetbuttcap%
\pgfsetroundjoin%
\pgfsetlinewidth{0.501875pt}%
\definecolor{currentstroke}{rgb}{0.000000,0.000000,0.000000}%
\pgfsetstrokecolor{currentstroke}%
\pgfsetdash{{1.000000pt}{3.000000pt}}{0.000000pt}%
\pgfpathmoveto{\pgfqpoint{7.062500in}{0.400000in}}%
\pgfpathlineto{\pgfqpoint{7.062500in}{3.600000in}}%
\pgfusepath{stroke}%
\end{pgfscope}%
\begin{pgfscope}%
\pgfsetbuttcap%
\pgfsetroundjoin%
\definecolor{currentfill}{rgb}{0.000000,0.000000,0.000000}%
\pgfsetfillcolor{currentfill}%
\pgfsetlinewidth{0.501875pt}%
\definecolor{currentstroke}{rgb}{0.000000,0.000000,0.000000}%
\pgfsetstrokecolor{currentstroke}%
\pgfsetdash{}{0pt}%
\pgfsys@defobject{currentmarker}{\pgfqpoint{0.000000in}{0.000000in}}{\pgfqpoint{0.000000in}{0.055556in}}{%
\pgfpathmoveto{\pgfqpoint{0.000000in}{0.000000in}}%
\pgfpathlineto{\pgfqpoint{0.000000in}{0.055556in}}%
\pgfusepath{stroke,fill}%
}%
\begin{pgfscope}%
\pgfsys@transformshift{7.062500in}{0.400000in}%
\pgfsys@useobject{currentmarker}{}%
\end{pgfscope}%
\end{pgfscope}%
\begin{pgfscope}%
\pgfsetbuttcap%
\pgfsetroundjoin%
\definecolor{currentfill}{rgb}{0.000000,0.000000,0.000000}%
\pgfsetfillcolor{currentfill}%
\pgfsetlinewidth{0.501875pt}%
\definecolor{currentstroke}{rgb}{0.000000,0.000000,0.000000}%
\pgfsetstrokecolor{currentstroke}%
\pgfsetdash{}{0pt}%
\pgfsys@defobject{currentmarker}{\pgfqpoint{0.000000in}{-0.055556in}}{\pgfqpoint{0.000000in}{0.000000in}}{%
\pgfpathmoveto{\pgfqpoint{0.000000in}{0.000000in}}%
\pgfpathlineto{\pgfqpoint{0.000000in}{-0.055556in}}%
\pgfusepath{stroke,fill}%
}%
\begin{pgfscope}%
\pgfsys@transformshift{7.062500in}{3.600000in}%
\pgfsys@useobject{currentmarker}{}%
\end{pgfscope}%
\end{pgfscope}%
\begin{pgfscope}%
\pgftext[left,bottom,x=7.009481in,y=0.222944in,rotate=0.000000]{{\sffamily\fontsize{12.000000}{14.400000}\selectfont 4}}
%
\end{pgfscope}%
\begin{pgfscope}%
\pgfpathrectangle{\pgfqpoint{1.250000in}{0.400000in}}{\pgfqpoint{7.750000in}{3.200000in}} %
\pgfusepath{clip}%
\pgfsetbuttcap%
\pgfsetroundjoin%
\pgfsetlinewidth{0.501875pt}%
\definecolor{currentstroke}{rgb}{0.000000,0.000000,0.000000}%
\pgfsetstrokecolor{currentstroke}%
\pgfsetdash{{1.000000pt}{3.000000pt}}{0.000000pt}%
\pgfpathmoveto{\pgfqpoint{8.031250in}{0.400000in}}%
\pgfpathlineto{\pgfqpoint{8.031250in}{3.600000in}}%
\pgfusepath{stroke}%
\end{pgfscope}%
\begin{pgfscope}%
\pgfsetbuttcap%
\pgfsetroundjoin%
\definecolor{currentfill}{rgb}{0.000000,0.000000,0.000000}%
\pgfsetfillcolor{currentfill}%
\pgfsetlinewidth{0.501875pt}%
\definecolor{currentstroke}{rgb}{0.000000,0.000000,0.000000}%
\pgfsetstrokecolor{currentstroke}%
\pgfsetdash{}{0pt}%
\pgfsys@defobject{currentmarker}{\pgfqpoint{0.000000in}{0.000000in}}{\pgfqpoint{0.000000in}{0.055556in}}{%
\pgfpathmoveto{\pgfqpoint{0.000000in}{0.000000in}}%
\pgfpathlineto{\pgfqpoint{0.000000in}{0.055556in}}%
\pgfusepath{stroke,fill}%
}%
\begin{pgfscope}%
\pgfsys@transformshift{8.031250in}{0.400000in}%
\pgfsys@useobject{currentmarker}{}%
\end{pgfscope}%
\end{pgfscope}%
\begin{pgfscope}%
\pgfsetbuttcap%
\pgfsetroundjoin%
\definecolor{currentfill}{rgb}{0.000000,0.000000,0.000000}%
\pgfsetfillcolor{currentfill}%
\pgfsetlinewidth{0.501875pt}%
\definecolor{currentstroke}{rgb}{0.000000,0.000000,0.000000}%
\pgfsetstrokecolor{currentstroke}%
\pgfsetdash{}{0pt}%
\pgfsys@defobject{currentmarker}{\pgfqpoint{0.000000in}{-0.055556in}}{\pgfqpoint{0.000000in}{0.000000in}}{%
\pgfpathmoveto{\pgfqpoint{0.000000in}{0.000000in}}%
\pgfpathlineto{\pgfqpoint{0.000000in}{-0.055556in}}%
\pgfusepath{stroke,fill}%
}%
\begin{pgfscope}%
\pgfsys@transformshift{8.031250in}{3.600000in}%
\pgfsys@useobject{currentmarker}{}%
\end{pgfscope}%
\end{pgfscope}%
\begin{pgfscope}%
\pgftext[left,bottom,x=7.978231in,y=0.220584in,rotate=0.000000]{{\sffamily\fontsize{12.000000}{14.400000}\selectfont 5}}
%
\end{pgfscope}%
\begin{pgfscope}%
\pgfpathrectangle{\pgfqpoint{1.250000in}{0.400000in}}{\pgfqpoint{7.750000in}{3.200000in}} %
\pgfusepath{clip}%
\pgfsetbuttcap%
\pgfsetroundjoin%
\pgfsetlinewidth{0.501875pt}%
\definecolor{currentstroke}{rgb}{0.000000,0.000000,0.000000}%
\pgfsetstrokecolor{currentstroke}%
\pgfsetdash{{1.000000pt}{3.000000pt}}{0.000000pt}%
\pgfpathmoveto{\pgfqpoint{9.000000in}{0.400000in}}%
\pgfpathlineto{\pgfqpoint{9.000000in}{3.600000in}}%
\pgfusepath{stroke}%
\end{pgfscope}%
\begin{pgfscope}%
\pgfsetbuttcap%
\pgfsetroundjoin%
\definecolor{currentfill}{rgb}{0.000000,0.000000,0.000000}%
\pgfsetfillcolor{currentfill}%
\pgfsetlinewidth{0.501875pt}%
\definecolor{currentstroke}{rgb}{0.000000,0.000000,0.000000}%
\pgfsetstrokecolor{currentstroke}%
\pgfsetdash{}{0pt}%
\pgfsys@defobject{currentmarker}{\pgfqpoint{0.000000in}{0.000000in}}{\pgfqpoint{0.000000in}{0.055556in}}{%
\pgfpathmoveto{\pgfqpoint{0.000000in}{0.000000in}}%
\pgfpathlineto{\pgfqpoint{0.000000in}{0.055556in}}%
\pgfusepath{stroke,fill}%
}%
\begin{pgfscope}%
\pgfsys@transformshift{9.000000in}{0.400000in}%
\pgfsys@useobject{currentmarker}{}%
\end{pgfscope}%
\end{pgfscope}%
\begin{pgfscope}%
\pgfsetbuttcap%
\pgfsetroundjoin%
\definecolor{currentfill}{rgb}{0.000000,0.000000,0.000000}%
\pgfsetfillcolor{currentfill}%
\pgfsetlinewidth{0.501875pt}%
\definecolor{currentstroke}{rgb}{0.000000,0.000000,0.000000}%
\pgfsetstrokecolor{currentstroke}%
\pgfsetdash{}{0pt}%
\pgfsys@defobject{currentmarker}{\pgfqpoint{0.000000in}{-0.055556in}}{\pgfqpoint{0.000000in}{0.000000in}}{%
\pgfpathmoveto{\pgfqpoint{0.000000in}{0.000000in}}%
\pgfpathlineto{\pgfqpoint{0.000000in}{-0.055556in}}%
\pgfusepath{stroke,fill}%
}%
\begin{pgfscope}%
\pgfsys@transformshift{9.000000in}{3.600000in}%
\pgfsys@useobject{currentmarker}{}%
\end{pgfscope}%
\end{pgfscope}%
\begin{pgfscope}%
\pgftext[left,bottom,x=8.946981in,y=0.218387in,rotate=0.000000]{{\sffamily\fontsize{12.000000}{14.400000}\selectfont 6}}
%
\end{pgfscope}%
\begin{pgfscope}%
\pgfpathrectangle{\pgfqpoint{1.250000in}{0.400000in}}{\pgfqpoint{7.750000in}{3.200000in}} %
\pgfusepath{clip}%
\pgfsetbuttcap%
\pgfsetroundjoin%
\pgfsetlinewidth{0.501875pt}%
\definecolor{currentstroke}{rgb}{0.000000,0.000000,0.000000}%
\pgfsetstrokecolor{currentstroke}%
\pgfsetdash{{1.000000pt}{3.000000pt}}{0.000000pt}%
\pgfpathmoveto{\pgfqpoint{1.250000in}{0.400000in}}%
\pgfpathlineto{\pgfqpoint{9.000000in}{0.400000in}}%
\pgfusepath{stroke}%
\end{pgfscope}%
\begin{pgfscope}%
\pgfsetbuttcap%
\pgfsetroundjoin%
\definecolor{currentfill}{rgb}{0.000000,0.000000,0.000000}%
\pgfsetfillcolor{currentfill}%
\pgfsetlinewidth{0.501875pt}%
\definecolor{currentstroke}{rgb}{0.000000,0.000000,0.000000}%
\pgfsetstrokecolor{currentstroke}%
\pgfsetdash{}{0pt}%
\pgfsys@defobject{currentmarker}{\pgfqpoint{0.000000in}{0.000000in}}{\pgfqpoint{0.055556in}{0.000000in}}{%
\pgfpathmoveto{\pgfqpoint{0.000000in}{0.000000in}}%
\pgfpathlineto{\pgfqpoint{0.055556in}{0.000000in}}%
\pgfusepath{stroke,fill}%
}%
\begin{pgfscope}%
\pgfsys@transformshift{1.250000in}{0.400000in}%
\pgfsys@useobject{currentmarker}{}%
\end{pgfscope}%
\end{pgfscope}%
\begin{pgfscope}%
\pgfsetbuttcap%
\pgfsetroundjoin%
\definecolor{currentfill}{rgb}{0.000000,0.000000,0.000000}%
\pgfsetfillcolor{currentfill}%
\pgfsetlinewidth{0.501875pt}%
\definecolor{currentstroke}{rgb}{0.000000,0.000000,0.000000}%
\pgfsetstrokecolor{currentstroke}%
\pgfsetdash{}{0pt}%
\pgfsys@defobject{currentmarker}{\pgfqpoint{-0.055556in}{0.000000in}}{\pgfqpoint{0.000000in}{0.000000in}}{%
\pgfpathmoveto{\pgfqpoint{0.000000in}{0.000000in}}%
\pgfpathlineto{\pgfqpoint{-0.055556in}{0.000000in}}%
\pgfusepath{stroke,fill}%
}%
\begin{pgfscope}%
\pgfsys@transformshift{9.000000in}{0.400000in}%
\pgfsys@useobject{currentmarker}{}%
\end{pgfscope}%
\end{pgfscope}%
\begin{pgfscope}%
\pgftext[left,bottom,x=0.789741in,y=0.338070in,rotate=0.000000]{{\sffamily\fontsize{12.000000}{14.400000}\selectfont −1.5}}
%
\end{pgfscope}%
\begin{pgfscope}%
\pgfpathrectangle{\pgfqpoint{1.250000in}{0.400000in}}{\pgfqpoint{7.750000in}{3.200000in}} %
\pgfusepath{clip}%
\pgfsetbuttcap%
\pgfsetroundjoin%
\pgfsetlinewidth{0.501875pt}%
\definecolor{currentstroke}{rgb}{0.000000,0.000000,0.000000}%
\pgfsetstrokecolor{currentstroke}%
\pgfsetdash{{1.000000pt}{3.000000pt}}{0.000000pt}%
\pgfpathmoveto{\pgfqpoint{1.250000in}{0.755556in}}%
\pgfpathlineto{\pgfqpoint{9.000000in}{0.755556in}}%
\pgfusepath{stroke}%
\end{pgfscope}%
\begin{pgfscope}%
\pgfsetbuttcap%
\pgfsetroundjoin%
\definecolor{currentfill}{rgb}{0.000000,0.000000,0.000000}%
\pgfsetfillcolor{currentfill}%
\pgfsetlinewidth{0.501875pt}%
\definecolor{currentstroke}{rgb}{0.000000,0.000000,0.000000}%
\pgfsetstrokecolor{currentstroke}%
\pgfsetdash{}{0pt}%
\pgfsys@defobject{currentmarker}{\pgfqpoint{0.000000in}{0.000000in}}{\pgfqpoint{0.055556in}{0.000000in}}{%
\pgfpathmoveto{\pgfqpoint{0.000000in}{0.000000in}}%
\pgfpathlineto{\pgfqpoint{0.055556in}{0.000000in}}%
\pgfusepath{stroke,fill}%
}%
\begin{pgfscope}%
\pgfsys@transformshift{1.250000in}{0.755556in}%
\pgfsys@useobject{currentmarker}{}%
\end{pgfscope}%
\end{pgfscope}%
\begin{pgfscope}%
\pgfsetbuttcap%
\pgfsetroundjoin%
\definecolor{currentfill}{rgb}{0.000000,0.000000,0.000000}%
\pgfsetfillcolor{currentfill}%
\pgfsetlinewidth{0.501875pt}%
\definecolor{currentstroke}{rgb}{0.000000,0.000000,0.000000}%
\pgfsetstrokecolor{currentstroke}%
\pgfsetdash{}{0pt}%
\pgfsys@defobject{currentmarker}{\pgfqpoint{-0.055556in}{0.000000in}}{\pgfqpoint{0.000000in}{0.000000in}}{%
\pgfpathmoveto{\pgfqpoint{0.000000in}{0.000000in}}%
\pgfpathlineto{\pgfqpoint{-0.055556in}{0.000000in}}%
\pgfusepath{stroke,fill}%
}%
\begin{pgfscope}%
\pgfsys@transformshift{9.000000in}{0.755556in}%
\pgfsys@useobject{currentmarker}{}%
\end{pgfscope}%
\end{pgfscope}%
\begin{pgfscope}%
\pgftext[left,bottom,x=0.789741in,y=0.692527in,rotate=0.000000]{{\sffamily\fontsize{12.000000}{14.400000}\selectfont −1.0}}
%
\end{pgfscope}%
\begin{pgfscope}%
\pgfpathrectangle{\pgfqpoint{1.250000in}{0.400000in}}{\pgfqpoint{7.750000in}{3.200000in}} %
\pgfusepath{clip}%
\pgfsetbuttcap%
\pgfsetroundjoin%
\pgfsetlinewidth{0.501875pt}%
\definecolor{currentstroke}{rgb}{0.000000,0.000000,0.000000}%
\pgfsetstrokecolor{currentstroke}%
\pgfsetdash{{1.000000pt}{3.000000pt}}{0.000000pt}%
\pgfpathmoveto{\pgfqpoint{1.250000in}{1.111111in}}%
\pgfpathlineto{\pgfqpoint{9.000000in}{1.111111in}}%
\pgfusepath{stroke}%
\end{pgfscope}%
\begin{pgfscope}%
\pgfsetbuttcap%
\pgfsetroundjoin%
\definecolor{currentfill}{rgb}{0.000000,0.000000,0.000000}%
\pgfsetfillcolor{currentfill}%
\pgfsetlinewidth{0.501875pt}%
\definecolor{currentstroke}{rgb}{0.000000,0.000000,0.000000}%
\pgfsetstrokecolor{currentstroke}%
\pgfsetdash{}{0pt}%
\pgfsys@defobject{currentmarker}{\pgfqpoint{0.000000in}{0.000000in}}{\pgfqpoint{0.055556in}{0.000000in}}{%
\pgfpathmoveto{\pgfqpoint{0.000000in}{0.000000in}}%
\pgfpathlineto{\pgfqpoint{0.055556in}{0.000000in}}%
\pgfusepath{stroke,fill}%
}%
\begin{pgfscope}%
\pgfsys@transformshift{1.250000in}{1.111111in}%
\pgfsys@useobject{currentmarker}{}%
\end{pgfscope}%
\end{pgfscope}%
\begin{pgfscope}%
\pgfsetbuttcap%
\pgfsetroundjoin%
\definecolor{currentfill}{rgb}{0.000000,0.000000,0.000000}%
\pgfsetfillcolor{currentfill}%
\pgfsetlinewidth{0.501875pt}%
\definecolor{currentstroke}{rgb}{0.000000,0.000000,0.000000}%
\pgfsetstrokecolor{currentstroke}%
\pgfsetdash{}{0pt}%
\pgfsys@defobject{currentmarker}{\pgfqpoint{-0.055556in}{0.000000in}}{\pgfqpoint{0.000000in}{0.000000in}}{%
\pgfpathmoveto{\pgfqpoint{0.000000in}{0.000000in}}%
\pgfpathlineto{\pgfqpoint{-0.055556in}{0.000000in}}%
\pgfusepath{stroke,fill}%
}%
\begin{pgfscope}%
\pgfsys@transformshift{9.000000in}{1.111111in}%
\pgfsys@useobject{currentmarker}{}%
\end{pgfscope}%
\end{pgfscope}%
\begin{pgfscope}%
\pgftext[left,bottom,x=0.789741in,y=1.048082in,rotate=0.000000]{{\sffamily\fontsize{12.000000}{14.400000}\selectfont −0.5}}
%
\end{pgfscope}%
\begin{pgfscope}%
\pgfpathrectangle{\pgfqpoint{1.250000in}{0.400000in}}{\pgfqpoint{7.750000in}{3.200000in}} %
\pgfusepath{clip}%
\pgfsetbuttcap%
\pgfsetroundjoin%
\pgfsetlinewidth{0.501875pt}%
\definecolor{currentstroke}{rgb}{0.000000,0.000000,0.000000}%
\pgfsetstrokecolor{currentstroke}%
\pgfsetdash{{1.000000pt}{3.000000pt}}{0.000000pt}%
\pgfpathmoveto{\pgfqpoint{1.250000in}{1.466667in}}%
\pgfpathlineto{\pgfqpoint{9.000000in}{1.466667in}}%
\pgfusepath{stroke}%
\end{pgfscope}%
\begin{pgfscope}%
\pgfsetbuttcap%
\pgfsetroundjoin%
\definecolor{currentfill}{rgb}{0.000000,0.000000,0.000000}%
\pgfsetfillcolor{currentfill}%
\pgfsetlinewidth{0.501875pt}%
\definecolor{currentstroke}{rgb}{0.000000,0.000000,0.000000}%
\pgfsetstrokecolor{currentstroke}%
\pgfsetdash{}{0pt}%
\pgfsys@defobject{currentmarker}{\pgfqpoint{0.000000in}{0.000000in}}{\pgfqpoint{0.055556in}{0.000000in}}{%
\pgfpathmoveto{\pgfqpoint{0.000000in}{0.000000in}}%
\pgfpathlineto{\pgfqpoint{0.055556in}{0.000000in}}%
\pgfusepath{stroke,fill}%
}%
\begin{pgfscope}%
\pgfsys@transformshift{1.250000in}{1.466667in}%
\pgfsys@useobject{currentmarker}{}%
\end{pgfscope}%
\end{pgfscope}%
\begin{pgfscope}%
\pgfsetbuttcap%
\pgfsetroundjoin%
\definecolor{currentfill}{rgb}{0.000000,0.000000,0.000000}%
\pgfsetfillcolor{currentfill}%
\pgfsetlinewidth{0.501875pt}%
\definecolor{currentstroke}{rgb}{0.000000,0.000000,0.000000}%
\pgfsetstrokecolor{currentstroke}%
\pgfsetdash{}{0pt}%
\pgfsys@defobject{currentmarker}{\pgfqpoint{-0.055556in}{0.000000in}}{\pgfqpoint{0.000000in}{0.000000in}}{%
\pgfpathmoveto{\pgfqpoint{0.000000in}{0.000000in}}%
\pgfpathlineto{\pgfqpoint{-0.055556in}{0.000000in}}%
\pgfusepath{stroke,fill}%
}%
\begin{pgfscope}%
\pgfsys@transformshift{9.000000in}{1.466667in}%
\pgfsys@useobject{currentmarker}{}%
\end{pgfscope}%
\end{pgfscope}%
\begin{pgfscope}%
\pgftext[left,bottom,x=0.929389in,y=1.403638in,rotate=0.000000]{{\sffamily\fontsize{12.000000}{14.400000}\selectfont 0.0}}
%
\end{pgfscope}%
\begin{pgfscope}%
\pgfpathrectangle{\pgfqpoint{1.250000in}{0.400000in}}{\pgfqpoint{7.750000in}{3.200000in}} %
\pgfusepath{clip}%
\pgfsetbuttcap%
\pgfsetroundjoin%
\pgfsetlinewidth{0.501875pt}%
\definecolor{currentstroke}{rgb}{0.000000,0.000000,0.000000}%
\pgfsetstrokecolor{currentstroke}%
\pgfsetdash{{1.000000pt}{3.000000pt}}{0.000000pt}%
\pgfpathmoveto{\pgfqpoint{1.250000in}{1.822222in}}%
\pgfpathlineto{\pgfqpoint{9.000000in}{1.822222in}}%
\pgfusepath{stroke}%
\end{pgfscope}%
\begin{pgfscope}%
\pgfsetbuttcap%
\pgfsetroundjoin%
\definecolor{currentfill}{rgb}{0.000000,0.000000,0.000000}%
\pgfsetfillcolor{currentfill}%
\pgfsetlinewidth{0.501875pt}%
\definecolor{currentstroke}{rgb}{0.000000,0.000000,0.000000}%
\pgfsetstrokecolor{currentstroke}%
\pgfsetdash{}{0pt}%
\pgfsys@defobject{currentmarker}{\pgfqpoint{0.000000in}{0.000000in}}{\pgfqpoint{0.055556in}{0.000000in}}{%
\pgfpathmoveto{\pgfqpoint{0.000000in}{0.000000in}}%
\pgfpathlineto{\pgfqpoint{0.055556in}{0.000000in}}%
\pgfusepath{stroke,fill}%
}%
\begin{pgfscope}%
\pgfsys@transformshift{1.250000in}{1.822222in}%
\pgfsys@useobject{currentmarker}{}%
\end{pgfscope}%
\end{pgfscope}%
\begin{pgfscope}%
\pgfsetbuttcap%
\pgfsetroundjoin%
\definecolor{currentfill}{rgb}{0.000000,0.000000,0.000000}%
\pgfsetfillcolor{currentfill}%
\pgfsetlinewidth{0.501875pt}%
\definecolor{currentstroke}{rgb}{0.000000,0.000000,0.000000}%
\pgfsetstrokecolor{currentstroke}%
\pgfsetdash{}{0pt}%
\pgfsys@defobject{currentmarker}{\pgfqpoint{-0.055556in}{0.000000in}}{\pgfqpoint{0.000000in}{0.000000in}}{%
\pgfpathmoveto{\pgfqpoint{0.000000in}{0.000000in}}%
\pgfpathlineto{\pgfqpoint{-0.055556in}{0.000000in}}%
\pgfusepath{stroke,fill}%
}%
\begin{pgfscope}%
\pgfsys@transformshift{9.000000in}{1.822222in}%
\pgfsys@useobject{currentmarker}{}%
\end{pgfscope}%
\end{pgfscope}%
\begin{pgfscope}%
\pgftext[left,bottom,x=0.929389in,y=1.759193in,rotate=0.000000]{{\sffamily\fontsize{12.000000}{14.400000}\selectfont 0.5}}
%
\end{pgfscope}%
\begin{pgfscope}%
\pgfpathrectangle{\pgfqpoint{1.250000in}{0.400000in}}{\pgfqpoint{7.750000in}{3.200000in}} %
\pgfusepath{clip}%
\pgfsetbuttcap%
\pgfsetroundjoin%
\pgfsetlinewidth{0.501875pt}%
\definecolor{currentstroke}{rgb}{0.000000,0.000000,0.000000}%
\pgfsetstrokecolor{currentstroke}%
\pgfsetdash{{1.000000pt}{3.000000pt}}{0.000000pt}%
\pgfpathmoveto{\pgfqpoint{1.250000in}{2.177778in}}%
\pgfpathlineto{\pgfqpoint{9.000000in}{2.177778in}}%
\pgfusepath{stroke}%
\end{pgfscope}%
\begin{pgfscope}%
\pgfsetbuttcap%
\pgfsetroundjoin%
\definecolor{currentfill}{rgb}{0.000000,0.000000,0.000000}%
\pgfsetfillcolor{currentfill}%
\pgfsetlinewidth{0.501875pt}%
\definecolor{currentstroke}{rgb}{0.000000,0.000000,0.000000}%
\pgfsetstrokecolor{currentstroke}%
\pgfsetdash{}{0pt}%
\pgfsys@defobject{currentmarker}{\pgfqpoint{0.000000in}{0.000000in}}{\pgfqpoint{0.055556in}{0.000000in}}{%
\pgfpathmoveto{\pgfqpoint{0.000000in}{0.000000in}}%
\pgfpathlineto{\pgfqpoint{0.055556in}{0.000000in}}%
\pgfusepath{stroke,fill}%
}%
\begin{pgfscope}%
\pgfsys@transformshift{1.250000in}{2.177778in}%
\pgfsys@useobject{currentmarker}{}%
\end{pgfscope}%
\end{pgfscope}%
\begin{pgfscope}%
\pgfsetbuttcap%
\pgfsetroundjoin%
\definecolor{currentfill}{rgb}{0.000000,0.000000,0.000000}%
\pgfsetfillcolor{currentfill}%
\pgfsetlinewidth{0.501875pt}%
\definecolor{currentstroke}{rgb}{0.000000,0.000000,0.000000}%
\pgfsetstrokecolor{currentstroke}%
\pgfsetdash{}{0pt}%
\pgfsys@defobject{currentmarker}{\pgfqpoint{-0.055556in}{0.000000in}}{\pgfqpoint{0.000000in}{0.000000in}}{%
\pgfpathmoveto{\pgfqpoint{0.000000in}{0.000000in}}%
\pgfpathlineto{\pgfqpoint{-0.055556in}{0.000000in}}%
\pgfusepath{stroke,fill}%
}%
\begin{pgfscope}%
\pgfsys@transformshift{9.000000in}{2.177778in}%
\pgfsys@useobject{currentmarker}{}%
\end{pgfscope}%
\end{pgfscope}%
\begin{pgfscope}%
\pgftext[left,bottom,x=0.929389in,y=2.114749in,rotate=0.000000]{{\sffamily\fontsize{12.000000}{14.400000}\selectfont 1.0}}
%
\end{pgfscope}%
\begin{pgfscope}%
\pgfpathrectangle{\pgfqpoint{1.250000in}{0.400000in}}{\pgfqpoint{7.750000in}{3.200000in}} %
\pgfusepath{clip}%
\pgfsetbuttcap%
\pgfsetroundjoin%
\pgfsetlinewidth{0.501875pt}%
\definecolor{currentstroke}{rgb}{0.000000,0.000000,0.000000}%
\pgfsetstrokecolor{currentstroke}%
\pgfsetdash{{1.000000pt}{3.000000pt}}{0.000000pt}%
\pgfpathmoveto{\pgfqpoint{1.250000in}{2.533333in}}%
\pgfpathlineto{\pgfqpoint{9.000000in}{2.533333in}}%
\pgfusepath{stroke}%
\end{pgfscope}%
\begin{pgfscope}%
\pgfsetbuttcap%
\pgfsetroundjoin%
\definecolor{currentfill}{rgb}{0.000000,0.000000,0.000000}%
\pgfsetfillcolor{currentfill}%
\pgfsetlinewidth{0.501875pt}%
\definecolor{currentstroke}{rgb}{0.000000,0.000000,0.000000}%
\pgfsetstrokecolor{currentstroke}%
\pgfsetdash{}{0pt}%
\pgfsys@defobject{currentmarker}{\pgfqpoint{0.000000in}{0.000000in}}{\pgfqpoint{0.055556in}{0.000000in}}{%
\pgfpathmoveto{\pgfqpoint{0.000000in}{0.000000in}}%
\pgfpathlineto{\pgfqpoint{0.055556in}{0.000000in}}%
\pgfusepath{stroke,fill}%
}%
\begin{pgfscope}%
\pgfsys@transformshift{1.250000in}{2.533333in}%
\pgfsys@useobject{currentmarker}{}%
\end{pgfscope}%
\end{pgfscope}%
\begin{pgfscope}%
\pgfsetbuttcap%
\pgfsetroundjoin%
\definecolor{currentfill}{rgb}{0.000000,0.000000,0.000000}%
\pgfsetfillcolor{currentfill}%
\pgfsetlinewidth{0.501875pt}%
\definecolor{currentstroke}{rgb}{0.000000,0.000000,0.000000}%
\pgfsetstrokecolor{currentstroke}%
\pgfsetdash{}{0pt}%
\pgfsys@defobject{currentmarker}{\pgfqpoint{-0.055556in}{0.000000in}}{\pgfqpoint{0.000000in}{0.000000in}}{%
\pgfpathmoveto{\pgfqpoint{0.000000in}{0.000000in}}%
\pgfpathlineto{\pgfqpoint{-0.055556in}{0.000000in}}%
\pgfusepath{stroke,fill}%
}%
\begin{pgfscope}%
\pgfsys@transformshift{9.000000in}{2.533333in}%
\pgfsys@useobject{currentmarker}{}%
\end{pgfscope}%
\end{pgfscope}%
\begin{pgfscope}%
\pgftext[left,bottom,x=0.929389in,y=2.471403in,rotate=0.000000]{{\sffamily\fontsize{12.000000}{14.400000}\selectfont 1.5}}
%
\end{pgfscope}%
\begin{pgfscope}%
\pgfpathrectangle{\pgfqpoint{1.250000in}{0.400000in}}{\pgfqpoint{7.750000in}{3.200000in}} %
\pgfusepath{clip}%
\pgfsetbuttcap%
\pgfsetroundjoin%
\pgfsetlinewidth{0.501875pt}%
\definecolor{currentstroke}{rgb}{0.000000,0.000000,0.000000}%
\pgfsetstrokecolor{currentstroke}%
\pgfsetdash{{1.000000pt}{3.000000pt}}{0.000000pt}%
\pgfpathmoveto{\pgfqpoint{1.250000in}{2.888889in}}%
\pgfpathlineto{\pgfqpoint{9.000000in}{2.888889in}}%
\pgfusepath{stroke}%
\end{pgfscope}%
\begin{pgfscope}%
\pgfsetbuttcap%
\pgfsetroundjoin%
\definecolor{currentfill}{rgb}{0.000000,0.000000,0.000000}%
\pgfsetfillcolor{currentfill}%
\pgfsetlinewidth{0.501875pt}%
\definecolor{currentstroke}{rgb}{0.000000,0.000000,0.000000}%
\pgfsetstrokecolor{currentstroke}%
\pgfsetdash{}{0pt}%
\pgfsys@defobject{currentmarker}{\pgfqpoint{0.000000in}{0.000000in}}{\pgfqpoint{0.055556in}{0.000000in}}{%
\pgfpathmoveto{\pgfqpoint{0.000000in}{0.000000in}}%
\pgfpathlineto{\pgfqpoint{0.055556in}{0.000000in}}%
\pgfusepath{stroke,fill}%
}%
\begin{pgfscope}%
\pgfsys@transformshift{1.250000in}{2.888889in}%
\pgfsys@useobject{currentmarker}{}%
\end{pgfscope}%
\end{pgfscope}%
\begin{pgfscope}%
\pgfsetbuttcap%
\pgfsetroundjoin%
\definecolor{currentfill}{rgb}{0.000000,0.000000,0.000000}%
\pgfsetfillcolor{currentfill}%
\pgfsetlinewidth{0.501875pt}%
\definecolor{currentstroke}{rgb}{0.000000,0.000000,0.000000}%
\pgfsetstrokecolor{currentstroke}%
\pgfsetdash{}{0pt}%
\pgfsys@defobject{currentmarker}{\pgfqpoint{-0.055556in}{0.000000in}}{\pgfqpoint{0.000000in}{0.000000in}}{%
\pgfpathmoveto{\pgfqpoint{0.000000in}{0.000000in}}%
\pgfpathlineto{\pgfqpoint{-0.055556in}{0.000000in}}%
\pgfusepath{stroke,fill}%
}%
\begin{pgfscope}%
\pgfsys@transformshift{9.000000in}{2.888889in}%
\pgfsys@useobject{currentmarker}{}%
\end{pgfscope}%
\end{pgfscope}%
\begin{pgfscope}%
\pgftext[left,bottom,x=0.929389in,y=2.825860in,rotate=0.000000]{{\sffamily\fontsize{12.000000}{14.400000}\selectfont 2.0}}
%
\end{pgfscope}%
\begin{pgfscope}%
\pgfpathrectangle{\pgfqpoint{1.250000in}{0.400000in}}{\pgfqpoint{7.750000in}{3.200000in}} %
\pgfusepath{clip}%
\pgfsetbuttcap%
\pgfsetroundjoin%
\pgfsetlinewidth{0.501875pt}%
\definecolor{currentstroke}{rgb}{0.000000,0.000000,0.000000}%
\pgfsetstrokecolor{currentstroke}%
\pgfsetdash{{1.000000pt}{3.000000pt}}{0.000000pt}%
\pgfpathmoveto{\pgfqpoint{1.250000in}{3.244444in}}%
\pgfpathlineto{\pgfqpoint{9.000000in}{3.244444in}}%
\pgfusepath{stroke}%
\end{pgfscope}%
\begin{pgfscope}%
\pgfsetbuttcap%
\pgfsetroundjoin%
\definecolor{currentfill}{rgb}{0.000000,0.000000,0.000000}%
\pgfsetfillcolor{currentfill}%
\pgfsetlinewidth{0.501875pt}%
\definecolor{currentstroke}{rgb}{0.000000,0.000000,0.000000}%
\pgfsetstrokecolor{currentstroke}%
\pgfsetdash{}{0pt}%
\pgfsys@defobject{currentmarker}{\pgfqpoint{0.000000in}{0.000000in}}{\pgfqpoint{0.055556in}{0.000000in}}{%
\pgfpathmoveto{\pgfqpoint{0.000000in}{0.000000in}}%
\pgfpathlineto{\pgfqpoint{0.055556in}{0.000000in}}%
\pgfusepath{stroke,fill}%
}%
\begin{pgfscope}%
\pgfsys@transformshift{1.250000in}{3.244444in}%
\pgfsys@useobject{currentmarker}{}%
\end{pgfscope}%
\end{pgfscope}%
\begin{pgfscope}%
\pgfsetbuttcap%
\pgfsetroundjoin%
\definecolor{currentfill}{rgb}{0.000000,0.000000,0.000000}%
\pgfsetfillcolor{currentfill}%
\pgfsetlinewidth{0.501875pt}%
\definecolor{currentstroke}{rgb}{0.000000,0.000000,0.000000}%
\pgfsetstrokecolor{currentstroke}%
\pgfsetdash{}{0pt}%
\pgfsys@defobject{currentmarker}{\pgfqpoint{-0.055556in}{0.000000in}}{\pgfqpoint{0.000000in}{0.000000in}}{%
\pgfpathmoveto{\pgfqpoint{0.000000in}{0.000000in}}%
\pgfpathlineto{\pgfqpoint{-0.055556in}{0.000000in}}%
\pgfusepath{stroke,fill}%
}%
\begin{pgfscope}%
\pgfsys@transformshift{9.000000in}{3.244444in}%
\pgfsys@useobject{currentmarker}{}%
\end{pgfscope}%
\end{pgfscope}%
\begin{pgfscope}%
\pgftext[left,bottom,x=0.929389in,y=3.181416in,rotate=0.000000]{{\sffamily\fontsize{12.000000}{14.400000}\selectfont 2.5}}
%
\end{pgfscope}%
\begin{pgfscope}%
\pgfpathrectangle{\pgfqpoint{1.250000in}{0.400000in}}{\pgfqpoint{7.750000in}{3.200000in}} %
\pgfusepath{clip}%
\pgfsetbuttcap%
\pgfsetroundjoin%
\pgfsetlinewidth{0.501875pt}%
\definecolor{currentstroke}{rgb}{0.000000,0.000000,0.000000}%
\pgfsetstrokecolor{currentstroke}%
\pgfsetdash{{1.000000pt}{3.000000pt}}{0.000000pt}%
\pgfpathmoveto{\pgfqpoint{1.250000in}{3.600000in}}%
\pgfpathlineto{\pgfqpoint{9.000000in}{3.600000in}}%
\pgfusepath{stroke}%
\end{pgfscope}%
\begin{pgfscope}%
\pgfsetbuttcap%
\pgfsetroundjoin%
\definecolor{currentfill}{rgb}{0.000000,0.000000,0.000000}%
\pgfsetfillcolor{currentfill}%
\pgfsetlinewidth{0.501875pt}%
\definecolor{currentstroke}{rgb}{0.000000,0.000000,0.000000}%
\pgfsetstrokecolor{currentstroke}%
\pgfsetdash{}{0pt}%
\pgfsys@defobject{currentmarker}{\pgfqpoint{0.000000in}{0.000000in}}{\pgfqpoint{0.055556in}{0.000000in}}{%
\pgfpathmoveto{\pgfqpoint{0.000000in}{0.000000in}}%
\pgfpathlineto{\pgfqpoint{0.055556in}{0.000000in}}%
\pgfusepath{stroke,fill}%
}%
\begin{pgfscope}%
\pgfsys@transformshift{1.250000in}{3.600000in}%
\pgfsys@useobject{currentmarker}{}%
\end{pgfscope}%
\end{pgfscope}%
\begin{pgfscope}%
\pgfsetbuttcap%
\pgfsetroundjoin%
\definecolor{currentfill}{rgb}{0.000000,0.000000,0.000000}%
\pgfsetfillcolor{currentfill}%
\pgfsetlinewidth{0.501875pt}%
\definecolor{currentstroke}{rgb}{0.000000,0.000000,0.000000}%
\pgfsetstrokecolor{currentstroke}%
\pgfsetdash{}{0pt}%
\pgfsys@defobject{currentmarker}{\pgfqpoint{-0.055556in}{0.000000in}}{\pgfqpoint{0.000000in}{0.000000in}}{%
\pgfpathmoveto{\pgfqpoint{0.000000in}{0.000000in}}%
\pgfpathlineto{\pgfqpoint{-0.055556in}{0.000000in}}%
\pgfusepath{stroke,fill}%
}%
\begin{pgfscope}%
\pgfsys@transformshift{9.000000in}{3.600000in}%
\pgfsys@useobject{currentmarker}{}%
\end{pgfscope}%
\end{pgfscope}%
\begin{pgfscope}%
\pgftext[left,bottom,x=0.929389in,y=3.536971in,rotate=0.000000]{{\sffamily\fontsize{12.000000}{14.400000}\selectfont 3.0}}
%
\end{pgfscope}%
\begin{pgfscope}%
\pgfsetrectcap%
\pgfsetroundjoin%
\pgfsetlinewidth{1.003750pt}%
\definecolor{currentstroke}{rgb}{0.000000,0.000000,0.000000}%
\pgfsetstrokecolor{currentstroke}%
\pgfsetdash{}{0pt}%
\pgfpathmoveto{\pgfqpoint{1.250000in}{3.600000in}}%
\pgfpathlineto{\pgfqpoint{9.000000in}{3.600000in}}%
\pgfusepath{stroke}%
\end{pgfscope}%
\begin{pgfscope}%
\pgfsetrectcap%
\pgfsetroundjoin%
\pgfsetlinewidth{1.003750pt}%
\definecolor{currentstroke}{rgb}{0.000000,0.000000,0.000000}%
\pgfsetstrokecolor{currentstroke}%
\pgfsetdash{}{0pt}%
\pgfpathmoveto{\pgfqpoint{9.000000in}{0.400000in}}%
\pgfpathlineto{\pgfqpoint{9.000000in}{3.600000in}}%
\pgfusepath{stroke}%
\end{pgfscope}%
\begin{pgfscope}%
\pgfsetrectcap%
\pgfsetroundjoin%
\pgfsetlinewidth{1.003750pt}%
\definecolor{currentstroke}{rgb}{0.000000,0.000000,0.000000}%
\pgfsetstrokecolor{currentstroke}%
\pgfsetdash{}{0pt}%
\pgfpathmoveto{\pgfqpoint{1.250000in}{0.400000in}}%
\pgfpathlineto{\pgfqpoint{9.000000in}{0.400000in}}%
\pgfusepath{stroke}%
\end{pgfscope}%
\begin{pgfscope}%
\pgfsetrectcap%
\pgfsetroundjoin%
\pgfsetlinewidth{1.003750pt}%
\definecolor{currentstroke}{rgb}{0.000000,0.000000,0.000000}%
\pgfsetstrokecolor{currentstroke}%
\pgfsetdash{}{0pt}%
\pgfpathmoveto{\pgfqpoint{1.250000in}{0.400000in}}%
\pgfpathlineto{\pgfqpoint{1.250000in}{3.600000in}}%
\pgfusepath{stroke}%
\end{pgfscope}%
\begin{pgfscope}%
\pgftext[left,bottom,x=3.785743in,y=3.646202in,rotate=0.000000]{{\sffamily\fontsize{14.400000}{17.280000}\selectfont euler scheme, dt=0.0001a}}
%
\end{pgfscope}%
\begin{pgfscope}%
\pgfsetrectcap%
\pgfsetroundjoin%
\definecolor{currentfill}{rgb}{1.000000,1.000000,1.000000}%
\pgfsetfillcolor{currentfill}%
\pgfsetlinewidth{1.003750pt}%
\definecolor{currentstroke}{rgb}{0.000000,0.000000,0.000000}%
\pgfsetstrokecolor{currentstroke}%
\pgfsetdash{}{0pt}%
\pgfpathmoveto{\pgfqpoint{4.648818in}{2.266278in}}%
\pgfpathlineto{\pgfqpoint{5.601182in}{2.266278in}}%
\pgfpathlineto{\pgfqpoint{5.601182in}{3.530583in}}%
\pgfpathlineto{\pgfqpoint{4.648818in}{3.530583in}}%
\pgfpathlineto{\pgfqpoint{4.648818in}{2.266278in}}%
\pgfpathclose%
\pgfusepath{stroke,fill}%
\end{pgfscope}%
\begin{pgfscope}%
\pgfsetrectcap%
\pgfsetroundjoin%
\pgfsetlinewidth{1.003750pt}%
\definecolor{currentstroke}{rgb}{0.000000,0.000000,1.000000}%
\pgfsetstrokecolor{currentstroke}%
\pgfsetdash{}{0pt}%
\pgfpathmoveto{\pgfqpoint{4.746001in}{3.418161in}}%
\pgfpathlineto{\pgfqpoint{4.940368in}{3.418161in}}%
\pgfusepath{stroke}%
\end{pgfscope}%
\begin{pgfscope}%
\pgftext[left,bottom,x=5.093085in,y=3.367603in,rotate=0.000000]{{\sffamily\fontsize{9.996000}{11.995200}\selectfont Sun}}
%
\end{pgfscope}%
\begin{pgfscope}%
\pgfsetrectcap%
\pgfsetroundjoin%
\pgfsetlinewidth{1.003750pt}%
\definecolor{currentstroke}{rgb}{0.000000,0.500000,0.000000}%
\pgfsetstrokecolor{currentstroke}%
\pgfsetdash{}{0pt}%
\pgfpathmoveto{\pgfqpoint{4.746001in}{3.214385in}}%
\pgfpathlineto{\pgfqpoint{4.940368in}{3.214385in}}%
\pgfusepath{stroke}%
\end{pgfscope}%
\begin{pgfscope}%
\pgftext[left,bottom,x=5.093085in,y=3.163828in,rotate=0.000000]{{\sffamily\fontsize{9.996000}{11.995200}\selectfont Earth}}
%
\end{pgfscope}%
\begin{pgfscope}%
\pgfsetrectcap%
\pgfsetroundjoin%
\pgfsetlinewidth{1.003750pt}%
\definecolor{currentstroke}{rgb}{1.000000,0.000000,0.000000}%
\pgfsetstrokecolor{currentstroke}%
\pgfsetdash{}{0pt}%
\pgfpathmoveto{\pgfqpoint{4.746001in}{3.010609in}}%
\pgfpathlineto{\pgfqpoint{4.940368in}{3.010609in}}%
\pgfusepath{stroke}%
\end{pgfscope}%
\begin{pgfscope}%
\pgftext[left,bottom,x=5.093085in,y=2.960052in,rotate=0.000000]{{\sffamily\fontsize{9.996000}{11.995200}\selectfont Moon}}
%
\end{pgfscope}%
\begin{pgfscope}%
\pgfsetrectcap%
\pgfsetroundjoin%
\pgfsetlinewidth{1.003750pt}%
\definecolor{currentstroke}{rgb}{0.000000,0.750000,0.750000}%
\pgfsetstrokecolor{currentstroke}%
\pgfsetdash{}{0pt}%
\pgfpathmoveto{\pgfqpoint{4.746001in}{2.806833in}}%
\pgfpathlineto{\pgfqpoint{4.940368in}{2.806833in}}%
\pgfusepath{stroke}%
\end{pgfscope}%
\begin{pgfscope}%
\pgftext[left,bottom,x=5.093085in,y=2.756276in,rotate=0.000000]{{\sffamily\fontsize{9.996000}{11.995200}\selectfont Mars}}
%
\end{pgfscope}%
\begin{pgfscope}%
\pgfsetrectcap%
\pgfsetroundjoin%
\pgfsetlinewidth{1.003750pt}%
\definecolor{currentstroke}{rgb}{0.750000,0.000000,0.750000}%
\pgfsetstrokecolor{currentstroke}%
\pgfsetdash{}{0pt}%
\pgfpathmoveto{\pgfqpoint{4.746001in}{2.603057in}}%
\pgfpathlineto{\pgfqpoint{4.940368in}{2.603057in}}%
\pgfusepath{stroke}%
\end{pgfscope}%
\begin{pgfscope}%
\pgftext[left,bottom,x=5.093085in,y=2.552500in,rotate=0.000000]{{\sffamily\fontsize{9.996000}{11.995200}\selectfont Venus}}
%
\end{pgfscope}%
\begin{pgfscope}%
\pgfsetrectcap%
\pgfsetroundjoin%
\pgfsetlinewidth{1.003750pt}%
\definecolor{currentstroke}{rgb}{0.750000,0.750000,0.000000}%
\pgfsetstrokecolor{currentstroke}%
\pgfsetdash{}{0pt}%
\pgfpathmoveto{\pgfqpoint{4.746001in}{2.399282in}}%
\pgfpathlineto{\pgfqpoint{4.940368in}{2.399282in}}%
\pgfusepath{stroke}%
\end{pgfscope}%
\begin{pgfscope}%
\pgftext[left,bottom,x=5.093085in,y=2.321812in,rotate=0.000000]{{\sffamily\fontsize{9.996000}{11.995200}\selectfont Jupiter}}
%
\end{pgfscope}%
\end{pgfpicture}%
\makeatother%
\endgroup%
}
	\caption{Simulation of a part of the solar system using the Euler integration scheme}\label{fig:s1}
\end{figure}

The simulated trajectories are as expected nearly ellical (see figure \ref{fig:s1}). Now we take a closer look on the trajectory of the moon in the rest frame of the earth for different time steps \verb dt. Therefor we run the simulation of the solar system for the time steps \verb dt {}= 0.0001, \verb dt = 0.001 and \verb dt {}= 0.01. Then for all simulations the difference of the location of the moon an the location of the earth are calculated in order to get the trajectory of the moon relative to the earth. The resulting plot is shown in figure \ref{fig:s2}.

\begin{figure}[H]
	\resizebox{1\textwidth}{!}{%% Creator: Matplotlib, PGF backend
%%
%% To include the figure in your LaTeX document, write
%%   \input{<filename>.pgf}
%%
%% Make sure the required packages are loaded in your preamble
%%   \usepackage{pgf}
%%
%% Figures using additional raster images can only be included by \input if
%% they are in the same directory as the main LaTeX file. For loading figures
%% from other directories you can use the `import` package
%%   \usepackage{import}
%% and then include the figures with
%%   \import{<path to file>}{<filename>.pgf}
%%
%% Matplotlib used the following preamble
%%   \usepackage{fontspec}
%%   \setmainfont{DejaVu Serif}
%%   \setsansfont{DejaVu Sans}
%%   \setmonofont{DejaVu Sans Mono}
%%
\begingroup%
\makeatletter%
\begin{pgfpicture}%
\pgfpathrectangle{\pgfpointorigin}{\pgfqpoint{10.000000in}{4.000000in}}%
\pgfusepath{use as bounding box}%
\begin{pgfscope}%
\pgfsetrectcap%
\pgfsetroundjoin%
\definecolor{currentfill}{rgb}{1.000000,1.000000,1.000000}%
\pgfsetfillcolor{currentfill}%
\pgfsetlinewidth{0.000000pt}%
\definecolor{currentstroke}{rgb}{1.000000,1.000000,1.000000}%
\pgfsetstrokecolor{currentstroke}%
\pgfsetdash{}{0pt}%
\pgfpathmoveto{\pgfqpoint{0.000000in}{0.000000in}}%
\pgfpathlineto{\pgfqpoint{10.000000in}{0.000000in}}%
\pgfpathlineto{\pgfqpoint{10.000000in}{4.000000in}}%
\pgfpathlineto{\pgfqpoint{0.000000in}{4.000000in}}%
\pgfpathclose%
\pgfusepath{fill}%
\end{pgfscope}%
\begin{pgfscope}%
\pgfsetrectcap%
\pgfsetroundjoin%
\definecolor{currentfill}{rgb}{1.000000,1.000000,1.000000}%
\pgfsetfillcolor{currentfill}%
\pgfsetlinewidth{0.000000pt}%
\definecolor{currentstroke}{rgb}{0.000000,0.000000,0.000000}%
\pgfsetstrokecolor{currentstroke}%
\pgfsetdash{}{0pt}%
\pgfpathmoveto{\pgfqpoint{1.250000in}{0.400000in}}%
\pgfpathlineto{\pgfqpoint{9.000000in}{0.400000in}}%
\pgfpathlineto{\pgfqpoint{9.000000in}{3.600000in}}%
\pgfpathlineto{\pgfqpoint{1.250000in}{3.600000in}}%
\pgfpathclose%
\pgfusepath{fill}%
\end{pgfscope}%
\begin{pgfscope}%
\pgfpathrectangle{\pgfqpoint{1.250000in}{0.400000in}}{\pgfqpoint{7.750000in}{3.200000in}} %
\pgfusepath{clip}%
\pgfsetrectcap%
\pgfsetroundjoin%
\pgfsetlinewidth{1.003750pt}%
\definecolor{currentstroke}{rgb}{0.000000,0.000000,1.000000}%
\pgfsetstrokecolor{currentstroke}%
\pgfsetdash{}{0pt}%
\pgfpathmoveto{\pgfqpoint{7.607583in}{2.000000in}}%
\pgfpathlineto{\pgfqpoint{7.605835in}{2.033941in}}%
\pgfpathlineto{\pgfqpoint{7.601289in}{2.067844in}}%
\pgfpathlineto{\pgfqpoint{7.593949in}{2.101668in}}%
\pgfpathlineto{\pgfqpoint{7.583822in}{2.135376in}}%
\pgfpathlineto{\pgfqpoint{7.570918in}{2.168929in}}%
\pgfpathlineto{\pgfqpoint{7.555248in}{2.202288in}}%
\pgfpathlineto{\pgfqpoint{7.536827in}{2.235414in}}%
\pgfpathlineto{\pgfqpoint{7.515675in}{2.268270in}}%
\pgfpathlineto{\pgfqpoint{7.491811in}{2.300816in}}%
\pgfpathlineto{\pgfqpoint{7.465260in}{2.333016in}}%
\pgfpathlineto{\pgfqpoint{7.436048in}{2.364831in}}%
\pgfpathlineto{\pgfqpoint{7.404206in}{2.396225in}}%
\pgfpathlineto{\pgfqpoint{7.369765in}{2.427160in}}%
\pgfpathlineto{\pgfqpoint{7.332761in}{2.457599in}}%
\pgfpathlineto{\pgfqpoint{7.293231in}{2.487507in}}%
\pgfpathlineto{\pgfqpoint{7.251218in}{2.516848in}}%
\pgfpathlineto{\pgfqpoint{7.206764in}{2.545586in}}%
\pgfpathlineto{\pgfqpoint{7.159916in}{2.573687in}}%
\pgfpathlineto{\pgfqpoint{7.098064in}{2.607864in}}%
\pgfpathlineto{\pgfqpoint{7.032650in}{2.640927in}}%
\pgfpathlineto{\pgfqpoint{6.963782in}{2.672811in}}%
\pgfpathlineto{\pgfqpoint{6.891575in}{2.703454in}}%
\pgfpathlineto{\pgfqpoint{6.816150in}{2.732795in}}%
\pgfpathlineto{\pgfqpoint{6.737635in}{2.760775in}}%
\pgfpathlineto{\pgfqpoint{6.656162in}{2.787337in}}%
\pgfpathlineto{\pgfqpoint{6.571871in}{2.812426in}}%
\pgfpathlineto{\pgfqpoint{6.484906in}{2.835990in}}%
\pgfpathlineto{\pgfqpoint{6.395419in}{2.857976in}}%
\pgfpathlineto{\pgfqpoint{6.303566in}{2.878338in}}%
\pgfpathlineto{\pgfqpoint{6.209509in}{2.897028in}}%
\pgfpathlineto{\pgfqpoint{6.113415in}{2.914005in}}%
\pgfpathlineto{\pgfqpoint{6.015455in}{2.929225in}}%
\pgfpathlineto{\pgfqpoint{5.915808in}{2.942652in}}%
\pgfpathlineto{\pgfqpoint{5.814656in}{2.954250in}}%
\pgfpathlineto{\pgfqpoint{5.712184in}{2.963987in}}%
\pgfpathlineto{\pgfqpoint{5.608584in}{2.971833in}}%
\pgfpathlineto{\pgfqpoint{5.504051in}{2.977762in}}%
\pgfpathlineto{\pgfqpoint{5.398784in}{2.981750in}}%
\pgfpathlineto{\pgfqpoint{5.292987in}{2.983779in}}%
\pgfpathlineto{\pgfqpoint{5.186865in}{2.983833in}}%
\pgfpathlineto{\pgfqpoint{5.080628in}{2.981897in}}%
\pgfpathlineto{\pgfqpoint{4.974488in}{2.977965in}}%
\pgfpathlineto{\pgfqpoint{4.868661in}{2.972029in}}%
\pgfpathlineto{\pgfqpoint{4.763363in}{2.964091in}}%
\pgfpathlineto{\pgfqpoint{4.658815in}{2.954151in}}%
\pgfpathlineto{\pgfqpoint{4.555236in}{2.942218in}}%
\pgfpathlineto{\pgfqpoint{4.452848in}{2.928303in}}%
\pgfpathlineto{\pgfqpoint{4.351874in}{2.912420in}}%
\pgfpathlineto{\pgfqpoint{4.252536in}{2.894591in}}%
\pgfpathlineto{\pgfqpoint{4.155056in}{2.874839in}}%
\pgfpathlineto{\pgfqpoint{4.059654in}{2.853194in}}%
\pgfpathlineto{\pgfqpoint{3.966552in}{2.829690in}}%
\pgfpathlineto{\pgfqpoint{3.875965in}{2.804365in}}%
\pgfpathlineto{\pgfqpoint{3.788110in}{2.777261in}}%
\pgfpathlineto{\pgfqpoint{3.703197in}{2.748428in}}%
\pgfpathlineto{\pgfqpoint{3.621434in}{2.717918in}}%
\pgfpathlineto{\pgfqpoint{3.543023in}{2.685788in}}%
\pgfpathlineto{\pgfqpoint{3.482841in}{2.658960in}}%
\pgfpathlineto{\pgfqpoint{3.425029in}{2.631169in}}%
\pgfpathlineto{\pgfqpoint{3.369681in}{2.602451in}}%
\pgfpathlineto{\pgfqpoint{3.316891in}{2.572844in}}%
\pgfpathlineto{\pgfqpoint{3.266747in}{2.542387in}}%
\pgfpathlineto{\pgfqpoint{3.219334in}{2.511123in}}%
\pgfpathlineto{\pgfqpoint{3.174734in}{2.479093in}}%
\pgfpathlineto{\pgfqpoint{3.133024in}{2.446343in}}%
\pgfpathlineto{\pgfqpoint{3.094277in}{2.412920in}}%
\pgfpathlineto{\pgfqpoint{3.058562in}{2.378870in}}%
\pgfpathlineto{\pgfqpoint{3.025943in}{2.344244in}}%
\pgfpathlineto{\pgfqpoint{2.996479in}{2.309092in}}%
\pgfpathlineto{\pgfqpoint{2.970222in}{2.273465in}}%
\pgfpathlineto{\pgfqpoint{2.947222in}{2.237418in}}%
\pgfpathlineto{\pgfqpoint{2.932135in}{2.210139in}}%
\pgfpathlineto{\pgfqpoint{2.918920in}{2.182676in}}%
\pgfpathlineto{\pgfqpoint{2.907590in}{2.155054in}}%
\pgfpathlineto{\pgfqpoint{2.898158in}{2.127295in}}%
\pgfpathlineto{\pgfqpoint{2.890635in}{2.099425in}}%
\pgfpathlineto{\pgfqpoint{2.885028in}{2.071466in}}%
\pgfpathlineto{\pgfqpoint{2.881344in}{2.043442in}}%
\pgfpathlineto{\pgfqpoint{2.879587in}{2.015380in}}%
\pgfpathlineto{\pgfqpoint{2.879759in}{1.987301in}}%
\pgfpathlineto{\pgfqpoint{2.881862in}{1.959232in}}%
\pgfpathlineto{\pgfqpoint{2.885894in}{1.931196in}}%
\pgfpathlineto{\pgfqpoint{2.891850in}{1.903217in}}%
\pgfpathlineto{\pgfqpoint{2.899726in}{1.875321in}}%
\pgfpathlineto{\pgfqpoint{2.909513in}{1.847531in}}%
\pgfpathlineto{\pgfqpoint{2.921203in}{1.819871in}}%
\pgfpathlineto{\pgfqpoint{2.934784in}{1.792365in}}%
\pgfpathlineto{\pgfqpoint{2.950242in}{1.765037in}}%
\pgfpathlineto{\pgfqpoint{2.967562in}{1.737910in}}%
\pgfpathlineto{\pgfqpoint{2.993522in}{1.702095in}}%
\pgfpathlineto{\pgfqpoint{3.022715in}{1.666733in}}%
\pgfpathlineto{\pgfqpoint{3.055090in}{1.631878in}}%
\pgfpathlineto{\pgfqpoint{3.090590in}{1.597582in}}%
\pgfpathlineto{\pgfqpoint{3.129152in}{1.563895in}}%
\pgfpathlineto{\pgfqpoint{3.170708in}{1.530867in}}%
\pgfpathlineto{\pgfqpoint{3.215187in}{1.498546in}}%
\pgfpathlineto{\pgfqpoint{3.262510in}{1.466978in}}%
\pgfpathlineto{\pgfqpoint{3.312596in}{1.436207in}}%
\pgfpathlineto{\pgfqpoint{3.365360in}{1.406278in}}%
\pgfpathlineto{\pgfqpoint{3.420711in}{1.377229in}}%
\pgfpathlineto{\pgfqpoint{3.478557in}{1.349102in}}%
\pgfpathlineto{\pgfqpoint{3.538800in}{1.321933in}}%
\pgfpathlineto{\pgfqpoint{3.601341in}{1.295757in}}%
\pgfpathlineto{\pgfqpoint{3.682592in}{1.264485in}}%
\pgfpathlineto{\pgfqpoint{3.767067in}{1.234878in}}%
\pgfpathlineto{\pgfqpoint{3.854555in}{1.206991in}}%
\pgfpathlineto{\pgfqpoint{3.944840in}{1.180875in}}%
\pgfpathlineto{\pgfqpoint{4.037703in}{1.156576in}}%
\pgfpathlineto{\pgfqpoint{4.132921in}{1.134134in}}%
\pgfpathlineto{\pgfqpoint{4.230269in}{1.113584in}}%
\pgfpathlineto{\pgfqpoint{4.329523in}{1.094956in}}%
\pgfpathlineto{\pgfqpoint{4.430454in}{1.078278in}}%
\pgfpathlineto{\pgfqpoint{4.532835in}{1.063568in}}%
\pgfpathlineto{\pgfqpoint{4.636440in}{1.050844in}}%
\pgfpathlineto{\pgfqpoint{4.741043in}{1.040118in}}%
\pgfpathlineto{\pgfqpoint{4.846419in}{1.031396in}}%
\pgfpathlineto{\pgfqpoint{4.952346in}{1.024681in}}%
\pgfpathlineto{\pgfqpoint{5.058604in}{1.019972in}}%
\pgfpathlineto{\pgfqpoint{5.164976in}{1.017264in}}%
\pgfpathlineto{\pgfqpoint{5.271246in}{1.016548in}}%
\pgfpathlineto{\pgfqpoint{5.377205in}{1.017811in}}%
\pgfpathlineto{\pgfqpoint{5.482645in}{1.021037in}}%
\pgfpathlineto{\pgfqpoint{5.587363in}{1.026205in}}%
\pgfpathlineto{\pgfqpoint{5.691159in}{1.033292in}}%
\pgfpathlineto{\pgfqpoint{5.793840in}{1.042273in}}%
\pgfpathlineto{\pgfqpoint{5.895215in}{1.053116in}}%
\pgfpathlineto{\pgfqpoint{5.995100in}{1.065792in}}%
\pgfpathlineto{\pgfqpoint{6.093313in}{1.080263in}}%
\pgfpathlineto{\pgfqpoint{6.189680in}{1.096492in}}%
\pgfpathlineto{\pgfqpoint{6.284031in}{1.114440in}}%
\pgfpathlineto{\pgfqpoint{6.376200in}{1.134063in}}%
\pgfpathlineto{\pgfqpoint{6.466029in}{1.155316in}}%
\pgfpathlineto{\pgfqpoint{6.553364in}{1.178152in}}%
\pgfpathlineto{\pgfqpoint{6.638055in}{1.202522in}}%
\pgfpathlineto{\pgfqpoint{6.719960in}{1.228373in}}%
\pgfpathlineto{\pgfqpoint{6.798942in}{1.255654in}}%
\pgfpathlineto{\pgfqpoint{6.874869in}{1.284307in}}%
\pgfpathlineto{\pgfqpoint{6.947616in}{1.314277in}}%
\pgfpathlineto{\pgfqpoint{7.017062in}{1.345505in}}%
\pgfpathlineto{\pgfqpoint{7.083094in}{1.377930in}}%
\pgfpathlineto{\pgfqpoint{7.145603in}{1.411491in}}%
\pgfpathlineto{\pgfqpoint{7.204488in}{1.446124in}}%
\pgfpathlineto{\pgfqpoint{7.248921in}{1.474559in}}%
\pgfpathlineto{\pgfqpoint{7.290927in}{1.503606in}}%
\pgfpathlineto{\pgfqpoint{7.330462in}{1.533230in}}%
\pgfpathlineto{\pgfqpoint{7.367484in}{1.563396in}}%
\pgfpathlineto{\pgfqpoint{7.401954in}{1.594069in}}%
\pgfpathlineto{\pgfqpoint{7.433838in}{1.625214in}}%
\pgfpathlineto{\pgfqpoint{7.463101in}{1.656795in}}%
\pgfpathlineto{\pgfqpoint{7.489712in}{1.688776in}}%
\pgfpathlineto{\pgfqpoint{7.513644in}{1.721119in}}%
\pgfpathlineto{\pgfqpoint{7.534871in}{1.753788in}}%
\pgfpathlineto{\pgfqpoint{7.553370in}{1.786745in}}%
\pgfpathlineto{\pgfqpoint{7.569121in}{1.819953in}}%
\pgfpathlineto{\pgfqpoint{7.582108in}{1.853374in}}%
\pgfpathlineto{\pgfqpoint{7.592315in}{1.886969in}}%
\pgfpathlineto{\pgfqpoint{7.599731in}{1.920701in}}%
\pgfpathlineto{\pgfqpoint{7.604347in}{1.954531in}}%
\pgfpathlineto{\pgfqpoint{7.606157in}{1.988420in}}%
\pgfpathlineto{\pgfqpoint{7.605158in}{2.022330in}}%
\pgfpathlineto{\pgfqpoint{7.601349in}{2.056222in}}%
\pgfpathlineto{\pgfqpoint{7.594734in}{2.090057in}}%
\pgfpathlineto{\pgfqpoint{7.585317in}{2.123796in}}%
\pgfpathlineto{\pgfqpoint{7.573107in}{2.157400in}}%
\pgfpathlineto{\pgfqpoint{7.558116in}{2.190831in}}%
\pgfpathlineto{\pgfqpoint{7.540357in}{2.224050in}}%
\pgfpathlineto{\pgfqpoint{7.519847in}{2.257018in}}%
\pgfpathlineto{\pgfqpoint{7.496608in}{2.289697in}}%
\pgfpathlineto{\pgfqpoint{7.470661in}{2.322048in}}%
\pgfpathlineto{\pgfqpoint{7.442033in}{2.354035in}}%
\pgfpathlineto{\pgfqpoint{7.410753in}{2.385618in}}%
\pgfpathlineto{\pgfqpoint{7.376852in}{2.416761in}}%
\pgfpathlineto{\pgfqpoint{7.340366in}{2.447427in}}%
\pgfpathlineto{\pgfqpoint{7.301333in}{2.477578in}}%
\pgfpathlineto{\pgfqpoint{7.259792in}{2.507179in}}%
\pgfpathlineto{\pgfqpoint{7.215788in}{2.536193in}}%
\pgfpathlineto{\pgfqpoint{7.169366in}{2.564586in}}%
\pgfpathlineto{\pgfqpoint{7.108015in}{2.599150in}}%
\pgfpathlineto{\pgfqpoint{7.043067in}{2.632622in}}%
\pgfpathlineto{\pgfqpoint{6.974631in}{2.664936in}}%
\pgfpathlineto{\pgfqpoint{6.902823in}{2.696030in}}%
\pgfpathlineto{\pgfqpoint{6.827765in}{2.725841in}}%
\pgfpathlineto{\pgfqpoint{6.749587in}{2.754309in}}%
\pgfpathlineto{\pgfqpoint{6.668423in}{2.781378in}}%
\pgfpathlineto{\pgfqpoint{6.584415in}{2.806990in}}%
\pgfpathlineto{\pgfqpoint{6.497711in}{2.831092in}}%
\pgfpathlineto{\pgfqpoint{6.408463in}{2.853634in}}%
\pgfpathlineto{\pgfqpoint{6.316831in}{2.874566in}}%
\pgfpathlineto{\pgfqpoint{6.222979in}{2.893842in}}%
\pgfpathlineto{\pgfqpoint{6.127077in}{2.911419in}}%
\pgfpathlineto{\pgfqpoint{6.029299in}{2.927257in}}%
\pgfpathlineto{\pgfqpoint{5.929825in}{2.941317in}}%
\pgfpathlineto{\pgfqpoint{5.828839in}{2.953564in}}%
\pgfpathlineto{\pgfqpoint{5.726528in}{2.963966in}}%
\pgfpathlineto{\pgfqpoint{5.623085in}{2.972496in}}%
\pgfpathlineto{\pgfqpoint{5.518707in}{2.979128in}}%
\pgfpathlineto{\pgfqpoint{5.413592in}{2.983840in}}%
\pgfpathlineto{\pgfqpoint{5.307943in}{2.986612in}}%
\pgfpathlineto{\pgfqpoint{5.201966in}{2.987431in}}%
\pgfpathlineto{\pgfqpoint{5.095868in}{2.986286in}}%
\pgfpathlineto{\pgfqpoint{4.989859in}{2.983167in}}%
\pgfpathlineto{\pgfqpoint{4.884153in}{2.978072in}}%
\pgfpathlineto{\pgfqpoint{4.778961in}{2.971000in}}%
\pgfpathlineto{\pgfqpoint{4.674499in}{2.961955in}}%
\pgfpathlineto{\pgfqpoint{4.570982in}{2.950945in}}%
\pgfpathlineto{\pgfqpoint{4.468625in}{2.937982in}}%
\pgfpathlineto{\pgfqpoint{4.367645in}{2.923082in}}%
\pgfpathlineto{\pgfqpoint{4.268255in}{2.906264in}}%
\pgfpathlineto{\pgfqpoint{4.170671in}{2.887553in}}%
\pgfpathlineto{\pgfqpoint{4.075104in}{2.866978in}}%
\pgfpathlineto{\pgfqpoint{3.981765in}{2.844570in}}%
\pgfpathlineto{\pgfqpoint{3.890863in}{2.820367in}}%
\pgfpathlineto{\pgfqpoint{3.802603in}{2.794411in}}%
\pgfpathlineto{\pgfqpoint{3.717187in}{2.766746in}}%
\pgfpathlineto{\pgfqpoint{3.634812in}{2.737421in}}%
\pgfpathlineto{\pgfqpoint{3.555674in}{2.706492in}}%
\pgfpathlineto{\pgfqpoint{3.479959in}{2.674016in}}%
\pgfpathlineto{\pgfqpoint{3.421975in}{2.646963in}}%
\pgfpathlineto{\pgfqpoint{3.366391in}{2.618994in}}%
\pgfpathlineto{\pgfqpoint{3.313293in}{2.590144in}}%
\pgfpathlineto{\pgfqpoint{3.262767in}{2.560449in}}%
\pgfpathlineto{\pgfqpoint{3.214896in}{2.529950in}}%
\pgfpathlineto{\pgfqpoint{3.169757in}{2.498685in}}%
\pgfpathlineto{\pgfqpoint{3.127425in}{2.466697in}}%
\pgfpathlineto{\pgfqpoint{3.087971in}{2.434028in}}%
\pgfpathlineto{\pgfqpoint{3.051464in}{2.400724in}}%
\pgfpathlineto{\pgfqpoint{3.017966in}{2.366830in}}%
\pgfpathlineto{\pgfqpoint{2.987535in}{2.332393in}}%
\pgfpathlineto{\pgfqpoint{2.960227in}{2.297462in}}%
\pgfpathlineto{\pgfqpoint{2.936091in}{2.262086in}}%
\pgfpathlineto{\pgfqpoint{2.915172in}{2.226317in}}%
\pgfpathlineto{\pgfqpoint{2.901618in}{2.199262in}}%
\pgfpathlineto{\pgfqpoint{2.889911in}{2.172036in}}%
\pgfpathlineto{\pgfqpoint{2.880064in}{2.144663in}}%
\pgfpathlineto{\pgfqpoint{2.872089in}{2.117163in}}%
\pgfpathlineto{\pgfqpoint{2.865995in}{2.089561in}}%
\pgfpathlineto{\pgfqpoint{2.861791in}{2.061879in}}%
\pgfpathlineto{\pgfqpoint{2.859482in}{2.034140in}}%
\pgfpathlineto{\pgfqpoint{2.859075in}{2.006368in}}%
\pgfpathlineto{\pgfqpoint{2.860570in}{1.978585in}}%
\pgfpathlineto{\pgfqpoint{2.863969in}{1.950816in}}%
\pgfpathlineto{\pgfqpoint{2.869272in}{1.923083in}}%
\pgfpathlineto{\pgfqpoint{2.876474in}{1.895410in}}%
\pgfpathlineto{\pgfqpoint{2.885572in}{1.867821in}}%
\pgfpathlineto{\pgfqpoint{2.896559in}{1.840340in}}%
\pgfpathlineto{\pgfqpoint{2.909426in}{1.812988in}}%
\pgfpathlineto{\pgfqpoint{2.924163in}{1.785791in}}%
\pgfpathlineto{\pgfqpoint{2.940758in}{1.758770in}}%
\pgfpathlineto{\pgfqpoint{2.965750in}{1.723058in}}%
\pgfpathlineto{\pgfqpoint{2.993982in}{1.687754in}}%
\pgfpathlineto{\pgfqpoint{3.025409in}{1.652913in}}%
\pgfpathlineto{\pgfqpoint{3.059983in}{1.618586in}}%
\pgfpathlineto{\pgfqpoint{3.097647in}{1.584826in}}%
\pgfpathlineto{\pgfqpoint{3.138342in}{1.551681in}}%
\pgfpathlineto{\pgfqpoint{3.182002in}{1.519202in}}%
\pgfpathlineto{\pgfqpoint{3.228556in}{1.487436in}}%
\pgfpathlineto{\pgfqpoint{3.277929in}{1.456430in}}%
\pgfpathlineto{\pgfqpoint{3.330040in}{1.426227in}}%
\pgfpathlineto{\pgfqpoint{3.384805in}{1.396872in}}%
\pgfpathlineto{\pgfqpoint{3.442136in}{1.368406in}}%
\pgfpathlineto{\pgfqpoint{3.501939in}{1.340868in}}%
\pgfpathlineto{\pgfqpoint{3.564118in}{1.314298in}}%
\pgfpathlineto{\pgfqpoint{3.628573in}{1.288729in}}%
\pgfpathlineto{\pgfqpoint{3.712186in}{1.258231in}}%
\pgfpathlineto{\pgfqpoint{3.798986in}{1.229414in}}%
\pgfpathlineto{\pgfqpoint{3.888761in}{1.202336in}}%
\pgfpathlineto{\pgfqpoint{3.981290in}{1.177049in}}%
\pgfpathlineto{\pgfqpoint{4.076349in}{1.153601in}}%
\pgfpathlineto{\pgfqpoint{4.173713in}{1.132031in}}%
\pgfpathlineto{\pgfqpoint{4.273148in}{1.112378in}}%
\pgfpathlineto{\pgfqpoint{4.374424in}{1.094672in}}%
\pgfpathlineto{\pgfqpoint{4.477305in}{1.078940in}}%
\pgfpathlineto{\pgfqpoint{4.581558in}{1.065202in}}%
\pgfpathlineto{\pgfqpoint{4.686947in}{1.053476in}}%
\pgfpathlineto{\pgfqpoint{4.793239in}{1.043771in}}%
\pgfpathlineto{\pgfqpoint{4.900200in}{1.036095in}}%
\pgfpathlineto{\pgfqpoint{5.007601in}{1.030449in}}%
\pgfpathlineto{\pgfqpoint{5.115213in}{1.026831in}}%
\pgfpathlineto{\pgfqpoint{5.222810in}{1.025234in}}%
\pgfpathlineto{\pgfqpoint{5.330172in}{1.025646in}}%
\pgfpathlineto{\pgfqpoint{5.437079in}{1.028052in}}%
\pgfpathlineto{\pgfqpoint{5.543318in}{1.032434in}}%
\pgfpathlineto{\pgfqpoint{5.648681in}{1.038768in}}%
\pgfpathlineto{\pgfqpoint{5.752962in}{1.047028in}}%
\pgfpathlineto{\pgfqpoint{5.855962in}{1.057185in}}%
\pgfpathlineto{\pgfqpoint{5.957487in}{1.069205in}}%
\pgfpathlineto{\pgfqpoint{6.057349in}{1.083053in}}%
\pgfpathlineto{\pgfqpoint{6.155366in}{1.098691in}}%
\pgfpathlineto{\pgfqpoint{6.251359in}{1.116075in}}%
\pgfpathlineto{\pgfqpoint{6.345158in}{1.135164in}}%
\pgfpathlineto{\pgfqpoint{6.436599in}{1.155909in}}%
\pgfpathlineto{\pgfqpoint{6.525521in}{1.178263in}}%
\pgfpathlineto{\pgfqpoint{6.611773in}{1.202174in}}%
\pgfpathlineto{\pgfqpoint{6.695209in}{1.227589in}}%
\pgfpathlineto{\pgfqpoint{6.775687in}{1.254455in}}%
\pgfpathlineto{\pgfqpoint{6.853074in}{1.282713in}}%
\pgfpathlineto{\pgfqpoint{6.927243in}{1.312306in}}%
\pgfpathlineto{\pgfqpoint{6.998072in}{1.343173in}}%
\pgfpathlineto{\pgfqpoint{7.065446in}{1.375254in}}%
\pgfpathlineto{\pgfqpoint{7.129257in}{1.408485in}}%
\pgfpathlineto{\pgfqpoint{7.189403in}{1.442803in}}%
\pgfpathlineto{\pgfqpoint{7.234816in}{1.470995in}}%
\pgfpathlineto{\pgfqpoint{7.277776in}{1.499806in}}%
\pgfpathlineto{\pgfqpoint{7.318239in}{1.529203in}}%
\pgfpathlineto{\pgfqpoint{7.356164in}{1.559149in}}%
\pgfpathlineto{\pgfqpoint{7.391513in}{1.589610in}}%
\pgfpathlineto{\pgfqpoint{7.424250in}{1.620550in}}%
\pgfpathlineto{\pgfqpoint{7.454342in}{1.651934in}}%
\pgfpathlineto{\pgfqpoint{7.481759in}{1.683724in}}%
\pgfpathlineto{\pgfqpoint{7.506474in}{1.715884in}}%
\pgfpathlineto{\pgfqpoint{7.528460in}{1.748377in}}%
\pgfpathlineto{\pgfqpoint{7.547696in}{1.781167in}}%
\pgfpathlineto{\pgfqpoint{7.564161in}{1.814214in}}%
\pgfpathlineto{\pgfqpoint{7.577840in}{1.847483in}}%
\pgfpathlineto{\pgfqpoint{7.588716in}{1.880935in}}%
\pgfpathlineto{\pgfqpoint{7.596780in}{1.914531in}}%
\pgfpathlineto{\pgfqpoint{7.602022in}{1.948234in}}%
\pgfpathlineto{\pgfqpoint{7.604435in}{1.982006in}}%
\pgfpathlineto{\pgfqpoint{7.604016in}{2.015807in}}%
\pgfpathlineto{\pgfqpoint{7.600766in}{2.049600in}}%
\pgfpathlineto{\pgfqpoint{7.594685in}{2.083346in}}%
\pgfpathlineto{\pgfqpoint{7.585779in}{2.117007in}}%
\pgfpathlineto{\pgfqpoint{7.574055in}{2.150543in}}%
\pgfpathlineto{\pgfqpoint{7.559526in}{2.183916in}}%
\pgfpathlineto{\pgfqpoint{7.542203in}{2.217088in}}%
\pgfpathlineto{\pgfqpoint{7.522103in}{2.250020in}}%
\pgfpathlineto{\pgfqpoint{7.499246in}{2.282674in}}%
\pgfpathlineto{\pgfqpoint{7.473654in}{2.315011in}}%
\pgfpathlineto{\pgfqpoint{7.445352in}{2.346995in}}%
\pgfpathlineto{\pgfqpoint{7.414368in}{2.378587in}}%
\pgfpathlineto{\pgfqpoint{7.380734in}{2.409749in}}%
\pgfpathlineto{\pgfqpoint{7.344482in}{2.440444in}}%
\pgfpathlineto{\pgfqpoint{7.305650in}{2.470636in}}%
\pgfpathlineto{\pgfqpoint{7.264278in}{2.500288in}}%
\pgfpathlineto{\pgfqpoint{7.220409in}{2.529364in}}%
\pgfpathlineto{\pgfqpoint{7.174088in}{2.557828in}}%
\pgfpathlineto{\pgfqpoint{7.112814in}{2.592494in}}%
\pgfpathlineto{\pgfqpoint{7.047887in}{2.626081in}}%
\pgfpathlineto{\pgfqpoint{6.979416in}{2.658524in}}%
\pgfpathlineto{\pgfqpoint{6.907516in}{2.689757in}}%
\pgfpathlineto{\pgfqpoint{6.832311in}{2.719719in}}%
\pgfpathlineto{\pgfqpoint{6.753928in}{2.748347in}}%
\pgfpathlineto{\pgfqpoint{6.672505in}{2.775582in}}%
\pgfpathlineto{\pgfqpoint{6.588185in}{2.801368in}}%
\pgfpathlineto{\pgfqpoint{6.501116in}{2.825650in}}%
\pgfpathlineto{\pgfqpoint{6.411455in}{2.848374in}}%
\pgfpathlineto{\pgfqpoint{6.319362in}{2.869493in}}%
\pgfpathlineto{\pgfqpoint{6.225005in}{2.888957in}}%
\pgfpathlineto{\pgfqpoint{6.128557in}{2.906722in}}%
\pgfpathlineto{\pgfqpoint{6.030198in}{2.922747in}}%
\pgfpathlineto{\pgfqpoint{5.930109in}{2.936994in}}%
\pgfpathlineto{\pgfqpoint{5.828481in}{2.949425in}}%
\pgfpathlineto{\pgfqpoint{5.725506in}{2.960011in}}%
\pgfpathlineto{\pgfqpoint{5.621381in}{2.968721in}}%
\pgfpathlineto{\pgfqpoint{5.516307in}{2.975530in}}%
\pgfpathlineto{\pgfqpoint{5.410489in}{2.980417in}}%
\pgfpathlineto{\pgfqpoint{5.304134in}{2.983363in}}%
\pgfpathlineto{\pgfqpoint{5.197453in}{2.984354in}}%
\pgfpathlineto{\pgfqpoint{5.090660in}{2.983380in}}%
\pgfpathlineto{\pgfqpoint{4.983967in}{2.980434in}}%
\pgfpathlineto{\pgfqpoint{4.877592in}{2.975513in}}%
\pgfpathlineto{\pgfqpoint{4.771752in}{2.968619in}}%
\pgfpathlineto{\pgfqpoint{4.666665in}{2.959758in}}%
\pgfpathlineto{\pgfqpoint{4.562548in}{2.948939in}}%
\pgfpathlineto{\pgfqpoint{4.459620in}{2.936175in}}%
\pgfpathlineto{\pgfqpoint{4.358096in}{2.921486in}}%
\pgfpathlineto{\pgfqpoint{4.258193in}{2.904893in}}%
\pgfpathlineto{\pgfqpoint{4.160124in}{2.886422in}}%
\pgfpathlineto{\pgfqpoint{4.064102in}{2.866105in}}%
\pgfpathlineto{\pgfqpoint{3.970333in}{2.843975in}}%
\pgfpathlineto{\pgfqpoint{3.879025in}{2.820072in}}%
\pgfpathlineto{\pgfqpoint{3.790379in}{2.794439in}}%
\pgfpathlineto{\pgfqpoint{3.704593in}{2.767123in}}%
\pgfpathlineto{\pgfqpoint{3.621858in}{2.738175in}}%
\pgfpathlineto{\pgfqpoint{3.542364in}{2.707649in}}%
\pgfpathlineto{\pgfqpoint{3.466292in}{2.675606in}}%
\pgfpathlineto{\pgfqpoint{3.408016in}{2.648920in}}%
\pgfpathlineto{\pgfqpoint{3.352129in}{2.621336in}}%
\pgfpathlineto{\pgfqpoint{3.298716in}{2.592890in}}%
\pgfpathlineto{\pgfqpoint{3.247857in}{2.563618in}}%
\pgfpathlineto{\pgfqpoint{3.199630in}{2.533558in}}%
\pgfpathlineto{\pgfqpoint{3.154110in}{2.502750in}}%
\pgfpathlineto{\pgfqpoint{3.111366in}{2.471235in}}%
\pgfpathlineto{\pgfqpoint{3.071467in}{2.439054in}}%
\pgfpathlineto{\pgfqpoint{3.034476in}{2.406251in}}%
\pgfpathlineto{\pgfqpoint{3.000451in}{2.372870in}}%
\pgfpathlineto{\pgfqpoint{2.969449in}{2.338957in}}%
\pgfpathlineto{\pgfqpoint{2.941520in}{2.304558in}}%
\pgfpathlineto{\pgfqpoint{2.916712in}{2.269721in}}%
\pgfpathlineto{\pgfqpoint{2.895066in}{2.234494in}}%
\pgfpathlineto{\pgfqpoint{2.880929in}{2.207847in}}%
\pgfpathlineto{\pgfqpoint{2.868607in}{2.181030in}}%
\pgfpathlineto{\pgfqpoint{2.858112in}{2.154063in}}%
\pgfpathlineto{\pgfqpoint{2.849455in}{2.126969in}}%
\pgfpathlineto{\pgfqpoint{2.842645in}{2.099767in}}%
\pgfpathlineto{\pgfqpoint{2.837691in}{2.072482in}}%
\pgfpathlineto{\pgfqpoint{2.834599in}{2.045134in}}%
\pgfpathlineto{\pgfqpoint{2.833374in}{2.017745in}}%
\pgfpathlineto{\pgfqpoint{2.834019in}{1.990338in}}%
\pgfpathlineto{\pgfqpoint{2.836535in}{1.962934in}}%
\pgfpathlineto{\pgfqpoint{2.840922in}{1.935557in}}%
\pgfpathlineto{\pgfqpoint{2.847179in}{1.908228in}}%
\pgfpathlineto{\pgfqpoint{2.855301in}{1.880969in}}%
\pgfpathlineto{\pgfqpoint{2.865284in}{1.853804in}}%
\pgfpathlineto{\pgfqpoint{2.877121in}{1.826753in}}%
\pgfpathlineto{\pgfqpoint{2.890804in}{1.799840in}}%
\pgfpathlineto{\pgfqpoint{2.906321in}{1.773087in}}%
\pgfpathlineto{\pgfqpoint{2.929845in}{1.737701in}}%
\pgfpathlineto{\pgfqpoint{2.956576in}{1.702689in}}%
\pgfpathlineto{\pgfqpoint{2.986478in}{1.668103in}}%
\pgfpathlineto{\pgfqpoint{3.019507in}{1.633992in}}%
\pgfpathlineto{\pgfqpoint{3.055614in}{1.600407in}}%
\pgfpathlineto{\pgfqpoint{3.094746in}{1.567397in}}%
\pgfpathlineto{\pgfqpoint{3.136844in}{1.535008in}}%
\pgfpathlineto{\pgfqpoint{3.181845in}{1.503289in}}%
\pgfpathlineto{\pgfqpoint{3.229682in}{1.472286in}}%
\pgfpathlineto{\pgfqpoint{3.280281in}{1.442042in}}%
\pgfpathlineto{\pgfqpoint{3.333567in}{1.412600in}}%
\pgfpathlineto{\pgfqpoint{3.389456in}{1.384004in}}%
\pgfpathlineto{\pgfqpoint{3.447864in}{1.356293in}}%
\pgfpathlineto{\pgfqpoint{3.508702in}{1.329505in}}%
\pgfpathlineto{\pgfqpoint{3.571877in}{1.303678in}}%
\pgfpathlineto{\pgfqpoint{3.653985in}{1.272800in}}%
\pgfpathlineto{\pgfqpoint{3.739396in}{1.243544in}}%
\pgfpathlineto{\pgfqpoint{3.827909in}{1.215970in}}%
\pgfpathlineto{\pgfqpoint{3.919315in}{1.190136in}}%
\pgfpathlineto{\pgfqpoint{4.013396in}{1.166091in}}%
\pgfpathlineto{\pgfqpoint{4.109934in}{1.143883in}}%
\pgfpathlineto{\pgfqpoint{4.208703in}{1.123553in}}%
\pgfpathlineto{\pgfqpoint{4.309474in}{1.105137in}}%
\pgfpathlineto{\pgfqpoint{4.412016in}{1.088667in}}%
\pgfpathlineto{\pgfqpoint{4.516096in}{1.074170in}}%
\pgfpathlineto{\pgfqpoint{4.621479in}{1.061667in}}%
\pgfpathlineto{\pgfqpoint{4.727928in}{1.051174in}}%
\pgfpathlineto{\pgfqpoint{4.835210in}{1.042702in}}%
\pgfpathlineto{\pgfqpoint{4.943089in}{1.036259in}}%
\pgfpathlineto{\pgfqpoint{5.051332in}{1.031845in}}%
\pgfpathlineto{\pgfqpoint{5.159708in}{1.029459in}}%
\pgfpathlineto{\pgfqpoint{5.267987in}{1.029093in}}%
\pgfpathlineto{\pgfqpoint{5.375945in}{1.030735in}}%
\pgfpathlineto{\pgfqpoint{5.483358in}{1.034370in}}%
\pgfpathlineto{\pgfqpoint{5.590010in}{1.039976in}}%
\pgfpathlineto{\pgfqpoint{5.695686in}{1.047531in}}%
\pgfpathlineto{\pgfqpoint{5.800178in}{1.057007in}}%
\pgfpathlineto{\pgfqpoint{5.903282in}{1.068372in}}%
\pgfpathlineto{\pgfqpoint{6.004801in}{1.081591in}}%
\pgfpathlineto{\pgfqpoint{6.104542in}{1.096627in}}%
\pgfpathlineto{\pgfqpoint{6.202320in}{1.113439in}}%
\pgfpathlineto{\pgfqpoint{6.297953in}{1.131982in}}%
\pgfpathlineto{\pgfqpoint{6.391269in}{1.152211in}}%
\pgfpathlineto{\pgfqpoint{6.482101in}{1.174076in}}%
\pgfpathlineto{\pgfqpoint{6.570288in}{1.197525in}}%
\pgfpathlineto{\pgfqpoint{6.655677in}{1.222505in}}%
\pgfpathlineto{\pgfqpoint{6.738120in}{1.248960in}}%
\pgfpathlineto{\pgfqpoint{6.817479in}{1.276832in}}%
\pgfpathlineto{\pgfqpoint{6.893619in}{1.306062in}}%
\pgfpathlineto{\pgfqpoint{6.966414in}{1.336588in}}%
\pgfpathlineto{\pgfqpoint{7.035746in}{1.368348in}}%
\pgfpathlineto{\pgfqpoint{7.101502in}{1.401277in}}%
\pgfpathlineto{\pgfqpoint{7.163576in}{1.435310in}}%
\pgfpathlineto{\pgfqpoint{7.210518in}{1.463287in}}%
\pgfpathlineto{\pgfqpoint{7.254993in}{1.491893in}}%
\pgfpathlineto{\pgfqpoint{7.296956in}{1.521094in}}%
\pgfpathlineto{\pgfqpoint{7.336367in}{1.550853in}}%
\pgfpathlineto{\pgfqpoint{7.373185in}{1.581135in}}%
\pgfpathlineto{\pgfqpoint{7.407375in}{1.611904in}}%
\pgfpathlineto{\pgfqpoint{7.438903in}{1.643124in}}%
\pgfpathlineto{\pgfqpoint{7.467738in}{1.674757in}}%
\pgfpathlineto{\pgfqpoint{7.493853in}{1.706768in}}%
\pgfpathlineto{\pgfqpoint{7.517222in}{1.739118in}}%
\pgfpathlineto{\pgfqpoint{7.537822in}{1.771770in}}%
\pgfpathlineto{\pgfqpoint{7.555634in}{1.804687in}}%
\pgfpathlineto{\pgfqpoint{7.570641in}{1.837831in}}%
\pgfpathlineto{\pgfqpoint{7.582827in}{1.871165in}}%
\pgfpathlineto{\pgfqpoint{7.592182in}{1.904649in}}%
\pgfpathlineto{\pgfqpoint{7.598697in}{1.938246in}}%
\pgfpathlineto{\pgfqpoint{7.602364in}{1.971918in}}%
\pgfpathlineto{\pgfqpoint{7.603181in}{2.005627in}}%
\pgfpathlineto{\pgfqpoint{7.601146in}{2.039334in}}%
\pgfpathlineto{\pgfqpoint{7.596262in}{2.073002in}}%
\pgfpathlineto{\pgfqpoint{7.588532in}{2.106591in}}%
\pgfpathlineto{\pgfqpoint{7.577965in}{2.140063in}}%
\pgfpathlineto{\pgfqpoint{7.564570in}{2.173381in}}%
\pgfpathlineto{\pgfqpoint{7.548360in}{2.206506in}}%
\pgfpathlineto{\pgfqpoint{7.529351in}{2.239400in}}%
\pgfpathlineto{\pgfqpoint{7.507561in}{2.272025in}}%
\pgfpathlineto{\pgfqpoint{7.483011in}{2.304344in}}%
\pgfpathlineto{\pgfqpoint{7.455725in}{2.336318in}}%
\pgfpathlineto{\pgfqpoint{7.425730in}{2.367910in}}%
\pgfpathlineto{\pgfqpoint{7.393056in}{2.399084in}}%
\pgfpathlineto{\pgfqpoint{7.357735in}{2.429801in}}%
\pgfpathlineto{\pgfqpoint{7.319801in}{2.460026in}}%
\pgfpathlineto{\pgfqpoint{7.279295in}{2.489723in}}%
\pgfpathlineto{\pgfqpoint{7.236256in}{2.518854in}}%
\pgfpathlineto{\pgfqpoint{7.190728in}{2.547385in}}%
\pgfpathlineto{\pgfqpoint{7.142759in}{2.575280in}}%
\pgfpathlineto{\pgfqpoint{7.079440in}{2.609202in}}%
\pgfpathlineto{\pgfqpoint{7.012487in}{2.642009in}}%
\pgfpathlineto{\pgfqpoint{6.942012in}{2.673637in}}%
\pgfpathlineto{\pgfqpoint{6.868131in}{2.704022in}}%
\pgfpathlineto{\pgfqpoint{6.790969in}{2.733101in}}%
\pgfpathlineto{\pgfqpoint{6.710658in}{2.760813in}}%
\pgfpathlineto{\pgfqpoint{6.627337in}{2.787100in}}%
\pgfpathlineto{\pgfqpoint{6.541151in}{2.811905in}}%
\pgfpathlineto{\pgfqpoint{6.452251in}{2.835174in}}%
\pgfpathlineto{\pgfqpoint{6.360798in}{2.856854in}}%
\pgfpathlineto{\pgfqpoint{6.266954in}{2.876897in}}%
\pgfpathlineto{\pgfqpoint{6.170893in}{2.895254in}}%
\pgfpathlineto{\pgfqpoint{6.072791in}{2.911882in}}%
\pgfpathlineto{\pgfqpoint{5.972831in}{2.926739in}}%
\pgfpathlineto{\pgfqpoint{5.871202in}{2.939787in}}%
\pgfpathlineto{\pgfqpoint{5.768098in}{2.950991in}}%
\pgfpathlineto{\pgfqpoint{5.663718in}{2.960320in}}%
\pgfpathlineto{\pgfqpoint{5.558265in}{2.967745in}}%
\pgfpathlineto{\pgfqpoint{5.451947in}{2.973242in}}%
\pgfpathlineto{\pgfqpoint{5.344976in}{2.976790in}}%
\pgfpathlineto{\pgfqpoint{5.237566in}{2.978373in}}%
\pgfpathlineto{\pgfqpoint{5.129936in}{2.977979in}}%
\pgfpathlineto{\pgfqpoint{5.022305in}{2.975598in}}%
\pgfpathlineto{\pgfqpoint{4.914896in}{2.971226in}}%
\pgfpathlineto{\pgfqpoint{4.807933in}{2.964863in}}%
\pgfpathlineto{\pgfqpoint{4.701641in}{2.956514in}}%
\pgfpathlineto{\pgfqpoint{4.596244in}{2.946187in}}%
\pgfpathlineto{\pgfqpoint{4.491968in}{2.933896in}}%
\pgfpathlineto{\pgfqpoint{4.389038in}{2.919659in}}%
\pgfpathlineto{\pgfqpoint{4.287675in}{2.903497in}}%
\pgfpathlineto{\pgfqpoint{4.188101in}{2.885437in}}%
\pgfpathlineto{\pgfqpoint{4.090535in}{2.865511in}}%
\pgfpathlineto{\pgfqpoint{3.995192in}{2.843754in}}%
\pgfpathlineto{\pgfqpoint{3.902283in}{2.820207in}}%
\pgfpathlineto{\pgfqpoint{3.812016in}{2.794913in}}%
\pgfpathlineto{\pgfqpoint{3.724594in}{2.767922in}}%
\pgfpathlineto{\pgfqpoint{3.640214in}{2.739287in}}%
\pgfpathlineto{\pgfqpoint{3.559067in}{2.709063in}}%
\pgfpathlineto{\pgfqpoint{3.481338in}{2.677313in}}%
\pgfpathlineto{\pgfqpoint{3.407205in}{2.644101in}}%
\pgfpathlineto{\pgfqpoint{3.350601in}{2.616524in}}%
\pgfpathlineto{\pgfqpoint{3.296492in}{2.588092in}}%
\pgfpathlineto{\pgfqpoint{3.244956in}{2.558843in}}%
\pgfpathlineto{\pgfqpoint{3.196071in}{2.528816in}}%
\pgfpathlineto{\pgfqpoint{3.149909in}{2.498051in}}%
\pgfpathlineto{\pgfqpoint{3.106538in}{2.466591in}}%
\pgfpathlineto{\pgfqpoint{3.066025in}{2.434478in}}%
\pgfpathlineto{\pgfqpoint{3.028430in}{2.401757in}}%
\pgfpathlineto{\pgfqpoint{2.993809in}{2.368473in}}%
\pgfpathlineto{\pgfqpoint{2.962215in}{2.334672in}}%
\pgfpathlineto{\pgfqpoint{2.933696in}{2.300400in}}%
\pgfpathlineto{\pgfqpoint{2.908295in}{2.265705in}}%
\pgfpathlineto{\pgfqpoint{2.886051in}{2.230636in}}%
\pgfpathlineto{\pgfqpoint{2.866998in}{2.195242in}}%
\pgfpathlineto{\pgfqpoint{2.854821in}{2.168513in}}%
\pgfpathlineto{\pgfqpoint{2.844467in}{2.141649in}}%
\pgfpathlineto{\pgfqpoint{2.835944in}{2.114673in}}%
\pgfpathlineto{\pgfqpoint{2.829261in}{2.087606in}}%
\pgfpathlineto{\pgfqpoint{2.824423in}{2.060469in}}%
\pgfpathlineto{\pgfqpoint{2.821435in}{2.033283in}}%
\pgfpathlineto{\pgfqpoint{2.820300in}{2.006071in}}%
\pgfpathlineto{\pgfqpoint{2.821020in}{1.978853in}}%
\pgfpathlineto{\pgfqpoint{2.823594in}{1.951652in}}%
\pgfpathlineto{\pgfqpoint{2.828021in}{1.924489in}}%
\pgfpathlineto{\pgfqpoint{2.834297in}{1.897385in}}%
\pgfpathlineto{\pgfqpoint{2.842418in}{1.870363in}}%
\pgfpathlineto{\pgfqpoint{2.852377in}{1.843444in}}%
\pgfpathlineto{\pgfqpoint{2.864166in}{1.816649in}}%
\pgfpathlineto{\pgfqpoint{2.877776in}{1.789999in}}%
\pgfpathlineto{\pgfqpoint{2.898736in}{1.754729in}}%
\pgfpathlineto{\pgfqpoint{2.922883in}{1.719805in}}%
\pgfpathlineto{\pgfqpoint{2.950181in}{1.685277in}}%
\pgfpathlineto{\pgfqpoint{2.980591in}{1.651193in}}%
\pgfpathlineto{\pgfqpoint{3.014068in}{1.617601in}}%
\pgfpathlineto{\pgfqpoint{3.050563in}{1.584549in}}%
\pgfpathlineto{\pgfqpoint{3.090022in}{1.552082in}}%
\pgfpathlineto{\pgfqpoint{3.132386in}{1.520246in}}%
\pgfpathlineto{\pgfqpoint{3.177595in}{1.489086in}}%
\pgfpathlineto{\pgfqpoint{3.225579in}{1.458645in}}%
\pgfpathlineto{\pgfqpoint{3.276270in}{1.428964in}}%
\pgfpathlineto{\pgfqpoint{3.329591in}{1.400086in}}%
\pgfpathlineto{\pgfqpoint{3.385465in}{1.372049in}}%
\pgfpathlineto{\pgfqpoint{3.443809in}{1.344892in}}%
\pgfpathlineto{\pgfqpoint{3.504536in}{1.318651in}}%
\pgfpathlineto{\pgfqpoint{3.583662in}{1.287193in}}%
\pgfpathlineto{\pgfqpoint{3.666191in}{1.257288in}}%
\pgfpathlineto{\pgfqpoint{3.751934in}{1.228998in}}%
\pgfpathlineto{\pgfqpoint{3.840697in}{1.202379in}}%
\pgfpathlineto{\pgfqpoint{3.932277in}{1.177486in}}%
\pgfpathlineto{\pgfqpoint{4.026466in}{1.154368in}}%
\pgfpathlineto{\pgfqpoint{4.123052in}{1.133068in}}%
\pgfpathlineto{\pgfqpoint{4.221818in}{1.113628in}}%
\pgfpathlineto{\pgfqpoint{4.322543in}{1.096082in}}%
\pgfpathlineto{\pgfqpoint{4.425004in}{1.080461in}}%
\pgfpathlineto{\pgfqpoint{4.528974in}{1.066792in}}%
\pgfpathlineto{\pgfqpoint{4.634224in}{1.055096in}}%
\pgfpathlineto{\pgfqpoint{4.740526in}{1.045389in}}%
\pgfpathlineto{\pgfqpoint{4.847648in}{1.037683in}}%
\pgfpathlineto{\pgfqpoint{4.955363in}{1.031987in}}%
\pgfpathlineto{\pgfqpoint{5.063439in}{1.028302in}}%
\pgfpathlineto{\pgfqpoint{5.171649in}{1.026628in}}%
\pgfpathlineto{\pgfqpoint{5.279767in}{1.026957in}}%
\pgfpathlineto{\pgfqpoint{5.387568in}{1.029279in}}%
\pgfpathlineto{\pgfqpoint{5.494831in}{1.033580in}}%
\pgfpathlineto{\pgfqpoint{5.601340in}{1.039840in}}%
\pgfpathlineto{\pgfqpoint{5.706879in}{1.048037in}}%
\pgfpathlineto{\pgfqpoint{5.811240in}{1.058144in}}%
\pgfpathlineto{\pgfqpoint{5.914216in}{1.070130in}}%
\pgfpathlineto{\pgfqpoint{6.015610in}{1.083961in}}%
\pgfpathlineto{\pgfqpoint{6.115225in}{1.099600in}}%
\pgfpathlineto{\pgfqpoint{6.212873in}{1.117006in}}%
\pgfpathlineto{\pgfqpoint{6.308372in}{1.136135in}}%
\pgfpathlineto{\pgfqpoint{6.401545in}{1.156941in}}%
\pgfpathlineto{\pgfqpoint{6.492221in}{1.179374in}}%
\pgfpathlineto{\pgfqpoint{6.580238in}{1.203381in}}%
\pgfpathlineto{\pgfqpoint{6.665438in}{1.228909in}}%
\pgfpathlineto{\pgfqpoint{6.747673in}{1.255901in}}%
\pgfpathlineto{\pgfqpoint{6.826798in}{1.284298in}}%
\pgfpathlineto{\pgfqpoint{6.902680in}{1.314038in}}%
\pgfpathlineto{\pgfqpoint{6.975188in}{1.345061in}}%
\pgfpathlineto{\pgfqpoint{7.044204in}{1.377300in}}%
\pgfpathlineto{\pgfqpoint{7.109612in}{1.410691in}}%
\pgfpathlineto{\pgfqpoint{7.171305in}{1.445167in}}%
\pgfpathlineto{\pgfqpoint{7.217919in}{1.473483in}}%
\pgfpathlineto{\pgfqpoint{7.262045in}{1.502414in}}%
\pgfpathlineto{\pgfqpoint{7.303639in}{1.531924in}}%
\pgfpathlineto{\pgfqpoint{7.342658in}{1.561977in}}%
\pgfpathlineto{\pgfqpoint{7.379064in}{1.592536in}}%
\pgfpathlineto{\pgfqpoint{7.412822in}{1.623565in}}%
\pgfpathlineto{\pgfqpoint{7.443898in}{1.655025in}}%
\pgfpathlineto{\pgfqpoint{7.472264in}{1.686881in}}%
\pgfpathlineto{\pgfqpoint{7.497891in}{1.719094in}}%
\pgfpathlineto{\pgfqpoint{7.520756in}{1.751626in}}%
\pgfpathlineto{\pgfqpoint{7.540837in}{1.784439in}}%
\pgfpathlineto{\pgfqpoint{7.558116in}{1.817496in}}%
\pgfpathlineto{\pgfqpoint{7.572576in}{1.850758in}}%
\pgfpathlineto{\pgfqpoint{7.584207in}{1.884187in}}%
\pgfpathlineto{\pgfqpoint{7.592996in}{1.917744in}}%
\pgfpathlineto{\pgfqpoint{7.598937in}{1.951392in}}%
\pgfpathlineto{\pgfqpoint{7.602025in}{1.985091in}}%
\pgfpathlineto{\pgfqpoint{7.602258in}{2.018804in}}%
\pgfpathlineto{\pgfqpoint{7.599637in}{2.052493in}}%
\pgfpathlineto{\pgfqpoint{7.594166in}{2.086118in}}%
\pgfpathlineto{\pgfqpoint{7.585851in}{2.119642in}}%
\pgfpathlineto{\pgfqpoint{7.574701in}{2.153027in}}%
\pgfpathlineto{\pgfqpoint{7.560727in}{2.186235in}}%
\pgfpathlineto{\pgfqpoint{7.543944in}{2.219228in}}%
\pgfpathlineto{\pgfqpoint{7.524368in}{2.251968in}}%
\pgfpathlineto{\pgfqpoint{7.502021in}{2.284418in}}%
\pgfpathlineto{\pgfqpoint{7.476923in}{2.316540in}}%
\pgfpathlineto{\pgfqpoint{7.449101in}{2.348298in}}%
\pgfpathlineto{\pgfqpoint{7.418581in}{2.379655in}}%
\pgfpathlineto{\pgfqpoint{7.385395in}{2.410574in}}%
\pgfpathlineto{\pgfqpoint{7.349576in}{2.441018in}}%
\pgfpathlineto{\pgfqpoint{7.311159in}{2.470953in}}%
\pgfpathlineto{\pgfqpoint{7.270184in}{2.500341in}}%
\pgfpathlineto{\pgfqpoint{7.226692in}{2.529149in}}%
\pgfpathlineto{\pgfqpoint{7.180726in}{2.557341in}}%
\pgfpathlineto{\pgfqpoint{7.132334in}{2.584882in}}%
\pgfpathlineto{\pgfqpoint{7.068508in}{2.618342in}}%
\pgfpathlineto{\pgfqpoint{7.001071in}{2.650668in}}%
\pgfpathlineto{\pgfqpoint{6.930134in}{2.681795in}}%
\pgfpathlineto{\pgfqpoint{6.855812in}{2.711661in}}%
\pgfpathlineto{\pgfqpoint{6.778230in}{2.740204in}}%
\pgfpathlineto{\pgfqpoint{6.697517in}{2.767365in}}%
\pgfpathlineto{\pgfqpoint{6.613809in}{2.793087in}}%
\pgfpathlineto{\pgfqpoint{6.527250in}{2.817313in}}%
\pgfpathlineto{\pgfqpoint{6.437990in}{2.839989in}}%
\pgfpathlineto{\pgfqpoint{6.346186in}{2.861064in}}%
\pgfpathlineto{\pgfqpoint{6.252000in}{2.880489in}}%
\pgfpathlineto{\pgfqpoint{6.155601in}{2.898216in}}%
\pgfpathlineto{\pgfqpoint{6.057166in}{2.914202in}}%
\pgfpathlineto{\pgfqpoint{5.956876in}{2.928404in}}%
\pgfpathlineto{\pgfqpoint{5.854918in}{2.940784in}}%
\pgfpathlineto{\pgfqpoint{5.751485in}{2.951307in}}%
\pgfpathlineto{\pgfqpoint{5.646777in}{2.959941in}}%
\pgfpathlineto{\pgfqpoint{5.540996in}{2.966656in}}%
\pgfpathlineto{\pgfqpoint{5.434350in}{2.971427in}}%
\pgfpathlineto{\pgfqpoint{5.327053in}{2.974234in}}%
\pgfpathlineto{\pgfqpoint{5.219322in}{2.975058in}}%
\pgfpathlineto{\pgfqpoint{5.111376in}{2.973887in}}%
\pgfpathlineto{\pgfqpoint{5.003440in}{2.970710in}}%
\pgfpathlineto{\pgfqpoint{4.895739in}{2.965523in}}%
\pgfpathlineto{\pgfqpoint{4.788502in}{2.958325in}}%
\pgfpathlineto{\pgfqpoint{4.681959in}{2.949121in}}%
\pgfpathlineto{\pgfqpoint{4.576341in}{2.937919in}}%
\pgfpathlineto{\pgfqpoint{4.471878in}{2.924732in}}%
\pgfpathlineto{\pgfqpoint{4.368802in}{2.909579in}}%
\pgfpathlineto{\pgfqpoint{4.267343in}{2.892483in}}%
\pgfpathlineto{\pgfqpoint{4.167729in}{2.873471in}}%
\pgfpathlineto{\pgfqpoint{4.070186in}{2.852576in}}%
\pgfpathlineto{\pgfqpoint{3.974937in}{2.829836in}}%
\pgfpathlineto{\pgfqpoint{3.882201in}{2.805294in}}%
\pgfpathlineto{\pgfqpoint{3.792192in}{2.778995in}}%
\pgfpathlineto{\pgfqpoint{3.705121in}{2.750993in}}%
\pgfpathlineto{\pgfqpoint{3.621191in}{2.721343in}}%
\pgfpathlineto{\pgfqpoint{3.540600in}{2.690106in}}%
\pgfpathlineto{\pgfqpoint{3.463536in}{2.657348in}}%
\pgfpathlineto{\pgfqpoint{3.404547in}{2.630091in}}%
\pgfpathlineto{\pgfqpoint{3.348023in}{2.601943in}}%
\pgfpathlineto{\pgfqpoint{3.294049in}{2.572940in}}%
\pgfpathlineto{\pgfqpoint{3.242706in}{2.543124in}}%
\pgfpathlineto{\pgfqpoint{3.194072in}{2.512537in}}%
\pgfpathlineto{\pgfqpoint{3.148222in}{2.481221in}}%
\pgfpathlineto{\pgfqpoint{3.105224in}{2.449220in}}%
\pgfpathlineto{\pgfqpoint{3.065144in}{2.416581in}}%
\pgfpathlineto{\pgfqpoint{3.028041in}{2.383349in}}%
\pgfpathlineto{\pgfqpoint{2.993972in}{2.349572in}}%
\pgfpathlineto{\pgfqpoint{2.962986in}{2.315299in}}%
\pgfpathlineto{\pgfqpoint{2.935131in}{2.280579in}}%
\pgfpathlineto{\pgfqpoint{2.910447in}{2.245461in}}%
\pgfpathlineto{\pgfqpoint{2.888970in}{2.209997in}}%
\pgfpathlineto{\pgfqpoint{2.874985in}{2.183202in}}%
\pgfpathlineto{\pgfqpoint{2.862833in}{2.156263in}}%
\pgfpathlineto{\pgfqpoint{2.852523in}{2.129201in}}%
\pgfpathlineto{\pgfqpoint{2.844063in}{2.102038in}}%
\pgfpathlineto{\pgfqpoint{2.837458in}{2.074796in}}%
\pgfpathlineto{\pgfqpoint{2.832714in}{2.047498in}}%
\pgfpathlineto{\pgfqpoint{2.829834in}{2.020164in}}%
\pgfpathlineto{\pgfqpoint{2.828819in}{1.992819in}}%
\pgfpathlineto{\pgfqpoint{2.829668in}{1.965482in}}%
\pgfpathlineto{\pgfqpoint{2.832380in}{1.938177in}}%
\pgfpathlineto{\pgfqpoint{2.836950in}{1.910925in}}%
\pgfpathlineto{\pgfqpoint{2.843375in}{1.883748in}}%
\pgfpathlineto{\pgfqpoint{2.851647in}{1.856668in}}%
\pgfpathlineto{\pgfqpoint{2.861757in}{1.829707in}}%
\pgfpathlineto{\pgfqpoint{2.873696in}{1.802885in}}%
\pgfpathlineto{\pgfqpoint{2.887453in}{1.776225in}}%
\pgfpathlineto{\pgfqpoint{2.908600in}{1.740965in}}%
\pgfpathlineto{\pgfqpoint{2.932921in}{1.706078in}}%
\pgfpathlineto{\pgfqpoint{2.960375in}{1.671613in}}%
\pgfpathlineto{\pgfqpoint{2.990918in}{1.637619in}}%
\pgfpathlineto{\pgfqpoint{3.024503in}{1.604141in}}%
\pgfpathlineto{\pgfqpoint{3.061076in}{1.571227in}}%
\pgfpathlineto{\pgfqpoint{3.100580in}{1.538921in}}%
\pgfpathlineto{\pgfqpoint{3.142954in}{1.507267in}}%
\pgfpathlineto{\pgfqpoint{3.188134in}{1.476309in}}%
\pgfpathlineto{\pgfqpoint{3.236049in}{1.446087in}}%
\pgfpathlineto{\pgfqpoint{3.286628in}{1.416642in}}%
\pgfpathlineto{\pgfqpoint{3.339794in}{1.388013in}}%
\pgfpathlineto{\pgfqpoint{3.395467in}{1.360239in}}%
\pgfpathlineto{\pgfqpoint{3.453565in}{1.333355in}}%
\pgfpathlineto{\pgfqpoint{3.514002in}{1.307396in}}%
\pgfpathlineto{\pgfqpoint{3.592700in}{1.276299in}}%
\pgfpathlineto{\pgfqpoint{3.674733in}{1.246762in}}%
\pgfpathlineto{\pgfqpoint{3.759915in}{1.218843in}}%
\pgfpathlineto{\pgfqpoint{3.848052in}{1.192595in}}%
\pgfpathlineto{\pgfqpoint{3.938949in}{1.168069in}}%
\pgfpathlineto{\pgfqpoint{4.032402in}{1.145312in}}%
\pgfpathlineto{\pgfqpoint{4.128205in}{1.124363in}}%
\pgfpathlineto{\pgfqpoint{4.226148in}{1.105262in}}%
\pgfpathlineto{\pgfqpoint{4.326016in}{1.088040in}}%
\pgfpathlineto{\pgfqpoint{4.427594in}{1.072728in}}%
\pgfpathlineto{\pgfqpoint{4.530662in}{1.059350in}}%
\pgfpathlineto{\pgfqpoint{4.635001in}{1.047926in}}%
\pgfpathlineto{\pgfqpoint{4.740389in}{1.038472in}}%
\pgfpathlineto{\pgfqpoint{4.846603in}{1.030999in}}%
\pgfpathlineto{\pgfqpoint{4.953422in}{1.025516in}}%
\pgfpathlineto{\pgfqpoint{5.060623in}{1.022024in}}%
\pgfpathlineto{\pgfqpoint{5.167986in}{1.020523in}}%
\pgfpathlineto{\pgfqpoint{5.275291in}{1.021007in}}%
\pgfpathlineto{\pgfqpoint{5.382320in}{1.023467in}}%
\pgfpathlineto{\pgfqpoint{5.488857in}{1.027889in}}%
\pgfpathlineto{\pgfqpoint{5.594690in}{1.034255in}}%
\pgfpathlineto{\pgfqpoint{5.699609in}{1.042545in}}%
\pgfpathlineto{\pgfqpoint{5.803408in}{1.052734in}}%
\pgfpathlineto{\pgfqpoint{5.905885in}{1.064791in}}%
\pgfpathlineto{\pgfqpoint{6.006842in}{1.078686in}}%
\pgfpathlineto{\pgfqpoint{6.106087in}{1.094383in}}%
\pgfpathlineto{\pgfqpoint{6.203431in}{1.111842in}}%
\pgfpathlineto{\pgfqpoint{6.298692in}{1.131021in}}%
\pgfpathlineto{\pgfqpoint{6.391693in}{1.151876in}}%
\pgfpathlineto{\pgfqpoint{6.482263in}{1.174357in}}%
\pgfpathlineto{\pgfqpoint{6.570237in}{1.198416in}}%
\pgfpathlineto{\pgfqpoint{6.655456in}{1.223998in}}%
\pgfpathlineto{\pgfqpoint{6.737770in}{1.251047in}}%
\pgfpathlineto{\pgfqpoint{6.817032in}{1.279506in}}%
\pgfpathlineto{\pgfqpoint{6.893104in}{1.309314in}}%
\pgfpathlineto{\pgfqpoint{6.965856in}{1.340410in}}%
\pgfpathlineto{\pgfqpoint{7.035163in}{1.372730in}}%
\pgfpathlineto{\pgfqpoint{7.100909in}{1.406207in}}%
\pgfpathlineto{\pgfqpoint{7.162983in}{1.440775in}}%
\pgfpathlineto{\pgfqpoint{7.209930in}{1.469170in}}%
\pgfpathlineto{\pgfqpoint{7.254414in}{1.498183in}}%
\pgfpathlineto{\pgfqpoint{7.296387in}{1.527778in}}%
\pgfpathlineto{\pgfqpoint{7.335807in}{1.557920in}}%
\pgfpathlineto{\pgfqpoint{7.372635in}{1.588571in}}%
\pgfpathlineto{\pgfqpoint{7.406833in}{1.619693in}}%
\pgfpathlineto{\pgfqpoint{7.438368in}{1.651250in}}%
\pgfpathlineto{\pgfqpoint{7.467209in}{1.683204in}}%
\pgfpathlineto{\pgfqpoint{7.493328in}{1.715516in}}%
\pgfpathlineto{\pgfqpoint{7.516700in}{1.748148in}}%
\pgfpathlineto{\pgfqpoint{7.537303in}{1.781062in}}%
\pgfpathlineto{\pgfqpoint{7.555119in}{1.814219in}}%
\pgfpathlineto{\pgfqpoint{7.570132in}{1.847580in}}%
\pgfpathlineto{\pgfqpoint{7.582328in}{1.881108in}}%
\pgfpathlineto{\pgfqpoint{7.591697in}{1.914763in}}%
\pgfpathlineto{\pgfqpoint{7.598231in}{1.948506in}}%
\pgfpathlineto{\pgfqpoint{7.601927in}{1.982299in}}%
\pgfpathlineto{\pgfqpoint{7.602783in}{2.016103in}}%
\pgfpathlineto{\pgfqpoint{7.600800in}{2.049880in}}%
\pgfpathlineto{\pgfqpoint{7.595981in}{2.083591in}}%
\pgfpathlineto{\pgfqpoint{7.588334in}{2.117197in}}%
\pgfpathlineto{\pgfqpoint{7.577867in}{2.150662in}}%
\pgfpathlineto{\pgfqpoint{7.564593in}{2.183946in}}%
\pgfpathlineto{\pgfqpoint{7.548527in}{2.217011in}}%
\pgfpathlineto{\pgfqpoint{7.529686in}{2.249821in}}%
\pgfpathlineto{\pgfqpoint{7.508091in}{2.282338in}}%
\pgfpathlineto{\pgfqpoint{7.483763in}{2.314525in}}%
\pgfpathlineto{\pgfqpoint{7.456730in}{2.346345in}}%
\pgfpathlineto{\pgfqpoint{7.427017in}{2.377761in}}%
\pgfpathlineto{\pgfqpoint{7.394658in}{2.408737in}}%
\pgfpathlineto{\pgfqpoint{7.359684in}{2.439238in}}%
\pgfpathlineto{\pgfqpoint{7.322131in}{2.469228in}}%
\pgfpathlineto{\pgfqpoint{7.282039in}{2.498671in}}%
\pgfpathlineto{\pgfqpoint{7.239448in}{2.527534in}}%
\pgfpathlineto{\pgfqpoint{7.194402in}{2.555782in}}%
\pgfpathlineto{\pgfqpoint{7.146947in}{2.583381in}}%
\pgfpathlineto{\pgfqpoint{7.084315in}{2.616916in}}%
\pgfpathlineto{\pgfqpoint{7.018097in}{2.649322in}}%
\pgfpathlineto{\pgfqpoint{6.948400in}{2.680535in}}%
\pgfpathlineto{\pgfqpoint{6.875336in}{2.710494in}}%
\pgfpathlineto{\pgfqpoint{6.799027in}{2.739140in}}%
\pgfpathlineto{\pgfqpoint{6.719597in}{2.766414in}}%
\pgfpathlineto{\pgfqpoint{6.637179in}{2.792260in}}%
\pgfpathlineto{\pgfqpoint{6.551911in}{2.816622in}}%
\pgfpathlineto{\pgfqpoint{6.463938in}{2.839449in}}%
\pgfpathlineto{\pgfqpoint{6.373411in}{2.860689in}}%
\pgfpathlineto{\pgfqpoint{6.280485in}{2.880293in}}%
\pgfpathlineto{\pgfqpoint{6.185324in}{2.898216in}}%
\pgfpathlineto{\pgfqpoint{6.088097in}{2.914412in}}%
\pgfpathlineto{\pgfqpoint{5.988977in}{2.928840in}}%
\pgfpathlineto{\pgfqpoint{5.888145in}{2.941461in}}%
\pgfpathlineto{\pgfqpoint{5.785787in}{2.952240in}}%
\pgfpathlineto{\pgfqpoint{5.682094in}{2.961142in}}%
\pgfpathlineto{\pgfqpoint{5.577262in}{2.968138in}}%
\pgfpathlineto{\pgfqpoint{5.471493in}{2.973201in}}%
\pgfpathlineto{\pgfqpoint{5.364994in}{2.976307in}}%
\pgfpathlineto{\pgfqpoint{5.257974in}{2.977438in}}%
\pgfpathlineto{\pgfqpoint{5.150648in}{2.976576in}}%
\pgfpathlineto{\pgfqpoint{5.043236in}{2.973710in}}%
\pgfpathlineto{\pgfqpoint{4.935959in}{2.968831in}}%
\pgfpathlineto{\pgfqpoint{4.829043in}{2.961937in}}%
\pgfpathlineto{\pgfqpoint{4.722714in}{2.953026in}}%
\pgfpathlineto{\pgfqpoint{4.617203in}{2.942106in}}%
\pgfpathlineto{\pgfqpoint{4.512740in}{2.929184in}}%
\pgfpathlineto{\pgfqpoint{4.409557in}{2.914277in}}%
\pgfpathlineto{\pgfqpoint{4.307886in}{2.897402in}}%
\pgfpathlineto{\pgfqpoint{4.207958in}{2.878585in}}%
\pgfpathlineto{\pgfqpoint{4.110004in}{2.857854in}}%
\pgfpathlineto{\pgfqpoint{4.014252in}{2.835243in}}%
\pgfpathlineto{\pgfqpoint{3.920929in}{2.810793in}}%
\pgfpathlineto{\pgfqpoint{3.830255in}{2.784547in}}%
\pgfpathlineto{\pgfqpoint{3.742450in}{2.756555in}}%
\pgfpathlineto{\pgfqpoint{3.657725in}{2.726870in}}%
\pgfpathlineto{\pgfqpoint{3.576290in}{2.695554in}}%
\pgfpathlineto{\pgfqpoint{3.498343in}{2.662669in}}%
\pgfpathlineto{\pgfqpoint{3.438628in}{2.635277in}}%
\pgfpathlineto{\pgfqpoint{3.381366in}{2.606963in}}%
\pgfpathlineto{\pgfqpoint{3.326647in}{2.577765in}}%
\pgfpathlineto{\pgfqpoint{3.274559in}{2.547725in}}%
\pgfpathlineto{\pgfqpoint{3.225186in}{2.516884in}}%
\pgfpathlineto{\pgfqpoint{3.178608in}{2.485286in}}%
\pgfpathlineto{\pgfqpoint{3.134900in}{2.452978in}}%
\pgfpathlineto{\pgfqpoint{3.094131in}{2.420006in}}%
\pgfpathlineto{\pgfqpoint{3.056367in}{2.386417in}}%
\pgfpathlineto{\pgfqpoint{3.021669in}{2.352263in}}%
\pgfpathlineto{\pgfqpoint{2.990091in}{2.317591in}}%
\pgfpathlineto{\pgfqpoint{2.961684in}{2.282456in}}%
\pgfpathlineto{\pgfqpoint{2.936491in}{2.246907in}}%
\pgfpathlineto{\pgfqpoint{2.914553in}{2.210999in}}%
\pgfpathlineto{\pgfqpoint{2.900254in}{2.183864in}}%
\pgfpathlineto{\pgfqpoint{2.887817in}{2.156580in}}%
\pgfpathlineto{\pgfqpoint{2.877251in}{2.129169in}}%
\pgfpathlineto{\pgfqpoint{2.868565in}{2.101655in}}%
\pgfpathlineto{\pgfqpoint{2.861765in}{2.074061in}}%
\pgfpathlineto{\pgfqpoint{2.856856in}{2.046409in}}%
\pgfpathlineto{\pgfqpoint{2.853840in}{2.018724in}}%
\pgfpathlineto{\pgfqpoint{2.852720in}{1.991028in}}%
\pgfpathlineto{\pgfqpoint{2.853493in}{1.963344in}}%
\pgfpathlineto{\pgfqpoint{2.856159in}{1.935696in}}%
\pgfpathlineto{\pgfqpoint{2.860712in}{1.908107in}}%
\pgfpathlineto{\pgfqpoint{2.867146in}{1.880598in}}%
\pgfpathlineto{\pgfqpoint{2.875454in}{1.853194in}}%
\pgfpathlineto{\pgfqpoint{2.885626in}{1.825916in}}%
\pgfpathlineto{\pgfqpoint{2.897652in}{1.798787in}}%
\pgfpathlineto{\pgfqpoint{2.911517in}{1.771829in}}%
\pgfpathlineto{\pgfqpoint{2.927209in}{1.745064in}}%
\pgfpathlineto{\pgfqpoint{2.950942in}{1.709715in}}%
\pgfpathlineto{\pgfqpoint{2.977851in}{1.674798in}}%
\pgfpathlineto{\pgfqpoint{3.007886in}{1.640362in}}%
\pgfpathlineto{\pgfqpoint{3.040996in}{1.606457in}}%
\pgfpathlineto{\pgfqpoint{3.077125in}{1.573129in}}%
\pgfpathlineto{\pgfqpoint{3.116212in}{1.540424in}}%
\pgfpathlineto{\pgfqpoint{3.158192in}{1.508388in}}%
\pgfpathlineto{\pgfqpoint{3.202994in}{1.477064in}}%
\pgfpathlineto{\pgfqpoint{3.250548in}{1.446495in}}%
\pgfpathlineto{\pgfqpoint{3.300775in}{1.416720in}}%
\pgfpathlineto{\pgfqpoint{3.353596in}{1.387779in}}%
\pgfpathlineto{\pgfqpoint{3.408927in}{1.359710in}}%
\pgfpathlineto{\pgfqpoint{3.466683in}{1.332548in}}%
\pgfpathlineto{\pgfqpoint{3.526773in}{1.306330in}}%
\pgfpathlineto{\pgfqpoint{3.605028in}{1.274931in}}%
\pgfpathlineto{\pgfqpoint{3.686601in}{1.245117in}}%
\pgfpathlineto{\pgfqpoint{3.771302in}{1.216944in}}%
\pgfpathlineto{\pgfqpoint{3.858934in}{1.190464in}}%
\pgfpathlineto{\pgfqpoint{3.949297in}{1.165724in}}%
\pgfpathlineto{\pgfqpoint{4.042186in}{1.142769in}}%
\pgfpathlineto{\pgfqpoint{4.137393in}{1.121639in}}%
\pgfpathlineto{\pgfqpoint{4.234707in}{1.102368in}}%
\pgfpathlineto{\pgfqpoint{4.333916in}{1.084987in}}%
\pgfpathlineto{\pgfqpoint{4.434803in}{1.069522in}}%
\pgfpathlineto{\pgfqpoint{4.537153in}{1.055998in}}%
\pgfpathlineto{\pgfqpoint{4.640747in}{1.044430in}}%
\pgfpathlineto{\pgfqpoint{4.745367in}{1.034834in}}%
\pgfpathlineto{\pgfqpoint{4.850795in}{1.027219in}}%
\pgfpathlineto{\pgfqpoint{4.956815in}{1.021592in}}%
\pgfpathlineto{\pgfqpoint{5.063207in}{1.017953in}}%
\pgfpathlineto{\pgfqpoint{5.169757in}{1.016300in}}%
\pgfpathlineto{\pgfqpoint{5.276251in}{1.016627in}}%
\pgfpathlineto{\pgfqpoint{5.382477in}{1.018924in}}%
\pgfpathlineto{\pgfqpoint{5.488224in}{1.023177in}}%
\pgfpathlineto{\pgfqpoint{5.593286in}{1.029369in}}%
\pgfpathlineto{\pgfqpoint{5.697459in}{1.037479in}}%
\pgfpathlineto{\pgfqpoint{5.800542in}{1.047480in}}%
\pgfpathlineto{\pgfqpoint{5.902339in}{1.059347in}}%
\pgfpathlineto{\pgfqpoint{6.002656in}{1.073047in}}%
\pgfpathlineto{\pgfqpoint{6.101306in}{1.088544in}}%
\pgfpathlineto{\pgfqpoint{6.198105in}{1.105803in}}%
\pgfpathlineto{\pgfqpoint{6.292874in}{1.124781in}}%
\pgfpathlineto{\pgfqpoint{6.385440in}{1.145436in}}%
\pgfpathlineto{\pgfqpoint{6.475633in}{1.167719in}}%
\pgfpathlineto{\pgfqpoint{6.563293in}{1.191583in}}%
\pgfpathlineto{\pgfqpoint{6.648261in}{1.216976in}}%
\pgfpathlineto{\pgfqpoint{6.730388in}{1.243843in}}%
\pgfpathlineto{\pgfqpoint{6.809529in}{1.272128in}}%
\pgfpathlineto{\pgfqpoint{6.885546in}{1.301773in}}%
\pgfpathlineto{\pgfqpoint{6.958308in}{1.332716in}}%
\pgfpathlineto{\pgfqpoint{7.027691in}{1.364895in}}%
\pgfpathlineto{\pgfqpoint{7.093575in}{1.398246in}}%
\pgfpathlineto{\pgfqpoint{7.155852in}{1.432703in}}%
\pgfpathlineto{\pgfqpoint{7.214416in}{1.468198in}}%
\pgfpathlineto{\pgfqpoint{7.258529in}{1.497295in}}%
\pgfpathlineto{\pgfqpoint{7.300159in}{1.526976in}}%
\pgfpathlineto{\pgfqpoint{7.339261in}{1.557206in}}%
\pgfpathlineto{\pgfqpoint{7.375797in}{1.587947in}}%
\pgfpathlineto{\pgfqpoint{7.409728in}{1.619161in}}%
\pgfpathlineto{\pgfqpoint{7.441020in}{1.650813in}}%
\pgfpathlineto{\pgfqpoint{7.469643in}{1.682863in}}%
\pgfpathlineto{\pgfqpoint{7.495567in}{1.715274in}}%
\pgfpathlineto{\pgfqpoint{7.518768in}{1.748007in}}%
\pgfpathlineto{\pgfqpoint{7.539224in}{1.781023in}}%
\pgfpathlineto{\pgfqpoint{7.556916in}{1.814285in}}%
\pgfpathlineto{\pgfqpoint{7.571826in}{1.847753in}}%
\pgfpathlineto{\pgfqpoint{7.583943in}{1.881387in}}%
\pgfpathlineto{\pgfqpoint{7.593255in}{1.915150in}}%
\pgfpathlineto{\pgfqpoint{7.599756in}{1.949003in}}%
\pgfpathlineto{\pgfqpoint{7.603441in}{1.982905in}}%
\pgfpathlineto{\pgfqpoint{7.604309in}{2.016819in}}%
\pgfpathlineto{\pgfqpoint{7.602360in}{2.050705in}}%
\pgfpathlineto{\pgfqpoint{7.597599in}{2.084526in}}%
\pgfpathlineto{\pgfqpoint{7.590034in}{2.118241in}}%
\pgfpathlineto{\pgfqpoint{7.579674in}{2.151813in}}%
\pgfpathlineto{\pgfqpoint{7.566532in}{2.185204in}}%
\pgfpathlineto{\pgfqpoint{7.550623in}{2.218375in}}%
\pgfpathlineto{\pgfqpoint{7.531966in}{2.251289in}}%
\pgfpathlineto{\pgfqpoint{7.510581in}{2.283908in}}%
\pgfpathlineto{\pgfqpoint{7.486492in}{2.316196in}}%
\pgfpathlineto{\pgfqpoint{7.459725in}{2.348115in}}%
\pgfpathlineto{\pgfqpoint{7.430310in}{2.379629in}}%
\pgfpathlineto{\pgfqpoint{7.398278in}{2.410701in}}%
\pgfpathlineto{\pgfqpoint{7.363662in}{2.441297in}}%
\pgfpathlineto{\pgfqpoint{7.326500in}{2.471381in}}%
\pgfpathlineto{\pgfqpoint{7.286832in}{2.500918in}}%
\pgfpathlineto{\pgfqpoint{7.244697in}{2.529873in}}%
\pgfpathlineto{\pgfqpoint{7.200142in}{2.558214in}}%
\pgfpathlineto{\pgfqpoint{7.153213in}{2.585905in}}%
\pgfpathlineto{\pgfqpoint{7.091288in}{2.619558in}}%
\pgfpathlineto{\pgfqpoint{7.025831in}{2.652084in}}%
\pgfpathlineto{\pgfqpoint{6.956951in}{2.683420in}}%
\pgfpathlineto{\pgfqpoint{6.884760in}{2.713509in}}%
\pgfpathlineto{\pgfqpoint{6.809378in}{2.742290in}}%
\pgfpathlineto{\pgfqpoint{6.730930in}{2.769707in}}%
\pgfpathlineto{\pgfqpoint{6.649545in}{2.795706in}}%
\pgfpathlineto{\pgfqpoint{6.565360in}{2.820233in}}%
\pgfpathlineto{\pgfqpoint{6.478516in}{2.843238in}}%
\pgfpathlineto{\pgfqpoint{6.389162in}{2.864670in}}%
\pgfpathlineto{\pgfqpoint{6.297448in}{2.884484in}}%
\pgfpathlineto{\pgfqpoint{6.203534in}{2.902633in}}%
\pgfpathlineto{\pgfqpoint{6.107583in}{2.919076in}}%
\pgfpathlineto{\pgfqpoint{6.009763in}{2.933771in}}%
\pgfpathlineto{\pgfqpoint{5.910248in}{2.946682in}}%
\pgfpathlineto{\pgfqpoint{5.809216in}{2.957773in}}%
\pgfpathlineto{\pgfqpoint{5.706851in}{2.967011in}}%
\pgfpathlineto{\pgfqpoint{5.603341in}{2.974368in}}%
\pgfpathlineto{\pgfqpoint{5.498880in}{2.979816in}}%
\pgfpathlineto{\pgfqpoint{5.393665in}{2.983332in}}%
\pgfpathlineto{\pgfqpoint{5.287896in}{2.984896in}}%
\pgfpathlineto{\pgfqpoint{5.181780in}{2.984491in}}%
\pgfpathlineto{\pgfqpoint{5.075526in}{2.982103in}}%
\pgfpathlineto{\pgfqpoint{4.969347in}{2.977724in}}%
\pgfpathlineto{\pgfqpoint{4.863458in}{2.971347in}}%
\pgfpathlineto{\pgfqpoint{4.758079in}{2.962970in}}%
\pgfpathlineto{\pgfqpoint{4.653430in}{2.952597in}}%
\pgfpathlineto{\pgfqpoint{4.549734in}{2.940232in}}%
\pgfpathlineto{\pgfqpoint{4.447217in}{2.925888in}}%
\pgfpathlineto{\pgfqpoint{4.346103in}{2.909579in}}%
\pgfpathlineto{\pgfqpoint{4.246618in}{2.891326in}}%
\pgfpathlineto{\pgfqpoint{4.148989in}{2.871154in}}%
\pgfpathlineto{\pgfqpoint{4.053441in}{2.849091in}}%
\pgfpathlineto{\pgfqpoint{3.960197in}{2.825172in}}%
\pgfpathlineto{\pgfqpoint{3.869480in}{2.799438in}}%
\pgfpathlineto{\pgfqpoint{3.781507in}{2.771931in}}%
\pgfpathlineto{\pgfqpoint{3.696495in}{2.742701in}}%
\pgfpathlineto{\pgfqpoint{3.614655in}{2.711803in}}%
\pgfpathlineto{\pgfqpoint{3.536192in}{2.679296in}}%
\pgfpathlineto{\pgfqpoint{3.475987in}{2.652176in}}%
\pgfpathlineto{\pgfqpoint{3.418171in}{2.624101in}}%
\pgfpathlineto{\pgfqpoint{3.362840in}{2.595110in}}%
\pgfpathlineto{\pgfqpoint{3.310086in}{2.565241in}}%
\pgfpathlineto{\pgfqpoint{3.259999in}{2.534536in}}%
\pgfpathlineto{\pgfqpoint{3.212664in}{2.503037in}}%
\pgfpathlineto{\pgfqpoint{3.168162in}{2.470787in}}%
\pgfpathlineto{\pgfqpoint{3.126571in}{2.437834in}}%
\pgfpathlineto{\pgfqpoint{3.087961in}{2.404224in}}%
\pgfpathlineto{\pgfqpoint{3.052401in}{2.370007in}}%
\pgfpathlineto{\pgfqpoint{3.019954in}{2.335233in}}%
\pgfpathlineto{\pgfqpoint{2.990675in}{2.299953in}}%
\pgfpathlineto{\pgfqpoint{2.964616in}{2.264221in}}%
\pgfpathlineto{\pgfqpoint{2.941824in}{2.228089in}}%
\pgfpathlineto{\pgfqpoint{2.926898in}{2.200762in}}%
\pgfpathlineto{\pgfqpoint{2.913847in}{2.173264in}}%
\pgfpathlineto{\pgfqpoint{2.902684in}{2.145620in}}%
\pgfpathlineto{\pgfqpoint{2.893419in}{2.117852in}}%
\pgfpathlineto{\pgfqpoint{2.886063in}{2.089986in}}%
\pgfpathlineto{\pgfqpoint{2.880621in}{2.062045in}}%
\pgfpathlineto{\pgfqpoint{2.877099in}{2.034052in}}%
\pgfpathlineto{\pgfqpoint{2.875500in}{2.006033in}}%
\pgfpathlineto{\pgfqpoint{2.875826in}{1.978012in}}%
\pgfpathlineto{\pgfqpoint{2.878076in}{1.950012in}}%
\pgfpathlineto{\pgfqpoint{2.882246in}{1.922059in}}%
\pgfpathlineto{\pgfqpoint{2.888332in}{1.894175in}}%
\pgfpathlineto{\pgfqpoint{2.896329in}{1.866385in}}%
\pgfpathlineto{\pgfqpoint{2.906226in}{1.838713in}}%
\pgfpathlineto{\pgfqpoint{2.918014in}{1.811182in}}%
\pgfpathlineto{\pgfqpoint{2.931680in}{1.783816in}}%
\pgfpathlineto{\pgfqpoint{2.947210in}{1.756638in}}%
\pgfpathlineto{\pgfqpoint{2.964588in}{1.729671in}}%
\pgfpathlineto{\pgfqpoint{2.990603in}{1.694082in}}%
\pgfpathlineto{\pgfqpoint{3.019825in}{1.658962in}}%
\pgfpathlineto{\pgfqpoint{3.052202in}{1.624362in}}%
\pgfpathlineto{\pgfqpoint{3.087676in}{1.590332in}}%
\pgfpathlineto{\pgfqpoint{3.126184in}{1.556922in}}%
\pgfpathlineto{\pgfqpoint{3.167660in}{1.524180in}}%
\pgfpathlineto{\pgfqpoint{3.212032in}{1.492152in}}%
\pgfpathlineto{\pgfqpoint{3.259224in}{1.460882in}}%
\pgfpathlineto{\pgfqpoint{3.309156in}{1.430415in}}%
\pgfpathlineto{\pgfqpoint{3.361744in}{1.400792in}}%
\pgfpathlineto{\pgfqpoint{3.416900in}{1.372052in}}%
\pgfpathlineto{\pgfqpoint{3.474533in}{1.344234in}}%
\pgfpathlineto{\pgfqpoint{3.534549in}{1.317374in}}%
\pgfpathlineto{\pgfqpoint{3.596852in}{1.291505in}}%
\pgfpathlineto{\pgfqpoint{3.677794in}{1.260615in}}%
\pgfpathlineto{\pgfqpoint{3.761952in}{1.231384in}}%
\pgfpathlineto{\pgfqpoint{3.849122in}{1.203868in}}%
\pgfpathlineto{\pgfqpoint{3.939095in}{1.178117in}}%
\pgfpathlineto{\pgfqpoint{4.031656in}{1.154176in}}%
\pgfpathlineto{\pgfqpoint{4.126591in}{1.132085in}}%
\pgfpathlineto{\pgfqpoint{4.223680in}{1.111880in}}%
\pgfpathlineto{\pgfqpoint{4.322702in}{1.093592in}}%
\pgfpathlineto{\pgfqpoint{4.423435in}{1.077247in}}%
\pgfpathlineto{\pgfqpoint{4.525657in}{1.062868in}}%
\pgfpathlineto{\pgfqpoint{4.629144in}{1.050472in}}%
\pgfpathlineto{\pgfqpoint{4.733674in}{1.040072in}}%
\pgfpathlineto{\pgfqpoint{4.839024in}{1.031677in}}%
\pgfpathlineto{\pgfqpoint{4.944974in}{1.025291in}}%
\pgfpathlineto{\pgfqpoint{5.051303in}{1.020915in}}%
\pgfpathlineto{\pgfqpoint{5.157795in}{1.018545in}}%
\pgfpathlineto{\pgfqpoint{5.264235in}{1.018173in}}%
\pgfpathlineto{\pgfqpoint{5.370410in}{1.019789in}}%
\pgfpathlineto{\pgfqpoint{5.476111in}{1.023378in}}%
\pgfpathlineto{\pgfqpoint{5.581131in}{1.028920in}}%
\pgfpathlineto{\pgfqpoint{5.685268in}{1.036394in}}%
\pgfpathlineto{\pgfqpoint{5.788324in}{1.045773in}}%
\pgfpathlineto{\pgfqpoint{5.890104in}{1.057031in}}%
\pgfpathlineto{\pgfqpoint{5.990418in}{1.070133in}}%
\pgfpathlineto{\pgfqpoint{6.089081in}{1.085046in}}%
\pgfpathlineto{\pgfqpoint{6.185911in}{1.101732in}}%
\pgfpathlineto{\pgfqpoint{6.280733in}{1.120149in}}%
\pgfpathlineto{\pgfqpoint{6.373376in}{1.140255in}}%
\pgfpathlineto{\pgfqpoint{6.463676in}{1.162003in}}%
\pgfpathlineto{\pgfqpoint{6.551473in}{1.185344in}}%
\pgfpathlineto{\pgfqpoint{6.636612in}{1.210228in}}%
\pgfpathlineto{\pgfqpoint{6.718947in}{1.236602in}}%
\pgfpathlineto{\pgfqpoint{6.798334in}{1.264410in}}%
\pgfpathlineto{\pgfqpoint{6.874638in}{1.293593in}}%
\pgfpathlineto{\pgfqpoint{6.947730in}{1.324094in}}%
\pgfpathlineto{\pgfqpoint{7.017485in}{1.355850in}}%
\pgfpathlineto{\pgfqpoint{7.083788in}{1.388799in}}%
\pgfpathlineto{\pgfqpoint{7.146529in}{1.422875in}}%
\pgfpathlineto{\pgfqpoint{7.205603in}{1.458013in}}%
\pgfpathlineto{\pgfqpoint{7.250158in}{1.486842in}}%
\pgfpathlineto{\pgfqpoint{7.292258in}{1.516272in}}%
\pgfpathlineto{\pgfqpoint{7.331860in}{1.546266in}}%
\pgfpathlineto{\pgfqpoint{7.368924in}{1.576789in}}%
\pgfpathlineto{\pgfqpoint{7.403411in}{1.607805in}}%
\pgfpathlineto{\pgfqpoint{7.435287in}{1.639275in}}%
\pgfpathlineto{\pgfqpoint{7.464519in}{1.671163in}}%
\pgfpathlineto{\pgfqpoint{7.491080in}{1.703431in}}%
\pgfpathlineto{\pgfqpoint{7.514942in}{1.736041in}}%
\pgfpathlineto{\pgfqpoint{7.536083in}{1.768955in}}%
\pgfpathlineto{\pgfqpoint{7.554482in}{1.802134in}}%
\pgfpathlineto{\pgfqpoint{7.570122in}{1.835539in}}%
\pgfpathlineto{\pgfqpoint{7.582989in}{1.869132in}}%
\pgfpathlineto{\pgfqpoint{7.593072in}{1.902874in}}%
\pgfpathlineto{\pgfqpoint{7.600361in}{1.936727in}}%
\pgfpathlineto{\pgfqpoint{7.604852in}{1.970650in}}%
\pgfpathlineto{\pgfqpoint{7.606542in}{2.004605in}}%
\pgfpathlineto{\pgfqpoint{7.605431in}{2.038553in}}%
\pgfpathlineto{\pgfqpoint{7.601523in}{2.072456in}}%
\pgfpathlineto{\pgfqpoint{7.594824in}{2.106274in}}%
\pgfpathlineto{\pgfqpoint{7.585342in}{2.139969in}}%
\pgfpathlineto{\pgfqpoint{7.573090in}{2.173503in}}%
\pgfpathlineto{\pgfqpoint{7.558083in}{2.206837in}}%
\pgfpathlineto{\pgfqpoint{7.540338in}{2.239932in}}%
\pgfpathlineto{\pgfqpoint{7.519875in}{2.272753in}}%
\pgfpathlineto{\pgfqpoint{7.496717in}{2.305260in}}%
\pgfpathlineto{\pgfqpoint{7.470891in}{2.337417in}}%
\pgfpathlineto{\pgfqpoint{7.442425in}{2.369187in}}%
\pgfpathlineto{\pgfqpoint{7.411350in}{2.400534in}}%
\pgfpathlineto{\pgfqpoint{7.377701in}{2.431422in}}%
\pgfpathlineto{\pgfqpoint{7.341513in}{2.461815in}}%
\pgfpathlineto{\pgfqpoint{7.302826in}{2.491678in}}%
\pgfpathlineto{\pgfqpoint{7.261683in}{2.520976in}}%
\pgfpathlineto{\pgfqpoint{7.218127in}{2.549676in}}%
\pgfpathlineto{\pgfqpoint{7.172204in}{2.577744in}}%
\pgfpathlineto{\pgfqpoint{7.111550in}{2.611891in}}%
\pgfpathlineto{\pgfqpoint{7.047377in}{2.644935in}}%
\pgfpathlineto{\pgfqpoint{6.979795in}{2.676816in}}%
\pgfpathlineto{\pgfqpoint{6.908915in}{2.707475in}}%
\pgfpathlineto{\pgfqpoint{6.834858in}{2.736852in}}%
\pgfpathlineto{\pgfqpoint{6.757747in}{2.764892in}}%
\pgfpathlineto{\pgfqpoint{6.677713in}{2.791540in}}%
\pgfpathlineto{\pgfqpoint{6.594890in}{2.816743in}}%
\pgfpathlineto{\pgfqpoint{6.509420in}{2.840452in}}%
\pgfpathlineto{\pgfqpoint{6.421448in}{2.862618in}}%
\pgfpathlineto{\pgfqpoint{6.331125in}{2.883194in}}%
\pgfpathlineto{\pgfqpoint{6.238606in}{2.902136in}}%
\pgfpathlineto{\pgfqpoint{6.144052in}{2.919404in}}%
\pgfpathlineto{\pgfqpoint{6.047627in}{2.934956in}}%
\pgfpathlineto{\pgfqpoint{5.949501in}{2.948757in}}%
\pgfpathlineto{\pgfqpoint{5.849847in}{2.960773in}}%
\pgfpathlineto{\pgfqpoint{5.748845in}{2.970971in}}%
\pgfpathlineto{\pgfqpoint{5.646676in}{2.979322in}}%
\pgfpathlineto{\pgfqpoint{5.543526in}{2.985802in}}%
\pgfpathlineto{\pgfqpoint{5.439585in}{2.990386in}}%
\pgfpathlineto{\pgfqpoint{5.335046in}{2.993055in}}%
\pgfpathlineto{\pgfqpoint{5.230107in}{2.993792in}}%
\pgfpathlineto{\pgfqpoint{5.124968in}{2.992583in}}%
\pgfpathlineto{\pgfqpoint{5.019831in}{2.989418in}}%
\pgfpathlineto{\pgfqpoint{4.914902in}{2.984290in}}%
\pgfpathlineto{\pgfqpoint{4.810390in}{2.977196in}}%
\pgfpathlineto{\pgfqpoint{4.706505in}{2.968136in}}%
\pgfpathlineto{\pgfqpoint{4.603458in}{2.957114in}}%
\pgfpathlineto{\pgfqpoint{4.501465in}{2.944139in}}%
\pgfpathlineto{\pgfqpoint{4.400738in}{2.929223in}}%
\pgfpathlineto{\pgfqpoint{4.301495in}{2.912382in}}%
\pgfpathlineto{\pgfqpoint{4.203951in}{2.893635in}}%
\pgfpathlineto{\pgfqpoint{4.108321in}{2.873009in}}%
\pgfpathlineto{\pgfqpoint{4.014820in}{2.850532in}}%
\pgfpathlineto{\pgfqpoint{3.923662in}{2.826238in}}%
\pgfpathlineto{\pgfqpoint{3.835060in}{2.800165in}}%
\pgfpathlineto{\pgfqpoint{3.749223in}{2.772355in}}%
\pgfpathlineto{\pgfqpoint{3.666356in}{2.742856in}}%
\pgfpathlineto{\pgfqpoint{3.586665in}{2.711720in}}%
\pgfpathlineto{\pgfqpoint{3.510346in}{2.679004in}}%
\pgfpathlineto{\pgfqpoint{3.451850in}{2.651734in}}%
\pgfpathlineto{\pgfqpoint{3.395733in}{2.623525in}}%
\pgfpathlineto{\pgfqpoint{3.342089in}{2.594413in}}%
\pgfpathlineto{\pgfqpoint{3.291009in}{2.564435in}}%
\pgfpathlineto{\pgfqpoint{3.242581in}{2.533630in}}%
\pgfpathlineto{\pgfqpoint{3.196889in}{2.502039in}}%
\pgfpathlineto{\pgfqpoint{3.154015in}{2.469704in}}%
\pgfpathlineto{\pgfqpoint{3.114034in}{2.436670in}}%
\pgfpathlineto{\pgfqpoint{3.077020in}{2.402983in}}%
\pgfpathlineto{\pgfqpoint{3.043041in}{2.368690in}}%
\pgfpathlineto{\pgfqpoint{3.012161in}{2.333840in}}%
\pgfpathlineto{\pgfqpoint{2.984439in}{2.298484in}}%
\pgfpathlineto{\pgfqpoint{2.959928in}{2.262673in}}%
\pgfpathlineto{\pgfqpoint{2.943682in}{2.235547in}}%
\pgfpathlineto{\pgfqpoint{2.929288in}{2.208219in}}%
\pgfpathlineto{\pgfqpoint{2.916764in}{2.180710in}}%
\pgfpathlineto{\pgfqpoint{2.906123in}{2.153044in}}%
\pgfpathlineto{\pgfqpoint{2.897379in}{2.125244in}}%
\pgfpathlineto{\pgfqpoint{2.890543in}{2.097334in}}%
\pgfpathlineto{\pgfqpoint{2.885625in}{2.069339in}}%
\pgfpathlineto{\pgfqpoint{2.882630in}{2.041283in}}%
\pgfpathlineto{\pgfqpoint{2.881566in}{2.013189in}}%
\pgfpathlineto{\pgfqpoint{2.882434in}{1.985082in}}%
\pgfpathlineto{\pgfqpoint{2.885237in}{1.956987in}}%
\pgfpathlineto{\pgfqpoint{2.889973in}{1.928928in}}%
\pgfpathlineto{\pgfqpoint{2.896639in}{1.900930in}}%
\pgfpathlineto{\pgfqpoint{2.905232in}{1.873017in}}%
\pgfpathlineto{\pgfqpoint{2.915743in}{1.845214in}}%
\pgfpathlineto{\pgfqpoint{2.928165in}{1.817546in}}%
\pgfpathlineto{\pgfqpoint{2.942486in}{1.790035in}}%
\pgfpathlineto{\pgfqpoint{2.958694in}{1.762706in}}%
\pgfpathlineto{\pgfqpoint{2.976774in}{1.735584in}}%
\pgfpathlineto{\pgfqpoint{3.003762in}{1.699782in}}%
\pgfpathlineto{\pgfqpoint{3.034003in}{1.664444in}}%
\pgfpathlineto{\pgfqpoint{3.067446in}{1.629623in}}%
\pgfpathlineto{\pgfqpoint{3.104033in}{1.595373in}}%
\pgfpathlineto{\pgfqpoint{3.143704in}{1.561745in}}%
\pgfpathlineto{\pgfqpoint{3.186390in}{1.528790in}}%
\pgfpathlineto{\pgfqpoint{3.232018in}{1.496557in}}%
\pgfpathlineto{\pgfqpoint{3.280511in}{1.465093in}}%
\pgfpathlineto{\pgfqpoint{3.331787in}{1.434444in}}%
\pgfpathlineto{\pgfqpoint{3.385758in}{1.404654in}}%
\pgfpathlineto{\pgfqpoint{3.442334in}{1.375765in}}%
\pgfpathlineto{\pgfqpoint{3.501419in}{1.347818in}}%
\pgfpathlineto{\pgfqpoint{3.562916in}{1.320850in}}%
\pgfpathlineto{\pgfqpoint{3.626722in}{1.294899in}}%
\pgfpathlineto{\pgfqpoint{3.709568in}{1.263940in}}%
\pgfpathlineto{\pgfqpoint{3.795647in}{1.234685in}}%
\pgfpathlineto{\pgfqpoint{3.884744in}{1.207190in}}%
\pgfpathlineto{\pgfqpoint{3.976636in}{1.181506in}}%
\pgfpathlineto{\pgfqpoint{4.071098in}{1.157682in}}%
\pgfpathlineto{\pgfqpoint{4.167900in}{1.135757in}}%
\pgfpathlineto{\pgfqpoint{4.266812in}{1.115769in}}%
\pgfpathlineto{\pgfqpoint{4.367600in}{1.097747in}}%
\pgfpathlineto{\pgfqpoint{4.470028in}{1.081718in}}%
\pgfpathlineto{\pgfqpoint{4.573864in}{1.067702in}}%
\pgfpathlineto{\pgfqpoint{4.678871in}{1.055716in}}%
\pgfpathlineto{\pgfqpoint{4.784817in}{1.045769in}}%
\pgfpathlineto{\pgfqpoint{4.891470in}{1.037869in}}%
\pgfpathlineto{\pgfqpoint{4.998598in}{1.032017in}}%
\pgfpathlineto{\pgfqpoint{5.105976in}{1.028210in}}%
\pgfpathlineto{\pgfqpoint{5.213377in}{1.026442in}}%
\pgfpathlineto{\pgfqpoint{5.320582in}{1.026700in}}%
\pgfpathlineto{\pgfqpoint{5.427371in}{1.028971in}}%
\pgfpathlineto{\pgfqpoint{5.533532in}{1.033234in}}%
\pgfpathlineto{\pgfqpoint{5.638855in}{1.039468in}}%
\pgfpathlineto{\pgfqpoint{5.743135in}{1.047646in}}%
\pgfpathlineto{\pgfqpoint{5.846174in}{1.057739in}}%
\pgfpathlineto{\pgfqpoint{5.947775in}{1.069714in}}%
\pgfpathlineto{\pgfqpoint{6.047751in}{1.083536in}}%
\pgfpathlineto{\pgfqpoint{6.145916in}{1.099165in}}%
\pgfpathlineto{\pgfqpoint{6.242093in}{1.116560in}}%
\pgfpathlineto{\pgfqpoint{6.336110in}{1.135676in}}%
\pgfpathlineto{\pgfqpoint{6.427798in}{1.156468in}}%
\pgfpathlineto{\pgfqpoint{6.516999in}{1.178886in}}%
\pgfpathlineto{\pgfqpoint{6.603558in}{1.202879in}}%
\pgfpathlineto{\pgfqpoint{6.687325in}{1.228392in}}%
\pgfpathlineto{\pgfqpoint{6.768159in}{1.255371in}}%
\pgfpathlineto{\pgfqpoint{6.845925in}{1.283757in}}%
\pgfpathlineto{\pgfqpoint{6.920492in}{1.313493in}}%
\pgfpathlineto{\pgfqpoint{6.991738in}{1.344515in}}%
\pgfpathlineto{\pgfqpoint{7.059547in}{1.376762in}}%
\pgfpathlineto{\pgfqpoint{7.123809in}{1.410170in}}%
\pgfpathlineto{\pgfqpoint{7.184421in}{1.444672in}}%
\pgfpathlineto{\pgfqpoint{7.230217in}{1.473017in}}%
\pgfpathlineto{\pgfqpoint{7.273568in}{1.501985in}}%
\pgfpathlineto{\pgfqpoint{7.314432in}{1.531540in}}%
\pgfpathlineto{\pgfqpoint{7.352767in}{1.561648in}}%
\pgfpathlineto{\pgfqpoint{7.388535in}{1.592271in}}%
\pgfpathlineto{\pgfqpoint{7.421701in}{1.623373in}}%
\pgfpathlineto{\pgfqpoint{7.452232in}{1.654917in}}%
\pgfpathlineto{\pgfqpoint{7.480100in}{1.686866in}}%
\pgfpathlineto{\pgfqpoint{7.505277in}{1.719182in}}%
\pgfpathlineto{\pgfqpoint{7.527740in}{1.751828in}}%
\pgfpathlineto{\pgfqpoint{7.547466in}{1.784766in}}%
\pgfpathlineto{\pgfqpoint{7.564439in}{1.817957in}}%
\pgfpathlineto{\pgfqpoint{7.578643in}{1.851363in}}%
\pgfpathlineto{\pgfqpoint{7.590065in}{1.884946in}}%
\pgfpathlineto{\pgfqpoint{7.598695in}{1.918666in}}%
\pgfpathlineto{\pgfqpoint{7.604528in}{1.952486in}}%
\pgfpathlineto{\pgfqpoint{7.607558in}{1.986366in}}%
\pgfpathlineto{\pgfqpoint{7.607786in}{2.020268in}}%
\pgfpathlineto{\pgfqpoint{7.605212in}{2.054153in}}%
\pgfpathlineto{\pgfqpoint{7.599842in}{2.087983in}}%
\pgfpathlineto{\pgfqpoint{7.591683in}{2.121718in}}%
\pgfpathlineto{\pgfqpoint{7.580745in}{2.155321in}}%
\pgfpathlineto{\pgfqpoint{7.567042in}{2.188753in}}%
\pgfpathlineto{\pgfqpoint{7.550591in}{2.221975in}}%
\pgfpathlineto{\pgfqpoint{7.531409in}{2.254951in}}%
\pgfpathlineto{\pgfqpoint{7.509519in}{2.287642in}}%
\pgfpathlineto{\pgfqpoint{7.484945in}{2.320012in}}%
\pgfpathlineto{\pgfqpoint{7.457715in}{2.352022in}}%
\pgfpathlineto{\pgfqpoint{7.427858in}{2.383636in}}%
\pgfpathlineto{\pgfqpoint{7.395409in}{2.414819in}}%
\pgfpathlineto{\pgfqpoint{7.360401in}{2.445533in}}%
\pgfpathlineto{\pgfqpoint{7.322874in}{2.475744in}}%
\pgfpathlineto{\pgfqpoint{7.282868in}{2.505416in}}%
\pgfpathlineto{\pgfqpoint{7.240428in}{2.534515in}}%
\pgfpathlineto{\pgfqpoint{7.195599in}{2.563007in}}%
\pgfpathlineto{\pgfqpoint{7.136278in}{2.597717in}}%
\pgfpathlineto{\pgfqpoint{7.073401in}{2.631363in}}%
\pgfpathlineto{\pgfqpoint{7.007077in}{2.663882in}}%
\pgfpathlineto{\pgfqpoint{6.937417in}{2.695214in}}%
\pgfpathlineto{\pgfqpoint{6.864542in}{2.725298in}}%
\pgfpathlineto{\pgfqpoint{6.788574in}{2.754079in}}%
\pgfpathlineto{\pgfqpoint{6.709644in}{2.781501in}}%
\pgfpathlineto{\pgfqpoint{6.627889in}{2.807510in}}%
\pgfpathlineto{\pgfqpoint{6.543448in}{2.832056in}}%
\pgfpathlineto{\pgfqpoint{6.456467in}{2.855090in}}%
\pgfpathlineto{\pgfqpoint{6.367098in}{2.876565in}}%
\pgfpathlineto{\pgfqpoint{6.275497in}{2.896437in}}%
\pgfpathlineto{\pgfqpoint{6.181824in}{2.914663in}}%
\pgfpathlineto{\pgfqpoint{6.086243in}{2.931204in}}%
\pgfpathlineto{\pgfqpoint{5.988924in}{2.946025in}}%
\pgfpathlineto{\pgfqpoint{5.890040in}{2.959089in}}%
\pgfpathlineto{\pgfqpoint{5.789769in}{2.970367in}}%
\pgfpathlineto{\pgfqpoint{5.688290in}{2.979829in}}%
\pgfpathlineto{\pgfqpoint{5.585789in}{2.987449in}}%
\pgfpathlineto{\pgfqpoint{5.482453in}{2.993206in}}%
\pgfpathlineto{\pgfqpoint{5.378473in}{2.997079in}}%
\pgfpathlineto{\pgfqpoint{5.274042in}{2.999051in}}%
\pgfpathlineto{\pgfqpoint{5.169357in}{2.999110in}}%
\pgfpathlineto{\pgfqpoint{5.064615in}{2.997245in}}%
\pgfpathlineto{\pgfqpoint{4.960018in}{2.993450in}}%
\pgfpathlineto{\pgfqpoint{4.855767in}{2.987720in}}%
\pgfpathlineto{\pgfqpoint{4.752068in}{2.980057in}}%
\pgfpathlineto{\pgfqpoint{4.649124in}{2.970464in}}%
\pgfpathlineto{\pgfqpoint{4.547142in}{2.958947in}}%
\pgfpathlineto{\pgfqpoint{4.446330in}{2.945519in}}%
\pgfpathlineto{\pgfqpoint{4.346893in}{2.930194in}}%
\pgfpathlineto{\pgfqpoint{4.249039in}{2.912990in}}%
\pgfpathlineto{\pgfqpoint{4.152975in}{2.893930in}}%
\pgfpathlineto{\pgfqpoint{4.058907in}{2.873040in}}%
\pgfpathlineto{\pgfqpoint{3.967038in}{2.850352in}}%
\pgfpathlineto{\pgfqpoint{3.877571in}{2.825898in}}%
\pgfpathlineto{\pgfqpoint{3.790708in}{2.799719in}}%
\pgfpathlineto{\pgfqpoint{3.706647in}{2.771855in}}%
\pgfpathlineto{\pgfqpoint{3.625582in}{2.742355in}}%
\pgfpathlineto{\pgfqpoint{3.547704in}{2.711270in}}%
\pgfpathlineto{\pgfqpoint{3.473201in}{2.678653in}}%
\pgfpathlineto{\pgfqpoint{3.416151in}{2.651498in}}%
\pgfpathlineto{\pgfqpoint{3.361469in}{2.623433in}}%
\pgfpathlineto{\pgfqpoint{3.309242in}{2.594493in}}%
\pgfpathlineto{\pgfqpoint{3.259558in}{2.564713in}}%
\pgfpathlineto{\pgfqpoint{3.212498in}{2.534130in}}%
\pgfpathlineto{\pgfqpoint{3.168144in}{2.502783in}}%
\pgfpathlineto{\pgfqpoint{3.126572in}{2.470713in}}%
\pgfpathlineto{\pgfqpoint{3.087855in}{2.437962in}}%
\pgfpathlineto{\pgfqpoint{3.052063in}{2.404574in}}%
\pgfpathlineto{\pgfqpoint{3.019262in}{2.370592in}}%
\pgfpathlineto{\pgfqpoint{2.989515in}{2.336064in}}%
\pgfpathlineto{\pgfqpoint{2.962878in}{2.301038in}}%
\pgfpathlineto{\pgfqpoint{2.939404in}{2.265562in}}%
\pgfpathlineto{\pgfqpoint{2.923904in}{2.238690in}}%
\pgfpathlineto{\pgfqpoint{2.910230in}{2.211615in}}%
\pgfpathlineto{\pgfqpoint{2.898398in}{2.184359in}}%
\pgfpathlineto{\pgfqpoint{2.888424in}{2.156945in}}%
\pgfpathlineto{\pgfqpoint{2.880322in}{2.129394in}}%
\pgfpathlineto{\pgfqpoint{2.874103in}{2.101729in}}%
\pgfpathlineto{\pgfqpoint{2.869777in}{2.073974in}}%
\pgfpathlineto{\pgfqpoint{2.867354in}{2.046153in}}%
\pgfpathlineto{\pgfqpoint{2.866840in}{2.018287in}}%
\pgfpathlineto{\pgfqpoint{2.868239in}{1.990401in}}%
\pgfpathlineto{\pgfqpoint{2.871555in}{1.962520in}}%
\pgfpathlineto{\pgfqpoint{2.876789in}{1.934666in}}%
\pgfpathlineto{\pgfqpoint{2.883939in}{1.906863in}}%
\pgfpathlineto{\pgfqpoint{2.893003in}{1.879136in}}%
\pgfpathlineto{\pgfqpoint{2.903976in}{1.851508in}}%
\pgfpathlineto{\pgfqpoint{2.916851in}{1.824004in}}%
\pgfpathlineto{\pgfqpoint{2.931620in}{1.796648in}}%
\pgfpathlineto{\pgfqpoint{2.948272in}{1.769462in}}%
\pgfpathlineto{\pgfqpoint{2.966794in}{1.742472in}}%
\pgfpathlineto{\pgfqpoint{2.994375in}{1.706829in}}%
\pgfpathlineto{\pgfqpoint{3.025215in}{1.671630in}}%
\pgfpathlineto{\pgfqpoint{3.059268in}{1.636929in}}%
\pgfpathlineto{\pgfqpoint{3.096483in}{1.602780in}}%
\pgfpathlineto{\pgfqpoint{3.136802in}{1.569236in}}%
\pgfpathlineto{\pgfqpoint{3.180161in}{1.536348in}}%
\pgfpathlineto{\pgfqpoint{3.226492in}{1.504167in}}%
\pgfpathlineto{\pgfqpoint{3.275720in}{1.472742in}}%
\pgfpathlineto{\pgfqpoint{3.327767in}{1.442121in}}%
\pgfpathlineto{\pgfqpoint{3.382546in}{1.412348in}}%
\pgfpathlineto{\pgfqpoint{3.439971in}{1.383469in}}%
\pgfpathlineto{\pgfqpoint{3.499946in}{1.355526in}}%
\pgfpathlineto{\pgfqpoint{3.562374in}{1.328559in}}%
\pgfpathlineto{\pgfqpoint{3.627154in}{1.302607in}}%
\pgfpathlineto{\pgfqpoint{3.694179in}{1.277706in}}%
\pgfpathlineto{\pgfqpoint{3.780954in}{1.248110in}}%
\pgfpathlineto{\pgfqpoint{3.870849in}{1.220273in}}%
\pgfpathlineto{\pgfqpoint{3.963639in}{1.194251in}}%
\pgfpathlineto{\pgfqpoint{4.059093in}{1.170097in}}%
\pgfpathlineto{\pgfqpoint{4.156975in}{1.147855in}}%
\pgfpathlineto{\pgfqpoint{4.257047in}{1.127567in}}%
\pgfpathlineto{\pgfqpoint{4.359065in}{1.109267in}}%
\pgfpathlineto{\pgfqpoint{4.462785in}{1.092984in}}%
\pgfpathlineto{\pgfqpoint{4.567963in}{1.078743in}}%
\pgfpathlineto{\pgfqpoint{4.674353in}{1.066560in}}%
\pgfpathlineto{\pgfqpoint{4.781710in}{1.056450in}}%
\pgfpathlineto{\pgfqpoint{4.889790in}{1.048419in}}%
\pgfpathlineto{\pgfqpoint{4.998350in}{1.042471in}}%
\pgfpathlineto{\pgfqpoint{5.107153in}{1.038603in}}%
\pgfpathlineto{\pgfqpoint{5.215961in}{1.036808in}}%
\pgfpathlineto{\pgfqpoint{5.324543in}{1.037074in}}%
\pgfpathlineto{\pgfqpoint{5.432669in}{1.039385in}}%
\pgfpathlineto{\pgfqpoint{5.540116in}{1.043721in}}%
\pgfpathlineto{\pgfqpoint{5.646666in}{1.050058in}}%
\pgfpathlineto{\pgfqpoint{5.752104in}{1.058366in}}%
\pgfpathlineto{\pgfqpoint{5.856224in}{1.068616in}}%
\pgfpathlineto{\pgfqpoint{5.958823in}{1.080770in}}%
\pgfpathlineto{\pgfqpoint{6.059706in}{1.094792in}}%
\pgfpathlineto{\pgfqpoint{6.158683in}{1.110638in}}%
\pgfpathlineto{\pgfqpoint{6.255572in}{1.128265in}}%
\pgfpathlineto{\pgfqpoint{6.350197in}{1.147626in}}%
\pgfpathlineto{\pgfqpoint{6.442388in}{1.168670in}}%
\pgfpathlineto{\pgfqpoint{6.531983in}{1.191346in}}%
\pgfpathlineto{\pgfqpoint{6.618826in}{1.215598in}}%
\pgfpathlineto{\pgfqpoint{6.702769in}{1.241372in}}%
\pgfpathlineto{\pgfqpoint{6.783670in}{1.268607in}}%
\pgfpathlineto{\pgfqpoint{6.861395in}{1.297245in}}%
\pgfpathlineto{\pgfqpoint{6.935815in}{1.327222in}}%
\pgfpathlineto{\pgfqpoint{7.006810in}{1.358476in}}%
\pgfpathlineto{\pgfqpoint{7.074268in}{1.390942in}}%
\pgfpathlineto{\pgfqpoint{7.138081in}{1.424554in}}%
\pgfpathlineto{\pgfqpoint{7.198149in}{1.459244in}}%
\pgfpathlineto{\pgfqpoint{7.243446in}{1.487726in}}%
\pgfpathlineto{\pgfqpoint{7.286243in}{1.516818in}}%
\pgfpathlineto{\pgfqpoint{7.326499in}{1.546485in}}%
\pgfpathlineto{\pgfqpoint{7.364175in}{1.576691in}}%
\pgfpathlineto{\pgfqpoint{7.399236in}{1.607398in}}%
\pgfpathlineto{\pgfqpoint{7.431650in}{1.638570in}}%
\pgfpathlineto{\pgfqpoint{7.461385in}{1.670170in}}%
\pgfpathlineto{\pgfqpoint{7.488415in}{1.702160in}}%
\pgfpathlineto{\pgfqpoint{7.512715in}{1.734502in}}%
\pgfpathlineto{\pgfqpoint{7.534264in}{1.767159in}}%
\pgfpathlineto{\pgfqpoint{7.553043in}{1.800093in}}%
\pgfpathlineto{\pgfqpoint{7.569037in}{1.833265in}}%
\pgfpathlineto{\pgfqpoint{7.582231in}{1.866638in}}%
\pgfpathlineto{\pgfqpoint{7.592616in}{1.900172in}}%
\pgfpathlineto{\pgfqpoint{7.600184in}{1.933830in}}%
\pgfpathlineto{\pgfqpoint{7.604930in}{1.967572in}}%
\pgfpathlineto{\pgfqpoint{7.606852in}{2.001362in}}%
\pgfpathlineto{\pgfqpoint{7.605952in}{2.035159in}}%
\pgfpathlineto{\pgfqpoint{7.602232in}{2.068926in}}%
\pgfpathlineto{\pgfqpoint{7.595699in}{2.102624in}}%
\pgfpathlineto{\pgfqpoint{7.586362in}{2.136215in}}%
\pgfpathlineto{\pgfqpoint{7.574233in}{2.169661in}}%
\pgfpathlineto{\pgfqpoint{7.559327in}{2.202925in}}%
\pgfpathlineto{\pgfqpoint{7.541662in}{2.235967in}}%
\pgfpathlineto{\pgfqpoint{7.521257in}{2.268750in}}%
\pgfpathlineto{\pgfqpoint{7.498135in}{2.301238in}}%
\pgfpathlineto{\pgfqpoint{7.472323in}{2.333393in}}%
\pgfpathlineto{\pgfqpoint{7.443849in}{2.365178in}}%
\pgfpathlineto{\pgfqpoint{7.412745in}{2.396557in}}%
\pgfpathlineto{\pgfqpoint{7.379043in}{2.427493in}}%
\pgfpathlineto{\pgfqpoint{7.342782in}{2.457951in}}%
\pgfpathlineto{\pgfqpoint{7.304000in}{2.487896in}}%
\pgfpathlineto{\pgfqpoint{7.262741in}{2.517293in}}%
\pgfpathlineto{\pgfqpoint{7.219047in}{2.546107in}}%
\pgfpathlineto{\pgfqpoint{7.172968in}{2.574304in}}%
\pgfpathlineto{\pgfqpoint{7.112090in}{2.608633in}}%
\pgfpathlineto{\pgfqpoint{7.047667in}{2.641883in}}%
\pgfpathlineto{\pgfqpoint{6.979807in}{2.673992in}}%
\pgfpathlineto{\pgfqpoint{6.908627in}{2.704900in}}%
\pgfpathlineto{\pgfqpoint{6.834250in}{2.734546in}}%
\pgfpathlineto{\pgfqpoint{6.756803in}{2.762875in}}%
\pgfpathlineto{\pgfqpoint{6.676419in}{2.789831in}}%
\pgfpathlineto{\pgfqpoint{6.593239in}{2.815362in}}%
\pgfpathlineto{\pgfqpoint{6.507408in}{2.839417in}}%
\pgfpathlineto{\pgfqpoint{6.419075in}{2.861946in}}%
\pgfpathlineto{\pgfqpoint{6.328397in}{2.882905in}}%
\pgfpathlineto{\pgfqpoint{6.235534in}{2.902249in}}%
\pgfpathlineto{\pgfqpoint{6.140651in}{2.919937in}}%
\pgfpathlineto{\pgfqpoint{6.043917in}{2.935931in}}%
\pgfpathlineto{\pgfqpoint{5.945508in}{2.950195in}}%
\pgfpathlineto{\pgfqpoint{5.845601in}{2.962695in}}%
\pgfpathlineto{\pgfqpoint{5.744379in}{2.973403in}}%
\pgfpathlineto{\pgfqpoint{5.642026in}{2.982291in}}%
\pgfpathlineto{\pgfqpoint{5.538733in}{2.989335in}}%
\pgfpathlineto{\pgfqpoint{5.434691in}{2.994514in}}%
\pgfpathlineto{\pgfqpoint{5.330095in}{2.997812in}}%
\pgfpathlineto{\pgfqpoint{5.225143in}{2.999213in}}%
\pgfpathlineto{\pgfqpoint{5.120033in}{2.998707in}}%
\pgfpathlineto{\pgfqpoint{5.014968in}{2.996286in}}%
\pgfpathlineto{\pgfqpoint{4.910149in}{2.991948in}}%
\pgfpathlineto{\pgfqpoint{4.805782in}{2.985690in}}%
\pgfpathlineto{\pgfqpoint{4.702070in}{2.977517in}}%
\pgfpathlineto{\pgfqpoint{4.599220in}{2.967435in}}%
\pgfpathlineto{\pgfqpoint{4.497438in}{2.955456in}}%
\pgfpathlineto{\pgfqpoint{4.396927in}{2.941592in}}%
\pgfpathlineto{\pgfqpoint{4.297895in}{2.925863in}}%
\pgfpathlineto{\pgfqpoint{4.200544in}{2.908290in}}%
\pgfpathlineto{\pgfqpoint{4.105077in}{2.888898in}}%
\pgfpathlineto{\pgfqpoint{4.011694in}{2.867717in}}%
\pgfpathlineto{\pgfqpoint{3.920596in}{2.844780in}}%
\pgfpathlineto{\pgfqpoint{3.831978in}{2.820125in}}%
\pgfpathlineto{\pgfqpoint{3.746032in}{2.793792in}}%
\pgfpathlineto{\pgfqpoint{3.662950in}{2.765825in}}%
\pgfpathlineto{\pgfqpoint{3.582917in}{2.736274in}}%
\pgfpathlineto{\pgfqpoint{3.506115in}{2.705190in}}%
\pgfpathlineto{\pgfqpoint{3.432720in}{2.672629in}}%
\pgfpathlineto{\pgfqpoint{3.376574in}{2.645558in}}%
\pgfpathlineto{\pgfqpoint{3.322804in}{2.617612in}}%
\pgfpathlineto{\pgfqpoint{3.271494in}{2.588825in}}%
\pgfpathlineto{\pgfqpoint{3.222723in}{2.559233in}}%
\pgfpathlineto{\pgfqpoint{3.176569in}{2.528872in}}%
\pgfpathlineto{\pgfqpoint{3.133106in}{2.497779in}}%
\pgfpathlineto{\pgfqpoint{3.092406in}{2.465996in}}%
\pgfpathlineto{\pgfqpoint{3.054537in}{2.433561in}}%
\pgfpathlineto{\pgfqpoint{3.019563in}{2.400518in}}%
\pgfpathlineto{\pgfqpoint{2.987545in}{2.366909in}}%
\pgfpathlineto{\pgfqpoint{2.958541in}{2.332780in}}%
\pgfpathlineto{\pgfqpoint{2.932603in}{2.298175in}}%
\pgfpathlineto{\pgfqpoint{2.909780in}{2.263142in}}%
\pgfpathlineto{\pgfqpoint{2.890118in}{2.227728in}}%
\pgfpathlineto{\pgfqpoint{2.877469in}{2.200948in}}%
\pgfpathlineto{\pgfqpoint{2.866637in}{2.174001in}}%
\pgfpathlineto{\pgfqpoint{2.857634in}{2.146910in}}%
\pgfpathlineto{\pgfqpoint{2.850473in}{2.119696in}}%
\pgfpathlineto{\pgfqpoint{2.845166in}{2.092380in}}%
\pgfpathlineto{\pgfqpoint{2.841720in}{2.064985in}}%
\pgfpathlineto{\pgfqpoint{2.840144in}{2.037533in}}%
\pgfpathlineto{\pgfqpoint{2.840444in}{2.010045in}}%
\pgfpathlineto{\pgfqpoint{2.842624in}{1.982545in}}%
\pgfpathlineto{\pgfqpoint{2.846687in}{1.955055in}}%
\pgfpathlineto{\pgfqpoint{2.852634in}{1.927597in}}%
\pgfpathlineto{\pgfqpoint{2.860464in}{1.900195in}}%
\pgfpathlineto{\pgfqpoint{2.870174in}{1.872870in}}%
\pgfpathlineto{\pgfqpoint{2.881760in}{1.845647in}}%
\pgfpathlineto{\pgfqpoint{2.895217in}{1.818547in}}%
\pgfpathlineto{\pgfqpoint{2.910536in}{1.791594in}}%
\pgfpathlineto{\pgfqpoint{2.927707in}{1.764810in}}%
\pgfpathlineto{\pgfqpoint{2.953465in}{1.729400in}}%
\pgfpathlineto{\pgfqpoint{2.982465in}{1.694386in}}%
\pgfpathlineto{\pgfqpoint{3.014668in}{1.659820in}}%
\pgfpathlineto{\pgfqpoint{3.050031in}{1.625753in}}%
\pgfpathlineto{\pgfqpoint{3.088506in}{1.592240in}}%
\pgfpathlineto{\pgfqpoint{3.130039in}{1.559329in}}%
\pgfpathlineto{\pgfqpoint{3.174568in}{1.527072in}}%
\pgfpathlineto{\pgfqpoint{3.222030in}{1.495517in}}%
\pgfpathlineto{\pgfqpoint{3.272354in}{1.464712in}}%
\pgfpathlineto{\pgfqpoint{3.325464in}{1.434704in}}%
\pgfpathlineto{\pgfqpoint{3.381280in}{1.405539in}}%
\pgfpathlineto{\pgfqpoint{3.439716in}{1.377259in}}%
\pgfpathlineto{\pgfqpoint{3.500683in}{1.349907in}}%
\pgfpathlineto{\pgfqpoint{3.564087in}{1.323525in}}%
\pgfpathlineto{\pgfqpoint{3.629830in}{1.298149in}}%
\pgfpathlineto{\pgfqpoint{3.715140in}{1.267902in}}%
\pgfpathlineto{\pgfqpoint{3.803736in}{1.239354in}}%
\pgfpathlineto{\pgfqpoint{3.895403in}{1.212569in}}%
\pgfpathlineto{\pgfqpoint{3.989917in}{1.187602in}}%
\pgfpathlineto{\pgfqpoint{4.087049in}{1.164509in}}%
\pgfpathlineto{\pgfqpoint{4.186564in}{1.143334in}}%
\pgfpathlineto{\pgfqpoint{4.288221in}{1.124119in}}%
\pgfpathlineto{\pgfqpoint{4.391779in}{1.106901in}}%
\pgfpathlineto{\pgfqpoint{4.496990in}{1.091710in}}%
\pgfpathlineto{\pgfqpoint{4.603606in}{1.078571in}}%
\pgfpathlineto{\pgfqpoint{4.711378in}{1.067502in}}%
\pgfpathlineto{\pgfqpoint{4.820056in}{1.058517in}}%
\pgfpathlineto{\pgfqpoint{4.929390in}{1.051625in}}%
\pgfpathlineto{\pgfqpoint{5.039134in}{1.046829in}}%
\pgfpathlineto{\pgfqpoint{5.149041in}{1.044125in}}%
\pgfpathlineto{\pgfqpoint{5.258868in}{1.043507in}}%
\pgfpathlineto{\pgfqpoint{5.368377in}{1.044961in}}%
\pgfpathlineto{\pgfqpoint{5.477331in}{1.048472in}}%
\pgfpathlineto{\pgfqpoint{5.585500in}{1.054017in}}%
\pgfpathlineto{\pgfqpoint{5.692659in}{1.061571in}}%
\pgfpathlineto{\pgfqpoint{5.798586in}{1.071102in}}%
\pgfpathlineto{\pgfqpoint{5.903069in}{1.082578in}}%
\pgfpathlineto{\pgfqpoint{6.005899in}{1.095959in}}%
\pgfpathlineto{\pgfqpoint{6.106875in}{1.111206in}}%
\pgfpathlineto{\pgfqpoint{6.205803in}{1.128272in}}%
\pgfpathlineto{\pgfqpoint{6.302496in}{1.147111in}}%
\pgfpathlineto{\pgfqpoint{6.396775in}{1.167671in}}%
\pgfpathlineto{\pgfqpoint{6.488467in}{1.189900in}}%
\pgfpathlineto{\pgfqpoint{6.577408in}{1.213741in}}%
\pgfpathlineto{\pgfqpoint{6.663441in}{1.239137in}}%
\pgfpathlineto{\pgfqpoint{6.746417in}{1.266026in}}%
\pgfpathlineto{\pgfqpoint{6.826194in}{1.294348in}}%
\pgfpathlineto{\pgfqpoint{6.902639in}{1.324038in}}%
\pgfpathlineto{\pgfqpoint{6.975627in}{1.355030in}}%
\pgfpathlineto{\pgfqpoint{7.045039in}{1.387257in}}%
\pgfpathlineto{\pgfqpoint{7.110764in}{1.420652in}}%
\pgfpathlineto{\pgfqpoint{7.172702in}{1.455145in}}%
\pgfpathlineto{\pgfqpoint{7.219460in}{1.483482in}}%
\pgfpathlineto{\pgfqpoint{7.263687in}{1.512440in}}%
\pgfpathlineto{\pgfqpoint{7.305342in}{1.541982in}}%
\pgfpathlineto{\pgfqpoint{7.344384in}{1.572071in}}%
\pgfpathlineto{\pgfqpoint{7.380779in}{1.602669in}}%
\pgfpathlineto{\pgfqpoint{7.414492in}{1.633740in}}%
\pgfpathlineto{\pgfqpoint{7.445495in}{1.665244in}}%
\pgfpathlineto{\pgfqpoint{7.473760in}{1.697143in}}%
\pgfpathlineto{\pgfqpoint{7.499262in}{1.729401in}}%
\pgfpathlineto{\pgfqpoint{7.521980in}{1.761977in}}%
\pgfpathlineto{\pgfqpoint{7.541897in}{1.794835in}}%
\pgfpathlineto{\pgfqpoint{7.558995in}{1.827935in}}%
\pgfpathlineto{\pgfqpoint{7.573264in}{1.861238in}}%
\pgfpathlineto{\pgfqpoint{7.584692in}{1.894707in}}%
\pgfpathlineto{\pgfqpoint{7.593273in}{1.928302in}}%
\pgfpathlineto{\pgfqpoint{7.599001in}{1.961985in}}%
\pgfpathlineto{\pgfqpoint{7.601877in}{1.995718in}}%
\pgfpathlineto{\pgfqpoint{7.601901in}{2.029462in}}%
\pgfpathlineto{\pgfqpoint{7.599076in}{2.063178in}}%
\pgfpathlineto{\pgfqpoint{7.593410in}{2.096829in}}%
\pgfpathlineto{\pgfqpoint{7.584911in}{2.130376in}}%
\pgfpathlineto{\pgfqpoint{7.573593in}{2.163782in}}%
\pgfpathlineto{\pgfqpoint{7.559469in}{2.197008in}}%
\pgfpathlineto{\pgfqpoint{7.542557in}{2.230017in}}%
\pgfpathlineto{\pgfqpoint{7.522878in}{2.262771in}}%
\pgfpathlineto{\pgfqpoint{7.500454in}{2.295234in}}%
\pgfpathlineto{\pgfqpoint{7.475311in}{2.327369in}}%
\pgfpathlineto{\pgfqpoint{7.447477in}{2.359138in}}%
\pgfpathlineto{\pgfqpoint{7.416983in}{2.390506in}}%
\pgfpathlineto{\pgfqpoint{7.383862in}{2.421437in}}%
\pgfpathlineto{\pgfqpoint{7.348151in}{2.451895in}}%
\pgfpathlineto{\pgfqpoint{7.309888in}{2.481845in}}%
\pgfpathlineto{\pgfqpoint{7.269116in}{2.511252in}}%
\pgfpathlineto{\pgfqpoint{7.225877in}{2.540083in}}%
\pgfpathlineto{\pgfqpoint{7.180219in}{2.568302in}}%
\pgfpathlineto{\pgfqpoint{7.119820in}{2.602668in}}%
\pgfpathlineto{\pgfqpoint{7.055820in}{2.635963in}}%
\pgfpathlineto{\pgfqpoint{6.988329in}{2.668125in}}%
\pgfpathlineto{\pgfqpoint{6.917459in}{2.699094in}}%
\pgfpathlineto{\pgfqpoint{6.843333in}{2.728811in}}%
\pgfpathlineto{\pgfqpoint{6.766076in}{2.757217in}}%
\pgfpathlineto{\pgfqpoint{6.685823in}{2.784258in}}%
\pgfpathlineto{\pgfqpoint{6.602711in}{2.809880in}}%
\pgfpathlineto{\pgfqpoint{6.516885in}{2.834031in}}%
\pgfpathlineto{\pgfqpoint{6.428495in}{2.856661in}}%
\pgfpathlineto{\pgfqpoint{6.337698in}{2.877724in}}%
\pgfpathlineto{\pgfqpoint{6.244654in}{2.897175in}}%
\pgfpathlineto{\pgfqpoint{6.149528in}{2.914971in}}%
\pgfpathlineto{\pgfqpoint{6.052494in}{2.931074in}}%
\pgfpathlineto{\pgfqpoint{5.953725in}{2.945446in}}%
\pgfpathlineto{\pgfqpoint{5.853404in}{2.958054in}}%
\pgfpathlineto{\pgfqpoint{5.751714in}{2.968866in}}%
\pgfpathlineto{\pgfqpoint{5.648843in}{2.977854in}}%
\pgfpathlineto{\pgfqpoint{5.544986in}{2.984995in}}%
\pgfpathlineto{\pgfqpoint{5.440337in}{2.990267in}}%
\pgfpathlineto{\pgfqpoint{5.335094in}{2.993651in}}%
\pgfpathlineto{\pgfqpoint{5.229461in}{2.995133in}}%
\pgfpathlineto{\pgfqpoint{5.123640in}{2.994702in}}%
\pgfpathlineto{\pgfqpoint{5.017838in}{2.992351in}}%
\pgfpathlineto{\pgfqpoint{4.912263in}{2.988076in}}%
\pgfpathlineto{\pgfqpoint{4.807123in}{2.981877in}}%
\pgfpathlineto{\pgfqpoint{4.702628in}{2.973758in}}%
\pgfpathlineto{\pgfqpoint{4.598988in}{2.963727in}}%
\pgfpathlineto{\pgfqpoint{4.496413in}{2.951795in}}%
\pgfpathlineto{\pgfqpoint{4.395114in}{2.937978in}}%
\pgfpathlineto{\pgfqpoint{4.295298in}{2.922296in}}%
\pgfpathlineto{\pgfqpoint{4.197173in}{2.904771in}}%
\pgfpathlineto{\pgfqpoint{4.100946in}{2.885431in}}%
\pgfpathlineto{\pgfqpoint{4.006818in}{2.864308in}}%
\pgfpathlineto{\pgfqpoint{3.914991in}{2.841436in}}%
\pgfpathlineto{\pgfqpoint{3.825663in}{2.816855in}}%
\pgfpathlineto{\pgfqpoint{3.739026in}{2.790608in}}%
\pgfpathlineto{\pgfqpoint{3.655271in}{2.762742in}}%
\pgfpathlineto{\pgfqpoint{3.574582in}{2.733307in}}%
\pgfpathlineto{\pgfqpoint{3.497140in}{2.702358in}}%
\pgfpathlineto{\pgfqpoint{3.423119in}{2.669953in}}%
\pgfpathlineto{\pgfqpoint{3.352688in}{2.636153in}}%
\pgfpathlineto{\pgfqpoint{3.299036in}{2.608152in}}%
\pgfpathlineto{\pgfqpoint{3.247866in}{2.579334in}}%
\pgfpathlineto{\pgfqpoint{3.199253in}{2.549737in}}%
\pgfpathlineto{\pgfqpoint{3.153272in}{2.519396in}}%
\pgfpathlineto{\pgfqpoint{3.109993in}{2.488352in}}%
\pgfpathlineto{\pgfqpoint{3.069485in}{2.456643in}}%
\pgfpathlineto{\pgfqpoint{3.031809in}{2.424311in}}%
\pgfpathlineto{\pgfqpoint{2.997027in}{2.391399in}}%
\pgfpathlineto{\pgfqpoint{2.965195in}{2.357949in}}%
\pgfpathlineto{\pgfqpoint{2.936365in}{2.324005in}}%
\pgfpathlineto{\pgfqpoint{2.910587in}{2.289614in}}%
\pgfpathlineto{\pgfqpoint{2.887903in}{2.254820in}}%
\pgfpathlineto{\pgfqpoint{2.868356in}{2.219672in}}%
\pgfpathlineto{\pgfqpoint{2.855775in}{2.193107in}}%
\pgfpathlineto{\pgfqpoint{2.844992in}{2.166389in}}%
\pgfpathlineto{\pgfqpoint{2.836018in}{2.139540in}}%
\pgfpathlineto{\pgfqpoint{2.828865in}{2.112579in}}%
\pgfpathlineto{\pgfqpoint{2.823541in}{2.085529in}}%
\pgfpathlineto{\pgfqpoint{2.820055in}{2.058410in}}%
\pgfpathlineto{\pgfqpoint{2.818413in}{2.031244in}}%
\pgfpathlineto{\pgfqpoint{2.818619in}{2.004051in}}%
\pgfpathlineto{\pgfqpoint{2.820676in}{1.976855in}}%
\pgfpathlineto{\pgfqpoint{2.824586in}{1.949675in}}%
\pgfpathlineto{\pgfqpoint{2.830350in}{1.922534in}}%
\pgfpathlineto{\pgfqpoint{2.837965in}{1.895454in}}%
\pgfpathlineto{\pgfqpoint{2.847429in}{1.868456in}}%
\pgfpathlineto{\pgfqpoint{2.858736in}{1.841562in}}%
\pgfpathlineto{\pgfqpoint{2.871881in}{1.814793in}}%
\pgfpathlineto{\pgfqpoint{2.886856in}{1.788172in}}%
\pgfpathlineto{\pgfqpoint{2.909652in}{1.752945in}}%
\pgfpathlineto{\pgfqpoint{2.935656in}{1.718068in}}%
\pgfpathlineto{\pgfqpoint{2.964837in}{1.683594in}}%
\pgfpathlineto{\pgfqpoint{2.997158in}{1.649573in}}%
\pgfpathlineto{\pgfqpoint{3.032575in}{1.616054in}}%
\pgfpathlineto{\pgfqpoint{3.071042in}{1.583086in}}%
\pgfpathlineto{\pgfqpoint{3.112507in}{1.550717in}}%
\pgfpathlineto{\pgfqpoint{3.156912in}{1.518996in}}%
\pgfpathlineto{\pgfqpoint{3.204196in}{1.487969in}}%
\pgfpathlineto{\pgfqpoint{3.254292in}{1.457681in}}%
\pgfpathlineto{\pgfqpoint{3.307128in}{1.428177in}}%
\pgfpathlineto{\pgfqpoint{3.362628in}{1.399499in}}%
\pgfpathlineto{\pgfqpoint{3.420713in}{1.371691in}}%
\pgfpathlineto{\pgfqpoint{3.481297in}{1.344793in}}%
\pgfpathlineto{\pgfqpoint{3.544292in}{1.318844in}}%
\pgfpathlineto{\pgfqpoint{3.626285in}{1.287798in}}%
\pgfpathlineto{\pgfqpoint{3.711710in}{1.258364in}}%
\pgfpathlineto{\pgfqpoint{3.800370in}{1.230606in}}%
\pgfpathlineto{\pgfqpoint{3.892059in}{1.204586in}}%
\pgfpathlineto{\pgfqpoint{3.986563in}{1.180361in}}%
\pgfpathlineto{\pgfqpoint{4.083663in}{1.157982in}}%
\pgfpathlineto{\pgfqpoint{4.183132in}{1.137497in}}%
\pgfpathlineto{\pgfqpoint{4.284740in}{1.118947in}}%
\pgfpathlineto{\pgfqpoint{4.388251in}{1.102370in}}%
\pgfpathlineto{\pgfqpoint{4.493426in}{1.087797in}}%
\pgfpathlineto{\pgfqpoint{4.600022in}{1.075254in}}%
\pgfpathlineto{\pgfqpoint{4.707796in}{1.064762in}}%
\pgfpathlineto{\pgfqpoint{4.816501in}{1.056337in}}%
\pgfpathlineto{\pgfqpoint{4.925893in}{1.049990in}}%
\pgfpathlineto{\pgfqpoint{5.035725in}{1.045725in}}%
\pgfpathlineto{\pgfqpoint{5.145753in}{1.043543in}}%
\pgfpathlineto{\pgfqpoint{5.255734in}{1.043437in}}%
\pgfpathlineto{\pgfqpoint{5.365427in}{1.045400in}}%
\pgfpathlineto{\pgfqpoint{5.474595in}{1.049414in}}%
\pgfpathlineto{\pgfqpoint{5.583005in}{1.055461in}}%
\pgfpathlineto{\pgfqpoint{5.690427in}{1.063515in}}%
\pgfpathlineto{\pgfqpoint{5.796637in}{1.073549in}}%
\pgfpathlineto{\pgfqpoint{5.901417in}{1.085530in}}%
\pgfpathlineto{\pgfqpoint{6.004553in}{1.099420in}}%
\pgfpathlineto{\pgfqpoint{6.105838in}{1.115178in}}%
\pgfpathlineto{\pgfqpoint{6.205073in}{1.132761in}}%
\pgfpathlineto{\pgfqpoint{6.302065in}{1.152120in}}%
\pgfpathlineto{\pgfqpoint{6.396628in}{1.173204in}}%
\pgfpathlineto{\pgfqpoint{6.488584in}{1.195960in}}%
\pgfpathlineto{\pgfqpoint{6.577763in}{1.220330in}}%
\pgfpathlineto{\pgfqpoint{6.664002in}{1.246255in}}%
\pgfpathlineto{\pgfqpoint{6.747148in}{1.273674in}}%
\pgfpathlineto{\pgfqpoint{6.827054in}{1.302522in}}%
\pgfpathlineto{\pgfqpoint{6.903583in}{1.332734in}}%
\pgfpathlineto{\pgfqpoint{6.976605in}{1.364242in}}%
\pgfpathlineto{\pgfqpoint{7.045998in}{1.396978in}}%
\pgfpathlineto{\pgfqpoint{7.111651in}{1.430870in}}%
\pgfpathlineto{\pgfqpoint{7.161409in}{1.458767in}}%
\pgfpathlineto{\pgfqpoint{7.208655in}{1.487322in}}%
\pgfpathlineto{\pgfqpoint{7.253344in}{1.516496in}}%
\pgfpathlineto{\pgfqpoint{7.295433in}{1.546252in}}%
\pgfpathlineto{\pgfqpoint{7.334880in}{1.576550in}}%
\pgfpathlineto{\pgfqpoint{7.371650in}{1.607353in}}%
\pgfpathlineto{\pgfqpoint{7.405710in}{1.638622in}}%
\pgfpathlineto{\pgfqpoint{7.437028in}{1.670317in}}%
\pgfpathlineto{\pgfqpoint{7.465579in}{1.702400in}}%
\pgfpathlineto{\pgfqpoint{7.491338in}{1.734831in}}%
\pgfpathlineto{\pgfqpoint{7.514283in}{1.767571in}}%
\pgfpathlineto{\pgfqpoint{7.534399in}{1.800581in}}%
\pgfpathlineto{\pgfqpoint{7.551669in}{1.833822in}}%
\pgfpathlineto{\pgfqpoint{7.566082in}{1.867254in}}%
\pgfpathlineto{\pgfqpoint{7.577629in}{1.900838in}}%
\pgfpathlineto{\pgfqpoint{7.586303in}{1.934535in}}%
\pgfpathlineto{\pgfqpoint{7.592103in}{1.968305in}}%
\pgfpathlineto{\pgfqpoint{7.595027in}{2.002110in}}%
\pgfpathlineto{\pgfqpoint{7.595078in}{2.035912in}}%
\pgfpathlineto{\pgfqpoint{7.592262in}{2.069670in}}%
\pgfpathlineto{\pgfqpoint{7.586585in}{2.103347in}}%
\pgfpathlineto{\pgfqpoint{7.578060in}{2.136905in}}%
\pgfpathlineto{\pgfqpoint{7.566699in}{2.170305in}}%
\pgfpathlineto{\pgfqpoint{7.552519in}{2.203510in}}%
\pgfpathlineto{\pgfqpoint{7.535539in}{2.236482in}}%
\pgfpathlineto{\pgfqpoint{7.515779in}{2.269184in}}%
\pgfpathlineto{\pgfqpoint{7.493264in}{2.301579in}}%
\pgfpathlineto{\pgfqpoint{7.468021in}{2.333630in}}%
\pgfpathlineto{\pgfqpoint{7.440078in}{2.365301in}}%
\pgfpathlineto{\pgfqpoint{7.409468in}{2.396556in}}%
\pgfpathlineto{\pgfqpoint{7.376225in}{2.427360in}}%
\pgfpathlineto{\pgfqpoint{7.340385in}{2.457676in}}%
\pgfpathlineto{\pgfqpoint{7.301989in}{2.487472in}}%
\pgfpathlineto{\pgfqpoint{7.261078in}{2.516712in}}%
\pgfpathlineto{\pgfqpoint{7.217696in}{2.545362in}}%
\pgfpathlineto{\pgfqpoint{7.171891in}{2.573389in}}%
\pgfpathlineto{\pgfqpoint{7.111303in}{2.607497in}}%
\pgfpathlineto{\pgfqpoint{7.047107in}{2.640518in}}%
\pgfpathlineto{\pgfqpoint{6.979413in}{2.672390in}}%
\pgfpathlineto{\pgfqpoint{6.908334in}{2.703052in}}%
\pgfpathlineto{\pgfqpoint{6.833990in}{2.732448in}}%
\pgfpathlineto{\pgfqpoint{6.756508in}{2.760519in}}%
\pgfpathlineto{\pgfqpoint{6.676020in}{2.787210in}}%
\pgfpathlineto{\pgfqpoint{6.592663in}{2.812469in}}%
\pgfpathlineto{\pgfqpoint{6.506581in}{2.836245in}}%
\pgfpathlineto{\pgfqpoint{6.417923in}{2.858487in}}%
\pgfpathlineto{\pgfqpoint{6.326843in}{2.879150in}}%
\pgfpathlineto{\pgfqpoint{6.233503in}{2.898188in}}%
\pgfpathlineto{\pgfqpoint{6.138067in}{2.915559in}}%
\pgfpathlineto{\pgfqpoint{6.040706in}{2.931224in}}%
\pgfpathlineto{\pgfqpoint{5.941596in}{2.945146in}}%
\pgfpathlineto{\pgfqpoint{5.840917in}{2.957289in}}%
\pgfpathlineto{\pgfqpoint{5.738855in}{2.967624in}}%
\pgfpathlineto{\pgfqpoint{5.635599in}{2.976121in}}%
\pgfpathlineto{\pgfqpoint{5.531343in}{2.982756in}}%
\pgfpathlineto{\pgfqpoint{5.426283in}{2.987505in}}%
\pgfpathlineto{\pgfqpoint{5.320622in}{2.990352in}}%
\pgfpathlineto{\pgfqpoint{5.214564in}{2.991281in}}%
\pgfpathlineto{\pgfqpoint{5.108315in}{2.990281in}}%
\pgfpathlineto{\pgfqpoint{5.002085in}{2.987343in}}%
\pgfpathlineto{\pgfqpoint{4.896086in}{2.982465in}}%
\pgfpathlineto{\pgfqpoint{4.790532in}{2.975645in}}%
\pgfpathlineto{\pgfqpoint{4.685637in}{2.966889in}}%
\pgfpathlineto{\pgfqpoint{4.581616in}{2.956205in}}%
\pgfpathlineto{\pgfqpoint{4.478686in}{2.943603in}}%
\pgfpathlineto{\pgfqpoint{4.377061in}{2.929102in}}%
\pgfpathlineto{\pgfqpoint{4.276956in}{2.912721in}}%
\pgfpathlineto{\pgfqpoint{4.178584in}{2.894486in}}%
\pgfpathlineto{\pgfqpoint{4.082158in}{2.874425in}}%
\pgfpathlineto{\pgfqpoint{3.987886in}{2.852571in}}%
\pgfpathlineto{\pgfqpoint{3.895975in}{2.828964in}}%
\pgfpathlineto{\pgfqpoint{3.806626in}{2.803643in}}%
\pgfpathlineto{\pgfqpoint{3.720040in}{2.776655in}}%
\pgfpathlineto{\pgfqpoint{3.636410in}{2.748051in}}%
\pgfpathlineto{\pgfqpoint{3.555924in}{2.717883in}}%
\pgfpathlineto{\pgfqpoint{3.478766in}{2.686211in}}%
\pgfpathlineto{\pgfqpoint{3.405113in}{2.653095in}}%
\pgfpathlineto{\pgfqpoint{3.348828in}{2.625607in}}%
\pgfpathlineto{\pgfqpoint{3.294978in}{2.597273in}}%
\pgfpathlineto{\pgfqpoint{3.243645in}{2.568128in}}%
\pgfpathlineto{\pgfqpoint{3.194905in}{2.538212in}}%
\pgfpathlineto{\pgfqpoint{3.148832in}{2.507562in}}%
\pgfpathlineto{\pgfqpoint{3.105496in}{2.476220in}}%
\pgfpathlineto{\pgfqpoint{3.064964in}{2.444228in}}%
\pgfpathlineto{\pgfqpoint{3.027298in}{2.411627in}}%
\pgfpathlineto{\pgfqpoint{2.992556in}{2.378463in}}%
\pgfpathlineto{\pgfqpoint{2.960793in}{2.344779in}}%
\pgfpathlineto{\pgfqpoint{2.932060in}{2.310623in}}%
\pgfpathlineto{\pgfqpoint{2.906401in}{2.276039in}}%
\pgfpathlineto{\pgfqpoint{2.883859in}{2.241077in}}%
\pgfpathlineto{\pgfqpoint{2.864470in}{2.205784in}}%
\pgfpathlineto{\pgfqpoint{2.852018in}{2.179126in}}%
\pgfpathlineto{\pgfqpoint{2.841369in}{2.152330in}}%
\pgfpathlineto{\pgfqpoint{2.832535in}{2.125418in}}%
\pgfpathlineto{\pgfqpoint{2.825523in}{2.098409in}}%
\pgfpathlineto{\pgfqpoint{2.820343in}{2.071326in}}%
\pgfpathlineto{\pgfqpoint{2.816999in}{2.044190in}}%
\pgfpathlineto{\pgfqpoint{2.815496in}{2.017021in}}%
\pgfpathlineto{\pgfqpoint{2.815836in}{1.989842in}}%
\pgfpathlineto{\pgfqpoint{2.818022in}{1.962674in}}%
\pgfpathlineto{\pgfqpoint{2.822053in}{1.935538in}}%
\pgfpathlineto{\pgfqpoint{2.827926in}{1.908457in}}%
\pgfpathlineto{\pgfqpoint{2.835639in}{1.881451in}}%
\pgfpathlineto{\pgfqpoint{2.845187in}{1.854541in}}%
\pgfpathlineto{\pgfqpoint{2.856562in}{1.827750in}}%
\pgfpathlineto{\pgfqpoint{2.869759in}{1.801099in}}%
\pgfpathlineto{\pgfqpoint{2.884765in}{1.774608in}}%
\pgfpathlineto{\pgfqpoint{2.907571in}{1.739574in}}%
\pgfpathlineto{\pgfqpoint{2.933546in}{1.704913in}}%
\pgfpathlineto{\pgfqpoint{2.962654in}{1.670674in}}%
\pgfpathlineto{\pgfqpoint{2.994856in}{1.636906in}}%
\pgfpathlineto{\pgfqpoint{3.030106in}{1.603656in}}%
\pgfpathlineto{\pgfqpoint{3.068356in}{1.570972in}}%
\pgfpathlineto{\pgfqpoint{3.109551in}{1.538900in}}%
\pgfpathlineto{\pgfqpoint{3.153634in}{1.507485in}}%
\pgfpathlineto{\pgfqpoint{3.200543in}{1.476772in}}%
\pgfpathlineto{\pgfqpoint{3.250209in}{1.446804in}}%
\pgfpathlineto{\pgfqpoint{3.302563in}{1.417624in}}%
\pgfpathlineto{\pgfqpoint{3.357529in}{1.389271in}}%
\pgfpathlineto{\pgfqpoint{3.415028in}{1.361787in}}%
\pgfpathlineto{\pgfqpoint{3.474978in}{1.335208in}}%
\pgfpathlineto{\pgfqpoint{3.537292in}{1.309574in}}%
\pgfpathlineto{\pgfqpoint{3.618373in}{1.278911in}}%
\pgfpathlineto{\pgfqpoint{3.702825in}{1.249844in}}%
\pgfpathlineto{\pgfqpoint{3.790458in}{1.222434in}}%
\pgfpathlineto{\pgfqpoint{3.881074in}{1.196741in}}%
\pgfpathlineto{\pgfqpoint{3.974469in}{1.172817in}}%
\pgfpathlineto{\pgfqpoint{4.070433in}{1.150713in}}%
\pgfpathlineto{\pgfqpoint{4.168749in}{1.130473in}}%
\pgfpathlineto{\pgfqpoint{4.269199in}{1.112138in}}%
\pgfpathlineto{\pgfqpoint{4.371555in}{1.095745in}}%
\pgfpathlineto{\pgfqpoint{4.475591in}{1.081323in}}%
\pgfpathlineto{\pgfqpoint{4.581074in}{1.068901in}}%
\pgfpathlineto{\pgfqpoint{4.687771in}{1.058498in}}%
\pgfpathlineto{\pgfqpoint{4.795446in}{1.050133in}}%
\pgfpathlineto{\pgfqpoint{4.903863in}{1.043816in}}%
\pgfpathlineto{\pgfqpoint{5.012784in}{1.039555in}}%
\pgfpathlineto{\pgfqpoint{5.121972in}{1.037352in}}%
\pgfpathlineto{\pgfqpoint{5.231192in}{1.037204in}}%
\pgfpathlineto{\pgfqpoint{5.340210in}{1.039103in}}%
\pgfpathlineto{\pgfqpoint{5.448792in}{1.043038in}}%
\pgfpathlineto{\pgfqpoint{5.556710in}{1.048991in}}%
\pgfpathlineto{\pgfqpoint{5.663738in}{1.056942in}}%
\pgfpathlineto{\pgfqpoint{5.769653in}{1.066865in}}%
\pgfpathlineto{\pgfqpoint{5.874237in}{1.078730in}}%
\pgfpathlineto{\pgfqpoint{5.977278in}{1.092503in}}%
\pgfpathlineto{\pgfqpoint{6.078567in}{1.108147in}}%
\pgfpathlineto{\pgfqpoint{6.177903in}{1.125619in}}%
\pgfpathlineto{\pgfqpoint{6.275090in}{1.144874in}}%
\pgfpathlineto{\pgfqpoint{6.369939in}{1.165865in}}%
\pgfpathlineto{\pgfqpoint{6.462268in}{1.188538in}}%
\pgfpathlineto{\pgfqpoint{6.551903in}{1.212840in}}%
\pgfpathlineto{\pgfqpoint{6.638675in}{1.238712in}}%
\pgfpathlineto{\pgfqpoint{6.722425in}{1.266094in}}%
\pgfpathlineto{\pgfqpoint{6.803001in}{1.294923in}}%
\pgfpathlineto{\pgfqpoint{6.880260in}{1.325135in}}%
\pgfpathlineto{\pgfqpoint{6.954066in}{1.356661in}}%
\pgfpathlineto{\pgfqpoint{7.024291in}{1.389433in}}%
\pgfpathlineto{\pgfqpoint{7.090818in}{1.423380in}}%
\pgfpathlineto{\pgfqpoint{7.153534in}{1.458430in}}%
\pgfpathlineto{\pgfqpoint{7.200896in}{1.487215in}}%
\pgfpathlineto{\pgfqpoint{7.245706in}{1.516621in}}%
\pgfpathlineto{\pgfqpoint{7.287919in}{1.546608in}}%
\pgfpathlineto{\pgfqpoint{7.327494in}{1.577139in}}%
\pgfpathlineto{\pgfqpoint{7.364392in}{1.608174in}}%
\pgfpathlineto{\pgfqpoint{7.398580in}{1.639673in}}%
\pgfpathlineto{\pgfqpoint{7.430025in}{1.671597in}}%
\pgfpathlineto{\pgfqpoint{7.458700in}{1.703905in}}%
\pgfpathlineto{\pgfqpoint{7.484580in}{1.736559in}}%
\pgfpathlineto{\pgfqpoint{7.507643in}{1.769516in}}%
\pgfpathlineto{\pgfqpoint{7.527872in}{1.802739in}}%
\pgfpathlineto{\pgfqpoint{7.545251in}{1.836186in}}%
\pgfpathlineto{\pgfqpoint{7.559768in}{1.869817in}}%
\pgfpathlineto{\pgfqpoint{7.571415in}{1.903592in}}%
\pgfpathlineto{\pgfqpoint{7.580186in}{1.937472in}}%
\pgfpathlineto{\pgfqpoint{7.586078in}{1.971416in}}%
\pgfpathlineto{\pgfqpoint{7.589091in}{2.005386in}}%
\pgfpathlineto{\pgfqpoint{7.589229in}{2.039340in}}%
\pgfpathlineto{\pgfqpoint{7.586497in}{2.073241in}}%
\pgfpathlineto{\pgfqpoint{7.580905in}{2.107050in}}%
\pgfpathlineto{\pgfqpoint{7.572464in}{2.140727in}}%
\pgfpathlineto{\pgfqpoint{7.561189in}{2.174233in}}%
\pgfpathlineto{\pgfqpoint{7.547096in}{2.207532in}}%
\pgfpathlineto{\pgfqpoint{7.530207in}{2.240586in}}%
\pgfpathlineto{\pgfqpoint{7.510542in}{2.273356in}}%
\pgfpathlineto{\pgfqpoint{7.488128in}{2.305805in}}%
\pgfpathlineto{\pgfqpoint{7.462992in}{2.337898in}}%
\pgfpathlineto{\pgfqpoint{7.435164in}{2.369598in}}%
\pgfpathlineto{\pgfqpoint{7.404678in}{2.400868in}}%
\pgfpathlineto{\pgfqpoint{7.371568in}{2.431674in}}%
\pgfpathlineto{\pgfqpoint{7.335872in}{2.461981in}}%
\pgfpathlineto{\pgfqpoint{7.297631in}{2.491754in}}%
\pgfpathlineto{\pgfqpoint{7.256887in}{2.520959in}}%
\pgfpathlineto{\pgfqpoint{7.213686in}{2.549562in}}%
\pgfpathlineto{\pgfqpoint{7.168074in}{2.577532in}}%
\pgfpathlineto{\pgfqpoint{7.107746in}{2.611553in}}%
\pgfpathlineto{\pgfqpoint{7.043832in}{2.644470in}}%
\pgfpathlineto{\pgfqpoint{6.976441in}{2.676223in}}%
\pgfpathlineto{\pgfqpoint{6.905686in}{2.706754in}}%
\pgfpathlineto{\pgfqpoint{6.831688in}{2.736005in}}%
\pgfpathlineto{\pgfqpoint{6.754570in}{2.763919in}}%
\pgfpathlineto{\pgfqpoint{6.674463in}{2.790444in}}%
\pgfpathlineto{\pgfqpoint{6.591504in}{2.815527in}}%
\pgfpathlineto{\pgfqpoint{6.505834in}{2.839118in}}%
\pgfpathlineto{\pgfqpoint{6.417599in}{2.861168in}}%
\pgfpathlineto{\pgfqpoint{6.326951in}{2.881632in}}%
\pgfpathlineto{\pgfqpoint{6.234047in}{2.900465in}}%
\pgfpathlineto{\pgfqpoint{6.139051in}{2.917626in}}%
\pgfpathlineto{\pgfqpoint{6.042128in}{2.933075in}}%
\pgfpathlineto{\pgfqpoint{5.943452in}{2.946775in}}%
\pgfpathlineto{\pgfqpoint{5.843199in}{2.958693in}}%
\pgfpathlineto{\pgfqpoint{5.741550in}{2.968797in}}%
\pgfpathlineto{\pgfqpoint{5.638693in}{2.977058in}}%
\pgfpathlineto{\pgfqpoint{5.534818in}{2.983451in}}%
\pgfpathlineto{\pgfqpoint{5.430118in}{2.987953in}}%
\pgfpathlineto{\pgfqpoint{5.324793in}{2.990544in}}%
\pgfpathlineto{\pgfqpoint{5.219044in}{2.991209in}}%
\pgfpathlineto{\pgfqpoint{5.113077in}{2.989935in}}%
\pgfpathlineto{\pgfqpoint{5.007100in}{2.986714in}}%
\pgfpathlineto{\pgfqpoint{4.901323in}{2.981539in}}%
\pgfpathlineto{\pgfqpoint{4.795960in}{2.974411in}}%
\pgfpathlineto{\pgfqpoint{4.691226in}{2.965330in}}%
\pgfpathlineto{\pgfqpoint{4.587337in}{2.954305in}}%
\pgfpathlineto{\pgfqpoint{4.484511in}{2.941345in}}%
\pgfpathlineto{\pgfqpoint{4.382966in}{2.926467in}}%
\pgfpathlineto{\pgfqpoint{4.282919in}{2.909688in}}%
\pgfpathlineto{\pgfqpoint{4.184588in}{2.891034in}}%
\pgfpathlineto{\pgfqpoint{4.088188in}{2.870532in}}%
\pgfpathlineto{\pgfqpoint{3.993935in}{2.848214in}}%
\pgfpathlineto{\pgfqpoint{3.902039in}{2.824119in}}%
\pgfpathlineto{\pgfqpoint{3.812710in}{2.798287in}}%
\pgfpathlineto{\pgfqpoint{3.726152in}{2.770765in}}%
\pgfpathlineto{\pgfqpoint{3.642567in}{2.741603in}}%
\pgfpathlineto{\pgfqpoint{3.562150in}{2.710856in}}%
\pgfpathlineto{\pgfqpoint{3.485092in}{2.678584in}}%
\pgfpathlineto{\pgfqpoint{3.411576in}{2.644850in}}%
\pgfpathlineto{\pgfqpoint{3.355433in}{2.616855in}}%
\pgfpathlineto{\pgfqpoint{3.301757in}{2.588004in}}%
\pgfpathlineto{\pgfqpoint{3.250632in}{2.558335in}}%
\pgfpathlineto{\pgfqpoint{3.202139in}{2.527888in}}%
\pgfpathlineto{\pgfqpoint{3.156353in}{2.496702in}}%
\pgfpathlineto{\pgfqpoint{3.113348in}{2.464822in}}%
\pgfpathlineto{\pgfqpoint{3.073192in}{2.432289in}}%
\pgfpathlineto{\pgfqpoint{3.035950in}{2.399150in}}%
\pgfpathlineto{\pgfqpoint{3.001680in}{2.365450in}}%
\pgfpathlineto{\pgfqpoint{2.970440in}{2.331237in}}%
\pgfpathlineto{\pgfqpoint{2.942279in}{2.296558in}}%
\pgfpathlineto{\pgfqpoint{2.917244in}{2.261463in}}%
\pgfpathlineto{\pgfqpoint{2.895375in}{2.226001in}}%
\pgfpathlineto{\pgfqpoint{2.876709in}{2.190223in}}%
\pgfpathlineto{\pgfqpoint{2.864830in}{2.163213in}}%
\pgfpathlineto{\pgfqpoint{2.854781in}{2.136076in}}%
\pgfpathlineto{\pgfqpoint{2.846572in}{2.108834in}}%
\pgfpathlineto{\pgfqpoint{2.840211in}{2.081508in}}%
\pgfpathlineto{\pgfqpoint{2.835704in}{2.054121in}}%
\pgfpathlineto{\pgfqpoint{2.833055in}{2.026695in}}%
\pgfpathlineto{\pgfqpoint{2.832268in}{1.999253in}}%
\pgfpathlineto{\pgfqpoint{2.833344in}{1.971816in}}%
\pgfpathlineto{\pgfqpoint{2.836282in}{1.944407in}}%
\pgfpathlineto{\pgfqpoint{2.841080in}{1.917049in}}%
\pgfpathlineto{\pgfqpoint{2.847735in}{1.889762in}}%
\pgfpathlineto{\pgfqpoint{2.856240in}{1.862570in}}%
\pgfpathlineto{\pgfqpoint{2.866590in}{1.835495in}}%
\pgfpathlineto{\pgfqpoint{2.878776in}{1.808558in}}%
\pgfpathlineto{\pgfqpoint{2.892786in}{1.781781in}}%
\pgfpathlineto{\pgfqpoint{2.914285in}{1.746365in}}%
\pgfpathlineto{\pgfqpoint{2.938974in}{1.711322in}}%
\pgfpathlineto{\pgfqpoint{2.966817in}{1.676704in}}%
\pgfpathlineto{\pgfqpoint{2.997770in}{1.642558in}}%
\pgfpathlineto{\pgfqpoint{3.031788in}{1.608934in}}%
\pgfpathlineto{\pgfqpoint{3.068817in}{1.575880in}}%
\pgfpathlineto{\pgfqpoint{3.108803in}{1.543440in}}%
\pgfpathlineto{\pgfqpoint{3.151684in}{1.511662in}}%
\pgfpathlineto{\pgfqpoint{3.197396in}{1.480589in}}%
\pgfpathlineto{\pgfqpoint{3.245870in}{1.450265in}}%
\pgfpathlineto{\pgfqpoint{3.297033in}{1.420730in}}%
\pgfpathlineto{\pgfqpoint{3.350808in}{1.392027in}}%
\pgfpathlineto{\pgfqpoint{3.407115in}{1.364193in}}%
\pgfpathlineto{\pgfqpoint{3.465871in}{1.337268in}}%
\pgfpathlineto{\pgfqpoint{3.526988in}{1.311285in}}%
\pgfpathlineto{\pgfqpoint{3.606567in}{1.280188in}}%
\pgfpathlineto{\pgfqpoint{3.689511in}{1.250683in}}%
\pgfpathlineto{\pgfqpoint{3.775630in}{1.222831in}}%
\pgfpathlineto{\pgfqpoint{3.864729in}{1.196689in}}%
\pgfpathlineto{\pgfqpoint{3.956606in}{1.172307in}}%
\pgfpathlineto{\pgfqpoint{4.051054in}{1.149735in}}%
\pgfpathlineto{\pgfqpoint{4.147861in}{1.129014in}}%
\pgfpathlineto{\pgfqpoint{4.246813in}{1.110185in}}%
\pgfpathlineto{\pgfqpoint{4.347689in}{1.093281in}}%
\pgfpathlineto{\pgfqpoint{4.450268in}{1.078332in}}%
\pgfpathlineto{\pgfqpoint{4.554325in}{1.065364in}}%
\pgfpathlineto{\pgfqpoint{4.659634in}{1.054397in}}%
\pgfpathlineto{\pgfqpoint{4.765967in}{1.045449in}}%
\pgfpathlineto{\pgfqpoint{4.873095in}{1.038530in}}%
\pgfpathlineto{\pgfqpoint{4.980790in}{1.033648in}}%
\pgfpathlineto{\pgfqpoint{5.088823in}{1.030806in}}%
\pgfpathlineto{\pgfqpoint{5.196966in}{1.030001in}}%
\pgfpathlineto{\pgfqpoint{5.304994in}{1.031229in}}%
\pgfpathlineto{\pgfqpoint{5.412680in}{1.034478in}}%
\pgfpathlineto{\pgfqpoint{5.519803in}{1.039734in}}%
\pgfpathlineto{\pgfqpoint{5.626144in}{1.046978in}}%
\pgfpathlineto{\pgfqpoint{5.731486in}{1.056187in}}%
\pgfpathlineto{\pgfqpoint{5.835616in}{1.067333in}}%
\pgfpathlineto{\pgfqpoint{5.938327in}{1.080387in}}%
\pgfpathlineto{\pgfqpoint{6.039413in}{1.095313in}}%
\pgfpathlineto{\pgfqpoint{6.138678in}{1.112074in}}%
\pgfpathlineto{\pgfqpoint{6.235926in}{1.130626in}}%
\pgfpathlineto{\pgfqpoint{6.330970in}{1.150926in}}%
\pgfpathlineto{\pgfqpoint{6.423628in}{1.172924in}}%
\pgfpathlineto{\pgfqpoint{6.513725in}{1.196568in}}%
\pgfpathlineto{\pgfqpoint{6.601092in}{1.221806in}}%
\pgfpathlineto{\pgfqpoint{6.685567in}{1.248578in}}%
\pgfpathlineto{\pgfqpoint{6.766995in}{1.276826in}}%
\pgfpathlineto{\pgfqpoint{6.845230in}{1.306487in}}%
\pgfpathlineto{\pgfqpoint{6.920131in}{1.337497in}}%
\pgfpathlineto{\pgfqpoint{6.991567in}{1.369787in}}%
\pgfpathlineto{\pgfqpoint{7.059412in}{1.403291in}}%
\pgfpathlineto{\pgfqpoint{7.123552in}{1.437938in}}%
\pgfpathlineto{\pgfqpoint{7.172123in}{1.466430in}}%
\pgfpathlineto{\pgfqpoint{7.218202in}{1.495569in}}%
\pgfpathlineto{\pgfqpoint{7.261740in}{1.525318in}}%
\pgfpathlineto{\pgfqpoint{7.302692in}{1.555639in}}%
\pgfpathlineto{\pgfqpoint{7.341018in}{1.586491in}}%
\pgfpathlineto{\pgfqpoint{7.376679in}{1.617837in}}%
\pgfpathlineto{\pgfqpoint{7.409640in}{1.649636in}}%
\pgfpathlineto{\pgfqpoint{7.439871in}{1.681848in}}%
\pgfpathlineto{\pgfqpoint{7.467343in}{1.714434in}}%
\pgfpathlineto{\pgfqpoint{7.492032in}{1.747352in}}%
\pgfpathlineto{\pgfqpoint{7.513917in}{1.780564in}}%
\pgfpathlineto{\pgfqpoint{7.532979in}{1.814027in}}%
\pgfpathlineto{\pgfqpoint{7.549205in}{1.847702in}}%
\pgfpathlineto{\pgfqpoint{7.562582in}{1.881549in}}%
\pgfpathlineto{\pgfqpoint{7.573102in}{1.915526in}}%
\pgfpathlineto{\pgfqpoint{7.580760in}{1.949594in}}%
\pgfpathlineto{\pgfqpoint{7.585554in}{1.983711in}}%
\pgfpathlineto{\pgfqpoint{7.587486in}{2.017839in}}%
\pgfpathlineto{\pgfqpoint{7.586559in}{2.051937in}}%
\pgfpathlineto{\pgfqpoint{7.582780in}{2.085966in}}%
\pgfpathlineto{\pgfqpoint{7.576160in}{2.119886in}}%
\pgfpathlineto{\pgfqpoint{7.566711in}{2.153658in}}%
\pgfpathlineto{\pgfqpoint{7.554450in}{2.187243in}}%
\pgfpathlineto{\pgfqpoint{7.539395in}{2.220603in}}%
\pgfpathlineto{\pgfqpoint{7.521567in}{2.253700in}}%
\pgfpathlineto{\pgfqpoint{7.500991in}{2.286496in}}%
\pgfpathlineto{\pgfqpoint{7.477692in}{2.318955in}}%
\pgfpathlineto{\pgfqpoint{7.451702in}{2.351038in}}%
\pgfpathlineto{\pgfqpoint{7.423052in}{2.382710in}}%
\pgfpathlineto{\pgfqpoint{7.391775in}{2.413935in}}%
\pgfpathlineto{\pgfqpoint{7.357911in}{2.444678in}}%
\pgfpathlineto{\pgfqpoint{7.321497in}{2.474903in}}%
\pgfpathlineto{\pgfqpoint{7.282576in}{2.504578in}}%
\pgfpathlineto{\pgfqpoint{7.241192in}{2.533667in}}%
\pgfpathlineto{\pgfqpoint{7.197393in}{2.562138in}}%
\pgfpathlineto{\pgfqpoint{7.151226in}{2.589958in}}%
\pgfpathlineto{\pgfqpoint{7.090268in}{2.623769in}}%
\pgfpathlineto{\pgfqpoint{7.025797in}{2.656454in}}%
\pgfpathlineto{\pgfqpoint{6.957924in}{2.687951in}}%
\pgfpathlineto{\pgfqpoint{6.886764in}{2.718204in}}%
\pgfpathlineto{\pgfqpoint{6.812440in}{2.747157in}}%
\pgfpathlineto{\pgfqpoint{6.735076in}{2.774756in}}%
\pgfpathlineto{\pgfqpoint{6.654805in}{2.800948in}}%
\pgfpathlineto{\pgfqpoint{6.571763in}{2.825683in}}%
\pgfpathlineto{\pgfqpoint{6.486091in}{2.848912in}}%
\pgfpathlineto{\pgfqpoint{6.397935in}{2.870589in}}%
\pgfpathlineto{\pgfqpoint{6.307447in}{2.890670in}}%
\pgfpathlineto{\pgfqpoint{6.214781in}{2.909113in}}%
\pgfpathlineto{\pgfqpoint{6.120099in}{2.925877in}}%
\pgfpathlineto{\pgfqpoint{6.023564in}{2.940926in}}%
\pgfpathlineto{\pgfqpoint{5.925346in}{2.954223in}}%
\pgfpathlineto{\pgfqpoint{5.825619in}{2.965736in}}%
\pgfpathlineto{\pgfqpoint{5.724560in}{2.975435in}}%
\pgfpathlineto{\pgfqpoint{5.622350in}{2.983294in}}%
\pgfpathlineto{\pgfqpoint{5.519177in}{2.989286in}}%
\pgfpathlineto{\pgfqpoint{5.415229in}{2.993391in}}%
\pgfpathlineto{\pgfqpoint{5.310699in}{2.995589in}}%
\pgfpathlineto{\pgfqpoint{5.205784in}{2.995866in}}%
\pgfpathlineto{\pgfqpoint{5.100685in}{2.994208in}}%
\pgfpathlineto{\pgfqpoint{4.995603in}{2.990608in}}%
\pgfpathlineto{\pgfqpoint{4.890744in}{2.985059in}}%
\pgfpathlineto{\pgfqpoint{4.786317in}{2.977560in}}%
\pgfpathlineto{\pgfqpoint{4.682532in}{2.968112in}}%
\pgfpathlineto{\pgfqpoint{4.579599in}{2.956721in}}%
\pgfpathlineto{\pgfqpoint{4.477734in}{2.943398in}}%
\pgfpathlineto{\pgfqpoint{4.377149in}{2.928155in}}%
\pgfpathlineto{\pgfqpoint{4.278059in}{2.911010in}}%
\pgfpathlineto{\pgfqpoint{4.180680in}{2.891986in}}%
\pgfpathlineto{\pgfqpoint{4.085225in}{2.871109in}}%
\pgfpathlineto{\pgfqpoint{3.991908in}{2.848409in}}%
\pgfpathlineto{\pgfqpoint{3.900941in}{2.823923in}}%
\pgfpathlineto{\pgfqpoint{3.812533in}{2.797690in}}%
\pgfpathlineto{\pgfqpoint{3.726890in}{2.769753in}}%
\pgfpathlineto{\pgfqpoint{3.644216in}{2.740163in}}%
\pgfpathlineto{\pgfqpoint{3.564710in}{2.708973in}}%
\pgfpathlineto{\pgfqpoint{3.488565in}{2.676241in}}%
\pgfpathlineto{\pgfqpoint{3.430197in}{2.648986in}}%
\pgfpathlineto{\pgfqpoint{3.374194in}{2.620818in}}%
\pgfpathlineto{\pgfqpoint{3.320647in}{2.591774in}}%
\pgfpathlineto{\pgfqpoint{3.269645in}{2.561889in}}%
\pgfpathlineto{\pgfqpoint{3.221271in}{2.531204in}}%
\pgfpathlineto{\pgfqpoint{3.175606in}{2.499759in}}%
\pgfpathlineto{\pgfqpoint{3.132728in}{2.467598in}}%
\pgfpathlineto{\pgfqpoint{3.092709in}{2.434763in}}%
\pgfpathlineto{\pgfqpoint{3.055618in}{2.401300in}}%
\pgfpathlineto{\pgfqpoint{3.021520in}{2.367257in}}%
\pgfpathlineto{\pgfqpoint{2.990475in}{2.332681in}}%
\pgfpathlineto{\pgfqpoint{2.962537in}{2.297622in}}%
\pgfpathlineto{\pgfqpoint{2.937756in}{2.262130in}}%
\pgfpathlineto{\pgfqpoint{2.916178in}{2.226256in}}%
\pgfpathlineto{\pgfqpoint{2.902120in}{2.199132in}}%
\pgfpathlineto{\pgfqpoint{2.889901in}{2.171845in}}%
\pgfpathlineto{\pgfqpoint{2.879533in}{2.144417in}}%
\pgfpathlineto{\pgfqpoint{2.871028in}{2.116873in}}%
\pgfpathlineto{\pgfqpoint{2.864395in}{2.089234in}}%
\pgfpathlineto{\pgfqpoint{2.859642in}{2.061523in}}%
\pgfpathlineto{\pgfqpoint{2.856775in}{2.033764in}}%
\pgfpathlineto{\pgfqpoint{2.855797in}{2.005980in}}%
\pgfpathlineto{\pgfqpoint{2.856712in}{1.978195in}}%
\pgfpathlineto{\pgfqpoint{2.859518in}{1.950432in}}%
\pgfpathlineto{\pgfqpoint{2.864215in}{1.922714in}}%
\pgfpathlineto{\pgfqpoint{2.870799in}{1.895064in}}%
\pgfpathlineto{\pgfqpoint{2.879266in}{1.867506in}}%
\pgfpathlineto{\pgfqpoint{2.889607in}{1.840063in}}%
\pgfpathlineto{\pgfqpoint{2.901815in}{1.812758in}}%
\pgfpathlineto{\pgfqpoint{2.915879in}{1.785615in}}%
\pgfpathlineto{\pgfqpoint{2.931786in}{1.758654in}}%
\pgfpathlineto{\pgfqpoint{2.955837in}{1.723032in}}%
\pgfpathlineto{\pgfqpoint{2.983102in}{1.687829in}}%
\pgfpathlineto{\pgfqpoint{3.013537in}{1.653097in}}%
\pgfpathlineto{\pgfqpoint{3.047092in}{1.618888in}}%
\pgfpathlineto{\pgfqpoint{3.083714in}{1.585250in}}%
\pgfpathlineto{\pgfqpoint{3.123344in}{1.552234in}}%
\pgfpathlineto{\pgfqpoint{3.165917in}{1.519886in}}%
\pgfpathlineto{\pgfqpoint{3.211366in}{1.488253in}}%
\pgfpathlineto{\pgfqpoint{3.259619in}{1.457380in}}%
\pgfpathlineto{\pgfqpoint{3.310597in}{1.427310in}}%
\pgfpathlineto{\pgfqpoint{3.364220in}{1.398085in}}%
\pgfpathlineto{\pgfqpoint{3.420404in}{1.369746in}}%
\pgfpathlineto{\pgfqpoint{3.479060in}{1.342331in}}%
\pgfpathlineto{\pgfqpoint{3.540097in}{1.315876in}}%
\pgfpathlineto{\pgfqpoint{3.603419in}{1.290418in}}%
\pgfpathlineto{\pgfqpoint{3.685636in}{1.260048in}}%
\pgfpathlineto{\pgfqpoint{3.771075in}{1.231348in}}%
\pgfpathlineto{\pgfqpoint{3.859532in}{1.204374in}}%
\pgfpathlineto{\pgfqpoint{3.950799in}{1.179180in}}%
\pgfpathlineto{\pgfqpoint{4.044663in}{1.155812in}}%
\pgfpathlineto{\pgfqpoint{4.140907in}{1.134314in}}%
\pgfpathlineto{\pgfqpoint{4.239310in}{1.114724in}}%
\pgfpathlineto{\pgfqpoint{4.339650in}{1.097075in}}%
\pgfpathlineto{\pgfqpoint{4.441701in}{1.081396in}}%
\pgfpathlineto{\pgfqpoint{4.545236in}{1.067712in}}%
\pgfpathlineto{\pgfqpoint{4.650027in}{1.056040in}}%
\pgfpathlineto{\pgfqpoint{4.755845in}{1.046398in}}%
\pgfpathlineto{\pgfqpoint{4.862462in}{1.038793in}}%
\pgfpathlineto{\pgfqpoint{4.969650in}{1.033234in}}%
\pgfpathlineto{\pgfqpoint{5.077182in}{1.029720in}}%
\pgfpathlineto{\pgfqpoint{5.184833in}{1.028250in}}%
\pgfpathlineto{\pgfqpoint{5.292378in}{1.028816in}}%
\pgfpathlineto{\pgfqpoint{5.399597in}{1.031406in}}%
\pgfpathlineto{\pgfqpoint{5.506272in}{1.036005in}}%
\pgfpathlineto{\pgfqpoint{5.612186in}{1.042595in}}%
\pgfpathlineto{\pgfqpoint{5.717130in}{1.051151in}}%
\pgfpathlineto{\pgfqpoint{5.820893in}{1.061647in}}%
\pgfpathlineto{\pgfqpoint{5.923275in}{1.074052in}}%
\pgfpathlineto{\pgfqpoint{6.024074in}{1.088332in}}%
\pgfpathlineto{\pgfqpoint{6.123099in}{1.104450in}}%
\pgfpathlineto{\pgfqpoint{6.220158in}{1.122364in}}%
\pgfpathlineto{\pgfqpoint{6.315071in}{1.142031in}}%
\pgfpathlineto{\pgfqpoint{6.407658in}{1.163404in}}%
\pgfpathlineto{\pgfqpoint{6.497748in}{1.186433in}}%
\pgfpathlineto{\pgfqpoint{6.585176in}{1.211065in}}%
\pgfpathlineto{\pgfqpoint{6.669783in}{1.237245in}}%
\pgfpathlineto{\pgfqpoint{6.751416in}{1.264915in}}%
\pgfpathlineto{\pgfqpoint{6.829931in}{1.294015in}}%
\pgfpathlineto{\pgfqpoint{6.905189in}{1.324482in}}%
\pgfpathlineto{\pgfqpoint{6.977058in}{1.356252in}}%
\pgfpathlineto{\pgfqpoint{7.045415in}{1.389259in}}%
\pgfpathlineto{\pgfqpoint{7.110144in}{1.423434in}}%
\pgfpathlineto{\pgfqpoint{7.171135in}{1.458708in}}%
\pgfpathlineto{\pgfqpoint{7.217168in}{1.487670in}}%
\pgfpathlineto{\pgfqpoint{7.260696in}{1.517252in}}%
\pgfpathlineto{\pgfqpoint{7.301674in}{1.547416in}}%
\pgfpathlineto{\pgfqpoint{7.340059in}{1.578125in}}%
\pgfpathlineto{\pgfqpoint{7.375814in}{1.609340in}}%
\pgfpathlineto{\pgfqpoint{7.408902in}{1.641021in}}%
\pgfpathlineto{\pgfqpoint{7.439293in}{1.673130in}}%
\pgfpathlineto{\pgfqpoint{7.466955in}{1.705626in}}%
\pgfpathlineto{\pgfqpoint{7.491865in}{1.738471in}}%
\pgfpathlineto{\pgfqpoint{7.514000in}{1.771624in}}%
\pgfpathlineto{\pgfqpoint{7.533340in}{1.805044in}}%
\pgfpathlineto{\pgfqpoint{7.549870in}{1.838693in}}%
\pgfpathlineto{\pgfqpoint{7.563576in}{1.872528in}}%
\pgfpathlineto{\pgfqpoint{7.574450in}{1.906511in}}%
\pgfpathlineto{\pgfqpoint{7.582485in}{1.940601in}}%
\pgfpathlineto{\pgfqpoint{7.587677in}{1.974757in}}%
\pgfpathlineto{\pgfqpoint{7.590027in}{2.008939in}}%
\pgfpathlineto{\pgfqpoint{7.589537in}{2.043108in}}%
\pgfpathlineto{\pgfqpoint{7.586213in}{2.077223in}}%
\pgfpathlineto{\pgfqpoint{7.580065in}{2.111245in}}%
\pgfpathlineto{\pgfqpoint{7.571105in}{2.145135in}}%
\pgfpathlineto{\pgfqpoint{7.559347in}{2.178853in}}%
\pgfpathlineto{\pgfqpoint{7.544809in}{2.212361in}}%
\pgfpathlineto{\pgfqpoint{7.527512in}{2.245621in}}%
\pgfpathlineto{\pgfqpoint{7.507480in}{2.278593in}}%
\pgfpathlineto{\pgfqpoint{7.484739in}{2.311242in}}%
\pgfpathlineto{\pgfqpoint{7.459318in}{2.343529in}}%
\pgfpathlineto{\pgfqpoint{7.431249in}{2.375417in}}%
\pgfpathlineto{\pgfqpoint{7.400566in}{2.406872in}}%
\pgfpathlineto{\pgfqpoint{7.367307in}{2.437856in}}%
\pgfpathlineto{\pgfqpoint{7.331510in}{2.468335in}}%
\pgfpathlineto{\pgfqpoint{7.293218in}{2.498274in}}%
\pgfpathlineto{\pgfqpoint{7.252475in}{2.527640in}}%
\pgfpathlineto{\pgfqpoint{7.209328in}{2.556398in}}%
\pgfpathlineto{\pgfqpoint{7.163827in}{2.584517in}}%
\pgfpathlineto{\pgfqpoint{7.103718in}{2.618716in}}%
\pgfpathlineto{\pgfqpoint{7.040117in}{2.651805in}}%
\pgfpathlineto{\pgfqpoint{6.973134in}{2.683723in}}%
\pgfpathlineto{\pgfqpoint{6.902887in}{2.714413in}}%
\pgfpathlineto{\pgfqpoint{6.829496in}{2.743819in}}%
\pgfpathlineto{\pgfqpoint{6.753088in}{2.771888in}}%
\pgfpathlineto{\pgfqpoint{6.673794in}{2.798566in}}%
\pgfpathlineto{\pgfqpoint{6.591750in}{2.823805in}}%
\pgfpathlineto{\pgfqpoint{6.507098in}{2.847557in}}%
\pgfpathlineto{\pgfqpoint{6.419983in}{2.869775in}}%
\pgfpathlineto{\pgfqpoint{6.330553in}{2.890416in}}%
\pgfpathlineto{\pgfqpoint{6.238963in}{2.909438in}}%
\pgfpathlineto{\pgfqpoint{6.145372in}{2.926803in}}%
\pgfpathlineto{\pgfqpoint{6.049940in}{2.942473in}}%
\pgfpathlineto{\pgfqpoint{5.952835in}{2.956415in}}%
\pgfpathlineto{\pgfqpoint{5.854225in}{2.968596in}}%
\pgfpathlineto{\pgfqpoint{5.754284in}{2.978987in}}%
\pgfpathlineto{\pgfqpoint{5.653190in}{2.987562in}}%
\pgfpathlineto{\pgfqpoint{5.551122in}{2.994295in}}%
\pgfpathlineto{\pgfqpoint{5.448264in}{2.999166in}}%
\pgfpathlineto{\pgfqpoint{5.344804in}{3.002157in}}%
\pgfpathlineto{\pgfqpoint{5.240929in}{3.003252in}}%
\pgfpathlineto{\pgfqpoint{5.136835in}{3.002437in}}%
\pgfpathlineto{\pgfqpoint{5.032714in}{2.999705in}}%
\pgfpathlineto{\pgfqpoint{4.928765in}{2.995048in}}%
\pgfpathlineto{\pgfqpoint{4.825187in}{2.988464in}}%
\pgfpathlineto{\pgfqpoint{4.722182in}{2.979953in}}%
\pgfpathlineto{\pgfqpoint{4.619954in}{2.969518in}}%
\pgfpathlineto{\pgfqpoint{4.518706in}{2.957168in}}%
\pgfpathlineto{\pgfqpoint{4.418645in}{2.942914in}}%
\pgfpathlineto{\pgfqpoint{4.319977in}{2.926770in}}%
\pgfpathlineto{\pgfqpoint{4.280947in}{2.919787in}}%
\pgfpathlineto{\pgfqpoint{4.280947in}{2.919787in}}%
\pgfusepath{stroke}%
\end{pgfscope}%
\begin{pgfscope}%
\pgfpathrectangle{\pgfqpoint{1.250000in}{0.400000in}}{\pgfqpoint{7.750000in}{3.200000in}} %
\pgfusepath{clip}%
\pgfsetrectcap%
\pgfsetroundjoin%
\pgfsetlinewidth{1.003750pt}%
\definecolor{currentstroke}{rgb}{0.000000,0.500000,0.000000}%
\pgfsetstrokecolor{currentstroke}%
\pgfsetdash{}{0pt}%
\pgfpathmoveto{\pgfqpoint{7.607583in}{2.000000in}}%
\pgfpathlineto{\pgfqpoint{7.590095in}{2.084868in}}%
\pgfpathlineto{\pgfqpoint{7.555052in}{2.169123in}}%
\pgfpathlineto{\pgfqpoint{7.502504in}{2.252141in}}%
\pgfpathlineto{\pgfqpoint{7.432625in}{2.333293in}}%
\pgfpathlineto{\pgfqpoint{7.345718in}{2.411949in}}%
\pgfpathlineto{\pgfqpoint{7.242215in}{2.487481in}}%
\pgfpathlineto{\pgfqpoint{7.122687in}{2.559268in}}%
\pgfpathlineto{\pgfqpoint{6.987838in}{2.626703in}}%
\pgfpathlineto{\pgfqpoint{6.838510in}{2.689195in}}%
\pgfpathlineto{\pgfqpoint{6.675685in}{2.746178in}}%
\pgfpathlineto{\pgfqpoint{6.500479in}{2.797114in}}%
\pgfpathlineto{\pgfqpoint{6.314141in}{2.841501in}}%
\pgfpathlineto{\pgfqpoint{6.118049in}{2.878879in}}%
\pgfpathlineto{\pgfqpoint{5.913703in}{2.908835in}}%
\pgfpathlineto{\pgfqpoint{5.702717in}{2.931010in}}%
\pgfpathlineto{\pgfqpoint{5.486808in}{2.945107in}}%
\pgfpathlineto{\pgfqpoint{5.267787in}{2.950892in}}%
\pgfpathlineto{\pgfqpoint{5.047538in}{2.948206in}}%
\pgfpathlineto{\pgfqpoint{4.828011in}{2.936964in}}%
\pgfpathlineto{\pgfqpoint{4.611192in}{2.917167in}}%
\pgfpathlineto{\pgfqpoint{4.399093in}{2.888898in}}%
\pgfpathlineto{\pgfqpoint{4.193717in}{2.852329in}}%
\pgfpathlineto{\pgfqpoint{3.997043in}{2.807724in}}%
\pgfpathlineto{\pgfqpoint{3.810993in}{2.755435in}}%
\pgfpathlineto{\pgfqpoint{3.637404in}{2.695906in}}%
\pgfpathlineto{\pgfqpoint{3.478004in}{2.629664in}}%
\pgfpathlineto{\pgfqpoint{3.334381in}{2.557318in}}%
\pgfpathlineto{\pgfqpoint{3.207958in}{2.479550in}}%
\pgfpathlineto{\pgfqpoint{3.099967in}{2.397107in}}%
\pgfpathlineto{\pgfqpoint{3.011433in}{2.310791in}}%
\pgfpathlineto{\pgfqpoint{2.943152in}{2.221446in}}%
\pgfpathlineto{\pgfqpoint{2.895686in}{2.129944in}}%
\pgfpathlineto{\pgfqpoint{2.869351in}{2.037176in}}%
\pgfpathlineto{\pgfqpoint{2.864219in}{1.944032in}}%
\pgfpathlineto{\pgfqpoint{2.880125in}{1.851391in}}%
\pgfpathlineto{\pgfqpoint{2.916675in}{1.760111in}}%
\pgfpathlineto{\pgfqpoint{2.973262in}{1.671011in}}%
\pgfpathlineto{\pgfqpoint{3.049083in}{1.584866in}}%
\pgfpathlineto{\pgfqpoint{3.143161in}{1.502398in}}%
\pgfpathlineto{\pgfqpoint{3.254371in}{1.424269in}}%
\pgfpathlineto{\pgfqpoint{3.381459in}{1.351077in}}%
\pgfpathlineto{\pgfqpoint{3.523068in}{1.283353in}}%
\pgfpathlineto{\pgfqpoint{3.677761in}{1.221559in}}%
\pgfpathlineto{\pgfqpoint{3.844042in}{1.166091in}}%
\pgfpathlineto{\pgfqpoint{4.020375in}{1.117275in}}%
\pgfpathlineto{\pgfqpoint{4.205203in}{1.075376in}}%
\pgfpathlineto{\pgfqpoint{4.396960in}{1.040595in}}%
\pgfpathlineto{\pgfqpoint{4.594085in}{1.013074in}}%
\pgfpathlineto{\pgfqpoint{4.795036in}{0.992903in}}%
\pgfpathlineto{\pgfqpoint{4.998293in}{0.980118in}}%
\pgfpathlineto{\pgfqpoint{5.202367in}{0.974706in}}%
\pgfpathlineto{\pgfqpoint{5.405809in}{0.976611in}}%
\pgfpathlineto{\pgfqpoint{5.607209in}{0.985735in}}%
\pgfpathlineto{\pgfqpoint{5.805201in}{1.001940in}}%
\pgfpathlineto{\pgfqpoint{5.998466in}{1.025052in}}%
\pgfpathlineto{\pgfqpoint{6.185733in}{1.054862in}}%
\pgfpathlineto{\pgfqpoint{6.365782in}{1.091127in}}%
\pgfpathlineto{\pgfqpoint{6.537444in}{1.133575in}}%
\pgfpathlineto{\pgfqpoint{6.699603in}{1.181901in}}%
\pgfpathlineto{\pgfqpoint{6.851195in}{1.235773in}}%
\pgfpathlineto{\pgfqpoint{6.991217in}{1.294830in}}%
\pgfpathlineto{\pgfqpoint{7.118722in}{1.358682in}}%
\pgfpathlineto{\pgfqpoint{7.232826in}{1.426912in}}%
\pgfpathlineto{\pgfqpoint{7.332709in}{1.499076in}}%
\pgfpathlineto{\pgfqpoint{7.417624in}{1.574706in}}%
\pgfpathlineto{\pgfqpoint{7.486896in}{1.653304in}}%
\pgfpathlineto{\pgfqpoint{7.539930in}{1.734351in}}%
\pgfpathlineto{\pgfqpoint{7.576217in}{1.817301in}}%
\pgfpathlineto{\pgfqpoint{7.595341in}{1.901589in}}%
\pgfpathlineto{\pgfqpoint{7.596986in}{1.986626in}}%
\pgfpathlineto{\pgfqpoint{7.580941in}{2.071806in}}%
\pgfpathlineto{\pgfqpoint{7.547109in}{2.156508in}}%
\pgfpathlineto{\pgfqpoint{7.495517in}{2.240095in}}%
\pgfpathlineto{\pgfqpoint{7.426317in}{2.321926in}}%
\pgfpathlineto{\pgfqpoint{7.339797in}{2.401350in}}%
\pgfpathlineto{\pgfqpoint{7.236385in}{2.477721in}}%
\pgfpathlineto{\pgfqpoint{7.116654in}{2.550397in}}%
\pgfpathlineto{\pgfqpoint{6.981326in}{2.618748in}}%
\pgfpathlineto{\pgfqpoint{6.831269in}{2.682162in}}%
\pgfpathlineto{\pgfqpoint{6.667502in}{2.740053in}}%
\pgfpathlineto{\pgfqpoint{6.491190in}{2.791867in}}%
\pgfpathlineto{\pgfqpoint{6.303640in}{2.837090in}}%
\pgfpathlineto{\pgfqpoint{6.106291in}{2.875256in}}%
\pgfpathlineto{\pgfqpoint{5.900708in}{2.905949in}}%
\pgfpathlineto{\pgfqpoint{5.688568in}{2.928816in}}%
\pgfpathlineto{\pgfqpoint{5.471648in}{2.943571in}}%
\pgfpathlineto{\pgfqpoint{5.251804in}{2.949997in}}%
\pgfpathlineto{\pgfqpoint{5.030957in}{2.947958in}}%
\pgfpathlineto{\pgfqpoint{4.811073in}{2.937393in}}%
\pgfpathlineto{\pgfqpoint{4.594138in}{2.918330in}}%
\pgfpathlineto{\pgfqpoint{4.382139in}{2.890879in}}%
\pgfpathlineto{\pgfqpoint{4.177036in}{2.855236in}}%
\pgfpathlineto{\pgfqpoint{3.980747in}{2.811682in}}%
\pgfpathlineto{\pgfqpoint{3.795113in}{2.760583in}}%
\pgfpathlineto{\pgfqpoint{3.621883in}{2.702383in}}%
\pgfpathlineto{\pgfqpoint{3.462689in}{2.637604in}}%
\pgfpathlineto{\pgfqpoint{3.319025in}{2.566836in}}%
\pgfpathlineto{\pgfqpoint{3.192225in}{2.490735in}}%
\pgfpathlineto{\pgfqpoint{3.083448in}{2.410010in}}%
\pgfpathlineto{\pgfqpoint{2.993664in}{2.325420in}}%
\pgfpathlineto{\pgfqpoint{2.923640in}{2.237757in}}%
\pgfpathlineto{\pgfqpoint{2.873930in}{2.147844in}}%
\pgfpathlineto{\pgfqpoint{2.844874in}{2.056518in}}%
\pgfpathlineto{\pgfqpoint{2.836594in}{1.964622in}}%
\pgfpathlineto{\pgfqpoint{2.848997in}{1.872991in}}%
\pgfpathlineto{\pgfqpoint{2.881778in}{1.782449in}}%
\pgfpathlineto{\pgfqpoint{2.934435in}{1.693789in}}%
\pgfpathlineto{\pgfqpoint{3.006276in}{1.607773in}}%
\pgfpathlineto{\pgfqpoint{3.096440in}{1.525117in}}%
\pgfpathlineto{\pgfqpoint{3.203907in}{1.446490in}}%
\pgfpathlineto{\pgfqpoint{3.327522in}{1.372504in}}%
\pgfpathlineto{\pgfqpoint{3.466016in}{1.303713in}}%
\pgfpathlineto{\pgfqpoint{3.618020in}{1.240607in}}%
\pgfpathlineto{\pgfqpoint{3.782090in}{1.183612in}}%
\pgfpathlineto{\pgfqpoint{3.956725in}{1.133091in}}%
\pgfpathlineto{\pgfqpoint{4.140383in}{1.089342in}}%
\pgfpathlineto{\pgfqpoint{4.331501in}{1.052601in}}%
\pgfpathlineto{\pgfqpoint{4.528505in}{1.023042in}}%
\pgfpathlineto{\pgfqpoint{4.729828in}{1.000783in}}%
\pgfpathlineto{\pgfqpoint{4.933919in}{0.985885in}}%
\pgfpathlineto{\pgfqpoint{5.139252in}{0.978358in}}%
\pgfpathlineto{\pgfqpoint{5.344336in}{0.978165in}}%
\pgfpathlineto{\pgfqpoint{5.547719in}{0.985221in}}%
\pgfpathlineto{\pgfqpoint{5.747994in}{0.999400in}}%
\pgfpathlineto{\pgfqpoint{5.943803in}{1.020536in}}%
\pgfpathlineto{\pgfqpoint{6.133841in}{1.048428in}}%
\pgfpathlineto{\pgfqpoint{6.316855in}{1.082838in}}%
\pgfpathlineto{\pgfqpoint{6.491650in}{1.123498in}}%
\pgfpathlineto{\pgfqpoint{6.657087in}{1.170108in}}%
\pgfpathlineto{\pgfqpoint{6.812087in}{1.222340in}}%
\pgfpathlineto{\pgfqpoint{6.955628in}{1.279838in}}%
\pgfpathlineto{\pgfqpoint{7.086751in}{1.342218in}}%
\pgfpathlineto{\pgfqpoint{7.204562in}{1.409071in}}%
\pgfpathlineto{\pgfqpoint{7.308229in}{1.479963in}}%
\pgfpathlineto{\pgfqpoint{7.396990in}{1.554434in}}%
\pgfpathlineto{\pgfqpoint{7.470154in}{1.631999in}}%
\pgfpathlineto{\pgfqpoint{7.527107in}{1.712151in}}%
\pgfpathlineto{\pgfqpoint{7.567315in}{1.794358in}}%
\pgfpathlineto{\pgfqpoint{7.590331in}{1.878065in}}%
\pgfpathlineto{\pgfqpoint{7.595801in}{1.962696in}}%
\pgfpathlineto{\pgfqpoint{7.583469in}{2.047654in}}%
\pgfpathlineto{\pgfqpoint{7.553192in}{2.132323in}}%
\pgfpathlineto{\pgfqpoint{7.504939in}{2.216074in}}%
\pgfpathlineto{\pgfqpoint{7.438805in}{2.298261in}}%
\pgfpathlineto{\pgfqpoint{7.355020in}{2.378231in}}%
\pgfpathlineto{\pgfqpoint{7.253953in}{2.455325in}}%
\pgfpathlineto{\pgfqpoint{7.136124in}{2.528885in}}%
\pgfpathlineto{\pgfqpoint{7.002205in}{2.598259in}}%
\pgfpathlineto{\pgfqpoint{6.853032in}{2.662809in}}%
\pgfpathlineto{\pgfqpoint{6.689598in}{2.721919in}}%
\pgfpathlineto{\pgfqpoint{6.513063in}{2.775002in}}%
\pgfpathlineto{\pgfqpoint{6.324742in}{2.821507in}}%
\pgfpathlineto{\pgfqpoint{6.126106in}{2.860934in}}%
\pgfpathlineto{\pgfqpoint{5.918767in}{2.892836in}}%
\pgfpathlineto{\pgfqpoint{5.704468in}{2.916830in}}%
\pgfpathlineto{\pgfqpoint{5.485064in}{2.932608in}}%
\pgfpathlineto{\pgfqpoint{5.262502in}{2.939939in}}%
\pgfpathlineto{\pgfqpoint{5.038800in}{2.938678in}}%
\pgfpathlineto{\pgfqpoint{4.816019in}{2.928767in}}%
\pgfpathlineto{\pgfqpoint{4.596238in}{2.910243in}}%
\pgfpathlineto{\pgfqpoint{4.381524in}{2.883234in}}%
\pgfpathlineto{\pgfqpoint{4.173905in}{2.847961in}}%
\pgfpathlineto{\pgfqpoint{3.975343in}{2.804735in}}%
\pgfpathlineto{\pgfqpoint{3.787706in}{2.753953in}}%
\pgfpathlineto{\pgfqpoint{3.612747in}{2.696090in}}%
\pgfpathlineto{\pgfqpoint{3.452078in}{2.631699in}}%
\pgfpathlineto{\pgfqpoint{3.307153in}{2.561394in}}%
\pgfpathlineto{\pgfqpoint{3.179252in}{2.485848in}}%
\pgfpathlineto{\pgfqpoint{3.069469in}{2.405783in}}%
\pgfpathlineto{\pgfqpoint{2.978701in}{2.321957in}}%
\pgfpathlineto{\pgfqpoint{2.907643in}{2.235156in}}%
\pgfpathlineto{\pgfqpoint{2.856782in}{2.146187in}}%
\pgfpathlineto{\pgfqpoint{2.826401in}{2.055863in}}%
\pgfpathlineto{\pgfqpoint{2.816580in}{1.964996in}}%
\pgfpathlineto{\pgfqpoint{2.827201in}{1.874391in}}%
\pgfpathlineto{\pgfqpoint{2.857952in}{1.784834in}}%
\pgfpathlineto{\pgfqpoint{2.908344in}{1.697082in}}%
\pgfpathlineto{\pgfqpoint{2.977717in}{1.611864in}}%
\pgfpathlineto{\pgfqpoint{3.065254in}{1.529868in}}%
\pgfpathlineto{\pgfqpoint{3.169995in}{1.451735in}}%
\pgfpathlineto{\pgfqpoint{3.290853in}{1.378060in}}%
\pgfpathlineto{\pgfqpoint{3.426631in}{1.309385in}}%
\pgfpathlineto{\pgfqpoint{3.576032in}{1.246195in}}%
\pgfpathlineto{\pgfqpoint{3.737685in}{1.188917in}}%
\pgfpathlineto{\pgfqpoint{3.910152in}{1.137921in}}%
\pgfpathlineto{\pgfqpoint{4.091948in}{1.093516in}}%
\pgfpathlineto{\pgfqpoint{4.281554in}{1.055954in}}%
\pgfpathlineto{\pgfqpoint{4.477435in}{1.025428in}}%
\pgfpathlineto{\pgfqpoint{4.678046in}{1.002074in}}%
\pgfpathlineto{\pgfqpoint{4.881848in}{0.985974in}}%
\pgfpathlineto{\pgfqpoint{5.087320in}{0.977159in}}%
\pgfpathlineto{\pgfqpoint{5.292964in}{0.975608in}}%
\pgfpathlineto{\pgfqpoint{5.497315in}{0.981255in}}%
\pgfpathlineto{\pgfqpoint{5.698949in}{0.993989in}}%
\pgfpathlineto{\pgfqpoint{5.896487in}{1.013656in}}%
\pgfpathlineto{\pgfqpoint{6.088599in}{1.040065in}}%
\pgfpathlineto{\pgfqpoint{6.274010in}{1.072990in}}%
\pgfpathlineto{\pgfqpoint{6.451502in}{1.112167in}}%
\pgfpathlineto{\pgfqpoint{6.619915in}{1.157305in}}%
\pgfpathlineto{\pgfqpoint{6.778153in}{1.208079in}}%
\pgfpathlineto{\pgfqpoint{6.925178in}{1.264140in}}%
\pgfpathlineto{\pgfqpoint{7.060020in}{1.325110in}}%
\pgfpathlineto{\pgfqpoint{7.181775in}{1.390585in}}%
\pgfpathlineto{\pgfqpoint{7.289603in}{1.460140in}}%
\pgfpathlineto{\pgfqpoint{7.382737in}{1.533325in}}%
\pgfpathlineto{\pgfqpoint{7.460478in}{1.609666in}}%
\pgfpathlineto{\pgfqpoint{7.522205in}{1.688669in}}%
\pgfpathlineto{\pgfqpoint{7.567371in}{1.769818in}}%
\pgfpathlineto{\pgfqpoint{7.595514in}{1.852577in}}%
\pgfpathlineto{\pgfqpoint{7.606256in}{1.936387in}}%
\pgfpathlineto{\pgfqpoint{7.599314in}{2.020673in}}%
\pgfpathlineto{\pgfqpoint{7.574502in}{2.104840in}}%
\pgfpathlineto{\pgfqpoint{7.531740in}{2.188277in}}%
\pgfpathlineto{\pgfqpoint{7.471065in}{2.270357in}}%
\pgfpathlineto{\pgfqpoint{7.392635in}{2.350444in}}%
\pgfpathlineto{\pgfqpoint{7.296737in}{2.427888in}}%
\pgfpathlineto{\pgfqpoint{7.183804in}{2.502040in}}%
\pgfpathlineto{\pgfqpoint{7.054414in}{2.572245in}}%
\pgfpathlineto{\pgfqpoint{6.909303in}{2.637860in}}%
\pgfpathlineto{\pgfqpoint{6.749372in}{2.698250in}}%
\pgfpathlineto{\pgfqpoint{6.575689in}{2.752805in}}%
\pgfpathlineto{\pgfqpoint{6.389492in}{2.800942in}}%
\pgfpathlineto{\pgfqpoint{6.192188in}{2.842120in}}%
\pgfpathlineto{\pgfqpoint{5.985348in}{2.875844in}}%
\pgfpathlineto{\pgfqpoint{5.770699in}{2.901682in}}%
\pgfpathlineto{\pgfqpoint{5.550106in}{2.919273in}}%
\pgfpathlineto{\pgfqpoint{5.325557in}{2.928333in}}%
\pgfpathlineto{\pgfqpoint{5.099137in}{2.928668in}}%
\pgfpathlineto{\pgfqpoint{4.873000in}{2.920180in}}%
\pgfpathlineto{\pgfqpoint{4.649341in}{2.902872in}}%
\pgfpathlineto{\pgfqpoint{4.430357in}{2.876850in}}%
\pgfpathlineto{\pgfqpoint{4.218215in}{2.842325in}}%
\pgfpathlineto{\pgfqpoint{4.015015in}{2.799609in}}%
\pgfpathlineto{\pgfqpoint{3.822759in}{2.749114in}}%
\pgfpathlineto{\pgfqpoint{3.643315in}{2.691342in}}%
\pgfpathlineto{\pgfqpoint{3.478394in}{2.626875in}}%
\pgfpathlineto{\pgfqpoint{3.329524in}{2.556369in}}%
\pgfpathlineto{\pgfqpoint{3.198034in}{2.480541in}}%
\pgfpathlineto{\pgfqpoint{3.085037in}{2.400153in}}%
\pgfpathlineto{\pgfqpoint{2.991426in}{2.316004in}}%
\pgfpathlineto{\pgfqpoint{2.917870in}{2.228917in}}%
\pgfpathlineto{\pgfqpoint{2.864814in}{2.139725in}}%
\pgfpathlineto{\pgfqpoint{2.832485in}{2.049264in}}%
\pgfpathlineto{\pgfqpoint{2.820899in}{1.958358in}}%
\pgfpathlineto{\pgfqpoint{2.829874in}{1.867815in}}%
\pgfpathlineto{\pgfqpoint{2.859038in}{1.778417in}}%
\pgfpathlineto{\pgfqpoint{2.907847in}{1.690912in}}%
\pgfpathlineto{\pgfqpoint{2.975599in}{1.606009in}}%
\pgfpathlineto{\pgfqpoint{3.061447in}{1.524375in}}%
\pgfpathlineto{\pgfqpoint{3.164414in}{1.446631in}}%
\pgfpathlineto{\pgfqpoint{3.283412in}{1.373347in}}%
\pgfpathlineto{\pgfqpoint{3.417253in}{1.305039in}}%
\pgfpathlineto{\pgfqpoint{3.564667in}{1.242172in}}%
\pgfpathlineto{\pgfqpoint{3.724312in}{1.185154in}}%
\pgfpathlineto{\pgfqpoint{3.894790in}{1.134340in}}%
\pgfpathlineto{\pgfqpoint{4.074660in}{1.090028in}}%
\pgfpathlineto{\pgfqpoint{4.262450in}{1.052462in}}%
\pgfpathlineto{\pgfqpoint{4.456667in}{1.021832in}}%
\pgfpathlineto{\pgfqpoint{4.655809in}{0.998275in}}%
\pgfpathlineto{\pgfqpoint{4.858377in}{0.981875in}}%
\pgfpathlineto{\pgfqpoint{5.062879in}{0.972670in}}%
\pgfpathlineto{\pgfqpoint{5.267845in}{0.970647in}}%
\pgfpathlineto{\pgfqpoint{5.471829in}{0.975747in}}%
\pgfpathlineto{\pgfqpoint{5.673419in}{0.987870in}}%
\pgfpathlineto{\pgfqpoint{5.871244in}{1.006870in}}%
\pgfpathlineto{\pgfqpoint{6.063976in}{1.032567in}}%
\pgfpathlineto{\pgfqpoint{6.250337in}{1.064739in}}%
\pgfpathlineto{\pgfqpoint{6.429105in}{1.103132in}}%
\pgfpathlineto{\pgfqpoint{6.599113in}{1.147459in}}%
\pgfpathlineto{\pgfqpoint{6.759257in}{1.197401in}}%
\pgfpathlineto{\pgfqpoint{6.908494in}{1.252611in}}%
\pgfpathlineto{\pgfqpoint{7.045847in}{1.312717in}}%
\pgfpathlineto{\pgfqpoint{7.170407in}{1.377318in}}%
\pgfpathlineto{\pgfqpoint{7.281334in}{1.445992in}}%
\pgfpathlineto{\pgfqpoint{7.377861in}{1.518294in}}%
\pgfpathlineto{\pgfqpoint{7.459291in}{1.593756in}}%
\pgfpathlineto{\pgfqpoint{7.525004in}{1.671893in}}%
\pgfpathlineto{\pgfqpoint{7.574460in}{1.752197in}}%
\pgfpathlineto{\pgfqpoint{7.607198in}{1.834146in}}%
\pgfpathlineto{\pgfqpoint{7.622841in}{1.917198in}}%
\pgfpathlineto{\pgfqpoint{7.621099in}{2.000794in}}%
\pgfpathlineto{\pgfqpoint{7.601779in}{2.084360in}}%
\pgfpathlineto{\pgfqpoint{7.564783in}{2.167310in}}%
\pgfpathlineto{\pgfqpoint{7.510116in}{2.249043in}}%
\pgfpathlineto{\pgfqpoint{7.437897in}{2.328946in}}%
\pgfpathlineto{\pgfqpoint{7.348360in}{2.406401in}}%
\pgfpathlineto{\pgfqpoint{7.241867in}{2.480779in}}%
\pgfpathlineto{\pgfqpoint{7.118914in}{2.551454in}}%
\pgfpathlineto{\pgfqpoint{6.980137in}{2.617797in}}%
\pgfpathlineto{\pgfqpoint{6.826323in}{2.679190in}}%
\pgfpathlineto{\pgfqpoint{6.658417in}{2.735025in}}%
\pgfpathlineto{\pgfqpoint{6.477526in}{2.784718in}}%
\pgfpathlineto{\pgfqpoint{6.284922in}{2.827710in}}%
\pgfpathlineto{\pgfqpoint{6.082044in}{2.863485in}}%
\pgfpathlineto{\pgfqpoint{5.870498in}{2.891572in}}%
\pgfpathlineto{\pgfqpoint{5.652045in}{2.911558in}}%
\pgfpathlineto{\pgfqpoint{5.428591in}{2.923104in}}%
\pgfpathlineto{\pgfqpoint{5.202174in}{2.925949in}}%
\pgfpathlineto{\pgfqpoint{4.974934in}{2.919924in}}%
\pgfpathlineto{\pgfqpoint{4.749090in}{2.904958in}}%
\pgfpathlineto{\pgfqpoint{4.526906in}{2.881090in}}%
\pgfpathlineto{\pgfqpoint{4.310651in}{2.848465in}}%
\pgfpathlineto{\pgfqpoint{4.102561in}{2.807347in}}%
\pgfpathlineto{\pgfqpoint{3.904800in}{2.758106in}}%
\pgfpathlineto{\pgfqpoint{3.719413in}{2.701222in}}%
\pgfpathlineto{\pgfqpoint{3.548296in}{2.637272in}}%
\pgfpathlineto{\pgfqpoint{3.393157in}{2.566921in}}%
\pgfpathlineto{\pgfqpoint{3.255491in}{2.490912in}}%
\pgfpathlineto{\pgfqpoint{3.136558in}{2.410044in}}%
\pgfpathlineto{\pgfqpoint{3.037367in}{2.325165in}}%
\pgfpathlineto{\pgfqpoint{2.958675in}{2.237151in}}%
\pgfpathlineto{\pgfqpoint{2.900980in}{2.146893in}}%
\pgfpathlineto{\pgfqpoint{2.864533in}{2.055282in}}%
\pgfpathlineto{\pgfqpoint{2.849346in}{1.963194in}}%
\pgfpathlineto{\pgfqpoint{2.855206in}{1.871484in}}%
\pgfpathlineto{\pgfqpoint{2.881697in}{1.780972in}}%
\pgfpathlineto{\pgfqpoint{2.928214in}{1.692436in}}%
\pgfpathlineto{\pgfqpoint{2.993988in}{1.606608in}}%
\pgfpathlineto{\pgfqpoint{3.078104in}{1.524168in}}%
\pgfpathlineto{\pgfqpoint{3.179520in}{1.445740in}}%
\pgfpathlineto{\pgfqpoint{3.297088in}{1.371892in}}%
\pgfpathlineto{\pgfqpoint{3.429571in}{1.303133in}}%
\pgfpathlineto{\pgfqpoint{3.575656in}{1.239916in}}%
\pgfpathlineto{\pgfqpoint{3.733975in}{1.182634in}}%
\pgfpathlineto{\pgfqpoint{3.903114in}{1.131625in}}%
\pgfpathlineto{\pgfqpoint{4.081627in}{1.087171in}}%
\pgfpathlineto{\pgfqpoint{4.268045in}{1.049498in}}%
\pgfpathlineto{\pgfqpoint{4.460890in}{1.018782in}}%
\pgfpathlineto{\pgfqpoint{4.658678in}{0.995148in}}%
\pgfpathlineto{\pgfqpoint{4.859934in}{0.978669in}}%
\pgfpathlineto{\pgfqpoint{5.063194in}{0.969373in}}%
\pgfpathlineto{\pgfqpoint{5.267014in}{0.967243in}}%
\pgfpathlineto{\pgfqpoint{5.469976in}{0.972217in}}%
\pgfpathlineto{\pgfqpoint{5.670693in}{0.984192in}}%
\pgfpathlineto{\pgfqpoint{5.867816in}{1.003024in}}%
\pgfpathlineto{\pgfqpoint{6.060038in}{1.028533in}}%
\pgfpathlineto{\pgfqpoint{6.246097in}{1.060499in}}%
\pgfpathlineto{\pgfqpoint{6.424781in}{1.098671in}}%
\pgfpathlineto{\pgfqpoint{6.594935in}{1.142763in}}%
\pgfpathlineto{\pgfqpoint{6.755460in}{1.192460in}}%
\pgfpathlineto{\pgfqpoint{6.905317in}{1.247416in}}%
\pgfpathlineto{\pgfqpoint{7.043535in}{1.307258in}}%
\pgfpathlineto{\pgfqpoint{7.169207in}{1.371588in}}%
\pgfpathlineto{\pgfqpoint{7.281496in}{1.439982in}}%
\pgfpathlineto{\pgfqpoint{7.379641in}{1.511995in}}%
\pgfpathlineto{\pgfqpoint{7.462951in}{1.587161in}}%
\pgfpathlineto{\pgfqpoint{7.530817in}{1.664994in}}%
\pgfpathlineto{\pgfqpoint{7.582707in}{1.744989in}}%
\pgfpathlineto{\pgfqpoint{7.618174in}{1.826625in}}%
\pgfpathlineto{\pgfqpoint{7.636854in}{1.909365in}}%
\pgfpathlineto{\pgfqpoint{7.638474in}{1.992660in}}%
\pgfpathlineto{\pgfqpoint{7.622851in}{2.075947in}}%
\pgfpathlineto{\pgfqpoint{7.589900in}{2.158651in}}%
\pgfpathlineto{\pgfqpoint{7.539634in}{2.240189in}}%
\pgfpathlineto{\pgfqpoint{7.472170in}{2.319970in}}%
\pgfpathlineto{\pgfqpoint{7.387736in}{2.397398in}}%
\pgfpathlineto{\pgfqpoint{7.286674in}{2.471871in}}%
\pgfpathlineto{\pgfqpoint{7.169446in}{2.542792in}}%
\pgfpathlineto{\pgfqpoint{7.036640in}{2.609561in}}%
\pgfpathlineto{\pgfqpoint{6.888980in}{2.671589in}}%
\pgfpathlineto{\pgfqpoint{6.727324in}{2.728296in}}%
\pgfpathlineto{\pgfqpoint{6.552677in}{2.779121in}}%
\pgfpathlineto{\pgfqpoint{6.366189in}{2.823525in}}%
\pgfpathlineto{\pgfqpoint{6.169166in}{2.860999in}}%
\pgfpathlineto{\pgfqpoint{5.963060in}{2.891073in}}%
\pgfpathlineto{\pgfqpoint{5.749478in}{2.913323in}}%
\pgfpathlineto{\pgfqpoint{5.530170in}{2.927384in}}%
\pgfpathlineto{\pgfqpoint{5.307021in}{2.932957in}}%
\pgfpathlineto{\pgfqpoint{5.082041in}{2.929819in}}%
\pgfpathlineto{\pgfqpoint{4.857337in}{2.917835in}}%
\pgfpathlineto{\pgfqpoint{4.635098in}{2.896966in}}%
\pgfpathlineto{\pgfqpoint{4.417558in}{2.867276in}}%
\pgfpathlineto{\pgfqpoint{4.206964in}{2.828941in}}%
\pgfpathlineto{\pgfqpoint{4.005535in}{2.782247in}}%
\pgfpathlineto{\pgfqpoint{3.815423in}{2.727595in}}%
\pgfpathlineto{\pgfqpoint{3.638669in}{2.665496in}}%
\pgfpathlineto{\pgfqpoint{3.477159in}{2.596563in}}%
\pgfpathlineto{\pgfqpoint{3.332592in}{2.521504in}}%
\pgfpathlineto{\pgfqpoint{3.206439in}{2.441105in}}%
\pgfpathlineto{\pgfqpoint{3.099927in}{2.356220in}}%
\pgfpathlineto{\pgfqpoint{3.014009in}{2.267750in}}%
\pgfpathlineto{\pgfqpoint{2.949364in}{2.176624in}}%
\pgfpathlineto{\pgfqpoint{2.906391in}{2.083788in}}%
\pgfpathlineto{\pgfqpoint{2.885214in}{1.990180in}}%
\pgfpathlineto{\pgfqpoint{2.885699in}{1.896722in}}%
\pgfpathlineto{\pgfqpoint{2.907472in}{1.804299in}}%
\pgfpathlineto{\pgfqpoint{2.949935in}{1.713757in}}%
\pgfpathlineto{\pgfqpoint{3.012300in}{1.625886in}}%
\pgfpathlineto{\pgfqpoint{3.093610in}{1.541419in}}%
\pgfpathlineto{\pgfqpoint{3.192762in}{1.461023in}}%
\pgfpathlineto{\pgfqpoint{3.308537in}{1.385301in}}%
\pgfpathlineto{\pgfqpoint{3.439622in}{1.314790in}}%
\pgfpathlineto{\pgfqpoint{3.584628in}{1.249957in}}%
\pgfpathlineto{\pgfqpoint{3.742112in}{1.191207in}}%
\pgfpathlineto{\pgfqpoint{3.910594in}{1.138879in}}%
\pgfpathlineto{\pgfqpoint{4.088568in}{1.093252in}}%
\pgfpathlineto{\pgfqpoint{4.274518in}{1.054548in}}%
\pgfpathlineto{\pgfqpoint{4.466926in}{1.022931in}}%
\pgfpathlineto{\pgfqpoint{4.664284in}{0.998514in}}%
\pgfpathlineto{\pgfqpoint{4.865096in}{0.981359in}}%
\pgfpathlineto{\pgfqpoint{5.067892in}{0.971481in}}%
\pgfpathlineto{\pgfqpoint{5.271226in}{0.968851in}}%
\pgfpathlineto{\pgfqpoint{5.473687in}{0.973397in}}%
\pgfpathlineto{\pgfqpoint{5.673897in}{0.985006in}}%
\pgfpathlineto{\pgfqpoint{5.870520in}{1.003527in}}%
\pgfpathlineto{\pgfqpoint{6.062265in}{1.028773in}}%
\pgfpathlineto{\pgfqpoint{6.247884in}{1.060521in}}%
\pgfpathlineto{\pgfqpoint{6.426183in}{1.098516in}}%
\pgfpathlineto{\pgfqpoint{6.596019in}{1.142470in}}%
\pgfpathlineto{\pgfqpoint{6.756306in}{1.192065in}}%
\pgfpathlineto{\pgfqpoint{6.906017in}{1.246955in}}%
\pgfpathlineto{\pgfqpoint{7.044188in}{1.306766in}}%
\pgfpathlineto{\pgfqpoint{7.169921in}{1.371097in}}%
\pgfpathlineto{\pgfqpoint{7.282386in}{1.439523in}}%
\pgfpathlineto{\pgfqpoint{7.380826in}{1.511596in}}%
\pgfpathlineto{\pgfqpoint{7.464558in}{1.586847in}}%
\pgfpathlineto{\pgfqpoint{7.532980in}{1.664785in}}%
\pgfpathlineto{\pgfqpoint{7.585570in}{1.744903in}}%
\pgfpathlineto{\pgfqpoint{7.621891in}{1.826675in}}%
\pgfpathlineto{\pgfqpoint{7.641594in}{1.909562in}}%
\pgfpathlineto{\pgfqpoint{7.644423in}{1.993009in}}%
\pgfpathlineto{\pgfqpoint{7.630216in}{2.076453in}}%
\pgfpathlineto{\pgfqpoint{7.598907in}{2.159320in}}%
\pgfpathlineto{\pgfqpoint{7.550535in}{2.241030in}}%
\pgfpathlineto{\pgfqpoint{7.485240in}{2.320996in}}%
\pgfpathlineto{\pgfqpoint{7.403274in}{2.398632in}}%
\pgfpathlineto{\pgfqpoint{7.304999in}{2.473349in}}%
\pgfpathlineto{\pgfqpoint{7.190890in}{2.544564in}}%
\pgfpathlineto{\pgfqpoint{7.061545in}{2.611700in}}%
\pgfpathlineto{\pgfqpoint{6.917682in}{2.674190in}}%
\pgfpathlineto{\pgfqpoint{6.760144in}{2.731481in}}%
\pgfpathlineto{\pgfqpoint{6.589903in}{2.783038in}}%
\pgfpathlineto{\pgfqpoint{6.408062in}{2.828353in}}%
\pgfpathlineto{\pgfqpoint{6.215854in}{2.866943in}}%
\pgfpathlineto{\pgfqpoint{6.014645in}{2.898363in}}%
\pgfpathlineto{\pgfqpoint{5.805931in}{2.922211in}}%
\pgfpathlineto{\pgfqpoint{5.591335in}{2.938133in}}%
\pgfpathlineto{\pgfqpoint{5.372599in}{2.945832in}}%
\pgfpathlineto{\pgfqpoint{5.151579in}{2.945078in}}%
\pgfpathlineto{\pgfqpoint{4.930226in}{2.935715in}}%
\pgfpathlineto{\pgfqpoint{4.710574in}{2.917667in}}%
\pgfpathlineto{\pgfqpoint{4.494717in}{2.890949in}}%
\pgfpathlineto{\pgfqpoint{4.284785in}{2.855671in}}%
\pgfpathlineto{\pgfqpoint{4.082908in}{2.812046in}}%
\pgfpathlineto{\pgfqpoint{3.891191in}{2.760390in}}%
\pgfpathlineto{\pgfqpoint{3.711672in}{2.701125in}}%
\pgfpathlineto{\pgfqpoint{3.546283in}{2.634777in}}%
\pgfpathlineto{\pgfqpoint{3.396816in}{2.561970in}}%
\pgfpathlineto{\pgfqpoint{3.264880in}{2.483419in}}%
\pgfpathlineto{\pgfqpoint{3.151874in}{2.399920in}}%
\pgfpathlineto{\pgfqpoint{3.058952in}{2.312333in}}%
\pgfpathlineto{\pgfqpoint{2.987008in}{2.221570in}}%
\pgfpathlineto{\pgfqpoint{2.936656in}{2.128575in}}%
\pgfpathlineto{\pgfqpoint{2.908229in}{2.034308in}}%
\pgfpathlineto{\pgfqpoint{2.901780in}{1.939727in}}%
\pgfpathlineto{\pgfqpoint{2.917089in}{1.845770in}}%
\pgfpathlineto{\pgfqpoint{2.953687in}{1.753342in}}%
\pgfpathlineto{\pgfqpoint{3.010872in}{1.663302in}}%
\pgfpathlineto{\pgfqpoint{3.087736in}{1.576452in}}%
\pgfpathlineto{\pgfqpoint{3.183196in}{1.493529in}}%
\pgfpathlineto{\pgfqpoint{3.296020in}{1.415201in}}%
\pgfpathlineto{\pgfqpoint{3.424855in}{1.342060in}}%
\pgfpathlineto{\pgfqpoint{3.568257in}{1.274627in}}%
\pgfpathlineto{\pgfqpoint{3.724712in}{1.213347in}}%
\pgfpathlineto{\pgfqpoint{3.892659in}{1.158594in}}%
\pgfpathlineto{\pgfqpoint{4.070512in}{1.110672in}}%
\pgfpathlineto{\pgfqpoint{4.256672in}{1.069817in}}%
\pgfpathlineto{\pgfqpoint{4.449546in}{1.036206in}}%
\pgfpathlineto{\pgfqpoint{4.647553in}{1.009955in}}%
\pgfpathlineto{\pgfqpoint{4.849139in}{0.991127in}}%
\pgfpathlineto{\pgfqpoint{5.052779in}{0.979731in}}%
\pgfpathlineto{\pgfqpoint{5.256985in}{0.975733in}}%
\pgfpathlineto{\pgfqpoint{5.460312in}{0.979052in}}%
\pgfpathlineto{\pgfqpoint{5.661360in}{0.989567in}}%
\pgfpathlineto{\pgfqpoint{5.858776in}{1.007117in}}%
\pgfpathlineto{\pgfqpoint{6.051258in}{1.031508in}}%
\pgfpathlineto{\pgfqpoint{6.237555in}{1.062508in}}%
\pgfpathlineto{\pgfqpoint{6.416473in}{1.099857in}}%
\pgfpathlineto{\pgfqpoint{6.586872in}{1.143260in}}%
\pgfpathlineto{\pgfqpoint{6.747670in}{1.192396in}}%
\pgfpathlineto{\pgfqpoint{6.897846in}{1.246914in}}%
\pgfpathlineto{\pgfqpoint{7.036441in}{1.306438in}}%
\pgfpathlineto{\pgfqpoint{7.162561in}{1.370566in}}%
\pgfpathlineto{\pgfqpoint{7.275380in}{1.438869in}}%
\pgfpathlineto{\pgfqpoint{7.374143in}{1.510899in}}%
\pgfpathlineto{\pgfqpoint{7.458170in}{1.586183in}}%
\pgfpathlineto{\pgfqpoint{7.526858in}{1.664227in}}%
\pgfpathlineto{\pgfqpoint{7.579685in}{1.744518in}}%
\pgfpathlineto{\pgfqpoint{7.616218in}{1.826527in}}%
\pgfpathlineto{\pgfqpoint{7.636109in}{1.909706in}}%
\pgfpathlineto{\pgfqpoint{7.639109in}{1.993496in}}%
\pgfpathlineto{\pgfqpoint{7.625062in}{2.077324in}}%
\pgfpathlineto{\pgfqpoint{7.593919in}{2.160608in}}%
\pgfpathlineto{\pgfqpoint{7.545732in}{2.242759in}}%
\pgfpathlineto{\pgfqpoint{7.480664in}{2.323185in}}%
\pgfpathlineto{\pgfqpoint{7.398993in}{2.401292in}}%
\pgfpathlineto{\pgfqpoint{7.301109in}{2.476488in}}%
\pgfpathlineto{\pgfqpoint{7.187522in}{2.548189in}}%
\pgfpathlineto{\pgfqpoint{7.058863in}{2.615819in}}%
\pgfpathlineto{\pgfqpoint{6.915884in}{2.678817in}}%
\pgfpathlineto{\pgfqpoint{6.759460in}{2.736639in}}%
\pgfpathlineto{\pgfqpoint{6.590588in}{2.788765in}}%
\pgfpathlineto{\pgfqpoint{6.410390in}{2.834702in}}%
\pgfpathlineto{\pgfqpoint{6.220106in}{2.873991in}}%
\pgfpathlineto{\pgfqpoint{6.021096in}{2.906210in}}%
\pgfpathlineto{\pgfqpoint{5.814832in}{2.930982in}}%
\pgfpathlineto{\pgfqpoint{5.602896in}{2.947980in}}%
\pgfpathlineto{\pgfqpoint{5.386970in}{2.956931in}}%
\pgfpathlineto{\pgfqpoint{5.168827in}{2.957625in}}%
\pgfpathlineto{\pgfqpoint{4.950320in}{2.949921in}}%
\pgfpathlineto{\pgfqpoint{4.733365in}{2.933751in}}%
\pgfpathlineto{\pgfqpoint{4.519927in}{2.909126in}}%
\pgfpathlineto{\pgfqpoint{4.312001in}{2.876142in}}%
\pgfpathlineto{\pgfqpoint{4.111583in}{2.834984in}}%
\pgfpathlineto{\pgfqpoint{3.920651in}{2.785927in}}%
\pgfpathlineto{\pgfqpoint{3.741132in}{2.729339in}}%
\pgfpathlineto{\pgfqpoint{3.574876in}{2.665683in}}%
\pgfpathlineto{\pgfqpoint{3.423623in}{2.595509in}}%
\pgfpathlineto{\pgfqpoint{3.288969in}{2.519454in}}%
\pgfpathlineto{\pgfqpoint{3.172342in}{2.438236in}}%
\pgfpathlineto{\pgfqpoint{3.074967in}{2.352639in}}%
\pgfpathlineto{\pgfqpoint{2.997849in}{2.263508in}}%
\pgfpathlineto{\pgfqpoint{2.941745in}{2.171731in}}%
\pgfpathlineto{\pgfqpoint{2.907158in}{2.078229in}}%
\pgfpathlineto{\pgfqpoint{2.894324in}{1.983935in}}%
\pgfpathlineto{\pgfqpoint{2.903215in}{1.889784in}}%
\pgfpathlineto{\pgfqpoint{2.933544in}{1.796691in}}%
\pgfpathlineto{\pgfqpoint{2.984778in}{1.705544in}}%
\pgfpathlineto{\pgfqpoint{3.056158in}{1.617185in}}%
\pgfpathlineto{\pgfqpoint{3.146716in}{1.532402in}}%
\pgfpathlineto{\pgfqpoint{3.255307in}{1.451921in}}%
\pgfpathlineto{\pgfqpoint{3.380635in}{1.376396in}}%
\pgfpathlineto{\pgfqpoint{3.521279in}{1.306410in}}%
\pgfpathlineto{\pgfqpoint{3.675720in}{1.242466in}}%
\pgfpathlineto{\pgfqpoint{3.842371in}{1.184993in}}%
\pgfpathlineto{\pgfqpoint{4.019599in}{1.134344in}}%
\pgfpathlineto{\pgfqpoint{4.205743in}{1.090798in}}%
\pgfpathlineto{\pgfqpoint{4.399138in}{1.054565in}}%
\pgfpathlineto{\pgfqpoint{4.598130in}{1.025789in}}%
\pgfpathlineto{\pgfqpoint{4.801084in}{1.004554in}}%
\pgfpathlineto{\pgfqpoint{5.006401in}{0.990883in}}%
\pgfpathlineto{\pgfqpoint{5.212522in}{0.984750in}}%
\pgfpathlineto{\pgfqpoint{5.417938in}{0.986079in}}%
\pgfpathlineto{\pgfqpoint{5.621190in}{0.994749in}}%
\pgfpathlineto{\pgfqpoint{5.820877in}{1.010597in}}%
\pgfpathlineto{\pgfqpoint{6.015656in}{1.033425in}}%
\pgfpathlineto{\pgfqpoint{6.204244in}{1.062998in}}%
\pgfpathlineto{\pgfqpoint{6.385420in}{1.099047in}}%
\pgfpathlineto{\pgfqpoint{6.558024in}{1.141277in}}%
\pgfpathlineto{\pgfqpoint{6.720961in}{1.189359in}}%
\pgfpathlineto{\pgfqpoint{6.873198in}{1.242941in}}%
\pgfpathlineto{\pgfqpoint{7.013770in}{1.301643in}}%
\pgfpathlineto{\pgfqpoint{7.141777in}{1.365062in}}%
\pgfpathlineto{\pgfqpoint{7.256386in}{1.432769in}}%
\pgfpathlineto{\pgfqpoint{7.356838in}{1.504315in}}%
\pgfpathlineto{\pgfqpoint{7.442445in}{1.579226in}}%
\pgfpathlineto{\pgfqpoint{7.512597in}{1.657011in}}%
\pgfpathlineto{\pgfqpoint{7.566762in}{1.737155in}}%
\pgfpathlineto{\pgfqpoint{7.604495in}{1.819127in}}%
\pgfpathlineto{\pgfqpoint{7.625438in}{1.902378in}}%
\pgfpathlineto{\pgfqpoint{7.629327in}{1.986344in}}%
\pgfpathlineto{\pgfqpoint{7.615997in}{2.070446in}}%
\pgfpathlineto{\pgfqpoint{7.585386in}{2.154096in}}%
\pgfpathlineto{\pgfqpoint{7.537540in}{2.236694in}}%
\pgfpathlineto{\pgfqpoint{7.472620in}{2.317637in}}%
\pgfpathlineto{\pgfqpoint{7.390904in}{2.396318in}}%
\pgfpathlineto{\pgfqpoint{7.292791in}{2.472132in}}%
\pgfpathlineto{\pgfqpoint{7.178806in}{2.544481in}}%
\pgfpathlineto{\pgfqpoint{7.049604in}{2.612774in}}%
\pgfpathlineto{\pgfqpoint{6.905966in}{2.676439in}}%
\pgfpathlineto{\pgfqpoint{6.748805in}{2.734922in}}%
\pgfpathlineto{\pgfqpoint{6.579163in}{2.787695in}}%
\pgfpathlineto{\pgfqpoint{6.398206in}{2.834262in}}%
\pgfpathlineto{\pgfqpoint{6.207224in}{2.874163in}}%
\pgfpathlineto{\pgfqpoint{6.007626in}{2.906984in}}%
\pgfpathlineto{\pgfqpoint{5.800928in}{2.932358in}}%
\pgfpathlineto{\pgfqpoint{5.588746in}{2.949971in}}%
\pgfpathlineto{\pgfqpoint{5.372790in}{2.959571in}}%
\pgfpathlineto{\pgfqpoint{5.154842in}{2.960971in}}%
\pgfpathlineto{\pgfqpoint{4.936750in}{2.954053in}}%
\pgfpathlineto{\pgfqpoint{4.720408in}{2.938775in}}%
\pgfpathlineto{\pgfqpoint{4.507737in}{2.915169in}}%
\pgfpathlineto{\pgfqpoint{4.300668in}{2.883351in}}%
\pgfpathlineto{\pgfqpoint{4.101120in}{2.843518in}}%
\pgfpathlineto{\pgfqpoint{3.910976in}{2.795952in}}%
\pgfpathlineto{\pgfqpoint{3.732061in}{2.741016in}}%
\pgfpathlineto{\pgfqpoint{3.566120in}{2.679156in}}%
\pgfpathlineto{\pgfqpoint{3.414787in}{2.610898in}}%
\pgfpathlineto{\pgfqpoint{3.279569in}{2.536842in}}%
\pgfpathlineto{\pgfqpoint{3.161817in}{2.457657in}}%
\pgfpathlineto{\pgfqpoint{3.062707in}{2.374074in}}%
\pgfpathlineto{\pgfqpoint{2.983219in}{2.286877in}}%
\pgfpathlineto{\pgfqpoint{2.924121in}{2.196894in}}%
\pgfpathlineto{\pgfqpoint{2.885958in}{2.104983in}}%
\pgfpathlineto{\pgfqpoint{2.869039in}{2.012022in}}%
\pgfpathlineto{\pgfqpoint{2.873437in}{1.918897in}}%
\pgfpathlineto{\pgfqpoint{2.898987in}{1.826488in}}%
\pgfpathlineto{\pgfqpoint{2.945294in}{1.735656in}}%
\pgfpathlineto{\pgfqpoint{3.011742in}{1.647233in}}%
\pgfpathlineto{\pgfqpoint{3.097507in}{1.562009in}}%
\pgfpathlineto{\pgfqpoint{3.201581in}{1.480722in}}%
\pgfpathlineto{\pgfqpoint{3.322785in}{1.404054in}}%
\pgfpathlineto{\pgfqpoint{3.459800in}{1.332621in}}%
\pgfpathlineto{\pgfqpoint{3.611188in}{1.266967in}}%
\pgfpathlineto{\pgfqpoint{3.775414in}{1.207567in}}%
\pgfpathlineto{\pgfqpoint{3.950874in}{1.154819in}}%
\pgfpathlineto{\pgfqpoint{4.135919in}{1.109050in}}%
\pgfpathlineto{\pgfqpoint{4.328870in}{1.070513in}}%
\pgfpathlineto{\pgfqpoint{4.528041in}{1.039394in}}%
\pgfpathlineto{\pgfqpoint{4.731758in}{1.015810in}}%
\pgfpathlineto{\pgfqpoint{4.938367in}{0.999816in}}%
\pgfpathlineto{\pgfqpoint{5.146251in}{0.991411in}}%
\pgfpathlineto{\pgfqpoint{5.353838in}{0.990539in}}%
\pgfpathlineto{\pgfqpoint{5.559607in}{0.997093in}}%
\pgfpathlineto{\pgfqpoint{5.762096in}{1.010923in}}%
\pgfpathlineto{\pgfqpoint{5.959906in}{1.031834in}}%
\pgfpathlineto{\pgfqpoint{6.151702in}{1.059597in}}%
\pgfpathlineto{\pgfqpoint{6.336218in}{1.093945in}}%
\pgfpathlineto{\pgfqpoint{6.512255in}{1.134581in}}%
\pgfpathlineto{\pgfqpoint{6.678682in}{1.181178in}}%
\pgfpathlineto{\pgfqpoint{6.834442in}{1.233381in}}%
\pgfpathlineto{\pgfqpoint{6.978544in}{1.290811in}}%
\pgfpathlineto{\pgfqpoint{7.110070in}{1.353064in}}%
\pgfpathlineto{\pgfqpoint{7.228171in}{1.419712in}}%
\pgfpathlineto{\pgfqpoint{7.332073in}{1.490308in}}%
\pgfpathlineto{\pgfqpoint{7.421074in}{1.564382in}}%
\pgfpathlineto{\pgfqpoint{7.494549in}{1.641446in}}%
\pgfpathlineto{\pgfqpoint{7.551951in}{1.720991in}}%
\pgfpathlineto{\pgfqpoint{7.592815in}{1.802490in}}%
\pgfpathlineto{\pgfqpoint{7.616763in}{1.885399in}}%
\pgfpathlineto{\pgfqpoint{7.623505in}{1.969158in}}%
\pgfpathlineto{\pgfqpoint{7.612848in}{2.053191in}}%
\pgfpathlineto{\pgfqpoint{7.584698in}{2.136910in}}%
\pgfpathlineto{\pgfqpoint{7.539071in}{2.219716in}}%
\pgfpathlineto{\pgfqpoint{7.476091in}{2.300999in}}%
\pgfpathlineto{\pgfqpoint{7.396005in}{2.380147in}}%
\pgfpathlineto{\pgfqpoint{7.299181in}{2.456544in}}%
\pgfpathlineto{\pgfqpoint{7.186119in}{2.529577in}}%
\pgfpathlineto{\pgfqpoint{7.057452in}{2.598639in}}%
\pgfpathlineto{\pgfqpoint{6.913952in}{2.663137in}}%
\pgfpathlineto{\pgfqpoint{6.756531in}{2.722495in}}%
\pgfpathlineto{\pgfqpoint{6.586238in}{2.776163in}}%
\pgfpathlineto{\pgfqpoint{6.404267in}{2.823622in}}%
\pgfpathlineto{\pgfqpoint{6.211941in}{2.864392in}}%
\pgfpathlineto{\pgfqpoint{6.010716in}{2.898036in}}%
\pgfpathlineto{\pgfqpoint{5.802166in}{2.924174in}}%
\pgfpathlineto{\pgfqpoint{5.587975in}{2.942482in}}%
\pgfpathlineto{\pgfqpoint{5.369921in}{2.952702in}}%
\pgfpathlineto{\pgfqpoint{5.149861in}{2.954649in}}%
\pgfpathlineto{\pgfqpoint{4.929711in}{2.948214in}}%
\pgfpathlineto{\pgfqpoint{4.711427in}{2.933367in}}%
\pgfpathlineto{\pgfqpoint{4.496982in}{2.910163in}}%
\pgfpathlineto{\pgfqpoint{4.288342in}{2.878741in}}%
\pgfpathlineto{\pgfqpoint{4.087444in}{2.839324in}}%
\pgfpathlineto{\pgfqpoint{3.896170in}{2.792221in}}%
\pgfpathlineto{\pgfqpoint{3.716324in}{2.737822in}}%
\pgfpathlineto{\pgfqpoint{3.549608in}{2.676596in}}%
\pgfpathlineto{\pgfqpoint{3.397603in}{2.609084in}}%
\pgfpathlineto{\pgfqpoint{3.261743in}{2.535895in}}%
\pgfpathlineto{\pgfqpoint{3.143302in}{2.457699in}}%
\pgfpathlineto{\pgfqpoint{3.043376in}{2.375219in}}%
\pgfpathlineto{\pgfqpoint{2.962868in}{2.289220in}}%
\pgfpathlineto{\pgfqpoint{2.902480in}{2.200502in}}%
\pgfpathlineto{\pgfqpoint{2.862704in}{2.109888in}}%
\pgfpathlineto{\pgfqpoint{2.843816in}{2.018218in}}%
\pgfpathlineto{\pgfqpoint{2.845878in}{1.926334in}}%
\pgfpathlineto{\pgfqpoint{2.868738in}{1.835070in}}%
\pgfpathlineto{\pgfqpoint{2.912038in}{1.745244in}}%
\pgfpathlineto{\pgfqpoint{2.975217in}{1.657651in}}%
\pgfpathlineto{\pgfqpoint{3.057529in}{1.573046in}}%
\pgfpathlineto{\pgfqpoint{3.158051in}{1.492145in}}%
\pgfpathlineto{\pgfqpoint{3.275705in}{1.415612in}}%
\pgfpathlineto{\pgfqpoint{3.409271in}{1.344056in}}%
\pgfpathlineto{\pgfqpoint{3.557406in}{1.278024in}}%
\pgfpathlineto{\pgfqpoint{3.718670in}{1.217999in}}%
\pgfpathlineto{\pgfqpoint{3.891539in}{1.164398in}}%
\pgfpathlineto{\pgfqpoint{4.074427in}{1.117571in}}%
\pgfpathlineto{\pgfqpoint{4.265709in}{1.077797in}}%
\pgfpathlineto{\pgfqpoint{4.463734in}{1.045291in}}%
\pgfpathlineto{\pgfqpoint{4.666844in}{1.020201in}}%
\pgfpathlineto{\pgfqpoint{4.873391in}{1.002613in}}%
\pgfpathlineto{\pgfqpoint{5.081748in}{0.992554in}}%
\pgfpathlineto{\pgfqpoint{5.290322in}{0.989992in}}%
\pgfpathlineto{\pgfqpoint{5.497561in}{0.994846in}}%
\pgfpathlineto{\pgfqpoint{5.701969in}{1.006984in}}%
\pgfpathlineto{\pgfqpoint{5.902106in}{1.026228in}}%
\pgfpathlineto{\pgfqpoint{6.096594in}{1.052361in}}%
\pgfpathlineto{\pgfqpoint{6.284126in}{1.085126in}}%
\pgfpathlineto{\pgfqpoint{6.463461in}{1.124234in}}%
\pgfpathlineto{\pgfqpoint{6.633434in}{1.169362in}}%
\pgfpathlineto{\pgfqpoint{6.792951in}{1.220160in}}%
\pgfpathlineto{\pgfqpoint{6.940992in}{1.276251in}}%
\pgfpathlineto{\pgfqpoint{7.076613in}{1.337232in}}%
\pgfpathlineto{\pgfqpoint{7.198942in}{1.402681in}}%
\pgfpathlineto{\pgfqpoint{7.307187in}{1.472152in}}%
\pgfpathlineto{\pgfqpoint{7.400627in}{1.545180in}}%
\pgfpathlineto{\pgfqpoint{7.478623in}{1.621282in}}%
\pgfpathlineto{\pgfqpoint{7.540610in}{1.699956in}}%
\pgfpathlineto{\pgfqpoint{7.586108in}{1.780685in}}%
\pgfpathlineto{\pgfqpoint{7.614717in}{1.862934in}}%
\pgfpathlineto{\pgfqpoint{7.626124in}{1.946152in}}%
\pgfpathlineto{\pgfqpoint{7.620108in}{2.029777in}}%
\pgfpathlineto{\pgfqpoint{7.596542in}{2.113230in}}%
\pgfpathlineto{\pgfqpoint{7.555399in}{2.195923in}}%
\pgfpathlineto{\pgfqpoint{7.496759in}{2.277256in}}%
\pgfpathlineto{\pgfqpoint{7.420815in}{2.356622in}}%
\pgfpathlineto{\pgfqpoint{7.327877in}{2.433410in}}%
\pgfpathlineto{\pgfqpoint{7.218385in}{2.507004in}}%
\pgfpathlineto{\pgfqpoint{7.092907in}{2.576795in}}%
\pgfpathlineto{\pgfqpoint{6.952152in}{2.642178in}}%
\pgfpathlineto{\pgfqpoint{6.796972in}{2.702561in}}%
\pgfpathlineto{\pgfqpoint{6.628367in}{2.757373in}}%
\pgfpathlineto{\pgfqpoint{6.447484in}{2.806069in}}%
\pgfpathlineto{\pgfqpoint{6.255619in}{2.848136in}}%
\pgfpathlineto{\pgfqpoint{6.054215in}{2.883105in}}%
\pgfpathlineto{\pgfqpoint{5.844850in}{2.910557in}}%
\pgfpathlineto{\pgfqpoint{5.629235in}{2.930132in}}%
\pgfpathlineto{\pgfqpoint{5.409192in}{2.941539in}}%
\pgfpathlineto{\pgfqpoint{5.186642in}{2.944561in}}%
\pgfpathlineto{\pgfqpoint{4.963585in}{2.939062in}}%
\pgfpathlineto{\pgfqpoint{4.742071in}{2.924995in}}%
\pgfpathlineto{\pgfqpoint{4.524175in}{2.902405in}}%
\pgfpathlineto{\pgfqpoint{4.311971in}{2.871430in}}%
\pgfpathlineto{\pgfqpoint{4.107500in}{2.832303in}}%
\pgfpathlineto{\pgfqpoint{3.912738in}{2.785349in}}%
\pgfpathlineto{\pgfqpoint{3.729571in}{2.730984in}}%
\pgfpathlineto{\pgfqpoint{3.559764in}{2.669708in}}%
\pgfpathlineto{\pgfqpoint{3.404939in}{2.602096in}}%
\pgfpathlineto{\pgfqpoint{3.266551in}{2.528794in}}%
\pgfpathlineto{\pgfqpoint{3.145869in}{2.450506in}}%
\pgfpathlineto{\pgfqpoint{3.043963in}{2.367985in}}%
\pgfpathlineto{\pgfqpoint{2.961696in}{2.282021in}}%
\pgfpathlineto{\pgfqpoint{2.899713in}{2.193432in}}%
\pgfpathlineto{\pgfqpoint{2.858442in}{2.103052in}}%
\pgfpathlineto{\pgfqpoint{2.838092in}{2.011720in}}%
\pgfpathlineto{\pgfqpoint{2.838661in}{1.920270in}}%
\pgfpathlineto{\pgfqpoint{2.859941in}{1.829521in}}%
\pgfpathlineto{\pgfqpoint{2.901527in}{1.740271in}}%
\pgfpathlineto{\pgfqpoint{2.962831in}{1.653285in}}%
\pgfpathlineto{\pgfqpoint{3.043090in}{1.569292in}}%
\pgfpathlineto{\pgfqpoint{3.141388in}{1.488975in}}%
\pgfpathlineto{\pgfqpoint{3.256665in}{1.412969in}}%
\pgfpathlineto{\pgfqpoint{3.387736in}{1.341856in}}%
\pgfpathlineto{\pgfqpoint{3.533309in}{1.276160in}}%
\pgfpathlineto{\pgfqpoint{3.691998in}{1.216347in}}%
\pgfpathlineto{\pgfqpoint{3.862343in}{1.162819in}}%
\pgfpathlineto{\pgfqpoint{4.042824in}{1.115920in}}%
\pgfpathlineto{\pgfqpoint{4.231878in}{1.075928in}}%
\pgfpathlineto{\pgfqpoint{4.427916in}{1.043062in}}%
\pgfpathlineto{\pgfqpoint{4.629334in}{1.017479in}}%
\pgfpathlineto{\pgfqpoint{4.834528in}{0.999276in}}%
\pgfpathlineto{\pgfqpoint{5.041906in}{0.988494in}}%
\pgfpathlineto{\pgfqpoint{5.249900in}{0.985120in}}%
\pgfpathlineto{\pgfqpoint{5.456977in}{0.989087in}}%
\pgfpathlineto{\pgfqpoint{5.661643in}{1.000281in}}%
\pgfpathlineto{\pgfqpoint{5.862458in}{1.018541in}}%
\pgfpathlineto{\pgfqpoint{6.058036in}{1.043663in}}%
\pgfpathlineto{\pgfqpoint{6.247054in}{1.075404in}}%
\pgfpathlineto{\pgfqpoint{6.428254in}{1.113486in}}%
\pgfpathlineto{\pgfqpoint{6.600449in}{1.157595in}}%
\pgfpathlineto{\pgfqpoint{6.762523in}{1.207388in}}%
\pgfpathlineto{\pgfqpoint{6.913435in}{1.262493in}}%
\pgfpathlineto{\pgfqpoint{7.052221in}{1.322514in}}%
\pgfpathlineto{\pgfqpoint{7.177992in}{1.387031in}}%
\pgfpathlineto{\pgfqpoint{7.289938in}{1.455602in}}%
\pgfpathlineto{\pgfqpoint{7.387327in}{1.527765in}}%
\pgfpathlineto{\pgfqpoint{7.469505in}{1.603044in}}%
\pgfpathlineto{\pgfqpoint{7.535900in}{1.680940in}}%
\pgfpathlineto{\pgfqpoint{7.586021in}{1.760945in}}%
\pgfpathlineto{\pgfqpoint{7.619459in}{1.842531in}}%
\pgfpathlineto{\pgfqpoint{7.635889in}{1.925161in}}%
\pgfpathlineto{\pgfqpoint{7.635077in}{2.008283in}}%
\pgfpathlineto{\pgfqpoint{7.616874in}{2.091333in}}%
\pgfpathlineto{\pgfqpoint{7.581231in}{2.173741in}}%
\pgfpathlineto{\pgfqpoint{7.528193in}{2.254922in}}%
\pgfpathlineto{\pgfqpoint{7.457912in}{2.334288in}}%
\pgfpathlineto{\pgfqpoint{7.370649in}{2.411244in}}%
\pgfpathlineto{\pgfqpoint{7.266778in}{2.485191in}}%
\pgfpathlineto{\pgfqpoint{7.146799in}{2.555532in}}%
\pgfpathlineto{\pgfqpoint{7.011340in}{2.621672in}}%
\pgfpathlineto{\pgfqpoint{6.861161in}{2.683025in}}%
\pgfpathlineto{\pgfqpoint{6.697167in}{2.739015in}}%
\pgfpathlineto{\pgfqpoint{6.520407in}{2.789090in}}%
\pgfpathlineto{\pgfqpoint{6.332080in}{2.832719in}}%
\pgfpathlineto{\pgfqpoint{6.133534in}{2.869410in}}%
\pgfpathlineto{\pgfqpoint{5.926270in}{2.898709in}}%
\pgfpathlineto{\pgfqpoint{5.711927in}{2.920216in}}%
\pgfpathlineto{\pgfqpoint{5.492286in}{2.933591in}}%
\pgfpathlineto{\pgfqpoint{5.269246in}{2.938564in}}%
\pgfpathlineto{\pgfqpoint{5.044814in}{2.934947in}}%
\pgfpathlineto{\pgfqpoint{4.821076in}{2.922638in}}%
\pgfpathlineto{\pgfqpoint{4.600177in}{2.901630in}}%
\pgfpathlineto{\pgfqpoint{4.384285in}{2.872018in}}%
\pgfpathlineto{\pgfqpoint{4.175558in}{2.833998in}}%
\pgfpathlineto{\pgfqpoint{3.976111in}{2.787872in}}%
\pgfpathlineto{\pgfqpoint{3.787977in}{2.734044in}}%
\pgfpathlineto{\pgfqpoint{3.613069in}{2.673016in}}%
\pgfpathlineto{\pgfqpoint{3.453152in}{2.605379in}}%
\pgfpathlineto{\pgfqpoint{3.309812in}{2.531804in}}%
\pgfpathlineto{\pgfqpoint{3.184427in}{2.453033in}}%
\pgfpathlineto{\pgfqpoint{3.078153in}{2.369862in}}%
\pgfpathlineto{\pgfqpoint{2.991910in}{2.283130in}}%
\pgfpathlineto{\pgfqpoint{2.926371in}{2.193703in}}%
\pgfpathlineto{\pgfqpoint{2.881967in}{2.102462in}}%
\pgfpathlineto{\pgfqpoint{2.858887in}{2.010289in}}%
\pgfpathlineto{\pgfqpoint{2.857086in}{1.918052in}}%
\pgfpathlineto{\pgfqpoint{2.876300in}{1.826601in}}%
\pgfpathlineto{\pgfqpoint{2.916061in}{1.736750in}}%
\pgfpathlineto{\pgfqpoint{2.975712in}{1.649277in}}%
\pgfpathlineto{\pgfqpoint{3.054426in}{1.564911in}}%
\pgfpathlineto{\pgfqpoint{3.151226in}{1.484331in}}%
\pgfpathlineto{\pgfqpoint{3.265004in}{1.408160in}}%
\pgfpathlineto{\pgfqpoint{3.394537in}{1.336963in}}%
\pgfpathlineto{\pgfqpoint{3.538509in}{1.271246in}}%
\pgfpathlineto{\pgfqpoint{3.695524in}{1.211452in}}%
\pgfpathlineto{\pgfqpoint{3.864125in}{1.157964in}}%
\pgfpathlineto{\pgfqpoint{4.042807in}{1.111105in}}%
\pgfpathlineto{\pgfqpoint{4.230033in}{1.071136in}}%
\pgfpathlineto{\pgfqpoint{4.424245in}{1.038261in}}%
\pgfpathlineto{\pgfqpoint{4.623878in}{1.012625in}}%
\pgfpathlineto{\pgfqpoint{4.827368in}{0.994318in}}%
\pgfpathlineto{\pgfqpoint{5.033165in}{0.983376in}}%
\pgfpathlineto{\pgfqpoint{5.239742in}{0.979786in}}%
\pgfpathlineto{\pgfqpoint{5.445600in}{0.983483in}}%
\pgfpathlineto{\pgfqpoint{5.649281in}{0.994356in}}%
\pgfpathlineto{\pgfqpoint{5.849367in}{1.012251in}}%
\pgfpathlineto{\pgfqpoint{6.044496in}{1.036972in}}%
\pgfpathlineto{\pgfqpoint{6.233360in}{1.068283in}}%
\pgfpathlineto{\pgfqpoint{6.414709in}{1.105914in}}%
\pgfpathlineto{\pgfqpoint{6.587363in}{1.149557in}}%
\pgfpathlineto{\pgfqpoint{6.750206in}{1.198878in}}%
\pgfpathlineto{\pgfqpoint{6.902196in}{1.253510in}}%
\pgfpathlineto{\pgfqpoint{7.042363in}{1.313061in}}%
\pgfpathlineto{\pgfqpoint{7.169815in}{1.377114in}}%
\pgfpathlineto{\pgfqpoint{7.283735in}{1.445232in}}%
\pgfpathlineto{\pgfqpoint{7.383387in}{1.516956in}}%
\pgfpathlineto{\pgfqpoint{7.468115in}{1.591809in}}%
\pgfpathlineto{\pgfqpoint{7.537345in}{1.669298in}}%
\pgfpathlineto{\pgfqpoint{7.590585in}{1.748916in}}%
\pgfpathlineto{\pgfqpoint{7.627427in}{1.830140in}}%
\pgfpathlineto{\pgfqpoint{7.647550in}{1.912438in}}%
\pgfpathlineto{\pgfqpoint{7.650719in}{1.995266in}}%
\pgfpathlineto{\pgfqpoint{7.636788in}{2.078071in}}%
\pgfpathlineto{\pgfqpoint{7.605705in}{2.160293in}}%
\pgfpathlineto{\pgfqpoint{7.557511in}{2.241364in}}%
\pgfpathlineto{\pgfqpoint{7.492343in}{2.320713in}}%
\pgfpathlineto{\pgfqpoint{7.410443in}{2.397762in}}%
\pgfpathlineto{\pgfqpoint{7.312158in}{2.471936in}}%
\pgfpathlineto{\pgfqpoint{7.197946in}{2.542659in}}%
\pgfpathlineto{\pgfqpoint{7.068379in}{2.609357in}}%
\pgfpathlineto{\pgfqpoint{6.924154in}{2.671465in}}%
\pgfpathlineto{\pgfqpoint{6.766092in}{2.728429in}}%
\pgfpathlineto{\pgfqpoint{6.595147in}{2.779708in}}%
\pgfpathlineto{\pgfqpoint{6.412408in}{2.824783in}}%
\pgfpathlineto{\pgfqpoint{6.219106in}{2.863162in}}%
\pgfpathlineto{\pgfqpoint{6.016611in}{2.894385in}}%
\pgfpathlineto{\pgfqpoint{5.806438in}{2.918036in}}%
\pgfpathlineto{\pgfqpoint{5.590235in}{2.933748in}}%
\pgfpathlineto{\pgfqpoint{5.369787in}{2.941212in}}%
\pgfpathlineto{\pgfqpoint{5.146996in}{2.940190in}}%
\pgfpathlineto{\pgfqpoint{4.923872in}{2.930520in}}%
\pgfpathlineto{\pgfqpoint{4.702511in}{2.912130in}}%
\pgfpathlineto{\pgfqpoint{4.485066in}{2.885042in}}%
\pgfpathlineto{\pgfqpoint{4.273724in}{2.849380in}}%
\pgfpathlineto{\pgfqpoint{4.070664in}{2.805374in}}%
\pgfpathlineto{\pgfqpoint{3.878023in}{2.753365in}}%
\pgfpathlineto{\pgfqpoint{3.697854in}{2.693801in}}%
\pgfpathlineto{\pgfqpoint{3.532088in}{2.627233in}}%
\pgfpathlineto{\pgfqpoint{3.382495in}{2.554312in}}%
\pgfpathlineto{\pgfqpoint{3.250644in}{2.475771in}}%
\pgfpathlineto{\pgfqpoint{3.137881in}{2.392420in}}%
\pgfpathlineto{\pgfqpoint{3.045299in}{2.305125in}}%
\pgfpathlineto{\pgfqpoint{2.973725in}{2.214794in}}%
\pgfpathlineto{\pgfqpoint{2.923713in}{2.122362in}}%
\pgfpathlineto{\pgfqpoint{2.895542in}{2.028769in}}%
\pgfpathlineto{\pgfqpoint{2.889223in}{1.934948in}}%
\pgfpathlineto{\pgfqpoint{2.904517in}{1.841810in}}%
\pgfpathlineto{\pgfqpoint{2.940945in}{1.750229in}}%
\pgfpathlineto{\pgfqpoint{2.997818in}{1.661034in}}%
\pgfpathlineto{\pgfqpoint{3.074254in}{1.575001in}}%
\pgfpathlineto{\pgfqpoint{3.169210in}{1.492843in}}%
\pgfpathlineto{\pgfqpoint{3.281500in}{1.415211in}}%
\pgfpathlineto{\pgfqpoint{3.409825in}{1.342685in}}%
\pgfpathlineto{\pgfqpoint{3.552792in}{1.275781in}}%
\pgfpathlineto{\pgfqpoint{3.708936in}{1.214942in}}%
\pgfpathlineto{\pgfqpoint{3.876741in}{1.160547in}}%
\pgfpathlineto{\pgfqpoint{4.054653in}{1.112907in}}%
\pgfpathlineto{\pgfqpoint{4.241098in}{1.072272in}}%
\pgfpathlineto{\pgfqpoint{4.434495in}{1.038830in}}%
\pgfpathlineto{\pgfqpoint{4.633266in}{1.012710in}}%
\pgfpathlineto{\pgfqpoint{4.835846in}{0.993987in}}%
\pgfpathlineto{\pgfqpoint{5.040693in}{0.982683in}}%
\pgfpathlineto{\pgfqpoint{5.246293in}{0.978771in}}%
\pgfpathlineto{\pgfqpoint{5.451168in}{0.982176in}}%
\pgfpathlineto{\pgfqpoint{5.653883in}{0.992780in}}%
\pgfpathlineto{\pgfqpoint{5.853048in}{1.010422in}}%
\pgfpathlineto{\pgfqpoint{6.047324in}{1.034903in}}%
\pgfpathlineto{\pgfqpoint{6.235427in}{1.065985in}}%
\pgfpathlineto{\pgfqpoint{6.416133in}{1.103397in}}%
\pgfpathlineto{\pgfqpoint{6.588278in}{1.146835in}}%
\pgfpathlineto{\pgfqpoint{6.750766in}{1.195964in}}%
\pgfpathlineto{\pgfqpoint{6.902567in}{1.250419in}}%
\pgfpathlineto{\pgfqpoint{7.042722in}{1.309813in}}%
\pgfpathlineto{\pgfqpoint{7.170347in}{1.373729in}}%
\pgfpathlineto{\pgfqpoint{7.284632in}{1.441731in}}%
\pgfpathlineto{\pgfqpoint{7.384845in}{1.513362in}}%
\pgfpathlineto{\pgfqpoint{7.470336in}{1.588145in}}%
\pgfpathlineto{\pgfqpoint{7.540534in}{1.665585in}}%
\pgfpathlineto{\pgfqpoint{7.594955in}{1.745176in}}%
\pgfpathlineto{\pgfqpoint{7.633198in}{1.826395in}}%
\pgfpathlineto{\pgfqpoint{7.654950in}{1.908708in}}%
\pgfpathlineto{\pgfqpoint{7.659987in}{1.991573in}}%
\pgfpathlineto{\pgfqpoint{7.648177in}{2.074438in}}%
\pgfpathlineto{\pgfqpoint{7.619479in}{2.156746in}}%
\pgfpathlineto{\pgfqpoint{7.573949in}{2.237935in}}%
\pgfpathlineto{\pgfqpoint{7.511738in}{2.317440in}}%
\pgfpathlineto{\pgfqpoint{7.433098in}{2.394698in}}%
\pgfpathlineto{\pgfqpoint{7.338382in}{2.469145in}}%
\pgfpathlineto{\pgfqpoint{7.228049in}{2.540220in}}%
\pgfpathlineto{\pgfqpoint{7.102664in}{2.607371in}}%
\pgfpathlineto{\pgfqpoint{6.962905in}{2.670054in}}%
\pgfpathlineto{\pgfqpoint{6.809564in}{2.727737in}}%
\pgfpathlineto{\pgfqpoint{6.643550in}{2.779904in}}%
\pgfpathlineto{\pgfqpoint{6.465892in}{2.826060in}}%
\pgfpathlineto{\pgfqpoint{6.277743in}{2.865735in}}%
\pgfpathlineto{\pgfqpoint{6.080379in}{2.898488in}}%
\pgfpathlineto{\pgfqpoint{5.875201in}{2.923914in}}%
\pgfpathlineto{\pgfqpoint{5.663734in}{2.941652in}}%
\pgfpathlineto{\pgfqpoint{5.447624in}{2.951390in}}%
\pgfpathlineto{\pgfqpoint{5.228629in}{2.952875in}}%
\pgfpathlineto{\pgfqpoint{5.008616in}{2.945919in}}%
\pgfpathlineto{\pgfqpoint{4.789539in}{2.930409in}}%
\pgfpathlineto{\pgfqpoint{4.573431in}{2.906313in}}%
\pgfpathlineto{\pgfqpoint{4.362377in}{2.873693in}}%
\pgfpathlineto{\pgfqpoint{4.158487in}{2.832702in}}%
\pgfpathlineto{\pgfqpoint{3.963868in}{2.783600in}}%
\pgfpathlineto{\pgfqpoint{3.780588in}{2.726750in}}%
\pgfpathlineto{\pgfqpoint{3.610637in}{2.662620in}}%
\pgfpathlineto{\pgfqpoint{3.455889in}{2.591782in}}%
\pgfpathlineto{\pgfqpoint{3.318062in}{2.514904in}}%
\pgfpathlineto{\pgfqpoint{3.198682in}{2.432744in}}%
\pgfpathlineto{\pgfqpoint{3.099047in}{2.346137in}}%
\pgfpathlineto{\pgfqpoint{3.020199in}{2.255977in}}%
\pgfpathlineto{\pgfqpoint{2.962910in}{2.163207in}}%
\pgfpathlineto{\pgfqpoint{2.927662in}{2.068794in}}%
\pgfpathlineto{\pgfqpoint{2.914646in}{1.973715in}}%
\pgfpathlineto{\pgfqpoint{2.923772in}{1.878938in}}%
\pgfpathlineto{\pgfqpoint{2.954675in}{1.785404in}}%
\pgfpathlineto{\pgfqpoint{3.006738in}{1.694014in}}%
\pgfpathlineto{\pgfqpoint{3.079118in}{1.605615in}}%
\pgfpathlineto{\pgfqpoint{3.170770in}{1.520993in}}%
\pgfpathlineto{\pgfqpoint{3.280481in}{1.440860in}}%
\pgfpathlineto{\pgfqpoint{3.406896in}{1.365856in}}%
\pgfpathlineto{\pgfqpoint{3.548551in}{1.296542in}}%
\pgfpathlineto{\pgfqpoint{3.703899in}{1.233401in}}%
\pgfpathlineto{\pgfqpoint{3.871331in}{1.176838in}}%
\pgfpathlineto{\pgfqpoint{4.049206in}{1.127185in}}%
\pgfpathlineto{\pgfqpoint{4.235864in}{1.084700in}}%
\pgfpathlineto{\pgfqpoint{4.429645in}{1.049577in}}%
\pgfpathlineto{\pgfqpoint{4.628902in}{1.021941in}}%
\pgfpathlineto{\pgfqpoint{4.832013in}{1.001861in}}%
\pgfpathlineto{\pgfqpoint{5.037389in}{0.989350in}}%
\pgfpathlineto{\pgfqpoint{5.243482in}{0.984368in}}%
\pgfpathlineto{\pgfqpoint{5.448791in}{0.986828in}}%
\pgfpathlineto{\pgfqpoint{5.651866in}{0.996600in}}%
\pgfpathlineto{\pgfqpoint{5.851312in}{1.013510in}}%
\pgfpathlineto{\pgfqpoint{6.045791in}{1.037348in}}%
\pgfpathlineto{\pgfqpoint{6.234028in}{1.067870in}}%
\pgfpathlineto{\pgfqpoint{6.414808in}{1.104794in}}%
\pgfpathlineto{\pgfqpoint{6.586981in}{1.147812in}}%
\pgfpathlineto{\pgfqpoint{6.749464in}{1.196584in}}%
\pgfpathlineto{\pgfqpoint{6.901242in}{1.250744in}}%
\pgfpathlineto{\pgfqpoint{7.041369in}{1.309900in}}%
\pgfpathlineto{\pgfqpoint{7.168973in}{1.373636in}}%
\pgfpathlineto{\pgfqpoint{7.283255in}{1.441515in}}%
\pgfpathlineto{\pgfqpoint{7.383491in}{1.513078in}}%
\pgfpathlineto{\pgfqpoint{7.469038in}{1.587848in}}%
\pgfpathlineto{\pgfqpoint{7.539332in}{1.665330in}}%
\pgfpathlineto{\pgfqpoint{7.593894in}{1.745013in}}%
\pgfpathlineto{\pgfqpoint{7.632330in}{1.826372in}}%
\pgfpathlineto{\pgfqpoint{7.654334in}{1.908871in}}%
\pgfpathlineto{\pgfqpoint{7.659691in}{1.991963in}}%
\pgfpathlineto{\pgfqpoint{7.648280in}{2.075091in}}%
\pgfpathlineto{\pgfqpoint{7.620074in}{2.157695in}}%
\pgfpathlineto{\pgfqpoint{7.575144in}{2.239209in}}%
\pgfpathlineto{\pgfqpoint{7.513663in}{2.319067in}}%
\pgfpathlineto{\pgfqpoint{7.435904in}{2.396702in}}%
\pgfpathlineto{\pgfqpoint{7.342244in}{2.471551in}}%
\pgfpathlineto{\pgfqpoint{7.233166in}{2.543059in}}%
\pgfpathlineto{\pgfqpoint{7.109260in}{2.610679in}}%
\pgfpathlineto{\pgfqpoint{6.971226in}{2.673877in}}%
\pgfpathlineto{\pgfqpoint{6.819872in}{2.732132in}}%
\pgfpathlineto{\pgfqpoint{6.656119in}{2.784946in}}%
\pgfpathlineto{\pgfqpoint{6.480996in}{2.831843in}}%
\pgfpathlineto{\pgfqpoint{6.295645in}{2.872373in}}%
\pgfpathlineto{\pgfqpoint{6.101318in}{2.906120in}}%
\pgfpathlineto{\pgfqpoint{5.899376in}{2.932702in}}%
\pgfpathlineto{\pgfqpoint{5.691284in}{2.951780in}}%
\pgfpathlineto{\pgfqpoint{5.478612in}{2.963064in}}%
\pgfpathlineto{\pgfqpoint{5.263024in}{2.966315in}}%
\pgfpathlineto{\pgfqpoint{5.263024in}{2.966315in}}%
\pgfusepath{stroke}%
\end{pgfscope}%
\begin{pgfscope}%
\pgfpathrectangle{\pgfqpoint{1.250000in}{0.400000in}}{\pgfqpoint{7.750000in}{3.200000in}} %
\pgfusepath{clip}%
\pgfsetrectcap%
\pgfsetroundjoin%
\pgfsetlinewidth{1.003750pt}%
\definecolor{currentstroke}{rgb}{1.000000,0.000000,0.000000}%
\pgfsetstrokecolor{currentstroke}%
\pgfsetdash{}{0pt}%
\pgfpathmoveto{\pgfqpoint{7.607583in}{2.000000in}}%
\pgfpathlineto{\pgfqpoint{5.858791in}{2.848682in}}%
\pgfpathlineto{\pgfqpoint{3.348034in}{2.804523in}}%
\pgfpathlineto{\pgfqpoint{1.882399in}{2.278071in}}%
\pgfpathlineto{\pgfqpoint{1.366435in}{1.664179in}}%
\pgfpathlineto{\pgfqpoint{1.540731in}{1.111160in}}%
\pgfpathlineto{\pgfqpoint{2.216683in}{0.687881in}}%
\pgfpathlineto{\pgfqpoint{3.256730in}{0.441543in}}%
\pgfpathlineto{\pgfqpoint{4.540117in}{0.421738in}}%
\pgfpathlineto{\pgfqpoint{5.910450in}{0.700018in}}%
\pgfpathlineto{\pgfqpoint{7.016010in}{1.395424in}}%
\pgfpathlineto{\pgfqpoint{6.613598in}{2.576976in}}%
\pgfpathlineto{\pgfqpoint{4.318200in}{3.024126in}}%
\pgfpathlineto{\pgfqpoint{2.532112in}{2.842125in}}%
\pgfpathlineto{\pgfqpoint{1.542242in}{2.400100in}}%
\pgfpathlineto{\pgfqpoint{1.319875in}{1.865108in}}%
\pgfpathlineto{\pgfqpoint{1.840715in}{1.346566in}}%
\pgfpathlineto{\pgfqpoint{3.094470in}{0.963974in}}%
\pgfpathlineto{\pgfqpoint{4.996626in}{0.906639in}}%
\pgfpathlineto{\pgfqpoint{6.961870in}{1.481373in}}%
\pgfpathlineto{\pgfqpoint{7.100032in}{2.568401in}}%
\pgfpathlineto{\pgfqpoint{5.693472in}{3.218485in}}%
\pgfpathlineto{\pgfqpoint{4.064021in}{3.389662in}}%
\pgfpathlineto{\pgfqpoint{2.707781in}{3.226959in}}%
\pgfpathlineto{\pgfqpoint{1.833048in}{2.835848in}}%
\pgfpathlineto{\pgfqpoint{1.594048in}{2.293993in}}%
\pgfpathlineto{\pgfqpoint{2.192776in}{1.687263in}}%
\pgfpathlineto{\pgfqpoint{3.955170in}{1.202628in}}%
\pgfpathlineto{\pgfqpoint{6.834395in}{1.467743in}}%
\pgfpathlineto{\pgfqpoint{7.813516in}{2.317613in}}%
\pgfpathlineto{\pgfqpoint{7.441826in}{3.007185in}}%
\pgfpathlineto{\pgfqpoint{6.390555in}{3.410842in}}%
\pgfpathlineto{\pgfqpoint{5.034878in}{3.502443in}}%
\pgfpathlineto{\pgfqpoint{3.679858in}{3.262550in}}%
\pgfpathlineto{\pgfqpoint{2.714704in}{2.676058in}}%
\pgfpathlineto{\pgfqpoint{2.800298in}{1.800229in}}%
\pgfpathlineto{\pgfqpoint{4.778243in}{1.092376in}}%
\pgfpathlineto{\pgfqpoint{7.112063in}{1.281150in}}%
\pgfpathlineto{\pgfqpoint{8.287829in}{1.888864in}}%
\pgfpathlineto{\pgfqpoint{8.398984in}{2.526602in}}%
\pgfpathlineto{\pgfqpoint{7.694593in}{3.024212in}}%
\pgfpathlineto{\pgfqpoint{6.375267in}{3.270189in}}%
\pgfpathlineto{\pgfqpoint{4.675958in}{3.133547in}}%
\pgfpathlineto{\pgfqpoint{3.189397in}{2.421339in}}%
\pgfpathlineto{\pgfqpoint{3.699751in}{1.272648in}}%
\pgfpathlineto{\pgfqpoint{5.540530in}{0.809368in}}%
\pgfpathlineto{\pgfqpoint{7.208118in}{0.874544in}}%
\pgfpathlineto{\pgfqpoint{8.333864in}{1.246217in}}%
\pgfpathlineto{\pgfqpoint{8.791224in}{1.782763in}}%
\pgfpathlineto{\pgfqpoint{8.490837in}{2.366378in}}%
\pgfpathlineto{\pgfqpoint{7.347072in}{2.854557in}}%
\pgfpathlineto{\pgfqpoint{5.357641in}{3.008011in}}%
\pgfpathlineto{\pgfqpoint{3.202463in}{2.413224in}}%
\pgfpathlineto{\pgfqpoint{3.094139in}{1.372855in}}%
\pgfpathlineto{\pgfqpoint{4.324803in}{0.750772in}}%
\pgfpathlineto{\pgfqpoint{5.824192in}{0.574729in}}%
\pgfpathlineto{\pgfqpoint{7.139042in}{0.758203in}}%
\pgfpathlineto{\pgfqpoint{7.987844in}{1.232575in}}%
\pgfpathlineto{\pgfqpoint{8.056087in}{1.923650in}}%
\pgfpathlineto{\pgfqpoint{6.873957in}{2.653774in}}%
\pgfpathlineto{\pgfqpoint{4.254443in}{2.845339in}}%
\pgfpathlineto{\pgfqpoint{2.505325in}{2.199078in}}%
\pgfpathlineto{\pgfqpoint{2.264215in}{1.432471in}}%
\pgfpathlineto{\pgfqpoint{2.987035in}{0.851698in}}%
\pgfpathlineto{\pgfqpoint{4.233500in}{0.552906in}}%
\pgfpathlineto{\pgfqpoint{5.667513in}{0.580681in}}%
\pgfpathlineto{\pgfqpoint{6.944324in}{0.970809in}}%
\pgfpathlineto{\pgfqpoint{7.533200in}{1.741271in}}%
\pgfpathlineto{\pgfqpoint{6.412347in}{2.697553in}}%
\pgfpathlineto{\pgfqpoint{3.831010in}{2.866345in}}%
\pgfpathlineto{\pgfqpoint{2.201138in}{2.408276in}}%
\pgfpathlineto{\pgfqpoint{1.665808in}{1.798123in}}%
\pgfpathlineto{\pgfqpoint{2.024126in}{1.235076in}}%
\pgfpathlineto{\pgfqpoint{3.106663in}{0.841623in}}%
\pgfpathlineto{\pgfqpoint{4.726131in}{0.745084in}}%
\pgfpathlineto{\pgfqpoint{6.496719in}{1.114483in}}%
\pgfpathlineto{\pgfqpoint{7.356877in}{2.062415in}}%
\pgfpathlineto{\pgfqpoint{6.036236in}{2.949230in}}%
\pgfpathlineto{\pgfqpoint{4.052466in}{3.157999in}}%
\pgfpathlineto{\pgfqpoint{2.498106in}{2.909291in}}%
\pgfpathlineto{\pgfqpoint{1.656362in}{2.423287in}}%
\pgfpathlineto{\pgfqpoint{1.619170in}{1.846916in}}%
\pgfpathlineto{\pgfqpoint{2.462960in}{1.313023in}}%
\pgfpathlineto{\pgfqpoint{4.255925in}{1.022008in}}%
\pgfpathlineto{\pgfqpoint{6.650324in}{1.391589in}}%
\pgfpathlineto{\pgfqpoint{7.308172in}{2.450666in}}%
\pgfpathlineto{\pgfqpoint{6.324098in}{3.169008in}}%
\pgfpathlineto{\pgfqpoint{4.876014in}{3.449963in}}%
\pgfpathlineto{\pgfqpoint{3.470559in}{3.372824in}}%
\pgfpathlineto{\pgfqpoint{2.395547in}{3.009808in}}%
\pgfpathlineto{\pgfqpoint{1.904303in}{2.430581in}}%
\pgfpathlineto{\pgfqpoint{2.301660in}{1.737818in}}%
\pgfpathlineto{\pgfqpoint{3.960996in}{1.169103in}}%
\pgfpathlineto{\pgfqpoint{6.634758in}{1.324182in}}%
\pgfpathlineto{\pgfqpoint{7.833528in}{2.144417in}}%
\pgfpathlineto{\pgfqpoint{7.597121in}{2.882419in}}%
\pgfpathlineto{\pgfqpoint{6.596145in}{3.338895in}}%
\pgfpathlineto{\pgfqpoint{5.239647in}{3.466378in}}%
\pgfpathlineto{\pgfqpoint{3.846989in}{3.232723in}}%
\pgfpathlineto{\pgfqpoint{2.845717in}{2.603437in}}%
\pgfpathlineto{\pgfqpoint{3.076533in}{1.641711in}}%
\pgfusepath{stroke}%
\end{pgfscope}%
\begin{pgfscope}%
\pgfpathrectangle{\pgfqpoint{1.250000in}{0.400000in}}{\pgfqpoint{7.750000in}{3.200000in}} %
\pgfusepath{clip}%
\pgfsetbuttcap%
\pgfsetroundjoin%
\pgfsetlinewidth{0.501875pt}%
\definecolor{currentstroke}{rgb}{0.000000,0.000000,0.000000}%
\pgfsetstrokecolor{currentstroke}%
\pgfsetdash{{1.000000pt}{3.000000pt}}{0.000000pt}%
\pgfpathmoveto{\pgfqpoint{1.250000in}{0.400000in}}%
\pgfpathlineto{\pgfqpoint{1.250000in}{3.600000in}}%
\pgfusepath{stroke}%
\end{pgfscope}%
\begin{pgfscope}%
\pgfsetbuttcap%
\pgfsetroundjoin%
\definecolor{currentfill}{rgb}{0.000000,0.000000,0.000000}%
\pgfsetfillcolor{currentfill}%
\pgfsetlinewidth{0.501875pt}%
\definecolor{currentstroke}{rgb}{0.000000,0.000000,0.000000}%
\pgfsetstrokecolor{currentstroke}%
\pgfsetdash{}{0pt}%
\pgfsys@defobject{currentmarker}{\pgfqpoint{0.000000in}{0.000000in}}{\pgfqpoint{0.000000in}{0.055556in}}{%
\pgfpathmoveto{\pgfqpoint{0.000000in}{0.000000in}}%
\pgfpathlineto{\pgfqpoint{0.000000in}{0.055556in}}%
\pgfusepath{stroke,fill}%
}%
\begin{pgfscope}%
\pgfsys@transformshift{1.250000in}{0.400000in}%
\pgfsys@useobject{currentmarker}{}%
\end{pgfscope}%
\end{pgfscope}%
\begin{pgfscope}%
\pgfsetbuttcap%
\pgfsetroundjoin%
\definecolor{currentfill}{rgb}{0.000000,0.000000,0.000000}%
\pgfsetfillcolor{currentfill}%
\pgfsetlinewidth{0.501875pt}%
\definecolor{currentstroke}{rgb}{0.000000,0.000000,0.000000}%
\pgfsetstrokecolor{currentstroke}%
\pgfsetdash{}{0pt}%
\pgfsys@defobject{currentmarker}{\pgfqpoint{0.000000in}{-0.055556in}}{\pgfqpoint{0.000000in}{0.000000in}}{%
\pgfpathmoveto{\pgfqpoint{0.000000in}{0.000000in}}%
\pgfpathlineto{\pgfqpoint{0.000000in}{-0.055556in}}%
\pgfusepath{stroke,fill}%
}%
\begin{pgfscope}%
\pgfsys@transformshift{1.250000in}{3.600000in}%
\pgfsys@useobject{currentmarker}{}%
\end{pgfscope}%
\end{pgfscope}%
\begin{pgfscope}%
\pgftext[left,bottom,x=0.941610in,y=0.218387in,rotate=0.000000]{{\sffamily\fontsize{12.000000}{14.400000}\selectfont −0.004}}
%
\end{pgfscope}%
\begin{pgfscope}%
\pgfpathrectangle{\pgfqpoint{1.250000in}{0.400000in}}{\pgfqpoint{7.750000in}{3.200000in}} %
\pgfusepath{clip}%
\pgfsetbuttcap%
\pgfsetroundjoin%
\pgfsetlinewidth{0.501875pt}%
\definecolor{currentstroke}{rgb}{0.000000,0.000000,0.000000}%
\pgfsetstrokecolor{currentstroke}%
\pgfsetdash{{1.000000pt}{3.000000pt}}{0.000000pt}%
\pgfpathmoveto{\pgfqpoint{2.218750in}{0.400000in}}%
\pgfpathlineto{\pgfqpoint{2.218750in}{3.600000in}}%
\pgfusepath{stroke}%
\end{pgfscope}%
\begin{pgfscope}%
\pgfsetbuttcap%
\pgfsetroundjoin%
\definecolor{currentfill}{rgb}{0.000000,0.000000,0.000000}%
\pgfsetfillcolor{currentfill}%
\pgfsetlinewidth{0.501875pt}%
\definecolor{currentstroke}{rgb}{0.000000,0.000000,0.000000}%
\pgfsetstrokecolor{currentstroke}%
\pgfsetdash{}{0pt}%
\pgfsys@defobject{currentmarker}{\pgfqpoint{0.000000in}{0.000000in}}{\pgfqpoint{0.000000in}{0.055556in}}{%
\pgfpathmoveto{\pgfqpoint{0.000000in}{0.000000in}}%
\pgfpathlineto{\pgfqpoint{0.000000in}{0.055556in}}%
\pgfusepath{stroke,fill}%
}%
\begin{pgfscope}%
\pgfsys@transformshift{2.218750in}{0.400000in}%
\pgfsys@useobject{currentmarker}{}%
\end{pgfscope}%
\end{pgfscope}%
\begin{pgfscope}%
\pgfsetbuttcap%
\pgfsetroundjoin%
\definecolor{currentfill}{rgb}{0.000000,0.000000,0.000000}%
\pgfsetfillcolor{currentfill}%
\pgfsetlinewidth{0.501875pt}%
\definecolor{currentstroke}{rgb}{0.000000,0.000000,0.000000}%
\pgfsetstrokecolor{currentstroke}%
\pgfsetdash{}{0pt}%
\pgfsys@defobject{currentmarker}{\pgfqpoint{0.000000in}{-0.055556in}}{\pgfqpoint{0.000000in}{0.000000in}}{%
\pgfpathmoveto{\pgfqpoint{0.000000in}{0.000000in}}%
\pgfpathlineto{\pgfqpoint{0.000000in}{-0.055556in}}%
\pgfusepath{stroke,fill}%
}%
\begin{pgfscope}%
\pgfsys@transformshift{2.218750in}{3.600000in}%
\pgfsys@useobject{currentmarker}{}%
\end{pgfscope}%
\end{pgfscope}%
\begin{pgfscope}%
\pgftext[left,bottom,x=1.910360in,y=0.218387in,rotate=0.000000]{{\sffamily\fontsize{12.000000}{14.400000}\selectfont −0.003}}
%
\end{pgfscope}%
\begin{pgfscope}%
\pgfpathrectangle{\pgfqpoint{1.250000in}{0.400000in}}{\pgfqpoint{7.750000in}{3.200000in}} %
\pgfusepath{clip}%
\pgfsetbuttcap%
\pgfsetroundjoin%
\pgfsetlinewidth{0.501875pt}%
\definecolor{currentstroke}{rgb}{0.000000,0.000000,0.000000}%
\pgfsetstrokecolor{currentstroke}%
\pgfsetdash{{1.000000pt}{3.000000pt}}{0.000000pt}%
\pgfpathmoveto{\pgfqpoint{3.187500in}{0.400000in}}%
\pgfpathlineto{\pgfqpoint{3.187500in}{3.600000in}}%
\pgfusepath{stroke}%
\end{pgfscope}%
\begin{pgfscope}%
\pgfsetbuttcap%
\pgfsetroundjoin%
\definecolor{currentfill}{rgb}{0.000000,0.000000,0.000000}%
\pgfsetfillcolor{currentfill}%
\pgfsetlinewidth{0.501875pt}%
\definecolor{currentstroke}{rgb}{0.000000,0.000000,0.000000}%
\pgfsetstrokecolor{currentstroke}%
\pgfsetdash{}{0pt}%
\pgfsys@defobject{currentmarker}{\pgfqpoint{0.000000in}{0.000000in}}{\pgfqpoint{0.000000in}{0.055556in}}{%
\pgfpathmoveto{\pgfqpoint{0.000000in}{0.000000in}}%
\pgfpathlineto{\pgfqpoint{0.000000in}{0.055556in}}%
\pgfusepath{stroke,fill}%
}%
\begin{pgfscope}%
\pgfsys@transformshift{3.187500in}{0.400000in}%
\pgfsys@useobject{currentmarker}{}%
\end{pgfscope}%
\end{pgfscope}%
\begin{pgfscope}%
\pgfsetbuttcap%
\pgfsetroundjoin%
\definecolor{currentfill}{rgb}{0.000000,0.000000,0.000000}%
\pgfsetfillcolor{currentfill}%
\pgfsetlinewidth{0.501875pt}%
\definecolor{currentstroke}{rgb}{0.000000,0.000000,0.000000}%
\pgfsetstrokecolor{currentstroke}%
\pgfsetdash{}{0pt}%
\pgfsys@defobject{currentmarker}{\pgfqpoint{0.000000in}{-0.055556in}}{\pgfqpoint{0.000000in}{0.000000in}}{%
\pgfpathmoveto{\pgfqpoint{0.000000in}{0.000000in}}%
\pgfpathlineto{\pgfqpoint{0.000000in}{-0.055556in}}%
\pgfusepath{stroke,fill}%
}%
\begin{pgfscope}%
\pgfsys@transformshift{3.187500in}{3.600000in}%
\pgfsys@useobject{currentmarker}{}%
\end{pgfscope}%
\end{pgfscope}%
\begin{pgfscope}%
\pgftext[left,bottom,x=2.879110in,y=0.218387in,rotate=0.000000]{{\sffamily\fontsize{12.000000}{14.400000}\selectfont −0.002}}
%
\end{pgfscope}%
\begin{pgfscope}%
\pgfpathrectangle{\pgfqpoint{1.250000in}{0.400000in}}{\pgfqpoint{7.750000in}{3.200000in}} %
\pgfusepath{clip}%
\pgfsetbuttcap%
\pgfsetroundjoin%
\pgfsetlinewidth{0.501875pt}%
\definecolor{currentstroke}{rgb}{0.000000,0.000000,0.000000}%
\pgfsetstrokecolor{currentstroke}%
\pgfsetdash{{1.000000pt}{3.000000pt}}{0.000000pt}%
\pgfpathmoveto{\pgfqpoint{4.156250in}{0.400000in}}%
\pgfpathlineto{\pgfqpoint{4.156250in}{3.600000in}}%
\pgfusepath{stroke}%
\end{pgfscope}%
\begin{pgfscope}%
\pgfsetbuttcap%
\pgfsetroundjoin%
\definecolor{currentfill}{rgb}{0.000000,0.000000,0.000000}%
\pgfsetfillcolor{currentfill}%
\pgfsetlinewidth{0.501875pt}%
\definecolor{currentstroke}{rgb}{0.000000,0.000000,0.000000}%
\pgfsetstrokecolor{currentstroke}%
\pgfsetdash{}{0pt}%
\pgfsys@defobject{currentmarker}{\pgfqpoint{0.000000in}{0.000000in}}{\pgfqpoint{0.000000in}{0.055556in}}{%
\pgfpathmoveto{\pgfqpoint{0.000000in}{0.000000in}}%
\pgfpathlineto{\pgfqpoint{0.000000in}{0.055556in}}%
\pgfusepath{stroke,fill}%
}%
\begin{pgfscope}%
\pgfsys@transformshift{4.156250in}{0.400000in}%
\pgfsys@useobject{currentmarker}{}%
\end{pgfscope}%
\end{pgfscope}%
\begin{pgfscope}%
\pgfsetbuttcap%
\pgfsetroundjoin%
\definecolor{currentfill}{rgb}{0.000000,0.000000,0.000000}%
\pgfsetfillcolor{currentfill}%
\pgfsetlinewidth{0.501875pt}%
\definecolor{currentstroke}{rgb}{0.000000,0.000000,0.000000}%
\pgfsetstrokecolor{currentstroke}%
\pgfsetdash{}{0pt}%
\pgfsys@defobject{currentmarker}{\pgfqpoint{0.000000in}{-0.055556in}}{\pgfqpoint{0.000000in}{0.000000in}}{%
\pgfpathmoveto{\pgfqpoint{0.000000in}{0.000000in}}%
\pgfpathlineto{\pgfqpoint{0.000000in}{-0.055556in}}%
\pgfusepath{stroke,fill}%
}%
\begin{pgfscope}%
\pgfsys@transformshift{4.156250in}{3.600000in}%
\pgfsys@useobject{currentmarker}{}%
\end{pgfscope}%
\end{pgfscope}%
\begin{pgfscope}%
\pgftext[left,bottom,x=3.847860in,y=0.218387in,rotate=0.000000]{{\sffamily\fontsize{12.000000}{14.400000}\selectfont −0.001}}
%
\end{pgfscope}%
\begin{pgfscope}%
\pgfpathrectangle{\pgfqpoint{1.250000in}{0.400000in}}{\pgfqpoint{7.750000in}{3.200000in}} %
\pgfusepath{clip}%
\pgfsetbuttcap%
\pgfsetroundjoin%
\pgfsetlinewidth{0.501875pt}%
\definecolor{currentstroke}{rgb}{0.000000,0.000000,0.000000}%
\pgfsetstrokecolor{currentstroke}%
\pgfsetdash{{1.000000pt}{3.000000pt}}{0.000000pt}%
\pgfpathmoveto{\pgfqpoint{5.125000in}{0.400000in}}%
\pgfpathlineto{\pgfqpoint{5.125000in}{3.600000in}}%
\pgfusepath{stroke}%
\end{pgfscope}%
\begin{pgfscope}%
\pgfsetbuttcap%
\pgfsetroundjoin%
\definecolor{currentfill}{rgb}{0.000000,0.000000,0.000000}%
\pgfsetfillcolor{currentfill}%
\pgfsetlinewidth{0.501875pt}%
\definecolor{currentstroke}{rgb}{0.000000,0.000000,0.000000}%
\pgfsetstrokecolor{currentstroke}%
\pgfsetdash{}{0pt}%
\pgfsys@defobject{currentmarker}{\pgfqpoint{0.000000in}{0.000000in}}{\pgfqpoint{0.000000in}{0.055556in}}{%
\pgfpathmoveto{\pgfqpoint{0.000000in}{0.000000in}}%
\pgfpathlineto{\pgfqpoint{0.000000in}{0.055556in}}%
\pgfusepath{stroke,fill}%
}%
\begin{pgfscope}%
\pgfsys@transformshift{5.125000in}{0.400000in}%
\pgfsys@useobject{currentmarker}{}%
\end{pgfscope}%
\end{pgfscope}%
\begin{pgfscope}%
\pgfsetbuttcap%
\pgfsetroundjoin%
\definecolor{currentfill}{rgb}{0.000000,0.000000,0.000000}%
\pgfsetfillcolor{currentfill}%
\pgfsetlinewidth{0.501875pt}%
\definecolor{currentstroke}{rgb}{0.000000,0.000000,0.000000}%
\pgfsetstrokecolor{currentstroke}%
\pgfsetdash{}{0pt}%
\pgfsys@defobject{currentmarker}{\pgfqpoint{0.000000in}{-0.055556in}}{\pgfqpoint{0.000000in}{0.000000in}}{%
\pgfpathmoveto{\pgfqpoint{0.000000in}{0.000000in}}%
\pgfpathlineto{\pgfqpoint{0.000000in}{-0.055556in}}%
\pgfusepath{stroke,fill}%
}%
\begin{pgfscope}%
\pgfsys@transformshift{5.125000in}{3.600000in}%
\pgfsys@useobject{currentmarker}{}%
\end{pgfscope}%
\end{pgfscope}%
\begin{pgfscope}%
\pgftext[left,bottom,x=4.886434in,y=0.218387in,rotate=0.000000]{{\sffamily\fontsize{12.000000}{14.400000}\selectfont 0.000}}
%
\end{pgfscope}%
\begin{pgfscope}%
\pgfpathrectangle{\pgfqpoint{1.250000in}{0.400000in}}{\pgfqpoint{7.750000in}{3.200000in}} %
\pgfusepath{clip}%
\pgfsetbuttcap%
\pgfsetroundjoin%
\pgfsetlinewidth{0.501875pt}%
\definecolor{currentstroke}{rgb}{0.000000,0.000000,0.000000}%
\pgfsetstrokecolor{currentstroke}%
\pgfsetdash{{1.000000pt}{3.000000pt}}{0.000000pt}%
\pgfpathmoveto{\pgfqpoint{6.093750in}{0.400000in}}%
\pgfpathlineto{\pgfqpoint{6.093750in}{3.600000in}}%
\pgfusepath{stroke}%
\end{pgfscope}%
\begin{pgfscope}%
\pgfsetbuttcap%
\pgfsetroundjoin%
\definecolor{currentfill}{rgb}{0.000000,0.000000,0.000000}%
\pgfsetfillcolor{currentfill}%
\pgfsetlinewidth{0.501875pt}%
\definecolor{currentstroke}{rgb}{0.000000,0.000000,0.000000}%
\pgfsetstrokecolor{currentstroke}%
\pgfsetdash{}{0pt}%
\pgfsys@defobject{currentmarker}{\pgfqpoint{0.000000in}{0.000000in}}{\pgfqpoint{0.000000in}{0.055556in}}{%
\pgfpathmoveto{\pgfqpoint{0.000000in}{0.000000in}}%
\pgfpathlineto{\pgfqpoint{0.000000in}{0.055556in}}%
\pgfusepath{stroke,fill}%
}%
\begin{pgfscope}%
\pgfsys@transformshift{6.093750in}{0.400000in}%
\pgfsys@useobject{currentmarker}{}%
\end{pgfscope}%
\end{pgfscope}%
\begin{pgfscope}%
\pgfsetbuttcap%
\pgfsetroundjoin%
\definecolor{currentfill}{rgb}{0.000000,0.000000,0.000000}%
\pgfsetfillcolor{currentfill}%
\pgfsetlinewidth{0.501875pt}%
\definecolor{currentstroke}{rgb}{0.000000,0.000000,0.000000}%
\pgfsetstrokecolor{currentstroke}%
\pgfsetdash{}{0pt}%
\pgfsys@defobject{currentmarker}{\pgfqpoint{0.000000in}{-0.055556in}}{\pgfqpoint{0.000000in}{0.000000in}}{%
\pgfpathmoveto{\pgfqpoint{0.000000in}{0.000000in}}%
\pgfpathlineto{\pgfqpoint{0.000000in}{-0.055556in}}%
\pgfusepath{stroke,fill}%
}%
\begin{pgfscope}%
\pgfsys@transformshift{6.093750in}{3.600000in}%
\pgfsys@useobject{currentmarker}{}%
\end{pgfscope}%
\end{pgfscope}%
\begin{pgfscope}%
\pgftext[left,bottom,x=5.855184in,y=0.218387in,rotate=0.000000]{{\sffamily\fontsize{12.000000}{14.400000}\selectfont 0.001}}
%
\end{pgfscope}%
\begin{pgfscope}%
\pgfpathrectangle{\pgfqpoint{1.250000in}{0.400000in}}{\pgfqpoint{7.750000in}{3.200000in}} %
\pgfusepath{clip}%
\pgfsetbuttcap%
\pgfsetroundjoin%
\pgfsetlinewidth{0.501875pt}%
\definecolor{currentstroke}{rgb}{0.000000,0.000000,0.000000}%
\pgfsetstrokecolor{currentstroke}%
\pgfsetdash{{1.000000pt}{3.000000pt}}{0.000000pt}%
\pgfpathmoveto{\pgfqpoint{7.062500in}{0.400000in}}%
\pgfpathlineto{\pgfqpoint{7.062500in}{3.600000in}}%
\pgfusepath{stroke}%
\end{pgfscope}%
\begin{pgfscope}%
\pgfsetbuttcap%
\pgfsetroundjoin%
\definecolor{currentfill}{rgb}{0.000000,0.000000,0.000000}%
\pgfsetfillcolor{currentfill}%
\pgfsetlinewidth{0.501875pt}%
\definecolor{currentstroke}{rgb}{0.000000,0.000000,0.000000}%
\pgfsetstrokecolor{currentstroke}%
\pgfsetdash{}{0pt}%
\pgfsys@defobject{currentmarker}{\pgfqpoint{0.000000in}{0.000000in}}{\pgfqpoint{0.000000in}{0.055556in}}{%
\pgfpathmoveto{\pgfqpoint{0.000000in}{0.000000in}}%
\pgfpathlineto{\pgfqpoint{0.000000in}{0.055556in}}%
\pgfusepath{stroke,fill}%
}%
\begin{pgfscope}%
\pgfsys@transformshift{7.062500in}{0.400000in}%
\pgfsys@useobject{currentmarker}{}%
\end{pgfscope}%
\end{pgfscope}%
\begin{pgfscope}%
\pgfsetbuttcap%
\pgfsetroundjoin%
\definecolor{currentfill}{rgb}{0.000000,0.000000,0.000000}%
\pgfsetfillcolor{currentfill}%
\pgfsetlinewidth{0.501875pt}%
\definecolor{currentstroke}{rgb}{0.000000,0.000000,0.000000}%
\pgfsetstrokecolor{currentstroke}%
\pgfsetdash{}{0pt}%
\pgfsys@defobject{currentmarker}{\pgfqpoint{0.000000in}{-0.055556in}}{\pgfqpoint{0.000000in}{0.000000in}}{%
\pgfpathmoveto{\pgfqpoint{0.000000in}{0.000000in}}%
\pgfpathlineto{\pgfqpoint{0.000000in}{-0.055556in}}%
\pgfusepath{stroke,fill}%
}%
\begin{pgfscope}%
\pgfsys@transformshift{7.062500in}{3.600000in}%
\pgfsys@useobject{currentmarker}{}%
\end{pgfscope}%
\end{pgfscope}%
\begin{pgfscope}%
\pgftext[left,bottom,x=6.823934in,y=0.218387in,rotate=0.000000]{{\sffamily\fontsize{12.000000}{14.400000}\selectfont 0.002}}
%
\end{pgfscope}%
\begin{pgfscope}%
\pgfpathrectangle{\pgfqpoint{1.250000in}{0.400000in}}{\pgfqpoint{7.750000in}{3.200000in}} %
\pgfusepath{clip}%
\pgfsetbuttcap%
\pgfsetroundjoin%
\pgfsetlinewidth{0.501875pt}%
\definecolor{currentstroke}{rgb}{0.000000,0.000000,0.000000}%
\pgfsetstrokecolor{currentstroke}%
\pgfsetdash{{1.000000pt}{3.000000pt}}{0.000000pt}%
\pgfpathmoveto{\pgfqpoint{8.031250in}{0.400000in}}%
\pgfpathlineto{\pgfqpoint{8.031250in}{3.600000in}}%
\pgfusepath{stroke}%
\end{pgfscope}%
\begin{pgfscope}%
\pgfsetbuttcap%
\pgfsetroundjoin%
\definecolor{currentfill}{rgb}{0.000000,0.000000,0.000000}%
\pgfsetfillcolor{currentfill}%
\pgfsetlinewidth{0.501875pt}%
\definecolor{currentstroke}{rgb}{0.000000,0.000000,0.000000}%
\pgfsetstrokecolor{currentstroke}%
\pgfsetdash{}{0pt}%
\pgfsys@defobject{currentmarker}{\pgfqpoint{0.000000in}{0.000000in}}{\pgfqpoint{0.000000in}{0.055556in}}{%
\pgfpathmoveto{\pgfqpoint{0.000000in}{0.000000in}}%
\pgfpathlineto{\pgfqpoint{0.000000in}{0.055556in}}%
\pgfusepath{stroke,fill}%
}%
\begin{pgfscope}%
\pgfsys@transformshift{8.031250in}{0.400000in}%
\pgfsys@useobject{currentmarker}{}%
\end{pgfscope}%
\end{pgfscope}%
\begin{pgfscope}%
\pgfsetbuttcap%
\pgfsetroundjoin%
\definecolor{currentfill}{rgb}{0.000000,0.000000,0.000000}%
\pgfsetfillcolor{currentfill}%
\pgfsetlinewidth{0.501875pt}%
\definecolor{currentstroke}{rgb}{0.000000,0.000000,0.000000}%
\pgfsetstrokecolor{currentstroke}%
\pgfsetdash{}{0pt}%
\pgfsys@defobject{currentmarker}{\pgfqpoint{0.000000in}{-0.055556in}}{\pgfqpoint{0.000000in}{0.000000in}}{%
\pgfpathmoveto{\pgfqpoint{0.000000in}{0.000000in}}%
\pgfpathlineto{\pgfqpoint{0.000000in}{-0.055556in}}%
\pgfusepath{stroke,fill}%
}%
\begin{pgfscope}%
\pgfsys@transformshift{8.031250in}{3.600000in}%
\pgfsys@useobject{currentmarker}{}%
\end{pgfscope}%
\end{pgfscope}%
\begin{pgfscope}%
\pgftext[left,bottom,x=7.792684in,y=0.218387in,rotate=0.000000]{{\sffamily\fontsize{12.000000}{14.400000}\selectfont 0.003}}
%
\end{pgfscope}%
\begin{pgfscope}%
\pgfpathrectangle{\pgfqpoint{1.250000in}{0.400000in}}{\pgfqpoint{7.750000in}{3.200000in}} %
\pgfusepath{clip}%
\pgfsetbuttcap%
\pgfsetroundjoin%
\pgfsetlinewidth{0.501875pt}%
\definecolor{currentstroke}{rgb}{0.000000,0.000000,0.000000}%
\pgfsetstrokecolor{currentstroke}%
\pgfsetdash{{1.000000pt}{3.000000pt}}{0.000000pt}%
\pgfpathmoveto{\pgfqpoint{9.000000in}{0.400000in}}%
\pgfpathlineto{\pgfqpoint{9.000000in}{3.600000in}}%
\pgfusepath{stroke}%
\end{pgfscope}%
\begin{pgfscope}%
\pgfsetbuttcap%
\pgfsetroundjoin%
\definecolor{currentfill}{rgb}{0.000000,0.000000,0.000000}%
\pgfsetfillcolor{currentfill}%
\pgfsetlinewidth{0.501875pt}%
\definecolor{currentstroke}{rgb}{0.000000,0.000000,0.000000}%
\pgfsetstrokecolor{currentstroke}%
\pgfsetdash{}{0pt}%
\pgfsys@defobject{currentmarker}{\pgfqpoint{0.000000in}{0.000000in}}{\pgfqpoint{0.000000in}{0.055556in}}{%
\pgfpathmoveto{\pgfqpoint{0.000000in}{0.000000in}}%
\pgfpathlineto{\pgfqpoint{0.000000in}{0.055556in}}%
\pgfusepath{stroke,fill}%
}%
\begin{pgfscope}%
\pgfsys@transformshift{9.000000in}{0.400000in}%
\pgfsys@useobject{currentmarker}{}%
\end{pgfscope}%
\end{pgfscope}%
\begin{pgfscope}%
\pgfsetbuttcap%
\pgfsetroundjoin%
\definecolor{currentfill}{rgb}{0.000000,0.000000,0.000000}%
\pgfsetfillcolor{currentfill}%
\pgfsetlinewidth{0.501875pt}%
\definecolor{currentstroke}{rgb}{0.000000,0.000000,0.000000}%
\pgfsetstrokecolor{currentstroke}%
\pgfsetdash{}{0pt}%
\pgfsys@defobject{currentmarker}{\pgfqpoint{0.000000in}{-0.055556in}}{\pgfqpoint{0.000000in}{0.000000in}}{%
\pgfpathmoveto{\pgfqpoint{0.000000in}{0.000000in}}%
\pgfpathlineto{\pgfqpoint{0.000000in}{-0.055556in}}%
\pgfusepath{stroke,fill}%
}%
\begin{pgfscope}%
\pgfsys@transformshift{9.000000in}{3.600000in}%
\pgfsys@useobject{currentmarker}{}%
\end{pgfscope}%
\end{pgfscope}%
\begin{pgfscope}%
\pgftext[left,bottom,x=8.761434in,y=0.218387in,rotate=0.000000]{{\sffamily\fontsize{12.000000}{14.400000}\selectfont 0.004}}
%
\end{pgfscope}%
\begin{pgfscope}%
\pgfpathrectangle{\pgfqpoint{1.250000in}{0.400000in}}{\pgfqpoint{7.750000in}{3.200000in}} %
\pgfusepath{clip}%
\pgfsetbuttcap%
\pgfsetroundjoin%
\pgfsetlinewidth{0.501875pt}%
\definecolor{currentstroke}{rgb}{0.000000,0.000000,0.000000}%
\pgfsetstrokecolor{currentstroke}%
\pgfsetdash{{1.000000pt}{3.000000pt}}{0.000000pt}%
\pgfpathmoveto{\pgfqpoint{1.250000in}{0.400000in}}%
\pgfpathlineto{\pgfqpoint{9.000000in}{0.400000in}}%
\pgfusepath{stroke}%
\end{pgfscope}%
\begin{pgfscope}%
\pgfsetbuttcap%
\pgfsetroundjoin%
\definecolor{currentfill}{rgb}{0.000000,0.000000,0.000000}%
\pgfsetfillcolor{currentfill}%
\pgfsetlinewidth{0.501875pt}%
\definecolor{currentstroke}{rgb}{0.000000,0.000000,0.000000}%
\pgfsetstrokecolor{currentstroke}%
\pgfsetdash{}{0pt}%
\pgfsys@defobject{currentmarker}{\pgfqpoint{0.000000in}{0.000000in}}{\pgfqpoint{0.055556in}{0.000000in}}{%
\pgfpathmoveto{\pgfqpoint{0.000000in}{0.000000in}}%
\pgfpathlineto{\pgfqpoint{0.055556in}{0.000000in}}%
\pgfusepath{stroke,fill}%
}%
\begin{pgfscope}%
\pgfsys@transformshift{1.250000in}{0.400000in}%
\pgfsys@useobject{currentmarker}{}%
\end{pgfscope}%
\end{pgfscope}%
\begin{pgfscope}%
\pgfsetbuttcap%
\pgfsetroundjoin%
\definecolor{currentfill}{rgb}{0.000000,0.000000,0.000000}%
\pgfsetfillcolor{currentfill}%
\pgfsetlinewidth{0.501875pt}%
\definecolor{currentstroke}{rgb}{0.000000,0.000000,0.000000}%
\pgfsetstrokecolor{currentstroke}%
\pgfsetdash{}{0pt}%
\pgfsys@defobject{currentmarker}{\pgfqpoint{-0.055556in}{0.000000in}}{\pgfqpoint{0.000000in}{0.000000in}}{%
\pgfpathmoveto{\pgfqpoint{0.000000in}{0.000000in}}%
\pgfpathlineto{\pgfqpoint{-0.055556in}{0.000000in}}%
\pgfusepath{stroke,fill}%
}%
\begin{pgfscope}%
\pgfsys@transformshift{9.000000in}{0.400000in}%
\pgfsys@useobject{currentmarker}{}%
\end{pgfscope}%
\end{pgfscope}%
\begin{pgfscope}%
\pgftext[left,bottom,x=0.577664in,y=0.336971in,rotate=0.000000]{{\sffamily\fontsize{12.000000}{14.400000}\selectfont −0.004}}
%
\end{pgfscope}%
\begin{pgfscope}%
\pgfpathrectangle{\pgfqpoint{1.250000in}{0.400000in}}{\pgfqpoint{7.750000in}{3.200000in}} %
\pgfusepath{clip}%
\pgfsetbuttcap%
\pgfsetroundjoin%
\pgfsetlinewidth{0.501875pt}%
\definecolor{currentstroke}{rgb}{0.000000,0.000000,0.000000}%
\pgfsetstrokecolor{currentstroke}%
\pgfsetdash{{1.000000pt}{3.000000pt}}{0.000000pt}%
\pgfpathmoveto{\pgfqpoint{1.250000in}{0.800000in}}%
\pgfpathlineto{\pgfqpoint{9.000000in}{0.800000in}}%
\pgfusepath{stroke}%
\end{pgfscope}%
\begin{pgfscope}%
\pgfsetbuttcap%
\pgfsetroundjoin%
\definecolor{currentfill}{rgb}{0.000000,0.000000,0.000000}%
\pgfsetfillcolor{currentfill}%
\pgfsetlinewidth{0.501875pt}%
\definecolor{currentstroke}{rgb}{0.000000,0.000000,0.000000}%
\pgfsetstrokecolor{currentstroke}%
\pgfsetdash{}{0pt}%
\pgfsys@defobject{currentmarker}{\pgfqpoint{0.000000in}{0.000000in}}{\pgfqpoint{0.055556in}{0.000000in}}{%
\pgfpathmoveto{\pgfqpoint{0.000000in}{0.000000in}}%
\pgfpathlineto{\pgfqpoint{0.055556in}{0.000000in}}%
\pgfusepath{stroke,fill}%
}%
\begin{pgfscope}%
\pgfsys@transformshift{1.250000in}{0.800000in}%
\pgfsys@useobject{currentmarker}{}%
\end{pgfscope}%
\end{pgfscope}%
\begin{pgfscope}%
\pgfsetbuttcap%
\pgfsetroundjoin%
\definecolor{currentfill}{rgb}{0.000000,0.000000,0.000000}%
\pgfsetfillcolor{currentfill}%
\pgfsetlinewidth{0.501875pt}%
\definecolor{currentstroke}{rgb}{0.000000,0.000000,0.000000}%
\pgfsetstrokecolor{currentstroke}%
\pgfsetdash{}{0pt}%
\pgfsys@defobject{currentmarker}{\pgfqpoint{-0.055556in}{0.000000in}}{\pgfqpoint{0.000000in}{0.000000in}}{%
\pgfpathmoveto{\pgfqpoint{0.000000in}{0.000000in}}%
\pgfpathlineto{\pgfqpoint{-0.055556in}{0.000000in}}%
\pgfusepath{stroke,fill}%
}%
\begin{pgfscope}%
\pgfsys@transformshift{9.000000in}{0.800000in}%
\pgfsys@useobject{currentmarker}{}%
\end{pgfscope}%
\end{pgfscope}%
\begin{pgfscope}%
\pgftext[left,bottom,x=0.577664in,y=0.736971in,rotate=0.000000]{{\sffamily\fontsize{12.000000}{14.400000}\selectfont −0.003}}
%
\end{pgfscope}%
\begin{pgfscope}%
\pgfpathrectangle{\pgfqpoint{1.250000in}{0.400000in}}{\pgfqpoint{7.750000in}{3.200000in}} %
\pgfusepath{clip}%
\pgfsetbuttcap%
\pgfsetroundjoin%
\pgfsetlinewidth{0.501875pt}%
\definecolor{currentstroke}{rgb}{0.000000,0.000000,0.000000}%
\pgfsetstrokecolor{currentstroke}%
\pgfsetdash{{1.000000pt}{3.000000pt}}{0.000000pt}%
\pgfpathmoveto{\pgfqpoint{1.250000in}{1.200000in}}%
\pgfpathlineto{\pgfqpoint{9.000000in}{1.200000in}}%
\pgfusepath{stroke}%
\end{pgfscope}%
\begin{pgfscope}%
\pgfsetbuttcap%
\pgfsetroundjoin%
\definecolor{currentfill}{rgb}{0.000000,0.000000,0.000000}%
\pgfsetfillcolor{currentfill}%
\pgfsetlinewidth{0.501875pt}%
\definecolor{currentstroke}{rgb}{0.000000,0.000000,0.000000}%
\pgfsetstrokecolor{currentstroke}%
\pgfsetdash{}{0pt}%
\pgfsys@defobject{currentmarker}{\pgfqpoint{0.000000in}{0.000000in}}{\pgfqpoint{0.055556in}{0.000000in}}{%
\pgfpathmoveto{\pgfqpoint{0.000000in}{0.000000in}}%
\pgfpathlineto{\pgfqpoint{0.055556in}{0.000000in}}%
\pgfusepath{stroke,fill}%
}%
\begin{pgfscope}%
\pgfsys@transformshift{1.250000in}{1.200000in}%
\pgfsys@useobject{currentmarker}{}%
\end{pgfscope}%
\end{pgfscope}%
\begin{pgfscope}%
\pgfsetbuttcap%
\pgfsetroundjoin%
\definecolor{currentfill}{rgb}{0.000000,0.000000,0.000000}%
\pgfsetfillcolor{currentfill}%
\pgfsetlinewidth{0.501875pt}%
\definecolor{currentstroke}{rgb}{0.000000,0.000000,0.000000}%
\pgfsetstrokecolor{currentstroke}%
\pgfsetdash{}{0pt}%
\pgfsys@defobject{currentmarker}{\pgfqpoint{-0.055556in}{0.000000in}}{\pgfqpoint{0.000000in}{0.000000in}}{%
\pgfpathmoveto{\pgfqpoint{0.000000in}{0.000000in}}%
\pgfpathlineto{\pgfqpoint{-0.055556in}{0.000000in}}%
\pgfusepath{stroke,fill}%
}%
\begin{pgfscope}%
\pgfsys@transformshift{9.000000in}{1.200000in}%
\pgfsys@useobject{currentmarker}{}%
\end{pgfscope}%
\end{pgfscope}%
\begin{pgfscope}%
\pgftext[left,bottom,x=0.577664in,y=1.136971in,rotate=0.000000]{{\sffamily\fontsize{12.000000}{14.400000}\selectfont −0.002}}
%
\end{pgfscope}%
\begin{pgfscope}%
\pgfpathrectangle{\pgfqpoint{1.250000in}{0.400000in}}{\pgfqpoint{7.750000in}{3.200000in}} %
\pgfusepath{clip}%
\pgfsetbuttcap%
\pgfsetroundjoin%
\pgfsetlinewidth{0.501875pt}%
\definecolor{currentstroke}{rgb}{0.000000,0.000000,0.000000}%
\pgfsetstrokecolor{currentstroke}%
\pgfsetdash{{1.000000pt}{3.000000pt}}{0.000000pt}%
\pgfpathmoveto{\pgfqpoint{1.250000in}{1.600000in}}%
\pgfpathlineto{\pgfqpoint{9.000000in}{1.600000in}}%
\pgfusepath{stroke}%
\end{pgfscope}%
\begin{pgfscope}%
\pgfsetbuttcap%
\pgfsetroundjoin%
\definecolor{currentfill}{rgb}{0.000000,0.000000,0.000000}%
\pgfsetfillcolor{currentfill}%
\pgfsetlinewidth{0.501875pt}%
\definecolor{currentstroke}{rgb}{0.000000,0.000000,0.000000}%
\pgfsetstrokecolor{currentstroke}%
\pgfsetdash{}{0pt}%
\pgfsys@defobject{currentmarker}{\pgfqpoint{0.000000in}{0.000000in}}{\pgfqpoint{0.055556in}{0.000000in}}{%
\pgfpathmoveto{\pgfqpoint{0.000000in}{0.000000in}}%
\pgfpathlineto{\pgfqpoint{0.055556in}{0.000000in}}%
\pgfusepath{stroke,fill}%
}%
\begin{pgfscope}%
\pgfsys@transformshift{1.250000in}{1.600000in}%
\pgfsys@useobject{currentmarker}{}%
\end{pgfscope}%
\end{pgfscope}%
\begin{pgfscope}%
\pgfsetbuttcap%
\pgfsetroundjoin%
\definecolor{currentfill}{rgb}{0.000000,0.000000,0.000000}%
\pgfsetfillcolor{currentfill}%
\pgfsetlinewidth{0.501875pt}%
\definecolor{currentstroke}{rgb}{0.000000,0.000000,0.000000}%
\pgfsetstrokecolor{currentstroke}%
\pgfsetdash{}{0pt}%
\pgfsys@defobject{currentmarker}{\pgfqpoint{-0.055556in}{0.000000in}}{\pgfqpoint{0.000000in}{0.000000in}}{%
\pgfpathmoveto{\pgfqpoint{0.000000in}{0.000000in}}%
\pgfpathlineto{\pgfqpoint{-0.055556in}{0.000000in}}%
\pgfusepath{stroke,fill}%
}%
\begin{pgfscope}%
\pgfsys@transformshift{9.000000in}{1.600000in}%
\pgfsys@useobject{currentmarker}{}%
\end{pgfscope}%
\end{pgfscope}%
\begin{pgfscope}%
\pgftext[left,bottom,x=0.577664in,y=1.536971in,rotate=0.000000]{{\sffamily\fontsize{12.000000}{14.400000}\selectfont −0.001}}
%
\end{pgfscope}%
\begin{pgfscope}%
\pgfpathrectangle{\pgfqpoint{1.250000in}{0.400000in}}{\pgfqpoint{7.750000in}{3.200000in}} %
\pgfusepath{clip}%
\pgfsetbuttcap%
\pgfsetroundjoin%
\pgfsetlinewidth{0.501875pt}%
\definecolor{currentstroke}{rgb}{0.000000,0.000000,0.000000}%
\pgfsetstrokecolor{currentstroke}%
\pgfsetdash{{1.000000pt}{3.000000pt}}{0.000000pt}%
\pgfpathmoveto{\pgfqpoint{1.250000in}{2.000000in}}%
\pgfpathlineto{\pgfqpoint{9.000000in}{2.000000in}}%
\pgfusepath{stroke}%
\end{pgfscope}%
\begin{pgfscope}%
\pgfsetbuttcap%
\pgfsetroundjoin%
\definecolor{currentfill}{rgb}{0.000000,0.000000,0.000000}%
\pgfsetfillcolor{currentfill}%
\pgfsetlinewidth{0.501875pt}%
\definecolor{currentstroke}{rgb}{0.000000,0.000000,0.000000}%
\pgfsetstrokecolor{currentstroke}%
\pgfsetdash{}{0pt}%
\pgfsys@defobject{currentmarker}{\pgfqpoint{0.000000in}{0.000000in}}{\pgfqpoint{0.055556in}{0.000000in}}{%
\pgfpathmoveto{\pgfqpoint{0.000000in}{0.000000in}}%
\pgfpathlineto{\pgfqpoint{0.055556in}{0.000000in}}%
\pgfusepath{stroke,fill}%
}%
\begin{pgfscope}%
\pgfsys@transformshift{1.250000in}{2.000000in}%
\pgfsys@useobject{currentmarker}{}%
\end{pgfscope}%
\end{pgfscope}%
\begin{pgfscope}%
\pgfsetbuttcap%
\pgfsetroundjoin%
\definecolor{currentfill}{rgb}{0.000000,0.000000,0.000000}%
\pgfsetfillcolor{currentfill}%
\pgfsetlinewidth{0.501875pt}%
\definecolor{currentstroke}{rgb}{0.000000,0.000000,0.000000}%
\pgfsetstrokecolor{currentstroke}%
\pgfsetdash{}{0pt}%
\pgfsys@defobject{currentmarker}{\pgfqpoint{-0.055556in}{0.000000in}}{\pgfqpoint{0.000000in}{0.000000in}}{%
\pgfpathmoveto{\pgfqpoint{0.000000in}{0.000000in}}%
\pgfpathlineto{\pgfqpoint{-0.055556in}{0.000000in}}%
\pgfusepath{stroke,fill}%
}%
\begin{pgfscope}%
\pgfsys@transformshift{9.000000in}{2.000000in}%
\pgfsys@useobject{currentmarker}{}%
\end{pgfscope}%
\end{pgfscope}%
\begin{pgfscope}%
\pgftext[left,bottom,x=0.717312in,y=1.936971in,rotate=0.000000]{{\sffamily\fontsize{12.000000}{14.400000}\selectfont 0.000}}
%
\end{pgfscope}%
\begin{pgfscope}%
\pgfpathrectangle{\pgfqpoint{1.250000in}{0.400000in}}{\pgfqpoint{7.750000in}{3.200000in}} %
\pgfusepath{clip}%
\pgfsetbuttcap%
\pgfsetroundjoin%
\pgfsetlinewidth{0.501875pt}%
\definecolor{currentstroke}{rgb}{0.000000,0.000000,0.000000}%
\pgfsetstrokecolor{currentstroke}%
\pgfsetdash{{1.000000pt}{3.000000pt}}{0.000000pt}%
\pgfpathmoveto{\pgfqpoint{1.250000in}{2.400000in}}%
\pgfpathlineto{\pgfqpoint{9.000000in}{2.400000in}}%
\pgfusepath{stroke}%
\end{pgfscope}%
\begin{pgfscope}%
\pgfsetbuttcap%
\pgfsetroundjoin%
\definecolor{currentfill}{rgb}{0.000000,0.000000,0.000000}%
\pgfsetfillcolor{currentfill}%
\pgfsetlinewidth{0.501875pt}%
\definecolor{currentstroke}{rgb}{0.000000,0.000000,0.000000}%
\pgfsetstrokecolor{currentstroke}%
\pgfsetdash{}{0pt}%
\pgfsys@defobject{currentmarker}{\pgfqpoint{0.000000in}{0.000000in}}{\pgfqpoint{0.055556in}{0.000000in}}{%
\pgfpathmoveto{\pgfqpoint{0.000000in}{0.000000in}}%
\pgfpathlineto{\pgfqpoint{0.055556in}{0.000000in}}%
\pgfusepath{stroke,fill}%
}%
\begin{pgfscope}%
\pgfsys@transformshift{1.250000in}{2.400000in}%
\pgfsys@useobject{currentmarker}{}%
\end{pgfscope}%
\end{pgfscope}%
\begin{pgfscope}%
\pgfsetbuttcap%
\pgfsetroundjoin%
\definecolor{currentfill}{rgb}{0.000000,0.000000,0.000000}%
\pgfsetfillcolor{currentfill}%
\pgfsetlinewidth{0.501875pt}%
\definecolor{currentstroke}{rgb}{0.000000,0.000000,0.000000}%
\pgfsetstrokecolor{currentstroke}%
\pgfsetdash{}{0pt}%
\pgfsys@defobject{currentmarker}{\pgfqpoint{-0.055556in}{0.000000in}}{\pgfqpoint{0.000000in}{0.000000in}}{%
\pgfpathmoveto{\pgfqpoint{0.000000in}{0.000000in}}%
\pgfpathlineto{\pgfqpoint{-0.055556in}{0.000000in}}%
\pgfusepath{stroke,fill}%
}%
\begin{pgfscope}%
\pgfsys@transformshift{9.000000in}{2.400000in}%
\pgfsys@useobject{currentmarker}{}%
\end{pgfscope}%
\end{pgfscope}%
\begin{pgfscope}%
\pgftext[left,bottom,x=0.717312in,y=2.336971in,rotate=0.000000]{{\sffamily\fontsize{12.000000}{14.400000}\selectfont 0.001}}
%
\end{pgfscope}%
\begin{pgfscope}%
\pgfpathrectangle{\pgfqpoint{1.250000in}{0.400000in}}{\pgfqpoint{7.750000in}{3.200000in}} %
\pgfusepath{clip}%
\pgfsetbuttcap%
\pgfsetroundjoin%
\pgfsetlinewidth{0.501875pt}%
\definecolor{currentstroke}{rgb}{0.000000,0.000000,0.000000}%
\pgfsetstrokecolor{currentstroke}%
\pgfsetdash{{1.000000pt}{3.000000pt}}{0.000000pt}%
\pgfpathmoveto{\pgfqpoint{1.250000in}{2.800000in}}%
\pgfpathlineto{\pgfqpoint{9.000000in}{2.800000in}}%
\pgfusepath{stroke}%
\end{pgfscope}%
\begin{pgfscope}%
\pgfsetbuttcap%
\pgfsetroundjoin%
\definecolor{currentfill}{rgb}{0.000000,0.000000,0.000000}%
\pgfsetfillcolor{currentfill}%
\pgfsetlinewidth{0.501875pt}%
\definecolor{currentstroke}{rgb}{0.000000,0.000000,0.000000}%
\pgfsetstrokecolor{currentstroke}%
\pgfsetdash{}{0pt}%
\pgfsys@defobject{currentmarker}{\pgfqpoint{0.000000in}{0.000000in}}{\pgfqpoint{0.055556in}{0.000000in}}{%
\pgfpathmoveto{\pgfqpoint{0.000000in}{0.000000in}}%
\pgfpathlineto{\pgfqpoint{0.055556in}{0.000000in}}%
\pgfusepath{stroke,fill}%
}%
\begin{pgfscope}%
\pgfsys@transformshift{1.250000in}{2.800000in}%
\pgfsys@useobject{currentmarker}{}%
\end{pgfscope}%
\end{pgfscope}%
\begin{pgfscope}%
\pgfsetbuttcap%
\pgfsetroundjoin%
\definecolor{currentfill}{rgb}{0.000000,0.000000,0.000000}%
\pgfsetfillcolor{currentfill}%
\pgfsetlinewidth{0.501875pt}%
\definecolor{currentstroke}{rgb}{0.000000,0.000000,0.000000}%
\pgfsetstrokecolor{currentstroke}%
\pgfsetdash{}{0pt}%
\pgfsys@defobject{currentmarker}{\pgfqpoint{-0.055556in}{0.000000in}}{\pgfqpoint{0.000000in}{0.000000in}}{%
\pgfpathmoveto{\pgfqpoint{0.000000in}{0.000000in}}%
\pgfpathlineto{\pgfqpoint{-0.055556in}{0.000000in}}%
\pgfusepath{stroke,fill}%
}%
\begin{pgfscope}%
\pgfsys@transformshift{9.000000in}{2.800000in}%
\pgfsys@useobject{currentmarker}{}%
\end{pgfscope}%
\end{pgfscope}%
\begin{pgfscope}%
\pgftext[left,bottom,x=0.717312in,y=2.736971in,rotate=0.000000]{{\sffamily\fontsize{12.000000}{14.400000}\selectfont 0.002}}
%
\end{pgfscope}%
\begin{pgfscope}%
\pgfpathrectangle{\pgfqpoint{1.250000in}{0.400000in}}{\pgfqpoint{7.750000in}{3.200000in}} %
\pgfusepath{clip}%
\pgfsetbuttcap%
\pgfsetroundjoin%
\pgfsetlinewidth{0.501875pt}%
\definecolor{currentstroke}{rgb}{0.000000,0.000000,0.000000}%
\pgfsetstrokecolor{currentstroke}%
\pgfsetdash{{1.000000pt}{3.000000pt}}{0.000000pt}%
\pgfpathmoveto{\pgfqpoint{1.250000in}{3.200000in}}%
\pgfpathlineto{\pgfqpoint{9.000000in}{3.200000in}}%
\pgfusepath{stroke}%
\end{pgfscope}%
\begin{pgfscope}%
\pgfsetbuttcap%
\pgfsetroundjoin%
\definecolor{currentfill}{rgb}{0.000000,0.000000,0.000000}%
\pgfsetfillcolor{currentfill}%
\pgfsetlinewidth{0.501875pt}%
\definecolor{currentstroke}{rgb}{0.000000,0.000000,0.000000}%
\pgfsetstrokecolor{currentstroke}%
\pgfsetdash{}{0pt}%
\pgfsys@defobject{currentmarker}{\pgfqpoint{0.000000in}{0.000000in}}{\pgfqpoint{0.055556in}{0.000000in}}{%
\pgfpathmoveto{\pgfqpoint{0.000000in}{0.000000in}}%
\pgfpathlineto{\pgfqpoint{0.055556in}{0.000000in}}%
\pgfusepath{stroke,fill}%
}%
\begin{pgfscope}%
\pgfsys@transformshift{1.250000in}{3.200000in}%
\pgfsys@useobject{currentmarker}{}%
\end{pgfscope}%
\end{pgfscope}%
\begin{pgfscope}%
\pgfsetbuttcap%
\pgfsetroundjoin%
\definecolor{currentfill}{rgb}{0.000000,0.000000,0.000000}%
\pgfsetfillcolor{currentfill}%
\pgfsetlinewidth{0.501875pt}%
\definecolor{currentstroke}{rgb}{0.000000,0.000000,0.000000}%
\pgfsetstrokecolor{currentstroke}%
\pgfsetdash{}{0pt}%
\pgfsys@defobject{currentmarker}{\pgfqpoint{-0.055556in}{0.000000in}}{\pgfqpoint{0.000000in}{0.000000in}}{%
\pgfpathmoveto{\pgfqpoint{0.000000in}{0.000000in}}%
\pgfpathlineto{\pgfqpoint{-0.055556in}{0.000000in}}%
\pgfusepath{stroke,fill}%
}%
\begin{pgfscope}%
\pgfsys@transformshift{9.000000in}{3.200000in}%
\pgfsys@useobject{currentmarker}{}%
\end{pgfscope}%
\end{pgfscope}%
\begin{pgfscope}%
\pgftext[left,bottom,x=0.717312in,y=3.136971in,rotate=0.000000]{{\sffamily\fontsize{12.000000}{14.400000}\selectfont 0.003}}
%
\end{pgfscope}%
\begin{pgfscope}%
\pgfpathrectangle{\pgfqpoint{1.250000in}{0.400000in}}{\pgfqpoint{7.750000in}{3.200000in}} %
\pgfusepath{clip}%
\pgfsetbuttcap%
\pgfsetroundjoin%
\pgfsetlinewidth{0.501875pt}%
\definecolor{currentstroke}{rgb}{0.000000,0.000000,0.000000}%
\pgfsetstrokecolor{currentstroke}%
\pgfsetdash{{1.000000pt}{3.000000pt}}{0.000000pt}%
\pgfpathmoveto{\pgfqpoint{1.250000in}{3.600000in}}%
\pgfpathlineto{\pgfqpoint{9.000000in}{3.600000in}}%
\pgfusepath{stroke}%
\end{pgfscope}%
\begin{pgfscope}%
\pgfsetbuttcap%
\pgfsetroundjoin%
\definecolor{currentfill}{rgb}{0.000000,0.000000,0.000000}%
\pgfsetfillcolor{currentfill}%
\pgfsetlinewidth{0.501875pt}%
\definecolor{currentstroke}{rgb}{0.000000,0.000000,0.000000}%
\pgfsetstrokecolor{currentstroke}%
\pgfsetdash{}{0pt}%
\pgfsys@defobject{currentmarker}{\pgfqpoint{0.000000in}{0.000000in}}{\pgfqpoint{0.055556in}{0.000000in}}{%
\pgfpathmoveto{\pgfqpoint{0.000000in}{0.000000in}}%
\pgfpathlineto{\pgfqpoint{0.055556in}{0.000000in}}%
\pgfusepath{stroke,fill}%
}%
\begin{pgfscope}%
\pgfsys@transformshift{1.250000in}{3.600000in}%
\pgfsys@useobject{currentmarker}{}%
\end{pgfscope}%
\end{pgfscope}%
\begin{pgfscope}%
\pgfsetbuttcap%
\pgfsetroundjoin%
\definecolor{currentfill}{rgb}{0.000000,0.000000,0.000000}%
\pgfsetfillcolor{currentfill}%
\pgfsetlinewidth{0.501875pt}%
\definecolor{currentstroke}{rgb}{0.000000,0.000000,0.000000}%
\pgfsetstrokecolor{currentstroke}%
\pgfsetdash{}{0pt}%
\pgfsys@defobject{currentmarker}{\pgfqpoint{-0.055556in}{0.000000in}}{\pgfqpoint{0.000000in}{0.000000in}}{%
\pgfpathmoveto{\pgfqpoint{0.000000in}{0.000000in}}%
\pgfpathlineto{\pgfqpoint{-0.055556in}{0.000000in}}%
\pgfusepath{stroke,fill}%
}%
\begin{pgfscope}%
\pgfsys@transformshift{9.000000in}{3.600000in}%
\pgfsys@useobject{currentmarker}{}%
\end{pgfscope}%
\end{pgfscope}%
\begin{pgfscope}%
\pgftext[left,bottom,x=0.717312in,y=3.536971in,rotate=0.000000]{{\sffamily\fontsize{12.000000}{14.400000}\selectfont 0.004}}
%
\end{pgfscope}%
\begin{pgfscope}%
\pgfsetrectcap%
\pgfsetroundjoin%
\pgfsetlinewidth{1.003750pt}%
\definecolor{currentstroke}{rgb}{0.000000,0.000000,0.000000}%
\pgfsetstrokecolor{currentstroke}%
\pgfsetdash{}{0pt}%
\pgfpathmoveto{\pgfqpoint{1.250000in}{3.600000in}}%
\pgfpathlineto{\pgfqpoint{9.000000in}{3.600000in}}%
\pgfusepath{stroke}%
\end{pgfscope}%
\begin{pgfscope}%
\pgfsetrectcap%
\pgfsetroundjoin%
\pgfsetlinewidth{1.003750pt}%
\definecolor{currentstroke}{rgb}{0.000000,0.000000,0.000000}%
\pgfsetstrokecolor{currentstroke}%
\pgfsetdash{}{0pt}%
\pgfpathmoveto{\pgfqpoint{9.000000in}{0.400000in}}%
\pgfpathlineto{\pgfqpoint{9.000000in}{3.600000in}}%
\pgfusepath{stroke}%
\end{pgfscope}%
\begin{pgfscope}%
\pgfsetrectcap%
\pgfsetroundjoin%
\pgfsetlinewidth{1.003750pt}%
\definecolor{currentstroke}{rgb}{0.000000,0.000000,0.000000}%
\pgfsetstrokecolor{currentstroke}%
\pgfsetdash{}{0pt}%
\pgfpathmoveto{\pgfqpoint{1.250000in}{0.400000in}}%
\pgfpathlineto{\pgfqpoint{9.000000in}{0.400000in}}%
\pgfusepath{stroke}%
\end{pgfscope}%
\begin{pgfscope}%
\pgfsetrectcap%
\pgfsetroundjoin%
\pgfsetlinewidth{1.003750pt}%
\definecolor{currentstroke}{rgb}{0.000000,0.000000,0.000000}%
\pgfsetstrokecolor{currentstroke}%
\pgfsetdash{}{0pt}%
\pgfpathmoveto{\pgfqpoint{1.250000in}{0.400000in}}%
\pgfpathlineto{\pgfqpoint{1.250000in}{3.600000in}}%
\pgfusepath{stroke}%
\end{pgfscope}%
\begin{pgfscope}%
\pgftext[left,bottom,x=3.065723in,y=3.646202in,rotate=0.000000]{{\sffamily\fontsize{14.400000}{17.280000}\selectfont moon in rest frame of earth, different dts}}
%
\end{pgfscope}%
\begin{pgfscope}%
\pgfsetrectcap%
\pgfsetroundjoin%
\definecolor{currentfill}{rgb}{1.000000,1.000000,1.000000}%
\pgfsetfillcolor{currentfill}%
\pgfsetlinewidth{1.003750pt}%
\definecolor{currentstroke}{rgb}{0.000000,0.000000,0.000000}%
\pgfsetstrokecolor{currentstroke}%
\pgfsetdash{}{0pt}%
\pgfpathmoveto{\pgfqpoint{4.416096in}{1.686968in}}%
\pgfpathlineto{\pgfqpoint{5.833904in}{1.686968in}}%
\pgfpathlineto{\pgfqpoint{5.833904in}{2.313032in}}%
\pgfpathlineto{\pgfqpoint{4.416096in}{2.313032in}}%
\pgfpathlineto{\pgfqpoint{4.416096in}{1.686968in}}%
\pgfpathclose%
\pgfusepath{stroke,fill}%
\end{pgfscope}%
\begin{pgfscope}%
\pgfsetrectcap%
\pgfsetroundjoin%
\pgfsetlinewidth{1.003750pt}%
\definecolor{currentstroke}{rgb}{0.000000,0.000000,1.000000}%
\pgfsetstrokecolor{currentstroke}%
\pgfsetdash{}{0pt}%
\pgfpathmoveto{\pgfqpoint{4.513279in}{2.200610in}}%
\pgfpathlineto{\pgfqpoint{4.707646in}{2.200610in}}%
\pgfusepath{stroke}%
\end{pgfscope}%
\begin{pgfscope}%
\pgftext[left,bottom,x=4.860362in,y=2.150053in,rotate=0.000000]{{\sffamily\fontsize{9.996000}{11.995200}\selectfont dt = 0.0001a}}
%
\end{pgfscope}%
\begin{pgfscope}%
\pgfsetrectcap%
\pgfsetroundjoin%
\pgfsetlinewidth{1.003750pt}%
\definecolor{currentstroke}{rgb}{0.000000,0.500000,0.000000}%
\pgfsetstrokecolor{currentstroke}%
\pgfsetdash{}{0pt}%
\pgfpathmoveto{\pgfqpoint{4.513279in}{1.996834in}}%
\pgfpathlineto{\pgfqpoint{4.707646in}{1.996834in}}%
\pgfusepath{stroke}%
\end{pgfscope}%
\begin{pgfscope}%
\pgftext[left,bottom,x=4.860362in,y=1.946277in,rotate=0.000000]{{\sffamily\fontsize{9.996000}{11.995200}\selectfont dt = 0.001a}}
%
\end{pgfscope}%
\begin{pgfscope}%
\pgfsetrectcap%
\pgfsetroundjoin%
\pgfsetlinewidth{1.003750pt}%
\definecolor{currentstroke}{rgb}{1.000000,0.000000,0.000000}%
\pgfsetstrokecolor{currentstroke}%
\pgfsetdash{}{0pt}%
\pgfpathmoveto{\pgfqpoint{4.513279in}{1.793058in}}%
\pgfpathlineto{\pgfqpoint{4.707646in}{1.793058in}}%
\pgfusepath{stroke}%
\end{pgfscope}%
\begin{pgfscope}%
\pgftext[left,bottom,x=4.860362in,y=1.742501in,rotate=0.000000]{{\sffamily\fontsize{9.996000}{11.995200}\selectfont dt = 0.01a}}
%
\end{pgfscope}%
\end{pgfpicture}%
\makeatother%
\endgroup%
}
	\caption{Simulated trajectory of the moon in the rest frame of the earth using different time steps}\label{fig:s2}
\end{figure}

As expected it seems that the tinier the chosen time step is the more precise the simulated trajectory gets. For the time steps \verb dt1 = 0.0001 and \verb dt = 0.001 the moons simulated trajectory relative to the earth is a ellipse as also observed in reality and therefore satisfactory.\\
In this small simulation of only six particles for each one of them the forces of all the others have to be computed every time step. (In this simulation that makes 15 forces per time step.) Therefor this part of the whole simulation is the most time-consuming.\\
For modern simulations with up to a few billion particles computing the forces takes an immense amount of the whole computing time.

\subsection{Integrators}
The simulation of the moons trajectory used the Euler integration scheme. Now this scheme is compared to the symplectic Euler scheme and the Velocity Verlet algorithm (see part \ref{theo}) by computing the solar system using the different algorithms and comparing the plots of the trajectory of the moon. The result is shown in figure \ref{fig:s3}.

\begin{figure}[H]
	\resizebox{1\textwidth}{!}{%% Creator: Matplotlib, PGF backend
%%
%% To include the figure in your LaTeX document, write
%%   \input{<filename>.pgf}
%%
%% Make sure the required packages are loaded in your preamble
%%   \usepackage{pgf}
%%
%% Figures using additional raster images can only be included by \input if
%% they are in the same directory as the main LaTeX file. For loading figures
%% from other directories you can use the `import` package
%%   \usepackage{import}
%% and then include the figures with
%%   \import{<path to file>}{<filename>.pgf}
%%
%% Matplotlib used the following preamble
%%   \usepackage{fontspec}
%%   \setmainfont{DejaVu Serif}
%%   \setsansfont{DejaVu Sans}
%%   \setmonofont{DejaVu Sans Mono}
%%
\begingroup%
\makeatletter%
\begin{pgfpicture}%
\pgfpathrectangle{\pgfpointorigin}{\pgfqpoint{10.000000in}{4.000000in}}%
\pgfusepath{use as bounding box}%
\begin{pgfscope}%
\pgfsetrectcap%
\pgfsetroundjoin%
\definecolor{currentfill}{rgb}{1.000000,1.000000,1.000000}%
\pgfsetfillcolor{currentfill}%
\pgfsetlinewidth{0.000000pt}%
\definecolor{currentstroke}{rgb}{1.000000,1.000000,1.000000}%
\pgfsetstrokecolor{currentstroke}%
\pgfsetdash{}{0pt}%
\pgfpathmoveto{\pgfqpoint{0.000000in}{0.000000in}}%
\pgfpathlineto{\pgfqpoint{10.000000in}{0.000000in}}%
\pgfpathlineto{\pgfqpoint{10.000000in}{4.000000in}}%
\pgfpathlineto{\pgfqpoint{0.000000in}{4.000000in}}%
\pgfpathclose%
\pgfusepath{fill}%
\end{pgfscope}%
\begin{pgfscope}%
\pgfsetrectcap%
\pgfsetroundjoin%
\definecolor{currentfill}{rgb}{1.000000,1.000000,1.000000}%
\pgfsetfillcolor{currentfill}%
\pgfsetlinewidth{0.000000pt}%
\definecolor{currentstroke}{rgb}{0.000000,0.000000,0.000000}%
\pgfsetstrokecolor{currentstroke}%
\pgfsetdash{}{0pt}%
\pgfpathmoveto{\pgfqpoint{1.250000in}{0.400000in}}%
\pgfpathlineto{\pgfqpoint{9.000000in}{0.400000in}}%
\pgfpathlineto{\pgfqpoint{9.000000in}{3.600000in}}%
\pgfpathlineto{\pgfqpoint{1.250000in}{3.600000in}}%
\pgfpathclose%
\pgfusepath{fill}%
\end{pgfscope}%
\begin{pgfscope}%
\pgfpathrectangle{\pgfqpoint{1.250000in}{0.400000in}}{\pgfqpoint{7.750000in}{3.200000in}} %
\pgfusepath{clip}%
\pgfsetrectcap%
\pgfsetroundjoin%
\pgfsetlinewidth{1.003750pt}%
\definecolor{currentstroke}{rgb}{0.000000,0.000000,1.000000}%
\pgfsetstrokecolor{currentstroke}%
\pgfsetdash{}{0pt}%
\pgfpathmoveto{\pgfqpoint{8.423607in}{3.142857in}}%
\pgfpathlineto{\pgfqpoint{8.423607in}{3.239849in}}%
\pgfpathlineto{\pgfqpoint{8.283703in}{3.336842in}}%
\pgfpathlineto{\pgfqpoint{8.080814in}{3.401975in}}%
\pgfpathlineto{\pgfqpoint{7.870068in}{3.437406in}}%
\pgfpathlineto{\pgfqpoint{7.667358in}{3.457128in}}%
\pgfpathlineto{\pgfqpoint{7.472190in}{3.468189in}}%
\pgfpathlineto{\pgfqpoint{7.281968in}{3.473445in}}%
\pgfpathlineto{\pgfqpoint{7.094534in}{3.473974in}}%
\pgfpathlineto{\pgfqpoint{6.908381in}{3.470073in}}%
\pgfpathlineto{\pgfqpoint{6.722550in}{3.461663in}}%
\pgfpathlineto{\pgfqpoint{6.536522in}{3.448480in}}%
\pgfpathlineto{\pgfqpoint{6.350137in}{3.430174in}}%
\pgfpathlineto{\pgfqpoint{6.163532in}{3.406363in}}%
\pgfpathlineto{\pgfqpoint{5.977100in}{3.376680in}}%
\pgfpathlineto{\pgfqpoint{5.791441in}{3.340795in}}%
\pgfpathlineto{\pgfqpoint{5.607335in}{3.298438in}}%
\pgfpathlineto{\pgfqpoint{5.425703in}{3.249417in}}%
\pgfpathlineto{\pgfqpoint{5.247571in}{3.193626in}}%
\pgfpathlineto{\pgfqpoint{5.074041in}{3.131056in}}%
\pgfpathlineto{\pgfqpoint{4.906255in}{3.061799in}}%
\pgfpathlineto{\pgfqpoint{4.745366in}{2.986047in}}%
\pgfpathlineto{\pgfqpoint{4.592505in}{2.904095in}}%
\pgfpathlineto{\pgfqpoint{4.448755in}{2.816331in}}%
\pgfpathlineto{\pgfqpoint{4.315123in}{2.723233in}}%
\pgfpathlineto{\pgfqpoint{4.192524in}{2.625358in}}%
\pgfpathlineto{\pgfqpoint{4.081755in}{2.523334in}}%
\pgfpathlineto{\pgfqpoint{3.983484in}{2.417844in}}%
\pgfpathlineto{\pgfqpoint{3.898240in}{2.309618in}}%
\pgfpathlineto{\pgfqpoint{3.826402in}{2.199416in}}%
\pgfpathlineto{\pgfqpoint{3.768201in}{2.088020in}}%
\pgfpathlineto{\pgfqpoint{3.723713in}{1.976213in}}%
\pgfpathlineto{\pgfqpoint{3.692868in}{1.864777in}}%
\pgfpathlineto{\pgfqpoint{3.675454in}{1.754475in}}%
\pgfpathlineto{\pgfqpoint{3.671126in}{1.646042in}}%
\pgfpathlineto{\pgfqpoint{3.679412in}{1.540179in}}%
\pgfpathlineto{\pgfqpoint{3.699733in}{1.437541in}}%
\pgfpathlineto{\pgfqpoint{3.731405in}{1.338736in}}%
\pgfpathlineto{\pgfqpoint{3.773664in}{1.244316in}}%
\pgfpathlineto{\pgfqpoint{3.825670in}{1.154775in}}%
\pgfpathlineto{\pgfqpoint{3.886526in}{1.070547in}}%
\pgfpathlineto{\pgfqpoint{3.955290in}{0.992002in}}%
\pgfpathlineto{\pgfqpoint{4.030987in}{0.919452in}}%
\pgfpathlineto{\pgfqpoint{4.112623in}{0.853143in}}%
\pgfpathlineto{\pgfqpoint{4.199194in}{0.793265in}}%
\pgfpathlineto{\pgfqpoint{4.289697in}{0.739948in}}%
\pgfpathlineto{\pgfqpoint{4.383140in}{0.693266in}}%
\pgfpathlineto{\pgfqpoint{4.478552in}{0.653243in}}%
\pgfpathlineto{\pgfqpoint{4.574986in}{0.619852in}}%
\pgfpathlineto{\pgfqpoint{4.671526in}{0.593022in}}%
\pgfpathlineto{\pgfqpoint{4.767297in}{0.572640in}}%
\pgfpathlineto{\pgfqpoint{4.861465in}{0.558553in}}%
\pgfpathlineto{\pgfqpoint{4.953239in}{0.550577in}}%
\pgfpathlineto{\pgfqpoint{5.041879in}{0.548496in}}%
\pgfpathlineto{\pgfqpoint{5.126693in}{0.552066in}}%
\pgfpathlineto{\pgfqpoint{5.207043in}{0.561023in}}%
\pgfpathlineto{\pgfqpoint{5.282341in}{0.575079in}}%
\pgfpathlineto{\pgfqpoint{5.352051in}{0.593932in}}%
\pgfpathlineto{\pgfqpoint{5.415692in}{0.617265in}}%
\pgfpathlineto{\pgfqpoint{5.472831in}{0.644751in}}%
\pgfpathlineto{\pgfqpoint{5.523088in}{0.676051in}}%
\pgfpathlineto{\pgfqpoint{5.566133in}{0.710826in}}%
\pgfpathlineto{\pgfqpoint{5.601683in}{0.748727in}}%
\pgfpathlineto{\pgfqpoint{5.629501in}{0.789407in}}%
\pgfpathlineto{\pgfqpoint{5.649397in}{0.832517in}}%
\pgfpathlineto{\pgfqpoint{5.661222in}{0.877710in}}%
\pgfpathlineto{\pgfqpoint{5.664870in}{0.924642in}}%
\pgfpathlineto{\pgfqpoint{5.660274in}{0.972973in}}%
\pgfpathlineto{\pgfqpoint{5.647403in}{1.022369in}}%
\pgfpathlineto{\pgfqpoint{5.626263in}{1.072501in}}%
\pgfpathlineto{\pgfqpoint{5.596893in}{1.123047in}}%
\pgfpathlineto{\pgfqpoint{5.559364in}{1.173693in}}%
\pgfpathlineto{\pgfqpoint{5.513774in}{1.224135in}}%
\pgfpathlineto{\pgfqpoint{5.460250in}{1.274075in}}%
\pgfpathlineto{\pgfqpoint{5.398945in}{1.323226in}}%
\pgfpathlineto{\pgfqpoint{5.330035in}{1.371310in}}%
\pgfpathlineto{\pgfqpoint{5.253717in}{1.418059in}}%
\pgfpathlineto{\pgfqpoint{5.170208in}{1.463215in}}%
\pgfpathlineto{\pgfqpoint{5.079742in}{1.506530in}}%
\pgfpathlineto{\pgfqpoint{4.982570in}{1.547766in}}%
\pgfpathlineto{\pgfqpoint{4.878958in}{1.586698in}}%
\pgfpathlineto{\pgfqpoint{4.769184in}{1.623109in}}%
\pgfpathlineto{\pgfqpoint{4.653538in}{1.656794in}}%
\pgfpathlineto{\pgfqpoint{4.532319in}{1.687561in}}%
\pgfpathlineto{\pgfqpoint{4.405838in}{1.715226in}}%
\pgfpathlineto{\pgfqpoint{4.274410in}{1.739618in}}%
\pgfpathlineto{\pgfqpoint{4.138362in}{1.760577in}}%
\pgfpathlineto{\pgfqpoint{3.998024in}{1.777956in}}%
\pgfpathlineto{\pgfqpoint{3.853732in}{1.791615in}}%
\pgfpathlineto{\pgfqpoint{3.705827in}{1.801430in}}%
\pgfpathlineto{\pgfqpoint{3.554657in}{1.807286in}}%
\pgfpathlineto{\pgfqpoint{3.400569in}{1.809077in}}%
\pgfpathlineto{\pgfqpoint{3.243917in}{1.806710in}}%
\pgfpathlineto{\pgfqpoint{3.085055in}{1.800103in}}%
\pgfpathlineto{\pgfqpoint{2.924340in}{1.789183in}}%
\pgfpathlineto{\pgfqpoint{2.762130in}{1.773887in}}%
\pgfpathlineto{\pgfqpoint{2.598786in}{1.754162in}}%
\pgfpathlineto{\pgfqpoint{2.434666in}{1.729966in}}%
\pgfpathlineto{\pgfqpoint{2.270130in}{1.701265in}}%
\pgfpathlineto{\pgfqpoint{2.105538in}{1.668034in}}%
\pgfpathlineto{\pgfqpoint{1.941249in}{1.630258in}}%
\pgfusepath{stroke}%
\end{pgfscope}%
\begin{pgfscope}%
\pgfpathrectangle{\pgfqpoint{1.250000in}{0.400000in}}{\pgfqpoint{7.750000in}{3.200000in}} %
\pgfusepath{clip}%
\pgfsetrectcap%
\pgfsetroundjoin%
\pgfsetlinewidth{1.003750pt}%
\definecolor{currentstroke}{rgb}{0.000000,0.500000,0.000000}%
\pgfsetstrokecolor{currentstroke}%
\pgfsetdash{}{0pt}%
\pgfpathmoveto{\pgfqpoint{8.423607in}{3.142857in}}%
\pgfpathlineto{\pgfqpoint{8.283703in}{3.239849in}}%
\pgfpathlineto{\pgfqpoint{8.082843in}{3.234803in}}%
\pgfpathlineto{\pgfqpoint{7.965592in}{3.174637in}}%
\pgfpathlineto{\pgfqpoint{7.924315in}{3.104478in}}%
\pgfpathlineto{\pgfqpoint{7.938258in}{3.041275in}}%
\pgfpathlineto{\pgfqpoint{7.992335in}{2.992901in}}%
\pgfpathlineto{\pgfqpoint{8.075538in}{2.964748in}}%
\pgfpathlineto{\pgfqpoint{8.178209in}{2.962484in}}%
\pgfpathlineto{\pgfqpoint{8.287836in}{2.994288in}}%
\pgfpathlineto{\pgfqpoint{8.376281in}{3.073763in}}%
\pgfpathlineto{\pgfqpoint{8.344088in}{3.208797in}}%
\pgfpathlineto{\pgfqpoint{8.160456in}{3.259900in}}%
\pgfpathlineto{\pgfqpoint{8.017569in}{3.239100in}}%
\pgfpathlineto{\pgfqpoint{7.938379in}{3.188583in}}%
\pgfpathlineto{\pgfqpoint{7.920590in}{3.127441in}}%
\pgfpathlineto{\pgfqpoint{7.962257in}{3.068179in}}%
\pgfpathlineto{\pgfqpoint{8.062558in}{3.024454in}}%
\pgfpathlineto{\pgfqpoint{8.214730in}{3.017902in}}%
\pgfpathlineto{\pgfqpoint{8.371950in}{3.083585in}}%
\pgfpathlineto{\pgfqpoint{8.383003in}{3.207817in}}%
\pgfpathlineto{\pgfqpoint{8.270478in}{3.282113in}}%
\pgfpathlineto{\pgfqpoint{8.140122in}{3.301676in}}%
\pgfpathlineto{\pgfqpoint{8.031622in}{3.283081in}}%
\pgfpathlineto{\pgfqpoint{7.961644in}{3.238383in}}%
\pgfpathlineto{\pgfqpoint{7.942524in}{3.176456in}}%
\pgfpathlineto{\pgfqpoint{7.990422in}{3.107116in}}%
\pgfpathlineto{\pgfqpoint{8.131414in}{3.051729in}}%
\pgfpathlineto{\pgfqpoint{8.361752in}{3.082028in}}%
\pgfpathlineto{\pgfqpoint{8.440081in}{3.179156in}}%
\pgfpathlineto{\pgfqpoint{8.410346in}{3.257964in}}%
\pgfpathlineto{\pgfqpoint{8.326244in}{3.304096in}}%
\pgfpathlineto{\pgfqpoint{8.217790in}{3.314565in}}%
\pgfpathlineto{\pgfqpoint{8.109389in}{3.287149in}}%
\pgfpathlineto{\pgfqpoint{8.032176in}{3.220121in}}%
\pgfpathlineto{\pgfqpoint{8.039024in}{3.120026in}}%
\pgfpathlineto{\pgfqpoint{8.197259in}{3.039129in}}%
\pgfpathlineto{\pgfqpoint{8.383965in}{3.060703in}}%
\pgfpathlineto{\pgfqpoint{8.478026in}{3.130156in}}%
\pgfpathlineto{\pgfqpoint{8.486919in}{3.203040in}}%
\pgfpathlineto{\pgfqpoint{8.430567in}{3.259910in}}%
\pgfpathlineto{\pgfqpoint{8.325021in}{3.288022in}}%
\pgfpathlineto{\pgfqpoint{8.189077in}{3.272405in}}%
\pgfpathlineto{\pgfqpoint{8.070152in}{3.191010in}}%
\pgfpathlineto{\pgfqpoint{8.110980in}{3.059731in}}%
\pgfpathlineto{\pgfqpoint{8.258242in}{3.006785in}}%
\pgfpathlineto{\pgfqpoint{8.391649in}{3.014234in}}%
\pgfpathlineto{\pgfqpoint{8.481709in}{3.056711in}}%
\pgfpathlineto{\pgfqpoint{8.518298in}{3.118030in}}%
\pgfpathlineto{\pgfqpoint{8.494267in}{3.184729in}}%
\pgfpathlineto{\pgfqpoint{8.402766in}{3.240521in}}%
\pgfpathlineto{\pgfqpoint{8.243611in}{3.258058in}}%
\pgfpathlineto{\pgfqpoint{8.071197in}{3.190083in}}%
\pgfpathlineto{\pgfqpoint{8.062531in}{3.071183in}}%
\pgfpathlineto{\pgfqpoint{8.160984in}{3.000088in}}%
\pgfpathlineto{\pgfqpoint{8.280935in}{2.979969in}}%
\pgfpathlineto{\pgfqpoint{8.386123in}{3.000938in}}%
\pgfpathlineto{\pgfqpoint{8.454028in}{3.055151in}}%
\pgfpathlineto{\pgfqpoint{8.459487in}{3.134131in}}%
\pgfpathlineto{\pgfqpoint{8.364917in}{3.217574in}}%
\pgfpathlineto{\pgfqpoint{8.155355in}{3.239467in}}%
\pgfpathlineto{\pgfqpoint{8.015426in}{3.165609in}}%
\pgfpathlineto{\pgfqpoint{7.996137in}{3.077997in}}%
\pgfpathlineto{\pgfqpoint{8.053963in}{3.011623in}}%
\pgfpathlineto{\pgfqpoint{8.153680in}{2.977475in}}%
\pgfpathlineto{\pgfqpoint{8.268401in}{2.980649in}}%
\pgfpathlineto{\pgfqpoint{8.370546in}{3.025235in}}%
\pgfpathlineto{\pgfqpoint{8.417656in}{3.113288in}}%
\pgfpathlineto{\pgfqpoint{8.327988in}{3.222577in}}%
\pgfpathlineto{\pgfqpoint{8.121481in}{3.241868in}}%
\pgfpathlineto{\pgfqpoint{7.991091in}{3.189517in}}%
\pgfpathlineto{\pgfqpoint{7.948265in}{3.119786in}}%
\pgfpathlineto{\pgfqpoint{7.976930in}{3.055437in}}%
\pgfpathlineto{\pgfqpoint{8.063533in}{3.010471in}}%
\pgfpathlineto{\pgfqpoint{8.193090in}{2.999438in}}%
\pgfpathlineto{\pgfqpoint{8.334738in}{3.041655in}}%
\pgfpathlineto{\pgfqpoint{8.403550in}{3.149990in}}%
\pgfpathlineto{\pgfqpoint{8.297899in}{3.251341in}}%
\pgfpathlineto{\pgfqpoint{8.139197in}{3.275200in}}%
\pgfpathlineto{\pgfqpoint{8.014848in}{3.246776in}}%
\pgfpathlineto{\pgfqpoint{7.947509in}{3.191233in}}%
\pgfpathlineto{\pgfqpoint{7.944534in}{3.125362in}}%
\pgfpathlineto{\pgfqpoint{8.012037in}{3.064346in}}%
\pgfpathlineto{\pgfqpoint{8.155474in}{3.031087in}}%
\pgfpathlineto{\pgfqpoint{8.347026in}{3.073324in}}%
\pgfpathlineto{\pgfqpoint{8.399654in}{3.194362in}}%
\pgfpathlineto{\pgfqpoint{8.320928in}{3.276458in}}%
\pgfpathlineto{\pgfqpoint{8.205081in}{3.308567in}}%
\pgfpathlineto{\pgfqpoint{8.092645in}{3.299751in}}%
\pgfpathlineto{\pgfqpoint{8.006644in}{3.258264in}}%
\pgfpathlineto{\pgfqpoint{7.967344in}{3.192066in}}%
\pgfpathlineto{\pgfqpoint{7.999133in}{3.112894in}}%
\pgfpathlineto{\pgfqpoint{8.131880in}{3.047898in}}%
\pgfpathlineto{\pgfqpoint{8.345781in}{3.065621in}}%
\pgfpathlineto{\pgfqpoint{8.441682in}{3.159362in}}%
\pgfpathlineto{\pgfqpoint{8.422770in}{3.243705in}}%
\pgfpathlineto{\pgfqpoint{8.342692in}{3.295874in}}%
\pgfpathlineto{\pgfqpoint{8.234172in}{3.310443in}}%
\pgfpathlineto{\pgfqpoint{8.122759in}{3.283740in}}%
\pgfpathlineto{\pgfqpoint{8.042657in}{3.211821in}}%
\pgfpathlineto{\pgfqpoint{8.061123in}{3.101910in}}%
\pgfusepath{stroke}%
\end{pgfscope}%
\begin{pgfscope}%
\pgfpathrectangle{\pgfqpoint{1.250000in}{0.400000in}}{\pgfqpoint{7.750000in}{3.200000in}} %
\pgfusepath{clip}%
\pgfsetrectcap%
\pgfsetroundjoin%
\pgfsetlinewidth{1.003750pt}%
\definecolor{currentstroke}{rgb}{1.000000,0.000000,0.000000}%
\pgfsetstrokecolor{currentstroke}%
\pgfsetdash{}{0pt}%
\pgfpathmoveto{\pgfqpoint{8.423607in}{3.142857in}}%
\pgfpathlineto{\pgfqpoint{8.353655in}{3.239849in}}%
\pgfpathlineto{\pgfqpoint{8.205950in}{3.277053in}}%
\pgfpathlineto{\pgfqpoint{8.067371in}{3.250808in}}%
\pgfpathlineto{\pgfqpoint{7.990434in}{3.181231in}}%
\pgfpathlineto{\pgfqpoint{8.002747in}{3.096287in}}%
\pgfpathlineto{\pgfqpoint{8.107797in}{3.030822in}}%
\pgfpathlineto{\pgfqpoint{8.270166in}{3.021574in}}%
\pgfpathlineto{\pgfqpoint{8.404844in}{3.085207in}}%
\pgfpathlineto{\pgfqpoint{8.422938in}{3.186948in}}%
\pgfpathlineto{\pgfqpoint{8.325706in}{3.263312in}}%
\pgfpathlineto{\pgfqpoint{8.182390in}{3.283087in}}%
\pgfpathlineto{\pgfqpoint{8.057196in}{3.247300in}}%
\pgfpathlineto{\pgfqpoint{7.997155in}{3.170954in}}%
\pgfpathlineto{\pgfqpoint{8.037841in}{3.081451in}}%
\pgfpathlineto{\pgfqpoint{8.184081in}{3.026000in}}%
\pgfpathlineto{\pgfqpoint{8.356784in}{3.048741in}}%
\pgfpathlineto{\pgfqpoint{8.445590in}{3.130734in}}%
\pgfpathlineto{\pgfqpoint{8.420840in}{3.219533in}}%
\pgfpathlineto{\pgfqpoint{8.312187in}{3.276345in}}%
\pgfpathlineto{\pgfqpoint{8.170131in}{3.281971in}}%
\pgfpathlineto{\pgfqpoint{8.050604in}{3.233740in}}%
\pgfpathlineto{\pgfqpoint{8.009288in}{3.145620in}}%
\pgfpathlineto{\pgfqpoint{8.088570in}{3.055711in}}%
\pgfpathlineto{\pgfqpoint{8.261754in}{3.024811in}}%
\pgfpathlineto{\pgfqpoint{8.410297in}{3.071202in}}%
\pgfpathlineto{\pgfqpoint{8.463054in}{3.153795in}}%
\pgfpathlineto{\pgfqpoint{8.417304in}{3.231893in}}%
\pgfpathlineto{\pgfqpoint{8.297185in}{3.276506in}}%
\pgfpathlineto{\pgfqpoint{8.146383in}{3.266554in}}%
\pgfpathlineto{\pgfqpoint{8.034305in}{3.196158in}}%
\pgfpathlineto{\pgfqpoint{8.035690in}{3.094847in}}%
\pgfpathlineto{\pgfqpoint{8.158563in}{3.024297in}}%
\pgfpathlineto{\pgfqpoint{8.321534in}{3.022980in}}%
\pgfpathlineto{\pgfqpoint{8.438087in}{3.079498in}}%
\pgfpathlineto{\pgfqpoint{8.467570in}{3.161089in}}%
\pgfpathlineto{\pgfqpoint{8.404326in}{3.235002in}}%
\pgfpathlineto{\pgfqpoint{8.265939in}{3.270378in}}%
\pgfpathlineto{\pgfqpoint{8.104917in}{3.239123in}}%
\pgfpathlineto{\pgfqpoint{8.023240in}{3.143863in}}%
\pgfpathlineto{\pgfqpoint{8.078024in}{3.048600in}}%
\pgfpathlineto{\pgfqpoint{8.213300in}{3.005190in}}%
\pgfpathlineto{\pgfqpoint{8.354882in}{3.022075in}}%
\pgfpathlineto{\pgfqpoint{8.447295in}{3.085980in}}%
\pgfpathlineto{\pgfqpoint{8.454195in}{3.171687in}}%
\pgfpathlineto{\pgfqpoint{8.362917in}{3.244277in}}%
\pgfpathlineto{\pgfqpoint{8.200066in}{3.263462in}}%
\pgfpathlineto{\pgfqpoint{8.052361in}{3.205314in}}%
\pgfpathlineto{\pgfqpoint{8.019134in}{3.105234in}}%
\pgfpathlineto{\pgfqpoint{8.099221in}{3.026455in}}%
\pgfpathlineto{\pgfqpoint{8.233291in}{2.998869in}}%
\pgfpathlineto{\pgfqpoint{8.364294in}{3.026978in}}%
\pgfpathlineto{\pgfqpoint{8.440618in}{3.101874in}}%
\pgfpathlineto{\pgfqpoint{8.415008in}{3.196752in}}%
\pgfpathlineto{\pgfqpoint{8.277855in}{3.259565in}}%
\pgfpathlineto{\pgfqpoint{8.107253in}{3.246259in}}%
\pgfpathlineto{\pgfqpoint{8.006185in}{3.171160in}}%
\pgfpathlineto{\pgfqpoint{8.012528in}{3.081458in}}%
\pgfpathlineto{\pgfqpoint{8.106706in}{3.017206in}}%
\pgfpathlineto{\pgfqpoint{8.244885in}{3.001265in}}%
\pgfpathlineto{\pgfqpoint{8.373985in}{3.042032in}}%
\pgfpathlineto{\pgfqpoint{8.431439in}{3.131548in}}%
\pgfpathlineto{\pgfqpoint{8.360210in}{3.228828in}}%
\pgfpathlineto{\pgfqpoint{8.194162in}{3.265629in}}%
\pgfpathlineto{\pgfqpoint{8.047741in}{3.229025in}}%
\pgfpathlineto{\pgfqpoint{7.983570in}{3.152376in}}%
\pgfpathlineto{\pgfqpoint{8.014470in}{3.071210in}}%
\pgfpathlineto{\pgfqpoint{8.126577in}{3.016926in}}%
\pgfpathlineto{\pgfqpoint{8.279960in}{3.015833in}}%
\pgfpathlineto{\pgfqpoint{8.404316in}{3.078775in}}%
\pgfpathlineto{\pgfqpoint{8.419271in}{3.180583in}}%
\pgfpathlineto{\pgfqpoint{8.307262in}{3.258455in}}%
\pgfpathlineto{\pgfqpoint{8.147812in}{3.270095in}}%
\pgfpathlineto{\pgfqpoint{8.024978in}{3.223502in}}%
\pgfpathlineto{\pgfqpoint{7.981593in}{3.145659in}}%
\pgfpathlineto{\pgfqpoint{8.033265in}{3.066592in}}%
\pgfpathlineto{\pgfqpoint{8.172138in}{3.021299in}}%
\pgfpathlineto{\pgfqpoint{8.341917in}{3.045508in}}%
\pgfpathlineto{\pgfqpoint{8.432579in}{3.134549in}}%
\pgfpathlineto{\pgfqpoint{8.393906in}{3.228455in}}%
\pgfpathlineto{\pgfqpoint{8.268950in}{3.279481in}}%
\pgfpathlineto{\pgfqpoint{8.124418in}{3.273195in}}%
\pgfpathlineto{\pgfqpoint{8.018544in}{3.217158in}}%
\pgfpathlineto{\pgfqpoint{7.994006in}{3.132781in}}%
\pgfpathlineto{\pgfqpoint{8.073628in}{3.053518in}}%
\pgfpathlineto{\pgfqpoint{8.240191in}{3.025052in}}%
\pgfpathlineto{\pgfqpoint{8.397287in}{3.077856in}}%
\pgfpathlineto{\pgfqpoint{8.442703in}{3.172259in}}%
\pgfpathlineto{\pgfqpoint{8.379039in}{3.251267in}}%
\pgfpathlineto{\pgfqpoint{8.252261in}{3.286088in}}%
\pgfpathlineto{\pgfqpoint{8.113704in}{3.266213in}}%
\pgfpathlineto{\pgfqpoint{8.020277in}{3.195008in}}%
\pgfpathlineto{\pgfqpoint{8.028533in}{3.098221in}}%
\pgfpathlineto{\pgfqpoint{8.153719in}{3.028764in}}%
\pgfpathlineto{\pgfqpoint{8.324859in}{3.032355in}}%
\pgfpathlineto{\pgfqpoint{8.439163in}{3.100364in}}%
\pgfpathlineto{\pgfqpoint{8.451536in}{3.188556in}}%
\pgfpathlineto{\pgfqpoint{8.373119in}{3.257605in}}%
\pgfpathlineto{\pgfqpoint{8.238747in}{3.282243in}}%
\pgfpathlineto{\pgfqpoint{8.098257in}{3.247712in}}%
\pgfpathlineto{\pgfqpoint{8.024541in}{3.156636in}}%
\pgfusepath{stroke}%
\end{pgfscope}%
\begin{pgfscope}%
\pgfpathrectangle{\pgfqpoint{1.250000in}{0.400000in}}{\pgfqpoint{7.750000in}{3.200000in}} %
\pgfusepath{clip}%
\pgfsetbuttcap%
\pgfsetroundjoin%
\pgfsetlinewidth{0.501875pt}%
\definecolor{currentstroke}{rgb}{0.000000,0.000000,0.000000}%
\pgfsetstrokecolor{currentstroke}%
\pgfsetdash{{1.000000pt}{3.000000pt}}{0.000000pt}%
\pgfpathmoveto{\pgfqpoint{2.025000in}{0.400000in}}%
\pgfpathlineto{\pgfqpoint{2.025000in}{3.600000in}}%
\pgfusepath{stroke}%
\end{pgfscope}%
\begin{pgfscope}%
\pgfsetbuttcap%
\pgfsetroundjoin%
\definecolor{currentfill}{rgb}{0.000000,0.000000,0.000000}%
\pgfsetfillcolor{currentfill}%
\pgfsetlinewidth{0.501875pt}%
\definecolor{currentstroke}{rgb}{0.000000,0.000000,0.000000}%
\pgfsetstrokecolor{currentstroke}%
\pgfsetdash{}{0pt}%
\pgfsys@defobject{currentmarker}{\pgfqpoint{0.000000in}{0.000000in}}{\pgfqpoint{0.000000in}{0.055556in}}{%
\pgfpathmoveto{\pgfqpoint{0.000000in}{0.000000in}}%
\pgfpathlineto{\pgfqpoint{0.000000in}{0.055556in}}%
\pgfusepath{stroke,fill}%
}%
\begin{pgfscope}%
\pgfsys@transformshift{2.025000in}{0.400000in}%
\pgfsys@useobject{currentmarker}{}%
\end{pgfscope}%
\end{pgfscope}%
\begin{pgfscope}%
\pgfsetbuttcap%
\pgfsetroundjoin%
\definecolor{currentfill}{rgb}{0.000000,0.000000,0.000000}%
\pgfsetfillcolor{currentfill}%
\pgfsetlinewidth{0.501875pt}%
\definecolor{currentstroke}{rgb}{0.000000,0.000000,0.000000}%
\pgfsetstrokecolor{currentstroke}%
\pgfsetdash{}{0pt}%
\pgfsys@defobject{currentmarker}{\pgfqpoint{0.000000in}{-0.055556in}}{\pgfqpoint{0.000000in}{0.000000in}}{%
\pgfpathmoveto{\pgfqpoint{0.000000in}{0.000000in}}%
\pgfpathlineto{\pgfqpoint{0.000000in}{-0.055556in}}%
\pgfusepath{stroke,fill}%
}%
\begin{pgfscope}%
\pgfsys@transformshift{2.025000in}{3.600000in}%
\pgfsys@useobject{currentmarker}{}%
\end{pgfscope}%
\end{pgfscope}%
\begin{pgfscope}%
\pgftext[left,bottom,x=1.769629in,y=0.218387in,rotate=0.000000]{{\sffamily\fontsize{12.000000}{14.400000}\selectfont −0.08}}
%
\end{pgfscope}%
\begin{pgfscope}%
\pgfpathrectangle{\pgfqpoint{1.250000in}{0.400000in}}{\pgfqpoint{7.750000in}{3.200000in}} %
\pgfusepath{clip}%
\pgfsetbuttcap%
\pgfsetroundjoin%
\pgfsetlinewidth{0.501875pt}%
\definecolor{currentstroke}{rgb}{0.000000,0.000000,0.000000}%
\pgfsetstrokecolor{currentstroke}%
\pgfsetdash{{1.000000pt}{3.000000pt}}{0.000000pt}%
\pgfpathmoveto{\pgfqpoint{3.575000in}{0.400000in}}%
\pgfpathlineto{\pgfqpoint{3.575000in}{3.600000in}}%
\pgfusepath{stroke}%
\end{pgfscope}%
\begin{pgfscope}%
\pgfsetbuttcap%
\pgfsetroundjoin%
\definecolor{currentfill}{rgb}{0.000000,0.000000,0.000000}%
\pgfsetfillcolor{currentfill}%
\pgfsetlinewidth{0.501875pt}%
\definecolor{currentstroke}{rgb}{0.000000,0.000000,0.000000}%
\pgfsetstrokecolor{currentstroke}%
\pgfsetdash{}{0pt}%
\pgfsys@defobject{currentmarker}{\pgfqpoint{0.000000in}{0.000000in}}{\pgfqpoint{0.000000in}{0.055556in}}{%
\pgfpathmoveto{\pgfqpoint{0.000000in}{0.000000in}}%
\pgfpathlineto{\pgfqpoint{0.000000in}{0.055556in}}%
\pgfusepath{stroke,fill}%
}%
\begin{pgfscope}%
\pgfsys@transformshift{3.575000in}{0.400000in}%
\pgfsys@useobject{currentmarker}{}%
\end{pgfscope}%
\end{pgfscope}%
\begin{pgfscope}%
\pgfsetbuttcap%
\pgfsetroundjoin%
\definecolor{currentfill}{rgb}{0.000000,0.000000,0.000000}%
\pgfsetfillcolor{currentfill}%
\pgfsetlinewidth{0.501875pt}%
\definecolor{currentstroke}{rgb}{0.000000,0.000000,0.000000}%
\pgfsetstrokecolor{currentstroke}%
\pgfsetdash{}{0pt}%
\pgfsys@defobject{currentmarker}{\pgfqpoint{0.000000in}{-0.055556in}}{\pgfqpoint{0.000000in}{0.000000in}}{%
\pgfpathmoveto{\pgfqpoint{0.000000in}{0.000000in}}%
\pgfpathlineto{\pgfqpoint{0.000000in}{-0.055556in}}%
\pgfusepath{stroke,fill}%
}%
\begin{pgfscope}%
\pgfsys@transformshift{3.575000in}{3.600000in}%
\pgfsys@useobject{currentmarker}{}%
\end{pgfscope}%
\end{pgfscope}%
\begin{pgfscope}%
\pgftext[left,bottom,x=3.319629in,y=0.218387in,rotate=0.000000]{{\sffamily\fontsize{12.000000}{14.400000}\selectfont −0.06}}
%
\end{pgfscope}%
\begin{pgfscope}%
\pgfpathrectangle{\pgfqpoint{1.250000in}{0.400000in}}{\pgfqpoint{7.750000in}{3.200000in}} %
\pgfusepath{clip}%
\pgfsetbuttcap%
\pgfsetroundjoin%
\pgfsetlinewidth{0.501875pt}%
\definecolor{currentstroke}{rgb}{0.000000,0.000000,0.000000}%
\pgfsetstrokecolor{currentstroke}%
\pgfsetdash{{1.000000pt}{3.000000pt}}{0.000000pt}%
\pgfpathmoveto{\pgfqpoint{5.125000in}{0.400000in}}%
\pgfpathlineto{\pgfqpoint{5.125000in}{3.600000in}}%
\pgfusepath{stroke}%
\end{pgfscope}%
\begin{pgfscope}%
\pgfsetbuttcap%
\pgfsetroundjoin%
\definecolor{currentfill}{rgb}{0.000000,0.000000,0.000000}%
\pgfsetfillcolor{currentfill}%
\pgfsetlinewidth{0.501875pt}%
\definecolor{currentstroke}{rgb}{0.000000,0.000000,0.000000}%
\pgfsetstrokecolor{currentstroke}%
\pgfsetdash{}{0pt}%
\pgfsys@defobject{currentmarker}{\pgfqpoint{0.000000in}{0.000000in}}{\pgfqpoint{0.000000in}{0.055556in}}{%
\pgfpathmoveto{\pgfqpoint{0.000000in}{0.000000in}}%
\pgfpathlineto{\pgfqpoint{0.000000in}{0.055556in}}%
\pgfusepath{stroke,fill}%
}%
\begin{pgfscope}%
\pgfsys@transformshift{5.125000in}{0.400000in}%
\pgfsys@useobject{currentmarker}{}%
\end{pgfscope}%
\end{pgfscope}%
\begin{pgfscope}%
\pgfsetbuttcap%
\pgfsetroundjoin%
\definecolor{currentfill}{rgb}{0.000000,0.000000,0.000000}%
\pgfsetfillcolor{currentfill}%
\pgfsetlinewidth{0.501875pt}%
\definecolor{currentstroke}{rgb}{0.000000,0.000000,0.000000}%
\pgfsetstrokecolor{currentstroke}%
\pgfsetdash{}{0pt}%
\pgfsys@defobject{currentmarker}{\pgfqpoint{0.000000in}{-0.055556in}}{\pgfqpoint{0.000000in}{0.000000in}}{%
\pgfpathmoveto{\pgfqpoint{0.000000in}{0.000000in}}%
\pgfpathlineto{\pgfqpoint{0.000000in}{-0.055556in}}%
\pgfusepath{stroke,fill}%
}%
\begin{pgfscope}%
\pgfsys@transformshift{5.125000in}{3.600000in}%
\pgfsys@useobject{currentmarker}{}%
\end{pgfscope}%
\end{pgfscope}%
\begin{pgfscope}%
\pgftext[left,bottom,x=4.869629in,y=0.218387in,rotate=0.000000]{{\sffamily\fontsize{12.000000}{14.400000}\selectfont −0.04}}
%
\end{pgfscope}%
\begin{pgfscope}%
\pgfpathrectangle{\pgfqpoint{1.250000in}{0.400000in}}{\pgfqpoint{7.750000in}{3.200000in}} %
\pgfusepath{clip}%
\pgfsetbuttcap%
\pgfsetroundjoin%
\pgfsetlinewidth{0.501875pt}%
\definecolor{currentstroke}{rgb}{0.000000,0.000000,0.000000}%
\pgfsetstrokecolor{currentstroke}%
\pgfsetdash{{1.000000pt}{3.000000pt}}{0.000000pt}%
\pgfpathmoveto{\pgfqpoint{6.675000in}{0.400000in}}%
\pgfpathlineto{\pgfqpoint{6.675000in}{3.600000in}}%
\pgfusepath{stroke}%
\end{pgfscope}%
\begin{pgfscope}%
\pgfsetbuttcap%
\pgfsetroundjoin%
\definecolor{currentfill}{rgb}{0.000000,0.000000,0.000000}%
\pgfsetfillcolor{currentfill}%
\pgfsetlinewidth{0.501875pt}%
\definecolor{currentstroke}{rgb}{0.000000,0.000000,0.000000}%
\pgfsetstrokecolor{currentstroke}%
\pgfsetdash{}{0pt}%
\pgfsys@defobject{currentmarker}{\pgfqpoint{0.000000in}{0.000000in}}{\pgfqpoint{0.000000in}{0.055556in}}{%
\pgfpathmoveto{\pgfqpoint{0.000000in}{0.000000in}}%
\pgfpathlineto{\pgfqpoint{0.000000in}{0.055556in}}%
\pgfusepath{stroke,fill}%
}%
\begin{pgfscope}%
\pgfsys@transformshift{6.675000in}{0.400000in}%
\pgfsys@useobject{currentmarker}{}%
\end{pgfscope}%
\end{pgfscope}%
\begin{pgfscope}%
\pgfsetbuttcap%
\pgfsetroundjoin%
\definecolor{currentfill}{rgb}{0.000000,0.000000,0.000000}%
\pgfsetfillcolor{currentfill}%
\pgfsetlinewidth{0.501875pt}%
\definecolor{currentstroke}{rgb}{0.000000,0.000000,0.000000}%
\pgfsetstrokecolor{currentstroke}%
\pgfsetdash{}{0pt}%
\pgfsys@defobject{currentmarker}{\pgfqpoint{0.000000in}{-0.055556in}}{\pgfqpoint{0.000000in}{0.000000in}}{%
\pgfpathmoveto{\pgfqpoint{0.000000in}{0.000000in}}%
\pgfpathlineto{\pgfqpoint{0.000000in}{-0.055556in}}%
\pgfusepath{stroke,fill}%
}%
\begin{pgfscope}%
\pgfsys@transformshift{6.675000in}{3.600000in}%
\pgfsys@useobject{currentmarker}{}%
\end{pgfscope}%
\end{pgfscope}%
\begin{pgfscope}%
\pgftext[left,bottom,x=6.419629in,y=0.218387in,rotate=0.000000]{{\sffamily\fontsize{12.000000}{14.400000}\selectfont −0.02}}
%
\end{pgfscope}%
\begin{pgfscope}%
\pgfpathrectangle{\pgfqpoint{1.250000in}{0.400000in}}{\pgfqpoint{7.750000in}{3.200000in}} %
\pgfusepath{clip}%
\pgfsetbuttcap%
\pgfsetroundjoin%
\pgfsetlinewidth{0.501875pt}%
\definecolor{currentstroke}{rgb}{0.000000,0.000000,0.000000}%
\pgfsetstrokecolor{currentstroke}%
\pgfsetdash{{1.000000pt}{3.000000pt}}{0.000000pt}%
\pgfpathmoveto{\pgfqpoint{8.225000in}{0.400000in}}%
\pgfpathlineto{\pgfqpoint{8.225000in}{3.600000in}}%
\pgfusepath{stroke}%
\end{pgfscope}%
\begin{pgfscope}%
\pgfsetbuttcap%
\pgfsetroundjoin%
\definecolor{currentfill}{rgb}{0.000000,0.000000,0.000000}%
\pgfsetfillcolor{currentfill}%
\pgfsetlinewidth{0.501875pt}%
\definecolor{currentstroke}{rgb}{0.000000,0.000000,0.000000}%
\pgfsetstrokecolor{currentstroke}%
\pgfsetdash{}{0pt}%
\pgfsys@defobject{currentmarker}{\pgfqpoint{0.000000in}{0.000000in}}{\pgfqpoint{0.000000in}{0.055556in}}{%
\pgfpathmoveto{\pgfqpoint{0.000000in}{0.000000in}}%
\pgfpathlineto{\pgfqpoint{0.000000in}{0.055556in}}%
\pgfusepath{stroke,fill}%
}%
\begin{pgfscope}%
\pgfsys@transformshift{8.225000in}{0.400000in}%
\pgfsys@useobject{currentmarker}{}%
\end{pgfscope}%
\end{pgfscope}%
\begin{pgfscope}%
\pgfsetbuttcap%
\pgfsetroundjoin%
\definecolor{currentfill}{rgb}{0.000000,0.000000,0.000000}%
\pgfsetfillcolor{currentfill}%
\pgfsetlinewidth{0.501875pt}%
\definecolor{currentstroke}{rgb}{0.000000,0.000000,0.000000}%
\pgfsetstrokecolor{currentstroke}%
\pgfsetdash{}{0pt}%
\pgfsys@defobject{currentmarker}{\pgfqpoint{0.000000in}{-0.055556in}}{\pgfqpoint{0.000000in}{0.000000in}}{%
\pgfpathmoveto{\pgfqpoint{0.000000in}{0.000000in}}%
\pgfpathlineto{\pgfqpoint{0.000000in}{-0.055556in}}%
\pgfusepath{stroke,fill}%
}%
\begin{pgfscope}%
\pgfsys@transformshift{8.225000in}{3.600000in}%
\pgfsys@useobject{currentmarker}{}%
\end{pgfscope}%
\end{pgfscope}%
\begin{pgfscope}%
\pgftext[left,bottom,x=8.039453in,y=0.218387in,rotate=0.000000]{{\sffamily\fontsize{12.000000}{14.400000}\selectfont 0.00}}
%
\end{pgfscope}%
\begin{pgfscope}%
\pgfpathrectangle{\pgfqpoint{1.250000in}{0.400000in}}{\pgfqpoint{7.750000in}{3.200000in}} %
\pgfusepath{clip}%
\pgfsetbuttcap%
\pgfsetroundjoin%
\pgfsetlinewidth{0.501875pt}%
\definecolor{currentstroke}{rgb}{0.000000,0.000000,0.000000}%
\pgfsetstrokecolor{currentstroke}%
\pgfsetdash{{1.000000pt}{3.000000pt}}{0.000000pt}%
\pgfpathmoveto{\pgfqpoint{1.250000in}{0.400000in}}%
\pgfpathlineto{\pgfqpoint{9.000000in}{0.400000in}}%
\pgfusepath{stroke}%
\end{pgfscope}%
\begin{pgfscope}%
\pgfsetbuttcap%
\pgfsetroundjoin%
\definecolor{currentfill}{rgb}{0.000000,0.000000,0.000000}%
\pgfsetfillcolor{currentfill}%
\pgfsetlinewidth{0.501875pt}%
\definecolor{currentstroke}{rgb}{0.000000,0.000000,0.000000}%
\pgfsetstrokecolor{currentstroke}%
\pgfsetdash{}{0pt}%
\pgfsys@defobject{currentmarker}{\pgfqpoint{0.000000in}{0.000000in}}{\pgfqpoint{0.055556in}{0.000000in}}{%
\pgfpathmoveto{\pgfqpoint{0.000000in}{0.000000in}}%
\pgfpathlineto{\pgfqpoint{0.055556in}{0.000000in}}%
\pgfusepath{stroke,fill}%
}%
\begin{pgfscope}%
\pgfsys@transformshift{1.250000in}{0.400000in}%
\pgfsys@useobject{currentmarker}{}%
\end{pgfscope}%
\end{pgfscope}%
\begin{pgfscope}%
\pgfsetbuttcap%
\pgfsetroundjoin%
\definecolor{currentfill}{rgb}{0.000000,0.000000,0.000000}%
\pgfsetfillcolor{currentfill}%
\pgfsetlinewidth{0.501875pt}%
\definecolor{currentstroke}{rgb}{0.000000,0.000000,0.000000}%
\pgfsetstrokecolor{currentstroke}%
\pgfsetdash{}{0pt}%
\pgfsys@defobject{currentmarker}{\pgfqpoint{-0.055556in}{0.000000in}}{\pgfqpoint{0.000000in}{0.000000in}}{%
\pgfpathmoveto{\pgfqpoint{0.000000in}{0.000000in}}%
\pgfpathlineto{\pgfqpoint{-0.055556in}{0.000000in}}%
\pgfusepath{stroke,fill}%
}%
\begin{pgfscope}%
\pgfsys@transformshift{9.000000in}{0.400000in}%
\pgfsys@useobject{currentmarker}{}%
\end{pgfscope}%
\end{pgfscope}%
\begin{pgfscope}%
\pgftext[left,bottom,x=0.683702in,y=0.336971in,rotate=0.000000]{{\sffamily\fontsize{12.000000}{14.400000}\selectfont −0.06}}
%
\end{pgfscope}%
\begin{pgfscope}%
\pgfpathrectangle{\pgfqpoint{1.250000in}{0.400000in}}{\pgfqpoint{7.750000in}{3.200000in}} %
\pgfusepath{clip}%
\pgfsetbuttcap%
\pgfsetroundjoin%
\pgfsetlinewidth{0.501875pt}%
\definecolor{currentstroke}{rgb}{0.000000,0.000000,0.000000}%
\pgfsetstrokecolor{currentstroke}%
\pgfsetdash{{1.000000pt}{3.000000pt}}{0.000000pt}%
\pgfpathmoveto{\pgfqpoint{1.250000in}{0.857143in}}%
\pgfpathlineto{\pgfqpoint{9.000000in}{0.857143in}}%
\pgfusepath{stroke}%
\end{pgfscope}%
\begin{pgfscope}%
\pgfsetbuttcap%
\pgfsetroundjoin%
\definecolor{currentfill}{rgb}{0.000000,0.000000,0.000000}%
\pgfsetfillcolor{currentfill}%
\pgfsetlinewidth{0.501875pt}%
\definecolor{currentstroke}{rgb}{0.000000,0.000000,0.000000}%
\pgfsetstrokecolor{currentstroke}%
\pgfsetdash{}{0pt}%
\pgfsys@defobject{currentmarker}{\pgfqpoint{0.000000in}{0.000000in}}{\pgfqpoint{0.055556in}{0.000000in}}{%
\pgfpathmoveto{\pgfqpoint{0.000000in}{0.000000in}}%
\pgfpathlineto{\pgfqpoint{0.055556in}{0.000000in}}%
\pgfusepath{stroke,fill}%
}%
\begin{pgfscope}%
\pgfsys@transformshift{1.250000in}{0.857143in}%
\pgfsys@useobject{currentmarker}{}%
\end{pgfscope}%
\end{pgfscope}%
\begin{pgfscope}%
\pgfsetbuttcap%
\pgfsetroundjoin%
\definecolor{currentfill}{rgb}{0.000000,0.000000,0.000000}%
\pgfsetfillcolor{currentfill}%
\pgfsetlinewidth{0.501875pt}%
\definecolor{currentstroke}{rgb}{0.000000,0.000000,0.000000}%
\pgfsetstrokecolor{currentstroke}%
\pgfsetdash{}{0pt}%
\pgfsys@defobject{currentmarker}{\pgfqpoint{-0.055556in}{0.000000in}}{\pgfqpoint{0.000000in}{0.000000in}}{%
\pgfpathmoveto{\pgfqpoint{0.000000in}{0.000000in}}%
\pgfpathlineto{\pgfqpoint{-0.055556in}{0.000000in}}%
\pgfusepath{stroke,fill}%
}%
\begin{pgfscope}%
\pgfsys@transformshift{9.000000in}{0.857143in}%
\pgfsys@useobject{currentmarker}{}%
\end{pgfscope}%
\end{pgfscope}%
\begin{pgfscope}%
\pgftext[left,bottom,x=0.683702in,y=0.794114in,rotate=0.000000]{{\sffamily\fontsize{12.000000}{14.400000}\selectfont −0.05}}
%
\end{pgfscope}%
\begin{pgfscope}%
\pgfpathrectangle{\pgfqpoint{1.250000in}{0.400000in}}{\pgfqpoint{7.750000in}{3.200000in}} %
\pgfusepath{clip}%
\pgfsetbuttcap%
\pgfsetroundjoin%
\pgfsetlinewidth{0.501875pt}%
\definecolor{currentstroke}{rgb}{0.000000,0.000000,0.000000}%
\pgfsetstrokecolor{currentstroke}%
\pgfsetdash{{1.000000pt}{3.000000pt}}{0.000000pt}%
\pgfpathmoveto{\pgfqpoint{1.250000in}{1.314286in}}%
\pgfpathlineto{\pgfqpoint{9.000000in}{1.314286in}}%
\pgfusepath{stroke}%
\end{pgfscope}%
\begin{pgfscope}%
\pgfsetbuttcap%
\pgfsetroundjoin%
\definecolor{currentfill}{rgb}{0.000000,0.000000,0.000000}%
\pgfsetfillcolor{currentfill}%
\pgfsetlinewidth{0.501875pt}%
\definecolor{currentstroke}{rgb}{0.000000,0.000000,0.000000}%
\pgfsetstrokecolor{currentstroke}%
\pgfsetdash{}{0pt}%
\pgfsys@defobject{currentmarker}{\pgfqpoint{0.000000in}{0.000000in}}{\pgfqpoint{0.055556in}{0.000000in}}{%
\pgfpathmoveto{\pgfqpoint{0.000000in}{0.000000in}}%
\pgfpathlineto{\pgfqpoint{0.055556in}{0.000000in}}%
\pgfusepath{stroke,fill}%
}%
\begin{pgfscope}%
\pgfsys@transformshift{1.250000in}{1.314286in}%
\pgfsys@useobject{currentmarker}{}%
\end{pgfscope}%
\end{pgfscope}%
\begin{pgfscope}%
\pgfsetbuttcap%
\pgfsetroundjoin%
\definecolor{currentfill}{rgb}{0.000000,0.000000,0.000000}%
\pgfsetfillcolor{currentfill}%
\pgfsetlinewidth{0.501875pt}%
\definecolor{currentstroke}{rgb}{0.000000,0.000000,0.000000}%
\pgfsetstrokecolor{currentstroke}%
\pgfsetdash{}{0pt}%
\pgfsys@defobject{currentmarker}{\pgfqpoint{-0.055556in}{0.000000in}}{\pgfqpoint{0.000000in}{0.000000in}}{%
\pgfpathmoveto{\pgfqpoint{0.000000in}{0.000000in}}%
\pgfpathlineto{\pgfqpoint{-0.055556in}{0.000000in}}%
\pgfusepath{stroke,fill}%
}%
\begin{pgfscope}%
\pgfsys@transformshift{9.000000in}{1.314286in}%
\pgfsys@useobject{currentmarker}{}%
\end{pgfscope}%
\end{pgfscope}%
\begin{pgfscope}%
\pgftext[left,bottom,x=0.683702in,y=1.251257in,rotate=0.000000]{{\sffamily\fontsize{12.000000}{14.400000}\selectfont −0.04}}
%
\end{pgfscope}%
\begin{pgfscope}%
\pgfpathrectangle{\pgfqpoint{1.250000in}{0.400000in}}{\pgfqpoint{7.750000in}{3.200000in}} %
\pgfusepath{clip}%
\pgfsetbuttcap%
\pgfsetroundjoin%
\pgfsetlinewidth{0.501875pt}%
\definecolor{currentstroke}{rgb}{0.000000,0.000000,0.000000}%
\pgfsetstrokecolor{currentstroke}%
\pgfsetdash{{1.000000pt}{3.000000pt}}{0.000000pt}%
\pgfpathmoveto{\pgfqpoint{1.250000in}{1.771429in}}%
\pgfpathlineto{\pgfqpoint{9.000000in}{1.771429in}}%
\pgfusepath{stroke}%
\end{pgfscope}%
\begin{pgfscope}%
\pgfsetbuttcap%
\pgfsetroundjoin%
\definecolor{currentfill}{rgb}{0.000000,0.000000,0.000000}%
\pgfsetfillcolor{currentfill}%
\pgfsetlinewidth{0.501875pt}%
\definecolor{currentstroke}{rgb}{0.000000,0.000000,0.000000}%
\pgfsetstrokecolor{currentstroke}%
\pgfsetdash{}{0pt}%
\pgfsys@defobject{currentmarker}{\pgfqpoint{0.000000in}{0.000000in}}{\pgfqpoint{0.055556in}{0.000000in}}{%
\pgfpathmoveto{\pgfqpoint{0.000000in}{0.000000in}}%
\pgfpathlineto{\pgfqpoint{0.055556in}{0.000000in}}%
\pgfusepath{stroke,fill}%
}%
\begin{pgfscope}%
\pgfsys@transformshift{1.250000in}{1.771429in}%
\pgfsys@useobject{currentmarker}{}%
\end{pgfscope}%
\end{pgfscope}%
\begin{pgfscope}%
\pgfsetbuttcap%
\pgfsetroundjoin%
\definecolor{currentfill}{rgb}{0.000000,0.000000,0.000000}%
\pgfsetfillcolor{currentfill}%
\pgfsetlinewidth{0.501875pt}%
\definecolor{currentstroke}{rgb}{0.000000,0.000000,0.000000}%
\pgfsetstrokecolor{currentstroke}%
\pgfsetdash{}{0pt}%
\pgfsys@defobject{currentmarker}{\pgfqpoint{-0.055556in}{0.000000in}}{\pgfqpoint{0.000000in}{0.000000in}}{%
\pgfpathmoveto{\pgfqpoint{0.000000in}{0.000000in}}%
\pgfpathlineto{\pgfqpoint{-0.055556in}{0.000000in}}%
\pgfusepath{stroke,fill}%
}%
\begin{pgfscope}%
\pgfsys@transformshift{9.000000in}{1.771429in}%
\pgfsys@useobject{currentmarker}{}%
\end{pgfscope}%
\end{pgfscope}%
\begin{pgfscope}%
\pgftext[left,bottom,x=0.683702in,y=1.708400in,rotate=0.000000]{{\sffamily\fontsize{12.000000}{14.400000}\selectfont −0.03}}
%
\end{pgfscope}%
\begin{pgfscope}%
\pgfpathrectangle{\pgfqpoint{1.250000in}{0.400000in}}{\pgfqpoint{7.750000in}{3.200000in}} %
\pgfusepath{clip}%
\pgfsetbuttcap%
\pgfsetroundjoin%
\pgfsetlinewidth{0.501875pt}%
\definecolor{currentstroke}{rgb}{0.000000,0.000000,0.000000}%
\pgfsetstrokecolor{currentstroke}%
\pgfsetdash{{1.000000pt}{3.000000pt}}{0.000000pt}%
\pgfpathmoveto{\pgfqpoint{1.250000in}{2.228571in}}%
\pgfpathlineto{\pgfqpoint{9.000000in}{2.228571in}}%
\pgfusepath{stroke}%
\end{pgfscope}%
\begin{pgfscope}%
\pgfsetbuttcap%
\pgfsetroundjoin%
\definecolor{currentfill}{rgb}{0.000000,0.000000,0.000000}%
\pgfsetfillcolor{currentfill}%
\pgfsetlinewidth{0.501875pt}%
\definecolor{currentstroke}{rgb}{0.000000,0.000000,0.000000}%
\pgfsetstrokecolor{currentstroke}%
\pgfsetdash{}{0pt}%
\pgfsys@defobject{currentmarker}{\pgfqpoint{0.000000in}{0.000000in}}{\pgfqpoint{0.055556in}{0.000000in}}{%
\pgfpathmoveto{\pgfqpoint{0.000000in}{0.000000in}}%
\pgfpathlineto{\pgfqpoint{0.055556in}{0.000000in}}%
\pgfusepath{stroke,fill}%
}%
\begin{pgfscope}%
\pgfsys@transformshift{1.250000in}{2.228571in}%
\pgfsys@useobject{currentmarker}{}%
\end{pgfscope}%
\end{pgfscope}%
\begin{pgfscope}%
\pgfsetbuttcap%
\pgfsetroundjoin%
\definecolor{currentfill}{rgb}{0.000000,0.000000,0.000000}%
\pgfsetfillcolor{currentfill}%
\pgfsetlinewidth{0.501875pt}%
\definecolor{currentstroke}{rgb}{0.000000,0.000000,0.000000}%
\pgfsetstrokecolor{currentstroke}%
\pgfsetdash{}{0pt}%
\pgfsys@defobject{currentmarker}{\pgfqpoint{-0.055556in}{0.000000in}}{\pgfqpoint{0.000000in}{0.000000in}}{%
\pgfpathmoveto{\pgfqpoint{0.000000in}{0.000000in}}%
\pgfpathlineto{\pgfqpoint{-0.055556in}{0.000000in}}%
\pgfusepath{stroke,fill}%
}%
\begin{pgfscope}%
\pgfsys@transformshift{9.000000in}{2.228571in}%
\pgfsys@useobject{currentmarker}{}%
\end{pgfscope}%
\end{pgfscope}%
\begin{pgfscope}%
\pgftext[left,bottom,x=0.683702in,y=2.165543in,rotate=0.000000]{{\sffamily\fontsize{12.000000}{14.400000}\selectfont −0.02}}
%
\end{pgfscope}%
\begin{pgfscope}%
\pgfpathrectangle{\pgfqpoint{1.250000in}{0.400000in}}{\pgfqpoint{7.750000in}{3.200000in}} %
\pgfusepath{clip}%
\pgfsetbuttcap%
\pgfsetroundjoin%
\pgfsetlinewidth{0.501875pt}%
\definecolor{currentstroke}{rgb}{0.000000,0.000000,0.000000}%
\pgfsetstrokecolor{currentstroke}%
\pgfsetdash{{1.000000pt}{3.000000pt}}{0.000000pt}%
\pgfpathmoveto{\pgfqpoint{1.250000in}{2.685714in}}%
\pgfpathlineto{\pgfqpoint{9.000000in}{2.685714in}}%
\pgfusepath{stroke}%
\end{pgfscope}%
\begin{pgfscope}%
\pgfsetbuttcap%
\pgfsetroundjoin%
\definecolor{currentfill}{rgb}{0.000000,0.000000,0.000000}%
\pgfsetfillcolor{currentfill}%
\pgfsetlinewidth{0.501875pt}%
\definecolor{currentstroke}{rgb}{0.000000,0.000000,0.000000}%
\pgfsetstrokecolor{currentstroke}%
\pgfsetdash{}{0pt}%
\pgfsys@defobject{currentmarker}{\pgfqpoint{0.000000in}{0.000000in}}{\pgfqpoint{0.055556in}{0.000000in}}{%
\pgfpathmoveto{\pgfqpoint{0.000000in}{0.000000in}}%
\pgfpathlineto{\pgfqpoint{0.055556in}{0.000000in}}%
\pgfusepath{stroke,fill}%
}%
\begin{pgfscope}%
\pgfsys@transformshift{1.250000in}{2.685714in}%
\pgfsys@useobject{currentmarker}{}%
\end{pgfscope}%
\end{pgfscope}%
\begin{pgfscope}%
\pgfsetbuttcap%
\pgfsetroundjoin%
\definecolor{currentfill}{rgb}{0.000000,0.000000,0.000000}%
\pgfsetfillcolor{currentfill}%
\pgfsetlinewidth{0.501875pt}%
\definecolor{currentstroke}{rgb}{0.000000,0.000000,0.000000}%
\pgfsetstrokecolor{currentstroke}%
\pgfsetdash{}{0pt}%
\pgfsys@defobject{currentmarker}{\pgfqpoint{-0.055556in}{0.000000in}}{\pgfqpoint{0.000000in}{0.000000in}}{%
\pgfpathmoveto{\pgfqpoint{0.000000in}{0.000000in}}%
\pgfpathlineto{\pgfqpoint{-0.055556in}{0.000000in}}%
\pgfusepath{stroke,fill}%
}%
\begin{pgfscope}%
\pgfsys@transformshift{9.000000in}{2.685714in}%
\pgfsys@useobject{currentmarker}{}%
\end{pgfscope}%
\end{pgfscope}%
\begin{pgfscope}%
\pgftext[left,bottom,x=0.683702in,y=2.622685in,rotate=0.000000]{{\sffamily\fontsize{12.000000}{14.400000}\selectfont −0.01}}
%
\end{pgfscope}%
\begin{pgfscope}%
\pgfpathrectangle{\pgfqpoint{1.250000in}{0.400000in}}{\pgfqpoint{7.750000in}{3.200000in}} %
\pgfusepath{clip}%
\pgfsetbuttcap%
\pgfsetroundjoin%
\pgfsetlinewidth{0.501875pt}%
\definecolor{currentstroke}{rgb}{0.000000,0.000000,0.000000}%
\pgfsetstrokecolor{currentstroke}%
\pgfsetdash{{1.000000pt}{3.000000pt}}{0.000000pt}%
\pgfpathmoveto{\pgfqpoint{1.250000in}{3.142857in}}%
\pgfpathlineto{\pgfqpoint{9.000000in}{3.142857in}}%
\pgfusepath{stroke}%
\end{pgfscope}%
\begin{pgfscope}%
\pgfsetbuttcap%
\pgfsetroundjoin%
\definecolor{currentfill}{rgb}{0.000000,0.000000,0.000000}%
\pgfsetfillcolor{currentfill}%
\pgfsetlinewidth{0.501875pt}%
\definecolor{currentstroke}{rgb}{0.000000,0.000000,0.000000}%
\pgfsetstrokecolor{currentstroke}%
\pgfsetdash{}{0pt}%
\pgfsys@defobject{currentmarker}{\pgfqpoint{0.000000in}{0.000000in}}{\pgfqpoint{0.055556in}{0.000000in}}{%
\pgfpathmoveto{\pgfqpoint{0.000000in}{0.000000in}}%
\pgfpathlineto{\pgfqpoint{0.055556in}{0.000000in}}%
\pgfusepath{stroke,fill}%
}%
\begin{pgfscope}%
\pgfsys@transformshift{1.250000in}{3.142857in}%
\pgfsys@useobject{currentmarker}{}%
\end{pgfscope}%
\end{pgfscope}%
\begin{pgfscope}%
\pgfsetbuttcap%
\pgfsetroundjoin%
\definecolor{currentfill}{rgb}{0.000000,0.000000,0.000000}%
\pgfsetfillcolor{currentfill}%
\pgfsetlinewidth{0.501875pt}%
\definecolor{currentstroke}{rgb}{0.000000,0.000000,0.000000}%
\pgfsetstrokecolor{currentstroke}%
\pgfsetdash{}{0pt}%
\pgfsys@defobject{currentmarker}{\pgfqpoint{-0.055556in}{0.000000in}}{\pgfqpoint{0.000000in}{0.000000in}}{%
\pgfpathmoveto{\pgfqpoint{0.000000in}{0.000000in}}%
\pgfpathlineto{\pgfqpoint{-0.055556in}{0.000000in}}%
\pgfusepath{stroke,fill}%
}%
\begin{pgfscope}%
\pgfsys@transformshift{9.000000in}{3.142857in}%
\pgfsys@useobject{currentmarker}{}%
\end{pgfscope}%
\end{pgfscope}%
\begin{pgfscope}%
\pgftext[left,bottom,x=0.823351in,y=3.079828in,rotate=0.000000]{{\sffamily\fontsize{12.000000}{14.400000}\selectfont 0.00}}
%
\end{pgfscope}%
\begin{pgfscope}%
\pgfpathrectangle{\pgfqpoint{1.250000in}{0.400000in}}{\pgfqpoint{7.750000in}{3.200000in}} %
\pgfusepath{clip}%
\pgfsetbuttcap%
\pgfsetroundjoin%
\pgfsetlinewidth{0.501875pt}%
\definecolor{currentstroke}{rgb}{0.000000,0.000000,0.000000}%
\pgfsetstrokecolor{currentstroke}%
\pgfsetdash{{1.000000pt}{3.000000pt}}{0.000000pt}%
\pgfpathmoveto{\pgfqpoint{1.250000in}{3.600000in}}%
\pgfpathlineto{\pgfqpoint{9.000000in}{3.600000in}}%
\pgfusepath{stroke}%
\end{pgfscope}%
\begin{pgfscope}%
\pgfsetbuttcap%
\pgfsetroundjoin%
\definecolor{currentfill}{rgb}{0.000000,0.000000,0.000000}%
\pgfsetfillcolor{currentfill}%
\pgfsetlinewidth{0.501875pt}%
\definecolor{currentstroke}{rgb}{0.000000,0.000000,0.000000}%
\pgfsetstrokecolor{currentstroke}%
\pgfsetdash{}{0pt}%
\pgfsys@defobject{currentmarker}{\pgfqpoint{0.000000in}{0.000000in}}{\pgfqpoint{0.055556in}{0.000000in}}{%
\pgfpathmoveto{\pgfqpoint{0.000000in}{0.000000in}}%
\pgfpathlineto{\pgfqpoint{0.055556in}{0.000000in}}%
\pgfusepath{stroke,fill}%
}%
\begin{pgfscope}%
\pgfsys@transformshift{1.250000in}{3.600000in}%
\pgfsys@useobject{currentmarker}{}%
\end{pgfscope}%
\end{pgfscope}%
\begin{pgfscope}%
\pgfsetbuttcap%
\pgfsetroundjoin%
\definecolor{currentfill}{rgb}{0.000000,0.000000,0.000000}%
\pgfsetfillcolor{currentfill}%
\pgfsetlinewidth{0.501875pt}%
\definecolor{currentstroke}{rgb}{0.000000,0.000000,0.000000}%
\pgfsetstrokecolor{currentstroke}%
\pgfsetdash{}{0pt}%
\pgfsys@defobject{currentmarker}{\pgfqpoint{-0.055556in}{0.000000in}}{\pgfqpoint{0.000000in}{0.000000in}}{%
\pgfpathmoveto{\pgfqpoint{0.000000in}{0.000000in}}%
\pgfpathlineto{\pgfqpoint{-0.055556in}{0.000000in}}%
\pgfusepath{stroke,fill}%
}%
\begin{pgfscope}%
\pgfsys@transformshift{9.000000in}{3.600000in}%
\pgfsys@useobject{currentmarker}{}%
\end{pgfscope}%
\end{pgfscope}%
\begin{pgfscope}%
\pgftext[left,bottom,x=0.823351in,y=3.536971in,rotate=0.000000]{{\sffamily\fontsize{12.000000}{14.400000}\selectfont 0.01}}
%
\end{pgfscope}%
\begin{pgfscope}%
\pgfsetrectcap%
\pgfsetroundjoin%
\pgfsetlinewidth{1.003750pt}%
\definecolor{currentstroke}{rgb}{0.000000,0.000000,0.000000}%
\pgfsetstrokecolor{currentstroke}%
\pgfsetdash{}{0pt}%
\pgfpathmoveto{\pgfqpoint{1.250000in}{3.600000in}}%
\pgfpathlineto{\pgfqpoint{9.000000in}{3.600000in}}%
\pgfusepath{stroke}%
\end{pgfscope}%
\begin{pgfscope}%
\pgfsetrectcap%
\pgfsetroundjoin%
\pgfsetlinewidth{1.003750pt}%
\definecolor{currentstroke}{rgb}{0.000000,0.000000,0.000000}%
\pgfsetstrokecolor{currentstroke}%
\pgfsetdash{}{0pt}%
\pgfpathmoveto{\pgfqpoint{9.000000in}{0.400000in}}%
\pgfpathlineto{\pgfqpoint{9.000000in}{3.600000in}}%
\pgfusepath{stroke}%
\end{pgfscope}%
\begin{pgfscope}%
\pgfsetrectcap%
\pgfsetroundjoin%
\pgfsetlinewidth{1.003750pt}%
\definecolor{currentstroke}{rgb}{0.000000,0.000000,0.000000}%
\pgfsetstrokecolor{currentstroke}%
\pgfsetdash{}{0pt}%
\pgfpathmoveto{\pgfqpoint{1.250000in}{0.400000in}}%
\pgfpathlineto{\pgfqpoint{9.000000in}{0.400000in}}%
\pgfusepath{stroke}%
\end{pgfscope}%
\begin{pgfscope}%
\pgfsetrectcap%
\pgfsetroundjoin%
\pgfsetlinewidth{1.003750pt}%
\definecolor{currentstroke}{rgb}{0.000000,0.000000,0.000000}%
\pgfsetstrokecolor{currentstroke}%
\pgfsetdash{}{0pt}%
\pgfpathmoveto{\pgfqpoint{1.250000in}{0.400000in}}%
\pgfpathlineto{\pgfqpoint{1.250000in}{3.600000in}}%
\pgfusepath{stroke}%
\end{pgfscope}%
\begin{pgfscope}%
\pgftext[left,bottom,x=3.270215in,y=3.646202in,rotate=0.000000]{{\sffamily\fontsize{14.400000}{17.280000}\selectfont moon in rest frame of earth, dt=0.1a}}
%
\end{pgfscope}%
\begin{pgfscope}%
\pgfsetrectcap%
\pgfsetroundjoin%
\definecolor{currentfill}{rgb}{1.000000,1.000000,1.000000}%
\pgfsetfillcolor{currentfill}%
\pgfsetlinewidth{1.003750pt}%
\definecolor{currentstroke}{rgb}{0.000000,0.000000,0.000000}%
\pgfsetstrokecolor{currentstroke}%
\pgfsetdash{}{0pt}%
\pgfpathmoveto{\pgfqpoint{1.319417in}{2.877606in}}%
\pgfpathlineto{\pgfqpoint{2.965338in}{2.877606in}}%
\pgfpathlineto{\pgfqpoint{2.965338in}{3.530583in}}%
\pgfpathlineto{\pgfqpoint{1.319417in}{3.530583in}}%
\pgfpathlineto{\pgfqpoint{1.319417in}{2.877606in}}%
\pgfpathclose%
\pgfusepath{stroke,fill}%
\end{pgfscope}%
\begin{pgfscope}%
\pgfsetrectcap%
\pgfsetroundjoin%
\pgfsetlinewidth{1.003750pt}%
\definecolor{currentstroke}{rgb}{0.000000,0.000000,1.000000}%
\pgfsetstrokecolor{currentstroke}%
\pgfsetdash{}{0pt}%
\pgfpathmoveto{\pgfqpoint{1.416600in}{3.418161in}}%
\pgfpathlineto{\pgfqpoint{1.610967in}{3.418161in}}%
\pgfusepath{stroke}%
\end{pgfscope}%
\begin{pgfscope}%
\pgftext[left,bottom,x=1.763683in,y=3.367603in,rotate=0.000000]{{\sffamily\fontsize{9.996000}{11.995200}\selectfont euler}}
%
\end{pgfscope}%
\begin{pgfscope}%
\pgfsetrectcap%
\pgfsetroundjoin%
\pgfsetlinewidth{1.003750pt}%
\definecolor{currentstroke}{rgb}{0.000000,0.500000,0.000000}%
\pgfsetstrokecolor{currentstroke}%
\pgfsetdash{}{0pt}%
\pgfpathmoveto{\pgfqpoint{1.416600in}{3.214385in}}%
\pgfpathlineto{\pgfqpoint{1.610967in}{3.214385in}}%
\pgfusepath{stroke}%
\end{pgfscope}%
\begin{pgfscope}%
\pgftext[left,bottom,x=1.763683in,y=3.136915in,rotate=0.000000]{{\sffamily\fontsize{9.996000}{11.995200}\selectfont symplectic euler}}
%
\end{pgfscope}%
\begin{pgfscope}%
\pgfsetrectcap%
\pgfsetroundjoin%
\pgfsetlinewidth{1.003750pt}%
\definecolor{currentstroke}{rgb}{1.000000,0.000000,0.000000}%
\pgfsetstrokecolor{currentstroke}%
\pgfsetdash{}{0pt}%
\pgfpathmoveto{\pgfqpoint{1.416600in}{3.010609in}}%
\pgfpathlineto{\pgfqpoint{1.610967in}{3.010609in}}%
\pgfusepath{stroke}%
\end{pgfscope}%
\begin{pgfscope}%
\pgftext[left,bottom,x=1.763683in,y=2.933139in,rotate=0.000000]{{\sffamily\fontsize{9.996000}{11.995200}\selectfont velocity verlet}}
%
\end{pgfscope}%
\end{pgfpicture}%
\makeatother%
\endgroup%
}
	\caption{Simulated trajectory of the moon in the rest frame of the earth using different algorithms}\label{fig:s3}
\end{figure}

Obviously the simple Euler integration scheme ist the poorest of the three compared algorithms. After a short time it diverges from the expected trajectory. In contrast the trajectories compluted by the symplectic Euler and the Velocity Verlet algorithm follow more or less the expected ellipse. For this simulation the Velocity Verlet seems to be the best of the three algorithms.

\subsection{Long-term stability}
This program is a simple modification of the one used in the part before. The only difference is that the simulated time is increased from one to ten years.

\begin{figure}[H]
	\resizebox{1\textwidth}{!}{%% Creator: Matplotlib, PGF backend
%%
%% To include the figure in your LaTeX document, write
%%   \input{<filename>.pgf}
%%
%% Make sure the required packages are loaded in your preamble
%%   \usepackage{pgf}
%%
%% Figures using additional raster images can only be included by \input if
%% they are in the same directory as the main LaTeX file. For loading figures
%% from other directories you can use the `import` package
%%   \usepackage{import}
%% and then include the figures with
%%   \import{<path to file>}{<filename>.pgf}
%%
%% Matplotlib used the following preamble
%%   \usepackage{fontspec}
%%   \setmainfont{DejaVu Serif}
%%   \setsansfont{DejaVu Sans}
%%   \setmonofont{DejaVu Sans Mono}
%%
\begingroup%
\makeatletter%
\begin{pgfpicture}%
\pgfpathrectangle{\pgfpointorigin}{\pgfqpoint{10.000000in}{4.000000in}}%
\pgfusepath{use as bounding box}%
\begin{pgfscope}%
\pgfsetrectcap%
\pgfsetroundjoin%
\definecolor{currentfill}{rgb}{1.000000,1.000000,1.000000}%
\pgfsetfillcolor{currentfill}%
\pgfsetlinewidth{0.000000pt}%
\definecolor{currentstroke}{rgb}{1.000000,1.000000,1.000000}%
\pgfsetstrokecolor{currentstroke}%
\pgfsetdash{}{0pt}%
\pgfpathmoveto{\pgfqpoint{0.000000in}{0.000000in}}%
\pgfpathlineto{\pgfqpoint{10.000000in}{0.000000in}}%
\pgfpathlineto{\pgfqpoint{10.000000in}{4.000000in}}%
\pgfpathlineto{\pgfqpoint{0.000000in}{4.000000in}}%
\pgfpathclose%
\pgfusepath{fill}%
\end{pgfscope}%
\begin{pgfscope}%
\pgfsetrectcap%
\pgfsetroundjoin%
\definecolor{currentfill}{rgb}{1.000000,1.000000,1.000000}%
\pgfsetfillcolor{currentfill}%
\pgfsetlinewidth{0.000000pt}%
\definecolor{currentstroke}{rgb}{0.000000,0.000000,0.000000}%
\pgfsetstrokecolor{currentstroke}%
\pgfsetdash{}{0pt}%
\pgfpathmoveto{\pgfqpoint{1.250000in}{0.400000in}}%
\pgfpathlineto{\pgfqpoint{9.000000in}{0.400000in}}%
\pgfpathlineto{\pgfqpoint{9.000000in}{3.600000in}}%
\pgfpathlineto{\pgfqpoint{1.250000in}{3.600000in}}%
\pgfpathclose%
\pgfusepath{fill}%
\end{pgfscope}%
\begin{pgfscope}%
\pgfpathrectangle{\pgfqpoint{1.250000in}{0.400000in}}{\pgfqpoint{7.750000in}{3.200000in}} %
\pgfusepath{clip}%
\pgfsetrectcap%
\pgfsetroundjoin%
\pgfsetlinewidth{1.003750pt}%
\definecolor{currentstroke}{rgb}{0.000000,0.000000,1.000000}%
\pgfsetstrokecolor{currentstroke}%
\pgfsetdash{}{0pt}%
\pgfpathmoveto{\pgfqpoint{5.129965in}{2.000000in}}%
\pgfpathlineto{\pgfqpoint{5.129965in}{2.001697in}}%
\pgfpathlineto{\pgfqpoint{5.126468in}{2.003395in}}%
\pgfpathlineto{\pgfqpoint{5.116127in}{2.005155in}}%
\pgfpathlineto{\pgfqpoint{5.092085in}{2.005726in}}%
\pgfpathlineto{\pgfqpoint{5.068802in}{2.004092in}}%
\pgfpathlineto{\pgfqpoint{5.050564in}{2.000888in}}%
\pgfpathlineto{\pgfqpoint{5.034188in}{1.995822in}}%
\pgfpathlineto{\pgfqpoint{5.021419in}{1.989158in}}%
\pgfpathlineto{\pgfqpoint{5.015035in}{1.983490in}}%
\pgfpathlineto{\pgfqpoint{5.011697in}{1.977634in}}%
\pgfpathlineto{\pgfqpoint{5.011360in}{1.971953in}}%
\pgfpathlineto{\pgfqpoint{5.013717in}{1.966776in}}%
\pgfpathlineto{\pgfqpoint{5.018257in}{1.962360in}}%
\pgfpathlineto{\pgfqpoint{5.026617in}{1.957949in}}%
\pgfpathlineto{\pgfqpoint{5.036163in}{1.955378in}}%
\pgfpathlineto{\pgfqpoint{5.047542in}{1.954661in}}%
\pgfpathlineto{\pgfqpoint{5.056196in}{1.956283in}}%
\pgfpathlineto{\pgfqpoint{5.060113in}{1.958815in}}%
\pgfpathlineto{\pgfqpoint{5.060997in}{1.961181in}}%
\pgfpathlineto{\pgfqpoint{5.060032in}{1.963769in}}%
\pgfpathlineto{\pgfqpoint{5.055881in}{1.967296in}}%
\pgfpathlineto{\pgfqpoint{5.046369in}{1.971364in}}%
\pgfpathlineto{\pgfqpoint{5.032683in}{1.974532in}}%
\pgfpathlineto{\pgfqpoint{5.015718in}{1.976353in}}%
\pgfpathlineto{\pgfqpoint{4.996501in}{1.976502in}}%
\pgfpathlineto{\pgfqpoint{4.976128in}{1.974772in}}%
\pgfpathlineto{\pgfqpoint{4.955719in}{1.971069in}}%
\pgfpathlineto{\pgfqpoint{4.936381in}{1.965395in}}%
\pgfpathlineto{\pgfqpoint{4.919180in}{1.957845in}}%
\pgfpathlineto{\pgfqpoint{4.907630in}{1.950568in}}%
\pgfpathlineto{\pgfqpoint{4.898558in}{1.942324in}}%
\pgfpathlineto{\pgfqpoint{4.892388in}{1.933264in}}%
\pgfpathlineto{\pgfqpoint{4.889892in}{1.926040in}}%
\pgfpathlineto{\pgfqpoint{4.889376in}{1.918543in}}%
\pgfpathlineto{\pgfqpoint{4.890955in}{1.910865in}}%
\pgfpathlineto{\pgfqpoint{4.894718in}{1.903104in}}%
\pgfpathlineto{\pgfqpoint{4.900732in}{1.895367in}}%
\pgfpathlineto{\pgfqpoint{4.909033in}{1.887763in}}%
\pgfpathlineto{\pgfqpoint{4.923665in}{1.878032in}}%
\pgfpathlineto{\pgfqpoint{4.942288in}{1.869025in}}%
\pgfpathlineto{\pgfqpoint{4.964705in}{1.861027in}}%
\pgfpathlineto{\pgfqpoint{4.990589in}{1.854317in}}%
\pgfpathlineto{\pgfqpoint{5.019481in}{1.849162in}}%
\pgfpathlineto{\pgfqpoint{5.050783in}{1.845802in}}%
\pgfpathlineto{\pgfqpoint{5.083764in}{1.844436in}}%
\pgfpathlineto{\pgfqpoint{5.117576in}{1.845209in}}%
\pgfpathlineto{\pgfqpoint{5.151272in}{1.848196in}}%
\pgfpathlineto{\pgfqpoint{5.183848in}{1.853390in}}%
\pgfpathlineto{\pgfqpoint{5.214292in}{1.860692in}}%
\pgfpathlineto{\pgfqpoint{5.241643in}{1.869907in}}%
\pgfpathlineto{\pgfqpoint{5.259615in}{1.877904in}}%
\pgfpathlineto{\pgfqpoint{5.275080in}{1.886667in}}%
\pgfpathlineto{\pgfqpoint{5.287822in}{1.896026in}}%
\pgfpathlineto{\pgfqpoint{5.297704in}{1.905799in}}%
\pgfpathlineto{\pgfqpoint{5.304680in}{1.915801in}}%
\pgfpathlineto{\pgfqpoint{5.308788in}{1.925850in}}%
\pgfpathlineto{\pgfqpoint{5.310151in}{1.935778in}}%
\pgfpathlineto{\pgfqpoint{5.308961in}{1.945436in}}%
\pgfpathlineto{\pgfqpoint{5.305468in}{1.954701in}}%
\pgfpathlineto{\pgfqpoint{5.299956in}{1.963482in}}%
\pgfpathlineto{\pgfqpoint{5.289992in}{1.974332in}}%
\pgfpathlineto{\pgfqpoint{5.277685in}{1.984148in}}%
\pgfpathlineto{\pgfqpoint{5.259969in}{1.994993in}}%
\pgfpathlineto{\pgfqpoint{5.236543in}{2.006139in}}%
\pgfpathlineto{\pgfqpoint{5.207467in}{2.017079in}}%
\pgfpathlineto{\pgfqpoint{5.172527in}{2.027609in}}%
\pgfpathlineto{\pgfqpoint{5.130666in}{2.037639in}}%
\pgfpathlineto{\pgfqpoint{5.085178in}{2.046122in}}%
\pgfpathlineto{\pgfqpoint{5.034932in}{2.053037in}}%
\pgfpathlineto{\pgfqpoint{4.985603in}{2.057525in}}%
\pgfpathlineto{\pgfqpoint{4.931735in}{2.059974in}}%
\pgfpathlineto{\pgfqpoint{4.881115in}{2.060040in}}%
\pgfpathlineto{\pgfqpoint{4.827708in}{2.057822in}}%
\pgfpathlineto{\pgfqpoint{4.772152in}{2.053017in}}%
\pgfpathlineto{\pgfqpoint{4.723413in}{2.046643in}}%
\pgfpathlineto{\pgfqpoint{4.674232in}{2.038044in}}%
\pgfpathlineto{\pgfqpoint{4.625224in}{2.027114in}}%
\pgfpathlineto{\pgfqpoint{4.577043in}{2.013774in}}%
\pgfpathlineto{\pgfqpoint{4.538019in}{2.000780in}}%
\pgfpathlineto{\pgfqpoint{4.500460in}{1.986064in}}%
\pgfpathlineto{\pgfqpoint{4.464792in}{1.969627in}}%
\pgfpathlineto{\pgfqpoint{4.431447in}{1.951484in}}%
\pgfpathlineto{\pgfqpoint{4.406743in}{1.935762in}}%
\pgfpathlineto{\pgfqpoint{4.384040in}{1.918989in}}%
\pgfpathlineto{\pgfqpoint{4.363572in}{1.901195in}}%
\pgfpathlineto{\pgfqpoint{4.345573in}{1.882416in}}%
\pgfpathlineto{\pgfqpoint{4.330281in}{1.862693in}}%
\pgfpathlineto{\pgfqpoint{4.320731in}{1.847313in}}%
\pgfpathlineto{\pgfqpoint{4.312939in}{1.831454in}}%
\pgfpathlineto{\pgfqpoint{4.307004in}{1.815143in}}%
\pgfpathlineto{\pgfqpoint{4.303027in}{1.798410in}}%
\pgfpathlineto{\pgfqpoint{4.301109in}{1.781286in}}%
\pgfpathlineto{\pgfqpoint{4.301349in}{1.763808in}}%
\pgfpathlineto{\pgfqpoint{4.303845in}{1.746012in}}%
\pgfpathlineto{\pgfqpoint{4.308694in}{1.727940in}}%
\pgfpathlineto{\pgfqpoint{4.315993in}{1.709636in}}%
\pgfpathlineto{\pgfqpoint{4.325833in}{1.691150in}}%
\pgfpathlineto{\pgfqpoint{4.338305in}{1.672533in}}%
\pgfpathlineto{\pgfqpoint{4.353497in}{1.653841in}}%
\pgfpathlineto{\pgfqpoint{4.371491in}{1.635133in}}%
\pgfpathlineto{\pgfqpoint{4.392366in}{1.616473in}}%
\pgfpathlineto{\pgfqpoint{4.416194in}{1.597930in}}%
\pgfpathlineto{\pgfqpoint{4.443042in}{1.579577in}}%
\pgfpathlineto{\pgfqpoint{4.472968in}{1.561489in}}%
\pgfpathlineto{\pgfqpoint{4.506021in}{1.543751in}}%
\pgfpathlineto{\pgfqpoint{4.542240in}{1.526447in}}%
\pgfpathlineto{\pgfqpoint{4.581653in}{1.509669in}}%
\pgfpathlineto{\pgfqpoint{4.624275in}{1.493513in}}%
\pgfpathlineto{\pgfqpoint{4.670105in}{1.478080in}}%
\pgfpathlineto{\pgfqpoint{4.719124in}{1.463473in}}%
\pgfpathlineto{\pgfqpoint{4.771297in}{1.449803in}}%
\pgfpathlineto{\pgfqpoint{4.826566in}{1.437181in}}%
\pgfpathlineto{\pgfqpoint{4.884851in}{1.425723in}}%
\pgfpathlineto{\pgfqpoint{4.946045in}{1.415548in}}%
\pgfpathlineto{\pgfqpoint{5.010016in}{1.406775in}}%
\pgfpathlineto{\pgfqpoint{5.076601in}{1.399526in}}%
\pgfpathlineto{\pgfqpoint{5.145604in}{1.393920in}}%
\pgfpathlineto{\pgfqpoint{5.216797in}{1.390075in}}%
\pgfpathlineto{\pgfqpoint{5.289916in}{1.388107in}}%
\pgfpathlineto{\pgfqpoint{5.364660in}{1.388126in}}%
\pgfpathlineto{\pgfqpoint{5.440690in}{1.390232in}}%
\pgfpathlineto{\pgfqpoint{5.517633in}{1.394520in}}%
\pgfpathlineto{\pgfqpoint{5.595074in}{1.401070in}}%
\pgfpathlineto{\pgfqpoint{5.672568in}{1.409948in}}%
\pgfpathlineto{\pgfqpoint{5.749636in}{1.421204in}}%
\pgfpathlineto{\pgfqpoint{5.825769in}{1.434865in}}%
\pgfpathlineto{\pgfqpoint{5.900436in}{1.450942in}}%
\pgfpathlineto{\pgfqpoint{5.973091in}{1.469415in}}%
\pgfpathlineto{\pgfqpoint{6.043177in}{1.490243in}}%
\pgfpathlineto{\pgfqpoint{6.110137in}{1.513354in}}%
\pgfpathlineto{\pgfqpoint{6.152769in}{1.529982in}}%
\pgfpathlineto{\pgfqpoint{6.193610in}{1.547546in}}%
\pgfpathlineto{\pgfqpoint{6.232509in}{1.566003in}}%
\pgfpathlineto{\pgfqpoint{6.269321in}{1.585305in}}%
\pgfpathlineto{\pgfqpoint{6.303907in}{1.605401in}}%
\pgfpathlineto{\pgfqpoint{6.336142in}{1.626232in}}%
\pgfpathlineto{\pgfqpoint{6.365908in}{1.647737in}}%
\pgfpathlineto{\pgfqpoint{6.393103in}{1.669849in}}%
\pgfpathlineto{\pgfqpoint{6.417635in}{1.692500in}}%
\pgfpathlineto{\pgfqpoint{6.439428in}{1.715616in}}%
\pgfpathlineto{\pgfqpoint{6.458422in}{1.739122in}}%
\pgfpathlineto{\pgfqpoint{6.474569in}{1.762943in}}%
\pgfpathlineto{\pgfqpoint{6.487842in}{1.787001in}}%
\pgfpathlineto{\pgfqpoint{6.498226in}{1.811218in}}%
\pgfpathlineto{\pgfqpoint{6.505723in}{1.835517in}}%
\pgfpathlineto{\pgfqpoint{6.510351in}{1.859822in}}%
\pgfpathlineto{\pgfqpoint{6.512144in}{1.884059in}}%
\pgfpathlineto{\pgfqpoint{6.511148in}{1.908156in}}%
\pgfpathlineto{\pgfqpoint{6.507423in}{1.932046in}}%
\pgfpathlineto{\pgfqpoint{6.501043in}{1.955663in}}%
\pgfpathlineto{\pgfqpoint{6.492092in}{1.978945in}}%
\pgfpathlineto{\pgfqpoint{6.480664in}{2.001838in}}%
\pgfpathlineto{\pgfqpoint{6.466859in}{2.024288in}}%
\pgfpathlineto{\pgfqpoint{6.450788in}{2.046249in}}%
\pgfpathlineto{\pgfqpoint{6.432565in}{2.067679in}}%
\pgfpathlineto{\pgfqpoint{6.412309in}{2.088541in}}%
\pgfpathlineto{\pgfqpoint{6.390141in}{2.108803in}}%
\pgfpathlineto{\pgfqpoint{6.366184in}{2.128437in}}%
\pgfpathlineto{\pgfqpoint{6.340561in}{2.147422in}}%
\pgfpathlineto{\pgfqpoint{6.299273in}{2.174644in}}%
\pgfpathlineto{\pgfqpoint{6.254917in}{2.200323in}}%
\pgfpathlineto{\pgfqpoint{6.207885in}{2.224438in}}%
\pgfpathlineto{\pgfqpoint{6.158545in}{2.246983in}}%
\pgfpathlineto{\pgfqpoint{6.107244in}{2.267968in}}%
\pgfpathlineto{\pgfqpoint{6.054302in}{2.287416in}}%
\pgfpathlineto{\pgfqpoint{6.000009in}{2.305358in}}%
\pgfpathlineto{\pgfqpoint{5.925972in}{2.327010in}}%
\pgfpathlineto{\pgfqpoint{5.850551in}{2.346167in}}%
\pgfpathlineto{\pgfqpoint{5.774213in}{2.362953in}}%
\pgfpathlineto{\pgfqpoint{5.697349in}{2.377492in}}%
\pgfpathlineto{\pgfqpoint{5.620282in}{2.389913in}}%
\pgfpathlineto{\pgfqpoint{5.543273in}{2.400339in}}%
\pgfpathlineto{\pgfqpoint{5.447414in}{2.410745in}}%
\pgfpathlineto{\pgfqpoint{5.352296in}{2.418433in}}%
\pgfpathlineto{\pgfqpoint{5.258156in}{2.423596in}}%
\pgfpathlineto{\pgfqpoint{5.165168in}{2.426406in}}%
\pgfpathlineto{\pgfqpoint{5.073458in}{2.427014in}}%
\pgfpathlineto{\pgfqpoint{4.983120in}{2.425549in}}%
\pgfpathlineto{\pgfqpoint{4.894222in}{2.422125in}}%
\pgfpathlineto{\pgfqpoint{4.806820in}{2.416837in}}%
\pgfpathlineto{\pgfqpoint{4.720960in}{2.409763in}}%
\pgfpathlineto{\pgfqpoint{4.620025in}{2.399015in}}%
\pgfpathlineto{\pgfqpoint{4.521451in}{2.385891in}}%
\pgfpathlineto{\pgfqpoint{4.425322in}{2.370470in}}%
\pgfpathlineto{\pgfqpoint{4.331735in}{2.352815in}}%
\pgfpathlineto{\pgfqpoint{4.240802in}{2.332979in}}%
\pgfpathlineto{\pgfqpoint{4.167145in}{2.314812in}}%
\pgfpathlineto{\pgfqpoint{4.095505in}{2.295182in}}%
\pgfpathlineto{\pgfqpoint{4.025975in}{2.274106in}}%
\pgfpathlineto{\pgfqpoint{3.958657in}{2.251602in}}%
\pgfpathlineto{\pgfqpoint{3.893663in}{2.227685in}}%
\pgfpathlineto{\pgfqpoint{3.831113in}{2.202372in}}%
\pgfpathlineto{\pgfqpoint{3.771136in}{2.175678in}}%
\pgfpathlineto{\pgfqpoint{3.713870in}{2.147619in}}%
\pgfpathlineto{\pgfqpoint{3.659463in}{2.118213in}}%
\pgfpathlineto{\pgfqpoint{3.608069in}{2.087478in}}%
\pgfpathlineto{\pgfqpoint{3.569234in}{2.061948in}}%
\pgfpathlineto{\pgfqpoint{3.532521in}{2.035591in}}%
\pgfpathlineto{\pgfqpoint{3.498022in}{2.008422in}}%
\pgfpathlineto{\pgfqpoint{3.465833in}{1.980454in}}%
\pgfpathlineto{\pgfqpoint{3.436052in}{1.951703in}}%
\pgfpathlineto{\pgfqpoint{3.408782in}{1.922184in}}%
\pgfpathlineto{\pgfqpoint{3.384128in}{1.891916in}}%
\pgfpathlineto{\pgfqpoint{3.362201in}{1.860920in}}%
\pgfpathlineto{\pgfqpoint{3.343114in}{1.829217in}}%
\pgfpathlineto{\pgfqpoint{3.330732in}{1.804992in}}%
\pgfpathlineto{\pgfqpoint{3.320063in}{1.780394in}}%
\pgfpathlineto{\pgfqpoint{3.311158in}{1.755435in}}%
\pgfpathlineto{\pgfqpoint{3.304072in}{1.730129in}}%
\pgfpathlineto{\pgfqpoint{3.298856in}{1.704488in}}%
\pgfpathlineto{\pgfqpoint{3.295566in}{1.678528in}}%
\pgfpathlineto{\pgfqpoint{3.294258in}{1.652265in}}%
\pgfpathlineto{\pgfqpoint{3.294988in}{1.625715in}}%
\pgfpathlineto{\pgfqpoint{3.297814in}{1.598896in}}%
\pgfpathlineto{\pgfqpoint{3.302794in}{1.571827in}}%
\pgfpathlineto{\pgfqpoint{3.309988in}{1.544528in}}%
\pgfpathlineto{\pgfqpoint{3.319456in}{1.517022in}}%
\pgfpathlineto{\pgfqpoint{3.331258in}{1.489332in}}%
\pgfpathlineto{\pgfqpoint{3.345455in}{1.461481in}}%
\pgfpathlineto{\pgfqpoint{3.362110in}{1.433498in}}%
\pgfpathlineto{\pgfqpoint{3.381284in}{1.405409in}}%
\pgfpathlineto{\pgfqpoint{3.403040in}{1.377245in}}%
\pgfpathlineto{\pgfqpoint{3.427439in}{1.349037in}}%
\pgfpathlineto{\pgfqpoint{3.454544in}{1.320819in}}%
\pgfpathlineto{\pgfqpoint{3.484416in}{1.292626in}}%
\pgfpathlineto{\pgfqpoint{3.517117in}{1.264497in}}%
\pgfpathlineto{\pgfqpoint{3.552708in}{1.236471in}}%
\pgfpathlineto{\pgfqpoint{3.591249in}{1.208592in}}%
\pgfpathlineto{\pgfqpoint{3.632799in}{1.180903in}}%
\pgfpathlineto{\pgfqpoint{3.677416in}{1.153453in}}%
\pgfpathlineto{\pgfqpoint{3.725157in}{1.126295in}}%
\pgfpathlineto{\pgfqpoint{3.776073in}{1.099481in}}%
\pgfpathlineto{\pgfqpoint{3.830215in}{1.073069in}}%
\pgfpathlineto{\pgfqpoint{3.887628in}{1.047120in}}%
\pgfpathlineto{\pgfqpoint{3.948352in}{1.021698in}}%
\pgfpathlineto{\pgfqpoint{4.012424in}{0.996869in}}%
\pgfpathlineto{\pgfqpoint{4.079873in}{0.972705in}}%
\pgfpathlineto{\pgfqpoint{4.150724in}{0.949279in}}%
\pgfpathlineto{\pgfqpoint{4.224991in}{0.926669in}}%
\pgfpathlineto{\pgfqpoint{4.302681in}{0.904956in}}%
\pgfpathlineto{\pgfqpoint{4.383794in}{0.884224in}}%
\pgfpathlineto{\pgfqpoint{4.468317in}{0.864561in}}%
\pgfpathlineto{\pgfqpoint{4.556224in}{0.846059in}}%
\pgfpathlineto{\pgfqpoint{4.647480in}{0.828813in}}%
\pgfpathlineto{\pgfqpoint{4.742031in}{0.812920in}}%
\pgfpathlineto{\pgfqpoint{4.839812in}{0.798481in}}%
\pgfpathlineto{\pgfqpoint{4.940738in}{0.785599in}}%
\pgfpathlineto{\pgfqpoint{5.044706in}{0.774380in}}%
\pgfpathlineto{\pgfqpoint{5.151593in}{0.764932in}}%
\pgfpathlineto{\pgfqpoint{5.261255in}{0.757362in}}%
\pgfpathlineto{\pgfqpoint{5.373525in}{0.751780in}}%
\pgfpathlineto{\pgfqpoint{5.488210in}{0.748294in}}%
\pgfpathlineto{\pgfqpoint{5.605094in}{0.747013in}}%
\pgfpathlineto{\pgfqpoint{5.723932in}{0.748042in}}%
\pgfpathlineto{\pgfqpoint{5.844451in}{0.751484in}}%
\pgfpathlineto{\pgfqpoint{5.966352in}{0.757436in}}%
\pgfpathlineto{\pgfqpoint{6.089305in}{0.765992in}}%
\pgfpathlineto{\pgfqpoint{6.212950in}{0.777236in}}%
\pgfpathlineto{\pgfqpoint{6.336900in}{0.791246in}}%
\pgfpathlineto{\pgfqpoint{6.460738in}{0.808087in}}%
\pgfpathlineto{\pgfqpoint{6.584022in}{0.827814in}}%
\pgfpathlineto{\pgfqpoint{6.706282in}{0.850468in}}%
\pgfpathlineto{\pgfqpoint{6.827028in}{0.876076in}}%
\pgfpathlineto{\pgfqpoint{6.906433in}{0.894793in}}%
\pgfpathlineto{\pgfqpoint{6.984783in}{0.914825in}}%
\pgfpathlineto{\pgfqpoint{7.061919in}{0.936167in}}%
\pgfpathlineto{\pgfqpoint{7.137683in}{0.958808in}}%
\pgfpathlineto{\pgfqpoint{7.211912in}{0.982735in}}%
\pgfpathlineto{\pgfqpoint{7.284445in}{1.007930in}}%
\pgfpathlineto{\pgfqpoint{7.355120in}{1.034370in}}%
\pgfpathlineto{\pgfqpoint{7.423777in}{1.062030in}}%
\pgfpathlineto{\pgfqpoint{7.490257in}{1.090876in}}%
\pgfpathlineto{\pgfqpoint{7.554404in}{1.120874in}}%
\pgfpathlineto{\pgfqpoint{7.616065in}{1.151984in}}%
\pgfpathlineto{\pgfqpoint{7.675094in}{1.184159in}}%
\pgfpathlineto{\pgfqpoint{7.731348in}{1.217353in}}%
\pgfpathlineto{\pgfqpoint{7.784692in}{1.251510in}}%
\pgfpathlineto{\pgfqpoint{7.834997in}{1.286575in}}%
\pgfpathlineto{\pgfqpoint{7.882144in}{1.322485in}}%
\pgfpathlineto{\pgfqpoint{7.926023in}{1.359177in}}%
\pgfpathlineto{\pgfqpoint{7.966531in}{1.396582in}}%
\pgfpathlineto{\pgfqpoint{8.003580in}{1.434630in}}%
\pgfpathlineto{\pgfqpoint{8.037090in}{1.473247in}}%
\pgfpathlineto{\pgfqpoint{8.066993in}{1.512358in}}%
\pgfpathlineto{\pgfqpoint{8.093235in}{1.551886in}}%
\pgfpathlineto{\pgfqpoint{8.115774in}{1.591751in}}%
\pgfpathlineto{\pgfqpoint{8.134581in}{1.631875in}}%
\pgfpathlineto{\pgfqpoint{8.149638in}{1.672177in}}%
\pgfpathlineto{\pgfqpoint{8.155759in}{1.692370in}}%
\pgfpathlineto{\pgfqpoint{8.155759in}{1.692370in}}%
\pgfusepath{stroke}%
\end{pgfscope}%
\begin{pgfscope}%
\pgfpathrectangle{\pgfqpoint{1.250000in}{0.400000in}}{\pgfqpoint{7.750000in}{3.200000in}} %
\pgfusepath{clip}%
\pgfsetrectcap%
\pgfsetroundjoin%
\pgfsetlinewidth{1.003750pt}%
\definecolor{currentstroke}{rgb}{0.000000,0.500000,0.000000}%
\pgfsetstrokecolor{currentstroke}%
\pgfsetdash{}{0pt}%
\pgfpathmoveto{\pgfqpoint{5.129965in}{2.000000in}}%
\pgfpathlineto{\pgfqpoint{5.126468in}{2.001697in}}%
\pgfpathlineto{\pgfqpoint{5.118515in}{2.000556in}}%
\pgfpathlineto{\pgfqpoint{5.117831in}{1.998222in}}%
\pgfpathlineto{\pgfqpoint{5.121263in}{1.996883in}}%
\pgfpathlineto{\pgfqpoint{5.126571in}{1.997400in}}%
\pgfpathlineto{\pgfqpoint{5.128782in}{1.998791in}}%
\pgfpathlineto{\pgfqpoint{5.127977in}{2.001154in}}%
\pgfpathlineto{\pgfqpoint{5.119814in}{2.001684in}}%
\pgfpathlineto{\pgfqpoint{5.117390in}{1.999730in}}%
\pgfpathlineto{\pgfqpoint{5.118431in}{1.998693in}}%
\pgfpathlineto{\pgfqpoint{5.124743in}{1.997813in}}%
\pgfpathlineto{\pgfqpoint{5.128674in}{1.998963in}}%
\pgfpathlineto{\pgfqpoint{5.128950in}{2.001137in}}%
\pgfpathlineto{\pgfqpoint{5.122878in}{2.002779in}}%
\pgfpathlineto{\pgfqpoint{5.118416in}{2.001672in}}%
\pgfpathlineto{\pgfqpoint{5.117938in}{2.000588in}}%
\pgfpathlineto{\pgfqpoint{5.119136in}{1.999375in}}%
\pgfpathlineto{\pgfqpoint{5.122660in}{1.998405in}}%
\pgfpathlineto{\pgfqpoint{5.128419in}{1.998935in}}%
\pgfpathlineto{\pgfqpoint{5.130377in}{2.000635in}}%
\pgfpathlineto{\pgfqpoint{5.129634in}{2.002014in}}%
\pgfpathlineto{\pgfqpoint{5.124820in}{2.003005in}}%
\pgfpathlineto{\pgfqpoint{5.120179in}{2.001352in}}%
\pgfpathlineto{\pgfqpoint{5.120351in}{1.999600in}}%
\pgfpathlineto{\pgfqpoint{5.124306in}{1.998185in}}%
\pgfpathlineto{\pgfqpoint{5.131326in}{1.999778in}}%
\pgfpathlineto{\pgfqpoint{5.131548in}{2.001053in}}%
\pgfpathlineto{\pgfqpoint{5.130139in}{2.002048in}}%
\pgfpathlineto{\pgfqpoint{5.124102in}{2.002267in}}%
\pgfpathlineto{\pgfqpoint{5.121129in}{2.000843in}}%
\pgfpathlineto{\pgfqpoint{5.122150in}{1.998545in}}%
\pgfpathlineto{\pgfqpoint{5.129166in}{1.997749in}}%
\pgfpathlineto{\pgfqpoint{5.132332in}{1.999566in}}%
\pgfpathlineto{\pgfqpoint{5.131732in}{2.000733in}}%
\pgfpathlineto{\pgfqpoint{5.129444in}{2.001709in}}%
\pgfpathlineto{\pgfqpoint{5.125465in}{2.002016in}}%
\pgfpathlineto{\pgfqpoint{5.121155in}{2.000826in}}%
\pgfpathlineto{\pgfqpoint{5.120938in}{1.998746in}}%
\pgfpathlineto{\pgfqpoint{5.126398in}{1.997149in}}%
\pgfpathlineto{\pgfqpoint{5.130726in}{1.998465in}}%
\pgfpathlineto{\pgfqpoint{5.130862in}{1.999847in}}%
\pgfpathlineto{\pgfqpoint{5.128498in}{2.001308in}}%
\pgfpathlineto{\pgfqpoint{5.123259in}{2.001691in}}%
\pgfpathlineto{\pgfqpoint{5.119761in}{2.000398in}}%
\pgfpathlineto{\pgfqpoint{5.119278in}{1.998865in}}%
\pgfpathlineto{\pgfqpoint{5.123217in}{1.997106in}}%
\pgfpathlineto{\pgfqpoint{5.128639in}{1.997942in}}%
\pgfpathlineto{\pgfqpoint{5.129816in}{1.999483in}}%
\pgfpathlineto{\pgfqpoint{5.127575in}{2.001395in}}%
\pgfpathlineto{\pgfqpoint{5.122412in}{2.001733in}}%
\pgfpathlineto{\pgfqpoint{5.118082in}{1.999596in}}%
\pgfpathlineto{\pgfqpoint{5.118798in}{1.998470in}}%
\pgfpathlineto{\pgfqpoint{5.124202in}{1.997490in}}%
\pgfpathlineto{\pgfqpoint{5.127743in}{1.998229in}}%
\pgfpathlineto{\pgfqpoint{5.129464in}{2.000125in}}%
\pgfpathlineto{\pgfqpoint{5.126822in}{2.001898in}}%
\pgfpathlineto{\pgfqpoint{5.119746in}{2.001819in}}%
\pgfpathlineto{\pgfqpoint{5.117988in}{1.999694in}}%
\pgfpathlineto{\pgfqpoint{5.119676in}{1.998626in}}%
\pgfpathlineto{\pgfqpoint{5.123262in}{1.998044in}}%
\pgfpathlineto{\pgfqpoint{5.128051in}{1.998783in}}%
\pgfpathlineto{\pgfqpoint{5.129366in}{2.000901in}}%
\pgfpathlineto{\pgfqpoint{5.127398in}{2.002338in}}%
\pgfpathlineto{\pgfqpoint{5.121691in}{2.002746in}}%
\pgfpathlineto{\pgfqpoint{5.118559in}{2.000861in}}%
\pgfpathlineto{\pgfqpoint{5.119353in}{1.999476in}}%
\pgfpathlineto{\pgfqpoint{5.122672in}{1.998338in}}%
\pgfpathlineto{\pgfqpoint{5.128020in}{1.998648in}}%
\pgfpathlineto{\pgfqpoint{5.130417in}{2.000289in}}%
\pgfpathlineto{\pgfqpoint{5.129944in}{2.001765in}}%
\pgfpathlineto{\pgfqpoint{5.125229in}{2.002933in}}%
\pgfpathlineto{\pgfqpoint{5.120441in}{2.001207in}}%
\pgfpathlineto{\pgfqpoint{5.120903in}{1.999283in}}%
\pgfpathlineto{\pgfqpoint{5.125389in}{1.998073in}}%
\pgfpathlineto{\pgfqpoint{5.131716in}{1.999610in}}%
\pgfpathlineto{\pgfqpoint{5.132059in}{2.000830in}}%
\pgfpathlineto{\pgfqpoint{5.128554in}{2.002517in}}%
\pgfpathlineto{\pgfqpoint{5.122078in}{2.001606in}}%
\pgfpathlineto{\pgfqpoint{5.121000in}{1.999450in}}%
\pgfpathlineto{\pgfqpoint{5.124550in}{1.997925in}}%
\pgfpathlineto{\pgfqpoint{5.130995in}{1.998468in}}%
\pgfpathlineto{\pgfqpoint{5.132118in}{1.999522in}}%
\pgfpathlineto{\pgfqpoint{5.131594in}{2.000705in}}%
\pgfpathlineto{\pgfqpoint{5.129269in}{2.001712in}}%
\pgfpathlineto{\pgfqpoint{5.125204in}{2.002028in}}%
\pgfpathlineto{\pgfqpoint{5.121018in}{2.000864in}}%
\pgfpathlineto{\pgfqpoint{5.120661in}{1.998869in}}%
\pgfpathlineto{\pgfqpoint{5.126230in}{1.997194in}}%
\pgfpathlineto{\pgfqpoint{5.130819in}{1.998415in}}%
\pgfpathlineto{\pgfqpoint{5.131102in}{1.999706in}}%
\pgfpathlineto{\pgfqpoint{5.129066in}{2.001110in}}%
\pgfpathlineto{\pgfqpoint{5.123952in}{2.001688in}}%
\pgfpathlineto{\pgfqpoint{5.120057in}{2.000324in}}%
\pgfpathlineto{\pgfqpoint{5.119620in}{1.998739in}}%
\pgfpathlineto{\pgfqpoint{5.123791in}{1.997098in}}%
\pgfpathlineto{\pgfqpoint{5.129189in}{1.998143in}}%
\pgfpathlineto{\pgfqpoint{5.130105in}{1.999704in}}%
\pgfpathlineto{\pgfqpoint{5.127768in}{2.001440in}}%
\pgfpathlineto{\pgfqpoint{5.122676in}{2.001755in}}%
\pgfpathlineto{\pgfqpoint{5.118352in}{1.999428in}}%
\pgfpathlineto{\pgfqpoint{5.119214in}{1.998294in}}%
\pgfpathlineto{\pgfqpoint{5.124758in}{1.997548in}}%
\pgfpathlineto{\pgfqpoint{5.128159in}{1.998528in}}%
\pgfpathlineto{\pgfqpoint{5.129047in}{2.000688in}}%
\pgfpathlineto{\pgfqpoint{5.125740in}{2.002124in}}%
\pgfpathlineto{\pgfqpoint{5.119234in}{2.001533in}}%
\pgfpathlineto{\pgfqpoint{5.118021in}{2.000420in}}%
\pgfpathlineto{\pgfqpoint{5.118489in}{1.999210in}}%
\pgfpathlineto{\pgfqpoint{5.120730in}{1.998229in}}%
\pgfpathlineto{\pgfqpoint{5.124625in}{1.997954in}}%
\pgfpathlineto{\pgfqpoint{5.128787in}{1.999140in}}%
\pgfpathlineto{\pgfqpoint{5.128887in}{2.001259in}}%
\pgfpathlineto{\pgfqpoint{5.123079in}{2.002706in}}%
\pgfpathlineto{\pgfqpoint{5.119009in}{2.001116in}}%
\pgfpathlineto{\pgfqpoint{5.119354in}{1.999681in}}%
\pgfpathlineto{\pgfqpoint{5.122322in}{1.998388in}}%
\pgfpathlineto{\pgfqpoint{5.127458in}{1.998417in}}%
\pgfpathlineto{\pgfqpoint{5.130320in}{1.999915in}}%
\pgfpathlineto{\pgfqpoint{5.130130in}{2.001472in}}%
\pgfpathlineto{\pgfqpoint{5.125347in}{2.002883in}}%
\pgfpathlineto{\pgfqpoint{5.120444in}{2.001391in}}%
\pgfpathlineto{\pgfqpoint{5.120582in}{1.999588in}}%
\pgfpathlineto{\pgfqpoint{5.124859in}{1.998161in}}%
\pgfpathlineto{\pgfqpoint{5.131436in}{1.999672in}}%
\pgfpathlineto{\pgfqpoint{5.131551in}{2.000919in}}%
\pgfpathlineto{\pgfqpoint{5.129940in}{2.001941in}}%
\pgfpathlineto{\pgfqpoint{5.123320in}{2.002097in}}%
\pgfpathlineto{\pgfqpoint{5.120639in}{2.000563in}}%
\pgfpathlineto{\pgfqpoint{5.121960in}{1.998515in}}%
\pgfpathlineto{\pgfqpoint{5.129281in}{1.997976in}}%
\pgfpathlineto{\pgfqpoint{5.131755in}{2.000077in}}%
\pgfpathlineto{\pgfqpoint{5.130210in}{2.001246in}}%
\pgfpathlineto{\pgfqpoint{5.126563in}{2.001922in}}%
\pgfpathlineto{\pgfqpoint{5.121749in}{2.001194in}}%
\pgfpathlineto{\pgfqpoint{5.120377in}{1.999214in}}%
\pgfpathlineto{\pgfqpoint{5.122247in}{1.997783in}}%
\pgfpathlineto{\pgfqpoint{5.128185in}{1.997447in}}%
\pgfpathlineto{\pgfqpoint{5.130966in}{1.999647in}}%
\pgfpathlineto{\pgfqpoint{5.129228in}{2.001107in}}%
\pgfpathlineto{\pgfqpoint{5.124542in}{2.001792in}}%
\pgfpathlineto{\pgfqpoint{5.120312in}{2.000602in}}%
\pgfpathlineto{\pgfqpoint{5.119517in}{1.998980in}}%
\pgfpathlineto{\pgfqpoint{5.120920in}{1.997779in}}%
\pgfpathlineto{\pgfqpoint{5.126493in}{1.997391in}}%
\pgfpathlineto{\pgfqpoint{5.129055in}{1.998397in}}%
\pgfpathlineto{\pgfqpoint{5.129658in}{2.000188in}}%
\pgfpathlineto{\pgfqpoint{5.126325in}{2.001811in}}%
\pgfpathlineto{\pgfqpoint{5.118998in}{2.000764in}}%
\pgfpathlineto{\pgfqpoint{5.118277in}{1.999478in}}%
\pgfpathlineto{\pgfqpoint{5.119338in}{1.998340in}}%
\pgfpathlineto{\pgfqpoint{5.125299in}{1.997695in}}%
\pgfpathlineto{\pgfqpoint{5.128615in}{1.998907in}}%
\pgfpathlineto{\pgfqpoint{5.128452in}{2.001184in}}%
\pgfpathlineto{\pgfqpoint{5.121550in}{2.002301in}}%
\pgfpathlineto{\pgfqpoint{5.118288in}{2.000397in}}%
\pgfpathlineto{\pgfqpoint{5.119244in}{1.999111in}}%
\pgfpathlineto{\pgfqpoint{5.122319in}{1.998148in}}%
\pgfpathlineto{\pgfqpoint{5.126995in}{1.998312in}}%
\pgfpathlineto{\pgfqpoint{5.129807in}{2.000045in}}%
\pgfpathlineto{\pgfqpoint{5.128865in}{2.001755in}}%
\pgfpathlineto{\pgfqpoint{5.123177in}{2.002774in}}%
\pgfpathlineto{\pgfqpoint{5.119422in}{2.000955in}}%
\pgfpathlineto{\pgfqpoint{5.120386in}{1.999356in}}%
\pgfpathlineto{\pgfqpoint{5.124831in}{1.998230in}}%
\pgfpathlineto{\pgfqpoint{5.129442in}{1.999046in}}%
\pgfpathlineto{\pgfqpoint{5.130946in}{2.000526in}}%
\pgfpathlineto{\pgfqpoint{5.130129in}{2.001815in}}%
\pgfpathlineto{\pgfqpoint{5.124839in}{2.002748in}}%
\pgfpathlineto{\pgfqpoint{5.120217in}{2.000601in}}%
\pgfpathlineto{\pgfqpoint{5.121913in}{1.998691in}}%
\pgfpathlineto{\pgfqpoint{5.126733in}{1.998089in}}%
\pgfpathlineto{\pgfqpoint{5.131710in}{2.000065in}}%
\pgfpathlineto{\pgfqpoint{5.131159in}{2.001249in}}%
\pgfpathlineto{\pgfqpoint{5.128918in}{2.002095in}}%
\pgfpathlineto{\pgfqpoint{5.125347in}{2.002252in}}%
\pgfpathlineto{\pgfqpoint{5.121562in}{2.001219in}}%
\pgfpathlineto{\pgfqpoint{5.120984in}{1.999057in}}%
\pgfpathlineto{\pgfqpoint{5.123970in}{1.997735in}}%
\pgfpathlineto{\pgfqpoint{5.130269in}{1.998138in}}%
\pgfpathlineto{\pgfqpoint{5.131697in}{1.999202in}}%
\pgfpathlineto{\pgfqpoint{5.131434in}{2.000454in}}%
\pgfpathlineto{\pgfqpoint{5.129175in}{2.001549in}}%
\pgfpathlineto{\pgfqpoint{5.124856in}{2.001869in}}%
\pgfpathlineto{\pgfqpoint{5.120649in}{2.000431in}}%
\pgfpathlineto{\pgfqpoint{5.120644in}{1.998581in}}%
\pgfpathlineto{\pgfqpoint{5.125743in}{1.997167in}}%
\pgfpathlineto{\pgfqpoint{5.130222in}{1.998776in}}%
\pgfpathlineto{\pgfqpoint{5.129921in}{2.000421in}}%
\pgfpathlineto{\pgfqpoint{5.126278in}{2.001774in}}%
\pgfpathlineto{\pgfqpoint{5.121408in}{2.001402in}}%
\pgfpathlineto{\pgfqpoint{5.119106in}{2.000012in}}%
\pgfpathlineto{\pgfqpoint{5.119271in}{1.998613in}}%
\pgfpathlineto{\pgfqpoint{5.123885in}{1.997257in}}%
\pgfpathlineto{\pgfqpoint{5.129228in}{1.999062in}}%
\pgfpathlineto{\pgfqpoint{5.128166in}{2.001196in}}%
\pgfpathlineto{\pgfqpoint{5.123641in}{2.002015in}}%
\pgfpathlineto{\pgfqpoint{5.118340in}{2.000369in}}%
\pgfpathlineto{\pgfqpoint{5.118534in}{1.999138in}}%
\pgfpathlineto{\pgfqpoint{5.120499in}{1.998131in}}%
\pgfpathlineto{\pgfqpoint{5.123942in}{1.997737in}}%
\pgfpathlineto{\pgfqpoint{5.127885in}{1.998469in}}%
\pgfpathlineto{\pgfqpoint{5.129383in}{2.000490in}}%
\pgfpathlineto{\pgfqpoint{5.126805in}{2.002065in}}%
\pgfpathlineto{\pgfqpoint{5.120281in}{2.001963in}}%
\pgfpathlineto{\pgfqpoint{5.118756in}{2.000904in}}%
\pgfpathlineto{\pgfqpoint{5.119133in}{1.999568in}}%
\pgfpathlineto{\pgfqpoint{5.121938in}{1.998401in}}%
\pgfpathlineto{\pgfqpoint{5.126962in}{1.998379in}}%
\pgfpathlineto{\pgfqpoint{5.129949in}{2.000019in}}%
\pgfpathlineto{\pgfqpoint{5.129360in}{2.001648in}}%
\pgfpathlineto{\pgfqpoint{5.123880in}{2.002762in}}%
\pgfpathlineto{\pgfqpoint{5.119643in}{2.000926in}}%
\pgfpathlineto{\pgfqpoint{5.120530in}{1.999273in}}%
\pgfpathlineto{\pgfqpoint{5.124948in}{1.998191in}}%
\pgfpathlineto{\pgfqpoint{5.129433in}{1.998974in}}%
\pgfpathlineto{\pgfqpoint{5.130972in}{2.000426in}}%
\pgfpathlineto{\pgfqpoint{5.130156in}{2.001704in}}%
\pgfpathlineto{\pgfqpoint{5.124476in}{2.002475in}}%
\pgfpathlineto{\pgfqpoint{5.121378in}{2.001506in}}%
\pgfpathlineto{\pgfqpoint{5.120585in}{1.999583in}}%
\pgfpathlineto{\pgfqpoint{5.123935in}{1.998060in}}%
\pgfpathlineto{\pgfqpoint{5.130881in}{1.998957in}}%
\pgfpathlineto{\pgfqpoint{5.131696in}{2.000199in}}%
\pgfpathlineto{\pgfqpoint{5.130604in}{2.001370in}}%
\pgfpathlineto{\pgfqpoint{5.127750in}{2.002095in}}%
\pgfpathlineto{\pgfqpoint{5.123637in}{2.001862in}}%
\pgfpathlineto{\pgfqpoint{5.120634in}{2.000074in}}%
\pgfpathlineto{\pgfqpoint{5.122168in}{1.998216in}}%
\pgfpathlineto{\pgfqpoint{5.128642in}{1.997751in}}%
\pgfpathlineto{\pgfqpoint{5.130704in}{1.998708in}}%
\pgfpathlineto{\pgfqpoint{5.131010in}{2.000075in}}%
\pgfpathlineto{\pgfqpoint{5.128950in}{2.001409in}}%
\pgfpathlineto{\pgfqpoint{5.124502in}{2.001875in}}%
\pgfpathlineto{\pgfqpoint{5.120508in}{2.000618in}}%
\pgfpathlineto{\pgfqpoint{5.120170in}{1.998849in}}%
\pgfpathlineto{\pgfqpoint{5.122217in}{1.997658in}}%
\pgfpathlineto{\pgfqpoint{5.128082in}{1.997742in}}%
\pgfpathlineto{\pgfqpoint{5.129885in}{1.999009in}}%
\pgfpathlineto{\pgfqpoint{5.129066in}{2.000824in}}%
\pgfpathlineto{\pgfqpoint{5.124461in}{2.001873in}}%
\pgfpathlineto{\pgfqpoint{5.118747in}{1.999886in}}%
\pgfpathlineto{\pgfqpoint{5.119322in}{1.998590in}}%
\pgfpathlineto{\pgfqpoint{5.124837in}{1.997511in}}%
\pgfpathlineto{\pgfqpoint{5.128136in}{1.998301in}}%
\pgfpathlineto{\pgfqpoint{5.129434in}{2.000162in}}%
\pgfpathlineto{\pgfqpoint{5.126350in}{2.001874in}}%
\pgfpathlineto{\pgfqpoint{5.119341in}{2.001302in}}%
\pgfpathlineto{\pgfqpoint{5.118349in}{2.000125in}}%
\pgfpathlineto{\pgfqpoint{5.119344in}{1.998903in}}%
\pgfpathlineto{\pgfqpoint{5.122374in}{1.998058in}}%
\pgfpathlineto{\pgfqpoint{5.126872in}{1.998279in}}%
\pgfpathlineto{\pgfqpoint{5.129647in}{2.000059in}}%
\pgfpathlineto{\pgfqpoint{5.128370in}{2.001788in}}%
\pgfpathlineto{\pgfqpoint{5.122110in}{2.002497in}}%
\pgfpathlineto{\pgfqpoint{5.118942in}{2.000435in}}%
\pgfpathlineto{\pgfqpoint{5.120416in}{1.999013in}}%
\pgfpathlineto{\pgfqpoint{5.124802in}{1.998216in}}%
\pgfpathlineto{\pgfqpoint{5.129424in}{1.999330in}}%
\pgfpathlineto{\pgfqpoint{5.130363in}{2.000989in}}%
\pgfpathlineto{\pgfqpoint{5.128978in}{2.002215in}}%
\pgfpathlineto{\pgfqpoint{5.123407in}{2.002629in}}%
\pgfpathlineto{\pgfqpoint{5.120865in}{2.001668in}}%
\pgfpathlineto{\pgfqpoint{5.120152in}{1.999967in}}%
\pgfpathlineto{\pgfqpoint{5.123131in}{1.998303in}}%
\pgfpathlineto{\pgfqpoint{5.130828in}{1.999322in}}%
\pgfpathlineto{\pgfqpoint{5.131513in}{2.000650in}}%
\pgfpathlineto{\pgfqpoint{5.130342in}{2.001775in}}%
\pgfpathlineto{\pgfqpoint{5.124259in}{2.002246in}}%
\pgfpathlineto{\pgfqpoint{5.121142in}{2.000895in}}%
\pgfpathlineto{\pgfqpoint{5.121874in}{1.998642in}}%
\pgfpathlineto{\pgfqpoint{5.128901in}{1.997806in}}%
\pgfpathlineto{\pgfqpoint{5.131844in}{1.999805in}}%
\pgfpathlineto{\pgfqpoint{5.130751in}{2.001043in}}%
\pgfpathlineto{\pgfqpoint{5.127683in}{2.001916in}}%
\pgfpathlineto{\pgfqpoint{5.123126in}{2.001702in}}%
\pgfpathlineto{\pgfqpoint{5.120333in}{1.999882in}}%
\pgfpathlineto{\pgfqpoint{5.121493in}{1.998155in}}%
\pgfpathlineto{\pgfqpoint{5.127260in}{1.997280in}}%
\pgfpathlineto{\pgfqpoint{5.130681in}{1.999235in}}%
\pgfpathlineto{\pgfqpoint{5.129417in}{2.000828in}}%
\pgfpathlineto{\pgfqpoint{5.124800in}{2.001801in}}%
\pgfpathlineto{\pgfqpoint{5.120416in}{2.000889in}}%
\pgfpathlineto{\pgfqpoint{5.119116in}{1.999373in}}%
\pgfpathlineto{\pgfqpoint{5.120095in}{1.998087in}}%
\pgfpathlineto{\pgfqpoint{5.125426in}{1.997237in}}%
\pgfpathlineto{\pgfqpoint{5.128256in}{1.997936in}}%
\pgfpathlineto{\pgfqpoint{5.129724in}{1.999475in}}%
\pgfpathlineto{\pgfqpoint{5.127658in}{2.001409in}}%
\pgfpathlineto{\pgfqpoint{5.122731in}{2.001837in}}%
\pgfpathlineto{\pgfqpoint{5.118278in}{1.999760in}}%
\pgfpathlineto{\pgfqpoint{5.119076in}{1.998595in}}%
\pgfpathlineto{\pgfqpoint{5.125144in}{1.997739in}}%
\pgfpathlineto{\pgfqpoint{5.128702in}{1.998858in}}%
\pgfpathlineto{\pgfqpoint{5.128979in}{2.000977in}}%
\pgfpathlineto{\pgfqpoint{5.125741in}{2.002240in}}%
\pgfpathlineto{\pgfqpoint{5.119430in}{2.001639in}}%
\pgfpathlineto{\pgfqpoint{5.118282in}{2.000520in}}%
\pgfpathlineto{\pgfqpoint{5.118975in}{1.999274in}}%
\pgfpathlineto{\pgfqpoint{5.121808in}{1.998291in}}%
\pgfpathlineto{\pgfqpoint{5.126614in}{1.998351in}}%
\pgfpathlineto{\pgfqpoint{5.129677in}{2.000202in}}%
\pgfpathlineto{\pgfqpoint{5.128820in}{2.001873in}}%
\pgfpathlineto{\pgfqpoint{5.123467in}{2.002885in}}%
\pgfpathlineto{\pgfqpoint{5.119362in}{2.001188in}}%
\pgfpathlineto{\pgfqpoint{5.119714in}{1.999649in}}%
\pgfpathlineto{\pgfqpoint{5.123051in}{1.998321in}}%
\pgfpathlineto{\pgfqpoint{5.128250in}{1.998594in}}%
\pgfpathlineto{\pgfqpoint{5.130763in}{2.000040in}}%
\pgfpathlineto{\pgfqpoint{5.130688in}{2.001459in}}%
\pgfpathlineto{\pgfqpoint{5.126341in}{2.002827in}}%
\pgfpathlineto{\pgfqpoint{5.120933in}{2.001215in}}%
\pgfpathlineto{\pgfqpoint{5.121289in}{1.999114in}}%
\pgfpathlineto{\pgfqpoint{5.125723in}{1.997997in}}%
\pgfpathlineto{\pgfqpoint{5.131760in}{1.999383in}}%
\pgfpathlineto{\pgfqpoint{5.132169in}{2.000568in}}%
\pgfpathlineto{\pgfqpoint{5.131026in}{2.001632in}}%
\pgfpathlineto{\pgfqpoint{5.125039in}{2.002321in}}%
\pgfpathlineto{\pgfqpoint{5.121552in}{2.001176in}}%
\pgfpathlineto{\pgfqpoint{5.121422in}{1.998877in}}%
\pgfpathlineto{\pgfqpoint{5.128077in}{1.997521in}}%
\pgfpathlineto{\pgfqpoint{5.131724in}{1.999139in}}%
\pgfpathlineto{\pgfqpoint{5.131208in}{2.000403in}}%
\pgfpathlineto{\pgfqpoint{5.128520in}{2.001543in}}%
\pgfpathlineto{\pgfqpoint{5.123586in}{2.001705in}}%
\pgfpathlineto{\pgfqpoint{5.120159in}{2.000085in}}%
\pgfpathlineto{\pgfqpoint{5.120450in}{1.998390in}}%
\pgfpathlineto{\pgfqpoint{5.125330in}{1.996950in}}%
\pgfpathlineto{\pgfqpoint{5.129970in}{1.998044in}}%
\pgfpathlineto{\pgfqpoint{5.130588in}{1.999396in}}%
\pgfpathlineto{\pgfqpoint{5.128652in}{2.001018in}}%
\pgfpathlineto{\pgfqpoint{5.122980in}{2.001611in}}%
\pgfpathlineto{\pgfqpoint{5.118357in}{1.999269in}}%
\pgfpathlineto{\pgfqpoint{5.119137in}{1.998172in}}%
\pgfpathlineto{\pgfqpoint{5.124263in}{1.997353in}}%
\pgfpathlineto{\pgfqpoint{5.127556in}{1.998079in}}%
\pgfpathlineto{\pgfqpoint{5.129379in}{1.999907in}}%
\pgfpathlineto{\pgfqpoint{5.126670in}{2.001821in}}%
\pgfpathlineto{\pgfqpoint{5.119420in}{2.001708in}}%
\pgfpathlineto{\pgfqpoint{5.117468in}{1.999687in}}%
\pgfpathlineto{\pgfqpoint{5.118721in}{1.998689in}}%
\pgfpathlineto{\pgfqpoint{5.121540in}{1.998034in}}%
\pgfpathlineto{\pgfqpoint{5.125850in}{1.998229in}}%
\pgfpathlineto{\pgfqpoint{5.129300in}{2.000270in}}%
\pgfpathlineto{\pgfqpoint{5.128218in}{2.002017in}}%
\pgfpathlineto{\pgfqpoint{5.123040in}{2.002991in}}%
\pgfpathlineto{\pgfqpoint{5.119277in}{2.001435in}}%
\pgfpathlineto{\pgfqpoint{5.119449in}{1.999967in}}%
\pgfpathlineto{\pgfqpoint{5.122408in}{1.998518in}}%
\pgfpathlineto{\pgfqpoint{5.127954in}{1.998565in}}%
\pgfpathlineto{\pgfqpoint{5.130805in}{1.999937in}}%
\pgfpathlineto{\pgfqpoint{5.131116in}{2.001340in}}%
\pgfpathlineto{\pgfqpoint{5.127692in}{2.002958in}}%
\pgfpathlineto{\pgfqpoint{5.122390in}{2.002220in}}%
\pgfpathlineto{\pgfqpoint{5.120822in}{2.000578in}}%
\pgfpathlineto{\pgfqpoint{5.123498in}{1.998338in}}%
\pgfpathlineto{\pgfqpoint{5.130805in}{1.998468in}}%
\pgfpathlineto{\pgfqpoint{5.132158in}{2.000534in}}%
\pgfpathlineto{\pgfqpoint{5.130455in}{2.001532in}}%
\pgfpathlineto{\pgfqpoint{5.127123in}{2.002081in}}%
\pgfpathlineto{\pgfqpoint{5.122686in}{2.001531in}}%
\pgfpathlineto{\pgfqpoint{5.120685in}{1.999383in}}%
\pgfpathlineto{\pgfqpoint{5.122630in}{1.997789in}}%
\pgfpathlineto{\pgfqpoint{5.128354in}{1.997219in}}%
\pgfpathlineto{\pgfqpoint{5.131312in}{1.998933in}}%
\pgfpathlineto{\pgfqpoint{5.130458in}{2.000302in}}%
\pgfpathlineto{\pgfqpoint{5.126737in}{2.001520in}}%
\pgfpathlineto{\pgfqpoint{5.120771in}{2.000785in}}%
\pgfpathlineto{\pgfqpoint{5.118207in}{1.998131in}}%
\pgfpathlineto{\pgfqpoint{5.120532in}{1.996543in}}%
\pgfpathlineto{\pgfqpoint{5.124830in}{1.996297in}}%
\pgfpathlineto{\pgfqpoint{5.129164in}{1.997774in}}%
\pgfpathlineto{\pgfqpoint{5.130025in}{1.999361in}}%
\pgfpathlineto{\pgfqpoint{5.127872in}{2.001319in}}%
\pgfpathlineto{\pgfqpoint{5.118844in}{2.001121in}}%
\pgfpathlineto{\pgfqpoint{5.116259in}{1.999100in}}%
\pgfpathlineto{\pgfqpoint{5.117963in}{1.997534in}}%
\pgfpathlineto{\pgfqpoint{5.122863in}{1.997161in}}%
\pgfpathlineto{\pgfqpoint{5.126199in}{1.997849in}}%
\pgfpathlineto{\pgfqpoint{5.128882in}{1.999766in}}%
\pgfpathlineto{\pgfqpoint{5.125983in}{2.002013in}}%
\pgfpathlineto{\pgfqpoint{5.119301in}{2.003017in}}%
\pgfpathlineto{\pgfqpoint{5.113209in}{2.002047in}}%
\pgfpathlineto{\pgfqpoint{5.110727in}{2.000202in}}%
\pgfpathlineto{\pgfqpoint{5.111100in}{1.998221in}}%
\pgfpathlineto{\pgfqpoint{5.113992in}{1.996578in}}%
\pgfpathlineto{\pgfqpoint{5.119241in}{1.995835in}}%
\pgfpathlineto{\pgfqpoint{5.126442in}{1.997216in}}%
\pgfpathlineto{\pgfqpoint{5.128535in}{1.998807in}}%
\pgfpathlineto{\pgfqpoint{5.127374in}{2.001494in}}%
\pgfpathlineto{\pgfqpoint{5.120456in}{2.002830in}}%
\pgfpathlineto{\pgfqpoint{5.115343in}{2.001553in}}%
\pgfpathlineto{\pgfqpoint{5.115965in}{1.999976in}}%
\pgfpathlineto{\pgfqpoint{5.121233in}{1.998493in}}%
\pgfpathlineto{\pgfqpoint{5.126927in}{1.998734in}}%
\pgfpathlineto{\pgfqpoint{5.129080in}{2.001288in}}%
\pgfpathlineto{\pgfqpoint{5.127475in}{2.004283in}}%
\pgfpathlineto{\pgfqpoint{5.121675in}{2.007316in}}%
\pgfpathlineto{\pgfqpoint{5.108923in}{2.010299in}}%
\pgfpathlineto{\pgfqpoint{5.088573in}{2.012716in}}%
\pgfpathlineto{\pgfqpoint{5.065998in}{2.013444in}}%
\pgfpathlineto{\pgfqpoint{5.042129in}{2.012370in}}%
\pgfpathlineto{\pgfqpoint{5.013370in}{2.008748in}}%
\pgfpathlineto{\pgfqpoint{4.989343in}{2.003705in}}%
\pgfpathlineto{\pgfqpoint{4.964259in}{1.996184in}}%
\pgfpathlineto{\pgfqpoint{4.939452in}{1.985821in}}%
\pgfpathlineto{\pgfqpoint{4.923876in}{1.977205in}}%
\pgfpathlineto{\pgfqpoint{4.909720in}{1.967175in}}%
\pgfpathlineto{\pgfqpoint{4.897592in}{1.955730in}}%
\pgfpathlineto{\pgfqpoint{4.888133in}{1.942908in}}%
\pgfpathlineto{\pgfqpoint{4.882000in}{1.928795in}}%
\pgfpathlineto{\pgfqpoint{4.879851in}{1.913519in}}%
\pgfpathlineto{\pgfqpoint{4.880472in}{1.905500in}}%
\pgfpathlineto{\pgfqpoint{4.882326in}{1.897262in}}%
\pgfpathlineto{\pgfqpoint{4.885488in}{1.888836in}}%
\pgfpathlineto{\pgfqpoint{4.890029in}{1.880255in}}%
\pgfpathlineto{\pgfqpoint{4.896012in}{1.871557in}}%
\pgfpathlineto{\pgfqpoint{4.903501in}{1.862780in}}%
\pgfpathlineto{\pgfqpoint{4.912548in}{1.853968in}}%
\pgfpathlineto{\pgfqpoint{4.935504in}{1.836424in}}%
\pgfpathlineto{\pgfqpoint{4.965174in}{1.819322in}}%
\pgfpathlineto{\pgfqpoint{5.001708in}{1.803097in}}%
\pgfpathlineto{\pgfqpoint{5.045092in}{1.788212in}}%
\pgfpathlineto{\pgfqpoint{5.095131in}{1.775151in}}%
\pgfpathlineto{\pgfqpoint{5.151427in}{1.764403in}}%
\pgfpathlineto{\pgfqpoint{5.213372in}{1.756451in}}%
\pgfpathlineto{\pgfqpoint{5.280136in}{1.751752in}}%
\pgfpathlineto{\pgfqpoint{5.350676in}{1.750724in}}%
\pgfpathlineto{\pgfqpoint{5.423738in}{1.753725in}}%
\pgfpathlineto{\pgfqpoint{5.497884in}{1.761033in}}%
\pgfpathlineto{\pgfqpoint{5.534870in}{1.766364in}}%
\pgfpathlineto{\pgfqpoint{5.571518in}{1.772833in}}%
\pgfpathlineto{\pgfqpoint{5.607610in}{1.780446in}}%
\pgfpathlineto{\pgfqpoint{5.642926in}{1.789199in}}%
\pgfpathlineto{\pgfqpoint{5.677242in}{1.799084in}}%
\pgfpathlineto{\pgfqpoint{5.710332in}{1.810082in}}%
\pgfpathlineto{\pgfqpoint{5.741973in}{1.822166in}}%
\pgfpathlineto{\pgfqpoint{5.771946in}{1.835302in}}%
\pgfpathlineto{\pgfqpoint{5.800036in}{1.849446in}}%
\pgfpathlineto{\pgfqpoint{5.826036in}{1.864548in}}%
\pgfpathlineto{\pgfqpoint{5.849749in}{1.880548in}}%
\pgfpathlineto{\pgfqpoint{5.870989in}{1.897380in}}%
\pgfpathlineto{\pgfqpoint{5.889583in}{1.914971in}}%
\pgfpathlineto{\pgfqpoint{5.905373in}{1.933239in}}%
\pgfpathlineto{\pgfqpoint{5.918219in}{1.952100in}}%
\pgfpathlineto{\pgfqpoint{5.927998in}{1.971462in}}%
\pgfpathlineto{\pgfqpoint{5.934605in}{1.991230in}}%
\pgfpathlineto{\pgfqpoint{5.937958in}{2.011306in}}%
\pgfpathlineto{\pgfqpoint{5.937994in}{2.031587in}}%
\pgfpathlineto{\pgfqpoint{5.934672in}{2.051970in}}%
\pgfpathlineto{\pgfqpoint{5.927971in}{2.072351in}}%
\pgfpathlineto{\pgfqpoint{5.917894in}{2.092626in}}%
\pgfpathlineto{\pgfqpoint{5.904464in}{2.112693in}}%
\pgfpathlineto{\pgfqpoint{5.887724in}{2.132448in}}%
\pgfpathlineto{\pgfqpoint{5.867738in}{2.151793in}}%
\pgfpathlineto{\pgfqpoint{5.844588in}{2.170634in}}%
\pgfpathlineto{\pgfqpoint{5.818372in}{2.188877in}}%
\pgfpathlineto{\pgfqpoint{5.789208in}{2.206437in}}%
\pgfpathlineto{\pgfqpoint{5.757226in}{2.223230in}}%
\pgfpathlineto{\pgfqpoint{5.722569in}{2.239181in}}%
\pgfpathlineto{\pgfqpoint{5.685394in}{2.254218in}}%
\pgfpathlineto{\pgfqpoint{5.645866in}{2.268277in}}%
\pgfpathlineto{\pgfqpoint{5.604158in}{2.281300in}}%
\pgfpathlineto{\pgfqpoint{5.560451in}{2.293235in}}%
\pgfpathlineto{\pgfqpoint{5.514932in}{2.304035in}}%
\pgfpathlineto{\pgfqpoint{5.467789in}{2.313660in}}%
\pgfpathlineto{\pgfqpoint{5.419214in}{2.322079in}}%
\pgfpathlineto{\pgfqpoint{5.369401in}{2.329261in}}%
\pgfpathlineto{\pgfqpoint{5.318542in}{2.335187in}}%
\pgfpathlineto{\pgfqpoint{5.266829in}{2.339839in}}%
\pgfpathlineto{\pgfqpoint{5.214453in}{2.343206in}}%
\pgfpathlineto{\pgfqpoint{5.161601in}{2.345281in}}%
\pgfpathlineto{\pgfqpoint{5.108457in}{2.346061in}}%
\pgfpathlineto{\pgfqpoint{5.055202in}{2.345550in}}%
\pgfpathlineto{\pgfqpoint{5.002012in}{2.343752in}}%
\pgfpathlineto{\pgfqpoint{4.949059in}{2.340678in}}%
\pgfpathlineto{\pgfqpoint{4.896511in}{2.336339in}}%
\pgfpathlineto{\pgfqpoint{4.844531in}{2.330752in}}%
\pgfpathlineto{\pgfqpoint{4.793276in}{2.323935in}}%
\pgfpathlineto{\pgfqpoint{4.742902in}{2.315908in}}%
\pgfpathlineto{\pgfqpoint{4.693558in}{2.306695in}}%
\pgfpathlineto{\pgfqpoint{4.645388in}{2.296321in}}%
\pgfpathlineto{\pgfqpoint{4.598534in}{2.284814in}}%
\pgfpathlineto{\pgfqpoint{4.553133in}{2.272202in}}%
\pgfpathlineto{\pgfqpoint{4.509318in}{2.258518in}}%
\pgfpathlineto{\pgfqpoint{4.467220in}{2.243793in}}%
\pgfpathlineto{\pgfqpoint{4.426964in}{2.228062in}}%
\pgfpathlineto{\pgfqpoint{4.388673in}{2.211362in}}%
\pgfpathlineto{\pgfqpoint{4.352470in}{2.193731in}}%
\pgfpathlineto{\pgfqpoint{4.318470in}{2.175208in}}%
\pgfpathlineto{\pgfqpoint{4.286787in}{2.155836in}}%
\pgfpathlineto{\pgfqpoint{4.257535in}{2.135657in}}%
\pgfpathlineto{\pgfqpoint{4.230821in}{2.114718in}}%
\pgfpathlineto{\pgfqpoint{4.206752in}{2.093065in}}%
\pgfpathlineto{\pgfqpoint{4.185430in}{2.070749in}}%
\pgfpathlineto{\pgfqpoint{4.166954in}{2.047821in}}%
\pgfpathlineto{\pgfqpoint{4.151423in}{2.024335in}}%
\pgfpathlineto{\pgfqpoint{4.138927in}{2.000348in}}%
\pgfpathlineto{\pgfqpoint{4.129556in}{1.975918in}}%
\pgfpathlineto{\pgfqpoint{4.123395in}{1.951108in}}%
\pgfpathlineto{\pgfqpoint{4.120522in}{1.925981in}}%
\pgfpathlineto{\pgfqpoint{4.121013in}{1.900604in}}%
\pgfpathlineto{\pgfqpoint{4.124937in}{1.875047in}}%
\pgfpathlineto{\pgfqpoint{4.132356in}{1.849382in}}%
\pgfpathlineto{\pgfqpoint{4.143326in}{1.823684in}}%
\pgfpathlineto{\pgfqpoint{4.157895in}{1.798032in}}%
\pgfpathlineto{\pgfqpoint{4.176102in}{1.772505in}}%
\pgfpathlineto{\pgfqpoint{4.197980in}{1.747187in}}%
\pgfpathlineto{\pgfqpoint{4.223548in}{1.722165in}}%
\pgfpathlineto{\pgfqpoint{4.252819in}{1.697527in}}%
\pgfpathlineto{\pgfqpoint{4.285790in}{1.673363in}}%
\pgfpathlineto{\pgfqpoint{4.322450in}{1.649767in}}%
\pgfpathlineto{\pgfqpoint{4.362772in}{1.626833in}}%
\pgfpathlineto{\pgfqpoint{4.406718in}{1.604659in}}%
\pgfpathlineto{\pgfqpoint{4.454232in}{1.583341in}}%
\pgfpathlineto{\pgfqpoint{4.505245in}{1.562980in}}%
\pgfpathlineto{\pgfqpoint{4.559672in}{1.543675in}}%
\pgfpathlineto{\pgfqpoint{4.617409in}{1.525525in}}%
\pgfpathlineto{\pgfqpoint{4.678337in}{1.508630in}}%
\pgfpathlineto{\pgfqpoint{4.742317in}{1.493089in}}%
\pgfpathlineto{\pgfqpoint{4.809192in}{1.479000in}}%
\pgfpathlineto{\pgfqpoint{4.878787in}{1.466456in}}%
\pgfpathlineto{\pgfqpoint{4.950905in}{1.455552in}}%
\pgfpathlineto{\pgfqpoint{5.025334in}{1.446377in}}%
\pgfpathlineto{\pgfqpoint{5.101837in}{1.439014in}}%
\pgfpathlineto{\pgfqpoint{5.180164in}{1.433546in}}%
\pgfpathlineto{\pgfqpoint{5.260040in}{1.430047in}}%
\pgfpathlineto{\pgfqpoint{5.341176in}{1.428584in}}%
\pgfpathlineto{\pgfqpoint{5.423265in}{1.429220in}}%
\pgfpathlineto{\pgfqpoint{5.505980in}{1.432006in}}%
\pgfpathlineto{\pgfqpoint{5.588983in}{1.436989in}}%
\pgfpathlineto{\pgfqpoint{5.671918in}{1.444202in}}%
\pgfpathlineto{\pgfqpoint{5.754420in}{1.453669in}}%
\pgfpathlineto{\pgfqpoint{5.836111in}{1.465406in}}%
\pgfpathlineto{\pgfqpoint{5.916605in}{1.479413in}}%
\pgfpathlineto{\pgfqpoint{5.995510in}{1.495680in}}%
\pgfpathlineto{\pgfqpoint{6.072431in}{1.514186in}}%
\pgfpathlineto{\pgfqpoint{6.146969in}{1.534893in}}%
\pgfpathlineto{\pgfqpoint{6.218730in}{1.557755in}}%
\pgfpathlineto{\pgfqpoint{6.287323in}{1.582707in}}%
\pgfpathlineto{\pgfqpoint{6.352363in}{1.609675in}}%
\pgfpathlineto{\pgfqpoint{6.413478in}{1.638570in}}%
\pgfpathlineto{\pgfqpoint{6.470309in}{1.669290in}}%
\pgfpathlineto{\pgfqpoint{6.522513in}{1.701720in}}%
\pgfpathlineto{\pgfqpoint{6.569766in}{1.735733in}}%
\pgfpathlineto{\pgfqpoint{6.611768in}{1.771191in}}%
\pgfpathlineto{\pgfqpoint{6.648245in}{1.807944in}}%
\pgfpathlineto{\pgfqpoint{6.678948in}{1.845835in}}%
\pgfpathlineto{\pgfqpoint{6.703661in}{1.884695in}}%
\pgfpathlineto{\pgfqpoint{6.722198in}{1.924348in}}%
\pgfpathlineto{\pgfqpoint{6.734410in}{1.964614in}}%
\pgfpathlineto{\pgfqpoint{6.740180in}{2.005305in}}%
\pgfpathlineto{\pgfqpoint{6.739429in}{2.046231in}}%
\pgfpathlineto{\pgfqpoint{6.732116in}{2.087199in}}%
\pgfpathlineto{\pgfqpoint{6.718236in}{2.128016in}}%
\pgfpathlineto{\pgfqpoint{6.697822in}{2.168489in}}%
\pgfpathlineto{\pgfqpoint{6.670944in}{2.208427in}}%
\pgfpathlineto{\pgfqpoint{6.637708in}{2.247642in}}%
\pgfpathlineto{\pgfqpoint{6.598256in}{2.285952in}}%
\pgfpathlineto{\pgfqpoint{6.552761in}{2.323181in}}%
\pgfpathlineto{\pgfqpoint{6.501430in}{2.359158in}}%
\pgfpathlineto{\pgfqpoint{6.444497in}{2.393723in}}%
\pgfpathlineto{\pgfqpoint{6.382226in}{2.426724in}}%
\pgfpathlineto{\pgfqpoint{6.314904in}{2.458019in}}%
\pgfpathlineto{\pgfqpoint{6.242840in}{2.487476in}}%
\pgfpathlineto{\pgfqpoint{6.166362in}{2.514975in}}%
\pgfpathlineto{\pgfqpoint{6.085816in}{2.540408in}}%
\pgfpathlineto{\pgfqpoint{6.001560in}{2.563678in}}%
\pgfpathlineto{\pgfqpoint{5.913965in}{2.584700in}}%
\pgfpathlineto{\pgfqpoint{5.823407in}{2.603403in}}%
\pgfpathlineto{\pgfqpoint{5.730271in}{2.619725in}}%
\pgfpathlineto{\pgfqpoint{5.634943in}{2.633618in}}%
\pgfpathlineto{\pgfqpoint{5.537812in}{2.645045in}}%
\pgfpathlineto{\pgfqpoint{5.439262in}{2.653979in}}%
\pgfpathlineto{\pgfqpoint{5.339678in}{2.660406in}}%
\pgfpathlineto{\pgfqpoint{5.239438in}{2.664319in}}%
\pgfpathlineto{\pgfqpoint{5.138914in}{2.665724in}}%
\pgfpathlineto{\pgfqpoint{5.038469in}{2.664633in}}%
\pgfpathlineto{\pgfqpoint{4.938459in}{2.661070in}}%
\pgfpathlineto{\pgfqpoint{4.839229in}{2.655064in}}%
\pgfpathlineto{\pgfqpoint{4.741114in}{2.646653in}}%
\pgfpathlineto{\pgfqpoint{4.644438in}{2.635881in}}%
\pgfpathlineto{\pgfqpoint{4.549512in}{2.622799in}}%
\pgfpathlineto{\pgfqpoint{4.456637in}{2.607463in}}%
\pgfpathlineto{\pgfqpoint{4.366101in}{2.589935in}}%
\pgfpathlineto{\pgfqpoint{4.278180in}{2.570282in}}%
\pgfpathlineto{\pgfqpoint{4.193140in}{2.548574in}}%
\pgfpathlineto{\pgfqpoint{4.111231in}{2.524886in}}%
\pgfpathlineto{\pgfqpoint{4.032694in}{2.499298in}}%
\pgfpathlineto{\pgfqpoint{3.957760in}{2.471892in}}%
\pgfpathlineto{\pgfqpoint{3.886645in}{2.442752in}}%
\pgfpathlineto{\pgfqpoint{3.819557in}{2.411968in}}%
\pgfpathlineto{\pgfqpoint{3.756692in}{2.379630in}}%
\pgfpathlineto{\pgfqpoint{3.698235in}{2.345833in}}%
\pgfpathlineto{\pgfqpoint{3.644361in}{2.310672in}}%
\pgfpathlineto{\pgfqpoint{3.595235in}{2.274248in}}%
\pgfpathlineto{\pgfqpoint{3.551013in}{2.236660in}}%
\pgfpathlineto{\pgfqpoint{3.511838in}{2.198012in}}%
\pgfpathlineto{\pgfqpoint{3.477847in}{2.158411in}}%
\pgfpathlineto{\pgfqpoint{3.449164in}{2.117965in}}%
\pgfpathlineto{\pgfqpoint{3.425904in}{2.076785in}}%
\pgfpathlineto{\pgfqpoint{3.408170in}{2.034983in}}%
\pgfpathlineto{\pgfqpoint{3.396059in}{1.992675in}}%
\pgfpathlineto{\pgfqpoint{3.389652in}{1.949978in}}%
\pgfpathlineto{\pgfqpoint{3.389023in}{1.907013in}}%
\pgfpathlineto{\pgfqpoint{3.394231in}{1.863903in}}%
\pgfpathlineto{\pgfqpoint{3.405326in}{1.820772in}}%
\pgfpathlineto{\pgfqpoint{3.422344in}{1.777747in}}%
\pgfpathlineto{\pgfqpoint{3.445307in}{1.734958in}}%
\pgfpathlineto{\pgfqpoint{3.474225in}{1.692535in}}%
\pgfpathlineto{\pgfqpoint{3.509092in}{1.650614in}}%
\pgfpathlineto{\pgfqpoint{3.549888in}{1.609329in}}%
\pgfpathlineto{\pgfqpoint{3.596576in}{1.568817in}}%
\pgfpathlineto{\pgfqpoint{3.649101in}{1.529217in}}%
\pgfpathlineto{\pgfqpoint{3.707394in}{1.490669in}}%
\pgfpathlineto{\pgfqpoint{3.771364in}{1.453313in}}%
\pgfpathlineto{\pgfqpoint{3.840902in}{1.417292in}}%
\pgfpathlineto{\pgfqpoint{3.915879in}{1.382746in}}%
\pgfpathlineto{\pgfqpoint{3.996144in}{1.349816in}}%
\pgfpathlineto{\pgfqpoint{4.081526in}{1.318644in}}%
\pgfpathlineto{\pgfqpoint{4.171832in}{1.289367in}}%
\pgfpathlineto{\pgfqpoint{4.266843in}{1.262124in}}%
\pgfpathlineto{\pgfqpoint{4.366320in}{1.237047in}}%
\pgfpathlineto{\pgfqpoint{4.469998in}{1.214269in}}%
\pgfpathlineto{\pgfqpoint{4.577590in}{1.193916in}}%
\pgfpathlineto{\pgfqpoint{4.688784in}{1.176110in}}%
\pgfpathlineto{\pgfqpoint{4.803243in}{1.160967in}}%
\pgfpathlineto{\pgfqpoint{4.920608in}{1.148597in}}%
\pgfpathlineto{\pgfqpoint{5.040494in}{1.139102in}}%
\pgfpathlineto{\pgfqpoint{5.162496in}{1.132575in}}%
\pgfpathlineto{\pgfqpoint{5.286187in}{1.129100in}}%
\pgfpathlineto{\pgfqpoint{5.411117in}{1.128751in}}%
\pgfpathlineto{\pgfqpoint{5.536818in}{1.131592in}}%
\pgfpathlineto{\pgfqpoint{5.662805in}{1.137671in}}%
\pgfpathlineto{\pgfqpoint{5.788576in}{1.147028in}}%
\pgfpathlineto{\pgfqpoint{5.913615in}{1.159685in}}%
\pgfpathlineto{\pgfqpoint{6.037395in}{1.175651in}}%
\pgfpathlineto{\pgfqpoint{6.159381in}{1.194921in}}%
\pgfpathlineto{\pgfqpoint{6.279030in}{1.217473in}}%
\pgfpathlineto{\pgfqpoint{6.395797in}{1.243267in}}%
\pgfpathlineto{\pgfqpoint{6.509137in}{1.272249in}}%
\pgfpathlineto{\pgfqpoint{6.618508in}{1.304347in}}%
\pgfpathlineto{\pgfqpoint{6.723377in}{1.339472in}}%
\pgfpathlineto{\pgfqpoint{6.823219in}{1.377516in}}%
\pgfpathlineto{\pgfqpoint{6.917525in}{1.418356in}}%
\pgfpathlineto{\pgfqpoint{7.005803in}{1.461853in}}%
\pgfpathlineto{\pgfqpoint{7.087583in}{1.507848in}}%
\pgfpathlineto{\pgfqpoint{7.162421in}{1.556171in}}%
\pgfpathlineto{\pgfqpoint{7.229898in}{1.606633in}}%
\pgfpathlineto{\pgfqpoint{7.289633in}{1.659034in}}%
\pgfpathlineto{\pgfqpoint{7.341275in}{1.713158in}}%
\pgfpathlineto{\pgfqpoint{7.384513in}{1.768780in}}%
\pgfpathlineto{\pgfqpoint{7.419078in}{1.825664in}}%
\pgfpathlineto{\pgfqpoint{7.444741in}{1.883563in}}%
\pgfpathlineto{\pgfqpoint{7.461321in}{1.942224in}}%
\pgfpathlineto{\pgfqpoint{7.468683in}{2.001388in}}%
\pgfpathlineto{\pgfqpoint{7.466738in}{2.060792in}}%
\pgfpathlineto{\pgfqpoint{7.455447in}{2.120169in}}%
\pgfpathlineto{\pgfqpoint{7.434821in}{2.179254in}}%
\pgfpathlineto{\pgfqpoint{7.404918in}{2.237780in}}%
\pgfpathlineto{\pgfqpoint{7.365847in}{2.295485in}}%
\pgfpathlineto{\pgfqpoint{7.317762in}{2.352110in}}%
\pgfpathlineto{\pgfqpoint{7.260866in}{2.407402in}}%
\pgfpathlineto{\pgfqpoint{7.195403in}{2.461118in}}%
\pgfpathlineto{\pgfqpoint{7.121664in}{2.513022in}}%
\pgfpathlineto{\pgfqpoint{7.039978in}{2.562890in}}%
\pgfpathlineto{\pgfqpoint{6.950710in}{2.610507in}}%
\pgfpathlineto{\pgfqpoint{6.854261in}{2.655676in}}%
\pgfpathlineto{\pgfqpoint{6.751065in}{2.698209in}}%
\pgfpathlineto{\pgfqpoint{6.641581in}{2.737936in}}%
\pgfpathlineto{\pgfqpoint{6.526294in}{2.774702in}}%
\pgfpathlineto{\pgfqpoint{6.405710in}{2.808367in}}%
\pgfpathlineto{\pgfqpoint{6.280351in}{2.838809in}}%
\pgfpathlineto{\pgfqpoint{6.150756in}{2.865920in}}%
\pgfpathlineto{\pgfqpoint{6.017471in}{2.889612in}}%
\pgfpathlineto{\pgfqpoint{5.881051in}{2.909812in}}%
\pgfpathlineto{\pgfqpoint{5.742054in}{2.926462in}}%
\pgfpathlineto{\pgfqpoint{5.601038in}{2.939523in}}%
\pgfpathlineto{\pgfqpoint{5.458560in}{2.948970in}}%
\pgfpathlineto{\pgfqpoint{5.315172in}{2.954793in}}%
\pgfpathlineto{\pgfqpoint{5.171417in}{2.956997in}}%
\pgfpathlineto{\pgfqpoint{5.027830in}{2.955601in}}%
\pgfpathlineto{\pgfqpoint{4.884934in}{2.950637in}}%
\pgfpathlineto{\pgfqpoint{4.743240in}{2.942148in}}%
\pgfpathlineto{\pgfqpoint{4.603242in}{2.930192in}}%
\pgfpathlineto{\pgfqpoint{4.465419in}{2.914835in}}%
\pgfpathlineto{\pgfqpoint{4.330233in}{2.896154in}}%
\pgfpathlineto{\pgfqpoint{4.198130in}{2.874235in}}%
\pgfpathlineto{\pgfqpoint{4.069536in}{2.849174in}}%
\pgfpathlineto{\pgfqpoint{3.944858in}{2.821074in}}%
\pgfpathlineto{\pgfqpoint{3.824484in}{2.790044in}}%
\pgfpathlineto{\pgfqpoint{3.708785in}{2.756203in}}%
\pgfpathlineto{\pgfqpoint{3.598109in}{2.719672in}}%
\pgfpathlineto{\pgfqpoint{3.492789in}{2.680580in}}%
\pgfpathlineto{\pgfqpoint{3.393137in}{2.639062in}}%
\pgfpathlineto{\pgfqpoint{3.299444in}{2.595255in}}%
\pgfpathlineto{\pgfqpoint{3.211987in}{2.549303in}}%
\pgfpathlineto{\pgfqpoint{3.131020in}{2.501350in}}%
\pgfpathlineto{\pgfqpoint{3.056782in}{2.451548in}}%
\pgfpathlineto{\pgfqpoint{2.989493in}{2.400050in}}%
\pgfpathlineto{\pgfqpoint{2.929355in}{2.347012in}}%
\pgfpathlineto{\pgfqpoint{2.876552in}{2.292594in}}%
\pgfpathlineto{\pgfqpoint{2.831251in}{2.236959in}}%
\pgfpathlineto{\pgfqpoint{2.793601in}{2.180269in}}%
\pgfpathlineto{\pgfqpoint{2.763735in}{2.122694in}}%
\pgfpathlineto{\pgfqpoint{2.741766in}{2.064403in}}%
\pgfpathlineto{\pgfqpoint{2.727792in}{2.005566in}}%
\pgfpathlineto{\pgfqpoint{2.721890in}{1.946360in}}%
\pgfpathlineto{\pgfqpoint{2.724120in}{1.886958in}}%
\pgfpathlineto{\pgfqpoint{2.734526in}{1.827540in}}%
\pgfpathlineto{\pgfqpoint{2.753128in}{1.768284in}}%
\pgfpathlineto{\pgfqpoint{2.779931in}{1.709373in}}%
\pgfpathlineto{\pgfqpoint{2.814917in}{1.650988in}}%
\pgfpathlineto{\pgfqpoint{2.858049in}{1.593313in}}%
\pgfpathlineto{\pgfqpoint{2.909268in}{1.536534in}}%
\pgfpathlineto{\pgfqpoint{2.968492in}{1.480835in}}%
\pgfpathlineto{\pgfqpoint{3.035618in}{1.426403in}}%
\pgfpathlineto{\pgfqpoint{3.110520in}{1.373424in}}%
\pgfpathlineto{\pgfqpoint{3.193046in}{1.322082in}}%
\pgfpathlineto{\pgfqpoint{3.283020in}{1.272562in}}%
\pgfpathlineto{\pgfqpoint{3.380240in}{1.225048in}}%
\pgfpathlineto{\pgfqpoint{3.484481in}{1.179721in}}%
\pgfpathlineto{\pgfqpoint{3.595486in}{1.136759in}}%
\pgfpathlineto{\pgfqpoint{3.712975in}{1.096338in}}%
\pgfpathlineto{\pgfqpoint{3.836639in}{1.058629in}}%
\pgfpathlineto{\pgfqpoint{3.966142in}{1.023798in}}%
\pgfpathlineto{\pgfqpoint{4.101119in}{0.992008in}}%
\pgfpathlineto{\pgfqpoint{4.241177in}{0.963412in}}%
\pgfpathlineto{\pgfqpoint{4.385897in}{0.938159in}}%
\pgfpathlineto{\pgfqpoint{4.534830in}{0.916386in}}%
\pgfpathlineto{\pgfqpoint{4.687503in}{0.898224in}}%
\pgfpathlineto{\pgfqpoint{4.843416in}{0.883792in}}%
\pgfpathlineto{\pgfqpoint{5.002042in}{0.873200in}}%
\pgfpathlineto{\pgfqpoint{5.162833in}{0.866543in}}%
\pgfpathlineto{\pgfqpoint{5.325219in}{0.863903in}}%
\pgfpathlineto{\pgfqpoint{5.488607in}{0.865352in}}%
\pgfpathlineto{\pgfqpoint{5.652387in}{0.870941in}}%
\pgfpathlineto{\pgfqpoint{5.815935in}{0.880710in}}%
\pgfpathlineto{\pgfqpoint{5.978611in}{0.894679in}}%
\pgfpathlineto{\pgfqpoint{6.139763in}{0.912853in}}%
\pgfpathlineto{\pgfqpoint{6.298735in}{0.935218in}}%
\pgfpathlineto{\pgfqpoint{6.454864in}{0.961741in}}%
\pgfpathlineto{\pgfqpoint{6.607485in}{0.992370in}}%
\pgfpathlineto{\pgfqpoint{6.755936in}{1.027035in}}%
\pgfpathlineto{\pgfqpoint{6.899563in}{1.065644in}}%
\pgfpathlineto{\pgfqpoint{7.037719in}{1.108088in}}%
\pgfpathlineto{\pgfqpoint{7.169770in}{1.154236in}}%
\pgfpathlineto{\pgfqpoint{7.295103in}{1.203939in}}%
\pgfpathlineto{\pgfqpoint{7.413122in}{1.257030in}}%
\pgfpathlineto{\pgfqpoint{7.523260in}{1.313322in}}%
\pgfpathlineto{\pgfqpoint{7.624975in}{1.372610in}}%
\pgfpathlineto{\pgfqpoint{7.717761in}{1.434673in}}%
\pgfpathlineto{\pgfqpoint{7.801145in}{1.499273in}}%
\pgfpathlineto{\pgfqpoint{7.874695in}{1.566159in}}%
\pgfpathlineto{\pgfqpoint{7.938020in}{1.635064in}}%
\pgfpathlineto{\pgfqpoint{7.990777in}{1.705710in}}%
\pgfpathlineto{\pgfqpoint{8.032667in}{1.777808in}}%
\pgfpathlineto{\pgfqpoint{8.063443in}{1.851059in}}%
\pgfpathlineto{\pgfqpoint{8.082910in}{1.925156in}}%
\pgfpathlineto{\pgfqpoint{8.090926in}{1.999787in}}%
\pgfpathlineto{\pgfqpoint{8.087403in}{2.074635in}}%
\pgfpathlineto{\pgfqpoint{8.072310in}{2.149381in}}%
\pgfpathlineto{\pgfqpoint{8.045669in}{2.223706in}}%
\pgfpathlineto{\pgfqpoint{8.007561in}{2.297290in}}%
\pgfpathlineto{\pgfqpoint{7.958121in}{2.369819in}}%
\pgfpathlineto{\pgfqpoint{7.897537in}{2.440981in}}%
\pgfpathlineto{\pgfqpoint{7.826054in}{2.510472in}}%
\pgfpathlineto{\pgfqpoint{7.743964in}{2.577998in}}%
\pgfpathlineto{\pgfqpoint{7.651614in}{2.643273in}}%
\pgfpathlineto{\pgfqpoint{7.549395in}{2.706022in}}%
\pgfpathlineto{\pgfqpoint{7.437743in}{2.765985in}}%
\pgfpathlineto{\pgfqpoint{7.317137in}{2.822916in}}%
\pgfpathlineto{\pgfqpoint{7.188095in}{2.876584in}}%
\pgfpathlineto{\pgfqpoint{7.051170in}{2.926775in}}%
\pgfpathlineto{\pgfqpoint{6.906945in}{2.973292in}}%
\pgfpathlineto{\pgfqpoint{6.756033in}{3.015958in}}%
\pgfpathlineto{\pgfqpoint{6.599071in}{3.054614in}}%
\pgfpathlineto{\pgfqpoint{6.436714in}{3.089121in}}%
\pgfpathlineto{\pgfqpoint{6.269636in}{3.119359in}}%
\pgfpathlineto{\pgfqpoint{6.098522in}{3.145228in}}%
\pgfpathlineto{\pgfqpoint{5.924065in}{3.166650in}}%
\pgfpathlineto{\pgfqpoint{5.746963in}{3.183564in}}%
\pgfpathlineto{\pgfqpoint{5.567916in}{3.195931in}}%
\pgfpathlineto{\pgfqpoint{5.387619in}{3.203730in}}%
\pgfpathlineto{\pgfqpoint{5.206765in}{3.206959in}}%
\pgfpathlineto{\pgfqpoint{5.026037in}{3.205633in}}%
\pgfpathlineto{\pgfqpoint{4.846104in}{3.199786in}}%
\pgfpathlineto{\pgfqpoint{4.667626in}{3.189468in}}%
\pgfpathlineto{\pgfqpoint{4.491242in}{3.174745in}}%
\pgfpathlineto{\pgfqpoint{4.317577in}{3.155697in}}%
\pgfpathlineto{\pgfqpoint{4.147233in}{3.132420in}}%
\pgfpathlineto{\pgfqpoint{3.980792in}{3.105021in}}%
\pgfpathlineto{\pgfqpoint{3.818813in}{3.073621in}}%
\pgfpathlineto{\pgfqpoint{3.661831in}{3.038352in}}%
\pgfpathlineto{\pgfqpoint{3.510357in}{2.999356in}}%
\pgfpathlineto{\pgfqpoint{3.364878in}{2.956785in}}%
\pgfpathlineto{\pgfqpoint{3.225852in}{2.910801in}}%
\pgfpathlineto{\pgfqpoint{3.093714in}{2.861573in}}%
\pgfpathlineto{\pgfqpoint{2.968871in}{2.809277in}}%
\pgfpathlineto{\pgfqpoint{2.851705in}{2.754098in}}%
\pgfpathlineto{\pgfqpoint{2.742572in}{2.696225in}}%
\pgfpathlineto{\pgfqpoint{2.641801in}{2.635853in}}%
\pgfpathlineto{\pgfqpoint{2.549694in}{2.573184in}}%
\pgfpathlineto{\pgfqpoint{2.466528in}{2.508421in}}%
\pgfpathlineto{\pgfqpoint{2.392554in}{2.441776in}}%
\pgfpathlineto{\pgfqpoint{2.327998in}{2.373460in}}%
\pgfpathlineto{\pgfqpoint{2.273059in}{2.303690in}}%
\pgfpathlineto{\pgfqpoint{2.227910in}{2.232686in}}%
\pgfpathlineto{\pgfqpoint{2.192700in}{2.160671in}}%
\pgfpathlineto{\pgfqpoint{2.167552in}{2.087868in}}%
\pgfpathlineto{\pgfqpoint{2.152561in}{2.014505in}}%
\pgfpathlineto{\pgfqpoint{2.147799in}{1.940810in}}%
\pgfpathlineto{\pgfqpoint{2.153310in}{1.867015in}}%
\pgfpathlineto{\pgfqpoint{2.169112in}{1.793350in}}%
\pgfpathlineto{\pgfqpoint{2.195197in}{1.720047in}}%
\pgfpathlineto{\pgfqpoint{2.231528in}{1.647341in}}%
\pgfpathlineto{\pgfqpoint{2.278044in}{1.575465in}}%
\pgfpathlineto{\pgfqpoint{2.334651in}{1.504651in}}%
\pgfpathlineto{\pgfqpoint{2.334651in}{1.504651in}}%
\pgfusepath{stroke}%
\end{pgfscope}%
\begin{pgfscope}%
\pgfpathrectangle{\pgfqpoint{1.250000in}{0.400000in}}{\pgfqpoint{7.750000in}{3.200000in}} %
\pgfusepath{clip}%
\pgfsetrectcap%
\pgfsetroundjoin%
\pgfsetlinewidth{1.003750pt}%
\definecolor{currentstroke}{rgb}{1.000000,0.000000,0.000000}%
\pgfsetstrokecolor{currentstroke}%
\pgfsetdash{}{0pt}%
\pgfpathmoveto{\pgfqpoint{5.129965in}{2.000000in}}%
\pgfpathlineto{\pgfqpoint{5.128216in}{2.001697in}}%
\pgfpathlineto{\pgfqpoint{5.121059in}{2.001889in}}%
\pgfpathlineto{\pgfqpoint{5.119136in}{2.000672in}}%
\pgfpathlineto{\pgfqpoint{5.119444in}{1.999185in}}%
\pgfpathlineto{\pgfqpoint{5.122070in}{1.998039in}}%
\pgfpathlineto{\pgfqpoint{5.126129in}{1.997878in}}%
\pgfpathlineto{\pgfqpoint{5.129496in}{1.998991in}}%
\pgfpathlineto{\pgfqpoint{5.129948in}{2.000772in}}%
\pgfpathlineto{\pgfqpoint{5.127518in}{2.002108in}}%
\pgfpathlineto{\pgfqpoint{5.120805in}{2.001828in}}%
\pgfpathlineto{\pgfqpoint{5.119304in}{2.000492in}}%
\pgfpathlineto{\pgfqpoint{5.120321in}{1.998925in}}%
\pgfpathlineto{\pgfqpoint{5.123977in}{1.997955in}}%
\pgfpathlineto{\pgfqpoint{5.128295in}{1.998353in}}%
\pgfpathlineto{\pgfqpoint{5.130515in}{1.999788in}}%
\pgfpathlineto{\pgfqpoint{5.129896in}{2.001342in}}%
\pgfpathlineto{\pgfqpoint{5.127180in}{2.002336in}}%
\pgfpathlineto{\pgfqpoint{5.120640in}{2.001590in}}%
\pgfpathlineto{\pgfqpoint{5.119607in}{2.000048in}}%
\pgfpathlineto{\pgfqpoint{5.121589in}{1.998475in}}%
\pgfpathlineto{\pgfqpoint{5.125919in}{1.997934in}}%
\pgfpathlineto{\pgfqpoint{5.129632in}{1.998746in}}%
\pgfpathlineto{\pgfqpoint{5.130951in}{2.000191in}}%
\pgfpathlineto{\pgfqpoint{5.129808in}{2.001558in}}%
\pgfpathlineto{\pgfqpoint{5.126805in}{2.002339in}}%
\pgfpathlineto{\pgfqpoint{5.120233in}{2.000933in}}%
\pgfpathlineto{\pgfqpoint{5.120267in}{1.999160in}}%
\pgfpathlineto{\pgfqpoint{5.123339in}{1.997925in}}%
\pgfpathlineto{\pgfqpoint{5.130327in}{1.998891in}}%
\pgfpathlineto{\pgfqpoint{5.131064in}{2.000319in}}%
\pgfpathlineto{\pgfqpoint{5.129483in}{2.001613in}}%
\pgfpathlineto{\pgfqpoint{5.126023in}{2.002232in}}%
\pgfpathlineto{\pgfqpoint{5.121998in}{2.001685in}}%
\pgfpathlineto{\pgfqpoint{5.119956in}{2.000018in}}%
\pgfpathlineto{\pgfqpoint{5.121326in}{1.998351in}}%
\pgfpathlineto{\pgfqpoint{5.128247in}{1.997886in}}%
\pgfpathlineto{\pgfqpoint{5.130557in}{1.999005in}}%
\pgfpathlineto{\pgfqpoint{5.130730in}{2.000505in}}%
\pgfpathlineto{\pgfqpoint{5.128448in}{2.001775in}}%
\pgfpathlineto{\pgfqpoint{5.124377in}{2.002111in}}%
\pgfpathlineto{\pgfqpoint{5.120684in}{2.001093in}}%
\pgfpathlineto{\pgfqpoint{5.119853in}{1.999342in}}%
\pgfpathlineto{\pgfqpoint{5.121856in}{1.997963in}}%
\pgfpathlineto{\pgfqpoint{5.128482in}{1.997972in}}%
\pgfpathlineto{\pgfqpoint{5.130390in}{1.999283in}}%
\pgfpathlineto{\pgfqpoint{5.129750in}{2.000943in}}%
\pgfpathlineto{\pgfqpoint{5.126321in}{2.002042in}}%
\pgfpathlineto{\pgfqpoint{5.122056in}{2.001810in}}%
\pgfpathlineto{\pgfqpoint{5.119530in}{2.000495in}}%
\pgfpathlineto{\pgfqpoint{5.119688in}{1.998926in}}%
\pgfpathlineto{\pgfqpoint{5.122043in}{1.997801in}}%
\pgfpathlineto{\pgfqpoint{5.128725in}{1.998236in}}%
\pgfpathlineto{\pgfqpoint{5.130161in}{1.999802in}}%
\pgfpathlineto{\pgfqpoint{5.128380in}{2.001504in}}%
\pgfpathlineto{\pgfqpoint{5.124229in}{2.002149in}}%
\pgfpathlineto{\pgfqpoint{5.118964in}{2.000167in}}%
\pgfpathlineto{\pgfqpoint{5.119737in}{1.998746in}}%
\pgfpathlineto{\pgfqpoint{5.122539in}{1.997796in}}%
\pgfpathlineto{\pgfqpoint{5.126374in}{1.997777in}}%
\pgfpathlineto{\pgfqpoint{5.129483in}{1.998879in}}%
\pgfpathlineto{\pgfqpoint{5.129857in}{2.000660in}}%
\pgfpathlineto{\pgfqpoint{5.127057in}{2.002023in}}%
\pgfpathlineto{\pgfqpoint{5.119999in}{2.001411in}}%
\pgfpathlineto{\pgfqpoint{5.118915in}{2.000049in}}%
\pgfpathlineto{\pgfqpoint{5.120207in}{1.998665in}}%
\pgfpathlineto{\pgfqpoint{5.123678in}{1.997873in}}%
\pgfpathlineto{\pgfqpoint{5.127923in}{1.998296in}}%
\pgfpathlineto{\pgfqpoint{5.130189in}{1.999855in}}%
\pgfpathlineto{\pgfqpoint{5.129223in}{2.001498in}}%
\pgfpathlineto{\pgfqpoint{5.122485in}{2.002281in}}%
\pgfpathlineto{\pgfqpoint{5.119839in}{2.001300in}}%
\pgfpathlineto{\pgfqpoint{5.119225in}{1.999824in}}%
\pgfpathlineto{\pgfqpoint{5.121216in}{1.998437in}}%
\pgfpathlineto{\pgfqpoint{5.125380in}{1.997938in}}%
\pgfpathlineto{\pgfqpoint{5.129307in}{1.998862in}}%
\pgfpathlineto{\pgfqpoint{5.130443in}{2.000515in}}%
\pgfpathlineto{\pgfqpoint{5.128851in}{2.001897in}}%
\pgfpathlineto{\pgfqpoint{5.122218in}{2.002159in}}%
\pgfpathlineto{\pgfqpoint{5.119882in}{2.000913in}}%
\pgfpathlineto{\pgfqpoint{5.120088in}{1.999219in}}%
\pgfpathlineto{\pgfqpoint{5.123218in}{1.998003in}}%
\pgfpathlineto{\pgfqpoint{5.127496in}{1.998066in}}%
\pgfpathlineto{\pgfqpoint{5.130354in}{1.999256in}}%
\pgfpathlineto{\pgfqpoint{5.130663in}{2.000800in}}%
\pgfpathlineto{\pgfqpoint{5.128703in}{2.002008in}}%
\pgfpathlineto{\pgfqpoint{5.121831in}{2.001835in}}%
\pgfpathlineto{\pgfqpoint{5.119989in}{2.000241in}}%
\pgfpathlineto{\pgfqpoint{5.121508in}{1.998484in}}%
\pgfpathlineto{\pgfqpoint{5.125337in}{1.997747in}}%
\pgfpathlineto{\pgfqpoint{5.130932in}{1.999474in}}%
\pgfpathlineto{\pgfqpoint{5.130605in}{2.000939in}}%
\pgfpathlineto{\pgfqpoint{5.128117in}{2.002039in}}%
\pgfpathlineto{\pgfqpoint{5.124303in}{2.002246in}}%
\pgfpathlineto{\pgfqpoint{5.120877in}{2.001258in}}%
\pgfpathlineto{\pgfqpoint{5.120109in}{1.999447in}}%
\pgfpathlineto{\pgfqpoint{5.122581in}{1.997996in}}%
\pgfpathlineto{\pgfqpoint{5.129495in}{1.998289in}}%
\pgfpathlineto{\pgfqpoint{5.130924in}{1.999606in}}%
\pgfpathlineto{\pgfqpoint{5.129968in}{2.001094in}}%
\pgfpathlineto{\pgfqpoint{5.126607in}{2.002064in}}%
\pgfpathlineto{\pgfqpoint{5.122308in}{2.001801in}}%
\pgfpathlineto{\pgfqpoint{5.119777in}{2.000372in}}%
\pgfpathlineto{\pgfqpoint{5.120313in}{1.998717in}}%
\pgfpathlineto{\pgfqpoint{5.123139in}{1.997674in}}%
\pgfpathlineto{\pgfqpoint{5.129614in}{1.998412in}}%
\pgfpathlineto{\pgfqpoint{5.130626in}{1.999873in}}%
\pgfpathlineto{\pgfqpoint{5.128876in}{2.001401in}}%
\pgfpathlineto{\pgfqpoint{5.124645in}{2.002048in}}%
\pgfpathlineto{\pgfqpoint{5.120673in}{2.001260in}}%
\pgfpathlineto{\pgfqpoint{5.119260in}{1.999741in}}%
\pgfpathlineto{\pgfqpoint{5.120468in}{1.998338in}}%
\pgfpathlineto{\pgfqpoint{5.127088in}{1.997740in}}%
\pgfpathlineto{\pgfqpoint{5.129805in}{1.998866in}}%
\pgfpathlineto{\pgfqpoint{5.130043in}{2.000562in}}%
\pgfpathlineto{\pgfqpoint{5.127179in}{2.001910in}}%
\pgfpathlineto{\pgfqpoint{5.122870in}{2.002021in}}%
\pgfpathlineto{\pgfqpoint{5.119764in}{2.000987in}}%
\pgfpathlineto{\pgfqpoint{5.119043in}{1.999518in}}%
\pgfpathlineto{\pgfqpoint{5.120653in}{1.998248in}}%
\pgfpathlineto{\pgfqpoint{5.123975in}{1.997664in}}%
\pgfpathlineto{\pgfqpoint{5.127800in}{1.998146in}}%
\pgfpathlineto{\pgfqpoint{5.130007in}{1.999710in}}%
\pgfpathlineto{\pgfqpoint{5.128854in}{2.001456in}}%
\pgfpathlineto{\pgfqpoint{5.125337in}{2.002304in}}%
\pgfpathlineto{\pgfqpoint{5.119294in}{2.000860in}}%
\pgfpathlineto{\pgfqpoint{5.119147in}{1.999394in}}%
\pgfpathlineto{\pgfqpoint{5.121294in}{1.998179in}}%
\pgfpathlineto{\pgfqpoint{5.125124in}{1.997805in}}%
\pgfpathlineto{\pgfqpoint{5.128901in}{1.998711in}}%
\pgfpathlineto{\pgfqpoint{5.129988in}{2.000517in}}%
\pgfpathlineto{\pgfqpoint{5.127913in}{2.001986in}}%
\pgfpathlineto{\pgfqpoint{5.121158in}{2.001958in}}%
\pgfpathlineto{\pgfqpoint{5.119337in}{2.000685in}}%
\pgfpathlineto{\pgfqpoint{5.119878in}{1.999117in}}%
\pgfpathlineto{\pgfqpoint{5.122987in}{1.997999in}}%
\pgfpathlineto{\pgfqpoint{5.127286in}{1.998093in}}%
\pgfpathlineto{\pgfqpoint{5.130130in}{1.999411in}}%
\pgfpathlineto{\pgfqpoint{5.130041in}{2.001078in}}%
\pgfpathlineto{\pgfqpoint{5.127591in}{2.002237in}}%
\pgfpathlineto{\pgfqpoint{5.121039in}{2.001798in}}%
\pgfpathlineto{\pgfqpoint{5.119640in}{2.000320in}}%
\pgfpathlineto{\pgfqpoint{5.121167in}{1.998655in}}%
\pgfpathlineto{\pgfqpoint{5.125288in}{1.997900in}}%
\pgfpathlineto{\pgfqpoint{5.129191in}{1.998528in}}%
\pgfpathlineto{\pgfqpoint{5.130885in}{1.999933in}}%
\pgfpathlineto{\pgfqpoint{5.130069in}{2.001381in}}%
\pgfpathlineto{\pgfqpoint{5.127299in}{2.002307in}}%
\pgfpathlineto{\pgfqpoint{5.120626in}{2.001343in}}%
\pgfpathlineto{\pgfqpoint{5.119979in}{1.999626in}}%
\pgfpathlineto{\pgfqpoint{5.122635in}{1.998143in}}%
\pgfpathlineto{\pgfqpoint{5.130060in}{1.998736in}}%
\pgfpathlineto{\pgfqpoint{5.131110in}{2.000122in}}%
\pgfpathlineto{\pgfqpoint{5.129809in}{2.001466in}}%
\pgfpathlineto{\pgfqpoint{5.126529in}{2.002224in}}%
\pgfpathlineto{\pgfqpoint{5.122485in}{2.001896in}}%
\pgfpathlineto{\pgfqpoint{5.119975in}{2.000431in}}%
\pgfpathlineto{\pgfqpoint{5.120723in}{1.998695in}}%
\pgfpathlineto{\pgfqpoint{5.123972in}{1.997708in}}%
\pgfpathlineto{\pgfqpoint{5.130381in}{1.998812in}}%
\pgfpathlineto{\pgfqpoint{5.130957in}{2.000246in}}%
\pgfpathlineto{\pgfqpoint{5.129100in}{2.001570in}}%
\pgfpathlineto{\pgfqpoint{5.125174in}{2.002111in}}%
\pgfpathlineto{\pgfqpoint{5.121108in}{2.001286in}}%
\pgfpathlineto{\pgfqpoint{5.119762in}{1.999570in}}%
\pgfpathlineto{\pgfqpoint{5.121418in}{1.998112in}}%
\pgfpathlineto{\pgfqpoint{5.128115in}{1.997855in}}%
\pgfpathlineto{\pgfqpoint{5.130360in}{1.999044in}}%
\pgfpathlineto{\pgfqpoint{5.130289in}{2.000639in}}%
\pgfpathlineto{\pgfqpoint{5.127479in}{2.001890in}}%
\pgfpathlineto{\pgfqpoint{5.123126in}{2.001965in}}%
\pgfpathlineto{\pgfqpoint{5.119962in}{2.000759in}}%
\pgfpathlineto{\pgfqpoint{5.119605in}{1.999132in}}%
\pgfpathlineto{\pgfqpoint{5.121643in}{1.997904in}}%
\pgfpathlineto{\pgfqpoint{5.128323in}{1.998082in}}%
\pgfpathlineto{\pgfqpoint{5.130154in}{1.999568in}}%
\pgfpathlineto{\pgfqpoint{5.128935in}{2.001313in}}%
\pgfpathlineto{\pgfqpoint{5.125098in}{2.002169in}}%
\pgfpathlineto{\pgfqpoint{5.119155in}{2.000405in}}%
\pgfpathlineto{\pgfqpoint{5.119510in}{1.998922in}}%
\pgfpathlineto{\pgfqpoint{5.121948in}{1.997855in}}%
\pgfpathlineto{\pgfqpoint{5.128900in}{1.998530in}}%
\pgfpathlineto{\pgfqpoint{5.129934in}{2.000296in}}%
\pgfpathlineto{\pgfqpoint{5.127525in}{2.001864in}}%
\pgfpathlineto{\pgfqpoint{5.120312in}{2.001568in}}%
\pgfpathlineto{\pgfqpoint{5.118932in}{2.000244in}}%
\pgfpathlineto{\pgfqpoint{5.119876in}{1.998811in}}%
\pgfpathlineto{\pgfqpoint{5.122993in}{1.997873in}}%
\pgfpathlineto{\pgfqpoint{5.127115in}{1.998022in}}%
\pgfpathlineto{\pgfqpoint{5.129926in}{1.999369in}}%
\pgfpathlineto{\pgfqpoint{5.129595in}{2.001126in}}%
\pgfpathlineto{\pgfqpoint{5.126645in}{2.002254in}}%
\pgfpathlineto{\pgfqpoint{5.120091in}{2.001498in}}%
\pgfpathlineto{\pgfqpoint{5.119146in}{2.000088in}}%
\pgfpathlineto{\pgfqpoint{5.120760in}{1.998633in}}%
\pgfpathlineto{\pgfqpoint{5.124755in}{1.997930in}}%
\pgfpathlineto{\pgfqpoint{5.128913in}{1.998638in}}%
\pgfpathlineto{\pgfqpoint{5.130508in}{2.000222in}}%
\pgfpathlineto{\pgfqpoint{5.129249in}{2.001694in}}%
\pgfpathlineto{\pgfqpoint{5.122629in}{2.002291in}}%
\pgfpathlineto{\pgfqpoint{5.120023in}{2.001229in}}%
\pgfpathlineto{\pgfqpoint{5.119652in}{1.999632in}}%
\pgfpathlineto{\pgfqpoint{5.122223in}{1.998241in}}%
\pgfpathlineto{\pgfqpoint{5.126659in}{1.998010in}}%
\pgfpathlineto{\pgfqpoint{5.130021in}{1.999074in}}%
\pgfpathlineto{\pgfqpoint{5.130754in}{2.000609in}}%
\pgfpathlineto{\pgfqpoint{5.129082in}{2.001881in}}%
\pgfpathlineto{\pgfqpoint{5.122203in}{2.001989in}}%
\pgfpathlineto{\pgfqpoint{5.119955in}{2.000552in}}%
\pgfpathlineto{\pgfqpoint{5.120782in}{1.998789in}}%
\pgfpathlineto{\pgfqpoint{5.124326in}{1.997814in}}%
\pgfpathlineto{\pgfqpoint{5.130708in}{1.999279in}}%
\pgfpathlineto{\pgfqpoint{5.130828in}{2.000749in}}%
\pgfpathlineto{\pgfqpoint{5.128738in}{2.001920in}}%
\pgfpathlineto{\pgfqpoint{5.125116in}{2.002292in}}%
\pgfpathlineto{\pgfqpoint{5.121416in}{2.001491in}}%
\pgfpathlineto{\pgfqpoint{5.120053in}{1.999703in}}%
\pgfpathlineto{\pgfqpoint{5.122109in}{1.998127in}}%
\pgfpathlineto{\pgfqpoint{5.129107in}{1.998150in}}%
\pgfpathlineto{\pgfqpoint{5.130881in}{1.999414in}}%
\pgfpathlineto{\pgfqpoint{5.130384in}{2.000912in}}%
\pgfpathlineto{\pgfqpoint{5.127550in}{2.002006in}}%
\pgfpathlineto{\pgfqpoint{5.123399in}{2.002036in}}%
\pgfpathlineto{\pgfqpoint{5.120249in}{2.000785in}}%
\pgfpathlineto{\pgfqpoint{5.120165in}{1.999014in}}%
\pgfpathlineto{\pgfqpoint{5.122730in}{1.997786in}}%
\pgfpathlineto{\pgfqpoint{5.129287in}{1.998243in}}%
\pgfpathlineto{\pgfqpoint{5.130614in}{1.999653in}}%
\pgfpathlineto{\pgfqpoint{5.129284in}{2.001242in}}%
\pgfpathlineto{\pgfqpoint{5.125381in}{2.002086in}}%
\pgfpathlineto{\pgfqpoint{5.121253in}{2.001535in}}%
\pgfpathlineto{\pgfqpoint{5.119360in}{2.000068in}}%
\pgfpathlineto{\pgfqpoint{5.120199in}{1.998555in}}%
\pgfpathlineto{\pgfqpoint{5.126562in}{1.997603in}}%
\pgfpathlineto{\pgfqpoint{5.129456in}{1.998534in}}%
\pgfpathlineto{\pgfqpoint{5.130227in}{2.000156in}}%
\pgfpathlineto{\pgfqpoint{5.127822in}{2.001694in}}%
\pgfpathlineto{\pgfqpoint{5.123434in}{2.002052in}}%
\pgfpathlineto{\pgfqpoint{5.120032in}{2.001156in}}%
\pgfpathlineto{\pgfqpoint{5.118988in}{1.999721in}}%
\pgfpathlineto{\pgfqpoint{5.120332in}{1.998388in}}%
\pgfpathlineto{\pgfqpoint{5.123494in}{1.997664in}}%
\pgfpathlineto{\pgfqpoint{5.127314in}{1.997937in}}%
\pgfpathlineto{\pgfqpoint{5.129918in}{1.999272in}}%
\pgfpathlineto{\pgfqpoint{5.129497in}{2.001044in}}%
\pgfpathlineto{\pgfqpoint{5.126200in}{2.002158in}}%
\pgfpathlineto{\pgfqpoint{5.119502in}{2.001034in}}%
\pgfpathlineto{\pgfqpoint{5.118977in}{1.999609in}}%
\pgfpathlineto{\pgfqpoint{5.120786in}{1.998342in}}%
\pgfpathlineto{\pgfqpoint{5.124483in}{1.997807in}}%
\pgfpathlineto{\pgfqpoint{5.128504in}{1.998521in}}%
\pgfpathlineto{\pgfqpoint{5.130116in}{2.000245in}}%
\pgfpathlineto{\pgfqpoint{5.128462in}{2.001802in}}%
\pgfpathlineto{\pgfqpoint{5.121566in}{2.002071in}}%
\pgfpathlineto{\pgfqpoint{5.119387in}{2.000903in}}%
\pgfpathlineto{\pgfqpoint{5.119415in}{1.999389in}}%
\pgfpathlineto{\pgfqpoint{5.121953in}{1.998163in}}%
\pgfpathlineto{\pgfqpoint{5.126195in}{1.997968in}}%
\pgfpathlineto{\pgfqpoint{5.129666in}{1.999137in}}%
\pgfpathlineto{\pgfqpoint{5.130129in}{2.000862in}}%
\pgfpathlineto{\pgfqpoint{5.127983in}{2.002134in}}%
\pgfpathlineto{\pgfqpoint{5.121379in}{2.001939in}}%
\pgfpathlineto{\pgfqpoint{5.119617in}{2.000543in}}%
\pgfpathlineto{\pgfqpoint{5.120571in}{1.998869in}}%
\pgfpathlineto{\pgfqpoint{5.124226in}{1.997899in}}%
\pgfpathlineto{\pgfqpoint{5.128368in}{1.998280in}}%
\pgfpathlineto{\pgfqpoint{5.130622in}{1.999644in}}%
\pgfpathlineto{\pgfqpoint{5.130255in}{2.001182in}}%
\pgfpathlineto{\pgfqpoint{5.127795in}{2.002241in}}%
\pgfpathlineto{\pgfqpoint{5.121113in}{2.001614in}}%
\pgfpathlineto{\pgfqpoint{5.119941in}{1.999940in}}%
\pgfpathlineto{\pgfqpoint{5.122145in}{1.998297in}}%
\pgfpathlineto{\pgfqpoint{5.129699in}{1.998550in}}%
\pgfpathlineto{\pgfqpoint{5.131080in}{1.999910in}}%
\pgfpathlineto{\pgfqpoint{5.130123in}{2.001320in}}%
\pgfpathlineto{\pgfqpoint{5.127166in}{2.002221in}}%
\pgfpathlineto{\pgfqpoint{5.123276in}{2.002141in}}%
\pgfpathlineto{\pgfqpoint{5.120335in}{2.000920in}}%
\pgfpathlineto{\pgfqpoint{5.120350in}{1.999112in}}%
\pgfpathlineto{\pgfqpoint{5.123372in}{1.997865in}}%
\pgfpathlineto{\pgfqpoint{5.130140in}{1.998639in}}%
\pgfpathlineto{\pgfqpoint{5.131022in}{2.000028in}}%
\pgfpathlineto{\pgfqpoint{5.129484in}{2.001413in}}%
\pgfpathlineto{\pgfqpoint{5.125762in}{2.002128in}}%
\pgfpathlineto{\pgfqpoint{5.121578in}{2.001546in}}%
\pgfpathlineto{\pgfqpoint{5.119697in}{1.999941in}}%
\pgfpathlineto{\pgfqpoint{5.120934in}{1.998369in}}%
\pgfpathlineto{\pgfqpoint{5.127702in}{1.997748in}}%
\pgfpathlineto{\pgfqpoint{5.130223in}{1.998781in}}%
\pgfpathlineto{\pgfqpoint{5.130623in}{2.000293in}}%
\pgfpathlineto{\pgfqpoint{5.128312in}{2.001659in}}%
\pgfpathlineto{\pgfqpoint{5.123934in}{2.002008in}}%
\pgfpathlineto{\pgfqpoint{5.120291in}{2.000951in}}%
\pgfpathlineto{\pgfqpoint{5.119495in}{1.999333in}}%
\pgfpathlineto{\pgfqpoint{5.121255in}{1.998031in}}%
\pgfpathlineto{\pgfqpoint{5.127983in}{1.997931in}}%
\pgfpathlineto{\pgfqpoint{5.130181in}{1.999258in}}%
\pgfpathlineto{\pgfqpoint{5.129659in}{2.000964in}}%
\pgfpathlineto{\pgfqpoint{5.126266in}{2.002066in}}%
\pgfpathlineto{\pgfqpoint{5.119441in}{2.000617in}}%
\pgfpathlineto{\pgfqpoint{5.119346in}{1.999105in}}%
\pgfpathlineto{\pgfqpoint{5.121453in}{1.997958in}}%
\pgfpathlineto{\pgfqpoint{5.128465in}{1.998343in}}%
\pgfpathlineto{\pgfqpoint{5.130008in}{2.000044in}}%
\pgfpathlineto{\pgfqpoint{5.128111in}{2.001709in}}%
\pgfpathlineto{\pgfqpoint{5.120786in}{2.001729in}}%
\pgfpathlineto{\pgfqpoint{5.119014in}{2.000441in}}%
\pgfpathlineto{\pgfqpoint{5.119518in}{1.998981in}}%
\pgfpathlineto{\pgfqpoint{5.122177in}{1.997931in}}%
\pgfpathlineto{\pgfqpoint{5.126100in}{1.997835in}}%
\pgfpathlineto{\pgfqpoint{5.129397in}{1.998975in}}%
\pgfpathlineto{\pgfqpoint{5.129739in}{2.000818in}}%
\pgfpathlineto{\pgfqpoint{5.127071in}{2.002139in}}%
\pgfpathlineto{\pgfqpoint{5.120387in}{2.001656in}}%
\pgfpathlineto{\pgfqpoint{5.119136in}{2.000284in}}%
\pgfpathlineto{\pgfqpoint{5.120348in}{1.998779in}}%
\pgfpathlineto{\pgfqpoint{5.123947in}{1.997899in}}%
\pgfpathlineto{\pgfqpoint{5.128168in}{1.998321in}}%
\pgfpathlineto{\pgfqpoint{5.130358in}{1.999816in}}%
\pgfpathlineto{\pgfqpoint{5.129547in}{2.001427in}}%
\pgfpathlineto{\pgfqpoint{5.123093in}{2.002394in}}%
\pgfpathlineto{\pgfqpoint{5.120316in}{2.001492in}}%
\pgfpathlineto{\pgfqpoint{5.119534in}{1.999959in}}%
\pgfpathlineto{\pgfqpoint{5.121677in}{1.998438in}}%
\pgfpathlineto{\pgfqpoint{5.126082in}{1.997968in}}%
\pgfpathlineto{\pgfqpoint{5.129729in}{1.998871in}}%
\pgfpathlineto{\pgfqpoint{5.130823in}{2.000376in}}%
\pgfpathlineto{\pgfqpoint{5.129421in}{2.001729in}}%
\pgfpathlineto{\pgfqpoint{5.122663in}{2.002183in}}%
\pgfpathlineto{\pgfqpoint{5.120090in}{2.000968in}}%
\pgfpathlineto{\pgfqpoint{5.120192in}{1.999238in}}%
\pgfpathlineto{\pgfqpoint{5.123382in}{1.997991in}}%
\pgfpathlineto{\pgfqpoint{5.130474in}{1.999110in}}%
\pgfpathlineto{\pgfqpoint{5.130961in}{2.000559in}}%
\pgfpathlineto{\pgfqpoint{5.129153in}{2.001788in}}%
\pgfpathlineto{\pgfqpoint{5.125641in}{2.002299in}}%
\pgfpathlineto{\pgfqpoint{5.121784in}{2.001681in}}%
\pgfpathlineto{\pgfqpoint{5.119915in}{2.000028in}}%
\pgfpathlineto{\pgfqpoint{5.121418in}{1.998369in}}%
\pgfpathlineto{\pgfqpoint{5.128616in}{1.998038in}}%
\pgfpathlineto{\pgfqpoint{5.130786in}{1.999217in}}%
\pgfpathlineto{\pgfqpoint{5.130750in}{2.000683in}}%
\pgfpathlineto{\pgfqpoint{5.128386in}{2.001859in}}%
\pgfpathlineto{\pgfqpoint{5.124360in}{2.002119in}}%
\pgfpathlineto{\pgfqpoint{5.120708in}{2.001042in}}%
\pgfpathlineto{\pgfqpoint{5.120052in}{1.999236in}}%
\pgfpathlineto{\pgfqpoint{5.122299in}{1.997892in}}%
\pgfpathlineto{\pgfqpoint{5.128963in}{1.998105in}}%
\pgfpathlineto{\pgfqpoint{5.130648in}{1.999442in}}%
\pgfpathlineto{\pgfqpoint{5.129864in}{2.001026in}}%
\pgfpathlineto{\pgfqpoint{5.126513in}{2.002055in}}%
\pgfpathlineto{\pgfqpoint{5.122236in}{2.001803in}}%
\pgfpathlineto{\pgfqpoint{5.119686in}{2.000405in}}%
\pgfpathlineto{\pgfqpoint{5.120031in}{1.998778in}}%
\pgfpathlineto{\pgfqpoint{5.122573in}{1.997703in}}%
\pgfpathlineto{\pgfqpoint{5.129071in}{1.998322in}}%
\pgfpathlineto{\pgfqpoint{5.130258in}{1.999896in}}%
\pgfpathlineto{\pgfqpoint{5.128333in}{2.001542in}}%
\pgfpathlineto{\pgfqpoint{5.124127in}{2.002121in}}%
\pgfpathlineto{\pgfqpoint{5.119042in}{1.999969in}}%
\pgfpathlineto{\pgfqpoint{5.120039in}{1.998548in}}%
\pgfpathlineto{\pgfqpoint{5.122925in}{1.997675in}}%
\pgfpathlineto{\pgfqpoint{5.129520in}{1.998835in}}%
\pgfpathlineto{\pgfqpoint{5.129809in}{2.000617in}}%
\pgfpathlineto{\pgfqpoint{5.126823in}{2.001992in}}%
\pgfpathlineto{\pgfqpoint{5.119739in}{2.001202in}}%
\pgfpathlineto{\pgfqpoint{5.118910in}{1.999806in}}%
\pgfpathlineto{\pgfqpoint{5.120427in}{1.998474in}}%
\pgfpathlineto{\pgfqpoint{5.123909in}{1.997780in}}%
\pgfpathlineto{\pgfqpoint{5.127958in}{1.998241in}}%
\pgfpathlineto{\pgfqpoint{5.130140in}{1.999791in}}%
\pgfpathlineto{\pgfqpoint{5.129049in}{2.001473in}}%
\pgfpathlineto{\pgfqpoint{5.125697in}{2.002349in}}%
\pgfpathlineto{\pgfqpoint{5.119535in}{2.001119in}}%
\pgfpathlineto{\pgfqpoint{5.119167in}{1.999664in}}%
\pgfpathlineto{\pgfqpoint{5.121309in}{1.998356in}}%
\pgfpathlineto{\pgfqpoint{5.125485in}{1.997938in}}%
\pgfpathlineto{\pgfqpoint{5.129345in}{1.998928in}}%
\pgfpathlineto{\pgfqpoint{5.130286in}{2.000625in}}%
\pgfpathlineto{\pgfqpoint{5.128440in}{2.001987in}}%
\pgfpathlineto{\pgfqpoint{5.121718in}{2.002082in}}%
\pgfpathlineto{\pgfqpoint{5.119602in}{2.000829in}}%
\pgfpathlineto{\pgfqpoint{5.119930in}{1.999213in}}%
\pgfpathlineto{\pgfqpoint{5.123041in}{1.998040in}}%
\pgfpathlineto{\pgfqpoint{5.127436in}{1.998131in}}%
\pgfpathlineto{\pgfqpoint{5.130300in}{1.999416in}}%
\pgfpathlineto{\pgfqpoint{5.130395in}{2.000997in}}%
\pgfpathlineto{\pgfqpoint{5.128227in}{2.002145in}}%
\pgfpathlineto{\pgfqpoint{5.121463in}{2.001776in}}%
\pgfpathlineto{\pgfqpoint{5.119851in}{2.000190in}}%
\pgfpathlineto{\pgfqpoint{5.121453in}{1.998490in}}%
\pgfpathlineto{\pgfqpoint{5.125376in}{1.997794in}}%
\pgfpathlineto{\pgfqpoint{5.130940in}{1.999693in}}%
\pgfpathlineto{\pgfqpoint{5.130413in}{2.001154in}}%
\pgfpathlineto{\pgfqpoint{5.127842in}{2.002166in}}%
\pgfpathlineto{\pgfqpoint{5.124132in}{2.002283in}}%
\pgfpathlineto{\pgfqpoint{5.120857in}{2.001254in}}%
\pgfpathlineto{\pgfqpoint{5.120222in}{1.999412in}}%
\pgfpathlineto{\pgfqpoint{5.122912in}{1.997978in}}%
\pgfpathlineto{\pgfqpoint{5.129841in}{1.998477in}}%
\pgfpathlineto{\pgfqpoint{5.131040in}{1.999839in}}%
\pgfpathlineto{\pgfqpoint{5.129890in}{2.001277in}}%
\pgfpathlineto{\pgfqpoint{5.126577in}{2.002151in}}%
\pgfpathlineto{\pgfqpoint{5.122445in}{2.001865in}}%
\pgfpathlineto{\pgfqpoint{5.119926in}{2.000413in}}%
\pgfpathlineto{\pgfqpoint{5.120600in}{1.998688in}}%
\pgfpathlineto{\pgfqpoint{5.123651in}{1.997666in}}%
\pgfpathlineto{\pgfqpoint{5.129976in}{1.998572in}}%
\pgfpathlineto{\pgfqpoint{5.130712in}{2.000038in}}%
\pgfpathlineto{\pgfqpoint{5.128769in}{2.001498in}}%
\pgfpathlineto{\pgfqpoint{5.124539in}{2.002061in}}%
\pgfpathlineto{\pgfqpoint{5.120648in}{2.001209in}}%
\pgfpathlineto{\pgfqpoint{5.119410in}{1.999629in}}%
\pgfpathlineto{\pgfqpoint{5.120873in}{1.998218in}}%
\pgfpathlineto{\pgfqpoint{5.127563in}{1.997765in}}%
\pgfpathlineto{\pgfqpoint{5.130040in}{1.998900in}}%
\pgfpathlineto{\pgfqpoint{5.130126in}{2.000544in}}%
\pgfpathlineto{\pgfqpoint{5.127176in}{2.001870in}}%
\pgfpathlineto{\pgfqpoint{5.122714in}{2.001923in}}%
\pgfpathlineto{\pgfqpoint{5.119677in}{2.000791in}}%
\pgfpathlineto{\pgfqpoint{5.119214in}{1.999291in}}%
\pgfpathlineto{\pgfqpoint{5.121068in}{1.998076in}}%
\pgfpathlineto{\pgfqpoint{5.128117in}{1.998150in}}%
\pgfpathlineto{\pgfqpoint{5.130121in}{1.999681in}}%
\pgfpathlineto{\pgfqpoint{5.128912in}{2.001400in}}%
\pgfpathlineto{\pgfqpoint{5.125227in}{2.002235in}}%
\pgfpathlineto{\pgfqpoint{5.119165in}{2.000631in}}%
\pgfpathlineto{\pgfqpoint{5.119238in}{1.999178in}}%
\pgfpathlineto{\pgfqpoint{5.121521in}{1.998053in}}%
\pgfpathlineto{\pgfqpoint{5.125325in}{1.997781in}}%
\pgfpathlineto{\pgfqpoint{5.128977in}{1.998749in}}%
\pgfpathlineto{\pgfqpoint{5.129889in}{2.000584in}}%
\pgfpathlineto{\pgfqpoint{5.127591in}{2.002022in}}%
\pgfpathlineto{\pgfqpoint{5.120739in}{2.001799in}}%
\pgfpathlineto{\pgfqpoint{5.119114in}{2.000487in}}%
\pgfpathlineto{\pgfqpoint{5.119814in}{1.998983in}}%
\pgfpathlineto{\pgfqpoint{5.122868in}{1.997954in}}%
\pgfpathlineto{\pgfqpoint{5.127092in}{1.998070in}}%
\pgfpathlineto{\pgfqpoint{5.129973in}{1.999447in}}%
\pgfpathlineto{\pgfqpoint{5.129718in}{2.001181in}}%
\pgfpathlineto{\pgfqpoint{5.127053in}{2.002300in}}%
\pgfpathlineto{\pgfqpoint{5.120626in}{2.001667in}}%
\pgfpathlineto{\pgfqpoint{5.119483in}{2.000179in}}%
\pgfpathlineto{\pgfqpoint{5.121145in}{1.998588in}}%
\pgfpathlineto{\pgfqpoint{5.125226in}{1.997892in}}%
\pgfpathlineto{\pgfqpoint{5.129135in}{1.998577in}}%
\pgfpathlineto{\pgfqpoint{5.130732in}{2.000066in}}%
\pgfpathlineto{\pgfqpoint{5.129700in}{2.001539in}}%
\pgfpathlineto{\pgfqpoint{5.123251in}{2.002348in}}%
\pgfpathlineto{\pgfqpoint{5.120453in}{2.001322in}}%
\pgfpathlineto{\pgfqpoint{5.119974in}{1.999620in}}%
\pgfpathlineto{\pgfqpoint{5.122776in}{1.998154in}}%
\pgfpathlineto{\pgfqpoint{5.130226in}{1.998922in}}%
\pgfpathlineto{\pgfqpoint{5.131033in}{2.000354in}}%
\pgfpathlineto{\pgfqpoint{5.129503in}{2.001661in}}%
\pgfpathlineto{\pgfqpoint{5.126191in}{2.002328in}}%
\pgfpathlineto{\pgfqpoint{5.122347in}{2.001952in}}%
\pgfpathlineto{\pgfqpoint{5.119975in}{2.000519in}}%
\pgfpathlineto{\pgfqpoint{5.120775in}{1.998762in}}%
\pgfpathlineto{\pgfqpoint{5.124239in}{1.997771in}}%
\pgfpathlineto{\pgfqpoint{5.130621in}{1.999029in}}%
\pgfpathlineto{\pgfqpoint{5.130941in}{2.000459in}}%
\pgfpathlineto{\pgfqpoint{5.128893in}{2.001706in}}%
\pgfpathlineto{\pgfqpoint{5.124966in}{2.002147in}}%
\pgfpathlineto{\pgfqpoint{5.121042in}{2.001272in}}%
\pgfpathlineto{\pgfqpoint{5.119849in}{1.999531in}}%
\pgfpathlineto{\pgfqpoint{5.121732in}{1.998077in}}%
\pgfpathlineto{\pgfqpoint{5.128600in}{1.997976in}}%
\pgfpathlineto{\pgfqpoint{5.130645in}{1.999190in}}%
\pgfpathlineto{\pgfqpoint{5.130399in}{2.000720in}}%
\pgfpathlineto{\pgfqpoint{5.127576in}{2.001895in}}%
\pgfpathlineto{\pgfqpoint{5.123190in}{2.001932in}}%
\pgfpathlineto{\pgfqpoint{5.120021in}{2.000642in}}%
\pgfpathlineto{\pgfqpoint{5.119885in}{1.998969in}}%
\pgfpathlineto{\pgfqpoint{5.122157in}{1.997794in}}%
\pgfpathlineto{\pgfqpoint{5.128776in}{1.998169in}}%
\pgfpathlineto{\pgfqpoint{5.130367in}{1.999647in}}%
\pgfpathlineto{\pgfqpoint{5.129082in}{2.001311in}}%
\pgfpathlineto{\pgfqpoint{5.125242in}{2.002139in}}%
\pgfpathlineto{\pgfqpoint{5.121242in}{2.001622in}}%
\pgfpathlineto{\pgfqpoint{5.119247in}{2.000221in}}%
\pgfpathlineto{\pgfqpoint{5.119808in}{1.998721in}}%
\pgfpathlineto{\pgfqpoint{5.122379in}{1.997731in}}%
\pgfpathlineto{\pgfqpoint{5.129083in}{1.998576in}}%
\pgfpathlineto{\pgfqpoint{5.129918in}{2.000348in}}%
\pgfpathlineto{\pgfqpoint{5.127340in}{2.001881in}}%
\pgfpathlineto{\pgfqpoint{5.120085in}{2.001391in}}%
\pgfpathlineto{\pgfqpoint{5.118908in}{2.000007in}}%
\pgfpathlineto{\pgfqpoint{5.120049in}{1.998608in}}%
\pgfpathlineto{\pgfqpoint{5.123159in}{1.997765in}}%
\pgfpathlineto{\pgfqpoint{5.127091in}{1.997959in}}%
\pgfpathlineto{\pgfqpoint{5.129822in}{1.999308in}}%
\pgfpathlineto{\pgfqpoint{5.129394in}{2.001121in}}%
\pgfpathlineto{\pgfqpoint{5.126211in}{2.002238in}}%
\pgfpathlineto{\pgfqpoint{5.119756in}{2.001301in}}%
\pgfpathlineto{\pgfqpoint{5.119084in}{1.999875in}}%
\pgfpathlineto{\pgfqpoint{5.120896in}{1.998489in}}%
\pgfpathlineto{\pgfqpoint{5.124851in}{1.997879in}}%
\pgfpathlineto{\pgfqpoint{5.128880in}{1.998617in}}%
\pgfpathlineto{\pgfqpoint{5.130375in}{2.000248in}}%
\pgfpathlineto{\pgfqpoint{5.128892in}{2.001751in}}%
\pgfpathlineto{\pgfqpoint{5.122116in}{2.002216in}}%
\pgfpathlineto{\pgfqpoint{5.119735in}{2.001124in}}%
\pgfpathlineto{\pgfqpoint{5.119579in}{1.999570in}}%
\pgfpathlineto{\pgfqpoint{5.122255in}{1.998237in}}%
\pgfpathlineto{\pgfqpoint{5.126776in}{1.998066in}}%
\pgfpathlineto{\pgfqpoint{5.130067in}{1.999222in}}%
\pgfpathlineto{\pgfqpoint{5.130550in}{2.000796in}}%
\pgfpathlineto{\pgfqpoint{5.128625in}{2.002026in}}%
\pgfpathlineto{\pgfqpoint{5.121795in}{2.001966in}}%
\pgfpathlineto{\pgfqpoint{5.119770in}{2.000558in}}%
\pgfpathlineto{\pgfqpoint{5.120635in}{1.998856in}}%
\pgfpathlineto{\pgfqpoint{5.124268in}{1.997880in}}%
\pgfpathlineto{\pgfqpoint{5.130738in}{1.999498in}}%
\pgfpathlineto{\pgfqpoint{5.130628in}{2.000975in}}%
\pgfpathlineto{\pgfqpoint{5.128361in}{2.002065in}}%
\pgfpathlineto{\pgfqpoint{5.124737in}{2.002325in}}%
\pgfpathlineto{\pgfqpoint{5.121210in}{2.001452in}}%
\pgfpathlineto{\pgfqpoint{5.120038in}{1.999670in}}%
\pgfpathlineto{\pgfqpoint{5.122248in}{1.998125in}}%
\pgfpathlineto{\pgfqpoint{5.129410in}{1.998322in}}%
\pgfpathlineto{\pgfqpoint{5.131018in}{1.999641in}}%
\pgfpathlineto{\pgfqpoint{5.130339in}{2.001095in}}%
\pgfpathlineto{\pgfqpoint{5.127498in}{2.002090in}}%
\pgfpathlineto{\pgfqpoint{5.123462in}{2.002064in}}%
\pgfpathlineto{\pgfqpoint{5.120349in}{2.000772in}}%
\pgfpathlineto{\pgfqpoint{5.120422in}{1.998941in}}%
\pgfpathlineto{\pgfqpoint{5.123218in}{1.997751in}}%
\pgfpathlineto{\pgfqpoint{5.129702in}{1.998411in}}%
\pgfpathlineto{\pgfqpoint{5.130782in}{1.999840in}}%
\pgfpathlineto{\pgfqpoint{5.129306in}{2.001351in}}%
\pgfpathlineto{\pgfqpoint{5.125498in}{2.002123in}}%
\pgfpathlineto{\pgfqpoint{5.121402in}{2.001550in}}%
\pgfpathlineto{\pgfqpoint{5.119542in}{2.000006in}}%
\pgfpathlineto{\pgfqpoint{5.120586in}{1.998446in}}%
\pgfpathlineto{\pgfqpoint{5.127069in}{1.997639in}}%
\pgfpathlineto{\pgfqpoint{5.129741in}{1.998624in}}%
\pgfpathlineto{\pgfqpoint{5.130275in}{2.000233in}}%
\pgfpathlineto{\pgfqpoint{5.127733in}{2.001719in}}%
\pgfpathlineto{\pgfqpoint{5.123284in}{2.002005in}}%
\pgfpathlineto{\pgfqpoint{5.119956in}{2.001017in}}%
\pgfpathlineto{\pgfqpoint{5.119134in}{1.999524in}}%
\pgfpathlineto{\pgfqpoint{5.120714in}{1.998212in}}%
\pgfpathlineto{\pgfqpoint{5.127587in}{1.997904in}}%
\pgfpathlineto{\pgfqpoint{5.129992in}{1.999212in}}%
\pgfpathlineto{\pgfqpoint{5.129523in}{2.000968in}}%
\pgfpathlineto{\pgfqpoint{5.126057in}{2.002095in}}%
\pgfpathlineto{\pgfqpoint{5.119343in}{2.000809in}}%
\pgfpathlineto{\pgfqpoint{5.119077in}{1.999372in}}%
\pgfpathlineto{\pgfqpoint{5.121093in}{1.998175in}}%
\pgfpathlineto{\pgfqpoint{5.124793in}{1.997744in}}%
\pgfpathlineto{\pgfqpoint{5.128621in}{1.998500in}}%
\pgfpathlineto{\pgfqpoint{5.130114in}{2.000211in}}%
\pgfpathlineto{\pgfqpoint{5.128310in}{2.001778in}}%
\pgfpathlineto{\pgfqpoint{5.121165in}{2.001918in}}%
\pgfpathlineto{\pgfqpoint{5.119156in}{2.000705in}}%
\pgfpathlineto{\pgfqpoint{5.119393in}{1.999237in}}%
\pgfpathlineto{\pgfqpoint{5.122008in}{1.998103in}}%
\pgfpathlineto{\pgfqpoint{5.126227in}{1.997976in}}%
\pgfpathlineto{\pgfqpoint{5.129649in}{1.999210in}}%
\pgfpathlineto{\pgfqpoint{5.129905in}{2.000976in}}%
\pgfpathlineto{\pgfqpoint{5.127520in}{2.002203in}}%
\pgfpathlineto{\pgfqpoint{5.120912in}{2.001818in}}%
\pgfpathlineto{\pgfqpoint{5.119374in}{2.000420in}}%
\pgfpathlineto{\pgfqpoint{5.120421in}{1.998833in}}%
\pgfpathlineto{\pgfqpoint{5.124009in}{1.997909in}}%
\pgfpathlineto{\pgfqpoint{5.128232in}{1.998320in}}%
\pgfpathlineto{\pgfqpoint{5.130478in}{1.999778in}}%
\pgfpathlineto{\pgfqpoint{5.129902in}{2.001351in}}%
\pgfpathlineto{\pgfqpoint{5.127274in}{2.002341in}}%
\pgfpathlineto{\pgfqpoint{5.120799in}{2.001527in}}%
\pgfpathlineto{\pgfqpoint{5.119867in}{1.999858in}}%
\pgfpathlineto{\pgfqpoint{5.122183in}{1.998275in}}%
\pgfpathlineto{\pgfqpoint{5.126387in}{1.997870in}}%
\pgfpathlineto{\pgfqpoint{5.131003in}{2.000125in}}%
\pgfpathlineto{\pgfqpoint{5.129840in}{2.001520in}}%
\pgfpathlineto{\pgfqpoint{5.126843in}{2.002335in}}%
\pgfpathlineto{\pgfqpoint{5.123120in}{2.002184in}}%
\pgfpathlineto{\pgfqpoint{5.120346in}{2.000954in}}%
\pgfpathlineto{\pgfqpoint{5.120465in}{1.999121in}}%
\pgfpathlineto{\pgfqpoint{5.123706in}{1.997885in}}%
\pgfpathlineto{\pgfqpoint{5.130411in}{1.998854in}}%
\pgfpathlineto{\pgfqpoint{5.131028in}{2.000269in}}%
\pgfpathlineto{\pgfqpoint{5.129286in}{2.001593in}}%
\pgfpathlineto{\pgfqpoint{5.125610in}{2.002211in}}%
\pgfpathlineto{\pgfqpoint{5.121617in}{2.001610in}}%
\pgfpathlineto{\pgfqpoint{5.119791in}{1.999993in}}%
\pgfpathlineto{\pgfqpoint{5.121194in}{1.998373in}}%
\pgfpathlineto{\pgfqpoint{5.128199in}{1.997870in}}%
\pgfpathlineto{\pgfqpoint{5.130515in}{1.998954in}}%
\pgfpathlineto{\pgfqpoint{5.130657in}{2.000440in}}%
\pgfpathlineto{\pgfqpoint{5.128192in}{2.001728in}}%
\pgfpathlineto{\pgfqpoint{5.123802in}{2.001995in}}%
\pgfpathlineto{\pgfqpoint{5.120256in}{2.000867in}}%
\pgfpathlineto{\pgfqpoint{5.119683in}{1.999209in}}%
\pgfpathlineto{\pgfqpoint{5.121702in}{1.997934in}}%
\pgfpathlineto{\pgfqpoint{5.128456in}{1.997995in}}%
\pgfpathlineto{\pgfqpoint{5.130423in}{1.999310in}}%
\pgfpathlineto{\pgfqpoint{5.129805in}{2.000941in}}%
\pgfpathlineto{\pgfqpoint{5.126386in}{2.002019in}}%
\pgfpathlineto{\pgfqpoint{5.122019in}{2.001760in}}%
\pgfpathlineto{\pgfqpoint{5.119471in}{2.000424in}}%
\pgfpathlineto{\pgfqpoint{5.119618in}{1.998894in}}%
\pgfpathlineto{\pgfqpoint{5.121927in}{1.997820in}}%
\pgfpathlineto{\pgfqpoint{5.128785in}{1.998391in}}%
\pgfpathlineto{\pgfqpoint{5.130118in}{2.000065in}}%
\pgfpathlineto{\pgfqpoint{5.128152in}{2.001687in}}%
\pgfpathlineto{\pgfqpoint{5.124185in}{2.002228in}}%
\pgfpathlineto{\pgfqpoint{5.118987in}{2.000212in}}%
\pgfpathlineto{\pgfqpoint{5.119687in}{1.998775in}}%
\pgfpathlineto{\pgfqpoint{5.122407in}{1.997821in}}%
\pgfpathlineto{\pgfqpoint{5.126227in}{1.997811in}}%
\pgfpathlineto{\pgfqpoint{5.129396in}{1.998993in}}%
\pgfpathlineto{\pgfqpoint{5.129585in}{2.000870in}}%
\pgfpathlineto{\pgfqpoint{5.126698in}{2.002155in}}%
\pgfpathlineto{\pgfqpoint{5.120035in}{2.001468in}}%
\pgfpathlineto{\pgfqpoint{5.119010in}{2.000067in}}%
\pgfpathlineto{\pgfqpoint{5.120372in}{1.998630in}}%
\pgfpathlineto{\pgfqpoint{5.123876in}{1.997835in}}%
\pgfpathlineto{\pgfqpoint{5.127985in}{1.998265in}}%
\pgfpathlineto{\pgfqpoint{5.130186in}{1.999809in}}%
\pgfpathlineto{\pgfqpoint{5.129205in}{2.001482in}}%
\pgfpathlineto{\pgfqpoint{5.122554in}{2.002314in}}%
\pgfpathlineto{\pgfqpoint{5.119978in}{2.001342in}}%
\pgfpathlineto{\pgfqpoint{5.119458in}{1.999815in}}%
\pgfpathlineto{\pgfqpoint{5.121744in}{1.998368in}}%
\pgfpathlineto{\pgfqpoint{5.126120in}{1.997966in}}%
\pgfpathlineto{\pgfqpoint{5.129712in}{1.998933in}}%
\pgfpathlineto{\pgfqpoint{5.130638in}{2.000503in}}%
\pgfpathlineto{\pgfqpoint{5.128998in}{2.001859in}}%
\pgfpathlineto{\pgfqpoint{5.122260in}{2.002172in}}%
\pgfpathlineto{\pgfqpoint{5.119920in}{2.000958in}}%
\pgfpathlineto{\pgfqpoint{5.120145in}{1.999268in}}%
\pgfpathlineto{\pgfqpoint{5.123451in}{1.998031in}}%
\pgfpathlineto{\pgfqpoint{5.130560in}{1.999314in}}%
\pgfpathlineto{\pgfqpoint{5.130786in}{2.000786in}}%
\pgfpathlineto{\pgfqpoint{5.128757in}{2.001957in}}%
\pgfpathlineto{\pgfqpoint{5.121571in}{2.001704in}}%
\pgfpathlineto{\pgfqpoint{5.119849in}{2.000094in}}%
\pgfpathlineto{\pgfqpoint{5.121413in}{1.998433in}}%
\pgfpathlineto{\pgfqpoint{5.125198in}{1.997730in}}%
\pgfpathlineto{\pgfqpoint{5.130931in}{1.999445in}}%
\pgfpathlineto{\pgfqpoint{5.130665in}{2.000885in}}%
\pgfpathlineto{\pgfqpoint{5.128168in}{2.001968in}}%
\pgfpathlineto{\pgfqpoint{5.124174in}{2.002132in}}%
\pgfpathlineto{\pgfqpoint{5.120653in}{2.001000in}}%
\pgfpathlineto{\pgfqpoint{5.120180in}{1.999173in}}%
\pgfpathlineto{\pgfqpoint{5.122654in}{1.997863in}}%
\pgfpathlineto{\pgfqpoint{5.129395in}{1.998262in}}%
\pgfpathlineto{\pgfqpoint{5.130874in}{1.999616in}}%
\pgfpathlineto{\pgfqpoint{5.129957in}{2.001120in}}%
\pgfpathlineto{\pgfqpoint{5.126668in}{2.002073in}}%
\pgfpathlineto{\pgfqpoint{5.122395in}{2.001798in}}%
\pgfpathlineto{\pgfqpoint{5.119832in}{2.000318in}}%
\pgfpathlineto{\pgfqpoint{5.120388in}{1.998646in}}%
\pgfpathlineto{\pgfqpoint{5.123131in}{1.997633in}}%
\pgfpathlineto{\pgfqpoint{5.129468in}{1.998449in}}%
\pgfpathlineto{\pgfqpoint{5.130404in}{2.000017in}}%
\pgfpathlineto{\pgfqpoint{5.128396in}{2.001579in}}%
\pgfpathlineto{\pgfqpoint{5.124209in}{2.002112in}}%
\pgfpathlineto{\pgfqpoint{5.120531in}{2.001309in}}%
\pgfpathlineto{\pgfqpoint{5.119206in}{1.999803in}}%
\pgfpathlineto{\pgfqpoint{5.120415in}{1.998373in}}%
\pgfpathlineto{\pgfqpoint{5.126955in}{1.997723in}}%
\pgfpathlineto{\pgfqpoint{5.129644in}{1.998863in}}%
\pgfpathlineto{\pgfqpoint{5.129762in}{2.000642in}}%
\pgfpathlineto{\pgfqpoint{5.126597in}{2.001988in}}%
\pgfpathlineto{\pgfqpoint{5.119564in}{2.001004in}}%
\pgfpathlineto{\pgfqpoint{5.118985in}{1.999569in}}%
\pgfpathlineto{\pgfqpoint{5.120708in}{1.998284in}}%
\pgfpathlineto{\pgfqpoint{5.124169in}{1.997683in}}%
\pgfpathlineto{\pgfqpoint{5.128002in}{1.998175in}}%
\pgfpathlineto{\pgfqpoint{5.130093in}{1.999705in}}%
\pgfpathlineto{\pgfqpoint{5.128906in}{2.001429in}}%
\pgfpathlineto{\pgfqpoint{5.125321in}{2.002292in}}%
\pgfpathlineto{\pgfqpoint{5.119289in}{2.000905in}}%
\pgfpathlineto{\pgfqpoint{5.119191in}{1.999461in}}%
\pgfpathlineto{\pgfqpoint{5.121515in}{1.998234in}}%
\pgfpathlineto{\pgfqpoint{5.125664in}{1.997912in}}%
\pgfpathlineto{\pgfqpoint{5.129380in}{1.998944in}}%
\pgfpathlineto{\pgfqpoint{5.130158in}{2.000671in}}%
\pgfpathlineto{\pgfqpoint{5.128072in}{2.002028in}}%
\pgfpathlineto{\pgfqpoint{5.121238in}{2.001971in}}%
\pgfpathlineto{\pgfqpoint{5.119341in}{2.000710in}}%
\pgfpathlineto{\pgfqpoint{5.119830in}{1.999165in}}%
\pgfpathlineto{\pgfqpoint{5.122968in}{1.998052in}}%
\pgfpathlineto{\pgfqpoint{5.127440in}{1.998190in}}%
\pgfpathlineto{\pgfqpoint{5.130253in}{1.999566in}}%
\pgfpathlineto{\pgfqpoint{5.130100in}{2.001174in}}%
\pgfpathlineto{\pgfqpoint{5.127708in}{2.002258in}}%
\pgfpathlineto{\pgfqpoint{5.121055in}{2.001706in}}%
\pgfpathlineto{\pgfqpoint{5.119666in}{2.000151in}}%
\pgfpathlineto{\pgfqpoint{5.121291in}{1.998521in}}%
\pgfpathlineto{\pgfqpoint{5.125272in}{1.997842in}}%
\pgfpathlineto{\pgfqpoint{5.129112in}{1.998478in}}%
\pgfpathlineto{\pgfqpoint{5.130863in}{1.999901in}}%
\pgfpathlineto{\pgfqpoint{5.130125in}{2.001359in}}%
\pgfpathlineto{\pgfqpoint{5.127440in}{2.002282in}}%
\pgfpathlineto{\pgfqpoint{5.120712in}{2.001208in}}%
\pgfpathlineto{\pgfqpoint{5.120263in}{1.999365in}}%
\pgfpathlineto{\pgfqpoint{5.123121in}{1.997972in}}%
\pgfpathlineto{\pgfqpoint{5.130082in}{1.998667in}}%
\pgfpathlineto{\pgfqpoint{5.131078in}{2.000074in}}%
\pgfpathlineto{\pgfqpoint{5.129760in}{2.001460in}}%
\pgfpathlineto{\pgfqpoint{5.126496in}{2.002240in}}%
\pgfpathlineto{\pgfqpoint{5.122532in}{2.001921in}}%
\pgfpathlineto{\pgfqpoint{5.120050in}{2.000449in}}%
\pgfpathlineto{\pgfqpoint{5.120866in}{1.998665in}}%
\pgfpathlineto{\pgfqpoint{5.124138in}{1.997676in}}%
\pgfpathlineto{\pgfqpoint{5.130308in}{1.998766in}}%
\pgfpathlineto{\pgfqpoint{5.130778in}{2.000236in}}%
\pgfpathlineto{\pgfqpoint{5.128684in}{2.001615in}}%
\pgfpathlineto{\pgfqpoint{5.124544in}{2.002101in}}%
\pgfpathlineto{\pgfqpoint{5.120730in}{2.001222in}}%
\pgfpathlineto{\pgfqpoint{5.119590in}{1.999579in}}%
\pgfpathlineto{\pgfqpoint{5.121282in}{1.998144in}}%
\pgfpathlineto{\pgfqpoint{5.128045in}{1.997830in}}%
\pgfpathlineto{\pgfqpoint{5.130284in}{1.998988in}}%
\pgfpathlineto{\pgfqpoint{5.130168in}{2.000591in}}%
\pgfpathlineto{\pgfqpoint{5.127115in}{2.001868in}}%
\pgfpathlineto{\pgfqpoint{5.122577in}{2.001852in}}%
\pgfpathlineto{\pgfqpoint{5.119645in}{2.000631in}}%
\pgfpathlineto{\pgfqpoint{5.119437in}{1.999098in}}%
\pgfpathlineto{\pgfqpoint{5.121523in}{1.997928in}}%
\pgfpathlineto{\pgfqpoint{5.128434in}{1.998147in}}%
\pgfpathlineto{\pgfqpoint{5.130249in}{1.999633in}}%
\pgfpathlineto{\pgfqpoint{5.129021in}{2.001320in}}%
\pgfpathlineto{\pgfqpoint{5.125186in}{2.002159in}}%
\pgfpathlineto{\pgfqpoint{5.119114in}{2.000401in}}%
\pgfpathlineto{\pgfqpoint{5.119430in}{1.998956in}}%
\pgfpathlineto{\pgfqpoint{5.121879in}{1.997918in}}%
\pgfpathlineto{\pgfqpoint{5.125659in}{1.997753in}}%
\pgfpathlineto{\pgfqpoint{5.129123in}{1.998770in}}%
\pgfpathlineto{\pgfqpoint{5.129884in}{2.000591in}}%
\pgfpathlineto{\pgfqpoint{5.127421in}{2.002010in}}%
\pgfpathlineto{\pgfqpoint{5.120409in}{2.001621in}}%
\pgfpathlineto{\pgfqpoint{5.118964in}{2.000274in}}%
\pgfpathlineto{\pgfqpoint{5.119831in}{1.998831in}}%
\pgfpathlineto{\pgfqpoint{5.122878in}{1.997897in}}%
\pgfpathlineto{\pgfqpoint{5.127028in}{1.998063in}}%
\pgfpathlineto{\pgfqpoint{5.129880in}{1.999499in}}%
\pgfpathlineto{\pgfqpoint{5.129424in}{2.001276in}}%
\pgfpathlineto{\pgfqpoint{5.126544in}{2.002339in}}%
\pgfpathlineto{\pgfqpoint{5.120220in}{2.001507in}}%
\pgfpathlineto{\pgfqpoint{5.119315in}{2.000018in}}%
\pgfpathlineto{\pgfqpoint{5.121064in}{1.998518in}}%
\pgfpathlineto{\pgfqpoint{5.125046in}{1.997873in}}%
\pgfpathlineto{\pgfqpoint{5.128984in}{1.998591in}}%
\pgfpathlineto{\pgfqpoint{5.130530in}{2.000169in}}%
\pgfpathlineto{\pgfqpoint{5.129284in}{2.001670in}}%
\pgfpathlineto{\pgfqpoint{5.122779in}{2.002312in}}%
\pgfpathlineto{\pgfqpoint{5.120209in}{2.001231in}}%
\pgfpathlineto{\pgfqpoint{5.119963in}{1.999542in}}%
\pgfpathlineto{\pgfqpoint{5.122898in}{1.998132in}}%
\pgfpathlineto{\pgfqpoint{5.127278in}{1.998026in}}%
\pgfpathlineto{\pgfqpoint{5.130297in}{1.999078in}}%
\pgfpathlineto{\pgfqpoint{5.130876in}{2.000564in}}%
\pgfpathlineto{\pgfqpoint{5.129130in}{2.001841in}}%
\pgfpathlineto{\pgfqpoint{5.122161in}{2.001994in}}%
\pgfpathlineto{\pgfqpoint{5.119957in}{2.000585in}}%
\pgfpathlineto{\pgfqpoint{5.120829in}{1.998817in}}%
\pgfpathlineto{\pgfqpoint{5.124508in}{1.997830in}}%
\pgfpathlineto{\pgfqpoint{5.130797in}{1.999262in}}%
\pgfpathlineto{\pgfqpoint{5.130840in}{2.000695in}}%
\pgfpathlineto{\pgfqpoint{5.128594in}{2.001865in}}%
\pgfpathlineto{\pgfqpoint{5.124705in}{2.002204in}}%
\pgfpathlineto{\pgfqpoint{5.120981in}{2.001303in}}%
\pgfpathlineto{\pgfqpoint{5.119893in}{1.999563in}}%
\pgfpathlineto{\pgfqpoint{5.121965in}{1.998093in}}%
\pgfpathlineto{\pgfqpoint{5.129033in}{1.998130in}}%
\pgfpathlineto{\pgfqpoint{5.130873in}{1.999376in}}%
\pgfpathlineto{\pgfqpoint{5.130412in}{2.000851in}}%
\pgfpathlineto{\pgfqpoint{5.127506in}{2.001937in}}%
\pgfpathlineto{\pgfqpoint{5.123121in}{2.001904in}}%
\pgfpathlineto{\pgfqpoint{5.120037in}{2.000538in}}%
\pgfpathlineto{\pgfqpoint{5.120142in}{1.998839in}}%
\pgfpathlineto{\pgfqpoint{5.122652in}{1.997721in}}%
\pgfpathlineto{\pgfqpoint{5.129224in}{1.998276in}}%
\pgfpathlineto{\pgfqpoint{5.130591in}{1.999734in}}%
\pgfpathlineto{\pgfqpoint{5.129254in}{2.001310in}}%
\pgfpathlineto{\pgfqpoint{5.125440in}{2.002111in}}%
\pgfpathlineto{\pgfqpoint{5.121321in}{2.001546in}}%
\pgfpathlineto{\pgfqpoint{5.119383in}{2.000050in}}%
\pgfpathlineto{\pgfqpoint{5.120168in}{1.998539in}}%
\pgfpathlineto{\pgfqpoint{5.122889in}{1.997630in}}%
\pgfpathlineto{\pgfqpoint{5.129354in}{1.998653in}}%
\pgfpathlineto{\pgfqpoint{5.129976in}{2.000404in}}%
\pgfpathlineto{\pgfqpoint{5.127309in}{2.001883in}}%
\pgfpathlineto{\pgfqpoint{5.119980in}{2.001228in}}%
\pgfpathlineto{\pgfqpoint{5.118975in}{1.999782in}}%
\pgfpathlineto{\pgfqpoint{5.120299in}{1.998415in}}%
\pgfpathlineto{\pgfqpoint{5.123411in}{1.997664in}}%
\pgfpathlineto{\pgfqpoint{5.127174in}{1.997920in}}%
\pgfpathlineto{\pgfqpoint{5.129778in}{1.999282in}}%
\pgfpathlineto{\pgfqpoint{5.129228in}{2.001131in}}%
\pgfpathlineto{\pgfqpoint{5.125830in}{2.002218in}}%
\pgfpathlineto{\pgfqpoint{5.119485in}{2.001088in}}%
\pgfpathlineto{\pgfqpoint{5.119065in}{1.999648in}}%
\pgfpathlineto{\pgfqpoint{5.121030in}{1.998335in}}%
\pgfpathlineto{\pgfqpoint{5.124878in}{1.997808in}}%
\pgfpathlineto{\pgfqpoint{5.128767in}{1.998549in}}%
\pgfpathlineto{\pgfqpoint{5.130234in}{2.000217in}}%
\pgfpathlineto{\pgfqpoint{5.128563in}{2.001767in}}%
\pgfpathlineto{\pgfqpoint{5.121626in}{2.002100in}}%
\pgfpathlineto{\pgfqpoint{5.119467in}{2.000964in}}%
\pgfpathlineto{\pgfqpoint{5.119560in}{1.999440in}}%
\pgfpathlineto{\pgfqpoint{5.122360in}{1.998184in}}%
\pgfpathlineto{\pgfqpoint{5.126869in}{1.998086in}}%
\pgfpathlineto{\pgfqpoint{5.130063in}{1.999310in}}%
\pgfpathlineto{\pgfqpoint{5.130324in}{2.000925in}}%
\pgfpathlineto{\pgfqpoint{5.128155in}{2.002128in}}%
\pgfpathlineto{\pgfqpoint{5.121383in}{2.001923in}}%
\pgfpathlineto{\pgfqpoint{5.119582in}{2.000543in}}%
\pgfpathlineto{\pgfqpoint{5.120517in}{1.998900in}}%
\pgfpathlineto{\pgfqpoint{5.124229in}{1.997934in}}%
\pgfpathlineto{\pgfqpoint{5.128483in}{1.998347in}}%
\pgfpathlineto{\pgfqpoint{5.130721in}{1.999708in}}%
\pgfpathlineto{\pgfqpoint{5.130355in}{2.001188in}}%
\pgfpathlineto{\pgfqpoint{5.127896in}{2.002200in}}%
\pgfpathlineto{\pgfqpoint{5.120956in}{2.001428in}}%
\pgfpathlineto{\pgfqpoint{5.119948in}{1.999689in}}%
\pgfpathlineto{\pgfqpoint{5.122235in}{1.998166in}}%
\pgfpathlineto{\pgfqpoint{5.129616in}{1.998501in}}%
\pgfpathlineto{\pgfqpoint{5.131066in}{1.999875in}}%
\pgfpathlineto{\pgfqpoint{5.130190in}{2.001287in}}%
\pgfpathlineto{\pgfqpoint{5.127294in}{2.002181in}}%
\pgfpathlineto{\pgfqpoint{5.123347in}{2.002071in}}%
\pgfpathlineto{\pgfqpoint{5.120352in}{2.000730in}}%
\pgfpathlineto{\pgfqpoint{5.120616in}{1.998874in}}%
\pgfpathlineto{\pgfqpoint{5.123627in}{1.997737in}}%
\pgfpathlineto{\pgfqpoint{5.130065in}{1.998600in}}%
\pgfpathlineto{\pgfqpoint{5.130921in}{2.000040in}}%
\pgfpathlineto{\pgfqpoint{5.129338in}{2.001467in}}%
\pgfpathlineto{\pgfqpoint{5.125648in}{2.002166in}}%
\pgfpathlineto{\pgfqpoint{5.121593in}{2.001581in}}%
\pgfpathlineto{\pgfqpoint{5.119735in}{1.999962in}}%
\pgfpathlineto{\pgfqpoint{5.120988in}{1.998358in}}%
\pgfpathlineto{\pgfqpoint{5.127605in}{1.997718in}}%
\pgfpathlineto{\pgfqpoint{5.130058in}{1.998772in}}%
\pgfpathlineto{\pgfqpoint{5.130331in}{2.000368in}}%
\pgfpathlineto{\pgfqpoint{5.127684in}{2.001773in}}%
\pgfpathlineto{\pgfqpoint{5.123267in}{2.001996in}}%
\pgfpathlineto{\pgfqpoint{5.119989in}{2.000935in}}%
\pgfpathlineto{\pgfqpoint{5.119348in}{1.999373in}}%
\pgfpathlineto{\pgfqpoint{5.121153in}{1.998069in}}%
\pgfpathlineto{\pgfqpoint{5.127924in}{1.997907in}}%
\pgfpathlineto{\pgfqpoint{5.130102in}{1.999213in}}%
\pgfpathlineto{\pgfqpoint{5.129508in}{2.000947in}}%
\pgfpathlineto{\pgfqpoint{5.125875in}{2.002057in}}%
\pgfpathlineto{\pgfqpoint{5.119244in}{2.000592in}}%
\pgfpathlineto{\pgfqpoint{5.119251in}{1.999142in}}%
\pgfpathlineto{\pgfqpoint{5.121472in}{1.998009in}}%
\pgfpathlineto{\pgfqpoint{5.128752in}{1.998458in}}%
\pgfpathlineto{\pgfqpoint{5.130142in}{2.000133in}}%
\pgfpathlineto{\pgfqpoint{5.128234in}{2.001720in}}%
\pgfpathlineto{\pgfqpoint{5.120825in}{2.001742in}}%
\pgfpathlineto{\pgfqpoint{5.119001in}{2.000481in}}%
\pgfpathlineto{\pgfqpoint{5.119478in}{1.999050in}}%
\pgfpathlineto{\pgfqpoint{5.122222in}{1.998011in}}%
\pgfpathlineto{\pgfqpoint{5.126402in}{1.997976in}}%
\pgfpathlineto{\pgfqpoint{5.129685in}{1.999262in}}%
\pgfpathlineto{\pgfqpoint{5.129740in}{2.001045in}}%
\pgfpathlineto{\pgfqpoint{5.127122in}{2.002232in}}%
\pgfpathlineto{\pgfqpoint{5.120481in}{2.001670in}}%
\pgfpathlineto{\pgfqpoint{5.119156in}{2.000275in}}%
\pgfpathlineto{\pgfqpoint{5.120308in}{1.998776in}}%
\pgfpathlineto{\pgfqpoint{5.123845in}{1.997911in}}%
\pgfpathlineto{\pgfqpoint{5.128121in}{1.998355in}}%
\pgfpathlineto{\pgfqpoint{5.130332in}{1.999904in}}%
\pgfpathlineto{\pgfqpoint{5.129521in}{2.001500in}}%
\pgfpathlineto{\pgfqpoint{5.129521in}{2.001500in}}%
\pgfusepath{stroke}%
\end{pgfscope}%
\begin{pgfscope}%
\pgfpathrectangle{\pgfqpoint{1.250000in}{0.400000in}}{\pgfqpoint{7.750000in}{3.200000in}} %
\pgfusepath{clip}%
\pgfsetbuttcap%
\pgfsetroundjoin%
\pgfsetlinewidth{0.501875pt}%
\definecolor{currentstroke}{rgb}{0.000000,0.000000,0.000000}%
\pgfsetstrokecolor{currentstroke}%
\pgfsetdash{{1.000000pt}{3.000000pt}}{0.000000pt}%
\pgfpathmoveto{\pgfqpoint{1.250000in}{0.400000in}}%
\pgfpathlineto{\pgfqpoint{1.250000in}{3.600000in}}%
\pgfusepath{stroke}%
\end{pgfscope}%
\begin{pgfscope}%
\pgfsetbuttcap%
\pgfsetroundjoin%
\definecolor{currentfill}{rgb}{0.000000,0.000000,0.000000}%
\pgfsetfillcolor{currentfill}%
\pgfsetlinewidth{0.501875pt}%
\definecolor{currentstroke}{rgb}{0.000000,0.000000,0.000000}%
\pgfsetstrokecolor{currentstroke}%
\pgfsetdash{}{0pt}%
\pgfsys@defobject{currentmarker}{\pgfqpoint{0.000000in}{0.000000in}}{\pgfqpoint{0.000000in}{0.055556in}}{%
\pgfpathmoveto{\pgfqpoint{0.000000in}{0.000000in}}%
\pgfpathlineto{\pgfqpoint{0.000000in}{0.055556in}}%
\pgfusepath{stroke,fill}%
}%
\begin{pgfscope}%
\pgfsys@transformshift{1.250000in}{0.400000in}%
\pgfsys@useobject{currentmarker}{}%
\end{pgfscope}%
\end{pgfscope}%
\begin{pgfscope}%
\pgfsetbuttcap%
\pgfsetroundjoin%
\definecolor{currentfill}{rgb}{0.000000,0.000000,0.000000}%
\pgfsetfillcolor{currentfill}%
\pgfsetlinewidth{0.501875pt}%
\definecolor{currentstroke}{rgb}{0.000000,0.000000,0.000000}%
\pgfsetstrokecolor{currentstroke}%
\pgfsetdash{}{0pt}%
\pgfsys@defobject{currentmarker}{\pgfqpoint{0.000000in}{-0.055556in}}{\pgfqpoint{0.000000in}{0.000000in}}{%
\pgfpathmoveto{\pgfqpoint{0.000000in}{0.000000in}}%
\pgfpathlineto{\pgfqpoint{0.000000in}{-0.055556in}}%
\pgfusepath{stroke,fill}%
}%
\begin{pgfscope}%
\pgfsys@transformshift{1.250000in}{3.600000in}%
\pgfsys@useobject{currentmarker}{}%
\end{pgfscope}%
\end{pgfscope}%
\begin{pgfscope}%
\pgftext[left,bottom,x=1.047648in,y=0.218387in,rotate=0.000000]{{\sffamily\fontsize{12.000000}{14.400000}\selectfont −2.0}}
%
\end{pgfscope}%
\begin{pgfscope}%
\pgfpathrectangle{\pgfqpoint{1.250000in}{0.400000in}}{\pgfqpoint{7.750000in}{3.200000in}} %
\pgfusepath{clip}%
\pgfsetbuttcap%
\pgfsetroundjoin%
\pgfsetlinewidth{0.501875pt}%
\definecolor{currentstroke}{rgb}{0.000000,0.000000,0.000000}%
\pgfsetstrokecolor{currentstroke}%
\pgfsetdash{{1.000000pt}{3.000000pt}}{0.000000pt}%
\pgfpathmoveto{\pgfqpoint{2.218750in}{0.400000in}}%
\pgfpathlineto{\pgfqpoint{2.218750in}{3.600000in}}%
\pgfusepath{stroke}%
\end{pgfscope}%
\begin{pgfscope}%
\pgfsetbuttcap%
\pgfsetroundjoin%
\definecolor{currentfill}{rgb}{0.000000,0.000000,0.000000}%
\pgfsetfillcolor{currentfill}%
\pgfsetlinewidth{0.501875pt}%
\definecolor{currentstroke}{rgb}{0.000000,0.000000,0.000000}%
\pgfsetstrokecolor{currentstroke}%
\pgfsetdash{}{0pt}%
\pgfsys@defobject{currentmarker}{\pgfqpoint{0.000000in}{0.000000in}}{\pgfqpoint{0.000000in}{0.055556in}}{%
\pgfpathmoveto{\pgfqpoint{0.000000in}{0.000000in}}%
\pgfpathlineto{\pgfqpoint{0.000000in}{0.055556in}}%
\pgfusepath{stroke,fill}%
}%
\begin{pgfscope}%
\pgfsys@transformshift{2.218750in}{0.400000in}%
\pgfsys@useobject{currentmarker}{}%
\end{pgfscope}%
\end{pgfscope}%
\begin{pgfscope}%
\pgfsetbuttcap%
\pgfsetroundjoin%
\definecolor{currentfill}{rgb}{0.000000,0.000000,0.000000}%
\pgfsetfillcolor{currentfill}%
\pgfsetlinewidth{0.501875pt}%
\definecolor{currentstroke}{rgb}{0.000000,0.000000,0.000000}%
\pgfsetstrokecolor{currentstroke}%
\pgfsetdash{}{0pt}%
\pgfsys@defobject{currentmarker}{\pgfqpoint{0.000000in}{-0.055556in}}{\pgfqpoint{0.000000in}{0.000000in}}{%
\pgfpathmoveto{\pgfqpoint{0.000000in}{0.000000in}}%
\pgfpathlineto{\pgfqpoint{0.000000in}{-0.055556in}}%
\pgfusepath{stroke,fill}%
}%
\begin{pgfscope}%
\pgfsys@transformshift{2.218750in}{3.600000in}%
\pgfsys@useobject{currentmarker}{}%
\end{pgfscope}%
\end{pgfscope}%
\begin{pgfscope}%
\pgftext[left,bottom,x=2.016398in,y=0.220584in,rotate=0.000000]{{\sffamily\fontsize{12.000000}{14.400000}\selectfont −1.5}}
%
\end{pgfscope}%
\begin{pgfscope}%
\pgfpathrectangle{\pgfqpoint{1.250000in}{0.400000in}}{\pgfqpoint{7.750000in}{3.200000in}} %
\pgfusepath{clip}%
\pgfsetbuttcap%
\pgfsetroundjoin%
\pgfsetlinewidth{0.501875pt}%
\definecolor{currentstroke}{rgb}{0.000000,0.000000,0.000000}%
\pgfsetstrokecolor{currentstroke}%
\pgfsetdash{{1.000000pt}{3.000000pt}}{0.000000pt}%
\pgfpathmoveto{\pgfqpoint{3.187500in}{0.400000in}}%
\pgfpathlineto{\pgfqpoint{3.187500in}{3.600000in}}%
\pgfusepath{stroke}%
\end{pgfscope}%
\begin{pgfscope}%
\pgfsetbuttcap%
\pgfsetroundjoin%
\definecolor{currentfill}{rgb}{0.000000,0.000000,0.000000}%
\pgfsetfillcolor{currentfill}%
\pgfsetlinewidth{0.501875pt}%
\definecolor{currentstroke}{rgb}{0.000000,0.000000,0.000000}%
\pgfsetstrokecolor{currentstroke}%
\pgfsetdash{}{0pt}%
\pgfsys@defobject{currentmarker}{\pgfqpoint{0.000000in}{0.000000in}}{\pgfqpoint{0.000000in}{0.055556in}}{%
\pgfpathmoveto{\pgfqpoint{0.000000in}{0.000000in}}%
\pgfpathlineto{\pgfqpoint{0.000000in}{0.055556in}}%
\pgfusepath{stroke,fill}%
}%
\begin{pgfscope}%
\pgfsys@transformshift{3.187500in}{0.400000in}%
\pgfsys@useobject{currentmarker}{}%
\end{pgfscope}%
\end{pgfscope}%
\begin{pgfscope}%
\pgfsetbuttcap%
\pgfsetroundjoin%
\definecolor{currentfill}{rgb}{0.000000,0.000000,0.000000}%
\pgfsetfillcolor{currentfill}%
\pgfsetlinewidth{0.501875pt}%
\definecolor{currentstroke}{rgb}{0.000000,0.000000,0.000000}%
\pgfsetstrokecolor{currentstroke}%
\pgfsetdash{}{0pt}%
\pgfsys@defobject{currentmarker}{\pgfqpoint{0.000000in}{-0.055556in}}{\pgfqpoint{0.000000in}{0.000000in}}{%
\pgfpathmoveto{\pgfqpoint{0.000000in}{0.000000in}}%
\pgfpathlineto{\pgfqpoint{0.000000in}{-0.055556in}}%
\pgfusepath{stroke,fill}%
}%
\begin{pgfscope}%
\pgfsys@transformshift{3.187500in}{3.600000in}%
\pgfsys@useobject{currentmarker}{}%
\end{pgfscope}%
\end{pgfscope}%
\begin{pgfscope}%
\pgftext[left,bottom,x=2.985148in,y=0.218387in,rotate=0.000000]{{\sffamily\fontsize{12.000000}{14.400000}\selectfont −1.0}}
%
\end{pgfscope}%
\begin{pgfscope}%
\pgfpathrectangle{\pgfqpoint{1.250000in}{0.400000in}}{\pgfqpoint{7.750000in}{3.200000in}} %
\pgfusepath{clip}%
\pgfsetbuttcap%
\pgfsetroundjoin%
\pgfsetlinewidth{0.501875pt}%
\definecolor{currentstroke}{rgb}{0.000000,0.000000,0.000000}%
\pgfsetstrokecolor{currentstroke}%
\pgfsetdash{{1.000000pt}{3.000000pt}}{0.000000pt}%
\pgfpathmoveto{\pgfqpoint{4.156250in}{0.400000in}}%
\pgfpathlineto{\pgfqpoint{4.156250in}{3.600000in}}%
\pgfusepath{stroke}%
\end{pgfscope}%
\begin{pgfscope}%
\pgfsetbuttcap%
\pgfsetroundjoin%
\definecolor{currentfill}{rgb}{0.000000,0.000000,0.000000}%
\pgfsetfillcolor{currentfill}%
\pgfsetlinewidth{0.501875pt}%
\definecolor{currentstroke}{rgb}{0.000000,0.000000,0.000000}%
\pgfsetstrokecolor{currentstroke}%
\pgfsetdash{}{0pt}%
\pgfsys@defobject{currentmarker}{\pgfqpoint{0.000000in}{0.000000in}}{\pgfqpoint{0.000000in}{0.055556in}}{%
\pgfpathmoveto{\pgfqpoint{0.000000in}{0.000000in}}%
\pgfpathlineto{\pgfqpoint{0.000000in}{0.055556in}}%
\pgfusepath{stroke,fill}%
}%
\begin{pgfscope}%
\pgfsys@transformshift{4.156250in}{0.400000in}%
\pgfsys@useobject{currentmarker}{}%
\end{pgfscope}%
\end{pgfscope}%
\begin{pgfscope}%
\pgfsetbuttcap%
\pgfsetroundjoin%
\definecolor{currentfill}{rgb}{0.000000,0.000000,0.000000}%
\pgfsetfillcolor{currentfill}%
\pgfsetlinewidth{0.501875pt}%
\definecolor{currentstroke}{rgb}{0.000000,0.000000,0.000000}%
\pgfsetstrokecolor{currentstroke}%
\pgfsetdash{}{0pt}%
\pgfsys@defobject{currentmarker}{\pgfqpoint{0.000000in}{-0.055556in}}{\pgfqpoint{0.000000in}{0.000000in}}{%
\pgfpathmoveto{\pgfqpoint{0.000000in}{0.000000in}}%
\pgfpathlineto{\pgfqpoint{0.000000in}{-0.055556in}}%
\pgfusepath{stroke,fill}%
}%
\begin{pgfscope}%
\pgfsys@transformshift{4.156250in}{3.600000in}%
\pgfsys@useobject{currentmarker}{}%
\end{pgfscope}%
\end{pgfscope}%
\begin{pgfscope}%
\pgftext[left,bottom,x=3.953898in,y=0.218387in,rotate=0.000000]{{\sffamily\fontsize{12.000000}{14.400000}\selectfont −0.5}}
%
\end{pgfscope}%
\begin{pgfscope}%
\pgfpathrectangle{\pgfqpoint{1.250000in}{0.400000in}}{\pgfqpoint{7.750000in}{3.200000in}} %
\pgfusepath{clip}%
\pgfsetbuttcap%
\pgfsetroundjoin%
\pgfsetlinewidth{0.501875pt}%
\definecolor{currentstroke}{rgb}{0.000000,0.000000,0.000000}%
\pgfsetstrokecolor{currentstroke}%
\pgfsetdash{{1.000000pt}{3.000000pt}}{0.000000pt}%
\pgfpathmoveto{\pgfqpoint{5.125000in}{0.400000in}}%
\pgfpathlineto{\pgfqpoint{5.125000in}{3.600000in}}%
\pgfusepath{stroke}%
\end{pgfscope}%
\begin{pgfscope}%
\pgfsetbuttcap%
\pgfsetroundjoin%
\definecolor{currentfill}{rgb}{0.000000,0.000000,0.000000}%
\pgfsetfillcolor{currentfill}%
\pgfsetlinewidth{0.501875pt}%
\definecolor{currentstroke}{rgb}{0.000000,0.000000,0.000000}%
\pgfsetstrokecolor{currentstroke}%
\pgfsetdash{}{0pt}%
\pgfsys@defobject{currentmarker}{\pgfqpoint{0.000000in}{0.000000in}}{\pgfqpoint{0.000000in}{0.055556in}}{%
\pgfpathmoveto{\pgfqpoint{0.000000in}{0.000000in}}%
\pgfpathlineto{\pgfqpoint{0.000000in}{0.055556in}}%
\pgfusepath{stroke,fill}%
}%
\begin{pgfscope}%
\pgfsys@transformshift{5.125000in}{0.400000in}%
\pgfsys@useobject{currentmarker}{}%
\end{pgfscope}%
\end{pgfscope}%
\begin{pgfscope}%
\pgfsetbuttcap%
\pgfsetroundjoin%
\definecolor{currentfill}{rgb}{0.000000,0.000000,0.000000}%
\pgfsetfillcolor{currentfill}%
\pgfsetlinewidth{0.501875pt}%
\definecolor{currentstroke}{rgb}{0.000000,0.000000,0.000000}%
\pgfsetstrokecolor{currentstroke}%
\pgfsetdash{}{0pt}%
\pgfsys@defobject{currentmarker}{\pgfqpoint{0.000000in}{-0.055556in}}{\pgfqpoint{0.000000in}{0.000000in}}{%
\pgfpathmoveto{\pgfqpoint{0.000000in}{0.000000in}}%
\pgfpathlineto{\pgfqpoint{0.000000in}{-0.055556in}}%
\pgfusepath{stroke,fill}%
}%
\begin{pgfscope}%
\pgfsys@transformshift{5.125000in}{3.600000in}%
\pgfsys@useobject{currentmarker}{}%
\end{pgfscope}%
\end{pgfscope}%
\begin{pgfscope}%
\pgftext[left,bottom,x=4.992472in,y=0.218387in,rotate=0.000000]{{\sffamily\fontsize{12.000000}{14.400000}\selectfont 0.0}}
%
\end{pgfscope}%
\begin{pgfscope}%
\pgfpathrectangle{\pgfqpoint{1.250000in}{0.400000in}}{\pgfqpoint{7.750000in}{3.200000in}} %
\pgfusepath{clip}%
\pgfsetbuttcap%
\pgfsetroundjoin%
\pgfsetlinewidth{0.501875pt}%
\definecolor{currentstroke}{rgb}{0.000000,0.000000,0.000000}%
\pgfsetstrokecolor{currentstroke}%
\pgfsetdash{{1.000000pt}{3.000000pt}}{0.000000pt}%
\pgfpathmoveto{\pgfqpoint{6.093750in}{0.400000in}}%
\pgfpathlineto{\pgfqpoint{6.093750in}{3.600000in}}%
\pgfusepath{stroke}%
\end{pgfscope}%
\begin{pgfscope}%
\pgfsetbuttcap%
\pgfsetroundjoin%
\definecolor{currentfill}{rgb}{0.000000,0.000000,0.000000}%
\pgfsetfillcolor{currentfill}%
\pgfsetlinewidth{0.501875pt}%
\definecolor{currentstroke}{rgb}{0.000000,0.000000,0.000000}%
\pgfsetstrokecolor{currentstroke}%
\pgfsetdash{}{0pt}%
\pgfsys@defobject{currentmarker}{\pgfqpoint{0.000000in}{0.000000in}}{\pgfqpoint{0.000000in}{0.055556in}}{%
\pgfpathmoveto{\pgfqpoint{0.000000in}{0.000000in}}%
\pgfpathlineto{\pgfqpoint{0.000000in}{0.055556in}}%
\pgfusepath{stroke,fill}%
}%
\begin{pgfscope}%
\pgfsys@transformshift{6.093750in}{0.400000in}%
\pgfsys@useobject{currentmarker}{}%
\end{pgfscope}%
\end{pgfscope}%
\begin{pgfscope}%
\pgfsetbuttcap%
\pgfsetroundjoin%
\definecolor{currentfill}{rgb}{0.000000,0.000000,0.000000}%
\pgfsetfillcolor{currentfill}%
\pgfsetlinewidth{0.501875pt}%
\definecolor{currentstroke}{rgb}{0.000000,0.000000,0.000000}%
\pgfsetstrokecolor{currentstroke}%
\pgfsetdash{}{0pt}%
\pgfsys@defobject{currentmarker}{\pgfqpoint{0.000000in}{-0.055556in}}{\pgfqpoint{0.000000in}{0.000000in}}{%
\pgfpathmoveto{\pgfqpoint{0.000000in}{0.000000in}}%
\pgfpathlineto{\pgfqpoint{0.000000in}{-0.055556in}}%
\pgfusepath{stroke,fill}%
}%
\begin{pgfscope}%
\pgfsys@transformshift{6.093750in}{3.600000in}%
\pgfsys@useobject{currentmarker}{}%
\end{pgfscope}%
\end{pgfscope}%
\begin{pgfscope}%
\pgftext[left,bottom,x=5.961222in,y=0.218387in,rotate=0.000000]{{\sffamily\fontsize{12.000000}{14.400000}\selectfont 0.5}}
%
\end{pgfscope}%
\begin{pgfscope}%
\pgfpathrectangle{\pgfqpoint{1.250000in}{0.400000in}}{\pgfqpoint{7.750000in}{3.200000in}} %
\pgfusepath{clip}%
\pgfsetbuttcap%
\pgfsetroundjoin%
\pgfsetlinewidth{0.501875pt}%
\definecolor{currentstroke}{rgb}{0.000000,0.000000,0.000000}%
\pgfsetstrokecolor{currentstroke}%
\pgfsetdash{{1.000000pt}{3.000000pt}}{0.000000pt}%
\pgfpathmoveto{\pgfqpoint{7.062500in}{0.400000in}}%
\pgfpathlineto{\pgfqpoint{7.062500in}{3.600000in}}%
\pgfusepath{stroke}%
\end{pgfscope}%
\begin{pgfscope}%
\pgfsetbuttcap%
\pgfsetroundjoin%
\definecolor{currentfill}{rgb}{0.000000,0.000000,0.000000}%
\pgfsetfillcolor{currentfill}%
\pgfsetlinewidth{0.501875pt}%
\definecolor{currentstroke}{rgb}{0.000000,0.000000,0.000000}%
\pgfsetstrokecolor{currentstroke}%
\pgfsetdash{}{0pt}%
\pgfsys@defobject{currentmarker}{\pgfqpoint{0.000000in}{0.000000in}}{\pgfqpoint{0.000000in}{0.055556in}}{%
\pgfpathmoveto{\pgfqpoint{0.000000in}{0.000000in}}%
\pgfpathlineto{\pgfqpoint{0.000000in}{0.055556in}}%
\pgfusepath{stroke,fill}%
}%
\begin{pgfscope}%
\pgfsys@transformshift{7.062500in}{0.400000in}%
\pgfsys@useobject{currentmarker}{}%
\end{pgfscope}%
\end{pgfscope}%
\begin{pgfscope}%
\pgfsetbuttcap%
\pgfsetroundjoin%
\definecolor{currentfill}{rgb}{0.000000,0.000000,0.000000}%
\pgfsetfillcolor{currentfill}%
\pgfsetlinewidth{0.501875pt}%
\definecolor{currentstroke}{rgb}{0.000000,0.000000,0.000000}%
\pgfsetstrokecolor{currentstroke}%
\pgfsetdash{}{0pt}%
\pgfsys@defobject{currentmarker}{\pgfqpoint{0.000000in}{-0.055556in}}{\pgfqpoint{0.000000in}{0.000000in}}{%
\pgfpathmoveto{\pgfqpoint{0.000000in}{0.000000in}}%
\pgfpathlineto{\pgfqpoint{0.000000in}{-0.055556in}}%
\pgfusepath{stroke,fill}%
}%
\begin{pgfscope}%
\pgfsys@transformshift{7.062500in}{3.600000in}%
\pgfsys@useobject{currentmarker}{}%
\end{pgfscope}%
\end{pgfscope}%
\begin{pgfscope}%
\pgftext[left,bottom,x=6.929972in,y=0.218387in,rotate=0.000000]{{\sffamily\fontsize{12.000000}{14.400000}\selectfont 1.0}}
%
\end{pgfscope}%
\begin{pgfscope}%
\pgfpathrectangle{\pgfqpoint{1.250000in}{0.400000in}}{\pgfqpoint{7.750000in}{3.200000in}} %
\pgfusepath{clip}%
\pgfsetbuttcap%
\pgfsetroundjoin%
\pgfsetlinewidth{0.501875pt}%
\definecolor{currentstroke}{rgb}{0.000000,0.000000,0.000000}%
\pgfsetstrokecolor{currentstroke}%
\pgfsetdash{{1.000000pt}{3.000000pt}}{0.000000pt}%
\pgfpathmoveto{\pgfqpoint{8.031250in}{0.400000in}}%
\pgfpathlineto{\pgfqpoint{8.031250in}{3.600000in}}%
\pgfusepath{stroke}%
\end{pgfscope}%
\begin{pgfscope}%
\pgfsetbuttcap%
\pgfsetroundjoin%
\definecolor{currentfill}{rgb}{0.000000,0.000000,0.000000}%
\pgfsetfillcolor{currentfill}%
\pgfsetlinewidth{0.501875pt}%
\definecolor{currentstroke}{rgb}{0.000000,0.000000,0.000000}%
\pgfsetstrokecolor{currentstroke}%
\pgfsetdash{}{0pt}%
\pgfsys@defobject{currentmarker}{\pgfqpoint{0.000000in}{0.000000in}}{\pgfqpoint{0.000000in}{0.055556in}}{%
\pgfpathmoveto{\pgfqpoint{0.000000in}{0.000000in}}%
\pgfpathlineto{\pgfqpoint{0.000000in}{0.055556in}}%
\pgfusepath{stroke,fill}%
}%
\begin{pgfscope}%
\pgfsys@transformshift{8.031250in}{0.400000in}%
\pgfsys@useobject{currentmarker}{}%
\end{pgfscope}%
\end{pgfscope}%
\begin{pgfscope}%
\pgfsetbuttcap%
\pgfsetroundjoin%
\definecolor{currentfill}{rgb}{0.000000,0.000000,0.000000}%
\pgfsetfillcolor{currentfill}%
\pgfsetlinewidth{0.501875pt}%
\definecolor{currentstroke}{rgb}{0.000000,0.000000,0.000000}%
\pgfsetstrokecolor{currentstroke}%
\pgfsetdash{}{0pt}%
\pgfsys@defobject{currentmarker}{\pgfqpoint{0.000000in}{-0.055556in}}{\pgfqpoint{0.000000in}{0.000000in}}{%
\pgfpathmoveto{\pgfqpoint{0.000000in}{0.000000in}}%
\pgfpathlineto{\pgfqpoint{0.000000in}{-0.055556in}}%
\pgfusepath{stroke,fill}%
}%
\begin{pgfscope}%
\pgfsys@transformshift{8.031250in}{3.600000in}%
\pgfsys@useobject{currentmarker}{}%
\end{pgfscope}%
\end{pgfscope}%
\begin{pgfscope}%
\pgftext[left,bottom,x=7.898722in,y=0.220584in,rotate=0.000000]{{\sffamily\fontsize{12.000000}{14.400000}\selectfont 1.5}}
%
\end{pgfscope}%
\begin{pgfscope}%
\pgfpathrectangle{\pgfqpoint{1.250000in}{0.400000in}}{\pgfqpoint{7.750000in}{3.200000in}} %
\pgfusepath{clip}%
\pgfsetbuttcap%
\pgfsetroundjoin%
\pgfsetlinewidth{0.501875pt}%
\definecolor{currentstroke}{rgb}{0.000000,0.000000,0.000000}%
\pgfsetstrokecolor{currentstroke}%
\pgfsetdash{{1.000000pt}{3.000000pt}}{0.000000pt}%
\pgfpathmoveto{\pgfqpoint{9.000000in}{0.400000in}}%
\pgfpathlineto{\pgfqpoint{9.000000in}{3.600000in}}%
\pgfusepath{stroke}%
\end{pgfscope}%
\begin{pgfscope}%
\pgfsetbuttcap%
\pgfsetroundjoin%
\definecolor{currentfill}{rgb}{0.000000,0.000000,0.000000}%
\pgfsetfillcolor{currentfill}%
\pgfsetlinewidth{0.501875pt}%
\definecolor{currentstroke}{rgb}{0.000000,0.000000,0.000000}%
\pgfsetstrokecolor{currentstroke}%
\pgfsetdash{}{0pt}%
\pgfsys@defobject{currentmarker}{\pgfqpoint{0.000000in}{0.000000in}}{\pgfqpoint{0.000000in}{0.055556in}}{%
\pgfpathmoveto{\pgfqpoint{0.000000in}{0.000000in}}%
\pgfpathlineto{\pgfqpoint{0.000000in}{0.055556in}}%
\pgfusepath{stroke,fill}%
}%
\begin{pgfscope}%
\pgfsys@transformshift{9.000000in}{0.400000in}%
\pgfsys@useobject{currentmarker}{}%
\end{pgfscope}%
\end{pgfscope}%
\begin{pgfscope}%
\pgfsetbuttcap%
\pgfsetroundjoin%
\definecolor{currentfill}{rgb}{0.000000,0.000000,0.000000}%
\pgfsetfillcolor{currentfill}%
\pgfsetlinewidth{0.501875pt}%
\definecolor{currentstroke}{rgb}{0.000000,0.000000,0.000000}%
\pgfsetstrokecolor{currentstroke}%
\pgfsetdash{}{0pt}%
\pgfsys@defobject{currentmarker}{\pgfqpoint{0.000000in}{-0.055556in}}{\pgfqpoint{0.000000in}{0.000000in}}{%
\pgfpathmoveto{\pgfqpoint{0.000000in}{0.000000in}}%
\pgfpathlineto{\pgfqpoint{0.000000in}{-0.055556in}}%
\pgfusepath{stroke,fill}%
}%
\begin{pgfscope}%
\pgfsys@transformshift{9.000000in}{3.600000in}%
\pgfsys@useobject{currentmarker}{}%
\end{pgfscope}%
\end{pgfscope}%
\begin{pgfscope}%
\pgftext[left,bottom,x=8.867472in,y=0.218387in,rotate=0.000000]{{\sffamily\fontsize{12.000000}{14.400000}\selectfont 2.0}}
%
\end{pgfscope}%
\begin{pgfscope}%
\pgfpathrectangle{\pgfqpoint{1.250000in}{0.400000in}}{\pgfqpoint{7.750000in}{3.200000in}} %
\pgfusepath{clip}%
\pgfsetbuttcap%
\pgfsetroundjoin%
\pgfsetlinewidth{0.501875pt}%
\definecolor{currentstroke}{rgb}{0.000000,0.000000,0.000000}%
\pgfsetstrokecolor{currentstroke}%
\pgfsetdash{{1.000000pt}{3.000000pt}}{0.000000pt}%
\pgfpathmoveto{\pgfqpoint{1.250000in}{0.400000in}}%
\pgfpathlineto{\pgfqpoint{9.000000in}{0.400000in}}%
\pgfusepath{stroke}%
\end{pgfscope}%
\begin{pgfscope}%
\pgfsetbuttcap%
\pgfsetroundjoin%
\definecolor{currentfill}{rgb}{0.000000,0.000000,0.000000}%
\pgfsetfillcolor{currentfill}%
\pgfsetlinewidth{0.501875pt}%
\definecolor{currentstroke}{rgb}{0.000000,0.000000,0.000000}%
\pgfsetstrokecolor{currentstroke}%
\pgfsetdash{}{0pt}%
\pgfsys@defobject{currentmarker}{\pgfqpoint{0.000000in}{0.000000in}}{\pgfqpoint{0.055556in}{0.000000in}}{%
\pgfpathmoveto{\pgfqpoint{0.000000in}{0.000000in}}%
\pgfpathlineto{\pgfqpoint{0.055556in}{0.000000in}}%
\pgfusepath{stroke,fill}%
}%
\begin{pgfscope}%
\pgfsys@transformshift{1.250000in}{0.400000in}%
\pgfsys@useobject{currentmarker}{}%
\end{pgfscope}%
\end{pgfscope}%
\begin{pgfscope}%
\pgfsetbuttcap%
\pgfsetroundjoin%
\definecolor{currentfill}{rgb}{0.000000,0.000000,0.000000}%
\pgfsetfillcolor{currentfill}%
\pgfsetlinewidth{0.501875pt}%
\definecolor{currentstroke}{rgb}{0.000000,0.000000,0.000000}%
\pgfsetstrokecolor{currentstroke}%
\pgfsetdash{}{0pt}%
\pgfsys@defobject{currentmarker}{\pgfqpoint{-0.055556in}{0.000000in}}{\pgfqpoint{0.000000in}{0.000000in}}{%
\pgfpathmoveto{\pgfqpoint{0.000000in}{0.000000in}}%
\pgfpathlineto{\pgfqpoint{-0.055556in}{0.000000in}}%
\pgfusepath{stroke,fill}%
}%
\begin{pgfscope}%
\pgfsys@transformshift{9.000000in}{0.400000in}%
\pgfsys@useobject{currentmarker}{}%
\end{pgfscope}%
\end{pgfscope}%
\begin{pgfscope}%
\pgftext[left,bottom,x=0.789741in,y=0.336971in,rotate=0.000000]{{\sffamily\fontsize{12.000000}{14.400000}\selectfont −2.0}}
%
\end{pgfscope}%
\begin{pgfscope}%
\pgfpathrectangle{\pgfqpoint{1.250000in}{0.400000in}}{\pgfqpoint{7.750000in}{3.200000in}} %
\pgfusepath{clip}%
\pgfsetbuttcap%
\pgfsetroundjoin%
\pgfsetlinewidth{0.501875pt}%
\definecolor{currentstroke}{rgb}{0.000000,0.000000,0.000000}%
\pgfsetstrokecolor{currentstroke}%
\pgfsetdash{{1.000000pt}{3.000000pt}}{0.000000pt}%
\pgfpathmoveto{\pgfqpoint{1.250000in}{0.800000in}}%
\pgfpathlineto{\pgfqpoint{9.000000in}{0.800000in}}%
\pgfusepath{stroke}%
\end{pgfscope}%
\begin{pgfscope}%
\pgfsetbuttcap%
\pgfsetroundjoin%
\definecolor{currentfill}{rgb}{0.000000,0.000000,0.000000}%
\pgfsetfillcolor{currentfill}%
\pgfsetlinewidth{0.501875pt}%
\definecolor{currentstroke}{rgb}{0.000000,0.000000,0.000000}%
\pgfsetstrokecolor{currentstroke}%
\pgfsetdash{}{0pt}%
\pgfsys@defobject{currentmarker}{\pgfqpoint{0.000000in}{0.000000in}}{\pgfqpoint{0.055556in}{0.000000in}}{%
\pgfpathmoveto{\pgfqpoint{0.000000in}{0.000000in}}%
\pgfpathlineto{\pgfqpoint{0.055556in}{0.000000in}}%
\pgfusepath{stroke,fill}%
}%
\begin{pgfscope}%
\pgfsys@transformshift{1.250000in}{0.800000in}%
\pgfsys@useobject{currentmarker}{}%
\end{pgfscope}%
\end{pgfscope}%
\begin{pgfscope}%
\pgfsetbuttcap%
\pgfsetroundjoin%
\definecolor{currentfill}{rgb}{0.000000,0.000000,0.000000}%
\pgfsetfillcolor{currentfill}%
\pgfsetlinewidth{0.501875pt}%
\definecolor{currentstroke}{rgb}{0.000000,0.000000,0.000000}%
\pgfsetstrokecolor{currentstroke}%
\pgfsetdash{}{0pt}%
\pgfsys@defobject{currentmarker}{\pgfqpoint{-0.055556in}{0.000000in}}{\pgfqpoint{0.000000in}{0.000000in}}{%
\pgfpathmoveto{\pgfqpoint{0.000000in}{0.000000in}}%
\pgfpathlineto{\pgfqpoint{-0.055556in}{0.000000in}}%
\pgfusepath{stroke,fill}%
}%
\begin{pgfscope}%
\pgfsys@transformshift{9.000000in}{0.800000in}%
\pgfsys@useobject{currentmarker}{}%
\end{pgfscope}%
\end{pgfscope}%
\begin{pgfscope}%
\pgftext[left,bottom,x=0.789741in,y=0.738070in,rotate=0.000000]{{\sffamily\fontsize{12.000000}{14.400000}\selectfont −1.5}}
%
\end{pgfscope}%
\begin{pgfscope}%
\pgfpathrectangle{\pgfqpoint{1.250000in}{0.400000in}}{\pgfqpoint{7.750000in}{3.200000in}} %
\pgfusepath{clip}%
\pgfsetbuttcap%
\pgfsetroundjoin%
\pgfsetlinewidth{0.501875pt}%
\definecolor{currentstroke}{rgb}{0.000000,0.000000,0.000000}%
\pgfsetstrokecolor{currentstroke}%
\pgfsetdash{{1.000000pt}{3.000000pt}}{0.000000pt}%
\pgfpathmoveto{\pgfqpoint{1.250000in}{1.200000in}}%
\pgfpathlineto{\pgfqpoint{9.000000in}{1.200000in}}%
\pgfusepath{stroke}%
\end{pgfscope}%
\begin{pgfscope}%
\pgfsetbuttcap%
\pgfsetroundjoin%
\definecolor{currentfill}{rgb}{0.000000,0.000000,0.000000}%
\pgfsetfillcolor{currentfill}%
\pgfsetlinewidth{0.501875pt}%
\definecolor{currentstroke}{rgb}{0.000000,0.000000,0.000000}%
\pgfsetstrokecolor{currentstroke}%
\pgfsetdash{}{0pt}%
\pgfsys@defobject{currentmarker}{\pgfqpoint{0.000000in}{0.000000in}}{\pgfqpoint{0.055556in}{0.000000in}}{%
\pgfpathmoveto{\pgfqpoint{0.000000in}{0.000000in}}%
\pgfpathlineto{\pgfqpoint{0.055556in}{0.000000in}}%
\pgfusepath{stroke,fill}%
}%
\begin{pgfscope}%
\pgfsys@transformshift{1.250000in}{1.200000in}%
\pgfsys@useobject{currentmarker}{}%
\end{pgfscope}%
\end{pgfscope}%
\begin{pgfscope}%
\pgfsetbuttcap%
\pgfsetroundjoin%
\definecolor{currentfill}{rgb}{0.000000,0.000000,0.000000}%
\pgfsetfillcolor{currentfill}%
\pgfsetlinewidth{0.501875pt}%
\definecolor{currentstroke}{rgb}{0.000000,0.000000,0.000000}%
\pgfsetstrokecolor{currentstroke}%
\pgfsetdash{}{0pt}%
\pgfsys@defobject{currentmarker}{\pgfqpoint{-0.055556in}{0.000000in}}{\pgfqpoint{0.000000in}{0.000000in}}{%
\pgfpathmoveto{\pgfqpoint{0.000000in}{0.000000in}}%
\pgfpathlineto{\pgfqpoint{-0.055556in}{0.000000in}}%
\pgfusepath{stroke,fill}%
}%
\begin{pgfscope}%
\pgfsys@transformshift{9.000000in}{1.200000in}%
\pgfsys@useobject{currentmarker}{}%
\end{pgfscope}%
\end{pgfscope}%
\begin{pgfscope}%
\pgftext[left,bottom,x=0.789741in,y=1.136971in,rotate=0.000000]{{\sffamily\fontsize{12.000000}{14.400000}\selectfont −1.0}}
%
\end{pgfscope}%
\begin{pgfscope}%
\pgfpathrectangle{\pgfqpoint{1.250000in}{0.400000in}}{\pgfqpoint{7.750000in}{3.200000in}} %
\pgfusepath{clip}%
\pgfsetbuttcap%
\pgfsetroundjoin%
\pgfsetlinewidth{0.501875pt}%
\definecolor{currentstroke}{rgb}{0.000000,0.000000,0.000000}%
\pgfsetstrokecolor{currentstroke}%
\pgfsetdash{{1.000000pt}{3.000000pt}}{0.000000pt}%
\pgfpathmoveto{\pgfqpoint{1.250000in}{1.600000in}}%
\pgfpathlineto{\pgfqpoint{9.000000in}{1.600000in}}%
\pgfusepath{stroke}%
\end{pgfscope}%
\begin{pgfscope}%
\pgfsetbuttcap%
\pgfsetroundjoin%
\definecolor{currentfill}{rgb}{0.000000,0.000000,0.000000}%
\pgfsetfillcolor{currentfill}%
\pgfsetlinewidth{0.501875pt}%
\definecolor{currentstroke}{rgb}{0.000000,0.000000,0.000000}%
\pgfsetstrokecolor{currentstroke}%
\pgfsetdash{}{0pt}%
\pgfsys@defobject{currentmarker}{\pgfqpoint{0.000000in}{0.000000in}}{\pgfqpoint{0.055556in}{0.000000in}}{%
\pgfpathmoveto{\pgfqpoint{0.000000in}{0.000000in}}%
\pgfpathlineto{\pgfqpoint{0.055556in}{0.000000in}}%
\pgfusepath{stroke,fill}%
}%
\begin{pgfscope}%
\pgfsys@transformshift{1.250000in}{1.600000in}%
\pgfsys@useobject{currentmarker}{}%
\end{pgfscope}%
\end{pgfscope}%
\begin{pgfscope}%
\pgfsetbuttcap%
\pgfsetroundjoin%
\definecolor{currentfill}{rgb}{0.000000,0.000000,0.000000}%
\pgfsetfillcolor{currentfill}%
\pgfsetlinewidth{0.501875pt}%
\definecolor{currentstroke}{rgb}{0.000000,0.000000,0.000000}%
\pgfsetstrokecolor{currentstroke}%
\pgfsetdash{}{0pt}%
\pgfsys@defobject{currentmarker}{\pgfqpoint{-0.055556in}{0.000000in}}{\pgfqpoint{0.000000in}{0.000000in}}{%
\pgfpathmoveto{\pgfqpoint{0.000000in}{0.000000in}}%
\pgfpathlineto{\pgfqpoint{-0.055556in}{0.000000in}}%
\pgfusepath{stroke,fill}%
}%
\begin{pgfscope}%
\pgfsys@transformshift{9.000000in}{1.600000in}%
\pgfsys@useobject{currentmarker}{}%
\end{pgfscope}%
\end{pgfscope}%
\begin{pgfscope}%
\pgftext[left,bottom,x=0.789741in,y=1.536971in,rotate=0.000000]{{\sffamily\fontsize{12.000000}{14.400000}\selectfont −0.5}}
%
\end{pgfscope}%
\begin{pgfscope}%
\pgfpathrectangle{\pgfqpoint{1.250000in}{0.400000in}}{\pgfqpoint{7.750000in}{3.200000in}} %
\pgfusepath{clip}%
\pgfsetbuttcap%
\pgfsetroundjoin%
\pgfsetlinewidth{0.501875pt}%
\definecolor{currentstroke}{rgb}{0.000000,0.000000,0.000000}%
\pgfsetstrokecolor{currentstroke}%
\pgfsetdash{{1.000000pt}{3.000000pt}}{0.000000pt}%
\pgfpathmoveto{\pgfqpoint{1.250000in}{2.000000in}}%
\pgfpathlineto{\pgfqpoint{9.000000in}{2.000000in}}%
\pgfusepath{stroke}%
\end{pgfscope}%
\begin{pgfscope}%
\pgfsetbuttcap%
\pgfsetroundjoin%
\definecolor{currentfill}{rgb}{0.000000,0.000000,0.000000}%
\pgfsetfillcolor{currentfill}%
\pgfsetlinewidth{0.501875pt}%
\definecolor{currentstroke}{rgb}{0.000000,0.000000,0.000000}%
\pgfsetstrokecolor{currentstroke}%
\pgfsetdash{}{0pt}%
\pgfsys@defobject{currentmarker}{\pgfqpoint{0.000000in}{0.000000in}}{\pgfqpoint{0.055556in}{0.000000in}}{%
\pgfpathmoveto{\pgfqpoint{0.000000in}{0.000000in}}%
\pgfpathlineto{\pgfqpoint{0.055556in}{0.000000in}}%
\pgfusepath{stroke,fill}%
}%
\begin{pgfscope}%
\pgfsys@transformshift{1.250000in}{2.000000in}%
\pgfsys@useobject{currentmarker}{}%
\end{pgfscope}%
\end{pgfscope}%
\begin{pgfscope}%
\pgfsetbuttcap%
\pgfsetroundjoin%
\definecolor{currentfill}{rgb}{0.000000,0.000000,0.000000}%
\pgfsetfillcolor{currentfill}%
\pgfsetlinewidth{0.501875pt}%
\definecolor{currentstroke}{rgb}{0.000000,0.000000,0.000000}%
\pgfsetstrokecolor{currentstroke}%
\pgfsetdash{}{0pt}%
\pgfsys@defobject{currentmarker}{\pgfqpoint{-0.055556in}{0.000000in}}{\pgfqpoint{0.000000in}{0.000000in}}{%
\pgfpathmoveto{\pgfqpoint{0.000000in}{0.000000in}}%
\pgfpathlineto{\pgfqpoint{-0.055556in}{0.000000in}}%
\pgfusepath{stroke,fill}%
}%
\begin{pgfscope}%
\pgfsys@transformshift{9.000000in}{2.000000in}%
\pgfsys@useobject{currentmarker}{}%
\end{pgfscope}%
\end{pgfscope}%
\begin{pgfscope}%
\pgftext[left,bottom,x=0.929389in,y=1.936971in,rotate=0.000000]{{\sffamily\fontsize{12.000000}{14.400000}\selectfont 0.0}}
%
\end{pgfscope}%
\begin{pgfscope}%
\pgfpathrectangle{\pgfqpoint{1.250000in}{0.400000in}}{\pgfqpoint{7.750000in}{3.200000in}} %
\pgfusepath{clip}%
\pgfsetbuttcap%
\pgfsetroundjoin%
\pgfsetlinewidth{0.501875pt}%
\definecolor{currentstroke}{rgb}{0.000000,0.000000,0.000000}%
\pgfsetstrokecolor{currentstroke}%
\pgfsetdash{{1.000000pt}{3.000000pt}}{0.000000pt}%
\pgfpathmoveto{\pgfqpoint{1.250000in}{2.400000in}}%
\pgfpathlineto{\pgfqpoint{9.000000in}{2.400000in}}%
\pgfusepath{stroke}%
\end{pgfscope}%
\begin{pgfscope}%
\pgfsetbuttcap%
\pgfsetroundjoin%
\definecolor{currentfill}{rgb}{0.000000,0.000000,0.000000}%
\pgfsetfillcolor{currentfill}%
\pgfsetlinewidth{0.501875pt}%
\definecolor{currentstroke}{rgb}{0.000000,0.000000,0.000000}%
\pgfsetstrokecolor{currentstroke}%
\pgfsetdash{}{0pt}%
\pgfsys@defobject{currentmarker}{\pgfqpoint{0.000000in}{0.000000in}}{\pgfqpoint{0.055556in}{0.000000in}}{%
\pgfpathmoveto{\pgfqpoint{0.000000in}{0.000000in}}%
\pgfpathlineto{\pgfqpoint{0.055556in}{0.000000in}}%
\pgfusepath{stroke,fill}%
}%
\begin{pgfscope}%
\pgfsys@transformshift{1.250000in}{2.400000in}%
\pgfsys@useobject{currentmarker}{}%
\end{pgfscope}%
\end{pgfscope}%
\begin{pgfscope}%
\pgfsetbuttcap%
\pgfsetroundjoin%
\definecolor{currentfill}{rgb}{0.000000,0.000000,0.000000}%
\pgfsetfillcolor{currentfill}%
\pgfsetlinewidth{0.501875pt}%
\definecolor{currentstroke}{rgb}{0.000000,0.000000,0.000000}%
\pgfsetstrokecolor{currentstroke}%
\pgfsetdash{}{0pt}%
\pgfsys@defobject{currentmarker}{\pgfqpoint{-0.055556in}{0.000000in}}{\pgfqpoint{0.000000in}{0.000000in}}{%
\pgfpathmoveto{\pgfqpoint{0.000000in}{0.000000in}}%
\pgfpathlineto{\pgfqpoint{-0.055556in}{0.000000in}}%
\pgfusepath{stroke,fill}%
}%
\begin{pgfscope}%
\pgfsys@transformshift{9.000000in}{2.400000in}%
\pgfsys@useobject{currentmarker}{}%
\end{pgfscope}%
\end{pgfscope}%
\begin{pgfscope}%
\pgftext[left,bottom,x=0.929389in,y=2.336971in,rotate=0.000000]{{\sffamily\fontsize{12.000000}{14.400000}\selectfont 0.5}}
%
\end{pgfscope}%
\begin{pgfscope}%
\pgfpathrectangle{\pgfqpoint{1.250000in}{0.400000in}}{\pgfqpoint{7.750000in}{3.200000in}} %
\pgfusepath{clip}%
\pgfsetbuttcap%
\pgfsetroundjoin%
\pgfsetlinewidth{0.501875pt}%
\definecolor{currentstroke}{rgb}{0.000000,0.000000,0.000000}%
\pgfsetstrokecolor{currentstroke}%
\pgfsetdash{{1.000000pt}{3.000000pt}}{0.000000pt}%
\pgfpathmoveto{\pgfqpoint{1.250000in}{2.800000in}}%
\pgfpathlineto{\pgfqpoint{9.000000in}{2.800000in}}%
\pgfusepath{stroke}%
\end{pgfscope}%
\begin{pgfscope}%
\pgfsetbuttcap%
\pgfsetroundjoin%
\definecolor{currentfill}{rgb}{0.000000,0.000000,0.000000}%
\pgfsetfillcolor{currentfill}%
\pgfsetlinewidth{0.501875pt}%
\definecolor{currentstroke}{rgb}{0.000000,0.000000,0.000000}%
\pgfsetstrokecolor{currentstroke}%
\pgfsetdash{}{0pt}%
\pgfsys@defobject{currentmarker}{\pgfqpoint{0.000000in}{0.000000in}}{\pgfqpoint{0.055556in}{0.000000in}}{%
\pgfpathmoveto{\pgfqpoint{0.000000in}{0.000000in}}%
\pgfpathlineto{\pgfqpoint{0.055556in}{0.000000in}}%
\pgfusepath{stroke,fill}%
}%
\begin{pgfscope}%
\pgfsys@transformshift{1.250000in}{2.800000in}%
\pgfsys@useobject{currentmarker}{}%
\end{pgfscope}%
\end{pgfscope}%
\begin{pgfscope}%
\pgfsetbuttcap%
\pgfsetroundjoin%
\definecolor{currentfill}{rgb}{0.000000,0.000000,0.000000}%
\pgfsetfillcolor{currentfill}%
\pgfsetlinewidth{0.501875pt}%
\definecolor{currentstroke}{rgb}{0.000000,0.000000,0.000000}%
\pgfsetstrokecolor{currentstroke}%
\pgfsetdash{}{0pt}%
\pgfsys@defobject{currentmarker}{\pgfqpoint{-0.055556in}{0.000000in}}{\pgfqpoint{0.000000in}{0.000000in}}{%
\pgfpathmoveto{\pgfqpoint{0.000000in}{0.000000in}}%
\pgfpathlineto{\pgfqpoint{-0.055556in}{0.000000in}}%
\pgfusepath{stroke,fill}%
}%
\begin{pgfscope}%
\pgfsys@transformshift{9.000000in}{2.800000in}%
\pgfsys@useobject{currentmarker}{}%
\end{pgfscope}%
\end{pgfscope}%
\begin{pgfscope}%
\pgftext[left,bottom,x=0.929389in,y=2.736971in,rotate=0.000000]{{\sffamily\fontsize{12.000000}{14.400000}\selectfont 1.0}}
%
\end{pgfscope}%
\begin{pgfscope}%
\pgfpathrectangle{\pgfqpoint{1.250000in}{0.400000in}}{\pgfqpoint{7.750000in}{3.200000in}} %
\pgfusepath{clip}%
\pgfsetbuttcap%
\pgfsetroundjoin%
\pgfsetlinewidth{0.501875pt}%
\definecolor{currentstroke}{rgb}{0.000000,0.000000,0.000000}%
\pgfsetstrokecolor{currentstroke}%
\pgfsetdash{{1.000000pt}{3.000000pt}}{0.000000pt}%
\pgfpathmoveto{\pgfqpoint{1.250000in}{3.200000in}}%
\pgfpathlineto{\pgfqpoint{9.000000in}{3.200000in}}%
\pgfusepath{stroke}%
\end{pgfscope}%
\begin{pgfscope}%
\pgfsetbuttcap%
\pgfsetroundjoin%
\definecolor{currentfill}{rgb}{0.000000,0.000000,0.000000}%
\pgfsetfillcolor{currentfill}%
\pgfsetlinewidth{0.501875pt}%
\definecolor{currentstroke}{rgb}{0.000000,0.000000,0.000000}%
\pgfsetstrokecolor{currentstroke}%
\pgfsetdash{}{0pt}%
\pgfsys@defobject{currentmarker}{\pgfqpoint{0.000000in}{0.000000in}}{\pgfqpoint{0.055556in}{0.000000in}}{%
\pgfpathmoveto{\pgfqpoint{0.000000in}{0.000000in}}%
\pgfpathlineto{\pgfqpoint{0.055556in}{0.000000in}}%
\pgfusepath{stroke,fill}%
}%
\begin{pgfscope}%
\pgfsys@transformshift{1.250000in}{3.200000in}%
\pgfsys@useobject{currentmarker}{}%
\end{pgfscope}%
\end{pgfscope}%
\begin{pgfscope}%
\pgfsetbuttcap%
\pgfsetroundjoin%
\definecolor{currentfill}{rgb}{0.000000,0.000000,0.000000}%
\pgfsetfillcolor{currentfill}%
\pgfsetlinewidth{0.501875pt}%
\definecolor{currentstroke}{rgb}{0.000000,0.000000,0.000000}%
\pgfsetstrokecolor{currentstroke}%
\pgfsetdash{}{0pt}%
\pgfsys@defobject{currentmarker}{\pgfqpoint{-0.055556in}{0.000000in}}{\pgfqpoint{0.000000in}{0.000000in}}{%
\pgfpathmoveto{\pgfqpoint{0.000000in}{0.000000in}}%
\pgfpathlineto{\pgfqpoint{-0.055556in}{0.000000in}}%
\pgfusepath{stroke,fill}%
}%
\begin{pgfscope}%
\pgfsys@transformshift{9.000000in}{3.200000in}%
\pgfsys@useobject{currentmarker}{}%
\end{pgfscope}%
\end{pgfscope}%
\begin{pgfscope}%
\pgftext[left,bottom,x=0.929389in,y=3.138070in,rotate=0.000000]{{\sffamily\fontsize{12.000000}{14.400000}\selectfont 1.5}}
%
\end{pgfscope}%
\begin{pgfscope}%
\pgfpathrectangle{\pgfqpoint{1.250000in}{0.400000in}}{\pgfqpoint{7.750000in}{3.200000in}} %
\pgfusepath{clip}%
\pgfsetbuttcap%
\pgfsetroundjoin%
\pgfsetlinewidth{0.501875pt}%
\definecolor{currentstroke}{rgb}{0.000000,0.000000,0.000000}%
\pgfsetstrokecolor{currentstroke}%
\pgfsetdash{{1.000000pt}{3.000000pt}}{0.000000pt}%
\pgfpathmoveto{\pgfqpoint{1.250000in}{3.600000in}}%
\pgfpathlineto{\pgfqpoint{9.000000in}{3.600000in}}%
\pgfusepath{stroke}%
\end{pgfscope}%
\begin{pgfscope}%
\pgfsetbuttcap%
\pgfsetroundjoin%
\definecolor{currentfill}{rgb}{0.000000,0.000000,0.000000}%
\pgfsetfillcolor{currentfill}%
\pgfsetlinewidth{0.501875pt}%
\definecolor{currentstroke}{rgb}{0.000000,0.000000,0.000000}%
\pgfsetstrokecolor{currentstroke}%
\pgfsetdash{}{0pt}%
\pgfsys@defobject{currentmarker}{\pgfqpoint{0.000000in}{0.000000in}}{\pgfqpoint{0.055556in}{0.000000in}}{%
\pgfpathmoveto{\pgfqpoint{0.000000in}{0.000000in}}%
\pgfpathlineto{\pgfqpoint{0.055556in}{0.000000in}}%
\pgfusepath{stroke,fill}%
}%
\begin{pgfscope}%
\pgfsys@transformshift{1.250000in}{3.600000in}%
\pgfsys@useobject{currentmarker}{}%
\end{pgfscope}%
\end{pgfscope}%
\begin{pgfscope}%
\pgfsetbuttcap%
\pgfsetroundjoin%
\definecolor{currentfill}{rgb}{0.000000,0.000000,0.000000}%
\pgfsetfillcolor{currentfill}%
\pgfsetlinewidth{0.501875pt}%
\definecolor{currentstroke}{rgb}{0.000000,0.000000,0.000000}%
\pgfsetstrokecolor{currentstroke}%
\pgfsetdash{}{0pt}%
\pgfsys@defobject{currentmarker}{\pgfqpoint{-0.055556in}{0.000000in}}{\pgfqpoint{0.000000in}{0.000000in}}{%
\pgfpathmoveto{\pgfqpoint{0.000000in}{0.000000in}}%
\pgfpathlineto{\pgfqpoint{-0.055556in}{0.000000in}}%
\pgfusepath{stroke,fill}%
}%
\begin{pgfscope}%
\pgfsys@transformshift{9.000000in}{3.600000in}%
\pgfsys@useobject{currentmarker}{}%
\end{pgfscope}%
\end{pgfscope}%
\begin{pgfscope}%
\pgftext[left,bottom,x=0.929389in,y=3.536971in,rotate=0.000000]{{\sffamily\fontsize{12.000000}{14.400000}\selectfont 2.0}}
%
\end{pgfscope}%
\begin{pgfscope}%
\pgfsetrectcap%
\pgfsetroundjoin%
\pgfsetlinewidth{1.003750pt}%
\definecolor{currentstroke}{rgb}{0.000000,0.000000,0.000000}%
\pgfsetstrokecolor{currentstroke}%
\pgfsetdash{}{0pt}%
\pgfpathmoveto{\pgfqpoint{1.250000in}{3.600000in}}%
\pgfpathlineto{\pgfqpoint{9.000000in}{3.600000in}}%
\pgfusepath{stroke}%
\end{pgfscope}%
\begin{pgfscope}%
\pgfsetrectcap%
\pgfsetroundjoin%
\pgfsetlinewidth{1.003750pt}%
\definecolor{currentstroke}{rgb}{0.000000,0.000000,0.000000}%
\pgfsetstrokecolor{currentstroke}%
\pgfsetdash{}{0pt}%
\pgfpathmoveto{\pgfqpoint{9.000000in}{0.400000in}}%
\pgfpathlineto{\pgfqpoint{9.000000in}{3.600000in}}%
\pgfusepath{stroke}%
\end{pgfscope}%
\begin{pgfscope}%
\pgfsetrectcap%
\pgfsetroundjoin%
\pgfsetlinewidth{1.003750pt}%
\definecolor{currentstroke}{rgb}{0.000000,0.000000,0.000000}%
\pgfsetstrokecolor{currentstroke}%
\pgfsetdash{}{0pt}%
\pgfpathmoveto{\pgfqpoint{1.250000in}{0.400000in}}%
\pgfpathlineto{\pgfqpoint{9.000000in}{0.400000in}}%
\pgfusepath{stroke}%
\end{pgfscope}%
\begin{pgfscope}%
\pgfsetrectcap%
\pgfsetroundjoin%
\pgfsetlinewidth{1.003750pt}%
\definecolor{currentstroke}{rgb}{0.000000,0.000000,0.000000}%
\pgfsetstrokecolor{currentstroke}%
\pgfsetdash{}{0pt}%
\pgfpathmoveto{\pgfqpoint{1.250000in}{0.400000in}}%
\pgfpathlineto{\pgfqpoint{1.250000in}{3.600000in}}%
\pgfusepath{stroke}%
\end{pgfscope}%
\begin{pgfscope}%
\pgftext[left,bottom,x=2.772413in,y=3.627843in,rotate=0.000000]{{\sffamily\fontsize{14.400000}{17.280000}\selectfont moon in rest frame of earth, dt=0.1a, 10 years}}
%
\end{pgfscope}%
\begin{pgfscope}%
\pgfsetrectcap%
\pgfsetroundjoin%
\definecolor{currentfill}{rgb}{1.000000,1.000000,1.000000}%
\pgfsetfillcolor{currentfill}%
\pgfsetlinewidth{1.003750pt}%
\definecolor{currentstroke}{rgb}{0.000000,0.000000,0.000000}%
\pgfsetstrokecolor{currentstroke}%
\pgfsetdash{}{0pt}%
\pgfpathmoveto{\pgfqpoint{7.284662in}{2.877606in}}%
\pgfpathlineto{\pgfqpoint{8.930583in}{2.877606in}}%
\pgfpathlineto{\pgfqpoint{8.930583in}{3.530583in}}%
\pgfpathlineto{\pgfqpoint{7.284662in}{3.530583in}}%
\pgfpathlineto{\pgfqpoint{7.284662in}{2.877606in}}%
\pgfpathclose%
\pgfusepath{stroke,fill}%
\end{pgfscope}%
\begin{pgfscope}%
\pgfsetrectcap%
\pgfsetroundjoin%
\pgfsetlinewidth{1.003750pt}%
\definecolor{currentstroke}{rgb}{0.000000,0.000000,1.000000}%
\pgfsetstrokecolor{currentstroke}%
\pgfsetdash{}{0pt}%
\pgfpathmoveto{\pgfqpoint{7.381846in}{3.418161in}}%
\pgfpathlineto{\pgfqpoint{7.576212in}{3.418161in}}%
\pgfusepath{stroke}%
\end{pgfscope}%
\begin{pgfscope}%
\pgftext[left,bottom,x=7.728929in,y=3.367603in,rotate=0.000000]{{\sffamily\fontsize{9.996000}{11.995200}\selectfont euler}}
%
\end{pgfscope}%
\begin{pgfscope}%
\pgfsetrectcap%
\pgfsetroundjoin%
\pgfsetlinewidth{1.003750pt}%
\definecolor{currentstroke}{rgb}{0.000000,0.500000,0.000000}%
\pgfsetstrokecolor{currentstroke}%
\pgfsetdash{}{0pt}%
\pgfpathmoveto{\pgfqpoint{7.381846in}{3.214385in}}%
\pgfpathlineto{\pgfqpoint{7.576212in}{3.214385in}}%
\pgfusepath{stroke}%
\end{pgfscope}%
\begin{pgfscope}%
\pgftext[left,bottom,x=7.728929in,y=3.136915in,rotate=0.000000]{{\sffamily\fontsize{9.996000}{11.995200}\selectfont symplectic euler}}
%
\end{pgfscope}%
\begin{pgfscope}%
\pgfsetrectcap%
\pgfsetroundjoin%
\pgfsetlinewidth{1.003750pt}%
\definecolor{currentstroke}{rgb}{1.000000,0.000000,0.000000}%
\pgfsetstrokecolor{currentstroke}%
\pgfsetdash{}{0pt}%
\pgfpathmoveto{\pgfqpoint{7.381846in}{3.010609in}}%
\pgfpathlineto{\pgfqpoint{7.576212in}{3.010609in}}%
\pgfusepath{stroke}%
\end{pgfscope}%
\begin{pgfscope}%
\pgftext[left,bottom,x=7.728929in,y=2.933139in,rotate=0.000000]{{\sffamily\fontsize{9.996000}{11.995200}\selectfont velocity verlet}}
%
\end{pgfscope}%
\end{pgfpicture}%
\makeatother%
\endgroup%
}
	\caption{Long-term simulation of the trajectory of the moon using different algorithms}\label{fig:s4}
\end{figure}

In the resulting plot (see figure \ref{fig:s4}) the Velocity Verlet algorithm computed trajektory can hardly be seen as litte red point in the center. This confirms the result of the previous simulation: Velocity Verlet seems not only to be short-term but also long-term the best of the compared algorithms.\\
Surprisingly in the long-term simulation the symplectic Euler algorithm diverges faster than the simple Euler integration algorithm.

\end{document}
